\documentclass[]{book}
\usepackage{lmodern}
\usepackage{amssymb,amsmath}
\usepackage{ifxetex,ifluatex}
\usepackage{fixltx2e} % provides \textsubscript
\ifnum 0\ifxetex 1\fi\ifluatex 1\fi=0 % if pdftex
  \usepackage[T1]{fontenc}
  \usepackage[utf8]{inputenc}
\else % if luatex or xelatex
  \ifxetex
    \usepackage{mathspec}
  \else
    \usepackage{fontspec}
  \fi
  \defaultfontfeatures{Ligatures=TeX,Scale=MatchLowercase}
\fi
% use upquote if available, for straight quotes in verbatim environments
\IfFileExists{upquote.sty}{\usepackage{upquote}}{}
% use microtype if available
\IfFileExists{microtype.sty}{%
\usepackage{microtype}
\UseMicrotypeSet[protrusion]{basicmath} % disable protrusion for tt fonts
}{}
\usepackage[margin=1in]{geometry}
\usepackage{hyperref}
\hypersetup{unicode=true,
            pdftitle={NBA Collective Bargaining Agreement - 2017},
            pdfauthor={Robert},
            pdfborder={0 0 0},
            breaklinks=true}
\urlstyle{same}  % don't use monospace font for urls
\usepackage{natbib}
\bibliographystyle{plainnat}
\usepackage{longtable,booktabs}
\usepackage{graphicx,grffile}
\makeatletter
\def\maxwidth{\ifdim\Gin@nat@width>\linewidth\linewidth\else\Gin@nat@width\fi}
\def\maxheight{\ifdim\Gin@nat@height>\textheight\textheight\else\Gin@nat@height\fi}
\makeatother
% Scale images if necessary, so that they will not overflow the page
% margins by default, and it is still possible to overwrite the defaults
% using explicit options in \includegraphics[width, height, ...]{}
\setkeys{Gin}{width=\maxwidth,height=\maxheight,keepaspectratio}
\IfFileExists{parskip.sty}{%
\usepackage{parskip}
}{% else
\setlength{\parindent}{0pt}
\setlength{\parskip}{6pt plus 2pt minus 1pt}
}
\setlength{\emergencystretch}{3em}  % prevent overfull lines
\providecommand{\tightlist}{%
  \setlength{\itemsep}{0pt}\setlength{\parskip}{0pt}}
\setcounter{secnumdepth}{5}
% Redefines (sub)paragraphs to behave more like sections
\ifx\paragraph\undefined\else
\let\oldparagraph\paragraph
\renewcommand{\paragraph}[1]{\oldparagraph{#1}\mbox{}}
\fi
\ifx\subparagraph\undefined\else
\let\oldsubparagraph\subparagraph
\renewcommand{\subparagraph}[1]{\oldsubparagraph{#1}\mbox{}}
\fi

%%% Use protect on footnotes to avoid problems with footnotes in titles
\let\rmarkdownfootnote\footnote%
\def\footnote{\protect\rmarkdownfootnote}

%%% Change title format to be more compact
\usepackage{titling}

% Create subtitle command for use in maketitle
\newcommand{\subtitle}[1]{
  \posttitle{
    \begin{center}\large#1\end{center}
    }
}

\setlength{\droptitle}{-2em}
  \title{NBA Collective Bargaining Agreement - 2017}
  \pretitle{\vspace{\droptitle}\centering\huge}
  \posttitle{\par}
  \author{Robert}
  \preauthor{\centering\large\emph}
  \postauthor{\par}
  \predate{\centering\large\emph}
  \postdate{\par}
  \date{2017-05-25}

\usepackage{booktabs}
\usepackage{enumitem}
\setlistdepth{9}

\renewlist{enumerate}{enumerate}{9}
\setlist[itemize]{labelsep=.5em}

\usepackage{fancyhdr}
\usepackage{tocstyle}
\usetocstyle{standard}

\renewcommand{\chaptername}{Article}
\renewcommand{\thechapter}{\Roman{chapter}}

\begin{document}
\maketitle

{
\setcounter{tocdepth}{1}
\tableofcontents
}
\chapter*{Preface}\label{preface}
\addcontentsline{toc}{chapter}{Preface}

I do not own any rights to the CBA and am simply redistributing it in a
different format. This is not the official version of the CBA and I am
not responsible for any errors that might be present in this document.
If you believe you have found an error, please let me know and I will
correct it. In addition, an html version of the CBA is available at
\url{https://atlhawksfanatic.github.io/NBA-CBA/}.

Contact:

\begin{itemize}
\tightlist
\item
  via email: \url{atlhawksfanatic@gmail.com}
\item
  through Github: \url{https://github.com/atlhawksfanatic}
\item
  on Github: \url{https://twitter.com/atlhawksfanatic}
\end{itemize}

\chapter{DEFINITIONS}\label{definitions}

\section{Definitions.}\label{definitions.}

As used in this Agreement, the following terms shall have the following
meanings:

\begin{enumerate}
\def\labelenumi{\arabic{enumi}.}
\tightlist
\item
  ``Active List'' means the list of players, maintained by the NBA, who
  have signed Player Contracts with a Team and are otherwise eligible to
  participate in a Regular Season game.
\item
  ``Agreement'' means this Collective Bargaining Agreement entered into
  as of January 19, 2017.
\item
  ``Audit Report'' or ``final Audit Report'' means the audit report
  prepared in accordance with Article VII, Section 10.
\item
  ``Average Player Salary'' means, with respect to any Salary Cap Year,
  Total Salaries (plus any amounts paid by the NBA for such Salary Cap
  Year pursuant to Article VII, Section 2(b)) divided by an amount equal
  to the product of the number of Teams in the NBA (other than Expansion
  Teams during their first two (2) Salary Cap Years) multiplied by 13.2.
\item
  ``Base Compensation'' means the component of Compensation other than
  bonuses of any kind.
\item
  ``Basketball Related Income'' or ``BRI'' means basketball related
  income as defined in Article VII, Section 1(a) and (b).
\item
  ``Benefits'' or ``Total Benefits'' means the sum of all amounts paid
  or to be paid on an accrual basis during any Salary Cap Year by the
  NBA or NBA Teams, other than Expansion Teams during their first two
  Salary Cap Years, for the specific benefits set forth in Article IV.
\item
  ``Commissioner'' means the Commissioner of the NBA.
\item
  ``Compensation'' means the compensation that is or could be earned by,
  or is paid or payable to, an NBA player (including players whose
  Player Contracts have been terminated) in accordance with a Player
  Contract (whether such payment is sent to the player directly or to a
  person or entity designated by a player).
\item
  ``Contract'' (see ``Uniform Player Contract'').
\item
  ``Current Base Compensation'' means the component of Base Compensation
  other than Deferred Base Compensation.
\item
  ``Current Compensation'' means the component of Compensation other
  than Deferred Compensation.
\item
  ``Deferred Base Compensation'' means the component of Deferred
  Compensation other than bonuses of any kind.
\item
  ``Deferred Compensation'' means the component of Compensation for a
  Season that is payable to a player during the period commencing after
  the May 1 following such Season, in accordance with the rules set
  forth in Article VII. The determination of whether Compensation is
  Deferred Compensation will be based upon the time set by the Player
  Contract for the player to receive the Compensation, without regard to
  whether the obligation is funded currently or secured in any fashion.
\item
  ``Designated Rookie Scale Player'' means a player with whom a Team
  has, pursuant to Article II, Section 7(d) and Article VII, Section
  7(b), entered into a Designated Player Rookie Scale Extension. A Team
  may at any point in time, in respect of any current or future Salary
  Cap Year, have Salary included in its Team Salary in respect of a
  maximum of two (2) Designated Rookie Scale Players, provided that not
  more than one (1) of such Designated Rookie Scale Players may have
  been acquired by the Team by assignment. For purposes of the preceding
  sentence, if Salary in respect of a terminated Designated Rookie Scale
  Player Extension is stretched for Salary Cap purposes in accordance
  with Article VII, Section 7(d)(6), the Salary in respect of such
  terminated player's Extension will not count towards the maximum of
  two (2) Designated Rookie Scale Players for any Salary Cap Year that
  was not covered by the term of the Extension.
\item
  ``Designated Rookie Scale Player Extension'' means an Extension of a
  Rookie Scale Contract entered into between a Team and its Designated
  Rookie Scale Player that covers six (6) Seasons from the date the
  Extension is signed and provides for Salary for the first Salary Cap
  Year covered by the extended term equal to the player's applicable
  Maximum Annual Salary under Article II, Section 7 (or, in the case of
  a First Round Pick who has satisfied or may satisfy the 5th Year 30\%
  Max Criteria (as set forth in Article II, Section 7(a)(i)), provides
  for Salary for the first Salary Cap Year equal to twenty-five percent
  (25\%) or thirty percent (30\%) (or such other percentage between 25\%
  and 30\% as agreed upon by the Team and the player) of the Salary Cap
  in effect during the first Season of the extended term), with annual
  increases in Salary for each Salary Cap Year following the first
  Salary Cap Year of the extended term equal to eight percent (8\%) of
  the Salary for the first Salary Cap Year covered by the extended term.
\item
  ``Designated Veteran Player'' means a player with whom a Team has,
  pursuant to Article II, Sections 7(a)(ii) or 7(e) and Article VII,
  Section 7(a), entered into either a Designated Veteran Player
  Extension or Designated Veteran Player Contract. A Team may at any
  point in time, in respect of any current or future Salary Cap Year,
  have Salary included in its Team Salary in respect of a maximum of two
  (2) Designated Veteran Players. For purposes of the preceding
  sentence, if Salary in respect of a terminated Designated Veteran
  Player Contract or Designated Veteran Player Extension is stretched
  for Salary Cap purposes in accordance with Article VII, Section
  7(d)(6), the Salary in respect of such terminated player's Contract or
  Extension will not count towards the maximum of two (2) Designated
  Veteran Players for any Salary Cap Year that was not covered by the
  term of the Contract or Extension (as applicable).
\item
  ``Designated Veteran Player Contract'' means a Contract entered into
  between a Team and its Designated Veteran Player who is a Qualifying
  Veteran Free Agent that covers five (5) Seasons and provides for
  Salary for the first Salary Cap Year equal to such percentage above
  30\% but not greater than 35\% (as agreed upon by the Team and the
  player) of the Salary Cap in effect at the time the Contract is
  executed. Annual increases and decreases in Salary and Unlikely
  Bonuses shall be governed by Article VII, Section 5(c)(2).
\item
  ``Designated Veteran Player Extension'' means an Extension of a
  Contract entered into between a Team and its Designated Veteran Player
  that covers six (6) Seasons from the date the Extension is signed and
  provides for Salary for the first Salary Cap Year covered by the
  extended term equal to thirty percent (30\%) or thirty-five percent
  (35\%) (or such other percentage between 30\% and 35\% as agreed upon
  by the Team and the player) of the Salary Cap in effect during the
  first Season of the extended term. Annual increases and decreases in
  Salary and Unlikely Bonuses shall be governed by Article VII, Section
  5(c)(3)).
\item
  ``Draft'' or ``NBA Draft'' means the NBA's annual draft of Rookie
  basketball players.
\item
  ``Early Qualifying Veteran Free Agent'' means a Veteran Free Agent
  who, prior to becoming a Veteran Free Agent, played under one (1) or
  more Player Contracts covering some or all of each of the two (2)
  preceding Seasons, and who either played exclusively with his Prior
  Team during such two (2) Seasons, or, if he played for more than one
  (1) Team during such period, changed Teams only (i) by means of trade,
  (ii) by means of assignment via the NBA's waiver procedures, or (iii)
  by signing with his Prior Team during the first of the two (2)
  Seasons.
\item
  ``Early Termination Option'' (or ``ETO'') means an option in favor of
  a player to shorten the stated number of years covered by a Player
  Contract in accordance with Article XII.
\item
  ``Effective Season'' means, with respect to an Early Termination
  Option, the first Season covered by the Early Termination Option. (For
  example, if a Contract were to contain an Early Termination Option
  exercisable following the 2021-22 Season, the Effective Season would
  be the 2022-23 Season.)
\item
  ``Estimated Average Player Salary'' means, for a particular Salary Cap
  Year, one hundred four and one-half percent (104.5\%) of the prior
  Salary Cap Year's Average Player Salary.
\item
  ``Exception'' means an exception to the rule that a Team's Team Salary
  may not exceed the Salary Cap.
\item
  ``Expansion Team'' means any Team that becomes a member of the NBA
  through expansion following the effective date of this Agreement and
  commences play during the term of this Agreement.
\item
  ``Extension'' means an amendment to a Player Contract lengthening the
  term of the Contract for a specified period of years.
\item
  ``First Round Pick'' means a player selected by a Team in the first
  round of the Draft.
\item
  ``Free Agent'' means: (i) a Veteran Free Agent; (ii) a Rookie Free
  Agent; (iii) a Veteran whose Player Contract has been terminated in
  accordance with the NBA waiver procedure; or (iv) a player whose last
  Player Contract was a 10-Day Contract and who either completed the
  Contract by rendering the playing services called for thereunder or
  was released early from such Contract.
\item
  ``Generally Recognized League Honors'' means the following NBA league
  honors awarded to players: NBA Most Valuable Player; NBA Finals Most
  Valuable Player; NBA Defensive Player of the Year; NBA Sixth Man
  Award; NBA Most Improved Player; All-NBA Team (First, Second, or
  Third); NBA All-Defensive Team (First or Second); and All-Star Team
  Selection.
\item
  ``Inactive List'' means the list of players, maintained by the NBA,
  who have signed Player Contracts with a Team and are otherwise
  ineligible to participate in a Regular Season game.
\item
  ``Incentive Compensation'' means the component of Compensation
  consisting of one (1) or more bonuses described in Article II,
  Sections 3(b)(ii) and (iii) and 3(c).
\item
  ``Likely Bonus'' means Incentive Compensation included in a player's
  Salary in accordance with Article VII, Section 3(d).
\item
  ``Maximum Annual Salary'' means the maximum amount of Salaries and
  Unlikely Bonuses a player is eligible to receive in the first Salary
  Cap Year covered by a Contract or Extension as calculated in
  accordance with Article II, Section 7.
\item
  ``Member'' or ``Team'' means any team that is a member of the NBA.
\item
  ``Minimum Annual Salary'' means the minimum Salary that must be
  included in a Player Contract (other than a Two-Way Contract) that
  covers the entire Regular Season in accordance with Article II,
  Section 6(a).
\item
  ``Minimum Annual Salary Scale'' means: (i) for the 2017-18 Salary Cap
  Year, the Minimum Annual Salary Scale table annexed hereto as Exhibit
  C; and (ii) for each of the 2018-19 through 2023-24 Salary Cap Years,
  the Minimum Annual Salary Scale table for such Salary Cap Year
  prepared immediately upon the determination of the Salary Cap for such
  Salary Cap Year in accordance with the provisions of this Agreement
  and including the adjusted Minimum Annual Salary Amounts for such
  Salary Cap Year as calculated in accordance with Section 1(jj) above.
\item
  ``Minimum Player Salary'' means: (i) with respect to a Contract (other
  than a Two-Way Contract) that covers the entire Regular Season, the
  Minimum Annual Salary called for under Article II, Section 6(a); (ii)
  with respect to a Contract that covers less than the entire Regular
  Season (other than a Two-Way Contract or 10-Day Contract), the Minimum
  Annual Salary called for under Article II, Section 6(a) multiplied by
  a fraction, the numerator of which is the number of days remaining in
  the NBA Regular Season as of the date such Contract is entered into,
  and the denominator of which is the total number of days of that NBA
  Regular Season; and (iii) with respect to a 10-Day Contract, the
  Minimum Annual Salary called for under Article II, Section 6(a)
  multiplied by a fraction, the numerator of which is the number of days
  covered by the Contract and the denominator of which is the total
  number of days of that NBA Regular Season.
\item
  ``Minimum Team Salary'' means the minimum Team Salary a Team must have
  for a Salary Cap Year as determined in accordance with Article VII,
  Section 2(b).
\item
  ``Moratorium Period'' means, with respect to a Salary Cap Year, the
  period from 12:01 a.m. eastern time on July 1 of such Salary Cap Year
  through 12:00 p.m. eastern time on the following July 6 (for clarity,
  regardless of whether July 6 is a business day).
\item
  The term ``negotiate'' means, with respect to a player or his
  representatives on the one hand, and a Team or its representatives on
  the other hand, to engage in any written or oral communication
  relating to the possible employment, or terms of employment, of such
  player by such Team as a basketball player, regardless of who
  initiates such communication.
\item
  ``NBADL Regular Season'' means, with respect to any NBADL Season, the
  period beginning on the first day and ending on the last day of
  regularly scheduled (as opposed to exhibition or playoff) competition
  between NBADL teams.
\item
  ``NBADL Season'' means the period beginning on the first day of NBADL
  training camp and ending immediately after the last game of the NBADL
  playoffs.
\item
  ``Non-Qualifying Veteran Free Agent'' means a Veteran Free Agent who
  is not a Qualifying Veteran Free Agent or an Early Qualifying Veteran
  Free Agent.
\item
  ``Option'' means an option in a Player Contract in favor of a Team or
  player to extend such Contract beyond its stated term.
\item
  ``Option Year'' means the year that would be added to a Player
  Contract if an Option were exercised.
\item
  ``Performance Bonus'' means any Incentive Compensation described in
  Article II, Section 3(b)(ii).
\item
  ``Player Contract'' (see ``Uniform Player Contract'').
\item
  ``Prior Team'' means the Team for which a player was last under
  Contract prior to becoming a Qualifying Veteran Free Agent, Early
  Qualifying Veteran Free Agent or a Non-Qualifying Veteran Free Agent.
\item
  ``Qualifying Offer'' means a qualifying offer as defined in Article
  XI, Section 1(c).
\item
  ``Qualifying Veteran Free Agent'' means a Veteran Free Agent who,
  prior to becoming a Veteran Free Agent, played under one (1) or more
  Player Contracts covering some or all of each of the three (3)
  preceding Seasons, and who either played exclusively with his Prior
  Team during such three (3) Seasons, or, if he played for more than one
  (1) Team during such period, changed Teams only (i) by means of trade,
  (ii) by means of assignment via the NBA's waiver procedures during the
  first of the three (3) Seasons, or (ii) by signing with his Prior Team
  during the first of the three (3) Seasons.
\item
  ``Regular Salary'' means a player's Salary, less any component thereof
  that is a signing bonus (or deemed a signing bonus in accordance with
  Article VII) and any component thereof that is Incentive Compensation.
\item
  ``Regular Season'' means, with respect to any Season, the period
  beginning on the first day and ending on the last day of regularly
  scheduled (as opposed to exhibition or playoff) competition between
  NBA Teams.
\item
  ``Renegotiation,'' ``renegotiate,'' or ``renegotiated'' means a
  Contract amendment that provides for an increase in Salary and/or
  Unlikely Bonuses.
\item
  ``Replacement Player'' means, where appropriate, either a player who
  is acquired by a Team pursuant to the Traded Player Exception, or a
  player who is signed or acquired by a Team pursuant to the Disabled
  Player Exception.
\item
  ``Required Tender'' means an offer of a Uniform Player Contract to a
  Draft Rookie, signed by the Team, that: (i) on or before the date
  specified in Article X is either personally delivered to the player or
  his representative or sent by email or pre-paid certified, registered,
  or overnight mail to the last known address of the player or his
  representative; (ii) with respect to a First Round Pick, (A) affords
  the player until at least the first day of the following Regular
  Season to accept, and (B) satisfies the requirements of a Rookie Scale
  Contract set forth in Article VIII, Section 1 or 2; and (iii) with
  respect to a Second Round Pick, (A) affords the player until at least
  the immediately following October 15 to accept, (B) has a stated term
  of one (1) Season, and (C) calls for at least the Minimum Annual
  Salary then applicable to the player. In addition, a Team shall be
  permitted to include in any Required Tender an Exhibit 6 to the
  Uniform Player Contract requiring that the player, if he signs the
  Required Tender, pass a physical examination to be performed by a
  physician designated by the Team as a condition precedent to the
  validity of the Contract.
\item
  ``Restricted Free Agent'' means a Veteran Free Agent who is subject to
  a Team's right of first refusal in accordance with Article XI.
\item
  ``Rookie'' means a person who has never signed a Player Contract with
  an NBA Team.

  \begin{enumerate}
  \def\labelenumii{\arabic{enumii}.}
  \tightlist
  \item
    Draft Rookie" means a Rookie who is selected in the NBA Draft.
  \item
    ``Non-Draft Rookie''means a Rookie who is not selected in the NBA
    Draft for which he is first eligible.
  \end{enumerate}
\item
  ``Rookie Free Agent'' means: (i) a Draft Rookie who, pursuant to the
  provisions of Article VIII, Section 3 or Article X, is no longer
  subject to the exclusive negotiating rights of any Team, and who may
  be signed by any Team; or (ii) a Non-Draft Rookie.
\item
  ``Rookie Salary Scales'' means: (i) for the 2017-18 Salary Cap Year,
  the Rookie Salary Scale table annexed hereto as Exhibit B-1; and (ii)
  for each of the 2018-19 through 2023-24 Salary Cap Years, the Rookie
  Salary Scale table for such Salary Cap Year prepared immediately upon
  the determination of the Salary Cap for such Salary Cap Year in
  accordance with the provisions of this Agreement and including the
  adjusted Rookie Scale Amounts for such Salary Cap Year as calculated
  in accordance with Section 1(iii) below.
\item
  ``Rookie Scale Amounts'' means: (i) for the 2017-18 Salary Cap Year,
  the amounts set forth in the Rookie Salary Scale table annexed hereto
  as Exhibit B-1; (ii) for the 2018-19 through 2019-20 Salary Cap Years,
  the Salary amounts calculated in accordance with Article VIII, Section
  4; and (iii) for the 2020-21 through 2023-24 Salary Cap Years, the
  Salary amounts set forth in the preceding Salary Cap Year's Rookie
  Salary Scale table adjusted by applying the percentage increase (or
  decrease) in the Salary Cap from the preceding Salary Cap Year to the
  current Salary Cap Year; provided, however, that the applicable ``4th
  Year Option'' and Qualifying Offer percentages specified in the Rookie
  Salary Scale for the 2017-18 Salary Cap Year annexed hereto as Exhibit
  B-1 shall remain the same for each Salary Cap Year during the Term and
  be included in the Rookie Salary Scales prepared for each Salary Cap
  Year in accordance with Section 1(hhh) above.
\item
  ``Rookie Scale Contract'' means the initial Uniform Player Contract
  entered into, in accordance with Article VIII, Section 1 or 2, between
  a First Round Pick and the Team that holds his draft rights.
\item
  ``Room'' means the extent to which: (i) a Team's then-current Team
  Salary is less than the Salary Cap; or (ii) a Team is entitled to use
  one of the Salary Cap Exceptions set forth in Article VII, Section
  6(c), (d), (e), (f), (g) and (j) (Disabled Player Exception, Bi-annual
  Exception, Non-Taxpayer Mid-Level Salary Exception, Taxpayer Mid-Level
  Salary Exception, Mid-Level Salary Exception for Room Teams and Traded
  Player Exception).
\item
  ``Salary'' means, with respect to a Salary Cap Year, a player's
  Compensation with respect to the Season covered by such Salary Cap
  Year, plus any other amount that is deemed to constitute Salary in
  accordance with the terms of this Agreement, not including Unlikely
  Bonuses, any benefits the player received in accordance with the terms
  of this Agreement (including, e.g., the benefits provided for by
  Article IV, per diem, and moving expenses), and any portion of the
  player's Compensation that is attributable to another Salary Cap Year
  in accordance with this Agreement. Salary also includes any
  consideration received by a retired player that is deemed to
  constitute Salary in accordance with the terms of Article XIII.
\item
  ``Salary Cap'' means the maximum allowable Team Salary for each Team
  for a Salary Cap Year, subject to the rules and exceptions set forth
  in this Agreement.
\item
  ``Salary Cap Year'' means the period from July 1 through the following
  June 30.
\item
  ``Season'' or ``NBA Season'' means the period beginning on the first
  day of NBA training camp and ending immediately after the last game of
  the NBA Finals.
\item
  ``Second Round Pick'' means a player selected by a Team in the second
  round of the Draft.
\item
  ``Standard NBA Contract'' means a Contract other than a Two-Way
  Contract.
\item
  ``Standard NBA Contract Conversion Option'' means an option in a
  Two-Way Contract in favor of a Team to convert the Contract to a
  Standard NBA Contract that provides for a Salary for each Salary Cap
  Year equal to the player's applicable Minimum Player Salary and a term
  equal to the remainder of the original term of the Two-Way Contract,
  in accordance with Article II, Section 11(g).
\item
  ``Team'' or ``NBA Team'' (see ``Member'').
\item
  ``Team Affiliate'' means:

  \begin{enumerate}
  \def\labelenumii{\roman{enumii}.}
  \tightlist
  \item
    any individual or entity who or which (directly or indirectly) holds
    an ownership interest in a Team (other than ownership of
    publicly-traded securities constituting less than five percent (5\%)
    of the ownership interests in a Team) (an ``Owner'');
  \item
    any individual or entity who or which (directly or indirectly)
    controls, is controlled by or is under common control with, or who
    or which is an entity affiliated with or an individual related to, a
    Team;
  \item
    any individual or entity who or which (directly or indirectly)
    controls, is controlled by or is under common control with, or who
    or which is an entity affiliated with or an individual related to,
    an individual or entity described in Section 1(ttt)(i) or (ii)
    above; or
  \item
    any entity as to which (x) an Owner, or (y) an individual or entity
    that holds (directly or indirectly) an ownership interest in an
    entity described in Section 1(ttt)(ii) above, either (a) holds
    (directly or indirectly) more than five percent (5\%) of its
    ownership interests, or (b) participates in or influences its
    management or operations. For the purposes of this Section 1(ttt):
    an individual shall only be deemed to be ``related to'' a Team or
    another individual or entity if such individual is an officer,
    director, trustee, or executive employee of such Team or entity, or
    is a member of such individual's immediate family; and ``controls''
    or ``is controlled by'' shall include (without limitation) the
    circumstance in which an individual or a Team or entity has or can
    exercise effective control.
  \end{enumerate}
\item
  ``Team Salary'' means, with respect to a Salary Cap Year, the sum of
  all Salaries attributable to a Team's active and former players plus
  other amounts as computed in accordance with Article VII, less
  applicable credit amounts as computed in accordance with Article VII.
\item
  ``Total Salaries'' means the total Salaries included in the Team
  Salary of all NBA Teams for or with respect to a Salary Cap Year in
  accordance with this Agreement, other than the Salaries included in
  the Team Salary of Expansion Teams during their first two Salary Cap
  Years,as determined in accordance with Article VII. For purposes of
  this definition: (i) Total Salaries shall include all Incentive
  Compensation excluded from Salaries in accordance with Article VII,
  Section 3(d) but actually earned by NBA players during such Salary Cap
  Year, and shall exclude all Incentive Compensation included in
  Salaries in accordance with Article VII, Section 3(d) but not actually
  earned by NBA players during such Salary Cap Year; (ii) Total Salaries
  shall include the aggregate Salaries, if any, that are excluded from
  Team Salaries pursuant to Article VII, Section 4(h); (iii) Total
  Salaries shall include any amounts that a Team pays to its players
  pursuant to Article VII, Section 2(b); (iv) Total Salaries shall
  include any consideration received by a retired player that is
  included in Team Salary in accordance with the terms of Article XIII;
  (v) Total Salaries shall include the sum of: (a) all Two-Way NBA
  Salaries, and (b) for each Two-Way Contract that is terminated whereby
  the Team is required to pay Compensation in accordance with Exhibit 2
  (or Exhibit 10 as applicable) of the Two-Way Contract, fifty percent
  (50\%) of all Compensation payable to the player under the Contract
  less the sum of: (x) the Two-Way NBA Salary earned by the player under
  the Contract, and (y) the Two-Way NBADL Salary earned by the player
  under the Contract; and (vi) Total Salaries shall include any Exhibit
  10 Bonus a Team pays its players pursuant to Article II, Section
  3(q)(i).
\item
  ``Total Salaries and Benefits'' means the sum of Total Salaries plus
  Total Benefits.
\item
  ``Traded Player'' means a player whose Player Contract is assigned by
  one Team to another Team other than by means of the NBA waiver
  procedure.
\item
  ``Two-Way Contract'' means a Contract between a Two-Way Player and a
  Team made in accordance with Article II, Section 11. In the event that
  a Two-Way Contract is converted to a Standard NBA Contract pursuant to
  the Team's exercise of its Standard NBA Contract Conversion Option,
  the Contract shall no longer be a Two-Way Contract for the purposes of
  this Agreement.
\item
  ``Two-Way List'' means the list of players, maintained by the NBA, who
  have signed Two-Way Contracts and are eligible to provide services to
  an NBADL team in accordance with the provisions of this Agreement.
\item
  ``Two-Way Player'' means a player under a Two-Way Contract in
  accordance with Article II, Section 11.
\item
  ``Two-Way Player Salary'' means, with respect to any Two-Way Contract,
  the Salary called for under Article II, Section 11(a).
\item
  ``Two-Way Player Conversion Option'' means an option in a Player
  Contract with an Exhibit 10 in favor of a Team to convert the Contract
  to a Two-Way Contract in accordance with Article II, Section 3(q)(ii)
  and Section 11(i).
\item
  ``Uniform Player Contract'' or ``Player Contract'' or ``Contract''
  means the standard form of written agreement between a person and a
  Team required for use in the NBA by Article II, pursuant to which such
  person is employed by such Team as a professional basketball player.
\item
  ``Unlikely Bonus'' means Incentive Compensation excluded from a
  player's Salary in accordance with Article VII, Section 3(d).
\item
  ``Unrestricted Free Agent'' means a Free Agent who is not subject to a
  Team's right of first refusal.
\item
  ``Veteran'' or ``Veteran Player'' means a person who has signed at
  least one Player Contract with an NBA Team.
\item
  ``Veteran Free Agent'' means a Veteran who completed his Player
  Contract (other than a 10-Day Contract) by rendering the playing
  services called for thereunder.
\item
  ``Years of Service'' means the number of years of NBA service credited
  to a player in accordance with the following: a player will be
  credited with one (1) year of NBA service for each year that he is on
  an NBA Active List or Inactive List for one (1) or more days during
  the Regular Season. Notwithstanding the above, a player will not
  receive credit for a Year of Service for any year in which he: (i)
  withholds playing services called for by a Player Contract or this
  Agreement for more than thirty (30) days after the Season begins, or
  (ii) is a Restricted Free Agent, has been tendered a Qualifying Offer
  by his Prior Team and the Prior Team has extended the date by which
  the player may accept the Qualifying Offer until March 1 in accordance
  with the Article XI, Section 4(c)(i), and has not signed a Player
  Contract with any Team by March 1. In addition, notwithstanding the
  above, a player will not receive credit for a Year of Service for
  being on an NBA Active List or Inactive List as a result of signing a
  Player Contract that is disapproved by the Commissioner. In no event
  can a player be credited with more than one (1) Year of Service with
  respect to any one NBA Season. A Year of Service will be credited to a
  player on the June 30 following the Season with respect to which it is
  being credited. Under no circumstances shall the definition of Years
  of Service herein be used for purposes of determining a player's years
  of credited eligibility, benefit, and/or vesting service under any
  benefit plan or program provided for under Article IV of this
  Agreement, including, without limitation, the Pension Plan, 401(k)
  Plan, Health and Welfare Benefit Plan (including the Retiree Medical
  Plan, HRA Benefit, and tuition reimbursement program), or Post-Career
  Income Plan. Players shall be credited with Years of Service pursuant
  to this Section 1(iiii) only in respect of Seasons covered by this
  Agreement. Years of Service credit for Seasons prior to the 2005
  NBA/NBPA Collective Bargaining Agreement shall be determined in
  accordance with the provisions of the 1999 NBA/NBPA Collective
  Bargaining Agreement.
\end{enumerate}

\chapter{UNIFORM PLAYER CONTRACT}\label{uniform-player-contract}

\section{Required Form.}\label{required-form.}

The Player Contract to be entered into by each player and the Team by
which he is employed shall be a Uniform Player Contract in the form
annexed hereto as Exhibit A.

\section{Limitation on Amendments.}\label{limitation-on-amendments.}

\begin{enumerate}
\def\labelenumi{(\alph{enumi})}
\tightlist
\item
  Except as provided in Sections 3, 6, 7(d), 9, 10 and 12 of this
  Article, and in Article VII, Section 7 (Extensions, Renegotiations and
  Other Amendments) or Article XII (Option Clauses), no amendments to
  the form of Uniform Player Contract provided for by Section 1 of this
  Article shall be permitted.
\item
  Notwithstanding Section 2(a) above, except as provided: (i) in Section
  3(f), (g), (h), (j), (k), (l), (m), (n), and (p), and Section 11 of
  this Article, no amendments to Two-Way Contracts shall be permitted;
  and (ii) in Section 3(e), (h), (j), (k), (l), (m), (n), (p), and (q),
  and Section 11 of this Article, no amendments to Contracts containing
  an Exhibit 10 shall be permitted. For the avoidance of doubt, in no
  event may a Team and a player extend, renegotiate, or include an
  Option Year or Early Termination Option in a Two-Way Contract or a
  Contract containing an Exhibit 10.
\item
  If a Team and a player enter into (i) a Uniform Player Contract
  containing an amendment not specifically permitted by this Agreement
  or (ii) a subsequent amendment to an existing Player Contract where
  such amendment is not specifically permitted by this Agreement, then
  such Contract or subsequent amendment, as the case may be, shall be
  disapproved by the Commissioner and, consequently, rendered null and
  void.
\end{enumerate}

\section{Allowable Amendments.}\label{allowable-amendments.}

In their individual contract negotiations, a player and a Team may amend
the provisions of a Uniform Player Contract, but only in the following
respects:

\begin{enumerate}
\def\labelenumi{(\alph{enumi})}
\tightlist
\item
  By agreeing upon provisions (to be set forth in Exhibit 1 to a Uniform
  Player Contract) setting forth the Compensation to be paid or amounts
  to be loaned to the player for each Season of the Contract for
  rendering the services and performing the obligations described in
  such Contract.
\item
  By agreeing upon provisions (to be set forth in Exhibit 1 to a Uniform
  Player Contract) setting forth lump sum bonuses, and the payment date
  for each such bonus, to be paid as a result of: (i) the player's
  execution of a Uniform Player Contract or Extension (a ``signing
  bonus''); (ii) the player's achievement of agreed-upon benchmarks
  relating to his performance as a player or the Team's performance
  during a particular NBA Season, subject to the limitations imposed by
  paragraph 3(c) of the Uniform Player Contract and Section 12(c) below;
  or (iii) the player's achievement of agreed-upon benchmarks relating
  to his physical condition or academic achievement (e.g., earning a
  college degree or completion of a certified leadership training
  program), including the player's attendance at and participation in an
  off-season summer league and/or an off-season skill and/or
  conditioning program upon terms and conditions agreed upon by the Team
  and player (subject to the provisions of Section 12(b) below). Any
  amendment agreed upon pursuant to subsections (ii) or (iii) of this
  Section 3(b) must be structured so as to provide an incentive for
  positive achievement by the player and/or the Team; and any amendment
  agreed upon pursuant to subsection (ii) must be based upon specific
  numerical benchmarks or Generally Recognized League Honors. By way of
  example and not limitation, an amendment agreed upon pursuant to
  Section 3(b)(ii) may provide for the player to receive a bonus if his
  free-throw percentage exceeds eighty percent (80\%), but may not
  provide for the player to receive a bonus if his free-throw percentage
  improves over his previous Season's percentage. For purposes of any
  bonus agreed upon pursuant to subsection (ii), the performance
  benchmarks must be based solely upon official NBA statistics, and the
  determination of whether a player has earned any such performance
  bonus shall be made solely by reference to official NBA statistics as
  published on NBA.com.
\item
  By agreeing upon provisions (to be set forth in Exhibit 1 to a Uniform
  Player Contract) with respect to extra promotional appearances to be
  performed by the player (in addition to those required by paragraph 13
  of such Contract) and the Compensation therefor.
\item
  By agreeing upon a Compensation payment schedule (to be set forth in
  Exhibit 1 to a Uniform Player Contract) different from that provided
  for by paragraph 3(a) of the Uniform Player Contract; provided,
  however, that such amendment shall comply with the provisions of
  Section 3(b) above (relating to lump sum bonus payments) and Section
  13(e) below and, provided, further that: (i) the only such amendment
  that shall be permitted with respect to any Season in which the
  player's Compensation is not greater than the Minimum Player Salary
  shall be as described in Section 6(g) below; and (ii) the only such
  amendments that shall be permitted with respect to any Season in which
  the player's Compensation is greater than the Minimum Player Salary
  shall be as follows: (y) a Uniform Player Contract may provide for the
  player's Compensation (other than advances pursuant to clause (z)
  below and amounts paid on a deferred basis in accordance with Article
  XXV of this Agreement) to be paid in either twelve (12) equal
  semi-monthly payments or thirty-six (36) equal semi-monthly payments
  beginning with the first of said payments on November 15 of each year
  covered by the Contract and continuing with such payments on the first
  and fifteenth of each month until said Compensation is paid in full;
  and (z) a Uniform Player Contract that, at the time the Contract is
  signed, is fully or partially protected for lack of skill and injury
  or illness for a Season may provide for the player to be paid a
  portion of his Compensation for such Season, up to the Maximum Advance
  Amount as defined below, prior to November 15 of such Season. The
  Maximum Advance Amount for a Season shall equal the lesser of eighty
  percent (80\%) of the amount of the player's Compensation for such
  Season that is protected for lack of skill and injury or illness, or
  fifty percent (50\%) of the player's Base Compensation for such
  Season; provided that no more than twenty-five percent (25\%) of the
  player's Base Compensation for such Season may be paid to the player
  prior to the October 1 immediately preceding the first day of the
  Regular Season.
\item
  By agreeing upon provisions (to be set forth in Exhibit 1A to a
  Uniform Player Contract) stating that the Contract is intended to
  provide for Base Compensation equal to the Minimum Player Salary (with
  no bonuses of any kind) for each Season of the Contract for rendering
  the services and performing the obligations described in such
  Contract, in accordance with Section 6 below.
\item
  By agreeing upon provisions (to be set forth in Exhibit 1B to a
  Uniform Player Contract) (i) setting forth the Compensation to be paid
  to the player (with no bonuses of any kind) for each Season of the
  Contract for rendering the services and performing the obligations
  described in such Contract as a Two-Way Player, in accordance with
  Section 11 below (a ``Two-Way Contract''), and (ii) containing a
  Standard NBA Contract Conversion Option in accordance with Section
  11(g) below.
\item
  By agreeing upon provisions (to be set forth in Exhibit 2 to a Uniform
  Player Contract) stating that the Base Compensation provided for by a
  Uniform Player Contract (as described in Exhibit 1, 1A, or 1B to such
  Contract) shall be, in whole or in part, and subject to the standard
  conditions or limitations set forth in Section 4 below (and in the
  form of Exhibit 2) and any additional conditions or limitations that
  are negotiated by the player and Team to the extent permitted in
  accordance with Section 4(l) below, protected (as provided for by, and
  in accordance with the definitions set forth in, Section 4 below) in
  the event that such Contract is terminated by the Team by reason of
  the player's:

  \begin{enumerate}
  \def\labelenumii{(\roman{enumii})}
  \tightlist
  \item
    lack of skill;
  \item
    death not covered by an insurance policy procured by a Team for the
    player's benefit (``death'');
  \item
    disability or unfitness to play skilled basketball resulting from a
    basketball-related injury not covered by an insurance policy
    procured by a Team for the player's benefit (``basketball-related
    injury''), or disability or unfitness to play skilled basketball
    resulting from any injury or illness not covered by an insurance
    policy procured by a Team for the player's benefit (``injury or
    illness''), provided that a Contract can contain protection in only
    one of the two categories set forth in this Section 3(g)(iii);
    and/or
  \item
    mental disability not covered by an insurance policy procured by a
    Team for the player's benefit (``mental disability'').
  \end{enumerate}
\item
  By agreeing upon provisions (to be set forth in Exhibit 3 to a Uniform
  Player Contract) limiting or eliminating the player's right to receive
  his Base Compensation (in accordance with paragraphs 7(c), 16(a)(iii),
  and 16(b) of the Uniform Player Contract) when the player's disability
  or unfitness to play skilled basketball is caused by the re-injury of
  one or more injuries sustained prior to, or by the aggravation of one
  or more conditions that existed prior to, the execution of the Uniform
  Player Contract providing for such Base Compensation. Notwithstanding
  the foregoing, the provisions set forth in Exhibit 3 to a Uniform
  Player Contract shall not apply for a Season in the event such
  Contract is terminated during the period from the February 1 of such
  Season through the end of that Season.
\item
  By agreeing upon provisions (to be set forth in Exhibit 4 to a Uniform
  Player Contract) (i) entitling a player to earn Compensation if such
  player's Uniform Player Contract is traded to another NBA team, or
  (ii) prohibiting or limiting the Team's right to trade such player's
  Contract to another Team, subject, however, in either case (i) or (ii)
  to the provisions of Article XXIV.
\item
  By agreeing upon provisions (to be set forth in Exhibit 5 to a Uniform
  Player Contract) permitting the player to participate or engage in
  some or all of the activities otherwise prohibited by paragraph 12 of
  the Uniform Player Contract; provided, however, that no amendment to
  paragraph 12 of the Uniform Player Contract shall permit a player to
  participate in any public game or public exhibition of basketball not
  approved in accordance with Article XXIII of this Agreement.
\item
  By agreeing upon provisions (to be set forth in Exhibit 6 to a Uniform
  Player Contract) establishing that the player must report for and
  submit to a physical examination to be performed by a physician
  designated by the Team, subject to the provisions of Section 13(h)
  below.
\item
  By agreeing to delete paragraph 7(b) of the Uniform Player Contract in
  its entirety and substituting therefor the provision set forth in
  Exhibit 7 to a Uniform Player Contract.
\item
  By agreeing either (i) to delete paragraph 13(b) of the Uniform Player
  Contract in its entirety, or (ii) to delete the last sixteen words of
  the first sentence of paragraph 13(b) of such Contract.
\item
  By agreeing upon provisions for the purpose of terminating an
  already-existing Uniform Player Contract prior to the expiration of
  its stated term, stating as follows: (i) the Team will request waivers
  on the player in accordance with paragraph 16 of the Contract
  immediately following the Commissioner's approval of such amendment;
  and (ii) should the player clear waivers and his Contract thereupon be
  terminated (x) the amount of any Compensation protection contained in
  the Contract will immediately be reduced or eliminated, and/or (y) the
  Team's right of set-off under Article XXVII of this Agreement will be
  modified or eliminated.
\item
  By agreeing upon provisions (to be set forth in Exhibit 8 to a Uniform
  Player Contract) stating that the Contract will be traded to another
  team within forty-eight (48) hours of its execution or amendment, such
  trade and the consummation of such trade to be conditions precedent to
  the validity of the Contract or an amendment thereto; provided,
  however, that any such sign-and-trade transaction must comply with
  Article VII, Section Section 8(e).
\item
  By agreeing upon provisions (to be set forth in Exhibit 9 to a Uniform
  Player Contract) eliminating the player's right to receive his Base
  Compensation (in accordance with paragraphs 7(c), 16(a)(iii), and
  16(b) of the Uniform Player Contract) in the event the Contract is
  terminated prior to the first day of the Regular Season covered by
  such Contract; provided, however, that such amendment shall be
  permitted only if: (i) the Contract is for one (1) Season in length,
  provides for the Minimum Player Salary (with no bonuses of any kind)
  or Two-Way Salary and does not provide for Compensation protection of
  any kind pursuant to Section 3(g) above (a ``Non-Guaranteed, Training
  Camp Contract''); (ii) at the time of signing the Non-Guaranteed,
  Training Camp Contract, the Team has no fewer than fourteen (14)
  players signed to Player Contracts (not including any player signed to
  a Non-Guaranteed, Training Camp Contract) on the Team's roster in
  respect of the upcoming (or, after the first day of training camp, the
  then-current) Season; and (iii) no Team may be a party at any one time
  to more than six (6) Non-Guaranteed, Training Camp Contracts.
\item
  By agreeing upon provisions (to be set forth in Exhibit 10 to a
  Uniform Player Contract), subject to Section 11(i) below:

  \begin{enumerate}
  \def\labelenumii{(\roman{enumii})}
  \tightlist
  \item
    entitling a player to receive a bonus in an amount between \$5,000
    and \$50,000 (the ``Exhibit 10 Bonus'') if (1) the Contract is
    terminated by the Team in accordance with the NBA waiver procedure,
    and (2) the player (a) signs with the NBA Development League
    (``NBADL'') prior to the deadline set by the NBADL for NBADL teams
    to designate affiliate players, (b) is initially assigned by the
    NBADL to such Team's NBADL affiliate as listed in Exhibit 10 and
    timely reports to such affiliate, and (c) does not leave the NBADL
    (e.g., by buying out his contract with the NBADL and signing a
    contract with an international team) for a period of sixty (60) days
    after signing with the NBADL; provided, however, that an Exhibit 10
    may only contain an Exhibit 10 Bonus if the Team has an NBADL
    affiliate at the time of the execution of the Contract.
    Notwithstanding the foregoing, if a Team with an NBADL affiliate
    acquires by assignment a Contract with a Conversion Protection
    Amount but without an Exhibit 10 Bonus (the ``Acquired Exhibit
    10''), the Acquired Exhibit 10 shall be deemed to include an Exhibit
    10 Bonus equal to the Conversion Protection Amount; and
  \item
    stating that, if the Team exercises the Two-Way Player Conversion
    Option prior to the first day of the NBA Regular Season in
    accordance with Section 11(i) below, the Compensation provided for
    by the Contract will be protected for lack of skill and injury or
    illness in an amount between \$5,000 and \$50,000 (the ``Conversion
    Protection Amount''); provided, however, that if the Exhibit 10
    contains an Exhibit 10 Bonus, the Exhibit 10 must also contain a
    Conversion Protection Amount and the Conversion Protection Amount
    must be equal to the Exhibit 10 Bonus.
  \end{enumerate}

  In the event that NBADL rules permit a Team, other than the Team that
  last requested waivers on the player, to designate the player as an
  affiliate player (the ``Designating Team''), the Designating Team
  shall be responsible for paying the Exhibit 10 Bonus to the player
  provided that (a) the Designating Team designates the player as an
  affiliate player, (b) prior to the waiver, the Designating Team was a
  party to the Contract containing the Exhibit 10 Bonus, and (c) the
  player satisfies the conditions set forth in Section 3(q)(i) above
  with respect to the Designating Team's NBADL affiliate.

  No Team may (a) be a party at any one time to more than six (6)
  Contracts containing an Exhibit 10, or (b) enter into a Player
  Contract with an Exhibit 10 unless such Contract is for one (1) Season
  in length, provides for the Minimum Player Salary (with no bonuses of
  any kind other than the Exhibit 10 Bonus), and does not provide for
  Compensation protection of any kind pursuant to Section 3(g) above
  (other than in connection with the Two-Way Player Conversion Option).

  A Team may enter into a Contract with both an Exhibit 9 and an Exhibit
  10 in accordance with the preceding terms; provided, however, that if
  a Team exercises its Two-Way Player Conversion Option, the Contract's
  Exhibit 9 shall be rendered null and void and of no further force or
  effect upon the exercise of such Two-Way Player Conversion Option.
\end{enumerate}

\section{Compensation Protection.}\label{compensation-protection.}

\begin{enumerate}
\def\labelenumi{(\alph{enumi})}
\tightlist
\item
  \textbf{Lack of Skill.} When a Team agrees to protect, in whole or in
  part, the Base Compensation provided for by a Uniform Player Contract
  in the event such Contract is terminated by the Team, pursuant to
  paragraph 16(a)(iii) thereof, by reason of the player's lack of skill,
  such agreement shall mean that, subject to any conditions or
  limitations set forth in this Section 4(a) or Exhibit 2 to the Uniform
  Player Contract, or expressly set forth elsewhere in this Agreement,
  notwithstanding the provisions of paragraphs 16(a)(iii), 16(d), 16(e),
  and 16(g) of such Contract, the termination of such Contract by the
  Team on account of the player's failure to exhibit sufficient skill or
  competitive ability shall in no way affect the player's right to
  receive, in whole or in part, the Base Compensation payable pursuant
  to Exhibit 1 to such Contract in the amounts and at the times called
  for by such Exhibit; provided, however, that: (i) such lack of skill
  does not result from the player's participation in activities
  prohibited by paragraph 12 of the Uniform Player Contract (as such
  paragraph may be modified by Exhibit 5 to the Player Contract),
  attempted suicide, intentional self-inflicted injury, abuse of
  alcohol, use of any Prohibited Substance or controlled substance,
  abuse of or addiction to prescription drugs, conduct occurring during
  the commission of any felony for which the player is convicted
  (including a plea of guilty, no contest or nolo contendere),
  participation in any riot, insurrection or war or other military
  activities, or failure to comply with the requirements of Paragraphs
  7(d)-(i) of the Uniform Player Contract; (ii) at the time of the
  player's failure to render playing services, the player is not in
  material breach of such Contract; (iii) if the Team, for its own
  benefit, seeks to procure an insurance policy covering the player's
  lack of skill, the player cooperates with the Team in procuring such
  an insurance policy, including by, among other things, supplying all
  information requested of him, completing application forms, or
  otherwise, and submitting to all examinations and tests requested of
  him by or on behalf of the insurance company in connection with the
  Team's efforts to procure such policy; and (iv) if the Team, for its
  own benefit, has procured such an insurance policy, the player
  cooperates (in the manner described above) with the Team and insurance
  company in the processing of the Team's claim under such policy.
\item
  \textbf{Death.} When a Team agrees to protect, in whole or in part,
  the Base Compensation provided for by a Uniform Player Contract in the
  event such Contract is terminated by the Team, pursuant to paragraph
  16(a)(iv) thereof, by reason of the player's failure to render his
  services thereunder, if such failure has been caused by the player's
  death, such agreement shall mean that, subject to any conditions or
  limitations set forth in this Section 4(b) or Exhibit 2 to the Uniform
  Player Contract, or expressly set forth elsewhere in this Agreement,
  notwithstanding the provisions of paragraphs 16(a)(iii), 16(b), 16(c),
  16(d), 16(e), and 16(g) of such Contract, the termination of such
  Contract by the Team shall in no way affect the player's (or his
  estate's or duly appointed beneficiary's) right to receive, in whole
  or in part, the Base Compensation payable pursuant to Exhibit 1 to
  such Contract in the amounts and at the times called for by such
  Exhibit; provided, however, that: (i) such death does not result from
  the player's participation in activities prohibited by paragraph 12 of
  the Uniform Player Contract (as such paragraph may be modified by
  Exhibit 5 to the Player Contract), suicide, intentional self-inflicted
  injury, abuse of alcohol, use of any Prohibited Substance or
  controlled substance, abuse of or addiction to prescription drugs,
  conduct occurring during the commission of any felony for which the
  player is convicted (including a plea of guilty, no contest or nolo
  contendere), participation in any riot, insurrection or war or other
  military activities, or failure to comply with the requirements of
  Paragraphs 7(d)-(i) of the Uniform Player Contract; (ii) at the time
  of the player's failure to render playing services, the player is not
  in material breach of such Contract; (iii) if the Team, for its own
  benefit, seeks to procure an insurance policy covering the player's
  death, the player cooperates with the Team in procuring such an
  insurance policy, including by, among other things, supplying all
  information requested of him, completing application forms, or
  otherwise, and submitting to all examinations and tests requested of
  him by or on behalf of the insurance company in connection with the
  Team's efforts to procure such policy; and (iv) if the Team, for its
  own benefit, has procured such an insurance policy, the player's
  estate and/or duly appointed beneficiary cooperates (in the manner
  described above) with the Team and insurance company in the processing
  of the Team's claim under such policy.
\item
  \textbf{Basketball-Related Injury.} When a Team agrees to protect, in
  whole or in part, the Base Compensation provided for by a Uniform
  Player Contract in the event such Contract is terminated by the Team,
  pursuant to paragraph 16(a)(iv) thereof, by reason of the player's
  failure to render his services thereunder, if such failure has been
  caused by the player's disability and/or unfitness to play skilled
  basketball as a direct result of an injury sustained while
  participating in any basketball practice or game played for the Team,
  such agreement shall mean that, subject to any conditions or
  limitations set forth in this Section 4(c) or Exhibit 2 to the Uniform
  Player Contract, or expressly set forth elsewhere in this Agreement,
  notwithstanding the provisions of paragraphs 7(b), 7(c), 16(a)(iii),
  16(b), 16(c), 16(d), and 16(g) of such Contract, the termination of
  such Contract by the Team shall in no way affect the player's right to
  receive, in whole or in part, the Base Compensation payable pursuant
  to Exhibit 1 to such Contract in the amounts and at the times called
  for by such Exhibit; provided, however, that: (i) such injury does not
  result from the player's participation in activities prohibited by
  paragraph 12 of the Uniform Player Contract (as such paragraph may be
  modified in Exhibit 5 to the Player Contract), attempted suicide,
  intentional self-inflicted injury, abuse of alcohol, use of any
  Prohibited Substance or controlled substance, abuse of or addiction to
  prescription drugs, conduct occurring during the commission of any
  felony for which the player is convicted (including a plea of guilty,
  no contest or nolo contendere), participation in any riot,
  insurrection or war or other military activities, or failure to comply
  with the requirements of Paragraphs 7(d)-(i) of the Uniform Player
  Contract; (ii) at the time of the player's termination, the player is
  not in material breach of such Contract; (iii) if the Team, for its
  own benefit, seeks to procure an insurance policy covering the
  player's injury, the player cooperates with the Team in procuring such
  an insurance policy, including by, among other things, supplying all
  information requested of him, completing application forms, or
  otherwise and submitting to all examinations and tests requested of
  him by or on behalf of the insurance company in connection with the
  Team's efforts to procure such policy; and (iv) if the Team, for its
  own benefit, has procured such an insurance policy, the player
  cooperates (in the manner described above) with the Team and the
  insurance company in the processing of the Team's claim under such
  policy.
\item
  \textbf{Injury or Illness.} When a Team agrees to protect, in whole or
  in part, the Base Compensation provided for by a Uniform Player
  Contract in the event such contract is terminated by the Team,
  pursuant to paragraph 16(a)(iv) thereof, by reason of the player's
  failure to render his services thereunder, if such failure has been
  caused by an injury, illness, or disability suffered or sustained by
  the player, such agreement shall mean that, subject to any conditions
  or limitations set forth in this Section 4(d) or Exhibit 2 to the
  Uniform Player Contract, or expressly set forth elsewhere in this
  Agreement, notwithstanding the provisions of paragraphs 7(b), 7(c),
  16(a)(iii), 16(b), 16(c), 16(d), and 16(g) of such Contract, the
  termination of such Contract by the Team shall in no way affect the
  player's right to receive, in whole or in part, the Base Compensation
  payable pursuant to Exhibit 1 to such Contract in the amounts and at
  the times called for by such Exhibit; provided, however, that: (i)
  such injury, illness, or disability does not result from the player's
  participation in activities prohibited by paragraph 12 of the Uniform
  Player Contract (as such paragraph may be modified in Exhibit 5 to the
  Player Contract), attempted suicide, intentional self-inflicted
  injury, abuse of alcohol, use of any Prohibited Substance or
  controlled substance, abuse of or addiction to prescription drugs,
  conduct occurring during the commission of any felony for which the
  player is convicted (including a plea of guilty, no contest or nolo
  contendere), participation in any riot, insurrection or war or other
  military activities, or failure to comply with the requirements of
  Paragraphs 7(d)-(i) of the Uniform Player Contract; (ii) at the time
  of such injury, illness, or disability the player is not in material
  breach of such Contract; (iii) if the Team, for its own benefit, seeks
  to procure an insurance policy covering the player's injury and/or
  illness, the player cooperates with the Team in procuring such an
  insurance policy, including by, among other things, supplying all
  information requested of him, completing application forms, or
  otherwise and submitting to all examinations and tests requested of
  him by or on behalf of the insurance company in connection with the
  Team's efforts to procure such policy; and (iv) if the Team, for its
  own benefit, has procured such an insurance policy, the player
  cooperates (in the manner described above) with the Team and insurance
  company in the processing of the Team's claim under such policy.
\item
  \textbf{Mental Disability.} When a Team agrees to protect, in whole or
  in part, the Base Compensation provided for by a Uniform Player
  Contract in the event such Contract is terminated by the Team,
  pursuant to paragraph 16(a)(iv) thereof, by reason of the player's
  failure to render his services thereunder, if such failure has been
  caused by the player's mental disability, such agreement shall mean
  that, subject to any conditions or limitations set forth in this
  Section 4(e) or Exhibit 2 to the Uniform Player Contract, or expressly
  set forth elsewhere in this Agreement, notwithstanding the provisions
  of paragraphs 16(a)(iii), 16(b), 16(c), 16(d), 16(e), and 16(g) of
  such Contract, the termination of such Contract by the Team shall in
  no way affect the player's (or his duly appointed legal
  representative's) right to receive, in whole or in part, the Base
  Compensation payable pursuant to Exhibit 1 to such Contract in the
  amounts and at the times called for by such Exhibit; provided,
  however, that: (i) such mental disability does not result from the
  player's participation in activities prohibited by paragraph 12 of the
  Uniform Player Contract (as such paragraph may be modified in Exhibit
  5 to the Player Contract), attempted suicide, intentional
  self-inflicted injury, the use of any Prohibited Substance or
  controlled substance, abuse of or addiction to prescription drugs,
  conduct occurring during the commission of any felony for which the
  player is convicted (including a plea of guilty, no contest or nolo
  contendere), participation in any riot, insurrection or war or other
  military activities, or failure to comply with the requirements of
  Paragraphs 7(d)-(i) of the Uniform Player Contract; (ii) at the time
  of the player's failure to render playing services, the player is not
  in material breach of such Contract; (iii) if the Team, for its own
  benefit, seeks to procure an insurance policy covering the player's
  mental disability, the player (and/or his duly appointed legal
  representative) cooperates with the Team in procuring such an
  insurance policy, including by, among other things, supplying all
  information requested of him, completing application forms, or
  otherwise and submitting to all examinations and tests requested of
  him by the insurance company in connection with the Team's efforts to
  procure such policy; and (iv) if the Team, for its own benefit, has
  procured such an insurance policy, the player (and/or his duly
  appointed legal representative) cooperates (in the manner described
  above) with the Team and insurance company in the processing of the
  Team's claim under such policy.
\item
  No agreement by a Team to protect, in whole or in part, the Base
  Compensation provided for by a Uniform Player Contract shall require
  (or be construed as requiring) such Team to continue to employ the
  player (whether on the Active List, Inactive List, Two-Way List or
  otherwise); nor shall any such agreement afford the player any right
  to be employed, or to be deemed as having been employed, by such Team
  for any purpose.
\item
  Notwithstanding any other provision of this Agreement, when a Team
  agrees to protect, in whole or in part, the Base Compensation provided
  for by a Uniform Player Contract, and such protection is contingent on
  the satisfaction of a condition expressly set forth in Exhibit 2 to
  that Contract, such protection shall be applicable and effective only
  if the Player Contract has not previously been terminated at the time
  such condition is satisfied.
\item
  Notwithstanding any other provision of this Agreement, when a Team
  agrees to protect, in whole or in part, the Base Compensation provided
  for in any Option Year in favor of the Team included in a Uniform
  Player Contract, such protection shall be applicable and effective
  only if the option to extend the term provided for in the Contract was
  exercised by the Team prior to the termination of the Contract. When a
  Team agrees to protect, in whole or in part, the Base Compensation
  provided for in any Option Year in favor of the player, the
  applicability of such protection in the circumstance where the Option
  has not been exercised by the player shall be governed by the
  provisions of Article XII, Section 2(a).
\item
  During the term of a Player Contract, the percentage of protected Base
  Compensation for any future Season shall not exceed the percentage of
  unearned protected Base Compensation for any prior Season. Thus, for
  example, a Team could not provide for fifty percent (50\%) Base
  Compensation protection in the first Season of a Player Contract and
  one hundred percent (100\%) Base Compensation protection in the second
  Season of the Contract. However, the foregoing rule does not prevent a
  Team from providing a percentage of Base Compensation protection in a
  future Season that is higher than in a prior Season if the higher
  level of Base Compensation for the future Season is conditional and
  the condition cannot be satisfied until the completion of the prior
  Season. For example, it is permissible for a Contract to provide that
  Base Compensation protection for the first Season of a Player Contract
  equals fifty percent (50\%) and Base Compensation protection for the
  second Season will be increased from fifty percent (50\%) to one
  hundred percent (100\%) if the player is on the Team's roster as of
  the August 1 prior to the second Season of the Player Contract.
\item
  With respect to Player Contracts entered into or extended on or after
  the effective date of this Agreement (other than Player Contracts that
  provide in any Season for the player to earn Compensation not greater
  than his applicable Minimum Player Salary):

  \begin{enumerate}
  \def\labelenumii{(\roman{enumii})}
  \tightlist
  \item
    The maximum amount of aggregate Base Compensation that can be
    protected for death is thirty million dollars (\$30,000,000).
  \item
    If a player elects to purchase term life insurance for his benefit,
    his Team shall be permitted to reimburse him each Season for the
    premiums paid for such insurance with respect to such Season and any
    other future Season(s); provided, however, that:

    \begin{enumerate}
    \def\labelenumiii{(\Alph{enumiii})}
    \tightlist
    \item
      The amount of coverage for which premiums are reimbursed by the
      Team in any Season shall not exceed the lesser of (x) the
      aggregate amount of the player's unearned Base Compensation for
      such Season and each remaining Season (excluding an Option Year if
      not yet exercised) that is not protected for death, and (y) the
      difference between (i) seventy million dollars (\$70,000,000) and
      (ii) the aggregate amount of the player's unearned Base
      Compensation for such Season and each remaining Season (excluding
      an Option Year if not yet exercised) that is protected for death.
    \item
      Any such premium reimbursement shall not exceed the cost for
      10-year guaranteed term coverage at preferred rates.
    \end{enumerate}
  \item
    If a Contract contains death protection covering ten million dollars
    (\$10,000,000) or more of Base Compensation, the player shall be
    precluded from purchasing life insurance for a period of ninety (90)
    days following the execution or extension (as applicable) of the
    Contract or until such earlier time as the Team notifies the player
    in writing that it is no longer attempting to purchase life
    insurance coverage on the player (up to the amount of the player's
    Base Compensation protection for death) for the Team's benefit.
    During such ninety (90) day period or until such time as the Team
    issues the foregoing written notification to the player, the Team's
    efforts to purchase life insurance on the player for the Team's
    benefit shall be conducted diligently and in good faith.
  \end{enumerate}
\item
  In the event that a Team terminates a Player Contract (resulting in
  the player's separation of service from the Team), and the Team is
  obligated thereafter to make payments to the player pursuant to
  Exhibit 2 of the Contract, such payments shall be made in accordance
  with the following schedule:

  \begin{enumerate}
  \def\labelenumii{(\roman{enumii})}
  \tightlist
  \item
    If, as of the date of the player's separation from service, the
    aggregate Base Compensation owed to the player pursuant to Exhibit 2
    of the Contract is two hundred fifty thousand dollars (\$250,000) or
    less, such amount shall be paid in accordance with the semi-monthly
    installments prescribed by the payment schedule set forth in the
    Contract. Each installment shall equal the amount of Base
    Compensation that was due per pay period for the applicable Season
    immediately before the Player's separation until the aggregate
    amount of the remaining Base Compensation owed to the player
    pursuant to Exhibit 2 of the Contract is paid in full.
  \item
    If, as of the date of the player's separation from service, the
    aggregate Base Compensation owed to the player pursuant to Exhibit 2
    of the Contract exceeds two hundred fifty thousand dollars
    (\$250,000), such amount shall be paid as follows:

    \begin{enumerate}
    \def\labelenumiii{(\alph{enumiii})}
    \setcounter{enumiii}{23}
    \tightlist
    \item
      The Base Compensation, if any, owed to the player pursuant to
      Exhibit 2 of the Contract with respect to the ``current season''
      (as defined below) at the time when the request for waivers on the
      player is made shall be paid in accordance with the payment
      schedule set forth in the Contract. Each installment shall equal
      the amount of Base Compensation that was due per pay period
      immediately before the player's separation until the aggregate
      amount of the remaining Base Compensation owed to the player
      pursuant to Exhibit 2 of the Contract with respect to the current
      season is paid in full. For purposes of this subparagraph 2 only,
      the ``current season'' means the period from September 1 through
      June 30.
    \item
      The remaining Base Compensation, if any, owed to the player
      pursuant to Exhibit 2 of the Contract shall be aggregated and paid
      in equal amounts per year over a period equal to twice the number
      of NBA Seasons (including any Season covered by a Player Option
      Year) remaining on this Contract following the date upon which the
      request for waivers occurred, plus one NBA Season. For this
      purpose, if the request for waivers is made during the period from
      September 1 through June 30, the number of NBA Seasons remaining
      on this Contract shall not include the current season (as defined
      in subparagraph (x) above). The rescheduled payments described
      above shall be paid over the applicable number of NBA Seasons in
      equal semi-monthly installments on the pay dates prescribed by
      paragraph 3(a) of the Uniform Player Contract. The following
      example is for clarity. A player has four Seasons remaining on his
      Contract with protected Base Compensation of the following
      amounts: \$4 million in Season 1, \$4.3 million in Season 2, \$4.7
      million in Season 3, and \$5 million in Season 4. The player is
      waived on December 1 of Season 1. Under Section 4(k)(ii)(x) above,
      the player would receive the remainder of his \$4 million in Base
      Compensation for Season 1 in accordance with the payment schedule
      set forth in his Contract. Under Section 4(k)(ii)(y) above, the
      \$14 million of protected Base Compensation remaining to be paid
      for Seasons 2 - 4 of the Contract would be paid at a rate of \$2
      million per Season for the next seven (7) Seasons in accordance
      with the payment schedule set forth in paragraph 3 of the
      Contract. If the same player is instead waived on July 30 of
      Season 1, the \$18 million of protected Base Compensation
      remaining to be paid for Seasons 1 - 4 of the Contract would be
      paid -- under Section 4(k)(ii) above -- at a rate of \$2 million
      per Season for the next nine (9) Seasons in accordance with the
      payment schedule set forth in paragraph 3 of the Contract.
    \end{enumerate}
  \end{enumerate}
\item
  With respect to Player Contracts entered into or extended on or after
  the effective date of this Agreement (but in the case of Extensions
  only with respect to the extended term), in addition to the standard
  conditions or limitations set forth in Section 4 above (as set forth
  in the form of Exhibit 2 to the Uniform Player Contract), a Team and a
  player are authorized under Article II, Section 4(a)-(e) to negotiate
  additional conditions or limitations applicable to the player's
  Compensation protection for such categories as the Team and player
  agree to protect that relate to only the following: (i) whether the
  Team waives a player by a certain time (e.g., providing that a
  player's Base Compensation protection increases if the Team does not
  request waivers on the player by a certain date); (ii) achievement of
  certain benchmarks relating to Team and/or player performance or a
  player's physical condition (e.g., providing that a player's Base
  Compensation protection increases if the player achieves certain
  performance criteria or meets specified weigh-in criteria), provided
  that any such performance benchmarks must be based solely upon
  official NBA statistics, and the determination of whether a player has
  met any such performance benchmark shall be made solely by reference
  to official NBA statistics as published on NBA.com; (iii) a player
  experiencing a particular injury, illness or other medical condition
  (e.g., providing that a player's Base Compensation protection does not
  apply if the Team terminates a Contract due to an injury to a player's
  left knee); and (iv) the Team's ability to obtain insurance, using
  best efforts, of a certain type and dollar amount within a specified
  period of time following execution or extension (as applicable) of the
  Contract. Other than the standard conditions or limitations set forth
  in Section 4 above (as set forth in the form of Exhibit 2 to the
  Uniform Player Contract) and any individually-negotiated conditions or
  limitations in accordance with this Section (l), no Player Contract
  entered into or extended on or after the effective date of this
  Agreement (but in the case of Extensions only with respect to the
  extended term) may contain any additional condition or limitation of
  any kind on a player's Compensation protection. For clarity, with
  respect to Player Contracts entered into prior to the effective date
  of this Agreement (but in the case of Player Contracts entered into
  prior to the effective date of this Agreement that are extended on or
  after the effective date of this Agreement, only with respect to the
  years remaining on the Contract as of the effective date of this
  Agreement), the Compensation protection in such Contracts shall be
  subject to the standard conditions or limitations set forth in Section
  4 of the 2011 NBA/NBPA Collective Bargaining Agreement and any
  individually-negotiated conditions or limitations contained in the
  applicable Contract.
\end{enumerate}

\section{Conformity.}\label{conformity.}

\begin{enumerate}
\def\labelenumi{(\alph{enumi})}
\tightlist
\item
  All currently effective Player Contracts, and all Player Contracts
  entered into on or after the effective date of this Agreement that do
  not otherwise so provide, shall be deemed amended in such manner to
  require the parties to comply with all terms of this Agreement,
  including the terms of the Uniform Player Contract annexed hereto as
  Exhibit A. All Player Contracts shall be subject to the terms of this
  Agreement, which shall supersede the terms of any Player Contract
  inconsistent herewith. No Player Contract shall provide for the waiver
  by a player or a Team of any benefits or the sacrifice of any rights
  to which the player or the Team is entitled by virtue of a Uniform
  Player Contract or this Agreement.
\item
  Notwithstanding Section 5(a) above, no Player Contract entered into
  prior to the effective date of this Agreement shall be affected by any
  provisions of this Agreement expressly indicating that they apply only
  to Player Contracts entered into on or after the effective date of
  this Agreement.
\end{enumerate}

\section{Minimum Player Salary.}\label{minimum-player-salary.}

\begin{enumerate}
\def\labelenumi{(\alph{enumi})}
\tightlist
\item
  Except with respect to 10-Day Contracts provided for in Section 9
  below, Rest-of-Season Contracts provided for in Section 10 below, and
  Two-Way Contracts provided for in Section 11 below, no Player Contract
  shall provide for a Salary of less than the applicable scale amount
  contained in the Minimum Annual Salary Scale applicable for such
  Salary Cap Year. For the 2017-18 Salary Cap Year, the Minimum Annual
  Salary Scale is set forth as Exhibit C hereto. For each Salary Cap
  Year commencing with the 2018-19 Salary Cap Year, the Minimum Annual
  Salary Scale amounts shall be adjusted by applying the percentage
  increase (or decrease) in the Salary Cap from the preceding Salary Cap
  Year to the current Salary Cap Year. The Minimum Annual Salary Scale
  applicable to a player is determined by the Salary Cap Year
  encompassing the first Season covered by the player's Contract.
  Accordingly, for example, if the first Season covered by a player's
  Contract is the 2018-19 Season, then the Minimum Annual Salary Scale
  for the 2018-19 Salary Cap Year shall apply for each Season of the
  Contract.
\item
  No 10-Day Contract or Rest-of-Season Contract (as those terms are
  defined in Sections 9 and 10 below) shall provide for a Salary of less
  than the Minimum Player Salary applicable to that player.
\item
  In determining whether a Player Contract provides for a Salary of no
  less than the Minimum Player Salary applicable to that player, the
  allocation of a deemed signing bonus in respect of an ``international
  player payment'' in excess of the Excluded International Player
  Payment Amount for such Salary Cap Year as set forth in Article VII,
  Section 3(e) (but no other bonuses) shall be considered as part of the
  Salary provided for by a Player Contract, provided that such Player
  Contract makes clear that the Salary for each Season (including the
  allocation of any such deemed signing bonus) equals or exceeds the
  Minimum Player Salary for such Season.
\item
  On July 1, 2017, any Player Contract entered into before the effective
  date of this Agreement that is not terminated on or before June 30,
  2017 and provides for a Salary for the 2017-18 Season or any future
  Season that is less than the applicable Minimum Player Salary for such
  Season as set forth in the Minimum Annual Salary Scale for the 2017-18
  Salary Cap Year (attached as Exhibit C hereto) shall be deemed amended
  to provide for the applicable Minimum Player Salary for such Season(s)
  under such Minimum Annual Salary Scale based upon the player's Years
  of Service for each of the applicable Seasons(s). On July 1 of each
  Salary Cap Year, any Player Contract (other than a Two-Way Contract)
  entered into on or after the effective date of this Agreement that
  provides for a Salary for the upcoming Season that is less than the
  applicable Minimum Player Salary based on the Minimum Annual Salary
  Scale applicable to the player shall be deemed amended to provide for
  the applicable Minimum Player Salary based on such Minimum Annual
  Salary Scale.
\item
  Nothing in this Section 6 shall alter the respective rights and
  liabilities of a player and a Team, as provided for in the Uniform
  Player Contract or in this Agreement, with respect to the termination
  of a Player Contract.
\item
  Every Contract entered into between a player and Team that is intended
  to provide for Compensation equal to the Minimum Player Salary (with
  no bonuses of any kind) for each Season must contain the following
  sentence in Exhibit 1A of such Contract and shall be deemed amended in
  the manner described in such sentence: ``This Contract is intended to
  provide for a Base Compensation for the \_\_\_\_\_\_\_\_\_\_\_\_
  Season(s) equal to the Minimum Player Salary for such Season(s) (with
  no bonuses of any kind) and shall be deemed amended to the extent
  necessary to so provide.'' The reference in the preceding sentence to
  ``no bonuses of any kind'' shall not be construed to limit the ability
  of a Team and player (i) to agree upon provisions entitling a player
  to earn Compensation if such player's Uniform Player Contract is
  traded to another NBA team in accordance with Section 3(i) above, or
  (ii) to enter into a Contract with an Exhibit 10 Bonus, subject to the
  limitations in Section 3(q) above and Section 11(i) below.
\item
  A Uniform Player Contract (other than a Two-Way Contract) that
  provides in any Season for the player to earn Compensation not greater
  than his applicable Minimum Player Salary (with no bonuses of any
  kind) that, at the time the Contract is signed, is fully or partially
  protected for lack of skill and injury or illness may be amended to
  provide for the player to be paid a portion of his Compensation for
  such Season (the ``Advance''), up to the Minimum Player Salary Advance
  Limit as defined below, prior to November 15 of such Season. The
  Minimum Player Salary Advance Limit for a Season shall equal the
  lesser of (i) eighty percent (80\%) of the amount of the player's
  Compensation for such Season that is protected for lack of skill and
  injury or illness, or (ii) seven and one half percent (7.5\%) of the
  player's Base Compensation for such Season. Any Advance paid to a
  player for a Season pursuant to the foregoing must be deducted in full
  from the first installment of Current Base Compensation (i.e., on
  November 15) and, if necessary after reducing in full the first
  installment, the second installment of Current Base Compensation
  (i.e., on December 1) for such Season that the player would have
  received pursuant to paragraph 3(a) of the Contract had there been no
  such Advance. To effectuate the requirement set forth in the preceding
  sentence, every such Contract that provides for an Advance must
  contain the following language (and only such language) under the
  ``Payment Schedule'' heading in Exhibit 1A with respect to each
  applicable Season: ``Player's Current Base Compensation with respect
  to the \_\_\_\_\_\_\_\_\_ Season(s) shall be paid in accordance with
  paragraph 3(a), except that the November 15 installment of such
  Current Base Compensation and, if necessary after reducing in full the
  November 15 installment, the December 1 installment of such Current
  Base Compensation shall be reduced by \${[}amount of Advance{]}, which
  amount shall be paid to Player in advance on {[}date{]}.''
\end{enumerate}

\section{Maximum Annual Salary.}\label{maximum-annual-salary.}

\begin{enumerate}
\def\labelenumi{(\alph{enumi})}
\tightlist
\item
  Notwithstanding any other provision of this Agreement, no Player
  Contract entered into on or after the effective date of this Agreement
  may provide for a Salary plus Unlikely Bonuses in the first Season
  covered by the Contract that exceeds the following amounts:

  \begin{enumerate}
  \def\labelenumii{(\roman{enumii})}
  \tightlist
  \item
    for any player who has completed fewer than seven (7) Years of
    Service, the greater of (x) twenty-five percent (25\%) of the Salary
    Cap in effect at the time the Contract is executed, or (y) one
    hundred five percent (105\%) of the Salary for the final Season of
    the player's prior Contract; provided, however, that a player who
    has four (4) Years of Service as of the June 30 following the end of
    the last Season covered by his Player Contract (``5th Year Eligible
    Players'') shall be eligible to receive from his Prior Team up to
    thirty percent (30\%) of the Salary Cap in effect at the time the
    Contract is executed (the ``5th Year 30\% Max Salary'') if the
    player has met the following criteria (the ``5th Year 30\% Max
    Criteria''):

    \begin{enumerate}
    \def\labelenumiii{(\Alph{enumiii})}
    \tightlist
    \item
      With respect to Player Contracts in which the first Season covered
      by the Contract (or, in the case of Extensions governed by Section
      7(c) below, the extended term) is the 2017-18 Season, the player
      has met at least one of the following criteria during his first
      four Seasons: (i) the player is named twice to the All-NBA first,
      second, or third team; (ii) the player is voted in twice as an
      All-Star starter; or (iii) the player is designated once as NBA
      MVP; and
    \item
      With respect to Player Contracts in which the first Season covered
      by the Contract (or, in the case of Extensions governed by Section
      7(c) below, the extended term) is the 2018-19 Season or a later
      Season, the player has met at least one of the following criteria
      as of the July 1 following the player's fourth Season: (i) the
      player was named to the All-NBA first, second, or third team, or
      was named Defensive Player of the Year, in the immediately
      preceding Season or in two (2) Seasons during the immediately
      preceding three (3) Seasons; or (ii) the player was named NBA MVP
      during one of the immediately preceding three (3) Seasons;
    \end{enumerate}
  \item
    for any player who has completed at least seven (7) but fewer than
    ten (10) Years of Service, the greater of (x) thirty percent (30\%)
    of the Salary Cap in effect at the time the Contract is executed, or
    (y) one hundred five percent (105\%) of the Salary for the final
    Season of the player's prior Contract; provided, however, that a
    player who has eight (8) or nine (9) Years of Service at the time
    the Contract is executed and rendered such Years of Service for the
    Team with which he first executed a Player Contract (or, if he was
    under a Player Contract for more than one Team during such period,
    changed Teams only by trade during the first four (4) Salary Cap
    Years in which he was under a Player Contract) shall be eligible to
    enter into a Designated Veteran Player Contract pursuant to which he
    receives from his Prior Team up to thirty-five percent (35\%) of the
    Salary Cap in effect at the time the Contract is executed (the
    ``Designated Veteran Player 35\% Max Salary'') if the player has met
    at least one of the following criteria at the time his Contract is
    executed: (i) the player was named to the All-NBA first, second, or
    third team, or was named Defensive Player of the Year, in the
    immediately preceding Season or in two (2) Seasons during the
    immediately preceding three (3) Seasons; or (ii) the player was
    named NBA MVP during one of the immediately preceding three (3)
    Seasons (the ``Designated Veteran Player 35\% Max Criteria''); or
  \item
    for any player who has completed ten (10) or more Years of Service,
    the greater of (x) thirty-five percent (35\%) of the Salary Cap in
    effect at the time the Contract is executed, or (y) one hundred five
    percent (105\%) of the Salary for the final Season of the player's
    prior Contract.
  \end{enumerate}
\item
  Notwithstanding any other provision of this Agreement, no
  Renegotiation may provide for a Salary plus Unlikely Bonuses in the
  Renegotiation Season (as defined in Article VII, Section 7(c)) that
  exceeds the following amounts:

  \begin{enumerate}
  \def\labelenumii{(\roman{enumii})}
  \tightlist
  \item
    for any player who has completed fewer than seven (7) Years of
    Service, the greater of (x) twenty-five percent (25\%) of the Salary
    Cap in effect at the time the Renegotiation is executed, or (y) one
    hundred five percent (105\%) of the Salary for the Season prior to
    the Renegotiation Season;
  \item
    for any player who has completed at least seven (7) but fewer than
    ten (10) Years of Service, the greater of (x) thirty percent (30\%)
    of the Salary Cap in effect at the time the Renegotiation is
    executed, or (y) one hundred five (105\%) of the Salary for the
    Season prior to the Renegotiation Season; or
  \item
    for any player who has completed ten (10) or more Years of Service,
    the greater of (x) thirty-five percent (35\%) of the Salary Cap in
    effect at the time the Renegotiation is executed, or (y) one hundred
    five percent (105\%) of the Salary for the Season prior to the
    Renegotiation Season.
  \end{enumerate}
\item
  The parties recognize that it may not be possible to ascertain at the
  time an Extension is executed whether the Salary plus Unlikely Bonuses
  called for in the first Season of the extended term will exceed the
  Maximum Annual Salary set forth in this Section 7. Accordingly, and
  notwithstanding any other provision of this Agreement, the following
  rule shall apply to any Extension in which the extended term begins on
  or after the effective date of this Agreement: if, on July 1 of the
  Salary Cap Year encompassing the first Season of the extended term of
  such Extension, the Salary plus Unlikely Bonuses provided for in such
  Season exceeds the following amounts:

  \begin{enumerate}
  \def\labelenumii{(\roman{enumii})}
  \tightlist
  \item
    for any player who has completed fewer than seven (7) Years of
    Service, the greater of (x) twenty-five percent (25\%) of the Salary
    Cap in effect on July 1 of the Salary Cap Year encompassing the
    first Season of the extended term of such Extension, or (y) one
    hundred five percent (105\%) of the Salary provided for in the final
    Season of the original term of the Contract; provided, however, that
    a 5th Year Eligible Player who signed a Rookie Scale Extension in
    accordance with Section 7(d) below shall be eligible to receive the
    percentage that is agreed upon by the Team and player, which shall
    be no less than twenty-five percent (25\%) or greater than thirty
    percent (30\%) of the Salary Cap in effect on July 1 of the Salary
    Cap Year encompassing the first Season of the extended term of such
    Extension if the player has met at least one of the 5th Year 30\%
    Max Criteria;
  \item
    for any player who has completed at least seven (7) but fewer than
    ten (10) Years of Service, the greater of (x) thirty percent (30\%)
    of the Salary Cap in effect on July 1 of the Salary Cap Year
    encompassing the first Season of the extended term of such
    Extension, or (y) one hundred five percent (105\%) of the Salary
    provided for in the final Season of the original term of the
    Contract; provided, however, that a player who (A) has one Season,
    or two Seasons (including any Option Year), remaining on his
    Contract, and (B) has seven (7) or eight (8) Years of Service at the
    time the Extension is executed (i.e., a player entering their 8th or
    9th year in the NBA), and (C) rendered such Years of Service for the
    Team with which he first executed a Player Contract (or, if he was
    under a Player Contract for more than one Team during such period,
    changed Teams only by trade during the first four (4) Salary Cap
    Years in which he was under a Player Contract) shall be eligible to
    enter into a Designated Veteran Player Extension pursuant to which
    the player receives the percentage that is agreed upon by the Team
    and player, which shall be no less than thirty percent (30\%) and no
    greater than thirty-five percent (35\%) of the Salary Cap in effect
    on July 1 of the Salary Cap Year encompassing the first Season of
    the extended term of such Extension if the player has met at least
    one of the Designated Veteran Player 35\% Max Criteria; or
  \item
    for any player who has completed ten (10) or more Years of Service,
    the greater of (x) thirty-five percent (35\%) of the Salary Cap in
    effect on July 1 of the Salary Cap Year encompassing the first
    Season of the extended term of such Extension, or (y) one hundred
    five percent (105\%) of the Salary provided for in the final Season
    of the original term of the Contract; then such Salary plus Unlikely
    Bonuses shall immediately be deemed amended to provide for the
    maximum amount allowed by the applicable subsection (c)(i), (c)(ii),
    or (c)(iii) set forth above. In such circumstance, (i) if the
    Extension provides for Incentive Compensation, the amount of Likely
    Bonuses and Unlikely Bonuses in the first Salary Cap Year covered by
    the extended term shall be reduced first (on a pro-rata basis), and
    then, if necessary, the amount of Base Compensation provided for in
    the first Salary Cap Year shall be reduced, and (ii) Salaries plus
    Unlikely Bonuses in subsequent Seasons of the extended term shall
    also immediately be deemed amended to the extent necessary to comply
    with the maximum allowable increases or decreases over the amended
    Salary plus Unlikely Bonuses in the first Season of the extended
    term in accordance with Article VII, Section 5(c).
  \end{enumerate}
\item
  A player and a Team may provide in a Rookie Scale Extension that the
  player's Salary (in the first Season of the extended term) will equal
  ``the Maximum Annual Salary applicable to such player in the first
  Season of the extended term'' or:

  \begin{enumerate}
  \def\labelenumii{(\roman{enumii})}
  \item
    in the case of a Rookie Scale Extension for a First Round Pick who
    at the time the Extension is executed has already met at least one
    of the 5th Year 30\% Max Criteria, the player and Team may instead
    provide in the Extension that the player's Salary (in the first
    Season of the extended term) will equal ``{[}\_\_\_\_\_{]}\% of the
    Salary Cap in effect during the first Season of the extended term.''
    The percentage to be included where brackets are indicated in the
    foregoing language shall equal the percentage that is agreed upon by
    the Team and player, which shall in no event be less than
    twenty-five percent (25\%) or greater than thirty percent (30\%); or
  \item
    in the case of a Rookie Scale Extension for any other First Round
    Pick (i.e., a First Round Pick who at the time the Extension is
    executed had not yet met at least one of the 5th Year 30\% Max
    Criteria), the player and Team may instead provide in the Extension
    that the player's Salary (in the first Season of the extended term)
    will equal ``25\% of the Salary Cap in effect during the first
    Season of the extended term, or, if the player meets at least one of
    the applicable 5th Year 30\% Max Criteria during the fourth Season
    of his Rookie Scale Contract, {[}\_\_\_\_\_{]}\% of the Salary Cap
    in effect during the first Season of the extended term.'' The
    percentage to be included where brackets are indicated in the
    foregoing language shall equal the percentage of the Salary Cap that
    is agreed upon by the Team and player, which shall in no event be
    less than twenty-five percent (25\%) or greater than thirty percent
    (30\%).
  \item
    As an alternative to (i) or (ii) above, the Team may instead provide
    in the Extension that the player's Salary (in the first Season of
    the extended term) will equal alternative percentages of the Salary
    Cap (which shall in no event be less than twenty-five percent (25\%)
    or greater than thirty percent (30\%)) based upon how and whether
    the player satisfies the applicable 5th Year 30\% Max Criteria.
    Accordingly, for example, with respect to a Rookie Scale Extension
    entered into on or after the effective date of this Agreement in
    which the first Season of the extended term commences with the
    2019-20 Season, the Team and player could agree that the player's
    Salary (in the first Season of the extended term) would be 25\% of
    the Salary Cap in effect during the first Season of the extended
    term, or the applicable percentage of the Salary Cap set forth below
    if, during the fourth Season of his Rookie Scale Contract, the
    player meets the 5th Year 30\% Max Criteria set forth opposite such
    percentage:

    \begin{longtable}[]{@{}cc@{}}
    \toprule
    5th Year 30\% Max Criteria & Percentage\tabularnewline
    \midrule
    \endhead
    All-NBA Second Team & 27\%\tabularnewline
    All-NBA First Team & 28\%\tabularnewline
    NBA MVP & 30\%\tabularnewline
    \bottomrule
    \end{longtable}
  \end{enumerate}

  The player and Team may provide in a Rookie Scale Extension that the
  Salaries in any Seasons after the first Season of the extended term
  will be increased or decreased based on percentages specified by the
  parties that comply with Article VII, Section 5(c). In the case of a
  Rookie Scale Extension entered into pursuant to (ii) or (iii) above,
  the player and Team may instead provide that Salaries in any Seasons
  after the first Season of the extended term will be increased or
  decreased by a different percentage based on the percentage of the
  Salary Cap that the player receives in Salary in the first Season of
  the extended term. Notwithstanding anything to the contrary in the
  foregoing, a Designated Rookie Scale Player Extension must provide for
  annual increases in Salary for each Season following the first Season
  of the extended term equal to eight percent (8\%) of Salary for the
  Salary Cap Year covered by the first Season of the extended term. Any
  such Rookie Scale Extension (including any such Designated Rookie
  Scale Player Extension) shall be deemed amended on July 1 of the
  Salary Cap Year covering the first Season of the extended term to
  provide for specific Salaries for each Season of the extended term,
  based on the Maximum Annual Salary applicable to such player on such
  July 1. A Rookie Scale Extension entered into pursuant to this
  subsection may not include any Incentive Compensation.
\item
  A player and a Team may provide in a Designated Veteran Player
  Extension that the player's Salary (in the first Season of the
  extended term) will equal ``{[}\_\_\_\_\_{]}\% of the Salary Cap in
  effect during the first Season of the extended term.'' The percentage
  to be included where brackets are indicated in the foregoing language
  shall equal the percentage that is agreed upon by the Team and player,
  which percentage shall in no event be less than thirty percent (30\%)
  or greater than thirty-five percent (35\%). The player and Team may
  provide in a Designated Veteran Player Extension that the Salaries in
  any Seasons after the first Season of the extended term will be
  increased or decreased based on percentages specified by the parties
  that comply with Article VII, Section 5(c). Any such Designated
  Veteran Player Extension shall be deemed amended on July 1 of the
  Salary Cap Year covering the first Season of the extended term to
  provide for specific Salaries for each Season of the extended term,
  based on the Maximum Annual Salary applicable to such player on such
  July 1. A Designated Veteran Player Extension entered into pursuant to
  this subsection may not include any Incentive Compensation.
\item
  Notwithstanding any other provision of this Agreement, if a trade of a
  Uniform Player Contract would, by reason of a trade bonus contained in
  such Contract, cause the player's Salary plus Unlikely Bonuses for the
  Salary Cap Year in which such trade occurs to exceed the following
  amounts:

  \begin{enumerate}
  \def\labelenumii{(\roman{enumii})}
  \tightlist
  \item
    for any player who has completed fewer than seven (7) Years of
    Service, the greater of (x) twenty-five percent (25\%) of the Salary
    Cap in effect at the time the trade bonus is earned, or (y) one
    hundred five percent (105\%) of the player's Salary for the Season
    prior to the Season in which the trade bonus is earned, or in the
    case of a 5th Year Eligible Player who met at least one of the 5th
    Year 30\% Max Criteria and signed a Contract or Rookie Scale
    Extension (as applicable) that provided for up to thirty percent
    (30\%) of the Salary Cap, {[}\_\_{]}\% of the Salary Cap in effect
    at the time the trade bonus is earned with the applicable percentage
    where brackets are indicated equal to the percentage of the Salary
    Cap paid to the player in the first year of his Contract or the
    first year of the extended term in the case of a Rookie Scale
    Extension;
  \item
    for any player who has completed at least seven (7) but fewer than
    ten (10) Years of Service, the greater of (x) thirty percent (30\%)
    of the Salary Cap in effect at the time the trade bonus is earned,
    or (y) one hundred five percent (105\%) of the player's Salary for
    the Season prior to the Season in which the trade bonus is earned,
    or in the case of a Designated Veteran Player who signed a
    Designated Veteran Player Contract or a Designated Veteran Player
    Extension (as applicable) that provided for up to thirty-five
    percent (35\%) of the Salary Cap, {[}\texttt{\_\_}{]}\% of the
    Salary Cap in effect at the time the trade bonus is earned with the
    applicable percentage where brackets are indicated equal to the
    percentage of the Salary Cap paid to the player in the first year of
    his Contract (or the first year of the extended term in the case of
    a Designated Veteran Player Extension); or
  \item
    for any player who has completed ten (10) or more Years of Service,
    the greater of (x) thirty-five percent (35\%) of the Salary Cap in
    effect at the time the trade bonus is earned, or (y) one hundred
    five percent (105\%) of the player's Salary for the Season prior to
    the Season in which the trade bonus is earned; then such player's
    trade bonus shall be deemed amended to the extent necessary to
    reduce the player's Salary plus Unlikely Bonuses to the maximum
    amount allowed by the applicable subsection (f)(i), (f)(ii), or
    (f)(iii) set forth above.
  \end{enumerate}
\item
  Notwithstanding any other provision of this Agreement, any Contract or
  Rookie Scale Extension entered into between a 5th Year Eligible Player
  and a Team that provides for Salary plus Unlikely Bonuses in the first
  Season covered by the Contract or Rookie Scale Extension (as
  applicable) greater than twenty-five percent (25\%) of the Salary Cap
  in effect during the first Season of the Contract or extended term (as
  applicable) in accordance with the rules set forth in this Section 7
  must be for at least four (4) Seasons (excluding any Option Year) and,
  in the case of a Rookie Scale Extension, excluding the last Season
  covered by the player's Rookie Scale Contract.
\end{enumerate}

\section{Promotional Activities.}\label{promotional-activities.}

\begin{enumerate}
\def\labelenumi{(\alph{enumi})}
\tightlist
\item
  A player's obligation (pursuant to paragraph 13(d) of a Uniform Player
  Contract) to participate, upon request, in all other reasonable
  promotional activities of the Team and the NBA shall be deemed
  satisfied if:

  \begin{enumerate}
  \def\labelenumii{(\roman{enumii})}
  \tightlist
  \item
    during each Salary Cap Year of the period covered by such Contract,
    the Player makes seven (7) individual personal appearances (at least
    two (2) of which shall be in connection with season ticketholder
    events) and five (5) group appearances for or on behalf of or at the
    request of the Team (or Team Affiliate) by which he is employed
    and/or the NBA. Up to two (2) of these twelve (12) appearances may
    be assigned by the Team and/or the NBA in any year to NBA
    Properties. The Player shall be reimbursed for the actual expenses
    incurred in connection with any such appearance, provided that such
    expenses result directly from the appearance and are ordinary and
    reasonable. The Player shall also receive compensation from the Team
    by which he is employed of \$3,500, in accordance with paragraph
    13(d) of the Uniform Player Contract, for each promotional
    appearance he makes for a commercial sponsor of such Team.
    Notwithstanding the preceding sentence, with respect to any Salary
    Cap Year during which a player makes at least eight (8) appearances
    pursuant to this Section 8(a)(i), for each subsequent appearance
    made by the player for a commercial sponsor of the Team during such
    Salary Cap Year, the player shall receive compensation from the Team
    by which he is employed of \$4,500.
  \item
    Any personal or group appearance required under this subsection (a)
    must:

    \begin{enumerate}
    \def\labelenumiii{(\Alph{enumiii})}
    \tightlist
    \item
      take place during (1) the period from the first day of a Season
      through the day of the NBA Draft following such Season, or (2) the
      off-season, provided that no player may be required to make more
      than one off-season appearance in any year covered by his Contract
      and no player may be required to make such an off-season
      appearance unless he resides in or is otherwise located in the
      area where the appearance is to take place;
    \item
      occur in the home city (or geographic vicinity thereof) of the
      player's Team (subject to Section 8(a)(ii)(A)(2) above) or in a
      city (or geographic vicinity thereof) to which the player has
      traveled to play in a scheduled NBA game;
    \item
      not occur at a time that would interfere with a player's
      reasonable preparation to play on the day of a Team game;
    \item
      not occur at a time that would interfere with a player's ability
      to attend and participate fully in any practice session conducted
      by the Team, taking into account the commuting time from the
      practice to the appearance;
    \item
      be scheduled with the player at least fourteen (14) days in
      advance (by providing written notice to the player of the time,
      nature, location, and expected duration of the appearance) and
      called to his attention again seven (7) days prior to the
      appearance;
    \item
      not exceed a reasonable period of time; and
    \item
      not require the player to sign autographs as the primary purpose
      of the appearance.
    \end{enumerate}
  \item
    The player participates in reasonable fan appreciation activities
    before and after home games, including but not limited to signing
    autographs for fans, greeting fans, and participating in merchandise
    giveaways to fans; provided, however, that no player shall be
    required to participate in more than four (4) such activities per
    Season.
  \item
    Teams shall be required to track promotional appearances made by
    players in accordance with this Section 8 and Paragraph 13(d) of the
    Uniform Player Contract and report such information to the NBA. The
    NBA shall provide the Players Association with a list of such player
    appearances made during the preceding reporting period as of the
    date two (2) weeks prior to the date of each report. The reports
    shall be provided on or before the following dates each Season:
    December 31; February 28; April 30; and July 31.
  \end{enumerate}
\item
  Upon request by the Team, the NBA, or a League-related entity, and
  subject to the conditions and limitations set forth below, the Player
  shall wear a wireless microphone during any game or practice,
  including warm-up periods and going to and from the locker room to the
  playing floor. The rights in any audio captured by such microphone
  shall belong to the NBA or a League-related entity and may be used in
  any manner for publicity or promotional purposes.

  \begin{enumerate}
  \def\labelenumii{(\roman{enumii})}
  \tightlist
  \item
    The NBA or a League-related entity will be responsible for providing
    the audio equipment and for the placement of the microphone on the
    player in a location and manner that minimizes interference with the
    player's performance.
  \item
    The audio captured by the wireless microphone worn by the player
    (``Player Audio'') will be screened and approved prior to airing by
    the telecast producer and an NBA representative, and no such audio
    will be aired live without the prior consent of the player.
  \item
    The NBA will use best efforts to ensure that a game telecast will
    not include any Player Audio that contains profanity or that could
    reasonably be considered prejudicial or detrimental to the player or
    other players.
  \item
    All audio tapes containing approved Player Audio will be returned by
    the telecaster to the NBA and archived.
  \item
    At the request of the player or the Players Association, the NBA
    shall make available a copy of the Player Audio.
  \item
    In the event a player believes that any Player Audio excerpt would
    be prejudicial or detrimental to him if replayed in any non-game
    programming (e.g., home videos) or other publicity or promotional
    content, and notifies the NBA to that effect in writing within one
    hundred twenty (120) hours of the recording of such audio, then
    neither the NBA nor any League-related entity, following receipt of
    such notice from the player, shall incorporate, or license others to
    incorporate, such excerpt into any such content.
  \item
    No player, without his consent, may be required to wear a wireless
    microphone (A) for nationally-televised games, more than one (1)
    game per month in any Regular Season covered by his Contract, (B)
    for locally-televised games, more than one (1) game per month in any
    Season covered by his Contract, or (C) for playoff games, more than
    two (2) games per playoff round in any Season covered by his
    Contract.
  \item
    At the beginning of each Season, players will receive written notice
    of the conditions and limitations set forth in Section 8(b)(i)-(vii)
    above.
  \item
    Notwithstanding anything to the contrary in this Agreement, Player
    Audio shall not be used as the basis for the imposition of
    discipline upon any player.
  \end{enumerate}
\item
  Each player shall be required to participate each Season, upon
  request, in promotional activities for the benefit of the NBA's
  television partners, provided that such participation does not exceed
  one (1) hour per player per Season and that the player is reimbursed
  for any reasonable expenses he incurs in connection with such
  participation.
\end{enumerate}

\section{10-Day Contracts.}\label{day-contracts.}

\begin{enumerate}
\def\labelenumi{(\alph{enumi})}
\tightlist
\item
  Beginning on January 5 of any NBA Season, a Team may enter into a
  Player Contract (other than a Two-Way Contract) with a player for the
  longer of (i) ten (10) days, or (ii) a period encompassing three (3)
  games played by such Team (a ``10-Day Contract'').
\item
  The Salary provided for by a 10-Day Contract shall not be less than
  the Minimum Player Salary.
\item
  No Team may enter into a 10-Day Contract with the same player more
  than twice during the course of any one Season. No Team may be a party
  at any one time to more 10-Day Contracts than the following:
\end{enumerate}

\begin{longtable}[]{@{}cc@{}}
\toprule
\begin{minipage}[b]{0.54\columnwidth}\centering\strut
Aggregate Number of Players on Team's Active List and Inactive List
(including players signed to 10-Day Contracts, but not including Two-Way
Players)\strut
\end{minipage} & \begin{minipage}[b]{0.30\columnwidth}\centering\strut
Maximum Number of the Team's players who can be signed to 10-Day
Contracts\strut
\end{minipage}\tabularnewline
\midrule
\endhead
\begin{minipage}[t]{0.54\columnwidth}\centering\strut
12\strut
\end{minipage} & \begin{minipage}[t]{0.30\columnwidth}\centering\strut
0\strut
\end{minipage}\tabularnewline
\begin{minipage}[t]{0.54\columnwidth}\centering\strut
13\strut
\end{minipage} & \begin{minipage}[t]{0.30\columnwidth}\centering\strut
1\strut
\end{minipage}\tabularnewline
\begin{minipage}[t]{0.54\columnwidth}\centering\strut
14\strut
\end{minipage} & \begin{minipage}[t]{0.30\columnwidth}\centering\strut
2\strut
\end{minipage}\tabularnewline
\begin{minipage}[t]{0.54\columnwidth}\centering\strut
15\strut
\end{minipage} & \begin{minipage}[t]{0.30\columnwidth}\centering\strut
3\strut
\end{minipage}\tabularnewline
\bottomrule
\end{longtable}

For example, if a Team has thirteen (13) players on its Active List (not
including any Two-Way Players) and no players on its Inactive List, then
the Team may have one player under a 10-Day Contract. If a Team has
thirteen (13) players on its Active List (including one (1) Two-Way
Player) and two (2) players on its Inactive List (not including any
Two-Way Players), then the Team may have two (2) players under a 10-Day
Contract. If a Team has twelve (12) players on its Active List (not
including any Two-Way Players) and three (3) players on its Inactive
List (not including any Two-Way Players), then the Team may have three
(3) players under a 10-Day Contract.

\begin{enumerate}
\def\labelenumi{(\alph{enumi})}
\setcounter{enumi}{3}
\tightlist
\item
  No Team may enter into a 10-Day Contract if the length of such
  Contract, in accordance with Section 9(a), would extend to or past the
  date of the Team's last Regular Season game for such Season.
\item
  Notwithstanding anything to the contrary contained in a Uniform Player
  Contract, a 10-Day Contract shall be terminated simply by providing
  written notice to the player (and not by following the waiver
  procedure set forth in paragraph 16 of the Uniform Player Contract)
  and paying only such sums as are set forth in Exhibit 1A (or, if
  applicable, Exhibit 1) of such Contract.
\end{enumerate}

\section{Rest-of-Season Contracts.}\label{rest-of-season-contracts.}

\begin{enumerate}
\def\labelenumi{(\alph{enumi})}
\tightlist
\item
  At any time after the first day of an NBA Regular Season, a Team may
  enter into a Player Contract that may provide Compensation to a player
  only for the remainder of that Season (a ``Rest-of-Season Contract'').
\item
  The Salary provided for in a Rest-of-Season Contract shall not be less
  than the Minimum Player Salary.
\item
  Notwithstanding the foregoing, Two-Way Contracts shall not be subject
  to the requirements set forth in this Section 10.
\end{enumerate}

\section{Two-Way Contracts.}\label{two-way-contracts.}

\begin{enumerate}
\def\labelenumi{(\alph{enumi})}
\tightlist
\item
  \textbf{Two-Way Player Salary.}

  \begin{enumerate}
  \def\labelenumii{(\roman{enumii})}
  \tightlist
  \item
    Subject to the limitations set forth in this Section 11, an NBA Team
    may enter into a Player Contract that provides a player (``Two-Way
    Player'') with a tiered Salary as set forth in Section 11(a)(ii)
    below based on whether the player is performing services on a
    particular day for (i) an NBADL team, or (ii) the NBA Team
    (``Two-Way Contract'').
  \item
    The Salary provided for in a Two-Way Contract shall be the sum of
    the Two-Way NBA Salary (as defined in Section 11(a)(ii)(A) below)
    and the Two-Way NBADL Salary (as defined in Section 11(a)(ii)(B)
    below) (together, the ``Two-Way Player Salary''), and subject to the
    following rules:

    \begin{enumerate}
    \def\labelenumiii{(\Alph{enumiii})}
    \tightlist
    \item
      A player's ``Two-Way NBA Salary'' shall equal the Minimum Annual
      Salary called for under Article II, Section 6(a) for a player with
      zero (0) Years of Service (irrespective of how many Years of
      Service the player has accrued prior to the Contract or accrues
      during the term of the Contract), multiplied by a fraction, the
      numerator of which is the number of NBA Days of Service (as
      defined in Section 11(b)(i)(A) below) that the player accrues
      during the NBA Regular Season, and the denominator of which is the
      total number of days of that NBA Regular Season.
    \item
      A player's ``Two-Way NBADL Salary'' shall equal the player's
      Two-Way Annual NBADL Salary called for below, multiplied by a
      fraction, the numerator of which is the number of NBADL Days of
      Service (as defined in Section 11(b)(i)(B) below) that the player
      accrues during the NBADL Regular Season, and the denominator of
      which is the total number of days of that NBADL Regular Season.
    \end{enumerate}

    \begin{longtable}[]{@{}cc@{}}
    \toprule
    Salary Cap Year & Two-Way Annual NBADL Salary\tabularnewline
    \midrule
    \endhead
    2017-18 & \$75,000\tabularnewline
    2018-19 & \$77,250\tabularnewline
    2019-20 & \$79,568\tabularnewline
    2020-21 & \$81,955\tabularnewline
    2021-22 & \$84,414\tabularnewline
    2022-23 & \$86,946\tabularnewline
    2023-24 & \$89,554\tabularnewline
    2024-25 & \$92,241\tabularnewline
    \bottomrule
    \end{longtable}

    \begin{enumerate}
    \def\labelenumiii{(\Alph{enumiii})}
    \setcounter{enumiii}{2}
    \tightlist
    \item
      Notwithstanding anything to the contrary in this Agreement, no
      Two-Way Contract may include or provide for any (i) bonuses or
      Incentive Compensation of any kind, (ii) deferred compensation,
      (iii) loans, or (iv) advances (other than pursuant to Paragraph
      3(b) of the Contract).
    \end{enumerate}
  \item
    Every Two-Way Contract must contain an Exhibit 1B and include the
    following sentence in such Exhibit (which shall be deemed amended in
    the manner described in such sentence): ``This Contract is intended
    to provide for a Base Compensation for the \_\_\_\_\_\_\_\_\_\_\_\_
    Season(s) equal to the Two-Way Player Salary for such Season(s)
    (with no bonuses of any kind) and shall be deemed amended to the
    extent necessary to so provide.''
  \end{enumerate}
\item
  \textbf{Days of Service.}

  \begin{enumerate}
  \def\labelenumii{(\roman{enumii})}
  \tightlist
  \item
    Days of service for Two-Way Players will accrue as follows:

    \begin{enumerate}
    \def\labelenumiii{(\Alph{enumiii})}
    \tightlist
    \item
      A Two-Way Player will accrue one day of service for an NBA Team
      (an ``NBA Day of Service'') for each calendar day during the NBA
      Regular Season during which the player (i) is on the NBA Team's
      Active List for an NBA game, (ii) participates in any practice,
      basketball drill, conditioning, workout, or other such activity
      with one or more players on the NBA Team under the direction and
      supervision of the NBA Team, or (iii) travels with or at the
      direction of (including remaining on the road with) the NBA Team,
      other than if the only travel during that day is return travel to
      the NBA Team's home city that takes place between 12:00 midnight
      and 1:00 a.m; provided, however, that a Two-Way Player will not
      accrue an NBA Day of Service for traveling between his NBADL team
      and NBA Team. For the avoidance of doubt, a Two-Way Player will
      not accrue an NBA Day of Service during NBA training camp or
      during the NBA playoffs.
    \item
      A Two-Way Player will accrue one day of service for an NBADL team
      (an ``NBADL Day of Service'') for each calendar day during the
      NBADL Regular Season during which the Two-Way Player is not
      providing an NBA Day of Service. For the avoidance of doubt, a
      Two-Way Player will not accrue an NBADL Day of Service during
      NBADL training camp or during the NBADL playoffs.
    \end{enumerate}
  \item
    No player under a Two-Way Contract may accrue more than forty-five
    (45) NBA Days of Service during an NBA Regular Season (the ``45-Day
    Two-Way Service Limit''). If a player is signed to a Two-Way
    Contract after the start of the NBA Regular Season, the 45-Day
    Two-Way Service Limit shall be prorated such that the maximum
    allowable number of NBA Days of Service that the player may accrue
    shall be: forty-five (45) multiplied by a fraction, the numerator of
    which is the number of days remaining in the NBA Regular Season as
    of the date such Two-Way Contract is entered into, and the
    denominator of which is the total number of days of that NBA Regular
    Season, rounded up to the nearest whole number. Notwithstanding the
    foregoing, if a player provides one or more NBA Days of Service
    before the first day of any NBADL training camp or after the final
    game of the player's team's NBADL Regular Season, such day(s) will
    not count toward the 45-Day Two-Way Service Limit.
  \end{enumerate}
\item
  \textbf{Compensation Protection.}

  \begin{enumerate}
  \def\labelenumii{(\roman{enumii})}
  \tightlist
  \item
    The maximum amount of aggregate Base Compensation protection for a
    Season in a Two-Way Contract is \$50,000, provided that, if such
    Contract is signed after the first day of the NBA Regular Season,
    the maximum amount of aggregate Base Compensation protection for
    such Season shall be: \$50,000 multiplied by a fraction, the
    numerator of which is the number of days remaining in the NBA
    Regular Season as of the date such Contract is entered into, and the
    denominator of which is the total number of days of that NBA Regular
    Season.
  \item
    If a player's Player Contract with a Team contains aggregate Base
    Compensation protection that exceeds \$50,000 and the Team assigns
    or terminates the player's Contract, then, during the Salary Cap
    Year in which such assignment or termination occurs, the player
    shall be precluded from: (x) playing under an NBADL contract for
    such Team's NBADL affiliate, and (y) entering into a Two-Way
    Contract with such Team.
  \end{enumerate}
\item
  \textbf{Contract Term.} The term of a Two-Way Contract may not exceed
  two (2) Seasons in length and may not include any Option Year or Early
  Termination Option.
\item
  \textbf{Roster Limitations.} No Team may have on its roster at any one
  time more than two (2) Two-Way Players.
\item
  \textbf{Eligibility.} The following eligibility rules shall apply to
  all Two-Way Contracts:

  \begin{enumerate}
  \def\labelenumii{(\roman{enumii})}
  \tightlist
  \item
    No Team may sign a player to a Two-Way Contract after January 15 of
    any Season.
  \item
    No Team may sign or convert a player to a Two-Way Contract if the
    player has or may have four (4) or more Years of Service at any
    point during the Contract. For example, a player with three (3)
    Years of Service would not be eligible to sign a Two-Way Contract
    with a term of two (2) years.
  \item
    No Team may sign or convert a player to a Two-Way Contract, or
    acquire a Two-Way Contract by means of assignment, if, as a result,
    the player would or could be under a Two-Way Contract for any part
    of more than three (3) Salary Cap Years with the same NBA Team. For
    example, a player who completes a two-year Two-Way Contract with a
    Team could not subsequently sign a two-year Two-Way Contract with
    that Team.
  \end{enumerate}
\item
  \textbf{Standard NBA Contract Conversion Option.} Every Two-Way
  Contract shall provide the Team with an option to convert the Two-Way
  Contract during its term to a Contract that is not a Two-Way Contract
  (``Standard NBA Contract'') that provides for a Salary for each Salary
  Cap Year equal to the player's applicable Minimum Player Salary and a
  term equal to the remainder of the original term of the Two-Way
  Contract beginning on the date such option is exercised (``Standard
  NBA Contract Conversion Option''). Such player's applicable Minimum
  Player Salary shall be determined in accordance with Section 6 above.
  The Standard NBA Contract Conversion Option may be exercised at any
  point during the period beginning on July 1 and ending just prior to
  the start of the Team's last Regular Season game in each Salary Cap
  Year covered by the Two-Way Contract. Upon conversion, such Contract
  shall become a Standard NBA Contract and shall no longer be governed
  by the provisions of this Agreement governing Two-Way Contracts. To
  effectuate the requirements set forth in the preceding sentences,
  every Two-Way Contract with an Exhibit 1B must contain the following
  language (and only such language) under the ``Standard NBA Contract
  Conversion Option'' heading: ``Team shall have the option to convert
  this Contract to a Standard NBA Contract (''Standard NBA Contract
  Conversion Option``). Team's Standard NBA Contract Conversion Option
  may be exercised by providing written notice to Player that is either
  personally delivered to Player or his representative or sent by email
  or pre-paid certified, registered, or overnight mail to the last known
  address of Player or his representative with a copy to the Players
  Association and the NBA. If Team exercises the Standard NBA Contract
  Conversion Option, the Base Compensation amount set forth above in
  this Exhibit 1B will immediately become null and void and of no
  further force or effect, Player's Compensation shall be equal to the
  Player's applicable Minimum Player Salary for a term equal to the
  remainder of the original term of this Contract beginning on the date
  such option is exercised, and all other terms and conditions of this
  Contract, including the Base Compensation protection set forth in
  Exhibit 2 (if any), shall remain applicable.''
\item
  \textbf{Exclusive Rights.}

  \begin{enumerate}
  \def\labelenumii{(\roman{enumii})}
  \tightlist
  \item
    During the term of a Two-Way Contract, the Team that is the party to
    the Two-Way Contract shall be the only Team with which the Two-Way
    Player may negotiate or sign a Standard NBA Contract.
  \item
    The Team and the Two-Way Player who are parties to such Two-Way
    Contract shall have the right to negotiate and agree to a Standard
    NBA Contract in accordance with the terms of this Agreement.
    Notwithstanding anything to the contrary in this Agreement or the
    Uniform Player Contract, upon execution of the Standard NBA
    Contract, the prior Two-Way Contract between the Team and player
    will immediately be rendered null and void and of no further force
    or effect.
  \end{enumerate}
\item
  \textbf{Exhibit 10.}

  \begin{enumerate}
  \def\labelenumii{(\roman{enumii})}
  \tightlist
  \item
    Every Contract with an Exhibit 10 shall provide the Team with an
    option (to be set forth in Exhibit 10) to convert the Contract to a
    Two-Way Contract that provides for the Two-Way Player Salary
    (``Two-Way Player Conversion Option''); provided, however, that the
    Two-Way Player Conversion Option (a) must be exercised prior to the
    first day of the NBA Regular Season, and (b) may not be exercised if
    it would result in a violation of Article X, Section 4(d). If a Team
    exercises the Two-Way Player Conversion Option, (w) the Contract's
    Exhibit 1A will immediately become null and void and of no further
    force or effect and the Player's Compensation shall be equal to the
    Two-Way Player Salary applicable for such Season, (x) the Player's
    right to an Exhibit 10 Bonus (if applicable) will be rescinded, (y)
    the Player's Contract, notwithstanding the absence of an Exhibit 2,
    shall have Base Compensation protection for lack of skill and injury
    or illness at an amount equal to the Conversion Protection Amount,
    and (z) all other terms and conditions of the Contract shall remain
    applicable.
  \item
    If a Team exercises a Two-Way Player Conversion Option pursuant to a
    Contract with an Exhibit 10, such Contract shall be considered a
    Two-Way Contract for the purposes of this Agreement and subject to
    all applicable Two-Way Contract rules herein (including, but not
    limited to, the Standard NBA Contract Conversion Option) except that
    such Contract need not contain an Exhibit 1B.
  \item
    To effectuate the requirements set forth above, every Contract with
    an Exhibit 10 must contain the following language (and only such
    language) under the ``Two-Way Player Conversion Option'' and
    ``Standard NBA Contract Conversion Option'' headings,
    respectively:\\
    \textbf{Two-Way Player Conversion Option:} Team shall have the
    option to convert this Contract to a Two-Way Contract (``Two-Way
    Player Conversion Option''); provided, however, that (a) such option
    must be exercised prior to the first day of the NBA Regular Season,
    and (b) may not be exercised if it would result in a violation of
    Article X, Section 4(d) of the CBA. Team's Two-Way Player Conversion
    Option may be exercised by providing written notice to Player that
    is either personally delivered to Player or his representative or
    sent by email or pre-paid certified, registered, or overnight mail
    to the last known address of Player or his representative with a
    copy to the Players Association and the NBA. If Team exercises the
    Two-Way Player Conversion Option, this Contract's Exhibit 1A will
    immediately become null and void and of no further force or effect
    and the Player's Compensation shall be equal to the Two-Way Player
    Salary applicable for such Season. Further, upon conversion, the
    Player's right to the Bonus Amount (if applicable) set forth above
    pursuant to this Exhibit 10 will be rescinded and the Player's
    Contract, notwithstanding the absence of an Exhibit 2, shall be
    protected for lack of skill and injury or illness at an amount equal
    to the Conversion Protection Amount in this Exhibit 10. All other
    terms and conditions of this Contract shall remain applicable.\\
    \textbf{Standard NBA Contract Conversion Option:} In the event the
    Two-Way Player Conversion Option is exercised by the Team, Team
    shall thereafter have the option to convert the Contract to a
    Standard NBA Contract (``Standard NBA Contract Conversion Option'').
    Team's Standard NBA Contract Conversion Option may be exercised by
    providing written notice to Player that is either personally
    delivered to Player or his representative or sent by email or
    pre-paid certified, registered, or overnight mail to the last known
    address of Player or his representative with a copy to the Players
    Association and the NBA. If Team exercises the Standard NBA Contract
    Conversion Option, the Base Compensation amount applicable to the
    Two-Way Contract as set forth in this Exhibit 10 will immediately
    become null and void and of no further force or effect, Player's
    Compensation shall be equal to the Player's applicable Minimum
    Player Salary for such Season beginning on the date such option is
    exercised, and all other terms and conditions of this Contract,
    including the Base Compensation protection set forth in this Exhibit
    10, shall remain applicable.
  \end{enumerate}
\end{enumerate}

\section{Bonuses.}\label{bonuses.}

\begin{enumerate}
\def\labelenumi{(\alph{enumi})}
\item
  \begin{enumerate}
  \def\labelenumii{(\roman{enumii})}
  \tightlist
  \item
    Notwithstanding any other provision of this Agreement, (A) no
    Uniform Player Contract may provide for a signing bonus that exceeds
    fifteen percent (15\%) of the Compensation (excluding Incentive
    Compensation) called for by the Contract (or, in the case of an
    Extension, in the extended term of the Extension), and (B) no Offer
    Sheet may provide for a signing bonus that exceeds ten percent
    (10\%) of the Compensation (excluding Incentive Compensation) called
    for by the Offer Sheet.
  \item
    If a player's Contract provides for a signing bonus and the player
    is suspended for the intentional failure or refusal to render the
    services required under his Contract, the Team shall be entitled to
    a return from the player of an amount equal to the product of the
    signing bonus multiplied by a fraction, the numerator of which is
    the number of Regular Season games that the player is suspended as a
    result of his failure or refusal to render such services and the
    denominator of which is the total number of Regular Season games to
    be played by the Team during the term of the Contract (excluding any
    Option Year). The foregoing shall not limit any other rights or
    remedies a Team may have under the Contract or by law.
  \end{enumerate}
\item
  \begin{enumerate}
  \def\labelenumii{(\roman{enumii})}
  \tightlist
  \item
    No Uniform Player Contract may provide for the player's attendance
    at and participation in an off-season skill and/or conditioning
    program that exceeds two (2) weeks in length.
  \item
    A Uniform Player Contract that contains a bonus to be paid as a
    result of the player's attendance at and participation in an
    off-season summer league and/or an off-season skill and/or
    conditioning program in accordance with subsection b(i) above may
    also contain a provision providing that such bonus will be paid if:
    (A) the Team elects in writing to waive the requirement that the
    player perform the specified services; (B) the player, in lieu of
    providing the specified services, participates in training and/or
    plays games with his national team during the off-season; and/or (C)
    the player has an injury, illness, or other medical condition that
    renders the player unable to participate in such summer league
    and/or skill and conditioning program. If a Contract contains a
    provision of the type described in (A) above and the Team exercises
    its right to waive the requirement that the player perform the
    specified services with respect to one or more off-seasons, the
    amounts paid to the player shall continue to be treated as a bonus
    for the player's participation in an off-season summer league or
    off-season skill and/or conditioning program and shall continue to
    be subject to the rules in this Agreement relating to such bonuses.
  \item
    If a Uniform Player Contract contains a bonus to be paid as a result
    of the player's attendance at and participation in an off-season
    summer league and/or an off-season skill and/or conditioning
    program, the Team shall be required to provide the player with a
    reasonable opportunity to earn the bonus by, for example, providing
    the player with the dates, times, and location(s) at which the
    specified services are to be performed. A Team's failure to comply
    with this requirement with respect to any off-season shall be deemed
    to constitute a waiver of the requirement that the player perform
    the specified services for such off-season.
  \item
    No Player Contract may provide for bonuses for any Season to be paid
    as a result of the player's attendance at and participation in an
    off-season summer league and/or an off-season skill and/or
    conditioning program that exceed twenty percent (20\%) of the
    player's Base Compensation for such Season.
  \end{enumerate}
\item
  No Uniform Player Contract may contain a bonus for the player being on
  a Team's roster as of a specified date or for a specified duration, or
  for the player dressing in uniform for or being eligible to play in a
  specified number of games.
\item
  If a Player Contract contains Incentive Compensation, a Team and
  player shall not be permitted at any time to amend the Contract to
  modify the conditions that the player must satisfy in order to earn
  all or any portion of such Incentive Compensation.
\end{enumerate}

\section{General.}\label{general.}

\begin{enumerate}
\def\labelenumi{(\alph{enumi})}
\item
  \begin{enumerate}
  \def\labelenumii{(\roman{enumii})}
  \tightlist
  \item
    Subject to Section 15 below, any oral or written agreement between a
    player and a Team concerning terms and conditions of employment
    shall be reduced to writing in the form of a Uniform Player Contract
    or an amendment thereto as soon as practicable. Immediately upon the
    consummation of any such oral or written agreement, the Team shall
    notify the NBA by e-mail and provide the NBA with all economic terms
    of such agreement. Upon its receipt of an executed Uniform Player
    Contract, the NBA shall provide a copy of the same to the Players
    Association by email within two (2) business days.
  \item
    Notwithstanding subsection (a)(i) above, neither the NBA, any Team,
    nor the Players Association, or any player, shall contend that any
    agreement concerning terms and conditions of employment is binding
    upon the player or the Team until a Player Contract embodying such
    terms and conditions has been duly executed by the parties. Nothing
    herein is intended to affect (A) any authority of the Commissioner
    to approve or disapprove Player Contracts, or (B) the effect of the
    Commissioner's approval or disapproval on the validity of such
    Player Contracts.
  \item
    A violation of the first sentence of subsection (a)(i) above may be
    considered evidence of a violation of Article XIII.
  \end{enumerate}
\item
  No player shall attend the regular training camp of any Team, or
  participate in games or organized practices with the Team at any time,
  unless he is a party to a Player Contract then in effect. For purposes
  of this Section 13(b), a player shall be considered to be a party to a
  Player Contract then in effect if such Contract has been extended in
  accordance with an Option permitted by this Agreement.
\item
  The only form of Compensation that a Team may pay a player under his
  Uniform Player Contract is cash via a check made payable to the player
  or via a direct deposit made to the player's bank account.
  Compensation of any other kind is prohibited.
\item
  No Team shall make any direct or indirect payment of any money,
  property, investments, loans, or anything else of value for fees or
  otherwise to an agent, attorney, or representative of a player (for or
  in connection with such person's representation of such player); nor
  shall any Player Contract provide for such payment. No player shall
  assign or otherwise transfer to any third party his right to receive
  Compensation from the Team under his Uniform Player Contract. Nothing
  in this subsection (d), however, shall prevent a Team from sending a
  player's regular paycheck to a player's agent, attorney, or
  representative if so instructed in writing by the player.
\item
  Every Uniform Player Contract must provide that for each Season of
  such Contract, the player will be paid at least twenty percent (20\%)
  of his Salary for such Season, excluding Likely Bonuses and any
  portion of the player's Salary attributable to a trade bonus, in
  Current Base Compensation in accordance with the payment schedule
  provided in paragraph 3 of the Contract or in twelve (12) equal
  semi-monthly payments beginning with the first of said payments on
  November 15 of each year covered by the Contract and continuing with
  such payments on the first and fifteenth of each month until said
  Compensation is paid in full.
\item
  No Uniform Player Contract may provide for the payment of any
  Compensation earned for a Season prior to the July 1 immediately
  preceding such Season.
\item
  A Team's termination of a Uniform Player Contract by reason of the
  player's ``lack of skill'' (under paragraph 16(a)(iii) of the Uniform
  Player Contract) shall be interpreted to include a termination based
  on the Team's determination that, in view of the player's level of
  skill (in the sole opinion of the Team), the Compensation paid (or to
  be paid) to the player is no longer commensurate with the Team's
  financial plans or needs. The foregoing sentence shall not affect any
  post-termination obligation to pay Compensation that may result from
  Compensation protection provisions included in a Uniform Player
  Contract.
\item
  The following provisions shall govern an agreement (to be set forth in
  Exhibit 6 to a Uniform Player Contract) establishing that the player
  must report for and submit to a physical examination to be performed
  by one or more physician(s) designated by the Team:

  \begin{enumerate}
  \def\labelenumii{(\roman{enumii})}
  \tightlist
  \item
    The player must report for such physical examination at the time
    designated by the Team (which shall be no later than the third
    business day following the execution of the Contract), and must,
    upon reporting, supply all information reasonably requested of him,
    provide complete and truthful answers to all questions posed to him,
    and submit to all examinations and tests requested of him. The
    determination of whether the player has passed the physical
    examination shall be made by the Team in its sole discretion,
    exercised in good faith, in consultation with one or more of the
    Team's physicians; and a Team shall have the right to determine in
    good faith that a player has failed to pass the physical examination
    due to the risk of a future injury, illness or other health
    condition notwithstanding that the player is currently able to play.
    If the player does not pass the physical examination, the Team shall
    so notify the player no later than the sixth business day following
    the execution of the Contract.
  \item
    The Team's determination that the player has passed the physical
    examination shall be a condition precedent to the validity of the
    Contract. Accordingly, and without limiting the generality of the
    preceding sentence, until such time as a player has passed the
    physical examination, the prohibitions set forth in Section 13(b)
    above shall continue to apply to the Team and player.
  \item
    A Required Tender or a Qualifying Offer may contain an Exhibit 6. If
    a player accepts such a Required Tender or Qualifying Offer but does
    not pass the required physical examination, the Required Tender or
    Qualifying Offer shall be deemed to have been withdrawn, which shall
    have the consequences described in Article X, Section 4 or Article
    XI, Section 4, as the case may be.
  \end{enumerate}
\item
  A player who knows he has an injury, illness, or condition that
  renders, or he knows will likely render, him physically unable to
  perform the playing services required under a Player Contract may not
  validly enter into such Contract without prior written disclosure of
  such injury, illness, or condition to the Team.
\end{enumerate}

\section{Void Contracts.}\label{void-contracts.}

If a Player Contract fails to take effect or becomes void as a result of
a Commissioner disapproval, the player's failure to pass a physical
examination conducted pursuant to Exhibit 6 to such Contract, or the
rescission of a trade conducted pursuant to Article VII, Section 8(e),
then, in each such case:

\begin{enumerate}
\def\labelenumi{(\alph{enumi})}
\tightlist
\item
  the Team shall continue to possess such rights with respect to the
  player as the Team possessed at the time of the execution of the
  Contract, including, without limitation, any such rights that the Team
  possessed pursuant to Article VII, Section 6(b), Article X, and
  Article XI;
\item
  any Required Tender or Qualifying Offer that was outstanding at the
  time the Contract was executed shall continue in effect as if the
  Contract had not been executed (including if the original deadline for
  accepting the Required Tender or Qualifying Offer expired following
  the execution of the Contract), but for no fewer than six (6) business
  days following the Commissioner's disapproval, the Team's issuance of
  notice to the player that he did not pass the physical examination, or
  the rescission of such trade, as the case may be; and
\item
  in the case of a player who does not pass a physical examination
  pursuant to Exhibit 6: (i) the player shall not be permitted to accept
  such Required Tender or Qualifying Offer for a period of two (2)
  business days following his receipt of notice from the Team that he
  did not pass his physical examination, during which period the Team
  may elect to withdraw the Required Tender or Qualifying Offer, which
  shall have the consequences described in Article X, Section 4 or
  Article XI, Section 4, as the case may be; and (ii) if the Required
  Tender or Qualifying Offer is not withdrawn by the Team during this
  period, the Required Tender or Qualifying Offer shall thereafter be
  deemed amended so as to eliminate any Exhibit 6 that may be contained
  therein.
\end{enumerate}

\section{Moratorium Period.}\label{moratorium-period.}

Except as permitted in the next sentence, notwithstanding any other
provision of this Agreement, no player and Team may enter into any oral
or written agreement concerning terms and conditions of the player's
employment, or reduce any such agreement to writing in the form of a
Uniform Player Contract or amendment, during the Moratorium Period. The
following shall be permitted during the Moratorium Period: (i) a player
may accept any Required Tender, Qualifying Offer, or ``Maximum
Qualifying Offer'' (as defined in Article XI, Section 4(a)(ii)) that is
outstanding; (ii) a player and a Team may negotiate over the terms and
conditions of a Player Contract that may be entered into following the
conclusion of the Moratorium Period; (iii) a player and a Team may enter
into an Offer Sheet; (iv) a First Round Pick and the Team that holds his
draft rights may enter into a Rookie Scale Contract; and (v) a player
and a Team may enter into a Player Contract, not to exceed two (2)
Seasons in length, that provides for a Salary for each Salary Cap Year
equal to the Two-Way Player Salary or the Minimum Player Salary
applicable to the player (with no bonuses of any kind), and (v) a Team
may exercise a Two-Way Contract's Standard NBA Contract Conversion
Option or, with respect to a Contract with an Exhibit 10, such
Contract's Two-Way Player Conversion Option, in accordance with Article
II, Section 11(g) and (i) above.

\chapter{PLAYER EXPENSES}\label{player-expenses}

\section{Moving Expenses.}\label{moving-expenses.}

\begin{enumerate}
\def\labelenumi{(\alph{enumi})}
\tightlist
\item
  \textbf{Moving Expenses.} A Team's obligation to reimburse a player
  for ``reasonable'' expenses related to the assignment of a Player
  Contract from one Team to another (in accordance with paragraph 10 of
  a Uniform Player Contract) shall extend to the reimbursement of the
  actual expenses incurred by such player in moving to the home
  territory of his new Team, provided that such expenses result directly
  from the assignment and are ordinary and reasonable, and provided
  further that, prior to his actually incurring such expenses, the
  player (i) consults with the Team to which his Contract has been
  assigned in advance concerning his move, and (ii) furnishes the Team
  with a written estimate of such proposed expenses from an established
  moving company so as to afford such assignee-Team an opportunity to
  make reasonably comparable alternative arrangements for the move of
  the player. The player shall furnish such written estimate to the Team
  within a reasonable time following the notice of the assignment of the
  Player Contract. Upon receipt of such estimate from the player, the
  Team shall, within ten (10) days, either agree to reimburse the player
  for the expenses set forth in such estimate or make alternative
  arrangements (at the Team's expense) for the move of the player.
  ``Reasonable'' moving expenses shall include the cost of moving not
  more than one (1) automobile for the player (and not more than two (2)
  automobiles if the player is married).
\item
  \textbf{Hotel Accommodations.} A player whose Contract is assigned
  from one Team to another shall be reimbursed by the assignee-Team for
  the cost of a hotel room in a hotel (comparable to that in which such
  Team's players are lodged while ``on the road'') in the
  assignee-Team's home city for up to forty-six (46) days following the
  assignment.
\item
  \textbf{Housing Costs Reimbursement.} A player whose Contract is
  assigned from one Team to another shall be reimbursed by the
  assignee-Team for the cost of his living quarters (either rent or
  mortgage expense) in the city from which he is assigned, for a period
  of three months after the date of the assignment; provided, however,
  that such payment shall: (i) be made only if and to the extent that
  the player is legally obligated for such costs; and (ii) not exceed
  \$4,500 per month. Any such payments shall be made on a pro rata basis
  if a full month is not involved.
\item
  \textbf{Proof of Expenses.} Prior to reimbursing an assigned player as
  provided in this Section, an assignee-Team may require satisfactory
  proof that the player has paid the amounts for which he seeks
  reimbursement, and, in the case of housing costs reimbursements,
  satisfactory proof that the player is legally obligated to pay such
  housing costs and the amount thereof. Upon notice to the player, the
  assignee-Team may, as an alternative to reimbursement, pay the
  expenses incurred upon assignment (in accordance with the foregoing
  provisions of this Section) directly to the persons, firms, or
  corporations involved.
\item
  \textbf{Player Obligation to Minimize Potential Liability.} So as to
  minimize the potential liability of NBA Teams under this Section, a
  player who does not establish permanent or year-round residence in the
  home city (or geographic vicinity thereof) of the Team by which he is
  employed shall use his best efforts (i) to obtain a short-term lease
  on the living quarters he selects, and (ii) to procure lease
  provisions authorizing him to sublet such premises and/or granting
  such Team the option to take over such lease in the event the Contract
  of such player is assigned to another NBA Team.
\end{enumerate}

\section{Meal Expense Allowance.}\label{meal-expense-allowance.}

\begin{enumerate}
\def\labelenumi{(\alph{enumi})}
\tightlist
\item
  The meal expense allowance, provided for in paragraph 4 of a Uniform
  Player Contract, shall be as follows:

  \begin{enumerate}
  \def\labelenumii{(\roman{enumii})}
  \tightlist
  \item
    For the 2017-18 Season: \$129 per day.
  \item
    For each subsequent Season of this Agreement: the preceding Season's
    meal expense allowance amount adjusted for cost of living by
    applying to the preceding Season's meal expense allowance amount the
    percentage increase (or decrease) in the national Consumer Price
    Index for All Urban Consumers (CPI-U) from the June 1 through the
    May 31 immediately preceding such Season, and which shall be rounded
    off to the nearest whole dollar per day.
  \end{enumerate}
\item
  When a Team is ``on the road'' for less than a full day, a partial
  meal expense shall be paid based upon the time of departure from or
  time of arrival in the Team's home city, in accordance with the
  following:

  \begin{enumerate}
  \def\labelenumii{(\roman{enumii})}
  \tightlist
  \item
    Departure after 9:00 a.m. or arrival before 7:00 a.m., no meal
    expense allowance for breakfast.
  \item
    Departure after 1:00 p.m. or arrival before 11:30 a.m., no meal
    expense allowance for lunch.
  \item
    Departure after 7:00 p.m. or arrival before 5:30 p.m., no meal
    expense allowance for dinner. For purposes of this Section 2(b), the
    meal expense allowance for breakfast shall be deemed to be eighteen
    percent (18\%) of the applicable daily meal expense allowance
    (rounded off to the nearest whole dollar); the meal expense
    allowance for lunch shall be deemed to be twenty-eight percent
    (28\%) of the applicable daily meal expense allowance (rounded off
    to the nearest whole dollar); and the meal expense allowance for
    dinner shall be deemed to be fifty-four percent (54\%) of the
    applicable daily mealexpense allowance (rounded off to the nearest
    whole dollar). For purposes of this Agreement and paragraph 4 of the
    Uniform Player Contract, the ``home city'' of an NBA Team shall be
    deemed to include only the city in which the facility regularly used
    by the Team for home games is located and any other location at
    which such home games are played, provided that such other
    location(s) is not more than seventy-five (75) miles from such city.
  \end{enumerate}
\end{enumerate}

\chapter{BENEFITS}\label{benefits}

\section{Player Pension Benefits.}\label{player-pension-benefits.}

Subject to approval by the Internal Revenue Service (the ``IRS'') and to
the extent permitted by applicable law, the NBA shall provide the
following pension benefits to NBA players and former NBA players in
accordance with and subject to the terms and conditions of the National
Basketball Association Players' Pension Plan, as restated effective
February 2, 2014, and as amended from time to time and as to be modified
as set forth herein (the ``Pension Plan''). (All capitalized terms used
in this Section 1 not otherwise defined in this Agreement shall have the
meanings set forth in the Pension Plan.)

\begin{enumerate}
\def\labelenumi{(\alph{enumi})}
\tightlist
\item
  \textbf{Benefits.}

  \begin{enumerate}
  \def\labelenumii{(\arabic{enumii})}
  \tightlist
  \item
    Current Benefit. As of the effective date of this Agreement, the
    monthly amount per Year of Credited Service payable as a Normal
    Retirement Pension (the ``Monthly Benefit'') is \$572.13.
  \item
    Benefit Increases. Effective for the Plan Year commencing February
    2, 2018, and for each subsequent Plan Year during the term of this
    Agreement:

    \begin{enumerate}
    \def\labelenumiii{(\roman{enumiii})}
    \tightlist
    \item
      The Monthly Benefit shall be adjusted (the Monthly Benefit
      following any such adjustment, the ``New Monthly Benefit'') such
      that, subject to Section 1(a)(4) below, the Normal Retirement
      Pension shall equal the maximum benefit amount permitted under the
      Internal Revenue Code of 1986, as amended (the ``Code'') (and the
      regulations issued thereunder), as in effect as of the effective
      date of this Agreement, as such maximum benefit amount may be
      adjusted for future increases in the cost-of-living in the manner
      prescribed by Section 415(d)(2) of the Code. Effective for the
      Plan Year commencing February 2, 2018, and for each subsequent
      Plan Year during the term of this Agreement, the amount of the New
      Monthly Benefit shall be determined using the modified actuarial
      reduction factors to be specified in the Pension Plan by amendment
      effective for the Plan Year beginning February 2, 2018.
    \item
      Any increase in the Normal Retirement Pension payable on or after
      the date of this Agreement: (A) shall apply only to those players
      and beneficiaries (x) who have not yet received or begun to
      receive a benefit under the Pension Plan as of the first day of
      the month following the beginning of the Plan Year to which the
      increase relates (the ``New Benefit Increase Commencement Date'')
      or (y) who are receiving monthly benefits under the Pension Plan
      as of the New Benefit Increase Commencement Date; (B) shall be
      effective as of the New Benefit Increase Commencement Date; (C)
      shall apply only to any benefit payment(s) to be made on or after
      the applicable New Benefit Increase Commencement Date; and (D)
      shall not require the recalculation of any benefit payment(s) made
      prior to the applicable New Benefit Increase Commencement Date.
    \item
      The Pension Plan shall provide that the amount of the Normal
      Retirement Pension that exceeds the Fixed Part A Benefit (defined
      below) (other than any death benefits payable to a beneficiary
      pursuant to Article VII of the Pension Plan) shall not be payable
      in the form of a Lump Sum Option. The ``Fixed Part A Benefit''
      shall mean that portion of the Normal Retirement Pension equal to
      the amount described in Section 1.21 of the Pension Plan as of the
      effective date of this Agreement; provided, however, that no
      adjustments for increases in the cost-of-living that may go into
      effect after the effective date of this Agreement shall be taken
      into account for purposes of calculating the Fixed Part A Benefit.
    \item
      The Pension Plan shall be amended to provide that a player shall
      not be considered to be on the Roster for a Regular Season solely
      because he was under a 10-Day Contract or Two-Way Contract as of
      February 2nd of such Regular Season.
    \end{enumerate}
  \item
    Pre-1965 Players and Pre-1965 Retirees. Effective for the Plan Year
    commencing February 2, 2018, and for each subsequent Plan Year
    during the term of this Agreement:

    \begin{enumerate}
    \def\labelenumiii{(\roman{enumiii})}
    \tightlist
    \item
      The Pension Plan shall be amended to provide that the Normal
      Retirement Benefit payable to a Pre-1965 Player and the ``A
      portion'' of the Retirement Benefit payable to a Pre-1965 Retiree
      shall be \$400 per month for each Year of Pre-1965 Credited
      Service or Year of Eligible Pre-1965 Retiree Service, respectively
      (the ``2017-18 Pre-1965 Benefit Increase'').
    \item
      Any pension benefits that are unable to be paid to Pre- 1965
      Players and Pre-1965 Retirees because of the benefit limitations
      imposed by Section 415 of the Code shall be paid to such Pre-1965
      Players and Pre-1965 Retirees pursuant to the National Basketball
      Association Excess Benefit Plan for Pre-1965 Players (the
      ``Pre-1965 Players Excess Benefit Plan'').
    \item
      The increase provided by this Section 1(a)(3) shall: (A) apply
      only to those Pre-1965 Players, Pre-1965 Retirees or their
      beneficiaries who are receiving monthly pension benefits as of
      March 1, 2018; (B) apply only with respect to any pension benefit
      payment(s) made on or after March 1, 2018; and (C) not require the
      recalculation of any pension benefit payment(s) made prior to
      March 1, 2018.
    \end{enumerate}
  \item
    Limitations on Benefits. Notwithstanding anything contained herein
    to the contrary:

    \begin{enumerate}
    \def\labelenumiii{(\roman{enumiii})}
    \tightlist
    \item
      Neither: (A) the pension benefits accrued or payable to any player
      or beneficiary for a Plan Year nor (B) the New Monthly Benefit for
      a Plan Year shall exceed the maximum benefit amount permitted
      under the Code (and the regulations issued thereunder) as in
      effect for that Plan Year (as adjusted in accordance with the
      actuarial factors specified in the Pension Plan and as in effect
      on the date that the benefit accrues or commences (or is paid) or
      for the Plan Year for which the New Monthly Benefit is
      determined), as such maximum benefit amount may be adjusted for
      future increases in the cost-of-living in the manner provided
      under Section 415(d)(2) of the Code.
    \item
      Neither the pension benefits accrued nor payable to any player or
      beneficiary for a Plan Year shall exceed the maximum benefit
      amount permitted under the Code (and the regulations issued
      thereunder), as in effect as of the effective date of this
      Agreement, as adjusted in accordance with the actuarial factors
      specified in the Pension Plan, and as may be adjusted for future
      increases in the cost-of-living in the manner prescribed by
      Section 415(d)(2) of the Code.
    \item
      If all or any portion of the actuarially-determined annual
      contributions to be made to the Pension Plan would not be fully
      deductible under the Code when paid to the Pension Plan, the New
      Monthly Benefit shall not exceed the amount which would result in
      all of such contributions being fully-deductible when paid. In the
      event that any such contribution or portion thereof is not fully
      deductible when paid, the NBA and the Players Association agree to
      bargain in good faith with respect to an alternative arrangement
      to be provided by the NBA Teams to the players. The costs of any
      such alternative arrangement shall be at an annual cost (as
      determined on an after-tax basis) to the NBA Teams substantially
      equal to but no greater than the annual accrual cost that such
      Teams would have incurred under the Pension Plan to fund the
      amount by which the New Monthly Benefit is reduced pursuant to
      this Section 1(a)(4)(iii). If despite good faith negotiations, the
      NBA and the Players Association fail to agree with respect to an
      alternative arrangement as described above, such failure to agree
      shall not create any right: (A) to unilaterally implement during
      the term of this Agreement any terms concerning the provision of
      pension benefits to the players; (B) to lockout; or (C) to strike.
    \end{enumerate}
  \end{enumerate}
\item
  \textbf{Administration.} Effective as of the effective date of this
  Agreement, the NBA and the Players Association shall cause to be
  amended the Pension Plan and the Agreement of Trust, by and among the
  NBA Teams and certain individual Trustees, dated January 17, 1997 (the
  ``Pension Trust Agreement'') to provide that the Pension Plan shall be
  maintained and operated as described in this Section 1(b).

  \begin{enumerate}
  \def\labelenumii{(\arabic{enumii})}
  \tightlist
  \item
    Subject to Section 1(b)(5), which is expressly designed to survive
    the expiration or termination of this Agreement, the Pension Plan
    shall be jointly operated and administered by the NBA and Players
    Association in accordance with Section 302(c)(5) of the Labor
    Management Relations Act of 1947, as amended, and the provisions of
    the Pension Trust Agreement and the Pension Plan. The Pension Trust
    Agreement shall provide for a six (6)-member Board of Trustees (the
    ``Pension Plan Trustees''), three (3) of whom are to be appointed by
    the NBA and three (3) of whom are to be appointed by the Players
    Association; provided, however, that the daily operations of the
    Pension Plan shall be delegated to one or more independent
    third-party administrators, as selected by the Pension Plan Trustees
    in their sole discretion. In the exercise of its responsibilities,
    each such independent third-party administrator shall be required to
    comply with ERISA and all other applicable laws and to act in a
    manner that is consistent with the provisions of the Pension Trust
    Agreement and the Pension Plan.
  \item
    It is intended by the NBA and the Players Association that: (i) the
    Pension Plan shall continue to constitute a collectively-bargained
    multiemployer defined benefit pension plan that is tax-qualified
    under Section 401(a) of the Code; and (ii) the Pension Plan's
    corresponding Trust is exempt from taxation under the provisions of
    Section 501(a) of the Code.
  \item
    Subject to the Employee Retirement Income Security Act of 1974, as
    amended (``ERISA''), including, without limitation, ERISA's
    requirements applicable to pension plan fiduciaries, or other
    applicable law, the Pension Plan Trustees will give due
    consideration to past practice with regard to administrative
    determinations and interpretations.
  \item
    An arbitration provision will be added to the Pension Trust
    Agreement substantially in the form as follows: In the event that
    the Trustees cannot decide any question of the administration
    (within the meaning of Section 302(c)(5) of the Labor Management
    Relations Act of 1947, as amended) of the Pension Plan or Trust
    because of a tie vote or lack of a quorum at two (2) successive
    meetings of the Trustees, then the Trustees shall, upon written
    application of the NBA Trustees or the Players Association Trustees,
    submit such dispute to an impartial umpire in accordance with the
    American Arbitration Association's Impartial Umpire Rules for
    Arbitration of Impasses Between Trustees of Joint Employee Benefit
    Trust Funds. The decision of said umpire shall be final, binding and
    conclusive upon the Trustees and all persons concerned. To the
    extent permitted by applicable laws, the fee of the impartial umpire
    and the American Arbitration Association, together with such other
    costs and expenses as may be authorized by the Trustees, shall be
    proper charges against the Trust, which the Trustees are authorized
    to pay. The impartial umpire in his or her decision shall be bound
    by the provisions of the Pension Plan, the Pension Trust Agreement
    and the charters, rules, policies and procedures put in place by
    past or current Pension Plan fiduciaries (the ``Governing
    Documents''). The impartial umpire shall have no power or authority
    to add to or subtract from the Governing Documents or to change,
    modify or amend the provisions of the Governing Documents or the
    Collective Bargaining Agreement between the NBA and the Players
    Association. Notwithstanding the foregoing, neither the Trustees
    appointed by the NBA nor the Trustees appointed by the Players
    Association may compel arbitration regarding a claim for benefits
    that would be time-barred under the Pension Plan or applicable law.
  \item
    Notwithstanding anything in this Agreement to the contrary,
    immediately upon expiration or termination of this Agreement: (i)
    the Players Association shall forfeit its right to appoint Pension
    Plan Trustees; (ii) the appointment and Trusteeship of the
    then-current Pension Plan Trustees who had been appointed by the
    Players Association shall automatically end; and (iii) sponsorship,
    administration and operation of the Pension Plan shall automatically
    revert back to the NBA. The Pension Plan shall again become jointly
    operated and administered by the NBA and Players Association
    immediately upon their entering into a new collective bargaining
    agreement, and the provisions of this Section 1(b) shall be
    incorporated into such new agreement. Furthermore, in the event that
    the sponsorship and administration has reverted back to the NBA
    under this paragraph, and subject to ERISA, including, without
    limitation, ERISA's requirements applicable to pension plan
    fiduciaries, the Trustees who are appointed by the NBA will continue
    to give due consideration to past practice, including past practice
    established during the period of joint administration by the NBA and
    Players Association, with regard to administrative determinations
    and interpretations.
  \end{enumerate}
\item
  \textbf{Contributions/Funding.} The NBA and Players Association
  acknowledge and agree that the Teams shall continue at all times to
  contribute to the Plan at least the amount necessary to meet the
  Pension Plan's statutory minimum funding requirements under Section
  412, Section 431 and, if applicable, Section 432 of the Code, or any
  other applicable law (the ``Minimum Funding Standards'') for such Plan
  Year, as determined by the actuaries of the Pension Plan. For any
  period during the term of this Agreement during which a new ``funding
  improvement plan'' (a ``FIP'') is required to be adopted by the
  Pension Plan under the Minimum Funding Standards, the funding
  benchmark for such FIP shall equal the funding benchmark required by
  the Minimum Funding Standards. The Teams may, in the sole discretion
  of the NBA, contribute to the Pension Plan more than the amount
  necessary to meet the Minimum Funding Standards; provided, however,
  that any such additional contribution amount shall not be greater than
  the contribution amount determined by the actuaries of the Pension
  Plan in accordance with the Pension Plan's historical scheduled
  contribution methodology. All contributions shall be conditioned on
  their being fully deductible by the Teams when paid.
\item
  \textbf{Players Employed by Toronto.}

  \begin{enumerate}
  \def\labelenumii{(\arabic{enumii})}
  \tightlist
  \item
    Players employed by Maple Leaf Sports \& Entertainment Partnership
    (or any successor thereto) (``Toronto'') or by an NBA Team located
    in any country other than the United States shall receive pension
    benefits of comparable value. Except as otherwise provided in
    Section 1(d)(2), players employed by Toronto (``Toronto Players'')
    shall continue to receive such benefits by means of the Pension Plan
    and a separate pension plan maintained by Toronto (the ``Toronto
    Plan''); provided, however, that a player shall not be eligible to
    participate (or continue to participate) in the Pension Plan for any
    period of time during which the player is both a resident of Canada
    for income tax purposes and a Toronto Player (a ``Canadian
    Resident'') but shall instead be eligible to receive a cash payment
    as described in Section 7 below.
  \item
    If the participation of Toronto Players in the Pension Plan would,
    at any time, result in the Pension Plan becoming subject to Canadian
    provincial pension legislation and/or Canadian federal income tax
    laws (to the extent that the application of such laws would result
    in adverse tax consequences to the Pension Plan, the NBA Teams or
    the Toronto Players) or result in the Toronto Plan's failure, at any
    future time, to either be qualified under the Code or registered
    under Canadian provincial pension legislation or Canadian federal
    tax laws, then any obligation to establish, maintain or make
    contributions to the Pension Plan in respect of Toronto Players and
    the Toronto Plan pursuant to this Agreement or pursuant to any prior
    collective bargaining agreement shall terminate; provided, however,
    that any such termination shall not impair the legally binding
    effect of any other provision of this Agreement or the legally
    binding effect (if any) of any other provision of any prior
    collective bargaining agreement, nor shall it create any right: (i)
    to unilaterally implement during the term of this Agreement any
    terms concerning the provision of pension benefits to the players;
    (ii) to lockout; or (iii) to strike. In the event of such
    termination, the NBA and Players Association agree to bargain in
    good faith with respect to an alternative arrangement to be provided
    by Toronto to the Toronto Players. Any such alternative arrangement
    shall be at an annual cost (as determined on an after-tax basis) to
    Toronto substantially equal to but no greater than the annual
    accrual cost that Toronto would have incurred under the Pension Plan
    and the Toronto Plan. If despite good faith negotiations, the NBA
    and the Players Association fail to agree with respect to an
    alternative arrangement as described above, such failure to agree
    shall not create any right: (A) to unilaterally implement during the
    term of this Agreement any terms concerning the provision of pension
    benefits to the players; (B) to lockout; or (C) to strike.
  \end{enumerate}
\item
  \textbf{Pension Plan Tax-Qualification Status.} Notwithstanding
  anything else in this Agreement: (1) if any change or amendment made
  to the Code, ERISA, or other applicable law, or to any regulations
  (whether final, temporary or proposed) or rulings issued thereunder;
  (2) if any interpretation, application or enforcement (or any proposed
  interpretation, application or enforcement), by a court of competent
  jurisdiction in the United States or by the IRS, of the Code, ERISA,
  or other applicable law, or any regulations or rulings issued
  thereunder; (3) if any regulations (whether final, temporary or
  proposed) or rulings issued by the IRS under the Code or ERISA; or (4)
  if any provisions of this Agreement, including, without limitation,
  any of the amendments or benefit increases to be provided under the
  Pension Plan pursuant to this Section 1, would result in the Pension
  Plan no longer being a tax-qualified plan under Section 401(a) of the
  Code, or would require NBA Teams to incur costs over and above any
  costs required to be incurred to implement the provisions of this
  Agreement or any prior collective bargaining agreement in order for
  the Pension Plan to maintain its tax-qualified status under Section
  401(a) of the Code (but only to the extent that such additional costs
  are incurred in connection with the provision of pension benefits to
  their non-player employees or to non-player employees of affiliates
  (within the meaning of Sections 414(b), (c) or (m) of the Code) of
  such Teams), then any obligation to continue to provide for the
  accrual of additional benefits under the Pension Plan pursuant to this
  Agreement or pursuant to any prior collective bargaining agreement
  shall terminate; provided, however, that any such termination shall
  not impair the legally binding effect of any other provision of this
  Agreement or the legally binding effect (if any) of any other
  provision of any prior collective bargaining agreement, nor shall it
  create any right: (i) to unilaterally implement during the term of
  this Agreement any terms concerning the provision of pension benefits
  to the players; (ii) to lockout; or (iii) to strike. In the event of
  such termination, the NBA and Players Association agree to bargain in
  good faith with respect to an alternative arrangement to be provided
  by the NBA Teams to the players. The costs of any such alternative
  arrangement shall be at an annual cost (as determined on an after-tax
  basis) to the NBA Teams substantially equal to but no greater than the
  annual accrual cost that such Teams would have incurred under the
  Pension Plan to fund the benefit described in this Section 1,
  commencing on the date of termination. If despite good faith
  negotiations, the NBA and the Players Association fail to agree with
  respect to an alternative arrangement as described above, such failure
  to agree shall not create any right: (A) to unilaterally implement
  during the term of this Agreement any terms concerning the provision
  of pension benefits to the players; (B) to lockout; or (C) to strike.
\item
  \textbf{Amounts to be Applied Against New Benefit Amount.} The
  following amounts shall be applied against the New Benefit Amount
  provided for by Section 8 below:

  \begin{enumerate}
  \def\labelenumii{(\arabic{enumii})}
  \tightlist
  \item
    \$4 million to be used in respect of the cost of the benefit
    increases described in Section 1(a)(2);
  \item
    Fifty percent (50\%) of the increase in the amount of the
    actuarially-determined annual contributions to be made for a Plan
    Year to the Pension Plan to fund the Normal Retirement Benefits for
    Pre-1965 Players described in Section 20.3(a)(v) of the Pension Plan
    over the amount of the actuarially determined annual contributions
    that would be required to be made for that Plan Year to the Pension
    Plan in order to fund the Normal Retirement Benefits for Pre-1965
    Players described in Section 20.3(a)(iv) of the Pension Plan had
    Section 20.3(a)(v) of the Pension Plan never been in effect (the
    ``Pre-1965 Player Cost Increase'');
  \item
    Fifty percent (50\%) of the costs incurred for a Plan Year in order
    to provide the excess portion of the benefit amount described in
    Pension Plan Section 20.3(a)(v) under the Pre-1965 Players Excess
    Benefit Plan over the costs that would be incurred for that Plan
    Year in order to provide for the excess portion of the benefit
    amount described in Pension Plan Section 20.3(a)(iv) under the
    Pre-1965 Players Excess Benefit Plan had Section 20.3(a)(v) of the
    Pension Plan never been in effect (the ``Pre-1965 Player Excess
    Benefit Cost Increase'');
  \item
    Fifty percent (50\%) of the amount of the actuarially-determined
    annual contributions to be made for a Plan Year to the Pension Plan
    to fund the Retirement Benefits for Pre-1965 Retirees described in
    Section 21.3(a) of the Pension Plan (the ``Pre-1965 Retiree Cost
    Increase''); and
  \item
    One hundred percent (100\%) of the costs, including the cost of
    professional fees (e.g., attorneys, accountants, actuaries and
    consultants) (``Professional Fees''), incurred in connection with
    the determination and implementation of any alternative benefits
    pursuant to Sections 1(d) and/or 1(e). For purposes of this Section
    1(f), in determining the Pre-1965 Player Cost Increase, the Pre-1965
    Player Excess Benefit Cost Increase, and the Pre- 1965 Retiree Cost
    Increase for a Plan Year, the annual contributions relating to such
    cost increases (and in the case of the Pre-1965 Player Excess
    Benefit Cost Increase, the costs incurred relating to such cost
    increase) shall be determined based on the applicable laws in effect
    for that Plan Year, taking into account (i) any new law or change or
    amendment made to ERISA, the Code and/or other applicable law, or to
    any regulations (whether final, temporary or proposed), rulings or
    formal guidance issued thereunder and (ii) any regulations (whether
    final, temporary or proposed), rulings or formal guidance issued
    under ERISA, the Code or other applicable law.
  \end{enumerate}
\item
  \textbf{Actuarial Determinations.} All actuarial determinations that
  need to be made in connection with, or under, the Pension Plan,
  including, without limitation, those necessary to implement this
  Section 1 and Sections 8 and 10 below, shall be made by the actuaries
  of the Pension Plan. Any such actuarial determinations shall be
  binding and conclusive.
\end{enumerate}

\section{Player 401(k) Benefits.}\label{player-401k-benefits.}

To the extent permitted by the Code and applicable law, the NBA shall
provide the following 401(k) benefits to NBA players and former NBA
players in accordance with and subject to the terms and conditions of
the NBA-NBPA 401(k) Savings Plan as restated effective November 1, 2014,
and as amended from time to time and to be modified as set forth herein
(the ``401(k) Plan''); provided, however, that, the 401(k) Plan shall be
amended, as of the effective date of this Agreement, to change the plan
name to the ``National Basketball Association Players' 401(k) Savings
Plan.'' (All capitalized terms used in this Section 2 not otherwise
defined in this Agreement shall have the meanings set forth in the
401(k) Plan and, for purposes of this Section 2, the term
``Compensation'' shall have the meaning set forth in the 401(k) Plan and
not Article I or Exhibit A of this Agreement.)

\begin{enumerate}
\def\labelenumi{(\alph{enumi})}
\tightlist
\item
  \textbf{Current Benefits.} For each Plan Year commencing during the
  term of this Agreement, the 401(k) Plan shall continue to provide for:
  (1) Salary Deferral Contributions by players, (2) except as may be
  limited below, Matching Contributions (other than Two-Way Matching
  Contributions, as defined below) by Teams in respect of Salary
  Deferral Contributions for a Salary Cap Year, as requested in writing
  by the Players Association, and (3) After Tax Contributions by
  players. The request for the Matching Contributions (other than
  Two-Way Matching Contributions) by the Players Association for a
  Season shall be made in writing prior to the commencement of that
  Season.
\item
  \textbf{Two-Way Player Benefits.} For each Plan Year commencing during
  the term of this Agreement, the 401(k) Plan shall provide the
  following benefits for Two-Way Players:

  \begin{enumerate}
  \def\labelenumii{(\arabic{enumii})}
  \tightlist
  \item
    Each Two-Way Player who is on the Active List, Inactive List or
    Two-Way List of any Team during a Regular Season shall be included
    as an Eligible Player subject to the terms and conditions of the
    401(k) Plan. Unless a Two-Way Player who is an Eligible Player
    affirmatively elects a different Salary Deferral Contribution
    (including, without limitation, no Salary Deferral Contribution) for
    a Season, such player's Compensation shall be automatically reduced
    for the payroll periods occurring between the beginning of the
    Regular Season in which the Two-Way Player is on a Roster and the
    June 30 immediately following the end of such Regular Season in
    substantially equal amounts which, when summed together, equal the
    maximum deferral amount permitted under Section 402(g) of the Code
    in effect for such Season. Such amounts shall be deposited to the
    401(k) Plan as Salary Deferral Contributions.
  \item
    Except as may be limited below, each Two-Way Player who is an
    Eligible Player shall be entitled to receive a ``Two-Way Matching
    Contribution'' equal to the lesser of: (i) twenty-five percent
    (25\%) of his Salary Deferral Contributions; and (ii) one percent
    (1\%) of his Compensation; provided, however, that if the Two-Way
    Player is signed or converted to a Standard NBA Contract during a
    Salary Cap Year and meets the requirements under the 401(k) Plan to
    be an Eligible Player without regard to the changes to the 401(k)
    Plan that are contemplated by this Section 2(b), then his Two-Way
    Matching Contribution in respect of his Salary Deferral
    Contributions made during such Salary Cap Year shall not be
    determined pursuant to this Section 2(b) but shall instead be
    determined pursuant to Section 2(a) above.
  \item
    The benefits payable to Two-Way Players under the 401(k) Plan,
    including, without limitation, the Two-Way Matching Contribution,
    shall otherwise be subject to the terms and conditions set forth in
    the 401(k) Plan.
  \end{enumerate}

  \begin{enumerate}
  \def\labelenumii{(\alph{enumii})}
  \setcounter{enumii}{2}
  \tightlist
  \item
    Timing of Matching Contributions and Two-Way Matching Contributions.
    Any Matching Contributions and Two-Way Matching Contributions to be
    made to the 401(k) Plan in respect of each Season shall be made no
    later than thirty (30) days following the completion of the Audit
    Report for the Salary Cap Year covering such Season.
  \item
    Limitations on Benefits. Notwithstanding anything contained herein
    to the contrary:

    \begin{enumerate}
    \def\labelenumiii{(\arabic{enumiii})}
    \tightlist
    \item
      Matching Contributions, Two-Way Matching Contributions, Salary
      Deferral Contributions and After Tax Contributions shall at all
      times be subject to all applicable limitations under the Code,
      including, without limitation, the maximum limitation on
      contributions under Code Section 415, the maximum limitation on
      compensation under Code Section 401(a)(17), and the maximum
      limitation on 401(k) deferrals under Code Section 402(g).
    \item
      The total amount of the Salary Deferral Contributions, Matching
      Contributions and Two-Way Matching Contributions to be made to the
      401(k) Plan shall be limited to an amount that, taking into
      account only Compensation paid to current players by the Teams,
      would result in all of such Salary Deferral Contributions,
      Matching Contributions and Two-Way Matching Contributions being
      fully deductible under the Code (and, where applicable, Canadian
      income tax laws) when paid to the 401(k) Plan. If, for any reason,
      all or a portion of the Salary Deferral Contributions, Matching
      Contributions and/or Two-Way Matching Contributions to be made to
      the 401(k) Plan will not, when paid to the 401(k) Plan, be fully
      deductible under the Code, the NBA and the Players Association
      agree that the contributions shall be reduced to result in all
      such contributions being fully deductible when paid.
    \end{enumerate}
  \end{enumerate}
\item
  \textbf{Players Employed by Toronto.} The terms of the 401(k) Plan
  shall continue to permit participation by Toronto Players on a
  tax-effective basis under Canadian income tax laws; provided, however,
  that a player shall not be eligible to participate in the 401(k) Plan
  for the period of time during which the player is a Canadian Resident
  but shall instead be eligible to receive a cash payment as described
  in Section 7 below. If the NBA and the Players Association should
  determine that the 401(k) Plan cannot continue to be provided to
  Toronto Players on a tax-effective basis under Canadian federal income
  tax laws, the NBA and Players Association agree to bargain in good
  faith with respect to an alternative arrangement to be provided by
  Toronto to the Toronto Players. The costs of any such alternative
  arrangement shall be at an annual cost (as determined on an after-tax
  basis) to Toronto substantially equal to but no greater than the
  annual cost that Toronto would have incurred under the 401(k) Plan
  with respect to the Matching Contributions for the Toronto Players.
  The cost to Toronto of providing for any such alternative arrangement:
  (1) shall be applied against the New Benefit Amount provided for by
  Section 8 below, and (2) shall be limited to the portion of the New
  Benefit Amount, if any, that is available for this purpose pursuant to
  Section 8(b)(2) below. If despite good faith negotiations, the NBA and
  the Players Association fail to agree with respect to an alternative
  arrangement as described above, such failure to agree shall not create
  any right: (i) to unilaterally implement during the term of this
  Agreement any terms concerning the provision of 401(k) benefits to the
  players; (ii) to lockout; or (iii) to strike.
\item
  \textbf{401(k) Plan Tax-Qualification Status.} Notwithstanding
  anything else in this Agreement: (1) if any change or amendment made
  to the Code, ERISA, or other applicable law, or to any regulations
  (whether final, temporary or proposed) or rulings issued thereunder;
  (2) if any interpretation, application or enforcement (or any proposed
  interpretation, application or enforcement), by a court of competent
  jurisdiction in the United States or by the IRS, of the Code, ERISA,
  or other applicable law, or any regulations or rulings issued
  thereunder; (3) if any regulations (whether final, temporary or
  proposed) or rulings issued by the IRS under the Code or ERISA; or (4)
  if any provisions of this Agreement would result in the 401(k) Plan no
  longer being a tax-qualified plan under Section 401(a) of the Code, or
  would require NBA Teams to incur costs over and above any costs
  required to be incurred to implement the provisions of this Agreement
  or any prior collective bargaining agreement in order for the 401(k)
  Plan to maintain its tax-qualified status under Section 401(a) of the
  Code (but only to the extent that such additional costs are incurred
  in connection with the provision of benefits to their non-player
  employees or to non-player employees of affiliates (within the meaning
  of Sections 414(b), (c) or (m) of the Code) of such Teams), then any
  obligation to maintain or make contributions to the 401(k) Plan
  pursuant to this Agreement or pursuant to any prior collective
  bargaining agreement shall terminate; provided, however, that any such
  termination shall not impair the legally binding effect of any other
  provision of this Agreement or the legally binding effect (if any) of
  any other provision of any prior collective bargaining agreement, nor
  shall it create any right: (i) to unilaterally implement during the
  term of this Agreement any terms concerning the provision of 401(k)
  benefits to the players; (ii) to lockout; or (iii) to strike. In the
  event of such termination, the NBA and Players Association agree to
  bargain in good faith with respect to an alternative arrangement to be
  provided by the NBA Teams to the players. Any such alternative
  arrangement shall be at an annual cost (as determined on an after-tax
  basis) to the NBA Teams substantially equal to but no greater than the
  annual cost that such Teams would have incurred under the 401(k) Plan
  with respect to Matching Contributions and Two-Way Matching
  Contributions commencing on the date of termination. The cost of any
  such alternative arrangement shall: (A) be applied against the New
  Benefit Amount provided by Section 8 below, and (B) be limited to the
  portion of the New Benefit Amount, if any, that is available for this
  purpose pursuant to Section 8(b)(2) below. If despite good faith
  negotiations, the NBA and the Players Association fail to agree with
  respect to an alternative arrangement as described above, such failure
  to agree shall not create any right: (x) to unilaterally implement
  during the term of this Agreement any terms concerning the provision
  of 401(k) benefits to the players; (y) to lockout; or (z) to strike.
\item
  \textbf{Amounts to be Applied Against New Benefit Amount.} The
  following amounts shall be applied against the New Benefit Amount
  provided for by Section 8 below:

  \begin{enumerate}
  \def\labelenumii{(\arabic{enumii})}
  \tightlist
  \item
    The cost of funding Matching Contributions shall: (i) be applied
    against the New Benefit Amount provided for by Section 8 below, and
    (ii) be limited to the portion of the New Benefit Amount, if any,
    that is available for this purpose pursuant to Section 8(b)(2)
    below.
  \item
    All costs incurred in connection with the operation and
    administration of the 401(k) Plan (and in connection with the
    determination and implementation of any alternative arrangement
    pursuant to Section 2(e) and/or Section 2(f)), including, without
    limitation, the cost of Professional Fees and the 401(k) Plan's
    recordkeeper's fixed fee for recordkeeping and other administrative
    services provided to the 401(k) Plan, shall be (i) paid by the
    Teams; (ii) applied against the New Benefit Amount provided for in
    Section 8 below; and (iii) limited to the portion of the New Benefit
    Amount, if any, that is available for this purpose pursuant to
    Section 8(b)(2) below. Notwithstanding the previous sentence, this
    Section 2(g)(2) shall not apply to: (A) any costs or fees
    attributable to a participant-initiated transaction under the 401(k)
    Plan or (B) any investment fees or expenses charged directly against
    the return on any investment options under the 401(k) Plan.
  \end{enumerate}
\end{enumerate}

\section{Player Health and Welfare
Benefits.}\label{player-health-and-welfare-benefits.}

Except as set forth below in this Section 3, as of the effective date of
this Agreement, and continuing until the expiration or termination of
this Agreement, to the extent permitted by applicable law, the NBA shall
provide the following health and welfare benefits to NBA players and
former NBA players in accordance with and subject to the terms and
conditions of the NBPA-NBA Supplemental Benefit Plan as restated July 1,
2016, and as amended from time to time and to be modified as set forth
herein, (the ``Health and Welfare Benefit Plan'') and the Agreement and
Declaration of Trust of the NBPA-NBA Supplemental Benefit Plan,
established July 22, 2004 (the ``Health and Welfare Benefit Trust
Agreement'' and the trust, the ``Health and Welfare Benefit Trust'');
provided, however, that the Health and Welfare Benefit Plan and the
Health and Welfare Benefit Trust Agreement shall be amended, as of the
effective date of this Agreement, to change the plan name to the ``NBA
Players' Health and Welfare Benefit Plan'' and to change the trust name
to the ``NBA Players' Health and Welfare Benefit Trust.'' (All
capitalized terms used in this Section 3 not otherwise defined in this
Agreement shall have the meanings set forth in the Health and Welfare
Benefit Plan.)

\begin{enumerate}
\def\labelenumi{(\alph{enumi})}
\tightlist
\item
  \textbf{Benefits.} The Health and Welfare Benefit Plan shall be
  amended to include the following benefits consistent with Sections
  3(a)(1) through 3(a)(9) below, and the benefits described in those
  Sections shall be operated and administered through the Health and
  Welfare Benefit Trust.

  \begin{enumerate}
  \def\labelenumii{(\arabic{enumii})}
  \tightlist
  \item
    A health reimbursement arrangement (the ``HRA Benefit'') for players
    who played in the NBA during and/or after the 2000-01 Season will
    continue to be operated in accordance with the Health and Welfare
    Benefit Plan, which arrangement shall be administered and operated
    in compliance with IRS and U.S. Department of Labor rules applicable
    to such arrangements.

    \begin{enumerate}
    \def\labelenumiii{(\roman{enumiii})}
    \tightlist
    \item
      An amount to fund an HRA Benefit for each eligible player in
      respect of each Salary Cap Year that, except as may be limited by
      Section 3(e), shall equal the lesser of: (A) \$30,000 and (B) the
      difference between \$150,000 and the sum of all contributions
      previously made to fund an HRA Benefit for such player in respect
      of prior Salary Cap Years (including, for clarity, contributions
      made prior to the 2017-18 Salary Cap Year) or, if such difference
      is \$0 or a negative number, then \$0; provided, however, that the
      Health and Welfare Benefit Plan shall be amended to provide that
      amounts to fund HRA Benefits for each Salary Cap Year shall be
      provided to eligible players by first re-allocating from amounts
      forfeited under the Health and Welfare Benefit Plan to the
      Individual Accounts of each such eligible player or, if such
      amounts are insufficient for this purpose, then as described in
      Section 3(e)(1) below.
    \item
      Except as may otherwise be agreed to by the NBA and the Players
      Association, any contributions to fund the HRA Benefit in respect
      of each Salary Cap Year shall be made no later than ninety (90)
      days following the completion of the Audit Report for such Salary
      Cap Year.
    \end{enumerate}
  \item
    The following insurance benefits provided to players:

    \begin{enumerate}
    \def\labelenumiii{(\roman{enumiii})}
    \tightlist
    \item
      Life insurance and accidental death and dismemberment benefits,
      which, as of the date of this Agreement, are being provided
      through the Metropolitan Life Insurance Company Policy No.
      0122986; provided, however, that life insurance and accidental
      death and dismemberment benefits for Two-Way Players shall be
      provided through a policy that is substantially similar to the
      life insurance and accidental death and dismemberment policy
      covering NBADL players.
    \item
      For players other than Two-Way Players, disability insurance
      benefits, which, as of the date of this Agreement, are being
      provided through the Houston Casualty Company Policy No.
      16/7007220.
    \item
      Except as otherwise provided in Section 3(a)(2)(iv), medical,
      dental and prescription drug insurance benefits which, as of the
      date of this Agreement, are being provided through the CIGNA
      HealthCare Policy No. 3211244; provided, however, that for Two-Way
      Players, medical and prescription drug insurance benefits shall be
      provided through a policy that is substantially similar to the
      Standard policy covering NBADL players with regard to coverage
      levels, scope of in-network and out-of-network coverage,
      deductibles, co-insurance, co-pays, out-of-pocket maximums and the
      employee share of premium contributions (which share shall be
      calculated as described below), and dental insurance benefits
      shall be provided through a policy that is substantially similar
      to the DPPO policy covering NBADL players with regard to coverage
      levels, scope of in-network and out-of-network coverage,
      deductibles, co-insurance, co-pays, out-of-pocket maximums and the
      employee share of premium contributions (which share shall be
      calculated as described below). For purposes of this Section
      3(a)(2)(iii) and Section 10(b)(6) below, the employee share of
      premium contributions for each coverage level under the relevant
      Two-Way Players' insurance policy shall be calculated by
      multiplying: (A) the total monthly premium payment for a Two-Way
      Player who elected that coverage level under the relevant Two-Way
      Players' insurance policy that Season; by (B) a fraction,
      expressed as a percentage of premium payment, the numerator of
      which is the portion of the total monthly premium payment
      contributed by an NBADL player for the same coverage level under
      the corresponding insurance policy covering NBADL players during
      the NBADL regular season occurring within the Salary Cap Year
      immediately preceding that Season, and the denominator of which is
      the total monthly premium payment for that NBADL player for the
      same coverage level under the corresponding insurance policy
      covering NBADL players during the NBADL regular season occurring
      within the Salary Cap Year immediately preceding that Season.
    \item
      For players other than Two-Way Players who are ``qualified
      expatriates'' under the Expatriate Health Coverage Clarification
      Act of 2014, expatriate medical and prescription drug insurance
      benefits.
    \item
      Vision insurance benefits which, as of the date of this Agreement,
      are being provided through the EyeMed Vision Care Policy No.
      9886987; provided, however, that Two-Way Players shall be provided
      vision benefits that are substantially similar to the vision
      benefits being provided to NBADL players as of the effective date
      of this Agreement.
    \end{enumerate}
  \item
    The Health and Welfare Benefit Plan shall be amended to provide that
    the period of coverage for the insurance benefits described in
    Section 3(a)(2)(iii) and 3(a)(2)(iv) shall extend: (a) for players
    (other than Veteran Free Agents and Two-Way Players), until the last
    day of the month during which the player ceases to be on either the
    Active List or Inactive List; (b) for Veteran Free Agents, until the
    August 31 following the last Season of the player's Contract; and
    (c) for Two-Way Players, from the first day of the Season until the
    earlier of (1) the last day of the month during which the player
    ceases to be on the Active List, Inactive List, or Two-Way List and
    (2) the last day of the month during which the Season ends.
  \item
    All of the benefits provided for in Section 3(a)(2) are subject to
    their permissibility and availability under applicable law.
  \item
    The Board of Trustees of the Health and Welfare Benefit Trust (the
    ``Health and Welfare Trustees'') may make changes to any of the
    insurance programs provided under Section 3(a)(2), provided that any
    such change that would result in an increase in the costs or a
    change in the types or levels of any of the benefits, or that would
    change any such program from an insured program to a self-insured
    program or vice versa, must be mutually agreed to in writing by the
    NBA and the Players Association.
  \item
    Subject to Section 3(a)(6)(i)-(ii) below, the NBA and the Players
    Association shall provide retiree health insurance benefits which,
    as of the effective date of this Agreement, are being provided
    through UnitedHealthcare Policy Numbers 908971, 16160, 16161 and
    16162 (collectively, the ``Retiree Medical Plan'').

    \begin{enumerate}
    \def\labelenumiii{(\roman{enumiii})}
    \tightlist
    \item
      The Retiree Medical Plan will be established only for the term of
      this Agreement; provided, however, that the NBA and the Players
      Association (or, if so delegated by the NBA and the Players
      Association in writing, the Health and Welfare Trustees) reserve
      the right, by mutual written agreement, to modify, amend or
      terminate, in whole or in part, the Retiree Medical Plan with
      respect to any or all eligible retirees and their eligible
      dependents at any time or for any reason, and no eligible retirees
      or eligible dependents (or other NBA players, retired NBA players
      or their dependents) shall under any circumstances have any vested
      rights of any nature with respect to the Retiree Medical Plan or
      any retiree health benefit (whether or not the player or retired
      player, or their dependents, has participated in the Retiree
      Medical Plan).
    \item
      The NBA and the Players Association (or, if so delegated by the
      NBA and the Players Association in writing, the Health and Welfare
      Trustees) reserve the right, by mutual written agreement, to
      increase or otherwise change the amount of monthly premiums under
      the Retiree Medical Plan charged to players at any time and for
      any reason.
    \end{enumerate}
  \item
    Effective as of the effective date of this Agreement, the Health and
    Welfare Benefit Plan shall be amended to provide for reimbursement
    of eligible tuition and career transition expenses, subject to the
    following reimbursement limits: (i) for an eligible player, the
    lesser of (A) a maximum amount of \$33,654 for each Salary Cap Year
    and (B) a maximum aggregate amount of \$101,000 over all Salary Cap
    Years; and (ii) for all eligible players, amounts not to exceed a
    maximum aggregate amount of \$4,276,185 for each Salary Cap Year. On
    a quarterly basis, the NBA shall cause to be provided to the Health
    and Welfare Benefit Trust (or directly to the independent
    third-party administrators) contributions equal to the aggregate
    amount of reimbursable eligible tuition and career transition
    expenses, if any, approved in the prior quarter.
  \item
    The Health and Welfare Benefit Plan shall be amended to provide that
    a player shall not be considered to be on an NBA Team Roster for a
    Regular Season solely because he was under a 10-Day Contract or
    Two-Way Contract as of February 2nd of such Regular Season.
  \item
    Following the execution of this Agreement, the NBA and the Players
    Association shall establish a mental wellness program for current
    players; provided, that (i) the terms of, and the benefits to be
    provided under, the mental wellness program shall be negotiated in
    good faith by the NBA and the Players Association and (ii) if
    despite good faith negotiations, the NBA and the Players Association
    fail to agree with respect to the terms of, and the benefits to be
    provided under, the mental wellness program, such failure to agree
    shall not create any right (A) to unilaterally implement during the
    term of this Agreement any terms concerning the provision of mental
    wellness benefits to the players; (B) to lockout; or (C) to strike.
    In addition, the NBA and the Players Association agree to meet and
    confer to discuss the establishment of a long-term care insurance
    benefit; provided, however, that, neither party shall have any
    obligation to establish a long-term care insurance benefit. The
    specific benefits to be provided under the mental wellness program
    and any long-term care insurance benefit shall be (x) subject to the
    agreement of the NBA and the Players Association and (y) governed by
    a definitive written amendment to the Health and Welfare Benefit
    Plan implementing such program or benefit.
  \end{enumerate}
\item
  \textbf{Administration.}

  \begin{enumerate}
  \def\labelenumii{(\arabic{enumii})}
  \tightlist
  \item
    The Health and Welfare Benefit Trust shall continue to be jointly
    operated and administered by the NBA and Players Association in
    accordance with Section 302(c)(5) of the Labor Management Relations
    Act of 1947, as amended, and the provisions of the Health and
    Welfare Benefit Trust Agreement and the Health and Welfare Benefit
    Plan, as to be amended pursuant to this Agreement. It is intended by
    the NBA and Players Association that the Health and Welfare Benefit
    Plan and Health and Welfare Benefit Trust shall continue to
    constitute a collectively-bargained voluntary employees' beneficiary
    association (``VEBA'') that qualifies as a tax exempt organization
    under the provisions of Section 501(c)(9) of the Code.
  \item
    The Health and Welfare Benefit Trust Agreement shall continue to
    provide that the Health and Welfare Benefit Trust and Health and
    Welfare Benefit Plan will be administered by the Health and Welfare
    Trustees. The daily operations of the Health and Welfare Benefit
    Plan and each of the benefits provided thereunder shall be delegated
    as follows: (i) in the case of the benefits set forth in Section
    3(a)(1) and 3(a)(7), to one or more independent third-party
    administrators; (ii) in the case of the benefits set forth in
    Section 3(a)(2), to insurers and/or one or more independent
    third-party administrators; and (iii) in the case of the benefits
    set forth in Section 3(a)(6), to an insurer and one or more
    independent third-party administrators. Each such insurer and
    independent third-party administrator referenced in this Section
    3(b)(2): (A) shall be selected by the Health and Welfare Trustees in
    their sole discretion; and (B) shall be required, in the exercise of
    its responsibilities, to comply with ERISA and all other applicable
    laws and to act in a manner that is consistent with the provisions
    of the Health and Welfare Benefit Trust and the Health and Welfare
    Benefit Plan.
  \item
    For the avoidance of doubt, nothing in this Section 3(b) shall
    prevent the Education Trust (defined below) from engaging or hiring
    an academic advisor or career counselor to assist with player
    outreach and similar functions with respect to the tuition
    reimbursement and career transition program set forth in Section
    3(a)(7).
  \end{enumerate}
\item
  \textbf{Players Employed by Toronto.} The terms of the Health and
  Welfare Benefit Plan shall permit participation by Toronto Players on
  the same basis as players who are not Toronto Players; provided,
  however, that a player shall not be eligible to participate in the HRA
  Benefit for the period of time during which the player is a Canadian
  Resident but shall instead be eligible to receive a cash payment as
  described in Section 7 below. If the NBA and the Players Association
  determine that the Health and Welfare Benefit Plan cannot provide one
  or more of the benefits described in Section 3(a) to Toronto Players
  (1) that are substantially equivalent to the benefits provided to
  players employed by Teams located in the United States or (2) on a
  tax-effective basis under Canadian federal income tax laws, the NBA
  and Players Association agree to bargain in good faith with respect to
  an alternative arrangement to be provided by Toronto to the Toronto
  Players. The annual cost incurred by the Teams in connection with any
  such alternative arrangement (as determined on an after-tax basis)
  shall not exceed the annual cost that such Teams would have incurred
  to fund the applicable benefit(s) described in Section 3(a) for such
  Toronto Player. The cost to Toronto of funding an alternative
  arrangement to the HRA Benefit shall be applied against the New
  Benefit Amount provided for by Section 8 below, and shall be limited
  to the portion of the New Benefit Amount, if any, that is available
  for this purpose pursuant to Section 8(b)(7) below. The cost to
  Toronto of funding any alternative arrangement(s) to any of the
  benefit(s) described in Section 3(a) shall be subject to the
  limitations set forth in this Agreement. If despite good faith
  negotiations, the NBA and the Players Association fail to agree with
  respect to any alternative arrangement(s) as described above, such
  failure to agree shall not create any right (i) to unilaterally
  implement during the term of this Agreement any terms concerning the
  provision of benefits provided or to be provided by the Health and
  Welfare Benefit Plan; (ii) to lockout; or (iii) to strike.
\item
  \textbf{Deductibility of Contributions/Regulatory Changes.}

  \begin{enumerate}
  \def\labelenumii{(\arabic{enumii})}
  \tightlist
  \item
    The Health and Welfare Benefit Trust and the Health and Welfare
    Benefit Plan shall be operated and administered in a manner that
    will result in all contributions by the Teams being fully deductible
    under the Code (and, where applicable, Canadian income tax laws)
    when paid to the Health and Welfare Benefit Trust (or directly to an
    insurance carrier for a benefit provided under the Health and
    Welfare Benefit Plan). If any Team is disallowed a deduction (in
    whole or in part) for such contributions, and unless the NBA
    determines otherwise, the obligation to provide the benefit (or
    portion of the benefit under the Health and Welfare Benefit Plan to
    which the contribution relates) and to make further contributions to
    provide the benefit (or portion of the benefit under the Health and
    Welfare Benefit Plan to which the contribution relates) shall
    immediately terminate and the provisions of Section 3(d)(3) shall
    apply.
  \item
    In the event that any benefit under the Health and Welfare Benefit
    Plan is no longer permissible or available due to applicable laws (a
    ``Regulatory Change''), the obligation to provide the benefit shall
    immediately terminate and the provisions of Section 3(d)(3) shall
    apply.
  \item
    Any termination of the Health and Welfare Benefit Plan or a benefit
    under such plan pursuant to Sections 3(d)(1)-(2) shall not impair
    the legally binding effect of any other provision of this Agreement,
    or the legally binding effect (if any) of any other provision of any
    prior collective bargaining agreement, nor shall it create any right
    (i) to unilaterally implement, during the term of this Agreement,
    any terms concerning the provision of the Health and Welfare Benefit
    Plan (or the applicable benefit provided or to be provided); (ii) to
    lockout; or (iii) to strike. In the event of any termination
    pursuant to Sections 3(d)(1)-(2) of the Health and Welfare Benefit
    Plan or a benefit under such plan, the NBA and Players Association
    agree to bargain in good faith with respect to alternative
    arrangement(s) to be provided by the NBA Teams to the players;
    provided, however, that any such alternative arrangement(s) shall be
    subject to the terms and conditions set forth in this Agreement,
    including, without limitation, with respect to an alternative
    arrangement to the Retiree Medical Plan, the terms and conditions
    set forth in Section 3(a)(6). The annual cost incurred by the NBA
    Teams in connection with any such alternative arrangement(s) (as
    determined on an after-tax basis) shall not exceed the annual cost
    that such Teams would have incurred in providing the relevant
    benefit(s) under the Health and Welfare Benefit Plan commencing on
    the date of termination. Any such alternative arrangement(s) shall,
    to the extent permitted by applicable law and the Health and Welfare
    Benefit Plan, be funded by such monies as may then remain in the
    Health and Welfare Benefit Trust and, if the monies remaining in the
    Health and Welfare Benefit Trust may not lawfully be used for, or
    are insufficient for, such purpose, such alternative arrangement(s)
    shall be funded by the NBA Teams. Any such alternative
    arrangement(s) shall be operated and administered in a manner that
    will result in all contributions by the Teams being fully deductible
    under the Code (and, where applicable, Canadian income tax laws)
    when paid. If funded by the Teams (and not funded out of monies
    remaining in the Health and Welfare Benefit Trust), the costs of
    funding an alternative arrangement to the HRA Benefit shall be
    applied against the New Benefit Amount provided for by Section 8
    below, and shall be limited to the portion of the New Benefit
    Amount, if any, that is available for this purpose pursuant to
    Section 8(b)(7) below. The costs of funding any alternative
    arrangement(s) shall be subject to the limitations set forth in this
    Agreement. If despite good faith negotiations, the NBA and the
    Players Association fail to agree with respect to any alternative
    arrangement(s) as described above, such failure to agree shall not
    create any right: (x) to unilaterally implement, during the term of
    this Agreement, any terms concerning the provision of benefits
    provided or to be provided by the Health and Welfare Benefit Plan;
    (y) to lockout; or (z) to strike.
  \end{enumerate}
\item
  \textbf{Amounts to be Applied Against New Benefit Amount.} The
  following amounts shall be applied against the New Benefit Amount
  provided for by Section 8 below:

  \begin{enumerate}
  \def\labelenumii{(\arabic{enumii})}
  \tightlist
  \item
    All costs of the HRA Benefit that are paid by the Teams (and not
    paid out of the amounts forfeited under the Health and Welfare
    Benefit Plan) shall be (i) applied against the New Benefit Amount
    provided for by Section 8 below, and (ii) limited to the portion of
    the New Benefit Amount, if any, that is available for this purpose
    pursuant to Section 8(b)(7) below. For the avoidance of doubt, to
    the extent that costs of the HRA Benefit in a Salary Cap Year are
    funded through amounts forfeited under the Health and Welfare
    Benefit Plan that were applied against the New Benefit Amount and
    included in the calculation of Benefits in a prior Salary Cap Year
    (or through earnings on such forfeited amounts), (A) such costs
    shall not be applied against the New Benefit Amount; and (B) such
    costs shall be excluded for purposes of all calculations called for
    under this Agreement of, or relating to, Benefits (including,
    without limitation, for purposes of: (1) preparing the Audit Report,
    Interim Audit Report, or Interim Escrow Audit Report, and (2)
    calculating Total Benefits, Total Salaries and Benefits, and
    Projected Benefits).
  \item
    All costs, including, without limitation, the cost of Professional
    Fees, incurred in connection with (i) the operation and
    administration of the Health and Welfare Benefit Plan Trust and the
    Health and Welfare Benefit Plan and (ii) the determination and
    implementation of any alternative arrangement pursuant to Section
    3(d)(3) shall be (if not funded out of the Health and Welfare
    Benefit Plan Trust) (A) paid by the Teams, (B) applied against the
    New Benefit Amount provided for by Section 8 below, and (C) limited
    to the portion of the New Benefit Amount, if any, that is available
    for this purpose pursuant to Section 8(b)(7) below. Notwithstanding
    the preceding sentence, this Section 3(e)(2) shall not apply to any
    costs or fees attributable to investment management fees in
    connection with the investment of Health and Welfare Benefit Trust
    assets. Such costs and fees shall: (x) be paid out of the assets of
    the Health and Welfare Benefit Trust; (y) not be applied against the
    New Benefit Amount; and (z) be excluded for purposes of all
    calculations called for under this Agreement of, or relating to,
    Benefits (including, without limitation, for purposes of: (1)
    preparing the Audit Report, Interim Audit Report, or Interim Escrow
    Audit Report, and (2) calculating Total Benefits, Total Salaries and
    Benefits, and Projected Benefits).
  \end{enumerate}
\end{enumerate}

\section{The Post-Career Income
Plan.}\label{the-post-career-income-plan.}

To the extent permitted by the Code and applicable law, the NBA shall
provide the following post-career income benefits to NBA players and
former NBA players in accordance with and subject to the terms and
conditions of the National Basketball Association Players' Qualified
Post-Career Income Plan, as restated effective November 1, 2012, and as
amended from time to time (the ``Qualified Plan'') and the National
Basketball Association Players' Non-Qualified Post-Career Income Plan,
as restated effective February 15, 2015, and as amended from time to
time (the ``Non-Qualified Plan,'' and, when referenced collectively with
the Qualified Plan, the ``Post-Career Income Plan''). (All capitalized
terms used in this Section 4 not otherwise defined in this Agreement
shall have the meanings set forth in the Post-Career Income Plan.)

\begin{enumerate}
\def\labelenumi{(\alph{enumi})}
\tightlist
\item
  \textbf{Current Benefits.}

  \begin{enumerate}
  \def\labelenumii{(\arabic{enumii})}
  \tightlist
  \item
    Effective for the Contribution Year (defined below) commencing
    November 1, 2017, and for each subsequent Contribution Year during
    the term of this Agreement, the Post-Career Income Plan shall
    continue to provide for (i) a Team contribution to the Post-Career
    Income Plan for Eligible Players to be used to purchase Post-Career
    Annuities (the ``Team Contribution'') and (ii) elective Player
    Contributions made by Qualifying Players to the Non-Qualified Plan
    to be used to purchase Post-Career Annuities on such players'
    behalf; provided, however, that the Post-Career Income Plan shall be
    amended, effective as of the 2017-18 Contribution Year, to provide
    that that a player shall not be considered to be on a Roster for a
    Contribution Year solely because he was under a 10-Day Contract or
    Two-Way Contract as of February 2nd of the Regular Season ending
    within such Contribution Year. The Team Contribution for each
    Eligible Player for each Contribution Year shall equal (A) the
    Additional Benefit Amount (defined below) divided by the total
    number of Eligible Players for such Contribution Year (including,
    for this purpose only, any Canadian Resident who but for the fact
    that he is a Canadian Resident would otherwise be an Eligible
    Player) (such quotient, an Eligible Player's ``Allocated Share''),
    less (B) tax withholding (solely with respect to contributions made
    to the Non-Qualified Plan) in the manner described in Section 3.3 of
    the Non-Qualified Plan (``Tax Withholding''). For each Contribution
    Year, a portion of a player's Allocated Share shall be contributed
    to the Qualified Plan on behalf of such player pursuant to the terms
    and conditions described in the Qualified Plan, and a portion to the
    Non-Qualified Plan pursuant to the terms and conditions described in
    the Non-Qualified Plan. For purposes of this Section 4, a
    ``Contribution Year'' means each November 1 through October 31 in
    respect of which a Team Funding Pool (defined below) is provided
    under this Section 4.
  \item
    Notwithstanding anything in this Section 4(a) to the contrary, and
    subject to the requirements of the Code and IRS rules and
    regulations, if the Board of Trustees of the Post-Career Income Plan
    (the ``PCIP Trustees'') determines, after Post-Career Annuities have
    been purchased for Eligible Players for a Contribution Year, that a
    present or former player should have received an Allocated Share for
    such Contribution Year but did not receive an Allocated Share, such
    present or former player shall be entitled to an Allocated Share
    equal to the amount of the Allocated Share made to the other
    Eligible Players for such Contribution Year, which shall be used to
    purchase one or more Post-Career Annuities in the same manner and on
    the same terms as the other Eligible Players for such Contribution
    Year. Unless practicable and otherwise agreed to by the PCIP
    Trustees, the cost of such Allocated Share shall not require a
    retroactive reduction in the Allocated Share and Post-Career
    Annuities of the other Eligible Players for such Contribution Year
    but rather shall be paid from the Additional Benefit Amount for the
    next Season (or, to the extent the Additional Benefit Amount for the
    next Season is insufficient, future Seasons). In addition, the cost
    of any additional fees or expenses charged by the Insurer for the
    purchase of such Post-Career Annuity (or for the purchase of any
    other Post-Career Annuity(ies) under the Plan on a retroactive
    basis) shall also be paid from the Additional Benefit Amount for the
    next Season (or, to the extent the Additional Benefit Amount for the
    next Season is insufficient, future Seasons).
  \end{enumerate}
\item
  \textbf{Deductibility of Team Contributions/Regulatory Changes.}

  \begin{enumerate}
  \def\labelenumii{(\arabic{enumii})}
  \tightlist
  \item
    The Post-Career Income Plan shall be structured and maintained in a
    manner that will result in the Team Funding Pool being fully
    deductible under the Code (and, where applicable, Canadian laws)
    when used toward Team Contributions contributed to the Post-Career
    Income Plan. In the event that a Team is disallowed a deduction (in
    whole or in part) for its portion of the Team Funding Pool, then the
    Team shall be returned such disallowed deduction from the
    Post-Career Income Plan; provided, however, that, if such portion
    may not be returned to the Team under the terms of the Plan or the
    applicable Group Annuity Contract or applicable law, then such Team
    shall instead be reimbursed for the lost tax benefit resulting from
    the disallowance of the deduction from the Additional Benefit Amount
    for the next Season (or, to the extent the Additional Benefit Amount
    for the next Season is insufficient, future Seasons) following the
    date such Team submits satisfactory documentation of the
    disallowance to the PCIP Trustees.
  \item
    Notwithstanding anything else in this Agreement, if any event or
    occurrence, including, without limitation, (i) any change
    oramendment made to the Code, ERISA, or other applicable law, or to
    any regulations (whether final, temporary or proposed regulations),
    or rulings or formal guidance issued thereunder, (ii) any
    interpretation, application or enforcement (or any proposed
    interpretation, application or enforcement), by a court of competent
    jurisdiction in the United States or by the IRS, of the Code, ERISA,
    or other applicable law, or any regulations or rulings issued
    thereunder, (iii) any regulations (whether final, temporary or
    proposed regulations), or rulings or formal guidance issued by the
    IRS under the Code or ERISA or (iv) any provisions of this
    Agreement, including, without limitation, the provisions of this
    Section 4(b), would result in the Teams being disallowed a deduction
    (in whole or in part) for contributions made to the Post-Career
    Income Plan, then any obligation to maintain the Post-Career Income
    Plan pursuant to this Agreement shall, at the option of the NBA,
    terminate; provided, however, that any such termination shall not
    impair the legally binding effect of any other provision of this
    Agreement or the legally binding effect (if any) of any other
    provision of any prior collective bargaining agreement, nor shall it
    create any right (A) to unilaterally implement during the term of
    this Agreement any terms concerning the provision of post-employment
    benefits to the players; (B) to lockout; or (C) to strike.
  \item
    Notwithstanding anything else in this Agreement, if any event or
    occurrence, including, without limitation, (i) any change or
    amendment made to the Code, ERISA, or other applicable law, to any
    regulations (whether final, temporary or proposed regulations), or
    rulings or formal guidance issued thereunder, (ii) any
    interpretation, application or enforcement (or any proposed
    interpretation, application or enforcement), by a court of competent
    jurisdiction in the United States or by the IRS, of the Code, ERISA,
    or other applicable law, or any regulations or rulings issued
    thereunder, (iii) any regulations (whether final, temporary or
    proposed regulations), or rulings or formal guidance issued by the
    IRS under the Code or ERISA or (iv) any provisions of this
    Agreement, including, without limitation, the provisions of this
    Section 4, would result in the Qualified Plan no longer being a
    tax-qualified plan under Section 401(a) of the Code or would require
    NBA Teams to incur costs over and above any costs required to be
    incurred to implement the Qualified Plan in order to maintain its
    tax-qualified status under Section 401(a) of the Code (but only to
    the extent that such additional costs are incurred in connection
    with the provision of benefits to their non-player employees or to
    non-player employees of affiliates (within the meaning of Sections
    414(b), (c) or (m) of the Code) of such Teams), then any obligation
    to maintain and/or make contributions in respect of the Qualified
    Plan pursuant to this Agreement shall terminate; provided, however,
    that any such termination shall not impair the legally binding
    effect of any other provision of this Agreement or the legally
    binding effect (if any) of any other provision of any prior
    collective bargaining agreement, nor shall it create any right (A)
    to unilaterally implement during the term of this Agreement any
    terms concerning the provision of post-employment benefits to the
    players; (B) to lockout; or (C) to strike.
  \item
    If the Taxable Allocated Share attributable to Eligible Players
    would be subject to a federal income tax rate higher than the rate
    that would apply if the Taxable Allocated Share were paid as Base
    Compensation, then any obligation to maintain the Post-Career Income
    Plan pursuant to this Agreement shall, at the option of the Players
    Association, terminate; provided, however, that any such termination
    shall not impair the legally binding effect of any other provision
    of this Agreement or the legally binding effect (if any) of any
    other provision of any prior collective bargaining agreement, nor
    shall it create any right (i) to unilaterally implement during the
    term of this Agreement any terms concerning the provision of
    post-employment benefits to the players; (ii) to lockout; or (iii)
    to strike.
  \item
    In the event of a termination described in Sections 4(b)(2)-(4), the
    NBA and Players Association agree to bargain in good faith with
    respect to an alternative arrangement to be provided by the NBA
    Teams to the players. The annual cost to the Teams of any such
    alternative arrangement (as determined on an after-tax basis) shall
    be substantially equal to but no greater than the annual cost that
    such Teams would have incurred under the Post-Career Income Plan on
    the date of termination. The cost of funding of any such alternative
    arrangement shall be as set forth in Section 4(d)(1). If despite
    good faith negotiations, the NBA and the Players Association fail to
    agree with respect to an alternative arrangement as described above,
    such failure to agree shall not create any right (i) to unilaterally
    implement during the term of this Agreement any terms concerning the
    provision of post-employment benefits to the players; (ii) to
    lockout; or (iii) to strike.
  \end{enumerate}
\item
  \textbf{Players Employed by Toronto.} The terms of the Post-Career
  Income Plan shall continue to permit participation by Toronto Players
  on a tax-effective basis under Canadian income tax laws; provided,
  however, that a player shall not be eligible to participate in the
  Post-Career Income Plan for the period of time during which the player
  is a Canadian Resident but shall instead be eligible to receive a cash
  payment as described in Section 7 below. If the NBA and the Players
  Association should determine that the Post-Career Income Plan cannot
  continue to be provided to Toronto Players on a tax-effective basis
  under Canadian federal income tax laws or that either the Qualified
  Plan or the Non-Qualified Plan would become subject to Ontario's
  Pension Benefits Act, the NBA and Players Association agree to bargain
  in good faith with respect to an alternative arrangement to be
  provided by Toronto to the Toronto Players. The cost of any such
  alternative arrangement to be provided in any Contribution Year shall
  come from such year's Team Funding Pool and shall equal an Eligible
  Player's Allocated Share for such Contribution Year as reduced by all
  federal, state, local, payroll or other tax obligations of any kind
  (including, where applicable, Canadian tax) applicable to such player
  as Toronto, in the exercise of its reasonable discretion, deems
  necessary. If despite good faith negotiations, the NBA and the Players
  Association fail to agree with respect to an alternative arrangement
  as described above, such failure to agree shall not create any right
  (i) to unilaterally implement during the term of this Agreement any
  terms concerning the provision of post-employment benefits to the
  players; (ii) to lockout; or (iii) to strike.
\item
  \textbf{Funding.}

  \begin{enumerate}
  \def\labelenumii{(\arabic{enumii})}
  \tightlist
  \item
    For each Season, except as provided below, one percent (1\%) of BRI
    for such Season (the ``Additional Benefit Amount'') shall be used to
    fund the Team Funding Pool (or the alternative arrangement
    referenced in Section 4(b)(5) and 4(c)); provided, however, that the
    Additional Benefit Amount for a Season shall be subject to reduction
    or elimination pursuant to Article VII, Section 12(b)(1). In no
    event shall the Additional Benefit Amount be used for any purpose
    other than as set forth in the immediately foregoing sentence. For
    purposes of all calculations called for under this Agreement of, or
    relating to, Benefits (including, without limitation, for purposes
    of (i) preparing the Audit Report, Interim Audit Report, or Interim
    Escrow Audit Report, and (ii) calculating Total Benefits, Total
    Salaries and Benefits, and Projected Benefits), the amount to be
    included with respect to the Additional Benefit Amount shall be the
    full Additional Benefit Amount specified in this Section 4(d) and
    not the reduced Additional Benefit Amount provided for under Article
    VII, Section 12(b)(1).
  \item
    For each Contribution Year, all or a portion of the Additional
    Benefit Amount as determined under Section 4(d)(1) (the ``Team
    Funding Pool'') shall be used to fund the Post-Career Income Plan in
    the manner described below. Each Team shall fund its portion of the
    Team Funding Pool for a Contribution Year in an amount equal to the
    Team Funding Pool for such Contribution Year divided by the number
    of all Teams in the NBA as of the beginning of such Contribution
    Year.
  \item
    Each Team shall pay its respective portion of the Team Funding Pool
    to the NBA, and the NBA, as the agent of the Teams, shall remit such
    pool, less Tax Withholding, into the Post-Career Income Plan each
    November following the Contribution Year to which it relates or, if
    later, within one hundred and twenty (120) days following the
    completion of the Audit Report covering the November 1 of such
    Contribution Year.
  \end{enumerate}
\item
  \textbf{Amounts to be Applied Against New Benefit Amount.} All costs,
  including, without limitation, the cost of Professional Fees and other
  administrative services provided to the Post-Career Income Plan, but
  excluding the cost of contributions made to the Post-Career Income
  Plan, incurred on or after the date of this Agreement in connection
  with the operation and administration of the Post-Career Income Plan
  (or any alternative arrangement pursuant to Section 4(b)(5) and 4(c))
  shall be (i) paid by the Teams; (ii) applied against the New Benefit
  Amount provided for in Section 8 below; and (iii) limited to the
  portion of the New Benefit Amount, if any, that is available for this
  purpose pursuant to Section 8(b)(6) below.
\end{enumerate}

\section{Labor-Management Cooperation and Education
Trust.}\label{labor-management-cooperation-and-education-trust.}

\begin{enumerate}
\def\labelenumi{(\alph{enumi})}
\item
  Except as set forth below in this Section 5, as of the effective date
  of this Agreement, and continuing until the expiration or termination
  of this Agreement, the National Basketball Players
  Association/National Basketball Association Labor-Management
  Cooperation and Education Trust (the ``Education Trust'') shall
  continue to be jointly operated and administered by the NBA and the
  Players Association in accordance with the provisions of the Agreement
  and Declaration of Trust Establishing the National Basketball Players
  Association/National Basketball Association Labor-Management
  Cooperation and Education Trust (the ``Education Trust Agreement'').
  It is intended by the NBA and the Players Association that, at all
  times, the Education Trust shall comply with the provisions of Section
  302(c)(9) of the Labor Management Relations Act of 1947, as amended,
  and shall qualify as an exempt organization under the provisions of
  Section 501(c)(5) or 501(c)(3) of the Code.
\item
  The Education Trust shall continue to be operated and administered for
  the purpose of establishing and providing (i) health education
  programs and (ii) education and career counseling programs designed to
  assist the NBA, NBA Teams and NBA players in solving problems of
  mutual concern not susceptible to resolution within the collective
  bargaining process and to enhance the involvement of NBA players in
  making decisions that affect their working lives. The NBA and the
  Players Association agree to meet and confer to discuss the Education
  Trust establishing a modified financial education program, subject to
  the modified financial education program being structured to qualify
  as a permitted activity of an exempt organization under the provisions
  of Section 501(c)(5) of the Code; provided, however, that neither
  party shall have any obligation to establish a modified financial
  education program. Notwithstanding the preceding sentence, in the
  event that an agreed-upon modified financial education program cannot
  be structured to qualify as a permitted activity of an exempt
  organization under the provisions of Section 501(c)(5) of the Code,
  the NBA and Players Association agree to meet and confer regarding the
  establishment of such program through a different vehicle than the
  Education Trust. The specific benefits to be provided under any
  financial education program shall be (A) subject to the agreement of
  the NBA and the Players Association and (B) governed by a definitive
  written arrangement implementing such program.
\item
  \begin{enumerate}
  \def\labelenumii{(\arabic{enumii})}
  \tightlist
  \item
    Except as provided in Section 5(c)(2) below, the costs of funding
    the Education Trust and the costs attributable to the operation and
    administration of the Education Trust, including, without
    limitation, the cost of Professional Fees incurred in connection
    with the administration of the Education Trust, shall be (i) paid by
    the Teams; (ii) applied against the New Benefit Amount provided for
    by Section 8 below; and (iii) limited to the portion of the New
    Benefit Amount, if any, that is available for this purpose pursuant
    to Section 8(b)(5) below. Payment of the amount necessary to fund
    the Education Trust in respect of each Salary Cap Year shall be made
    within thirty (30) days following the completion of the Audit Report
    for such Salary Cap Year. The NBA and Players Association agree
    that, subject to the limitations set forth in this Section 5, the
    amount to be paid by the Teams to fund the education and career
    counseling programs to be operated and administered by the Education
    Trust for the 2017-18 Salary Cap Year shall be no greater than
    \$1,507,608 and such maximum funding amount shall be increased by
    five percent (5\%) for each subsequent Salary Cap Year.
  \item
    The costs of funding the health education programs to be operated
    and administered by the Education Trust (or any programs that,
    pursuant to Section 5(f) below, are substituted for the health
    education programs) shall be paid by the Teams. Payment of the
    amount necessary to fund such programs in respect of each Salary Cap
    Year shall be made within thirty (30) days following the completion
    of the Audit Report for such Salary Cap Year. The NBA and Players
    Association agree that, subject to the limitations set forth in this
    Section 5, the amount to be paid by the Teams to fund the health
    education programs (or any programs that, pursuant to Section 5(f)
    below, are substituted for the health education programs) to be
    operated and administered by the Education Trust for the 2017-18
    Salary Cap Year shall be no greater than \$469,033 and such maximum
    funding amount shall be increased by five percent (5\%) for each
    subsequent Salary Cap Year.
  \end{enumerate}
\item
  The Education Trust shall be operated and administered in a manner
  that will result in all contributions by the Teams being fully
  deductible under the Code (and, where applicable, Canadian income tax
  laws) when paid. If any Team is disallowed a deduction (in whole or in
  part) for such contributions, and unless the NBA determines otherwise,
  the obligation to maintain the Education Trust and to make further
  contributions to the Education Trust shall immediately terminate;
  provided, however, that any such termination shall not impair the
  legally binding effect of any other provision of this Agreement, and
  shall not create any right (1) to unilaterally implement, during the
  term of this Agreement, any terms concerning the provision of
  education programs provided or to be provided by the Education Trust;
  (2) to lockout; or (3) to strike.
\item
  In the event of any termination pursuant to Section 5(d) above, the
  NBA and Players Association agree to bargain in good faith with
  respect to an alternative arrangement designed to provide the programs
  described in the Education Trust Agreement. Such alternative
  arrangement shall, to the extent permitted by applicable law, be
  funded by such monies as may then remain in the Education Trust and,
  if the monies remaining in the Education Trust may not lawfully be
  used for, or are insufficient for, such purpose, such alternative
  arrangement shall be funded, by the NBA Teams; provided, however, that
  the annual cost incurred by the Teams in connection with such
  alternative arrangement (as determined on an after-tax basis) shall
  not exceed the annual cost that such Teams would have incurred to fund
  the Education Trust commencing on the date of termination. Any such
  alternative arrangement shall be operated and administered in a manner
  that will result in all contributions by the Teams being fully
  deductible under the Code (and, where applicable, Canadian income tax
  laws) when paid; and, if funded by the Teams (and not out of existing
  monies remaining in the Education Trust), the costs of funding any
  alternative to the Education Trust shall be applied against the New
  Benefit Amount provided for by Section 8 below, shall be limited to
  the portion of the New Benefit Amount, if any, that is available for
  this purpose pursuant to Section 8(b)(5) below, and shall be subject
  to the limitations set forth in this Agreement. If despite good faith
  negotiations, the NBA and the Players Association fail to agree with
  respect to an alternative arrangement as described above, such failure
  to agree shall not create any right (1) to unilaterally implement,
  during the term of this Agreement, any terms concerning the provision
  of programs provided or to be provided by the Education Trust; (2) to
  lockout; or (3) to strike.
\item
  Upon written notice delivered to the NBA at least six (6) months prior
  to the commencement of any Salary Cap Year, the Players Association
  may elect to terminate the programs currently provided by the
  Education Trust and substitute alternative programs; provided,
  however, that the NBA consents to such substitution, which such
  consent shall not be unreasonably withheld; and provided, further,
  that any new programs shall comply with the provisions of Section
  302(c)(9) of the Labor Management Relations Act of 1947, as amended,
  and shall qualify as a permitted activity of an exempt organization
  under Section 501(c)(5) of the Code.
\end{enumerate}

\section{Additional Player Benefits}\label{additional-player-benefits}

Except as set forth below, the NBA shall provide the following
additional benefits:

\begin{enumerate}
\def\labelenumi{(\alph{enumi})}
\item
  Workers' compensation benefits in accordance with applicable statutes.
  Such benefits will be provided for players and Two-Way Players.
\item
  Funding for the annual Players Association High School Basketball Camp
  (or any substitute program mutually agreed upon by the parties) in the
  amount of \$1,034,012 for the 2017-18 Season, increasing by seven and
  one-half percent (7.5\%) per Season thereafter for the term of this
  Agreement.
\item
  Player Playoff Pool in the amount of \$20 million for the 2017-18
  Season, changing by a percentage in each Season thereafter for the
  term of this Agreement, which percentage shall be calculated by
  dividing: (1) the amount obtained by subtracting BRI for the
  immediately preceding Season from BRI for the then-current Season; by
  (2) BRI for the immediately preceding Season. If the NBA increases the
  number of Teams participating in the playoffs, the Player Playoff Pool
  shall be increased by \$615,000 for each Team added. The NBA will
  consult with the Players Association with respect to the method of
  allocation of the Player Playoff Pool.
\item
  The employer's portion of payroll taxes.
\item
  The Players Association's one-half share of the payment of fees and
  expenses to the Accountants (as defined in Article VII, Section 10(a)
  below) in connection with any audit conducted under this Agreement,
  and the Players Association's one-half share of the payment of fees
  and expenses payable with respect to the TV Expert (as defined in
  Article VII, Section 1(a)(7)(ii) below) and any expert selected in
  accordance with Article VII, Section 1(a)(7)(i).
\item
  The Players Association's share of the costs of the Anti-Drug Program
  as provided for by Article XXXIII.
\item
  \begin{enumerate}
  \def\labelenumii{(\arabic{enumii})}
  \tightlist
  \item
    The sum of the Compensation paid to each player with three (3) or
    more Years of Service who signs a one-year, 10-Day or Rest-of-Season
    Contract for the Minimum Player Salary during a Season, less, for
    each such player, the Minimum Player Salary for a player with two
    (2) Years of Service.
  \item
    The Compensation paid to any player with three (3) or more Years of
    Service who signs a one-year, 10-Day or Rest-of-Season Contract for
    the Minimum Player Salary in excess of the Minimum Player Salary for
    a player with two (2) Years of Service shall be paid by the player's
    Team pursuant to the terms of such player's Uniform Player Contract,
    and then reimbursed to the Team out of a League-wide fund created
    and maintained by the NBA. Such reimbursement shall be made at the
    conclusion of the Season covered by the Contract.
  \item
    The sum of the Compensation paid to players under Rookie Scale
    Contracts in respect of the Rookie Scale Conforming Increases (as
    defined in Article VIII, Section 5(a)) in accordance with the
    provisions of Article VIII, Section 5.
  \item
    Rookie Scale Conforming Increases required to be paid to any player
    pursuant to Article VIII, Section 5 shall be paid by the applicable
    player's Team in accordance with the payment schedule set forth in
    the player's Uniform Player Contract, and then reimbursed to the
    Team out of a League-wide fund created and maintained by the NBA.
    Such reimbursement shall be made at the conclusion of the Season
    covered by the Contract.
  \end{enumerate}
\item
  One-half of the annual funding of \$1 million for the NBA Players
  Legacy Fund that is provided jointly by the NBA and the Players
  Association.
\item
  Any additional contributions that may be required to be made to the
  Pension Plan because of any new law, change or amendment made to
  ERISA, the Code and/or any other applicable law or to any regulations
  (whether final, temporary or proposed), rulings or formal guidance
  issued thereunder that is effective for a Plan Year that first begins
  after the effective date of this Agreement.
\item
  Costs of player attendance at the partner forums as set forth in the
  following sentence. For the purposes of enhancing career exposure and
  professional development, the NBA agrees to permit current and former
  players to attend partner forums held from time-to-time with NBA
  business partners, subject to advance notice by the players and there
  being a reasonable number of player attendees such that the primary
  purpose of the forums (i.e., to facilitate interaction between the NBA
  and business partners) will be maintained. To the extent reasonably
  practicable, the NBA agrees to provide the Players Association with
  advance notice of partner forums that it is aware of.
\item
  The Players Association's one-half share of the costs of: (1) the
  Fitness to Play Panels as provided for by Article XXII, Section 11;
  (2) the player care survey as provided for by Article XXII, Section
  12; and (3) the Wearables Committee, including, without limitation,
  the costs of retaining experts, as provided for by Article XXII,
  Section 13.
\item
  Costs described in Sections 1(f), 2(g)(2), 3(e)(2), 4(e), and/or
  5(c)(1) above to the extent such costs are not paid as New Benefit
  Amounts pursuant to the provisions of Section 8 below due to the
  unavailability of any New Benefit Amount for payment of such costs
  (provided that as to Section 5(c)(1), solely those costs attributable
  to the operation and administration of the Education Trust, including,
  without limitation, Professional Fees, shall be included in this
  Section 6(l)).
\item
  The benefits funded by the New Benefit Amount set forth in Section 8
  below.
\end{enumerate}

\section{Canadian Residents.}\label{canadian-residents.}

As of the effective date of this Agreement, and continuing until the
expiration or termination of this Agreement, Toronto shall provide the
following benefits to Canadian Residents:

\begin{enumerate}
\def\labelenumi{(\alph{enumi})}
\tightlist
\item
  \textbf{Definitions.} All capitalized terms used in this Section 7 not
  otherwise defined in this Agreement shall have the meanings set forth
  below:

  \begin{enumerate}
  \def\labelenumii{(\arabic{enumii})}
  \tightlist
  \item
    ``Eligible Canadian Resident'' shall mean a Canadian Resident who
    would be eligible to participate in the Pension Plan, the
    Post-Career Income Plan, the HRA Benefit and/or the 401(k) Plan, in
    each case, but for the fact that he is a Canadian Resident.
  \item
    ``EHT'' shall mean the Ontario Employer Health Tax.
  \item
    ``Gross Amount'' for a Season shall mean, as applicable, the sum of:

    \begin{enumerate}
    \def\labelenumiii{(\roman{enumiii})}
    \tightlist
    \item
      if the player is an Eligible Canadian Resident in respect of the
      Pension Plan, the annual accrual cost that Toronto would have
      incurred under the Pension Plan for such Eligible Canadian
      Resident for such Season but for the fact that he was Canadian
      Resident; and
    \item
      if the player is an Eligible Canadian Resident in respect of the
      Post-Career Income Plan, the amount of the per-player Allocated
      Share for such Season; and
    \item
      if the player is an Eligible Canadian Resident in respect of the
      HRA Benefit, the amount of the contribution to fund the HRA
      Benefit for such Season that such player would be entitled to
      under Section 3(a)(1) but for the fact that he is a Canadian Tax
      Resident; provided that, for the avoidance of doubt, the Gross
      Amount(s) previously allocated to such player in lieu of the HRA
      Benefit for years in which he was an Eligible Canadian Resident in
      respect of the HRA Benefit shall be applied against the dollar
      limitations in Section 3(a)(1); and
    \item
      if the player is an Eligible Canadian Resident in respect of the
      401(k) Plan, the amount of the Matching Contribution (as defined
      in the 401(k) Plan) for such season assuming that the Eligible
      Canadian Resident had made the maximum player deferral permitted
      under the 401(k) Plan for such Season.
    \end{enumerate}
  \item
    ``Adjusted Gross Amount'' shall mean the adjusted gross amount that
    is equal to the Eligible Canadian Resident's Gross Amount less the
    amount of EHT on such adjusted gross amount.
  \end{enumerate}
\item
  \textbf{Cash payment.} For each Season during the term of this
  Agreement, each Eligible Canadian Resident shall be entitled to a
  single sum payment subject to the following terms and conditions:

  \begin{enumerate}
  \def\labelenumii{(\arabic{enumii})}
  \tightlist
  \item
    The amount of the payment shall equal the Eligible Canadian
    Resident's Adjusted Gross Amount in respect of such Season, less all
    amounts required to be withheld by any governmental authority, and
    less the employer's share of payroll taxes for the Eligible Canadian
    Resident (the ``Cash Payment'').
  \item
    The Cash Payment shall be paid in Canadian dollars to the Eligible
    Canadian Resident by no later than the December 31 immediately
    following the end of the Season to which the payment relates. For
    purposes of calculating the Cash Payment, the Adjusted Gross Amount
    shall be calculated in U.S. dollars and then converted to Canadian
    dollars using the nominal noon exchange rate quoted by the Bank of
    Canada for converting U.S. dollars into Canadian dollars on the
    first day of the month in which the Cash Payment is made, or if
    there is no such U.S. dollar to Canadian dollar exchange rate quoted
    for that date, the closest preceding date on which such exchange
    rate is quoted by the Bank of Canada.
  \item
    The Cash Payment shall not be considered part of an Eligible
    Canadian Resident's Escrow Amount under Article VII of this
    Agreement.
  \end{enumerate}
\item
  \textbf{Funding of Gross Amount.} All costs, including, without
  limitation, the Gross Amount and the cost of Professional Fees,
  incurred in connection with the determination and implementation of
  this Section 7: (i) that are attributable to the Pension Plan, the
  401(k) Plan and the HRA Benefit shall be applied against the New
  Benefit Amount provided for by Section 8 below and limited to the
  portion that is available for this purpose pursuant to Sections
  8(b)(1), (2) and (7) respectively and (ii) that are attributable to
  the Post-Career Income Plan shall be funded from the Team Funding
  Pool.
\end{enumerate}

\section{New Benefits Funding.}\label{new-benefits-funding.}

\begin{enumerate}
\def\labelenumi{(\alph{enumi})}
\item
  \begin{enumerate}
  \def\labelenumii{(\arabic{enumii})}
  \tightlist
  \item
    For each Salary Cap Year during the term of this Agreement, an
    aggregate amount (the ``New Benefit Amount'') equal to \$1.1 million
    multiplied by the number of Teams in the NBA during the Season that
    is covered by such Salary Cap Year shall be provided by the Teams to
    fund the benefits described in Section 8(b) below, unless the
    Players Association designates a lesser amount with respect to a
    Salary Cap Year, by notice in writing to the NBA delivered on or
    before the March 15 prior to the commencement of such Salary Cap
    Year.
  \item
    Notwithstanding subsection (a)(1) above, the New Benefit Amount for
    each Salary Cap Year shall be subject to reduction pursuant to
    Article VII, Section 12(b)(1). For purposes of all calculations
    called for under this Agreement of, or relating to Benefits
    (including, without limitation, for purposes of (i) preparing the
    Audit Report, Interim Audit Report, or Interim Escrow Audit Report,
    and (ii) calculating Total Benefits, Total Salaries and Benefits,
    and Projected Benefits), the amount to be included with respect to
    the New Benefit Amount shall be the full New Benefit Amount
    specified in Section 8(a)(1) above (less the amount excluded from
    such calculations, if any, as provided for under Section 8(b)(8))
    and not the reduced New Benefit Amount provided for under Article
    VII, Section 12(b)(1).
  \end{enumerate}
\item
  Subject to Section 8(c) below, the New Benefit Amount, after taking
  into account the reduction provided for in Section 8(a)(2) above,
  shall be utilized in the following manner for each Salary Cap Year:

  \begin{enumerate}
  \def\labelenumii{(\arabic{enumii})}
  \tightlist
  \item
    The New Benefit Amount shall first be utilized, to the extent
    necessary, to fund any pension contribution increases and costs
    described in Section 1(f) above. If the New Benefit Amount is
    insufficient for these purposes, the shortfall, over as short a
    period of time as is reasonably possible, shall be offset against:
    (i) the New Benefit Amount for the next Salary Cap Year; and/or (ii)
    the NBA's obligation to provide Benefits (other than Benefits funded
    via the New Benefit Amount) under this Article IV. The determination
    of the allocation of and type(s) of offset(s) to be applied (as
    between (i) and/or (ii) above) shall be made by the Players
    Association, subject to the NBA's consent, which shall not be
    unreasonably withheld.
  \item
    Subject to the provisions of Section 2 above, and after taking into
    account the expenditure described in Section 8(b)(1) above, the
    remainder of the New Benefit Amount, if any, shall be utilized, to
    the extent necessary, to fund the cost of Matching Contributions
    with respect to the 401(k) Plan (and, if applicable, to fund the
    cost of any alternative arrangement described in Sections 2(e) and
    (f) above) and to pay the costs described in Section 2(g)(2) above.
  \item
    After taking into account the expenditures described in Section
    8(b)(1) and (2) above, the remainder of the New Benefit Amount, if
    any, shall be utilized, to the extent necessary, to fund \$582,000
    of the cost of the life insurance and accidental death and
    dismemberment benefits described in Section 3(a)(2)(i) above.
  \item
    After taking into account the expenditures described in Section
    8(b)(1) -- (3) above, the remainder of the New Benefit Amount, if
    any, shall be utilized, to the extent necessary, to fund any
    incremental cost of changes in the medical and dental benefits made
    pursuant to a Regulatory Change in accordance with the provisions of
    Section 3(d) above or any change made in accordance with the
    provisions of Section 3(a)(5) above.
  \item
    After taking into account the expenditures described in Section
    8(b)(1) -- (4) above, the remainder of the New Benefit Amount, if
    any, shall be utilized, to the extent necessary, to fund the
    education and career counseling programs to be operated and
    administered by the Education Trust (or any programs that, pursuant
    to Section 5(f) above, are substituted for such education and career
    counseling programs) described in Section 5 above and to pay the
    costs described in Section 5(c) above.
  \item
    After taking into account the expenditures described in Section
    8(b)(1) -- (5) above, the remainder of the New Benefit Amount, if
    any, shall be utilized to pay the costs of the Post-Career Income
    Plan in accordance with the provisions of Section 4(e) above.
  \item
    After taking into account the expenditures described in Section
    8(b)(1) -- (6) above, the remainder of the New Benefit Amount, if
    any, shall be utilized to pay the costs of the HRA Benefit (and, if
    applicable, to pay the costs of any alternative arrangement with
    respect to the HRA Benefit as described in Sections 3(c) and 3(d)(3)
    above), in accordance with the provisions of Section 3 above and to
    pay the costs described in Section 3(e) above.
  \item
    After taking into account the expenditures described in Section
    8(b)(1) -- (7) above, the remainder of the New Benefit Amount, if
    any, shall be utilized to fund any additional player benefits as may
    be agreed to by the NBA and the Players Association by the March 15
    falling within such Salary Cap Year. Upon either the NBA's or the
    Players Association's request, the NBA and the Players Association
    agree to bargain in good faith with respect to additional player
    benefits; provided, that if despite good faith negotiations, the NBA
    and the Players Association fail to agree with respect to such
    additional player benefits, such failure shall not create any right
    (i) to unilaterally implement during the term of this Agreement any
    terms concerning the provision of such additional player benefits;
    (ii) to lockout; or (iii) to strike. In the event the NBA and the
    Players Association do not agree to use the remainder of the New
    Benefit Amount, if any, to fund additional player benefits by the
    March 15 falling within such Salary Cap Year, then the remainder of
    the New Benefit Amount, if any, shall be retained by the Teams, and
    the amount of such remainder shall be excluded for purposes of all
    calculations called for under this Agreement of, or relating to,
    Benefits (including, without limitation, for purposes of (A)
    preparing the Audit Report, Interim Audit Report, or Interim Escrow
    Audit Report, and (B) calculating Total Benefits, Total Salaries and
    Benefits, and Projected Benefits).
  \end{enumerate}
\item
  Notwithstanding anything to the contrary in this Article IV:

  \begin{enumerate}
  \def\labelenumii{(\arabic{enumii})}
  \tightlist
  \item
    In no event shall the Teams (or the NBA) pay amounts for any Salary
    Cap Year with respect to the benefits described in Section 8(b)
    above in excess of the New Benefit Amount for such Salary Cap Year.
  \item
    Until the final Audit Report (or, if applicable, the Interim Escrow
    Audit Report) for a Salary Cap Year is completed, the NBA shall not
    be required to spend, or commit to spend, any portion of the New
    Benefit Amount for such Salary Cap Year.
  \end{enumerate}
\end{enumerate}

\section{Projected Benefits.}\label{projected-benefits.}

\begin{enumerate}
\def\labelenumi{(\alph{enumi})}
\tightlist
\item
  For purposes of computing the Tax Level, Salary Cap and Minimum Team
  Salary in accordance with Article VII, ``Projected Benefits'' shall
  mean the projected amounts, as estimated by the NBA in good faith, to
  be paid or accrued by the NBA or the Teams, other than Expansion Teams
  during their first two Salary Cap Years, for the upcoming Salary Cap
  Year with respect to the benefits to be provided for such Salary Cap
  Year. In the event that the amount of any benefit for the upcoming
  Salary Cap Year is not reasonably calculable, then, for purposes of
  computing Projected Benefits, such amount shall be projected to be one
  hundred four and one-half percent (104.5\%) of the amount attributable
  to the same benefit for the prior Salary Cap Year.
\item
  For purposes of computing Projected Benefits, (i) the amount to be
  included with respect to players with three (3) or more Years of
  Service who receive the Minimum Player Salary shall be the same amount
  included in Benefits with respect to such players for the immediately
  preceding Season; and (ii) no amounts paid in respect of Rookie Scale
  Conforming Increases shall be included.
\item
  For purposes of computing Projected Benefits with respect to a Salary
  Cap Year, the amount to be included with respect to the Additional
  Benefit Amount shall be one percent (1\%) of Projected BRI for such
  Salary Cap Year.
\item
  For purposes of computing Projected Benefits with respect to a Salary
  Cap Year, there shall be taken into account any reduction in the New
  Benefit Amount with respect to such Salary Cap Year as designated by
  the Players Association, by notice in writing to the NBA delivered on
  or before the March 15 immediately preceding the commencement of such
  Salary Cap Year.
\end{enumerate}

\section{Benefit Exclusion Amount.}\label{benefit-exclusion-amount.}

\begin{enumerate}
\def\labelenumi{(\alph{enumi})}
\item
  An amount equal to the Benefit Exclusion Amount (defined below) shall
  be (i) paid by the Teams and (ii) excluded for purposes of all
  calculations called for under this Agreement of, or relating to,
  Benefits (including, without limitation, for purposes of: (A)
  preparing the Audit Report, Interim Audit Report, or Interim Escrow
  Audit Report, and (B) calculating Total Benefits, Total Salaries and
  Benefits, and Projected Benefits).
\item
  The ``Benefit Exclusion Amount,'' for each Salary Cap Year, shall mean
  the sum of:

  \begin{enumerate}
  \def\labelenumii{(\arabic{enumii})}
  \item
    The ``Pension Exclusion Amount,'' which shall equal fifty percent
    (50\%) of the portion of the increase in the amount of the
    actuarially-determined annual contributions to be made to the
    Pension Plan to fund the portion of the liabilities for the 2017-18
    Benefit Increase (defined below) that is attributable to the Current
    Retiree Group (defined below), as determined by the actuaries of the
    Pension Plan. The ``2017-18 Benefit Increase'' means the increase in
    the Monthly Benefit from \$572.13 to \$812.50; and
  \item
    Fifty percent (50\%) of the portion of the increase in the amount of
    the actuarially-determined annual contributions to be made to the
    Pension Plan, and fifty percent (50\%) of the portion of the
    increase in the cost under the Pre-1965 Players' Excess Benefit
    Plan, to fund the 2017-18 Pre-1965 Benefit Increase; and
  \item
    \begin{enumerate}
    \def\labelenumiii{(\roman{enumiii})}
    \tightlist
    \item
      Fifty percent (50\%) of the portion of the costs (including,
      without limitation, the cost of Professional Fees) that were
      approved by both an NBA designee and a Players Association
      designee as having been properly incurred in connection with the
      operation and administration of the Retiree Medical Plan
      (``Administrative Costs''), but only to the extent that such costs
      are attributable to the Current Retiree Group. The portion of the
      Administrative Costs for a Salary Cap Year that is attributable to
      the Current Retiree Group shall be determined by multiplying the
      total Administrative Costs for the Salary Cap Year by the
      ``Allocation Percentage'' (defined below) for such Salary Cap
      Year. The ``Allocation Percentage,'' for a Salary Cap Year, means
      the fraction, when expressed as a percentage, the numerator of
      which is the number of players in the Current Retiree Group who
      are enrolled in the Retiree Medical Plan on the day that is sixty
      (60) days prior to the last day of such Salary Cap Year, and the
      denominator of which is the total number of players who are
      enrolled in the Retiree Medical Plan on such date.
    \item
      Fifty percent (50\%) of the portion of the premium costs paid to
      the insurer of the Retiree Medical Plan (excluding the participant
      share of premium contributions) that is attributable to the
      Current Retiree Group. Such portion of the premium costs shall be
      calculated based on the schedules provided by the insurer of the
      Retiree Medical Plan that set forth the monthly premium payments
      for each eligible retiree or any eligible dependent based on the
      applicable coverage level elected.
    \item
      The calculations required by this Section 10(b)(3) shall be
      performed by the actuarial/consulting firm employed by the Pension
      Plan, whose calculations shall be binding and conclusive; and
    \end{enumerate}
  \item
    Fifty percent (50\%) of the portion of reimbursable tuition
    reimbursement and career transition benefits for players under the
    Health and Welfare Benefit Plan (as described in Section 3(a)(7)
    above) that is attributable to the Current Retiree Group; and
  \item
    The sum of the Two-Way 401(k) Exclusion Amounts (defined below) for
    all Two-Way Players who are Eligible Players under the 401(k) Plan.
    The ``Two-Way 401(k) Exclusion Amount'' for each such player is the
    amount equal to the lesser of: (A) twenty-five percent (25\%) of
    such player's Salary Deferral Contributions attributable to his
    Two-Way NBADL Salary only, and (B) one percent (1\%) of such
    player's Two-Way NBADL Salary only; and
  \item
    The portion of the premium costs paid to the applicable insurer(s)
    (excluding the employee share of premium contributions) to provide
    the medical, prescription drug, dental and life insurance and
    accidental death and dismemberment insurance benefits to Two-Way
    Players as described in Section 3(a)(2)(i) and 3(a)(2)(iii) above;
    and
  \item
    The premium costs paid to the insurer to provide the vision benefits
    to Two-Way Players as described in Section 3(a)(2)(v) above; and
  \item
    The amount equal to: (A) the premium costs under the workers'
    compensation policy covering NBADL players in the Salary Cap Year,
    divided by the average number per month of participants covered
    under such policy during the NBADL regular season occurring within
    such Salary Cap Year; multiplied by (B) the average number per month
    of Two-Way Players (excluding, for any month's calculation, Two-Way
    Players who were signed or converted to Standard NBA Contracts in
    that or a prior month) during the Regular Season; and
  \item
    The sum of the Two-Way Payroll Tax Exclusion Amounts (defined below)
    for all Two-Way Players. The ``Two-Way Payroll Tax Exclusion
    Amount'' for each Two-Way Player equals the employer's share of
    payroll taxes in respect of his Two-Way NBADL Salary only.
  \end{enumerate}
\item
  \begin{enumerate}
  \def\labelenumii{(\arabic{enumii})}
  \tightlist
  \item
    The ``Current Retiree Group'' shall mean those former players whose
    last day on an NBA Active List or Inactive List during a Regular
    Season occurred before the 2016-17 Season.
  \item
    The ``Current Player Group'' shall mean those players whose last day
    on an NBA Active List or Inactive List during a Regular Season will
    occur during or after the 2016-17 Season.
  \end{enumerate}
\item
  If a player who is included in the Current Retiree Group for one or
  more Salary Cap Years returns to an NBA Active List or Inactive List
  and thereby moves to the Current Player Group in a later Salary Cap
  Year, the Benefit Exclusion Amount for the Salary Cap Year during
  which he returns to an NBA Active List or Inactive List shall be
  reduced by the amount of the portion of the Benefit Exclusion Amount
  for the prior Salary Cap Year(s) that is attributable to such player.
\item
  If the NBA and Players Association provide an alternative arrangement
  to any benefit referenced in Sections 10(b)(1) through 10(b)(7) above,
  the amount to be included in the calculation of the Benefit Exclusion
  Amount with regard to that alternative arrangement shall not exceed
  the amount referenced in the applicable part of Section 10(b) with
  regard to the benefit being replaced by that alternative arrangement
  for the most recent Salary Cap Year before such benefit was replaced.
\item
  For the avoidance of doubt, other than the Benefit Exclusion Amount,
  all amounts paid or to be paid during any Salary Cap Year by the NBA
  or the NBA Teams for the benefits described in this Article IV shall
  be included for purposes of all calculations called for under this
  Agreement of, or relating to, Benefits (including, without limitation,
  for purposes of (1) preparing the Audit Report, Interim Audit Report,
  or Interim Escrow Audit Report, and (2) calculating Total Benefits,
  Total Salaries and Benefits, and Projected Benefits).
\end{enumerate}

\chapter{COMPENSATION AND EXPENSES IN CONNECTION WITH MILITARY
DUTY}\label{compensation-and-expenses-in-connection-with-military-duty}

\section{Salary.}\label{salary.}

A player drafted into military service during the Season, or a player
serving on active duty with a reserve unit during the Season, shall be
compensated for so long as the player remains on the Active or Inactive
List of the Team in such amount as may be negotiated between the player
and the Team by which he is employed, subject to the provisions of this
Agreement.

\section{Travel Expenses.}\label{travel-expenses.}

\begin{enumerate}
\def\labelenumi{(\alph{enumi})}
\tightlist
\item
  A player serving on military weekend duty with a reserve unit during
  the Season shall be entitled to reimbursement for any net
  out-of-pocket expenses incurred by such player in traveling to and
  from his place of duty to enable him to join his Team for purposes of
  participating in a Regular Season game.
\item
  In the event that the Player Contract of a player who is required to
  serve on military weekend duty with a reserve unit is assigned to
  another Team, the player shall be entitled to reimbursement for any
  out-of-pocket expenses incurred by such player in traveling during the
  off-season to and from his home and his place of military weekend duty
  with a reserve unit; provided that (i) the player makes reasonable
  efforts to change his reserve unit location to one located reasonably
  close to his home, and (ii) such obligation to reimburse the player
  shall cease six (6) months from the date that such player's Contract
  is assigned.
\end{enumerate}

\chapter{PLAYER CONDUCT}\label{player-conduct}

\section{General.}\label{general.-1}

In addition to any other rights a Team or the NBA may have by contract
(including but not limited to the rights set forth in paragraphs 9 and
16 of the Uniform Player Contract) or by law, when a player fails or
refuses, without proper and reasonable cause or excuse, to render the
services required by a Player Contract or this Agreement, or when a
player is, for proper cause, suspended by his Team or the NBA in
accordance with the terms of such Contract or this Agreement, the
Current Base Compensation payable to the player for the year of the
Contract during which such refusal or failure and/or suspension occurs
may be reduced (or, in the case of a suspension, shall be reduced) by
(a) 1/145th of the player's Base Compensation for each missed
Exhibition, Regular Season or Playoff game for any suspension of less
than twenty (20) games and (b) 1/110th of the player's Base Compensation
for each missed Exhibition, Regular Season or Playoff game for any
suspension of twenty (20) games or more (including any indefinite
suspension that persists for twenty (20) games or more or consecutive
suspensions for continuing acts or conduct that persist for twenty (20)
games or more).

\section{Practices.}\label{practices.}

\begin{enumerate}
\def\labelenumi{(\alph{enumi})}
\tightlist
\item
  When a player, without proper and reasonable excuse, fails to attend a
  practice session scheduled by his Team, he shall be subject to the
  following discipline: (i) for the first missed practice during a
  Season -- \$2,500; (ii) for the second missed practice during such
  Season -- \$5,000; (iii) for the third missed practice during such
  Season -- \$7,500; and (iv) for the fourth (or any additional) missed
  practice during such Season -- such discipline as is reasonable under
  the circumstances.
\item
  Notwithstanding Section 2(a) above, when a player, without proper and
  reasonable excuse, refuses or intentionally fails to attend any
  practice session scheduled by his Team, he shall be subject to such
  discipline as is reasonable under the circumstances.
\end{enumerate}

\section{Promotional Appearances.}\label{promotional-appearances.}

When a player, without proper and reasonable excuse, fails or refuses to
attend a promotional appearance required by and in accordance with
Article II, Section 8 and Paragraph 13(d) of the Uniform Player
Contract, he shall be fined \$20,000.

\section{Mandatory Programs.}\label{mandatory-programs.}

\begin{enumerate}
\def\labelenumi{(\alph{enumi})}
\tightlist
\item
  NBA players shall be required to attend and participate in educational
  and life skills programs designated as ``mandatory programs'' by the
  NBA and the Players Association. Such ``mandatory programs,'' which
  shall be jointly administered by the NBA and the Players Association,
  shall include a Rookie Transition Program (for rookies only), Team
  Awareness Meetings (which shall cover, among other things, substance
  abuse awareness, HIV awareness, and gambling awareness), and such
  other programs as the NBA and the Players Association shall jointly
  designate as mandatory.
\item
  When a player, without proper and reasonable excuse, fails or refuses
  to attend a ``mandatory program,'' he shall be fined \$20,000 by the
  NBA; provided, however, that if the player misses the Rookie
  Transition Program, he shall be suspended for five (5) games.
\end{enumerate}

\section{Media Training, Business of Basketball and Anti-Gambling
Training.}\label{media-training-business-of-basketball-and-anti-gambling-training.}

\begin{enumerate}
\def\labelenumi{(\alph{enumi})}
\tightlist
\item
  All players shall be required each Season to attend and participate in
  one (1) media training session conducted by their Team and/or the NBA.
  If a player, without proper and reasonable excuse, fails or refuses to
  attend a media training session, he shall be fined \$20,000.
\item
  All players shall be required to attend and participate each Season in
  one (1) ``business of basketball'' program conducted by their Team
  and/or the NBA. If a player, without proper and reasonable excuse,
  fails or refuses to attend such program, he shall be fined \$5,000.
\item
  All players shall be required each Season to attend and participate in
  one (1) anti-gambling training session conducted by their Team and/or
  the NBA. If a player, without proper and reasonable excuse, fails or
  refuses to attend an anti-gambling training session, he shall be fined
  \$20,000.
\end{enumerate}

\section{Charitable Contributions.}\label{charitable-contributions.}

\begin{enumerate}
\def\labelenumi{(\alph{enumi})}
\tightlist
\item
  In the event that (i) a fine or suspension is imposed on a player,
  (ii) such fine or suspension-related Compensation amount is collected
  by the League, and (iii) the fine or suspension is not grieved
  pursuant to Article XXXI, then the NBA shall remit fifty percent
  (50\%) of the amount collected to the National Basketball Players
  Association Foundation (the ``NBPA Foundation'') or such other
  charitable organization selected by the Players Association that
  qualifies for treatment under Section 501(c)(3) of the Internal
  Revenue Code of 1986, as now in effect or as it may hereafter be
  amended (a ``Section 501(c)(3) Organization''), and that is approved
  by the NBA (which approval shall not be unreasonably withheld) (both
  hereinafter, the ``NBPA-Selected Charitable Organization''); provided,
  however, that any contributions made by the NBPA-Selected Charitable
  Organization to a player charitable foundation cannot be intended to
  reimburse the player for the financial impact of a fine or suspension.
  The NBA shall remit the remaining fifty percent (50\%) of the amount
  collected to a Section 501(c)(3) Organization selected by the NBA and
  approved by the Players Association, which approval shall not be
  unreasonably withheld. For purposes of this Section 6(a), and with
  respect to any suspension imposed on a player by the NBA of five (5)
  games or more, the NBA shall be required to collect a
  suspension-related Compensation amount equal to at least five (5)
  games of such suspension.
\item
  The remittances made by the NBA pursuant to this Section 6 shall be
  made annually, ninety (90) days following the Accountants' (as defined
  in Article VII, Section 10(a)) submission to the NBA and the Players
  Association of a final Audit Report or an Interim Escrow Audit Report
  (as defined in Article VII, Section 10(a)) for the Salary Cap Year
  covering the Season during which the fines and suspension-related
  Compensation amounts are collected by the NBA.
\item
  If a timely Grievance is filed under Article XXXI challenging a fine
  or suspension of the kind designated in Section 6(a) above, and,
  following the disposition of the Grievance, the Grievance Arbitrator
  determines that all or part of the fine or suspension-related amount
  (plus any accrued interest thereon) is payable by the player to the
  League, then the League shall remit the amount collected by the League
  (plus any interest) in accordance with the provisions of Sections 6(a)
  and (b) above.
\end{enumerate}

\section{Unlawful Violence.}\label{unlawful-violence.}

When a player is convicted of (including a plea of guilty, no contest,
or nolo contendere to) a violent felony, he shall immediately be
suspended by the NBA for a minimum of ten (10) games.

\section{Counseling for Violent
Misconduct.}\label{counseling-for-violent-misconduct.}

\begin{enumerate}
\def\labelenumi{(\alph{enumi})}
\tightlist
\item
  In addition to any other rights a Team or the NBA may have by contract
  or law, when the NBA and the Players Association agree that there is
  reasonable cause to believe that a player has engaged in any type of
  off-court violent conduct, the player will (if the NBA and the Players
  Association so agree) be required to undergo a clinical evaluation by
  a neutral expert and, if deemed necessary by such expert, appropriate
  counseling, with such evaluation and counseling program to be
  developed and supervised by the NBA and the Players Association,
  unless the player has engaged in acts covered by the Joint NBA/NBPA
  Policy on Domestic Violence, Sexual Assault, and Child Abuse, in which
  case the terms of that Policy shall apply. For purposes of this
  paragraph, ``violent conduct'' shall include, but not be limited to,
  any conduct involving the use or threat of physical violence or the
  use of, or threat to use, a deadly weapon, any conduct which could be
  categorized as a ``hate crime,'' and any conduct involving dog
  fighting or animal cruelty.
\item
  Any player who is convicted of (including a plea of guilty, no
  contest, or nolo contendere to) a crime involving violent conduct
  shall be required to attend at least five (5) counseling sessions with
  a therapist or counselor jointly selected by the NBA and the Players
  Association, unless the player has engaged in acts covered by the
  Joint NBA/NBPA Policy on Domestic Violence, Sexual Assault, and Child
  Abuse, in which case the terms of that Policy shall apply. These
  sessions shall be in addition to any discipline imposed on the player
  by the NBA for the conduct underlying his conviction. The therapist or
  counselor who is jointly selected by the NBA and the Players
  Association shall determine the total number of counseling sessions to
  be attended by the player; however, in no event shall a player be
  required to attend more than ten (10) sessions.
\item
  Any player who, after being notified in writing by the NBA that he is
  required to undergo the clinical evaluation and/or counseling program
  authorized by Section 8(a) or 8(b) above, refuses or fails, without a
  reasonable explanation, to attend or participate in such evaluation
  and counseling program within seventy-two (72) hours following such
  notice, shall be fined by the NBA in the amount of \$10,000 for each
  day following such seventy-two (72) hours that the player refuses or
  fails to participate in such program.
\end{enumerate}

\section{Firearms and Other Weapons.}\label{firearms-and-other-weapons.}

\begin{enumerate}
\def\labelenumi{(\alph{enumi})}
\tightlist
\item
  Whenever a player is physically present at a facility or venue owned,
  operated, or being used by a Team, the NBA, or any League-related
  entity, and whenever a player is traveling on any NBA-related
  business, whether on behalf of the player's Team, the NBA, or any
  League-related entity, such player shall not possess a firearm of any
  kind or any other deadly weapon. For purposes of the foregoing, ``a
  facility or venue'' includes, but is not limited to: an arena; a
  practice facility; a Team or League office or facility; an All-Star or
  NBA Playoff venue; and the site of a promotional or charitable
  appearance.
\item
  At the commencement of each Season, and if the player owns or
  possesses any firearm, the player will provide the Team with proof
  that the player possesses a license or registration as required by law
  for any such firearm. Each player is also required to provide the Team
  with proof of any modifications or additions made to this information
  during the Season.
\item
  Any violation of Section 9(a) or Section 9(b) above shall be
  considered conduct prejudicial to the NBA under Article 35(d) of the
  NBA Constitution and By-Laws, and shall therefore subject the player
  to discipline by the NBA in accordance with such Article.
\end{enumerate}

\section{One Penalty.}\label{one-penalty.}

\begin{enumerate}
\def\labelenumi{(\alph{enumi})}
\tightlist
\item
  The NBA and a Team shall not discipline a player for the same act or
  conduct. The NBA's disciplinary action will preclude or supersede
  disciplinary action by any Team for the same act or conduct.
\item
  When the NBA becomes aware of any potential or actual disciplinary
  action which may be or has been imposed by a Team for a player's act
  or conduct, the NBA may, within forty-eight (48) hours, prohibit the
  discipline from being imposed or rescind the discipline that has been
  imposed, as applicable. If the NBA prohibits or rescinds the
  discipline, only the NBA shall thereafter be permitted to impose
  discipline on the player for that act or conduct. If the NBA does not
  prohibit or rescind the Team's discipline, the Team may impose its
  proposed discipline or the Team's discipline will remain in effect, as
  applicable, and, if the Team's discipline becomes effective or remains
  in effect, the NBA may not thereafter impose discipline on the player
  for that act or conduct.
\item
  Notwithstanding anything to the contrary contained in Sections 10(a)
  or 10(b), (i) the same act or conduct by a player may result in both a
  termination of the player's Uniform Player Contract by his Team and
  the suspension of the player by the NBA if the egregious nature of the
  act or conduct is so lacking in justification as to warrant such
  double penalty, and (ii) both the NBA and the Team to which a player
  is traded may impose discipline for a player's failure to report for a
  trade in accordance with paragraph 10(d) of the Uniform Player
  Contract.
\end{enumerate}

\section{League Investigations.}\label{league-investigations.}

\begin{enumerate}
\def\labelenumi{(\alph{enumi})}
\tightlist
\item
  Players are required to cooperate with investigations of alleged
  player misconduct conducted by the NBA. Failure to so cooperate, in
  the absence of a reasonable apprehension of criminal prosecution, will
  subject the player to reasonable fines and/or suspensions imposed by
  the NBA. Any investigations of alleged misconduct that is covered by
  the Joint NBA/NBPA Policy on Domestic Violence, Sexual Assault, and
  Child Abuse shall be governed by the terms of that Policy.
\item
  Except as set forth in Section 11(c) below, the NBA shall provide the
  Players Association with such advance notice as is reasonable in the
  circumstances of any interview or meeting to be held (in person or by
  telephone) between an NBA representative and a player under
  investigation by the NBA for alleged misconduct, and shall invite a
  representative of the Players Association to participate or attend.
  The failure or inability of a Players Association representative to
  participate in or attend the interview or meeting, however, shall not
  prevent the interview or meeting from proceeding as scheduled. A
  willful disregard by the NBA of its obligation to notify the Players
  Association as provided for by this Section 11(b) shall bar the NBA
  from using as evidence against the player in a proceeding involving
  such alleged misconduct any statements made by the player in the
  interview or meeting conducted by the NBA representative.
\item
  The provisions of Section 11(b) above shall not apply to interviews or
  meetings: (i) held by the NBA as part of an investigation with respect
  to alleged player misconduct that occurred at the site of a game; and
  (ii) which take place during the course of, or immediately preceding
  or following, such game. With respect to any such interview or
  meeting, the NBA's only obligation shall be to provide notice to the
  Players Association that the NBA will be conducting an investigation
  and holding an interview or meeting in connection therewith. Such
  notice may be given by telephone at a telephone number or by email at
  an email address to be designated in writing by the Players
  Association.
\end{enumerate}

\section{On-Court Conduct.}\label{on-court-conduct.}

\begin{enumerate}
\def\labelenumi{(\alph{enumi})}
\tightlist
\item
  The parties have agreed to all of the rules governing the conduct of
  players on the playing court (as that term is defined in Article XXXI,
  Section 9(c) below) that are contained in the 2016-17 Player Conduct,
  NBA Uniform Requirements, Dress Code and Other Player-Related Matters
  Memo distributed by the NBA and dated September 30, 2016. Beginning
  with the 2017-18 Season, the NBA and the Players Association will
  bargain over any new rules governing the conduct of players on the
  playing court (including disciplinary penalties associated therewith)
  or any change to the agreed-upon rules governing the conduct of
  players on the playing court (including disciplinary penalties
  associated therewith); provided, however, that this obligation to
  bargain does not apply to the official playing rules of the NBA (or
  any change or modification thereof) or any rule affecting the
  integrity of the game or game play (or any change or modification
  thereof), except with respect to any change or modification to the
  disciplinary penalties associated with a player's violation of such
  rules.
\item
  Nothing in Section 12(a) above shall be construed to modify or alter
  (i) the NBA's existing disciplinary authority in this Agreement or
  Article 35 of the NBA Constitution governing the conduct of players on
  the playing court (as that term is defined in Article XXXI, Section
  9(c) below), including, but not limited to, the NBA's ability to
  provide notice to players that it regards a type of on-court conduct
  to be violative of its disciplinary standards, (ii) the NBA's existing
  disciplinary authority in this Agreement and/or Article 35 of the NBA
  Constitution governing off-court conduct, or (iii) Article XXXVII,
  Section 2 of this Agreement governing player uniforms.
\item
  Prior to the imposition of a suspension on a player for conduct on the
  playing court (as defined in Article XXXI, Section 9(c)), the player
  will have the opportunity to request a telephonic meeting with the
  President, League Operations, the Executive Vice President, Basketball
  Operations or their designee to discuss the incident and be heard as
  to why a suspension is unwarranted; provided, however, that the player
  must promptly notify the NBA of his desire for such a meeting, which
  will be scheduled to take place within a reasonable time period that
  will not interfere with the NBA's investigatory process and will not
  preclude the NBA from issuing a suspension prior to the player's next
  game. Notice to the player of a possible suspension may be given by
  the NBA to the Players Association by telephone at a telephone number
  or by email at an email address to be designated in writing by the
  Players Association. Notice by the player of his request for a meeting
  pursuant to this Section 12(c) may be provided through the Players
  Association on the player's behalf, and a representative of the
  Players Association may participate in any such telephone call. The
  NBA will consider any information provided during the meeting before
  finalizing its decision; provided, however, that nothing contained
  herein will require the NBA to alter its disciplinary decision or
  affect any rights the player has under Article XXXI to appeal that
  decision.
\end{enumerate}

\section{Motor Vehicles.}\label{motor-vehicles.}

At the commencement of each Season, and if the player owns or operates
any motor vehicle, the player will provide the Team with proof that the
player possesses a valid driver's license, registration documents, and
insurance for any such vehicle. For players who sign Player Contracts
during the Season, the player will provide the Team with such
information within fourteen (14) days following the execution of his
Contract. Each player is also required to provide the Team with proof of
any modifications or additions made to this information during the
Season.

\section{Player Convictions Involving Alcohol or Controlled
Substances.}\label{player-convictions-involving-alcohol-or-controlled-substances.}

In addition to any other discipline imposed by the NBA for such conduct,
any player who is convicted of (including a plea of guilty, no contest,
or nolo contendere to) driving while intoxicated, driving under the
influence, driving under the influence of a controlled substance (if
that controlled substance is not a Prohibited Substance) or any similar
crime shall be required to submit to a mandatory evaluation by the
Medical Director of the Anti-Drug Program. After that mandatory
evaluation, the Medical Director may require the player to attend up to
ten (10) substance abuse counseling sessions.

\section{Player Arrests.}\label{player-arrests.}

A Team shall not impose discipline on a player solely on the basis of
the fact that the player has been arrested. Notwithstanding the
foregoing, (a) a Team may impose discipline on a player for the conduct
underlying the player's arrest if it has an independent basis for doing
so, (b) nothing herein shall permit a Team to discipline a player for
his failure to cooperate with a Team's investigation of his alleged
misconduct if he has a reasonable apprehension of criminal prosecution,
and (c) nothing herein shall prevent a Team from precluding a player
from participating in Team activities without loss of pay to the extent
it otherwise has the right to do so.

\section{Joint NBA/NBPA Policy on Domestic Violence, Sexual Assault, and
Child
Abuse.}\label{joint-nbanbpa-policy-on-domestic-violence-sexual-assault-and-child-abuse.}

Effective July 1, 2017, the Joint NBA/NBPA Policy on Domestic Violence,
Sexual Assault, and Child Abuse (and any amendments thereto), which is
attached as Exhibit F hereto, shall apply and remain in effect. Any
evaluation, counseling, treatment, and/or discipline of a player for
engaging in acts covered by this Policy shall be governed by the terms
of the Policy.

\chapter{BASKETBALL RELATED INCOME, SALARY CAP, MINIMUM TEAM SALARY, AND
ESCROW
ARRANGEMENT}\label{basketball-related-income-salary-cap-minimum-team-salary-and-escrow-arrangement}

\chaptermark{BASKETBALL RELATED INCOME, SALARY CAP, MINIMUM TEAM SALARY \ldots}

\section{Definitions.}\label{definitions.-1}

For purposes of this Agreement, the following terms shall have the
meanings set forth below:

\begin{enumerate}
\def\labelenumi{(\alph{enumi})}
\tightlist
\item
  \textbf{Basketball Related Income.}

  \begin{enumerate}
  \def\labelenumii{(\arabic{enumii})}
  \item
    ``Basketball Related Income'' (``BRI'') for a Salary Cap Year means
    the aggregate operating revenues (including the value of any
    property or services received in any barter transactions), accounted
    for in accordance with Section 1(b)(1) below, received or to be
    received for or with respect to such Salary Cap Year by the NBA, NBA
    Properties, Inc., including any of its subsidiaries whether now in
    existence or created in the future (hereinafter, ``Properties''),
    NBA Media Ventures LLC (``Media Ventures''), any other entity which
    is controlled, or in which at least fifty percent (50\%) of the
    issued and outstanding ownership interests are owned, by the NBA,
    Properties, Media Ventures, and/or a group of NBA Teams
    (hereinafter, ``League-related entity'') (but excluding the amount
    of such League-related entity's revenues equal to the portion of its
    total revenues that is proportionate to the share of the entity's
    profits to which ownership interests not owned by the NBA,
    Properties, Media Ventures and/or a group of NBA Teams are
    entitled), all NBA Teams other than Expansion Teams during their
    first two (2) Salary Cap Years (but including the Expansion Teams'
    shares of national television, radio, cable and other broadcast
    revenues, and any other League-wide revenues shared by the Expansion
    Teams, provided such revenues are otherwise included in BRI) and
    Related Parties (in accordance with Section 1(a)(7)(i) below), from
    all sources, whether known or unknown, whether now in existence or
    created in the future, to the extent derived from, relating to, or
    arising directly or indirectly out of, the performance of Players in
    NBA basketball games or in NBA-related activities. For purposes of
    this definition of BRI: (x) ``operating revenues'' shall include,
    but not be limited to, any type of revenue included in BRI for the
    1995-96 and 1996-97 Salary Cap Years (without regard to whether such
    type of revenue is received on a lump-sum, non-recurring or
    extraordinary basis, but subject to any specific rules set forth in
    this Article VII relating to the recognition or amortization of such
    amounts); and (y) ``Player'' means a person: who is under a Player
    Contract to an NBA Team; who completed the playing services called
    for under a Player Contract with an NBA Team at the conclusion of
    the prior Season; or who was under a Player Contract with an NBA
    Team during (but not at the conclusion of) the prior Season, but
    only with respect to the period for which he was under such
    Contract. Subject to the foregoing, BRI shall include, but not be
    limited to, the following revenues:

    \begin{enumerate}
    \def\labelenumiii{(\roman{enumiii})}
    \tightlist
    \item
      Regular Season gate receipts, net of applicable taxes, surcharges,
      imposts, facility fees, and other charges (including, without
      limitation, charges related to arena financings) imposed by
      governmental or quasi-governmental agencies other than income
      taxes (collectively, ``Taxes''), and net of all reasonable and
      customary Team and Related Party ticket-related expenses and
      premium seating ticket expenses related thereto, subject to the
      provisions of Section 1(a)(6) below, including, without
      limitation, gate receipts received or to be received by a Related
      Party in accordance with Section 1(a)(7)(i) below, including: (A)
      the value (determined on the basis of the price of the ticket) of
      all tickets traded by a Team for goods or services; and (B) the
      value (determined on the basis of the League-wide average ticket
      price for non-Season tickets) of all tickets for Regular Season
      games provided by a Team on a complimentary basis, without
      monetary or other compensation to a Team including complimentary
      admission to luxury suites (including standing room only tickets
      and tickets provided to Team employees other than Players);
      provided, however, that (x) the value of the ``Excluded
      Complimentary Tickets'' with respect to all Regular Season games
      in a Season shall be excluded from BRI, and (y) in addition,
      tickets provided as part of sponsorships and other transactions,
      where the proceeds from such transactions have been included in
      BRI, shall not be included in determining the number of
      complimentary tickets in any Season. For purposes of the
      foregoing, ``Excluded Complimentary Tickets'' shall mean (A) 1.9
      million tickets for the 2017-18 Season, increasing by 50,000
      tickets for each Season during the term of this Agreement, and (B)
      any tickets provided on a complimentary basis to or on behalf of
      Players;
    \item
      All proceeds of any kind, net of reasonable and customary expenses
      related thereto, subject to the provisions of Section 1(a)(6)
      below, from the broadcast or exhibition of, or the sale, license
      or other conveyance or exploitation of the right to broadcast or
      exhibit, NBA preseason, Regular Season and Playoff games and
      summer league and other NBA-related off-season games involving
      Players, highlights or portions of such games, and non-game NBA
      programming, on any and all forms of radio, television, telephone,
      internet, and any other communications media, forms of
      reproduction and other technologies, whether presently existing or
      not, anywhere in the world, whether live or on any form of delay,
      including, without limitation, network, local, cable, direct
      broadcast satellite and any form of pay television, and all other
      means of distribution and exploitation, whether presently existing
      or not and whether now known or hereafter developed, including,
      without limitation, such proceeds received or to be received by a
      Related Party (in accordance with Section 1(a)(7)(i) below), but
      not including the value of any broadcast, cablecast or telecast
      time provided as part of any such transaction that is used solely:
      (A) to promote or advertise the NBA, its Teams, League-related
      entities that generate BRI, Players, the NBA Development League
      (the ``NBADL'') (except to the extent the value of such time for
      the NBADL exceeds \$5 million), or the sport of basketball (but
      not the value of time used to promote or advertise the Women's
      National Basketball Association (the ``WNBA'') which shall be
      included in BRI); (B) to promote or advertise products,
      programming, merchandise, services or events that (i) produce
      revenues that are includable in BRI or (ii) are jointly licensed
      or otherwise agreed upon by the NBA and the Players Association;
      (C) to promote or advertise charitable, not-for-profit or
      governmental organizations or agencies; or (D) for public service
      announcements;
    \item
      All proceeds of any kind from Exhibition games including at least
      one NBA Team, net of Taxes and all reasonable and customary game,
      pre-season and training camp expenses (including summer league
      expenses), subject to the provisions of Section 1(a)(6) below,
      including, without limitation, such proceeds received or to be
      received by a Related Party (in accordance with Section 1(a)(7)(i)
      below);
    \item
      All playoff gate receipts of any kind, net of Taxes, arena rentals
      to the extent reasonable and customary, and all other reasonable
      and customary expenses, except the Player Playoff Pool, including,
      without limitation, such proceeds received or to be received by a
      Related Party (in accordance with Section 1(a)(7)(i) below);
    \item
      All proceeds of any kind, net of reasonable and customary expenses
      (including Taxes) related thereto, subject to the provisions of
      Section 1(a)(6) below, from: (A) in-arena sales of novelties and
      concessions (including revenues derived from the sale of novelties
      and concessions: (1) during (and immediately preceding or after)
      the Team's games or other public Team events at the arena (or
      practice facility), from carts and kiosks or other similar sales
      locations that are only operated on an intermittent basis (i.e.,
      principally when an NBA, NHL, or other public event is being held
      at the arena (or, respectively, the practice facility)) in (i) the
      arena plaza or elsewhere on the immediate perimeter of the arena
      (or, respectively, the practice facility), or (ii) directly across
      the street from the arena (or, respectively, the practice
      facility); and (2) from Team-organized viewing parties of NBA
      games), (B) sales of novelties and concessions in Team-identified
      stores located within such radius of the Team's home arena as is
      permitted by the NBA, (C) NBA game parking and programs, (D) Team
      sponsorships (whether or not the proceeds are directly or
      indirectly donated to charity), (E) Team promotions, (F) temporary
      arena signage (as defined in Section 1(a)(vi) below), (G) arena
      club revenues, (H) summer camps, (I) non-NBA basketball
      tournaments, (J) mascot and dance team appearances, (K) the sale
      of the right to pour beverages or (except as provided in Section
      1(a)(2)(xx) below) to provide concessions, and (L) sales of jersey
      patch rights, in each case, to the extent that such proceeds are
      related to the performance of Players in NBA basketball games or
      NBA-related activities, including, without limitation, such
      proceeds received or to be received by a Related Party (in
      accordance with Sections 1(a)(1)(vi) and 1(a)(7)(i) below). For
      the purposes of clarity, ``Team-identified stores'' includes
      stores owned by Teams or Related Parties that sell predominately
      Team-branded merchandise, whether or not the store is
      Team-identified;
    \item
      Fifty percent (50\%) of the gross proceeds, net of fifty percent
      (50\%) of Taxes, and net of fifty percent (50\%) of all reasonable
      and customary Team and Related Party expenses related thereto,
      subject to the provisions of Section 1(a)(6) below, from the sale
      of fixed arena signage within or outside of the arena in which an
      NBA Team plays more than one-half of its Regular Season home
      games, including, without limitation, such proceeds received or to
      be received by a Related Party (in accordance with this Section
      and Section 1(a)(7)(i) below). ``Fixed'' arena signage means signs
      (including, without limitation, electronic signs) that are
      displayed during all Regular Season NBA games and at least 75\% of
      non-NBA events at the arena during the Regular Season (in each
      case prorated to reflect contracts in effect beginning in-season),
      with all other signs being treated as ``temporary'' signage (for
      clarity, subject to applicable allocations). Fixed arena signage
      also includes ``sponsorship entitlement areas'' that are
      accessible or visible during all Regular Season NBA games and at
      least 75\% of non-NBA events at the arena during the Regular
      Season (in each case prorated to reflect contracts in effect
      beginning in-season). Revenues from sponsorship entitlement areas
      that do not qualify under the preceding sentence shall be treated
      as temporary signage. Revenues from signage outside the arena
      shall be included in BRI as fixed arena or temporary signage, as
      applicable, if: (a) the signage is attached to the arena or a
      physically connected parking facility; (b) the right to the
      signage revenues is conveyed in the Team's arena lease or other
      agreement, if applicable, governing a Team's use of an arena
      entered into by or on behalf of the Team (for clarity, in
      circumstances where the Team has a lease or similar agreement with
      a Related Party arena company, the foregoing is not intended to
      apply to any lease or similar agreement provisions, if any,
      between the Related Party arena company and the property owner
      governing the arena company's use of any property other than the
      arena itself); (c) only in the case of revenues received by the
      Team (and not by any Related Party), the signage is
      Team-identified (i.e., contains Team name, marks, logo,
      intellectual property); or (d) the signage is (x) in the area
      immediately proximate to the arena in an arena plaza in front of a
      main arena entrance or (y) attached to a standalone parking
      facility that is directly across the street from the arena (except
      that for fixed signage that falls within BRI solely under this
      subsection (d), twenty-five percent (25\%) of the gross proceeds
      (net of twenty-five percent (25\%) of Taxes, and net of
      twenty-five percent (25\%) of all reasonable and customary
      expenses related thereto subject to the provisions of this Section
      1(a)(1)(vi) and Section 1(a)(6) below) shall be included as BRI
      revenues). Other revenues received by a Team or Related Party from
      signage outside the arena shall be excluded from BRI;
    \item
      Fifty percent (50\%) of the gross proceeds of any kind, net of
      fifty percent (50\%) of Taxes, and net of fifty percent (50\%) of
      all reasonable and customary Team and Related Party expenses
      related thereto, subject to the provisions of Section 1(a)(6)
      below, from the sale, lease or licensing of luxury suites
      calculated on the basis of the actual proceeds received by the
      entity, including, without limitation, proceeds received or to be
      received by a Related Party (in accordance with Section 1(a)(7)(i)
      below), that sold, leased, or licensed such luxury suites;
      provided, however, that, other than the additional amounts paid by
      luxury suite holders to the Team for tickets pursuant to
      arrangements in which admission to games is not part of the
      agreement to buy, lease or license the luxury suite, thereby
      requiring the luxury suiteholder to make a separate payment for
      such admission, if any, this amount shall be the only amount
      included in BRI for the sale, lease or licensing of luxury suites
      and that, to the extent that the sale, lease or licensing of the
      luxury suite grants rights to the luxury suite for a period of
      more than one (1) year, for purposes of calculating the amount
      includable in BRI for any Salary Cap Year, the proceeds shall be
      determined on the basis of the annual fee or charge provided for
      in any such transaction and, if payments are made in addition to
      or in the absence of such an annual fee or charge, the value of
      such payments shall be amortized over the period of the sale,
      lease or license, unless such period exceeds twenty (20) years, in
      which event an amortization period of twenty (20) years shall be
      used;
    \item
      Fifty percent (50\%) of the gross proceeds, net of fifty percent
      (50\%) of Taxes, and net of fifty percent (50\%) of all reasonable
      and customary Team and Related Party expenses related thereto,
      subject to the provisions of Section 1(a)(6) below, from arena
      naming rights agreements with respect to arenas in which an NBA
      Team plays more than one-half of its Regular Season home games,
      including, without limitation, such proceeds received or to be
      received by a Related Party (in accordance with Section 1(a)(7)(i)
      below);
    \item
      Except as provided in Section 1(a)(2) below, proceeds received by
      Properties or any other League-related entity, net of reasonable
      and customary expenses (including Taxes) related thereto, subject
      to the provisions of Section 1(a)(6) below, from the following:
      (A) international television; (B) sponsorships; (C) NBA-related
      revenues from NBA Entertainment; (D) the All-Star Game; (E) other
      NBA special events; and (F) all other sources of revenue received
      by Properties or any other League-related entity, in each case
      under (A)-(F), to the extent that such proceeds are related to the
      performance of Players in NBA basketball games or NBA-related
      activities;
    \item
      Proceeds from premium seat licenses (other than licenses of luxury
      suites, which are governed by Section 1(a)(1)(vii) above), net of
      Taxes, and all reasonable and customary Team and Related Party
      expenses related thereto, subject to the provisions of Section
      1(a)(6) below, attributable to NBA-related events amortized over
      the period of the license (including, without limitation, such
      proceeds received or to be received by a Related Party (in
      accordance with Section 1(a)(7)(i) below), unless such period
      exceeds twenty (20) years, in which event an amortization period
      of twenty (20) years shall be used;
    \item
      Fifty percent (50\%) of the gross proceeds, net of fifty percent
      (50\%) of Taxes, and fifty percent (50\%) of reasonable and
      customary Team and Related Party expenses related thereto, subject
      to the provisions of Section 1(a)(6) below, from the sale of
      naming rights with respect to practice facilities used by NBA
      Teams, including, without limitation, such proceeds received or to
      be received by a Related Party (in accordance with Section
      1(a)(7)(i) below);
    \item
      If the right to receive revenues included in BRI is sold or
      transferred to an entity other than an entity referred to in
      Section 1(a)(1) above (such that those revenues would not be
      included in BRI pursuant to that subsection), then BRI shall be
      deemed to include the amount of revenues that would have been
      received by the seller or transferor and would have been included
      in BRI in such Salary Cap Year (subject to any applicable
      allocations provided for above), absent such sale or transfer,
      provided that a pledge, hypothecation, collateral assignment or
      other similar transaction involving such revenues, shall not be
      considered a sale or transfer within the meaning of this Section
      1(a)(1)(xii). The NBA will work in good faith to secure access to
      appropriate third-party books and records in the event the parties
      agree, or it is determined by an arbitrator, that a sale/transfer
      of BRI has occurred or been agreed to in accordance with this
      Section 1(a)(1)(xii). In any dispute over the value of BRI
      sold/transferred, subject to an arbitrator's determinations of
      admissibility and relevance, neither party shall be barred from
      seeking to rely on the terms of the underlying transaction;
    \item
      All proceeds, net of Taxes, less reasonable and customary expenses
      (which expenses shall be subject to New Venture treatment, if
      applicable, under Section 1(a)(6)(iii) below), subject to the
      provisions of Section 1(a)(6) below, from gambling on NBA games or
      any aspect of NBA games, subject to appropriate treatment of
      categories of excluded revenues or other amounts, if applicable,
      under Section 1(a)(2) below and allocations for multi-element
      deals. BRI shall exclude revenues from gambling on NBA games or
      any aspect of NBA games generated by casinos or other gambling
      businesses, owned or operated by a Team, Related Party, or a
      League-related entity, whose total revenues are not predominantly
      from gambling on NBA games or any aspect of NBA games;
    \item
      All proceeds, net of Taxes and reasonable and customary expenses
      related thereto, subject to Section 1(a)(6) below, from a Team's
      championship parade, provided, however, that in no event shall
      such expenses cause the amount included in BRI relating to the
      championship parade to be less than zero (0) for any Salary Cap
      Year;
    \item
      Fifty percent (50\%) of the gross proceeds, net of fifty percent
      (50\%) of reasonable and customary expenses (including Taxes)
      related thereto, subject to Section 1(a)(6) below, from (A) tours
      of the Team's home arena, and (B) fees from ATMs in the Team's
      home arena; and
    \item
      Player income or ``privilege'' tax payments to Teams or Related
      Parties, provided that such payments will continue to be excluded
      from BRI for any Team or Related Party that received such payments
      in the 2015-16 Salary Cap Year (e.g., Memphis, New Orleans).
    \item
      Consistent with the parties' practice under the 2011 CBA, payments
      from the NBA to Teams for participation in international Regular
      Season games will be included in miscellaneous BRI at the Team
      level, with the NBA recording its expenses (including such
      payments to Teams) at the League level in Special Events.
    \end{enumerate}
  \item
    Notwithstanding anything to the contrary in Section (a)(1) above, it
    is understood that the following is a non-exclusive list of examples
    of revenues that are or may be received by the NBA, Properties,
    Media Ventures, other League-related entities, NBA Teams and Related
    Parties (the foregoing persons or entities, beginning with ``NBA,''
    collectively referred to in this Section 1(a)(2) only as
    ``NBA-related entities'') that are not derived from, and do not
    relate to or arise out of, the performance of Players in NBA
    basketball games or in NBA-related activities or are otherwise
    expressly excluded from the definition of BRI:

    \begin{enumerate}
    \def\labelenumiii{(\roman{enumiii})}
    \tightlist
    \item
      Proceeds from the assignment of Player Contracts;
    \item
      Proceeds (A) from the sale, transfer or other disposition of any
      of the assets or property (excluding ordinary course sales of
      inventory and the revenues (if any) deemed to be included in BRI
      pursuant to Section 1(a)(1)(xii) above) of, or ownership interests
      in, any NBA-related entity, or (B) from loans or other financing
      transactions;
    \item
      Proceeds from the grant of Expansion Teams and relocation fees
      paid by existing Teams to NBA-related entities;
    \item
      Dues;
    \item
      Capital contributions received by an NBA-related entity from one
      of its owners, shareholders, members or partners;
    \item
      Fines and compensation withheld in connection with suspensions;
    \item
      Revenue sharing (by means of revenue transfers or otherwise) among
      Teams;
    \item
      Interest income;
    \item
      Insurance recoveries, except where, and only to the extent that,
      such recoveries are in respect of lost revenues that would have
      otherwise been included in BRI, in which event such recoveries
      shall be included in BRI in the Salary Cap Year in which they are
      received;
    \item
      Proceeds from the sale or rental of real estate;
    \item
      Any thing of value received in connection with the design or
      construction of a new or renovated arena or other team facility
      including, but not limited to, receipt of title to or a leasehold
      interest in real property or improvements, reimbursement of
      project-related expenses, benefits from project-related
      infrastructure improvements, or tax abatements, unless (and only
      to the extent that) such value is being provided to the Team or a
      Related Party in lieu of payments that the Team or Related Party
      would have otherwise received pursuant to an arena lease or other
      instrument concerning a Team's use of an arena (``lease'') and
      would have constituted BRI if paid to the Team or a Related Party;
      provided, however, that the determination of the amount, if any,
      to be included in BRI with respect to the value of any of the
      foregoing shall be made either (A) in accordance with the
      provisions of Section 1(a)(4) below or (B) based upon direct
      evidence that the Team or Related Party, after proposing that it
      would receive certain revenues constituting arena-generated BRI,
      subsequently agreed specifically to forego such revenues in direct
      exchange for a thing of value (as described above in this Section
      1(a)(2)(xi)) with the consequence that the arena-generated BRI
      revenues received or to be received by the Team or Related Party
      were or would be (in the opinion of the Accountants) less than the
      fair market value of arena-generated BRI revenues received or to
      be received by other NBA Teams in similar transactions, or (C)
      based upon direct evidence that the parties to the transaction had
      agreed that certain revenues constituting arena-generated BRI
      would be paid to the Team or Related Party and that such revenues
      were subsequently foregone by the Team or the Related Party in
      direct exchange for a thing of value (as described above in this
      Section 1(a)(2)(xi)); and provided further that, when a
      determination is made pursuant to clause (B) or clause (C) of this
      Section 1(a)(2)(xi), the amount(s), if any, to be included in BRI
      shall be allocated (with an appropriate interest adjustment to
      reflect the time value of money where the thing of value received
      by the Team or Related Party is in the form of cash or a cash
      equivalent, such as a check or wire transfer) over the Salary Cap
      Years in which the arena-generated BRI revenues foregone would
      have been received by the Team or Related Party (up to a maximum
      of twenty (20) Salary Cap Years) and not on a lump-sum basis;
    \item
      Any thing of value that induces or is intended to induce a Team
      either to relocate to or remain in a particular geographic
      location (whether or not provided in connection with a new or
      renegotiated arena lease), unless (and only to the extent that)
      such value is being provided to the Team or a Related Party in
      lieu of payments that the Team or Related Party would have
      otherwise received pursuant to an arena lease and that would have
      constituted BRI had they been paid to the Team or a Related Party;
      provided, however, that the determination of the amount, if any,
      to be included in BRI shall be made either (A) in accordance with
      the provisions of Section 1(a)(4) below or (B) based upon direct
      evidence that the parties to the transaction had agreed that
      certain revenues constituting arena-generated BRI would be
      foregone by the Team or Related Party, in direct exchange for a
      thing of value as described above in this Section 1(a)(2)(xii),
      and provided, further that, when a determination is made pursuant
      to clause (B) of this Section 1(a)(2)(xii), the amount(s), if any,
      to be included in BRI shall be allocated (with an appropriate
      interest adjustment to reflect the time value of money where the
      thing of value received by the Team or Related Party is in the
      form of cash or a cash equivalent, such as a check or wire
      transfer) over the Salary Cap Years in which the arena-generated
      BRI revenues foregone would have been received by the Team or
      Related Party (up to a maximum of fifteen (15) Salary Cap Years)
      and not on a lump-sum basis. With respect to transactions
      involving payments asserted to fall within the exclusion in this
      Section 1(a)(2)(xii), the NBA will provide the Players Association
      with the executed memoranda of understanding, term sheet, or other
      such executed summary of terms, if any, for such underlying
      transactions;
    \item
      Payments made to Teams or to the NBA pursuant to the provisions of
      Article VII, Section 12 (Escrow and Tax Arrangement) below;
    \item
      Distributions, dividends or royalties paid by any NBA-related
      entity to owners, shareholders, members or partners;
    \item
      Any category or source of revenue or proceeds that was expressly
      identified in any BRI Report (as defined in Section 10(b) below)
      or in any document or written communication (including debriefing
      memos) authored by the Accountants and provided to the Players
      Association and the NBA (but excluding any underlying work papers)
      in connection with the Audit Reports for any of the 1995-96
      through 2015-16 Salary Cap Years that was not included in BRI for
      such Salary Cap Years, unless such category or source was included
      on the ``open issues'' list prepared by the Accountants in
      connection with any of the Audit Reports for the 2005-06 through
      2015-16 Salary Cap Years, in which case such category or source
      shall be included in or excluded from BRI, as the case may be, in
      accordance with the other terms of this Article;
    \item
      Proceeds received by (A) Properties (and its related entities)
      that were treated or, consistent with past practice, would have
      been treated as within the scope of the Agreement between NBA
      Properties, Inc., and the National Basketball Players Association,
      dated as of September 18, 1995, as amended January 20, 1999, July
      29, 2005 and December 8, 2011 (the ``2011 Group License
      Agreement'') (including, but not limited to, proceeds received
      pursuant to the license of ``fantasy games,'' which proceeds would
      have been included in the computation of Player Merchandise
      Revenues in accordance with the 2011 Group License Agreement), or
      (B) a League-related entity relating to the following categories
      defined in the same manner as was used in the audited League
      Entities' Combined Financial Statements for the year ended
      September 30, 2015: (x) licensing; and/or (y) a League-related
      entity's representation of, and services performed for, third
      parties. For purposes of the foregoing sentence, ``third parties''
      refers to persons or entities that are not owned or controlled by
      persons or entities that own a majority interest in or otherwise
      control an NBA Team or, if such third party is a Related Party,
      proceeds received by the League-related entity shall not be
      included in BRI if representation of such Related Party does not
      relate either to such entity's NBA ownership or NBA Players;
    \item
      Monies collected from Team-related fundraising for charitable
      purposes or other charitable activities (including Team-organized
      ``50/50 raffles''), other than monies paid pursuant to Team
      sponsorship agreements that are included in BRI pursuant to
      Section 1(a)(1)(v) above; and
    \item
      Proceeds solely related to the NBADL and other leagues, teams and
      basketball organizations (e.g., an international league) that do
      not involve the playing of basketball by any then-current NBA
      players;
    \item
      Proceeds from the leasing or use of any Team physical assets
      (e.g., a Team plane);
    \item
      Any thing of value received from a concessionaire, food service
      vendor or other third party equipment or service provider that, if
      received in kind, is installed in an NBA arena or, if received in
      cash, is directed to defraying the costs of the construction or
      substantial renovation of an NBA arena; and
    \item
      Proceeds from businesses outside the arena (e.g., restaurants,
      casinos, hotels, retail businesses, etc.), except for revenues
      otherwise included in BRI for Team-identified stores. For clarity,
      the foregoing exclusion will not apply to revenues from the
      business operations of the NBA basketball team that are otherwise
      includable as BRI under other provisions of this Agreement,
      including, without limitation, revenues received from sales of
      Team game tickets, media rights, sponsorships, signage outside the
      arena (subject to the limitations set forth in Section 1(a)(1)(vi)
      above), and arena plaza game-day sales of novelties and
      concessions (subject to the limitations set forth in Section
      1(a)(1)(v) above).
    \end{enumerate}
  \item
    The parties agree that (i) in determining whether a category or
    source of revenue or proceeds constitutes BRI: (A) consideration
    shall be given to whether such category or source is more similar in
    kind or nature to the included categories and sources listed in
    Section 1(a)(1)(i) through (xvii) above, on the one hand, or to the
    excluded categories and sources listed in Section 1(a)(2)(i) through
    (xxi) above, on the other; and, (B) no inference may be drawn from
    the fact that such category or source was not included in the
    categories and sources listed in Section 1(a)(1)(i) through (xvii)
    above, or the fact that such category or source was not included in
    the categories and sources listed in Sections 1(a)(2)(i) through
    (xxi) above; and (ii) in any proceeding involving a dispute over (A)
    the includability or categorization of any revenue or expense item
    for BRI purposes; (B) the amount to be included in or deducted from
    BRI with respect to any revenue or expense item; or (C) the
    accounting methodology used by the Accountants in connection with
    any audit of BRI, the parties may refer to the past practice of the
    parties or the Accountants in connection with the Audit Reports for
    any of the 1999-2000 through 2015-16 Salary Cap Years; provided,
    however, that no reference may be made to the past practice of the
    parties or the Accountants with respect to any source or category of
    revenue or expense that was included on the ``open issues'' list
    prepared by the Accountants in connection with any of such Audit
    Reports; provided, further, that any such past practice shall be
    superseded to the extent changed or clarified by the terms of this
    Agreement. In addition, no reference may be made, with respect to
    expenses related to the NBA's non-international business, to the
    fact that such category of expenses falls within Section 1(a)(14)
    below, to argue for the inclusion or exclusion of expenses related
    to the League's non-international business.
  \item
    The parties agree that, with respect to any lease entered into after
    the date of this Agreement between a Team (or a Related Party) and
    an arena that is not a Related Party, the Accountants may attribute
    to the Team (or a Related Party) for purposes of computing BRI for a
    Salary Cap Year portions of arena revenues received by the arena or
    its related entities that would be included in BRI if received by
    the Team (or a Related Party) to the following extent: in the event
    of a renewal, extension or renegotiation of a lease between the same
    parties, or a new lease entered into by a Team (or a Related Party)
    with an arena that is not a Related Party, the Team will be deemed
    to receive in the first Salary Cap Year covered by the new lease or
    by the renewal, extension or renegotiation of the existing lease (as
    the case may be) the greater of (i) the amount of such revenues that
    the Team or the Related Party in fact receives under the lease or,
    (ii) if in the opinion of the Accountants, the Team (and/or the
    Related Party) is receiving substantially less than fair market
    value as determined by the Accountants (taking into account factors
    such as the rent paid by the Team or the Related Party, the number
    and identity of other major tenants in the arena, market conditions,
    the extent to which arena revenues are used to fund construction or
    renovations of the arena, and comparable lease arrangements in the
    NBA), an amount determined by the Accountants to constitute the fair
    market value of the revenues that a tenant, in the same
    circumstances as the Team or Related Party, would receive for such
    Salary Cap Year. In either of the preceding cases, the Accountants
    will also determine the amount to be included in BRI for Salary Cap
    Years beyond the first Salary Cap Year.
  \item
    \begin{enumerate}
    \def\labelenumiii{(\roman{enumiii})}
    \tightlist
    \item
      In no event shall the same revenues be included in BRI, directly
      or indirectly, more than once (including as a result of changes in
      accounting methods or practices), the purpose of this provision
      being to preclude the double-counting of revenues, whether in the
      same or in multiple Salary Cap Years.
    \item
      In no event shall the same expenses be deducted from BRI, directly
      or indirectly, more than once (including as a result of changes in
      accounting methods or practices), the purpose of this provision
      being to preclude the double-counting of expenses, whether in the
      same or in multiple Salary Cap Years.
    \end{enumerate}
  \item
    Subject to Section 11 below (Players Association Audit Rights):

    \begin{enumerate}
    \def\labelenumiii{(\roman{enumiii})}
    \tightlist
    \item
      With respect to expenses incurred in connection with all proceeds
      coming within Section 1(a)(1)(v) above, all reported expenses
      shall be conclusively presumed to be reasonable and customary, and
      such expenses shall not be the subject of the accounting
      procedures set forth in Section 10 below.
    \item
      With respect to expenses incurred in connection with all proceeds
      coming within Section 1(a)(1)(ix) above that are consistent with
      the types and categories of expenses incurred by Properties as
      reflected in the audited financial reports of Properties for the
      year ended July 31, 1994, (1) all such reported expenses shall be
      conclusively presumed to be reasonable and customary, and such
      expenses shall not be the subject of the accounting procedures set
      forth in Section 10 below, but (2) such expenses shall be
      disallowed to the extent they exceed the ratio of expenses to
      revenues for the category of revenues set forth in Exhibit D
      hereto.
    \item
      With respect to the NBA Store (the ``Store'') and any other new
      venture or business (whether or not involving the creation of a
      new entity) undertaken by the NBA, Properties, Media Ventures, or
      any other League-related entity requiring significant capital
      investment or start-up costs (``New Venture''), the League-related
      entities shall be able to deduct from BRI reasonable and customary
      expenses related thereto, including, but not limited to, cost of
      goods sold, sales tax, all reasonable operating expenses of the
      Store or New Venture (including, but not limited to, salaries and
      benefits directly related to the operations of the Store or New
      Venture, promotional and advertising costs, rent, direct overhead,
      general and administrative expenses of the Store or New Venture),
      reasonable financing costs and amortization of capital
      improvements and start-up costs; provided, however, that in no
      event shall the expenses attributable to the Store or New Venture
      cause the amount included in BRI for the Store or New Venture to
      be less than zero (0) for any Salary Cap Year.
    \item
      With respect to miscellaneous BRI or new categories of BRI (other
      than revenues attributable to the Store or a New Venture), the
      NBA, Properties, Media Ventures, other League-related entities,
      Teams and Related Parties shall be able to deduct all reasonable
      and customary expenses (including reasonable and customary Taxes),
      including, for example, in connection with All-Star Weekend,
      subject to the terms of this Section 1(a)(6).
    \item
      In each Salary Cap Year, except for Playoff-Related Revenues and
      Expenses (as defined below), all Team and Related Party revenues
      included in, and all Team and Related Party expenses deducted
      from, BRI are subject to an aggregate uniform
      percentage-of-revenues expense cap of nine and one-half percent
      (9.5\%) (see also Exhibit D hereto), with any such expenses
      disallowed to the extent they exceed that cap. Team and Related
      Party expenses that are deductible from BRI and subject to the
      nine and one-half percent (9.5\%) uniform expense cap shall
      include reasonable and customary expenses for the following
      categories, as referenced in the 2014-15 NBA Combined Financial
      Statements and Revenue Sharing Reference Manual: Ticket related
      expenses: 1002.15-.22, 1002.29-.34, 1012.27-.30; premium seating:
      1002.23-.28; suites: 1002.40-.46; naming rights: 1006.03-.09,
      1006.12-.18; and fixed signage: 1006.21-.27. For the avoidance of
      doubt, for the 2015-16 Salary Cap Year, the total Team and Related
      Party revenues and expenses that would have been subject to the
      new nine and one-half percent (9.5\%) expense cap were the amounts
      identified in the parties' letter agreement, dated January 19,
      2017. For the purposes of Article VII, Section 1,
      ``Playoff-Related Revenues and Expenses'' means the revenues and
      expenses reported in the ``playoff gate receipts, net'' amount
      shown in the Audit Report for the 2015-16 Salary Cap Year.
      Playoff-Related Revenues and Expenses are not subject to the above
      uniform expense cap. Such playoff-related expenses will continue
      to be deductible in accordance with the terms of the 2011 CBA as
      reflected in the Audit Report for the 2015-16 Salary Cap Year. For
      the avoidance of doubt, Taxes will be deducted from revenues
      included in BRI under Sections 1(a)(1)(i), (iii), (vi), (vii),
      (viii), (x), (xi), (xiii) and (xiv) (to the extent set forth in
      those subsections) before the application of the nine and one-half
      percent (9.5\%) ratio in calculating the uniform expense cap, and
      before the deduction of expenses.
    \end{enumerate}
  \item
    It is acknowledged by the parties hereto that for purposes of
    determining BRI:

    \begin{enumerate}
    \def\labelenumiii{(\roman{enumiii})}
    \tightlist
    \item
      Some NBA Teams have engaged or may engage in transactions with
      third parties that control, or own at least fifty percent (50\%)
      of, the NBA Team or that are controlled or owned at least fifty
      percent (50\%) by the persons or entities controlling or owning at
      least fifty percent (50\%) of the NBA Team (such third parties are
      referred to in this Agreement as a ``Related Party''), and Related
      Parties themselves engage in transactions with third parties that
      may result in a Related Party's receipt of revenues that
      constitute BRI. (Any entity that was an ``entity related to an NBA
      team'' as defined by Article VII, Section 1(a)(4)(i) of the
      September 18, 1995 Collective Bargaining Agreement between the NBA
      and the Players Association (the ``1995 CBA'') shall be deemed a
      Related Party under this Agreement for so long as such entity
      continues to be an entity related to an NBA Team within the
      meaning of the 1995 CBA.) As provided in Section 1(a)(1) above,
      the relevant proceeds received by any Related Party that come
      within such subsection and that relate to such Related Party's
      Team shall be included in BRI. However, except in connection with
      telecast agreements (which are subject to Section 1(a)(7)(ii)
      below), with respect to any such revenues or proceeds retained or
      received by a Related Party (other than arena revenues that relate
      to such Related Party's Team including, but not limited to,
      in-arena sales of novelties and concessions, NBA game parking,
      arena club revenues, suite and seat revenues and fixed and
      temporary in-arena signage, which shall be included in BRI as if
      received by the Team), or by a Team pursuant to a transaction with
      a Related Party, such revenues or proceeds shall be included in
      BRI only to the extent that the NBA and the Players Association
      agree or, if they fail to agree, the Accountants shall reasonably
      determine the amount, if any, of such revenues or proceeds to
      attribute to the Team (taking into account factors such as the
      nature of the transaction, arrangement and/or relationship between
      the Team and the Related Party or between the Related Party and a
      third party, any amounts included in BRI with respect to other
      Teams (or Related Parties) that have entered into comparable
      transactions, arrangements and/or relationships with third
      parties, market conditions, the nature of any services or
      activities performed by the Related Party for, or in connection
      with, the generation of revenues or proceeds and the amount of
      revenues or proceeds that the Related Party would be expected to
      retain or receive with respect to comparable transactions,
      arrangements and/or relationships with third parties), and the
      amount so attributed shall be the only amount included in BRI. To
      the extent that the amount of such proceeds to be included in BRI
      cannot reasonably be determined with respect to any particular
      transaction, the Accountants shall determine a reasonable amount
      with respect to such transaction, which shall be included in BRI.
      (In the event the Accountants refuse to make any such
      determination, such determination shall be made by a jointly
      selected expert with respect to any such transaction.) Without
      limiting the foregoing, in no event shall BRI include
      consideration paid to a Related Party in connection with rights
      acquired by such Related Party from a Team for fair market value,
      even if such consideration relates to NBA games or NBA-related
      activities (including, by way of example and not limitation,
      advertising revenue or subscriber fees earned by a Related Party
      television network that relate, directly or indirectly, to the
      telecast of NBA games licensed to the television network by a
      Team).
    \item
      In the event that, following the execution of this Agreement, a
      Team (other than the New York Knicks (``Knicks'')) enters into a
      local or regional telecast agreement with a Related Party, a copy
      of such agreement shall be provided to the Players Association
      within ten (10) days of approval of such agreement by the NBA. The
      Players Association and the NBA shall each have the right, not
      later than ten (10) days following the date on which the Players
      Association receives a copy of such agreement, to submit such
      agreement to a jointly-selected television valuation expert or (in
      the absence of such agreement) determined in accordance with the
      procedure set forth in this subsection (``TV Expert'') for the
      limited purpose set forth in this Section 1(a)(7)(ii). In the
      event that a party has so elected to submit such agreement to a TV
      Expert and the parties have not jointly selected a TV Expert
      within twenty (20) days following the date on which the Players
      Association receives a copy of such agreement, each party shall
      appoint its own television valuation designee and the two
      designees so appointed shall within ten (10) days of their
      appointment, jointly select a third party to serve as the TV
      Expert. Such TV Expert shall review such agreement to determine if
      the aggregate amount to be paid to the Team by the Related Party
      for the rights to telecast the Team's games pursuant to such
      agreement is more than fifteen percent (15\%) above or more than
      fifteen percent (15\%) below the fair market value of such rights
      over the term of such agreement. In making such determination, the
      TV Expert may take into account factors such as the nature of the
      transaction, arrangement and/or relationship between the Team and
      the Related Party, any amounts included in BRI with respect to
      other Teams (or Related Parties) that have entered into comparable
      transactions, arrangements and/or relationships with other
      programming licensors, market conditions, the nature of any
      services or activities performed by the Related Party for, or in
      connection with, the generation of revenues or proceeds and the
      amount of revenues or proceeds that the Related Party would be
      expected to retain or receive with respect to comparable
      transactions, arrangements and/or relationships with third
      parties; provided that in no event shall BRI include consideration
      paid to a Related Party in connection with rights acquired by such
      Related Party from a Team for fair market value, even if such
      consideration relates to NBA games or NBA-related activities
      (including, by way of example and not limitation, advertising
      revenue or subscriber fees earned by a Related Party television
      network that relate, directly or indirectly, to the telecast of
      NBA games licensed to the television network by a Team). In the
      event that the TV Expert determines that such aggregate amount is
      more than fifteen percent (15\%) above or below fair market value,
      the TV Expert shall be instructed to submit to the parties the
      amount for each Season of such agreement that he determines
      reflects the fair market value of such rights and such amounts,
      and no other amounts, shall be included in BRI with respect to
      such agreement for each Salary Cap Year covered by such agreement.
      Any determination made by the TV Expert pursuant to either of the
      preceding two sentences shall be submitted to the parties no later
      than twenty (20) days from the date on which such agreement was
      submitted to the TV Expert for his review. Any fees or costs
      associated with the retention or determination of the TV Expert
      shall be borne equally by the Players Association and NBA. The
      Players Association and the TV Expert shall maintain the
      confidentiality of any such agreement (and any determination made
      by the TV expert in accordance with this Section 1(a)(7)(ii))
      pursuant to the terms of Section 11(c) below relating to
      confidentiality of BRI Audits.
    \item
      With respect to the transactions listed below in this Section
      1(a)(7)(iii), the parties agree that, because the proceeds
      attributable to these transactions cannot be accurately
      ascertained, the following procedures shall be used for each NBA
      Season in which MSG Network is a Related Party of the Knicks (in
      the case of Section 1(a)(7)(iii)(A) below) and the Madison Square
      Garden arena is a Related Party of the Knicks (in the case of
      Section 1(a)(7)(iii)(B) below):

      \begin{enumerate}
      \def\labelenumiv{(\Alph{enumiv})}
      \tightlist
      \item
        New York Knicks transaction with MSG Network regarding the sale
        of local media rights: BRI for the Knicks for each NBA Season
        covered by this Agreement shall include an amount equal to the
        net proceeds included in BRI attributable to the Los Angeles
        Lakers' sale, license or other conveyance of all local media
        rights (including, but not limited to, broadcast and cable
        television and radio) for such NBA season.
      \item
        New York Knicks transactions with Related Parties involving
        signage: BRI for the Knicks for the 2015-2016 NBA Season shall
        include \$9,247,947 for signage. In each subsequent Season
        covered by this Agreement, this amount shall be increased (or
        decreased, as the case may be) by the League-wide percentage
        increase (or decrease) in signage as determined in accordance
        with Section 1(a)(1)(v) and (a)(1)(vi) above.\\
        At such time as the MSG Network and/or the Madison Square Garden
        Arena are no longer Related Parties, BRI for the New York Knicks
        in the categories described in Section 1(a)(7)(iii)(A) and/or
        (B) above, as the case may be, shall not be determined in
        accordance with the foregoing and will instead be determined by
        the applicable provisions of Section 1(a)(1) and (a)(7)(ii)
        above.
      \end{enumerate}
    \end{enumerate}
  \item
    In the event that, pursuant to the NBA's national broadcast,
    national telecast and network cable television agreements, NBA Teams
    receive revenue sharing proceeds that are attributable to NBA game
    telecasts in more than one Salary Cap Year, such proceeds shall be
    allocated over the same number of Salary Cap Years (beginning with
    first Salary Cap Year after the Salary Cap Year in which such
    proceeds are actually received) as the number of Salary Cap Years in
    which such games were televised. Any other contingent payments
    received by the NBA pursuant to such agreements shall be included in
    BRI to the extent and in a manner agreed upon by the parties, or, if
    the parties cannot agree, in a reasonable manner determined by the
    Accountants.
  \item
    The NBA and each NBA Team shall in good faith act and use their
    commercially reasonable efforts to increase BRI for each Salary Cap
    Year during the term of this Agreement. In the exercise of such
    commercially reasonable efforts, the NBA and each NBA Team shall be
    entitled to act in a manner consistent with their reasonable
    business judgment and shall not (i) take any action intended to
    benefit, at the expense of BRI, other commercial activities (such as
    the WNBA and the NBADL) unrelated to the performance of Players in
    NBA basketball games or in NBA-related activities, or (ii) shift or
    forgo revenues attributable to Salary Cap Years during the term of
    this Agreement in exchange for revenues or benefits during Salary
    Cap Years following the expiration of this Agreement (unless there
    is a reasonable business justification unrelated to collective
    bargaining for such shift or forgoing). There shall be no obligation
    on the part of the NBA or any NBA Team to accelerate into Salary Cap
    Years within the term of this Agreement revenues attributable to
    Salary Cap Years following the expiration of this Agreement. In
    evaluating compliance with this subsection, the parties and the
    System Arbitrator shall consider and give substantial weight to the
    reasonable business judgment of the NBA or the NBA Team but no
    deference will be applied where the NBA is alleged to have shifted
    or forgone revenues of \$350 million or more for the purpose of
    securing leverage in collective bargaining, in which case any
    finding of non-compliance shall require proof by a clear
    preponderance of the evidence. The following is a list of decisions
    in respect of which the business judgment of the NBA or an NBA Team
    shall conclusively be deemed reasonable: membership location; arena
    capacity or configuration; number and location of games played;
    whether to outsource or operate a line of business; and whether to
    accept or decline a sponsorship, advertising or naming rights
    opportunity. The foregoing list shall not limit in any manner the
    circumstances in which the business judgment of the NBA or an NBA
    Team may be deemed reasonable.
  \item
    The parties agree that upon a finding by the System Arbitrator
    (which, if appealed, is affirmed by the Appeals Panel) that the NBA
    or an NBA Team (or a Related Party) has willfully failed to provide
    to the Accountants information concerning revenues or expenses
    material to the Accountants' preparation of an Audit Report, and
    that such failure to provide information resulted in an
    understatement of BRI of more than \$4,232,345 with respect to the
    2017-18 Salary Cap Year (increasing by four and one-half percent
    (4.5\%) for each subsequent Salary Cap Year of this Agreement,
    beginning with the 2018-19 Salary Cap Year), then the amount by
    which BRI was understated shall be included in BRI in the Salary Cap
    Year in which such finding is made, with interest accruing from the
    date of the Audit Report for the Salary Cap Year in which such
    amount would have been included but for such understatement, with
    interest (at a rate equal to the one (1) year Treasury Bill rate as
    published in The Wall Street Journal on the date of the issuance of
    such Audit Report). In addition, if any Team, or if the NBA,
    violates the foregoing, it shall be fined \$3 million for its first
    violation during the term of this Agreement and an additional \$1.5
    million for each additional violation. (For example, if a Team
    violates the foregoing for the first time, it shall be fined \$3
    million; if such Team violates the foregoing a second time, it shall
    be fined \$4.5 million; and if such Team violates the foregoing a
    third time, it shall be fined \$6 million.) Fifty percent (50\%) of
    any such fine amounts shall be remitted by the NBA to an
    NBPA-Selected Charitable Organization (as defined in Article VI,
    Section 6 above) and fifty percent (50\%) shall be remitted by the
    NBA to a Section 501(c)(3) organization selected by the NBA.
  \item
    Neither the NBA or a League-related entity nor a Team or a Related
    Party will enter into any lease or other agreement providing for the
    receipt of revenues includable in BRI that contains provisions that
    purport to limit access of the Accountants to the books and records
    of the NBA, such League-related entity, such Team, or such Related
    Party in a manner inconsistent with the terms of this Agreement or
    that would preclude the calculation of revenues (if any) to be
    included in BRI pursuant to the provisions of Section 1(a)(1)(xii)
    above.
  \item
    Premium payments made by a Team for any insurance that, if paid,
    would be includable in BRI pursuant to Section 1(a)(2)(ix) above,
    shall be deducted from such Team's BRI for the Salary Cap Year in
    which any such insurance recovery is received.
  \item
    Equity Transactions.

    \begin{enumerate}
    \def\labelenumiii{(\roman{enumiii})}
    \tightlist
    \item
      The value of equity securities received by NBA-related entities
      (as defined in Section 1(a)(2) above) in entities that were not
      NBA-related entities prior to such receipt, to the extent
      otherwise constituting BRI under this Agreement, shall be included
      in BRI as follows:

      \begin{enumerate}
      \def\labelenumiv{(\Alph{enumiv})}
      \tightlist
      \item
        if the equity securities (including contingent securities, as
        defined below) are Publicly Tradable when received, the Publicly
        Traded Value of those securities will be included in BRI
        commencing in the Salary Cap Year in which they are received;
      \item
        if the equity securities consist of options, warrants,
        convertible securities or similar securities (``contingent
        securities''), and (x) those contingent securities are sold, the
        Net Proceeds will be included in BRI commencing in the Salary
        Cap Year in which the sale occurs, or (y) those contingent
        securities are exercised or converted into other securities that
        are or become Publicly Tradable, the Publicly Traded Value of
        the resulting securities (net of any exercise or conversion
        price and taxes, as determined below) will be included in BRI
        commencing in the Salary Cap Year in which the exercise or
        conversion occurs (if the resulting securities were Publicly
        Tradable at that time) or in the Salary Cap Year in which the
        resulting securities later become Publicly Tradable, whichever
        is first;
      \item
        if the equity securities (including contingent securities and
        any securities resulting from the exercise or conversion of
        contingent securities) are not Publicly Tradable at the time of
        receipt (or, in the case of contingent securities, at the time
        of exercise or conversion), no BRI value shall be attributable
        to such securities until they become Publicly Tradable or are
        sold or otherwise transferred for consideration other than
        securities that are not Publicly Tradable, whichever is first,
        at which time the Publicly Traded Value or Net Proceeds, as
        applicable, will be included in BRI commencing in the Salary Cap
        Year in which such event occurs; or
      \item
        notwithstanding the foregoing, if any contingent securities are
        exercisable or convertible into securities that are or become
        Publicly Tradable, but those contingent securities are not
        exercised or converted within one (1) year of any such right,
        the Players Association shall have the right, by written notice
        to the NBA, to have the Publicly Traded Value of such securities
        included in BRI as if those contingent securities had been
        exercised or converted on the date of such notice (net of any
        exercise or conversion price and taxes, as determined below).
      \end{enumerate}
    \item
      For purposes of this Section 1(a)(13), (A) ``Publicly Tradable''
      means (x) the applicable equity securities have been registered
      for sale under applicable state, federal and foreign laws, are
      listed and tradable on a generally recognized stock exchange or in
      the over-the-counter market, or (y) the applicable equity
      securities can be readily purchased and sold on a nationally
      recognized secondary market (e.g., without limitation on any
      example, shares in ``Facebook'' as of the date of the 2011 CBA),
      and in each case under (x) and (y), any contractual or other
      prohibition or limitation on sale would not preclude a sale; (B)
      ``Publicly Traded Value'' means the weighted average daily trading
      price of the applicable equity securities for the thirty (30)
      trading days (x) preceding the date of receipt if the securities
      are Publicly Tradable prior to that date or (y) following the date
      they become Publicly Tradable; provided that if such equity
      securities are sold during the Salary Cap Year in which their
      Publicly Traded Value is first included in BRI, the ``Publicly
      Traded Value'' of such equity securities shall be the Net Proceeds
      from such sale; (C) ``Net Proceeds'' means the proceeds received
      by the selling entity from the applicable sale, net of commissions
      and reasonable expenses relating to such sale, any exercise or
      conversion price with respect to securities resulting from the
      exercise or conversion of contingent securities, and any
      applicable taxes of the selling entity (or if the selling entity
      is a pass-through entity for income tax purposes, such entity's
      owners), which shall be determined using a tax rate equivalent to
      the highest marginal combined federal, state and local tax rate
      that would be applicable in the locale where the principal place
      of business of the selling entity is located, which in the case of
      a Team or a Related Party of a Team shall be deemed to be the
      locale of the arena in which the Team plays more than one-half of
      its Regular Season home games; and (D) a sale of equity securities
      shall not be subject to inclusion in BRI if the sale is part of a
      larger transaction in which (x) BRI has been fully accounted for
      or (y) all or substantially all of the assets of an NBA-related
      entity or business unit thereof are sold and such equity
      securities do not represent a majority of the value in such
      transaction. In all cases, the Publicly Traded Value of, or Net
      Proceeds from, the applicable equity securities will be included
      in BRI over a seven (7) year amortization period (inclusive of the
      Salary Cap Year in which such Publicly Traded Value is first
      included in BRI), even if such equity securities are sold during
      such seven (7) year period.
    \item
      For the avoidance of doubt, (A) in no event shall the value of, or
      proceeds or distributions from, equity securities in NBA-related
      entities be included in BRI, and (B) the value of, or proceeds or
      distributions from, equity securities in non-NBA-related entities
      shall be included in BRI exclusively pursuant to this Section
      1(a)(13), and only once under the applicable provision of Section
      1(a)(13)(i) above.
    \end{enumerate}
  \item
    International Development and Operations Expenses.

    \begin{enumerate}
    \def\labelenumiii{(\roman{enumiii})}
    \tightlist
    \item
      The NBA and League-related entities may deduct from BRI expenses
      related to the development and operation of the League's
      international business (``Newly-Deductible International
      Expenses''), subject to a limit of ten percent (10\%) of the
      League's gross BRI international revenues (the allowed amount of
      such expense following application of the ten percent (10\%) cap
      being the ``Allowed Newly-Deductible International Expenses'').
      Newly-Deductible International Expenses for any Salary Cap Year
      shall include all such international expenses incurred at the
      League level that are not otherwise deductible under this
      Agreement (excluding expenses in currently-deductible categories
      that are in excess of applicable percentage-of-revenue expense
      caps and the write-down of equity investments). For the purposes
      of this Agreement, the League's projected gross BRI international
      revenue and Newly-Deductible International Expenses for the
      2015-16 Salary Cap Year shall be deemed to be the amounts set
      forth in the parties' letter agreement dated January 19, 2017.
    \item
      For each of the 2017-18 through the 2020-21 Salary Cap Years, if
      (x) the sum of (i) ten percent (10\%) of the gross proceeds from
      fixed arena signage within or outside of the arena in which an NBA
      Team plays more than one-half of its Regular Season home games
      (excluding from such calculation any such gross proceeds that are
      subject to the twenty-five percent (25\%), rather than the fifty
      percent (50\%) ratio), and (ii) ten percent (10\%) of the gross
      proceeds from the sale, lease, or licensing of luxury suites for
      that Salary Cap Year, as those terms are used in Sections
      1(a)(1)(vi) and (vii), respectively (the ``Incremental
      Suites/Fixed Arena Signage Revenue''), minus (y) the Allowed
      Newly-Deductible International Expenses for that Salary Cap Year,
      is less than \$24 million, then the amount by which \$24 million
      exceeds that number shall be added to BRI for that Salary Cap
      Year.
    \item
      For the 2021-22 Salary Cap Year, if (x) the Allowed
      Newly-Deductible International Expenses for such Salary Cap Year,
      minus (y) the Incremental Suites/Fixed Arena Signage Revenue for
      such Salary Cap Year is greater than \$48 million, then the amount
      by which that number exceeds \$48 million shall be added to BRI
      for that Salary Cap Year.
    \item
      For the avoidance of doubt, for any Salary Cap Year after the
      2021-22 Salary Cap Year, no adjustments pursuant to the foregoing
      Sections 1(a)(14)(ii) and (iii) shall be made.
    \end{enumerate}
  \item
    Miscellaneous BRI Accounting Rules.

    \begin{enumerate}
    \def\labelenumiii{(\roman{enumiii})}
    \tightlist
    \item
      Team charter travel expenses for Regular Season games, associated
      with broadcast and other personnel for whom such travel expenses
      are otherwise deductible (for example, without limitation on any
      other example, Team personnel traveling in connection with the
      sale of Team sponsorships), shall be deductible at fifty percent
      (50\%).
    \item
      BRI for premium seating, in respect of: (a) bunker, super, and
      party suites, (b) theatre boxes, loge boxes, and other such
      non-traditional premium seating inventory, and (c) traditional
      club seats, shall be calculated by using the parties'
      previously-agreed upon methods, as reflected in the Audit Report
      for the 2015-16 Salary Cap Year.
    \item
      To the extent salary paid to a person who also owns an interest in
      the Team would otherwise be deductible from BRI, such salary shall
      only be deductible for BRI purposes only if all of the following
      criteria are met and if the expense otherwise qualifies for such
      deduction (for example, without limitation, the salary is related
      to the BRI against which it is deducted): (i) the owner owns less
      than seven and one-half percent (7.5\%) of the Team, (ii) the job
      being performed by the owner would otherwise be performed by a
      non-owner staff member, (iii) the job being performed by the owner
      is his/her full time job and he/she has no other roles with
      outside companies (with the exception of limited duty Board
      roles), (iv) the salary being earned is reasonable and customary,
      relative to what a non-owner staff member would earn, for the
      services being provided, and (v) there are no other individuals
      performing substantially the same role employed by the Team where
      the role is such that ordinarily there is only one person
      performing it (for example, without limitation, Team president).
    \item
      With respect to expenses associated with League-related entity
      advertising and public relations campaigns: (i) the expenses will
      be allocated to BRI and non-BRI revenue categories according to
      the methodology agreed to by the parties in connection with the
      final Audit Report for the 2015-16 Salary Cap Year, except that
      such expenses shall also be allocated to the NBA's Regular Season
      gate assessment during the Salary Cap Year in addition to the
      other categories previously included in the allocation; and (ii)
      the expenses shall be deducted from BRI, subject to Section
      1(a)(6)(ii) above.
    \end{enumerate}
  \end{enumerate}
\item
  \textbf{Accounting Methods/Lump Sum Payments.}

  \begin{enumerate}
  \def\labelenumii{(\arabic{enumii})}
  \tightlist
  \item
    Subject to Sections 1(b)(2) and (b)(3) below, and any provision
    hereof that expressly provides for an alternative accounting
    treatment, BRI for each Salary Cap Year shall be calculated
    exclusively pursuant to the accrual method of financial accounting
    (and not, for any purpose, the cash method of financial accounting)
    and in accordance with United States Generally Accepted Accounting
    Principles. By way of example, and not limitation, in the event a
    team receives a signing bonus in consideration for its agreement to
    enter into a five (5) year contract for the local telecast of its
    games, such signing bonus shall be amortized in equal annual amounts
    over the five (5) Salary Cap Years covered by such television
    contract.
  \item
    Except as otherwise provided in the case of luxury suites and
    premium seat licenses, in no event shall the amortization period for
    any lump sum payment exceed seven (7) years.
  \item
    Any payments that constitute BRI and that are subject to being
    repaid to the payor under certain circumstances (the
    ``Contingencies'') shall constitute BRI in the Salary Cap Year in
    which such payments would have been earned but for the Contingencies
    unless, at the time of such payments, the Contingencies under which
    the payments would be repaid are likely to occur, in which case the
    payments will not be included in BRI unless and until such time as
    the Contingencies under which such repayments would be made do not
    occur or are not likely to occur. In the event that a payment that
    has been included in BRI is subsequently repaid, BRI shall be
    reduced by the amount of such repayment in the Salary Cap Year in
    which such repayment is made. In any proceeding commenced before the
    System Arbitrator relating to the terms of this Section 1(b)(3), the
    NBA will bear the burden of demonstrating that the applicable
    Contingencies are likely to occur.

    \begin{enumerate}
    \def\labelenumiii{(\roman{enumiii})}
    \tightlist
    \item
      With respect to lump sum payments (e.g., signing bonuses) that
      constitute BRI and are received by a Team or Related Party under
      agreements entered into by a Team or Related Party after June 30,
      2017, for the period, if any, between (x) the date when the lump
      sum payment is received and (y) the beginning of the Salary Cap
      Year when the amortization period for the lump sum payment begins
      pursuant to Section 1(b)(1) above, BRI shall include in each
      applicable Salary Cap Year during this period imputed interest on
      the amount of such lump sum payment at a rate equal to the one (1)
      year Treasury Bill rate as published in The Wall Street Journal on
      the date the payment was received; provided, however, that such
      imputed interest shall only be calculated and included in BRI if
      each of the following is satisfied: (i) the lump sum payment
      received by the Team or Related Party is in the amount of one
      million dollars (\$1,000,000) or more; (ii) the lump sum payment
      is received by the Team or Related Party at least twelve (12)
      months before the start of the Salary Cap Year in which it will
      first be included in BRI under Section 1(b)(1) above; and (iii)
      the lump sum payment is not related to a ticket, luxury suite or
      seat licensing transaction including, without limitation, revenues
      included in BRI under Sections 1(a)(1)(i), 1(a)(1)(vii), and
      1(a)(1)(x) above.
    \end{enumerate}
  \end{enumerate}
\item
  ``Projected BRI'' for a Salary Cap Year means the amount determined as
  follows: Prior to the start of each Salary Cap Year, the NBA and the
  Players Association shall meet for the purpose of agreeing upon
  Projected BRI for that Salary Cap Year. In the absence of an agreement
  of the parties otherwise on or prior to the last day of the Moratorium
  Period of the applicable Salary Cap Year, Projected BRI for such
  Salary Cap Year shall be the sum of amounts determined in accordance
  with the following:

  \begin{enumerate}
  \def\labelenumii{(\arabic{enumii})}
  \tightlist
  \item
    With respect to BRI sources other than national broadcast, national
    telecast or network cable television contracts, Projected BRI shall
    include BRI for the preceding Salary Cap Year, increased by four and
    one-half percent (4.5\%). For purposes of this Section 1(c)(1), a
    contract between or among any League-related entities and/or Teams
    shall not be considered national broadcast, national telecast or
    network cable television contracts.
  \item
    With respect to national broadcast, national telecast or network
    cable television contracts including the NBA/ABC agreement dated
    October 3, 2014 (``NBA/ABC Agreement'') (a copy of which has been
    provided to the Players Association) and the NBA/TBS agreement,
    dated October 3, 2014 (``NBA/TBS Agreement'') (a copy of which has
    been provided to the Players Association), and national broadcast,
    national telecast or network cable television contracts covering
    Seasons that succeed the Seasons covered by the NBA/ABC and NBA/TBS
    Agreements (``Successor Agreements'') (copies of which shall be
    provided to the Players Association within ten (10) days of
    execution), Projected BRI for a Salary Cap Year shall include (i)
    the rights fees or other non-contingent payments stated in such
    contracts with respect to the Season covered by such Salary Cap Year
    (as such rights fees or non-contingent payments may be adjusted by
    agreement of the parties to such contracts); (ii) the amounts of
    revenue sharing proceeds, if any, that are includable in BRI for
    such Salary Cap Year pursuant to Section 1(a)(8) above; (iii) the
    amounts with respect to contingent payments (other than revenue
    sharing proceeds), if any, attributable to Salary Cap Years covered
    by this Agreement in Successor Agreements as such amounts are agreed
    upon by the parties, or if the parties do not reach agreement, by
    the Accountants; and (iv) the amount included in BRI for the
    preceding Salary Cap Year with respect to the value of advertising
    or promotional time provided to the NBA as part of the NBA/ABC and
    NBA/TBS Agreements (or any Successor Agreements) that is used to
    promote the WNBA or for any purpose other than those listed in
    Section 1(a)(1)(ii)(A)-(D).
  \end{enumerate}
\item
  ``Local Expansion Team BRI'' means the BRI of the Expansion Teams
  during their first two (2) Seasons, but not including the Expansion
  Teams' share of League-wide revenues that are otherwise included in
  BRI (including, but not limited to, their share of national
  television, cable, radio and other broadcast revenues).
\item
  ``Projected Local Expansion Team BRI'' means Local Expansion Team BRI
  for the immediately preceding Season, increased by four and one-half
  percent (4.5\%).
\item
  ``Interim Projected BRI'' means a projection of BRI for a Salary Cap
  Year using Estimated BRI in place of BRI for the previous Salary Cap
  Year.
\item
  ``Barter'' means to trade by exchanging one commodity, service or
  other non-cash item for another.
\item
  ``Estimated Total Benefits'' means the estimate of Total Benefits for
  a Salary Cap Year as set forth in the Interim Audit Report (as defined
  in Section 10(a) below) for such Salary Cap Year.
\item
  ``Estimated Total Salaries'' means the estimate of Total Salaries for
  a Salary Cap Year as set forth in the Interim Audit Report for such
  Salary Cap Year.
\item
  ``Estimated Total Salaries and Benefits'' means the sum of Estimated
  Total Benefits and Estimated Total Salaries for a Salary Cap Year as
  set forth in the Interim Audit Report for such Salary Cap Year.
\item
  ``Estimated BRI'' means the estimate of BRI for a Salary Cap Year as
  set forth in the Interim Audit Report for such Salary Cap Year.
\end{enumerate}

\section{Calculation of Salary Cap and Minimum Team
Salary.}\label{calculation-of-salary-cap-and-minimum-team-salary.}

\begin{enumerate}
\def\labelenumi{(\alph{enumi})}
\tightlist
\item
  \textbf{Salary Cap.}

  \begin{enumerate}
  \def\labelenumii{(\arabic{enumii})}
  \tightlist
  \item
    For each Salary Cap Year during the term of this Agreement, there
    shall be a Salary Cap. Subject to the adjustments set forth in
    Section 2(d) below, the Salary Cap for each Salary Cap Year covered
    by the Term of this Agreement will equal forty-four and seventy-four
    one hundredths percent (44.74\%) of Projected BRI for such Salary
    Cap Year, less Projected Benefits for such Salary Cap Year, divided
    by the number of Teams scheduled to play in the NBA during such
    Salary Cap Year, other than Expansion Teams during their first two
    (2) Salary Cap Years in the NBA.
  \item
    Notwithstanding Section 2(a)(1) above, in the event that, subject to
    the adjustments set forth in Section 2(d) below, Projected BRI for
    any Salary Cap Year in which one or more Expansion Teams is
    scheduled to play its second Season, plus Projected Local Expansion
    Team BRI for such Salary Cap Year, multiplied by the applicable
    percentage of Projected BRI set forth in Section 2(a)(1) above, less
    Projected Benefits for such Salary Cap Year (including for the
    Expansion Team(s)), divided by the number of Teams scheduled to play
    in the NBA during such Salary Cap Year (including the Expansion
    Team(s)), exceeds the Salary Cap calculated in accordance with
    Section 2(a)(1) above, the Salary Cap shall equal the amount
    calculated pursuant to this Section 2(a)(2).
  \item
    The Salary Cap for a Salary Cap Year will be in effect commencing on
    the first day of the Salary Cap Year (i.e., July 1) and shall
    continue through and including the last day of the Salary Cap Year
    (i.e., the following June 30).
  \item
    In the event that the Audit Report for a Salary Cap Year has not
    been completed as of the last day of such Salary Cap Year, and the
    NBA and the Players Association have not reached an agreement on
    Projected BRI and Projected Benefits pursuant to Article VII,
    Section 1 and Article IV, Section 9 for the immediately following
    Salary Cap Year, then the Salary Cap for the immediately following
    Salary Cap Year will be calculated pursuant to Section 2(a)(1)-(2)
    above, except that Interim Projected BRI shall be utilized instead
    of Projected BRI, Estimated BRI shall be utilized instead of BRI and
    Estimated Total Salaries and Benefits shall be utilized instead of
    Total Salaries and Benefits, for all purposes under this Section 2
    including, without limitation, the adjustments set forth in Section
    2(d) below. In the event that the Interim Audit Report for a Salary
    Cap Year has not been completed as of the last day of such Salary
    Cap Year, and the NBA and Players Association have not reached
    agreement on Projected BRI and Projected Benefits pursuant to
    Article VII, Section 1 and Article IV, Section 9, then the Salary
    Cap for the immediately following Salary Cap Year shall, until such
    Interim Audit Report is completed, be an amount that would have been
    the Salary Cap for the preceding Salary Cap Year had Projected BRI
    or Interim Projected BRI, as the case may be, for such preceding
    Salary Cap Year included, with respect to the NBA's national
    broadcast, national telecast or network cable television contracts,
    the rights fees or other non-contingent payments stated in such
    contracts for the Season following the Season covered by such
    preceding Salary Cap Year instead of for the Season covered by such
    preceding Salary Cap Year.
  \end{enumerate}
\item
  \textbf{Minimum Team Salary.}

  \begin{enumerate}
  \def\labelenumii{(\arabic{enumii})}
  \tightlist
  \item
    For each Salary Cap Year covered by the term of this Agreement,
    there shall be a Minimum Team Salary equal to ninety percent (90\%)
    of the Salary Cap for such Salary Cap Year.
  \item
    In the event that a Team's Team Salary for a Salary Cap Year as of
    the start of the Team's last Regular Season game of that Salary Cap
    Year is less than the applicable Minimum Team Salary for that Salary
    Cap Year, the NBA shall cause such Team to make payments to the
    players who were on the Team's roster during the Regular Season
    covered by such Salary Cap Year equal to the shortfall (to be
    disbursed to such players pro rata or in accordance with such other
    formula and roster requirements as may be reasonably determined by
    the Players Association). The Players Association shall provide the
    NBA with its proposed per-player distribution of any such shortfall
    within thirty (30) days after the completion of the Audit Report for
    such Salary Cap Year. The NBA shall cause the Team to make the
    required payments, less all amounts required to be withheld by any
    governmental authority, within thirty (30) days after receipt of the
    proposed distribution from the Players Association in accordance
    with the preceding sentence.
  \item
    Nothing contained herein shall preclude a Team from having a Team
    Salary in excess of the Minimum Team Salary, provided that the
    Team's Team Salary does not exceed the Salary Cap plus any
    additional amounts authorized pursuant to the Exceptions set forth
    in this Article VII.
  \item
    For purposes of determining whether a Team has met its Minimum Team
    Salary obligation for a Salary Cap Year in accordance with this
    Section 2(b), the Team's Team Salary shall:

    \begin{enumerate}
    \def\labelenumiii{(\roman{enumiii})}
    \tightlist
    \item
      be calculated in the same manner as Team Salary is calculated by
      the Accountants for purposes of computing Total Salaries and
      Benefits in the Audit Report (as defined in Section 10(a)(1)
      below), except that with respect to each player that was employed
      by more than one (1) Team under the same Player Contract (i.e., in
      cases where a player's Contract is acquired by trade or pursuant
      to the NBA waiver procedure) during the Salary Cap Year, the
      player's Salary in respect of such Player Contract for the
      applicable Salary Cap Year shall be allocated as follows: (Y) the
      amount to be included in Team Salary for each such Team other than
      the last Team by which the player was employed under the same
      Player Contract shall equal the Salary in respect of the Player
      Contract for the applicable Salary Cap Year multiplied by a
      fraction, the numerator of which is the total number of days of
      the Regular Season that the player was on the roster of such Team
      and the denominator of which is the total number of days in such
      Regular Season (or, if the Contract was not in effect for the
      entire Regular Season, the number of days in such Regular Season
      during which the Contract was in effect); and (Z) the amount to be
      included in Team Salary for the last such Team by which the player
      was employed under the same Player Contract shall equal the
      player's Salary in respect of the Player Contract less the amount
      of such Salary allocated to other Team(s) in accordance with
      clause (Y) above. For purposes of the foregoing calculation, any
      reduction in a player's Salary for the applicable Salary Cap Year
      resulting from the termination of the player's Contract shall be
      allocated to the last Team by which the player was employed under
      such Contract (after first performing the allocations described
      above before giving effect to such Salary reduction);
    \item
      include any Salary in respect of such Salary Cap Year that is
      excluded from the Team's Team Salary pursuant to Section 4(h)
      below; and
    \item
      exclude any Salary in respect of such Salary Cap Year that is
      included in the Team's Team Salary pursuant to Section 3(e) below.
    \end{enumerate}
  \item
    Notwithstanding any other provision of this Agreement, if the sum of
    (i) the amount of a proposed payment distribution to a player for a
    Salary Cap Year pursuant to this Section 2(b), and (ii) the player's
    Salary for such Salary Cap Year would exceed the player's applicable
    Maximum Annual Salary for such Salary Cap Year, then such player's
    proposed payment distribution for such Salary Cap Year pursuant to
    this Section 2(b) shall be reduced to the extent necessary for the
    sum of the player's Salary and such distribution to equal his
    applicable Maximum Annual Salary for such Salary Cap Year, and the
    amount of such reduction shall be disbursed to the other players on
    the Team pro rata based upon each such player's payment distribution
    amount under this Section 2(b) prior to any increases to such amount
    in accordance with this Section 2(b)(5). For purposes of this
    Section 2(b)(5): (i) a player's Salary shall be calculated in the
    same manner as it is calculated by the Accountants for purposes of
    computing Total Salaries and Benefits in the Audit Report (as
    defined in Section 10(a)(1) below); and (ii) a player's Maximum
    Annual Salary shall be deemed to be the amount calculated pursuant
    to Article II, Section 7(f)(i), (ii) or (iii) (as applicable)
    substituting references in such subsections to the ``Salary Cap in
    effect at the time the trade bonus is earned'' and ``the Season
    prior to the Season in which the trade bonus is earned'' with
    ``Salary Cap in effect for the Season in which the Minimum Team
    Salary shortfall is incurred'' and ``the Season prior to the Season
    in which the shortfall is incurred,'' respectively.
  \item
    In accordance with Article IV, Section 6(d), the amount paid by
    Teams in respect of the employer's portion of payroll taxes on
    payments made to players pursuant to this Section 2(b) shall be
    included in Benefits in the Salary Cap Year in which such payments
    are made.
  \end{enumerate}
\item
  \textbf{Expansion Team Salary Caps and Minimum Team Salaries.}
  Expansion Teams shall have the same Salary Caps and Minimum Team
  Salaries as all other Teams, except as follows: (1) During the first
  Salary Cap Year in which it begins play, an Expansion Team shall have
  a Salary Cap equal to sixty-six and two-thirds percent (66-2/3\%) of
  the Salary Cap applicable to all other Teams (the ``First Year
  Expansion Team Salary Cap''); and shall have a Minimum Team Salary
  equal to ninety percent (90\%) of the First Year Expansion Team Salary
  Cap. (2) During the second Salary Cap Year in which it begins play, an
  Expansion Team shall have a Salary Cap equal to eighty percent (80\%)
  of the Salary Cap applicable to all other Teams (the ``Second Year
  Expansion Team Salary Cap''); and shall have a Minimum Team Salary
  equal to ninety percent (90\%) of the Second Year Expansion Team
  Salary Cap.
\item
  \textbf{Adjustments to Salary Cap and Minimum Team Salary.}

  \begin{enumerate}
  \def\labelenumii{(\arabic{enumii})}
  \tightlist
  \item
    In the event that Total Salaries and Benefits for any Salary Cap
    Year are less than the Designated Share (as defined in Section
    12(b)(3) below) for such Salary Cap Year, then the Salary Cap for
    the subsequent Salary Cap Year shall be increased by the amount of
    the shortfall divided by the number of Teams in the NBA during such
    subsequent Salary Cap Year (other than Expansion Teams in their
    first two (2) Salary Cap Years in the NBA).
  \item
    In the event that there is an Overage (as defined in Section
    12(a)(14) below) in any Salary Cap Year, then the Salary Cap for the
    subsequent Salary Cap Year shall be adjusted as follows:

    \begin{enumerate}
    \def\labelenumiii{(\roman{enumiii})}
    \tightlist
    \item
      If the amount of the Overage in the Salary Cap Year is equal to or
      less than six percent (6\%) of Total Salaries and Benefits, then
      no adjustments shall be made to the Salary Cap for the subsequent
      Salary Cap Year.
    \item
      If the amount of the Overage in the Salary Cap Year equals more
      than six percent (6\%) of Total Salaries and Benefits, then the
      Salary Cap for the subsequent Salary Cap Year shall be reduced by
      an amount to be calculated as follows:\\
      STEP 1: Subtract six percent (6\%) of Total Salaries and Benefits
      from the Overage.\\
      STEP 2: If Projected BRI for the subsequent Salary Cap Year does
      not exceed BRI for the Salary Cap Year by more than eight percent
      (8\%) of BRI for the Salary Cap Year or the amount of the Overage
      described above exceeds nine percent (9\%) of Total Salaries and
      Benefits, then divide the result of Step 1 by the number of Teams
      in the NBA during the subsequent Salary Cap Year (other than
      Expansion Teams in their first two (2) Salary Cap Years in the
      NBA). The result of this calculation is the amount of the
      reduction in the Salary Cap for such subsequent Salary Cap Year,
      and no further steps are required. If Projected BRI for the
      subsequent Salary Cap Year exceeds BRI for the Salary Cap Year by
      more than eight percent (8\%) of BRI for the Salary Cap Year and
      the amount of the Overage described above does not exceed nine
      percent (9\%) of Total Salaries and Benefits, then proceed to Step
      3.\\
      STEP 3: Subtract BRI for the Salary Cap Year from Projected BRI
      for the subsequent Salary Cap Year.\\
      STEP 4: Subtract eight percent (8\%) of BRI for the Salary Cap
      Year from the result of Step 3.\\
      STEP 5: Multiply the result of Step 4 by forty-four and
      seventy-four one hundredths percent (44.74\%).\\
      STEP 6: Subtract the result of Step 5 from the result of Step 1.
      If the result of this step is less than zero, then no adjustments
      shall be made to the Salary Cap for the subsequent Salary Cap
      Year, and no further steps are required.\\
      STEP 7: Divide the result of Step 6 by the number of Teams in the
      NBA during such subsequent Salary Cap Year (other than Expansion
      Teams in their first two (2) Salary Cap Years in the NBA). The
      result of this calculation is the amount of the reduction in the
      Salary Cap for such subsequent Salary Cap Year.\\
      \emph{Example: Assume Total Salaries and Benefits for the 2013-14
      Season are \$2.27 billion and the Designated Share is \$2.1
      billion resulting in an Overage of \$170 million (which equals
      7.5\% of Total Salaries and Benefits). Assume Projected BRI for
      the 2014-15 Season (\$4.41 billion) exceeds actual BRI for the
      2013-14 Season (\$4.2 billion) by 5\%. Assume there are 30 Teams
      in the NBA in 2014-15. The Salary Cap for the 2014-15 Season would
      be reduced by \$1.1 million ((7.5\% of Total Salaries and Benefits
      (\$170 million) less 6\% of Total Salaries and Benefits (\$136
      million)) divided by 30).}\\
      \emph{Example: Same Total Salaries and Benefits as in the prior
      example, except assume Projected BRI for the 2014-15 Season
      (\$4.578 billion) exceeds actual BRI for the 2013-14 Season (\$4.2
      billion) by 9\%. The Salary Cap for the 2014-15 Season would be
      reduced by \$0.50 million (\$34 million (the Overage of 7.5\% of
      Total Salaries and Benefits (\$170 million) less 6\% of Total
      Salaries and Benefits (\$136 million)), less \$19 million (i.e.,
      the difference between (x) Projected BRI for the 2014-15 Season
      (\$4.578 billion) less actual BRI for the 2013-14 Season (\$4.2
      billion) (i.e., \$378 million) and (y) 8\% of actual BRI for the
      2013-14 Season (i.e., \$336 million) -- a difference of \$42
      million -- multiplied by 44.74\%), divided by 30).}
    \end{enumerate}
  \item
    In the event that the Salary Cap for a Salary Cap Year is calculated
    based upon an Interim Audit Report for the prior Salary Cap Year in
    accordance with Section 2(a)(4) above and BRI and Total Salaries and
    Benefits as set forth in the Audit Report for the prior Salary Cap
    Year are different from those in the Interim Audit Report such that
    the Salary Cap would have been different from that based upon the
    Interim Audit Report, any such difference in the Salary Cap shall be
    debited or credited, as the case may be, to the Salary Cap for the
    subsequent Salary Cap Year, except that, with respect to the 2023-24
    Salary Cap Year (or, in the alternative, if either the NBA or
    Players Association exercises its option to terminate this Agreement
    pursuant to Article XXXIX, the 2022-23 Salary Cap Year) any such
    differences shall be debited or credited, as the case may be, to the
    Salary Cap for the then current Salary Cap Year, in all such cases
    with interest (at a rate equal to the one (1) year Treasury Bill
    rate as published in The Wall Street Journal on the date of the
    issuance of the Interim Audit Report).
  \end{enumerate}
\item
  \textbf{Guarantee.}

  \begin{enumerate}
  \def\labelenumii{(\arabic{enumii})}
  \tightlist
  \item
    In the event that for any Salary Cap Year Total Salaries and
    Benefits is less than the Designated Share (as defined in Section
    12(b)(3) below) for such Salary Cap Year, the NBA shall be obligated
    to pay the difference (a ``Guarantee Shortfall'') to all NBA players
    who were on an NBA roster during the Regular Season covered by such
    Salary Cap Year. Any such Guarantee Shortfall obligation shall be
    effectuated and satisfied solely by the NBA paying such Guarantee
    Shortfall to Teams no later than sixty (60) days following the
    completion of the Audit Report for such Salary Cap Year and causing
    the Teams to distribute the Guarantee Shortfall to such players on
    such proportional basis as may be reasonably determined by the
    Players Association, less all amounts required to be withheld by any
    governmental authority. The Players Association shall provide the
    NBA with its proposed per-player distribution of any such Guarantee
    Shortfall for a Salary Cap Year within thirty (30) days after the
    completion of the Audit Report for such Salary Cap Year.
  \item
    In accordance with Article IV, Section 6(d) above, the amount paid
    by Teams in respect of the employer's portion of payroll taxes on
    Guarantee Shortfall payments made to players pursuant to this
    Section 2(e) shall be included in Benefits in the Salary Cap Year in
    such payments are made.
  \end{enumerate}
\end{enumerate}

\section{Determination of Salary.}\label{determination-of-salary.}

For the purposes of determining a player's Salary with respect to a
Salary Cap Year, the following rules shall apply:

\begin{enumerate}
\def\labelenumi{(\alph{enumi})}
\tightlist
\item
  \textbf{Deferred Compensation.}

  \begin{enumerate}
  \def\labelenumii{(\arabic{enumii})}
  \tightlist
  \item
    General Rule: All Player Contracts entered into, extended or
    renegotiated after the date of this Agreement shall specify the
    Season(s) in which any Deferred Compensation is earned. Deferred
    Compensation shall be included in a player's Salary for the Salary
    Cap Year encompassing the Season in which such Deferred Compensation
    is earned. (2) Over 38 Rule: The following provisions shall apply to
    any Player Contract entered into, extended, or renegotiated that,
    beginning with the date such Contract, Extension or Renegotiation is
    signed, covers four (4) or more Seasons, including one (1) or more
    Seasons commencing after such player will reach or has reached age
    thirty-eight (38) (an ``Over 38 Contract''):

    \begin{enumerate}
    \def\labelenumiii{(\roman{enumiii})}
    \tightlist
    \item
      Except as provided in Section 3(a)(2)(ii)-(iii) below, the
      aggregate Salaries in an Over 38 Contract for Salary Cap Years
      commencing with the fourth Salary Cap Year of such Over 38
      Contract or the first Salary Cap Year that covers a Season that
      follows the player's 38th birthday, whichever is later, shall be
      attributed to the prior Salary Cap Years pro rata on the basis of
      the Salaries for such prior Salary Cap Years.
    \item
      If a Qualifying Veteran Free Agent who is age 35 or 36 enters into
      an Over 38 Contract with his Prior Team covering five (5) Seasons,
      the Salary in such Over 38 Contract for the fifth Salary Cap Year
      shall be attributed to the prior Salary Cap Years pro rata on the
      basis of the Salaries for such prior Salary Cap Years. For
      purposes of this Section 3(a)(2)(ii), a Qualifying Veteran Free
      Agent who (x) enters into an Over 38 Contract with his Prior Team
      prior to October 1 of a Salary Cap Year, (y) is age 34 at the time
      he enters into the Contract, and (z) will turn age 35 on or before
      such October 1 shall be deemed to be 35 at the time he enters into
      such Over 38 Contract.
    \item
      For each Salary Cap Year of an Over 38 Contract beginning with the
      second Salary Cap Year prior to the First Zero Year (as defined in
      Section 3(a)(2)(vi) below), if the player's Contract has not been
      terminated as of the July 1 of such Salary Cap Year, then the
      Salaries of the player for such Salary Cap Year and the subsequent
      two (2) or fewer Salary Cap Years covered by the Contract
      (including any Zero Year (as defined in Section 3(a)(2)(vi)
      below)) shall, on such July 1, be aggregated and attributed in
      equal shares to each of such three (3) or fewer Salary Cap Years.
    \item
      Notwithstanding Section 3(a)(2)(i) above, there shall be no
      re-allocation of Salaries pursuant to this Section 3(a)(2) for any
      Contract between a Qualifying Veteran Free Agent and his Prior
      Team covering four (4) or fewer Seasons entered into by a player
      at age 35 or 36. For purposes of this Section 3(a)(2)(iv), a
      Qualifying Veteran Free Agent who (x) enters into an Over 38
      Contract with his Prior Team prior to October 1 of a Salary Cap
      Year, (y) is age 34 at the time he enters into the Contract, and
      (z) will turn age 35 on or before such October 1 shall be deemed
      to be 35 at the time he enters into such Over 38 Contract.
    \item
      For purposes of determining whether a Contract is an Over 38
      Contract pursuant to this Section 3(a)(2) only, Seasons shall be
      deemed to commence on October 1 and conclude on the last day of
      the Salary Cap Year.
    \item
      ``Zero Year'' means, with respect to an Over 38 Contract, any
      Salary Cap Year in which the Salary called for under the Contract
      has been attributed, in accordance with Section 3(a)(2)(i), (ii)
      or (iii) above, to prior Salary Cap Years of the Contract. ``First
      Zero Year'' means, with respect to an Over 38 Contract, the
      earliest Salary Cap Year in which the Salary called for under the
      Contract has been attributed, in accordance with Section
      3(a)(2)(i), (ii) or (iii) above, to prior Salary Cap Years of the
      Contract.
    \item
      For purposes of this subsection (a)(2): (i) a player (A) whose
      birthday is on a date during the Moratorium Period and (B) who
      signs a Contract, Extension or Renegotiation on or before the
      fifth day following the date on which the Moratorium Period
      concludes shall be treated as if his age, at the time of such
      signing, was his age on the immediately preceding June 30; and
      (ii) any player whose Over 38 Contract is signed pursuant to
      Section 8(e)(1) below shall not be considered a Qualifying Veteran
      Free Agent.
    \end{enumerate}
  \end{enumerate}
\item
  \textbf{Signing Bonuses.}

  \begin{enumerate}
  \def\labelenumii{(\arabic{enumii})}
  \tightlist
  \item
    Amounts Treated as Signing Bonuses: For purposes of determining a
    player's Salary, the term ``signing bonus'' shall include:

    \begin{enumerate}
    \def\labelenumiii{(\roman{enumiii})}
    \tightlist
    \item
      any amount provided for in a Player Contract that is earned upon
      the signing of such Contract;
    \item
      at the time of a trade of a Player Contract, any amount that,
      under the terms of the Contract, is earned in the form of a bonus
      upon the trade of the Contract; and
    \item
      payments in excess of the Excluded International Player Payment
      Amount, in accordance with Section 3(e) below.
    \end{enumerate}
  \item
    Proration: Any signing bonus contained in a Player Contract shall be
    allocated over the number of Salary Cap Years (or over the
    then-current and any remaining Salary Cap Years in the case of a
    signing bonus described in Section 3(b)(1)(ii) above) covered by
    such Contract in proportion to the percentage of Base Compensation
    in each such Salary Cap Year that, at the time of allocation, is
    protected for lack of skill; provided, however, that if the Player
    Contract provides for an Early Termination Option (``ETO''), the
    foregoing allocation shall be performed only over Salary Cap Years
    that precede the Effective Season of such ETO. In the event that, at
    the time of allocation, none of the Base Compensation provided for
    by a Player Contract (or none of the then-current or remaining Base
    Compensation in the case of a signing bonus described in Section
    3(b)(1)(ii) above) is protected for lack of skill, then the entire
    amount of the signing bonus shall be allocated to the first Salary
    Cap Year of the Contract (or, in the case of a signing bonus
    described in Section 3(b)(1)(ii) above, the Salary Cap Year during
    which the player's Contract is traded).
  \item
    Extensions:

    \begin{enumerate}
    \def\labelenumiii{(\roman{enumiii})}
    \tightlist
    \item
      In the event that a Team with a Team Salary at or over the Salary
      Cap enters into an Extension that calls for or contains a signing
      bonus, such signing bonus shall be paid no sooner than the first
      day of the first Salary Cap Year covered by the extended term and
      shall be allocated, in equal parts, over the number of Salary Cap
      Years covered by the extended term in proportion to the percentage
      of Base Compensation in each such Salary Cap Year that, at the
      time of allocation, is protected for lack of skill. In the event
      that, at the time of the allocation, none of the Base Compensation
      provided for during the extended term is protected for lack of
      skill, then the entire amount of the signing bonus shall be
      allocated to the first Salary Cap Year of the extended term.
    \item
      A Team with a Team Salary below the Salary Cap may enter into an
      Extension that calls for or contains a signing bonus to be paid at
      any time during the Contract's original or extended term. In the
      event that a Team with a Team Salary below the Salary Cap enters
      into an Extension that calls for or contains a signing bonus to be
      paid no sooner than the first day of the Salary Cap Year covered
      by such extended term, the bonus shall be allocated in accordance
      with the proration rules set forth in Section 3(b)(3)(i) above. In
      the event a Team with a Team Salary below the Salary Cap enters
      into an Extension that calls for or contains a signing bonus to be
      paid prior to the first day of the first Salary Cap Year covered
      by the extended term, the following rules shall apply:

      \begin{enumerate}
      \def\labelenumiv{(\Alph{enumiv})}
      \tightlist
      \item
        The signing bonus shall be allocated over the remaining Salary
        Cap Years (including the then-current Salary Cap Year) under the
        original term of the Contract and the extended term in
        proportion to the percentage of Base Compensation in each such
        Salary Cap Year that, at the time of allocation, is protected
        for lack of skill. In the event that, at the time of allocation,
        none of the Base Compensation provided for during the
        then-current and any remaining Salary Cap Years under the
        original term of the Contract or during the extended term is
        protected for lack of skill, then the entire amount of the
        signing bonus shall be allocated to the Salary Cap Year during
        which the Extension is signed; and
      \item
        The Extension shall be deemed a Renegotiation and shall be
        subject to the rules governing Renegotiations set forth in
        Section 7 below.
      \end{enumerate}
    \item
      If a Team and player enter into an Extension and provide that the
      trade bonus provision contained in the original Contract would not
      be applicable to the extended term in accordance with Article
      XXIV, Section 2(a)(v), then, in the case of an earned signing
      bonus described in Section 3(b)(1)(ii) above, the signing bonus
      shall be allocated over the then-current and any remaining Salary
      Cap Year(s) covered by the original term of the extended Contract
      (and not any of the Salary Cap Years covered by the extended term)
      in proportion to the percentage of Base Compensation in each such
      Salary Cap Year that, at the time of allocation, is protected for
      lack of skill. In the event that, at the time of allocation, none
      of the then-current or applicable remaining Base Compensation is
      protected for lack of skill, then the entire amount of the signing
      bonus shall be allocated to the Salary Cap Year during which the
      player's extended Contract is traded.
    \end{enumerate}
  \end{enumerate}
\item
  \textbf{Loans to Players.} The following rules shall apply to any loan
  made by any Team to a player:

  \begin{enumerate}
  \def\labelenumii{(\arabic{enumii})}
  \tightlist
  \item
    If any such loan bears no interest (or annual interest at an
    effective rate lower than the ``Target Rate'' (as defined below)),
    then the interest shall be imputed on the outstanding balance at a
    rate equal to the difference between the Target Rate and the actual
    rate of interest to be paid by the player and such imputed interest
    shall be included in the player's Salary. The ``Target Rate'' means
    the ``Prime Rate'' (as defined below) plus one percent (1\%) as of
    the date the loan is agreed upon, except that the ``Target Rate''
    shall be no lower than seven percent (7\%) or greater than nine
    percent (9\%). For purposes of this Section 3(c)(1), ``Prime Rate''
    means the prime rate reported in the ``Money Rates'' column or any
    successor column of The Wall Street Journal.
  \item
    No loan made to a player may (along with other outstanding loans to
    the player) exceed the amount of the player's Salary for the
    then-current Salary Cap Year that is protected for lack of skill.
    All loans must be repaid through deductions from the player's
    remaining Current Base Compensation over the years of the Contract
    that, at the time the loan is agreed upon, provide for Base
    Compensation that is fully protected for lack of skill (prior to the
    Effective Season of any ETO) in equal annual amounts (the ``annual
    allocable repayment amounts''). If a loan is made at a time when the
    remaining Current Base Compensation due for the relevant Season that
    is fully protected for lack of skill is less than the annual
    allocable repayment amount that would be owed on a loan for the full
    amount of the player's Current Base Compensation that is fully
    protected for lack of skill for the relevant Season (the ``maximum
    annual allocable repayment amount''), the maximum loan amount for
    that Season shall be reduced by the amount by which the maximum
    annual allocable repayment amount exceeds the amount of remaining
    Current Base Compensation that is fully protected for lack of skill.
    (For example, if a Player has \$2 million in Current Base
    Compensation (fully protected for lack of skill) in the first Season
    of a five-year Contract, and a loan is made during that Season at a
    time when the Player has already received his Current Base
    Compensation for that Season, the loan may not exceed \$1.6
    million.)
  \item
    In addition to the restrictions set forth in Section 3(c)(2) above:
    (i) no loan may be made that would result in a violation of Article
    II, Section 13(e); and (ii) no loan may be made to a player whose
    Contract provides for Base Compensation equal to the Minimum Player
    Salary.
  \item
    Any forgiveness by a Team of a loan to a player shall be deemed a
    Renegotiation in the Salary Cap Year of such forgiveness and shall
    be subject to the rules governing Renegotiations set forth in
    Section 7 below.
  \end{enumerate}
\item
  \textbf{Incentive Compensation.}

  \begin{enumerate}
  \def\labelenumii{(\arabic{enumii})}
  \tightlist
  \item
    For purposes of determining a player's Salary each Salary Cap Year,
    except as provided in Sections 3(d)(2)-(4) below, any Performance
    Bonus (provided such Performance Bonus may be included in a Player
    Contract in accordance with Section 5(d) below), shall be included
    in Salary only if such Performance Bonus would be earned if the
    Team's or player's performance were identical to the performance in
    the immediately preceding Salary Cap Year.
  \item
    Notwithstanding Section 3(d)(1) above, in the event that, at the
    time of the signing of a Contract, Renegotiation or Extension, the
    NBA or the Players Association believes that the performance of a
    player and/or his Team during the immediately preceding Salary Cap
    Year does not fairly predict the likelihood of the player earning a
    Performance Bonus during any Salary Cap Year covered by the
    Contract, Renegotiation or extended term of the Extension (as the
    case may be), the NBA or the Players Association may request that a
    jointly selected basketball expert (``Expert'') determine whether
    (i) in the case of an NBA challenge, it is very likely that the
    bonus will be earned, or (ii) in the case of a Players Association
    challenge, it is very likely that the bonus will not be earned. The
    party initiating a proceeding before the Expert shall carry the
    burden of proof. The Expert shall conduct a hearing within five (5)
    business days after the initiation of the proceeding, and shall
    render a determination within five (5) business days after the
    hearing. Notwithstanding anything to the contrary in this Section
    3(d)(2), no party may, in connection with any proceeding before the
    Expert, refer to the facts that, absent a challenge pursuant to this
    Section 3(d)(2), a Performance Bonus would or would not be included
    in a player's Salary pursuant to Section 3(d)(1) above, or would be
    termed ``Likely'' or ``Unlikely'' pursuant to Article I, Section
    1(gg) or (eeee). If, following an NBA challenge, the Expert
    determines that a Performance Bonus is very likely to be earned, the
    bonus shall be included in the player's Salary. If, following a
    Players Association challenge, the Expert determines that a
    Performance Bonus is very likely not to be earned, the bonus shall
    be excluded from the player's Salary. The Expert's determination
    that a Performance Bonus is very likely to be earned or very likely
    not to be earned shall be final, binding and unappealable. The fees
    and costs of the Expert in connection with any proceeding brought
    pursuant to this Section 3(d)(2) shall be borne equally by the
    parties.
  \item
    In the case of a Rookie or a Veteran who did not play during the
    immediately preceding Salary Cap Year who signs a Contract
    containing a Performance Bonus, or in the case of a player signed or
    acquired by an Expansion Team whose Contract contains a Performance
    Bonus to be paid as a result of, in whole or in part, the player's
    achievement of agreed-upon benchmarks relating to the Team's
    performance during its first Salary Cap Year, such Performance Bonus
    will be included in Salary if it is likely to be earned. In the
    event that the NBA and the Players Association cannot agree as to
    whether a Performance Bonus is likely to be earned, such dispute
    will be referred to the Expert, who will determine whether the bonus
    is likely to be earned or not likely to be earned. The Expert shall
    conduct a hearing within five (5) business days after the initiation
    of the proceeding, and shall render a determination within five (5)
    business days after the hearing. The Expert's determination that a
    Performance Bonus is likely to be earned or not likely to be earned
    shall be final, binding and unappealable. The fees and costs of the
    Expert in connection with any proceeding brought pursuant to this
    Section 3(d)(3) shall be borne equally by the parties.
  \item
    In the event that either party initiates a proceeding pursuant to
    Section 3(d)(2) or (3) above, the player's Salary plus the full
    amount of any disputed bonuses shall be included in Team Salary
    during the pendency of the proceeding.
  \item
    In the event the NBA and the Players Association cannot agree on an
    Expert, any challenge pursuant to Section 3(d)(2) and (3) above may
    be filed with the Grievance Arbitrator in accordance with Article
    XXXI, Sections 2-7 and 15.
  \item
    All Incentive Compensation described in Article II, Sections
    3(b)(iii) and 3(c) shall be included in Salary.
  \end{enumerate}
\item
  \textbf{International Player Payments.}

  \begin{enumerate}
  \def\labelenumii{(\arabic{enumii})}
  \tightlist
  \item
    Any amount in excess of the amounts set forth below (``Excluded
    International Player Payment Amounts'') paid or to be paid by or at
    the direction of any NBA Team to (i) any basketball team other than
    an NBA Team, or (ii) any other entity, organization, representative
    or person, for the purpose of inducing a player who is participating
    in the game of basketball as a professional outside of the United
    States to enter into a Player Contract or in connection with
    securing the right to enter into a Player Contract with such a
    player shall be deemed Salary (in the form of a signing bonus) to
    the player:
  \end{enumerate}

  \begin{longtable}[]{@{}cc@{}}
  \toprule
  Salary Cap Year & Excluded International Player Payment
  Amount\tabularnewline
  \midrule
  \endhead
  2017-18 & \$675,000\tabularnewline
  2018-19 & \$700,000\tabularnewline
  2019-20 & \$725,000\tabularnewline
  2020-21 & \$750,000\tabularnewline
  2021-22 & \$775,000\tabularnewline
  2022-23 & \$800,000\tabularnewline
  2023-24 & \$825,000\tabularnewline
  \bottomrule
  \end{longtable}

  \begin{enumerate}
  \def\labelenumii{(\arabic{enumii})}
  \setcounter{enumii}{1}
  \tightlist
  \item
    Subject to Article XIII, any payment up to the Excluded
    International Player Payment Amount for a Salary Cap Year paid by or
    at the direction of any NBA Team pursuant to Section 3(e)(1) above
    to a professional basketball team outside the United States to
    secure the contractual release of a player shall not be deemed
    Salary to the player.
  \item
    The Excluded International Player Payment Amount may be paid in a
    single installment or in multiple installments. The Excluded
    International Player Payment Amount, whether used in whole or in
    part, may be used by an NBA Team whenever it signs a player to a new
    Player Contract, except that the Excluded International Player
    Payment Amount may not be used, in whole or in part, more than once
    in any three-Season period with respect to the same player.
  \item
    The Excluded International Player Payment Amount, or any part of it,
    shall be deemed to have been used as of the date of the Player
    Contract to which it applies, regardless of when it is actually
    paid. A schedule of payments relating to the Excluded International
    Player Payment Amount, or any part of it, agreed upon at the time of
    the signing of the Player Contract to which it applies, shall not be
    deemed a multiple use of the Excluded International Player Payment
    Amount.
  \item
    Notwithstanding Section 3(e)(1) above, no amount paid or to be paid
    pursuant to this Section 3(e) shall be counted toward the Minimum
    Team Salary obligation of a Team in accordance with Section 2(b) or
    (c) above.
  \item
    Within two (2) business days following the NBA's receipt of notice
    of any payments made by any NBA Team that are governed by this
    Section 3(e), the NBA shall provide the Players Association with
    written notice of such payments.
  \item
    Notwithstanding anything to the contrary in this Section 3(e), Teams
    shall be prohibited from making an international player payment to
    (i) any basketball team other than an NBA Team, or (ii) any other
    entity, organization, representative or person, for the purpose of
    inducing a player who is participating in the game of basketball as
    a professional outside of the United States to enter into a Contract
    with an Exhibit 10 or a Two-Way Contract or in connection with
    securing the right to enter into any such Two-Way Contract or
    Contract with an Exhibit 10 with a player.
  \end{enumerate}
\item
  \textbf{One-Year Minimum Contracts.} Except where otherwise stated in
  this Agreement, the Salary of every player who signs a one-year,
  10-Day or Rest-of-Season Contract for the Minimum Player Salary
  applicable to such player shall be the lesser of (1) such Minimum
  Player Salary, or (2) the portion of such Minimum Player Salary that
  is not reimbursed out of the League-wide benefits fund described in
  Article IV, Section 6(g)(2).
\item
  \textbf{Insurance Premium Reimbursement.} If a Team reimburses a
  player for life insurance premiums pursuant to Article II, Section
  4(j)(ii), such premium reimbursement shall not be included in the
  computation of the player's Salary.
\item
  \textbf{Averaging.} In accordance with Article XI, Section 5(d)(iii),
  a player's Salary for each Salary Cap Year covered by his Contract
  shall be deemed in certain circumstances to be the average of the
  aggregate Salaries for each such Salary Cap Year.
\item
  \textbf{Existing Contracts.} A player's Salary with respect to any
  Salary Cap Year covered by a Contract entered into prior to the
  effective date of this Agreement shall continue to be calculated in
  accordance with the Salary Cap rules that were in existence at the
  time the Contract was entered into except as provided in Article II,
  Section 6(d) and Article VIII, Section 5. In no event shall the
  preceding sentence apply to the calculation of Salary with respect to
  any Contract, Extension (with respect to the extended term),
  Renegotiation, transaction, or event entered into or occurring on or
  after the effective date of this Agreement.
\item
  \textbf{Existing Rookie Scale Contract Increases.} Except where
  otherwise stated in this Agreement, the Salary of every player whose
  Compensation under his Rookie Scale Contract is increased pursuant to
  Article VIII, Section 5, shall equal the greater of: (i) the player's
  Salary under his Contract prior to application of the existing Rookie
  Scale Conforming Increases (as defined in Article VIII, Section 5); or
  (ii) the player's applicable Minimum Player Salary, and shall not
  include any portion of the Rookie Scale Conforming Increases paid to
  the player that is reimbursed out of the League-wide benefits fund
  described in Article IV, Section 6(g)(4). For clarity, except to the
  extent provide in clause (ii) of the preceding sentence, the Rookie
  Scale Conforming Increases shall be excluded from the calculation of
  an individual player's Salary and each Team's Team Salary (and thus
  will not affect, for example, the amount of room, if applicable, a
  Team has below the Salary Cap, the amount of Traded Player Exceptions,
  etc.).
\end{enumerate}

\section{Determination of Team
Salary.}\label{determination-of-team-salary.}

\begin{enumerate}
\def\labelenumi{(\alph{enumi})}
\tightlist
\item
  \textbf{Computation.} For purposes of computing Team Salary under this
  Agreement, all of the following amounts shall be included:

  \begin{enumerate}
  \def\labelenumii{(\arabic{enumii})}
  \item
    Subject to the rules set forth in this Article VII, the aggregate
    Salaries of all active players (and former players to the extent
    provided by the terms of this Agreement) attributable to a
    particular Salary Cap Year, including, without limitation:

    \begin{enumerate}
    \def\labelenumiii{(\roman{enumiii})}
    \tightlist
    \item
      Salaries paid or to be paid to players whose Player Contracts have
      been terminated pursuant to the NBA's waiver procedure (without
      regard to any revised payment schedule that might be provided for
      in the terminated Player Contracts), except that, with respect to
      any Player Contract that has been terminated pursuant to the NBA's
      waiver procedure, if the waiving Team elects in writing to have
      the player's Salary stretched in accordance with Section 7(d)(6)
      below, then the amount to be included in Team Salary for a Salary
      Cap Year in respect of the terminated Player Contract shall equal
      the ``Stretched Salary Amounts'' as calculated in accordance with
      Section 7(d)(6) below.
    \item
      Any amount called for in a retired player's Player Contract paid
      or to be paid to the player. When a player retires and the Team
      continues to pay such amounts, then, for purposes of computing the
      player's Salary for the then-current and any remaining Salary Cap
      Year covered by the Contract, the aggregate of such amounts,
      notwithstanding the payment schedule, shall be allocated pro rata
      over the then-current and each remaining Salary Cap Year on the
      basis of the remaining unearned protected Compensation in each
      such Salary Cap Year at the time of retirement.
    \item
      Amounts paid or to be paid pursuant to awards for, or settlements
      of, grievances between a player and a Team concerning Compensation
      obligations under a Player Contract in accordance with the
      following rules (which, except for purposes of Section
      4(a)(1)(iii)(C) below, shall be applied with respect to each
      Season for which there is any Compensation in dispute, as if the
      grievance relates only to such Season):

      \begin{enumerate}
      \def\labelenumiv{(\Alph{enumiv})}
      \item
        \begin{enumerate}
        \def\labelenumv{(\arabic{enumv})}
        \tightlist
        \item
          When a player initiates a Grievance (as defined in Article
          XXXI) against a Team seeking the payment of Compensation for a
          Season covered by the current or any future Salary Cap Year
          that the Team asserts is not owed, fifty percent (50\%) of the
          disputed amount shall be included in Team Salary for the
          Salary Cap Year to which the grievance relates. If the
          Grievance is resolved during or prior to the Salary Cap Year
          to which it relates, following resolution of the Grievance,
          whether by award or settlement, the disputed amount payable by
          the Team in excess of the fifty percent (50\%) allocation
          shall be included in Team Salary for the Salary Cap Year to
          which the Grievance relates, or, alternatively, the amount by
          which the fifty percent (50\%) allocation exceeds the disputed
          amount payable by the Team shall be subtracted from Team
          Salary for the Salary Cap Year to which the Grievance relates.
        \item
          If a Grievance described in the first sentence of Section
          4(a)(1)(iii)(A)(1) above is resolved after the conclusion of
          the Salary Cap Year to which it relates, the disputed amount
          payable by the Team related to such Salary Cap Year in excess
          of the fifty percent (50\%) allocation shall be included in
          Team Salary for the Salary Cap Year in which the Grievance is
          resolved, or, alternatively, the amount by which the fifty
          percent (50\%) allocation exceeds the disputed amount payable
          by the Team related to such Salary Cap Year shall be
          subtracted from Team Salary for the Salary Cap Year in which
          the grievance is resolved. Notwithstanding the preceding
          sentence: (i) a Team shall be required to pay additional tax
          to the NBA if and to the extent that, due to the operation of
          this Section 4(a)(1)(iii)(A)(2), the aggregate tax it pays to
          the NBA pursuant to Section 12(f) below for the two (2) Salary
          Cap Years in question (the Salary Cap Year for which the fifty
          percent (50\%) allocation was made and the subsequent Salary
          Cap Year in which the Grievance was resolved) is less than it
          would have been had the Grievance been resolved during the
          Salary Cap Year to which it related; and (ii) a Team shall be
          entitled to a tax refund from the NBA if and to the extent
          that, due to the operation of this Section 4(a)(1)(iii)(A)(2),
          the aggregate tax it pays to the NBA pursuant to Section 12(f)
          below for the two (2) Salary Cap Years in question is greater
          than it would have been had the grievance been resolved during
          the Salary Cap Year to which it related. In order to
          facilitate any such required tax refund from the NBA to the
          Team, the NBA shall set aside, pending resolution of the
          Grievance, the amount of tax paid by that Team in the Salary
          Cap Year to which the Grievance relates that is attributable
          to the fifty percent (50\%) allocation. Following resolution
          of the Grievance, the NBA shall pay to the Team the tax refund
          to which it is entitled (if any) based upon the resolution of
          the Grievance, and the remainder of the set aside tax funds
          shall be distributed by the NBA to one (1) or more Teams or
          otherwise used by the League in such manner as the NBA may
          reasonably determine, consistent with the provisions of
          Section 12(g) below.
        \end{enumerate}
      \item
        When a player initiates a Grievance against a Team seeking the
        payment of Compensation for a Season covered by a prior Salary
        Cap Year that the Team asserts is not owed, following resolution
        of the Grievance, whether by award or settlement, the disputed
        amount payable by the Team, if any, shall be included in Team
        Salary for the Salary Cap Year in which the Grievance is
        resolved (but only to the extent that it had been previously
        excluded from Team Salary). Notwithstanding the preceding
        sentence: (i) a Team shall be required to pay additional tax to
        the NBA if and to the extent that, due to the operation of this
        Section 4(a)(1)(iii)(B), the aggregate tax it pays to the NBA
        pursuant to Section 12(f) below for the two (2) Salary Cap Years
        in question (the Salary Cap Year to which the Grievance related
        and the subsequent Salary Cap Year in which the Grievance was
        resolved) is less than it would have been had the disputed
        amount payable by the Team been included in Team Salary during
        the Salary Cap Year to which it related; and (ii) a Team shall
        be entitled to a tax refund from the NBA if and to the extent
        that, due to the operation of this Section 4(a)(1)(iii)(B), the
        aggregate tax it pays to the NBA pursuant to Section 12(f) below
        for the two (2) Salary Cap Years in question is greater than it
        would have been had the disputed amount payable by the Team been
        included in Team Salary during the Salary Cap Year to which it
        related.
      \item
        If a Grievance relates to a player's Compensation for more than
        one (1) Season, for purposes of determining the disputed amount
        payable by the Team with respect to each such Season following
        the resolution of the Grievance, the aggregate amounts payable
        to the player for all Seasons pursuant to the resolution of the
        grievance, whether by award or settlement, shall be allocated to
        each such Season in proportion to the amount of Compensation
        that was in dispute for such Season, unless, in the case of an
        award, the Grievance Arbitrator allocates the amounts payable to
        the player to specific Seasons.
      \item
        Immediately upon reaching any agreement (oral or written) to
        resolve a Grievance relating to a player's Compensation, a Team
        shall notify the NBA by email and provide the NBA with the terms
        of such agreement. A Team's failure to comply with the preceding
        sentence may be considered evidence of a violation of Article
        XIII. If a Team delays or attempts to delay in any manner the
        processing or resolution of a Grievance relating to a player's
        Compensation for the purpose of creating or increasing its Room
        in any Salary Cap Year or for the purpose of reducing or
        deferring a tax payment to the NBA, such conduct shall
        constitute a violation of Article XIII.
      \end{enumerate}
    \item
      Salaries anticipated to be included in Team Salary based upon any
      agreement disclosed to the NBA pursuant to Article II, Section
      13(a)(i) (including, without limitation, any executed Player
      Contract whose validity is conditional on the passage of a
      physical examination by the player or on the assignment of the
      Contract), except to the extent that any such Salary is less than
      a player's Free Agent Amount (as defined in Section 4(d) below).
    \end{enumerate}
  \item
    \begin{enumerate}
    \def\labelenumiii{(\roman{enumiii})}
    \tightlist
    \item
      With respect to each Veteran Free Agent who last played for a Team
      who is an Unrestricted Free Agent, the Free Agent Amount (as
      defined in Section 4(d) below) attributable to such Veteran Free
      Agent.
    \item
      With respect to each Veteran Free Agent who last played for a Team
      who is a Restricted Free Agent, the greater of (A) the Free Agent
      Amount (as defined in Section 4(d) below) attributable to such
      Veteran Free Agent, (B) the Salary called for in any outstanding
      Qualifying Offer (other than a Two-Way Qualifying Offer, as
      defined Article XI, Section 1(c)(iii)(B) below) tendered to such
      Veteran Free Agent (or, if the Restricted Free Agent was also
      tendered a Maximum Qualifying Offer pursuant to Article XI,
      Section 4(a)(ii), the Salary called for in such outstanding
      Maximum Qualifying Offer), or (C) the Salary called for in any
      First Refusal Exercise Notice (as defined in Article XI, Section
      5(e)) issued with respect to such Veteran Free Agent.
    \end{enumerate}
  \item
    The aggregate Salaries called for under all outstanding Offer Sheets
    (as defined in Article XI, Section 5(b)).
  \item
    An amount with respect to a Team's unsigned First Round Pick, if
    any, as determined in accordance with Section 4(e) below.
  \item
    An amount with respect to the number of players fewer than twelve
    (12) included in a Team's Team Salary, as determined in accordance
    with Section 4(f) below.
  \item
    Value or consideration received by retired players that is
    determined to be includable in Team Salary in accordance with
    Article XIII, Section 5.
  \item
    The amount of any Salary Cap Exception that is deemed included in
    Team Salary in accordance with Section 6(m)(2) below.
  \end{enumerate}
\item
  \textbf{Expansion.} The Salary of any player selected by an Expansion
  Team in an expansion draft and terminated in accordance with the NBA
  waiver procedure before the first day of the Expansion Team's first
  Season shall not be included in the Expansion Team's Team Salary,
  except, to the extent such Salary is paid, for purposes of determining
  whether the Expansion Team has satisfied its Minimum Team Salary
  obligation for such Season.
\item
  \textbf{Assigned Contracts.} For purposes of calculating Team Salary,
  with respect to any Player Contract that is assigned, the assignee
  Team shall, upon assignment, have included in its Team Salary the
  entire Salary for the then-current Salary Cap Year and for all future
  Salary Cap Years.
\item
  \textbf{Free Agents.} Subject to Section 4(a)(2)(ii) above, until a
  Team's Veteran Free Agent re-signs with his Team, signs with another
  NBA Team, or is renounced, he will be included in his Prior Team's
  Team Salary at one of the following amounts (``Free Agent Amounts''):

  \begin{enumerate}
  \def\labelenumii{(\arabic{enumii})}
  \item
    \begin{enumerate}
    \def\labelenumiii{(\roman{enumiii})}
    \tightlist
    \item
      A Qualifying Veteran Free Agent, other than a Qualifying Veteran
      Free Agent described in Section 4(d)(1)(ii) or (iii) below, will
      be included at one hundred fifty percent (150\%) of his prior
      Salary if it was equal to or greater than the Estimated Average
      Player Salary for the prior Salary Cap Year, and one hundred
      ninety percent (190\%) of his prior Salary if it was less than the
      Estimated Average Player Salary for the prior Salary Cap Year.
    \item
      A Qualifying Veteran Free Agent following the second Option Year
      of his Rookie Scale Contract will be included as follows:

      \begin{enumerate}
      \def\labelenumiv{(\Alph{enumiv})}
      \tightlist
      \item
        For the 2017-18 Salary Cap Year: two hundred percent (200\%) of
        the player's prior Salary if it was equal to or greater than the
        Estimated Average Player Salary for the prior Salary Cap Year,
        and two hundred fifty percent (250\%) of his prior Salary if it
        was less than the Estimated Average Player Salary for the prior
        Salary Cap Year; and
      \item
        For each subsequent Salary Cap Year: two hundred fifty percent
        (250\%) of the player's prior Salary if it was equal to or
        greater than the Estimated Average Player Salary for the prior
        Salary Cap Year, and three hundred percent (300\%) of his prior
        Salary if it was less than the Estimated Average Player Salary
        for the prior Salary Cap Year.
      \end{enumerate}
    \end{enumerate}
  \item
    An Early Qualifying Veteran Free Agent will be included at one
    hundred thirty percent (130\%) of his prior Salary; provided,
    however, that the player's prior Team may, by written notice to the
    NBA, renounce its rights to sign the player pursuant to the Early
    Qualifying Veteran Free Agent Exception, in which case the player
    will be deemed a Non-Qualifying Veteran Free Agent for purposes of
    this Section 4(d) and Sections 6(b) and 6(j)(4) below.
  \item
    A Non-Qualifying Veteran Free Agent will be included at one hundred
    twenty percent (120\%) of his prior Salary.
  \item
    Notwithstanding Section 4(d)(1)-(3) above, if the player's prior
    Salary was equal to or less than the Minimum Player Salary
    applicable to such player, he will be included at the portion of the
    then-current Minimum Annual Salary applicable to such player that
    would not be reimbursed out of the League-wide benefits fund
    described in Article IV, Section 6(g).
  \item
    Notwithstanding Section 4(d)(1)-(3) above, at no time shall a
    player's Free Agent Amount exceed the Maximum Player Salary
    applicable to such player or be less than the portion of the Minimum
    Annual Salary applicable to such player that would not be reimbursed
    out of the League-wide benefits fund described in Article IV,
    Section 6(g).
  \item
    Notwithstanding Section 4(d)(1)-(3) above, at no time shall a Free
    Agent Amount for a Veteran Free Agent following the second or third
    Season of his Rookie Scale Contract exceed the maximum amount the
    Team may pay the player pursuant to Section 6(m)(4) below.
  \item
    Notwithstanding Section 4(d)(1)-(5) above, if a Two-Way Player
    completes a Two-Way Contract, the player's Free Agent Amount will be
    the Minimum Annual Salary applicable to a player completing a
    Standard NBA Contract for the zero (0) Years of Service Minimum
    Annual Salary.
  \item
    For purposes of this Section 4(d) only, a player's ``prior Salary''
    means his Regular Salary for the prior Season plus any signing bonus
    allocation and the amount of any Incentive Compensation actually
    earned for such Season under the Player Contract in effect when the
    player finished the prior Season.
  \end{enumerate}
\item
  \textbf{First Round Picks.}

  \begin{enumerate}
  \def\labelenumii{(\arabic{enumii})}
  \tightlist
  \item
    A First Round Pick, immediately upon selection in the Draft, shall
    be included in the Team Salary of the Team that holds his draft
    rights at one hundred twenty percent (120\%) of his applicable
    Rookie Scale Amount (``Rookie Scale Cap Hold Amount''), and, subject
    to Section 4(e)(2) and (3) below, shall continue to be included in
    the Team Salary of any Team that holds his draft rights (including
    any Team to which the player's draft rights are assigned) until such
    time as the player signs with such Team or until the Team loses or
    assigns its exclusive draft rights to the player.
  \item
    In the event that a First Round Pick signs with a non-NBA team, the
    player's applicable Rookie Scale Cap Hold Amount shall be excluded
    from the Team Salary of the Team that holds his draft rights,
    beginning on the date he signs such non-NBA contract or the first
    day of the Regular Season, whichever is later, and shall be included
    again in his Team's Team Salary at the applicable Rookie Scale Cap
    Hold Amount on the following July 1 or the date the player's
    contract ends (or the player is released from his non-NBA
    contractual obligations), whichever is earlier, unless the Team
    renounces its exclusive rights to the player in accordance with
    Article X, Section 4(g). If, after such following July 1, or any
    subsequent July 1, the player signs another, or remains under,
    contract with a non-NBA team, the player's applicable Rookie Scale
    Cap Hold Amount will again be excluded from Team Salary beginning on
    the date of the contract signing or the first day of the Regular
    Season commencing after such July 1, whichever is later, and will
    again be included in Team Salary at the applicable Rookie Scale Cap
    Hold Amount on the following July 1 or the date the player's
    contract ends (or the player is released from his non-NBA
    contractual obligations), whichever is earlier, unless the Team
    renounces its exclusive rights to the player in accordance with
    Article X, Section 4(g).
  \item
    A Team that holds draft rights to a First Round Pick may elect to
    have the player's applicable Rookie Scale Cap Hold Amount excluded
    from its Team Salary at any time prior to the first day of any
    Regular Season by providing the NBA with a written statement that
    the Team will not sign the player during that Salary Cap Year
    accompanied by a written statement from the First Round Pick
    renouncing his right to accept any outstanding Required Tender made
    to him by the Team. After making such an election, (i) the Team
    shall be prohibited from signing the player during that Salary Cap
    Year, except in accordance with Section 5(e)(4)(ii) below, (ii) the
    Team shall continue to possess such rights with respect to the
    player that the Team possessed pursuant to Article X immediately
    prior to such election, and (iii) the player's applicable Rookie
    Scale Amount shall be included again in his Team's Team Salary at
    the applicable Rookie Scale Cap Hold Amount on the following July 1.
    When a First Round Pick provides a Team with a written statement
    renouncing his right to accept that year's outstanding Required
    Tender, the Player shall no longer be permitted to accept it.
  \item
    For purposes of this Section 4(e), in the event that a First Round
    Pick does not sign a Contract with the Team that holds his draft
    rights during the Salary Cap Year immediately following the Draft in
    which he was selected (or during the same Salary Cap Year in which
    he was drafted if the Draft occurs on or after July 1), the
    ``applicable Rookie Scale Amount'' for such First Round Pick means,
    with respect to any subsequent Salary Cap Year, the Rookie Scale
    Amount that would apply if the player were drafted in the Draft
    immediately preceding such Salary Cap Year at the same draft
    position at which he was actually selected.
  \end{enumerate}
\item
  \textbf{Incomplete Rosters.}

  \begin{enumerate}
  \def\labelenumii{(\arabic{enumii})}
  \tightlist
  \item
    If at any time from July 1 through the day prior to the first day of
    the Regular Season a Team has fewer than twelve (12) players,
    determined in accordance with Section 4(f)(2) below, included in its
    Team Salary, then the Team's Team Salary shall be increased by an
    amount calculated as follows: STEP 1: Subtract from twelve (12) the
    number of players included in Team Salary. STEP 2: If the result in
    Step 1 is a positive number, multiply the result in Step 1 by the
    Minimum Annual Salary applicable to players with zero (0) Years of
    Service under the Minimum Annual Salary Scale for that Salary Cap
    Year.
  \item
    In determining whether a Team has fewer than twelve (12) players
    included in its Team Salary for purposes of Section 4(f)(1) above
    only, the only players who shall be counted are (i) players under
    Contract with the Team who are included in Team Salary, (ii) Free
    Agents who are included in Team Salary pursuant to Section 4(a)(2)
    above, (iii) players to whom Offer Sheets have been given, and (iv)
    unsigned First Round Picks who are included in Team Salary pursuant
    to Section 4(e) above.
  \end{enumerate}
\item
  \textbf{Renouncing.}

  \begin{enumerate}
  \def\labelenumii{(\arabic{enumii})}
  \tightlist
  \item
    To renounce a Veteran Free Agent, a Team must provide the NBA with a
    written statement renouncing its right to re-sign the player,
    effective no earlier than the July 1 following the last Season
    covered by the player's Contract. (The NBA shall notify the Players
    Association of any such renunciation by email within two (2)
    business days following receipt of notice of such renunciation.) If
    a Team renounces a Veteran Free Agent, the player will no longer
    qualify as a Qualifying Veteran Free Agent, Early Qualifying Veteran
    Free Agent, or Non-Qualifying Veteran Free Agent, as the case may
    be, and the Team will only be permitted to re-sign such player with
    Room (i.e., the Team cannot sign such player pursuant to Section
    6(b) below), pursuant to the Minimum Player Salary Exception, or to
    a Two-Way Contract. Notwithstanding the foregoing, in the event a
    Team renounces one or more players pursuant to this Section 6(g)
    (or, with respect to a First Round Pick, pursuant to Article X,
    Section 4(g)) in order to create Room for an Offer Sheet, and the
    offeree-player's Prior Team subsequently matches the Offer Sheet and
    enters into a Contract with that player, the Team may rescind the
    renunciation (in the case where a Team renounces only one player) or
    all such renunciations (in the case where the Team renounces more
    than one player) within two (2) business days of the date the Offer
    Sheet is matched (or, if the Prior Team conditions its match on the
    player reporting for and passing a physical, within two (2) business
    days of the player passing the physical), whereupon any such
    ``unrenounced'' player may again sign a Player Contract with the
    Team as a First Round Pick, Qualifying Veteran Free Agent, Early
    Qualifying Veteran Free Agent, or Non-Qualifying Veteran Free Agent,
    as the case may be, and will again be included in his Prior Team's
    Team Salary at his applicable Free Agent Amount; provided, however,
    that a Team may not rescind the renunciation of a player if (i) at
    the time the player was renounced the Team's Team Salary was at or
    below the Salary Cap and ``unrenouncing'' the player would cause the
    Team's Team Salary to exceed the Salary Cap, or (ii) at the time the
    player was renounced the Team's Team Salary was above the Salary Cap
    and ``unrenouncing'' the player would cause the Team's Team Salary
    to exceed the Salary Cap by more than the amount by which the Team's
    Team Salary exceeded the Salary Cap prior to the renunciation.
  \item
    A Team cannot renounce any player to whom the Team has made a
    Qualifying Offer until such time as the Qualifying Offer is no
    longer in effect.
  \end{enumerate}
\item
  \textbf{Long-Term Injuries.} Any player who suffers a career-ending
  injury or illness, and whose Contract is terminated by the Team in
  accordance with the NBA waiver procedure, will be excluded from his
  Team's Team Salary as follows:

  \begin{enumerate}
  \def\labelenumii{(\arabic{enumii})}
  \tightlist
  \item
    Subject to Section 4(h)(5) below, a Team may apply to the NBA to
    have the player's Salary for each remaining Salary Cap Year covered
    by the Contract excluded from Team Salary beginning on the first
    anniversary of the date of the last Regular Season or playoff game
    in which the player played; provided that if the player played in
    fewer than ten (10) Regular Season and playoff games in the last
    Season in which he played, then the earliest date upon which a Team
    may apply to the NBA to have the player's Salary excluded from its
    Team Salary in accordance with this Section 4(h) shall be the later
    of (A) sixty (60) days following the date during such Season in
    which the player last played in a Regular Season or playoff game,
    and (B) the first anniversary of the date during a prior Season in
    which the player last played in a Regular Season or playoff game
    under such Contract. Notwithstanding anything to the contrary in
    this Section 4(h)(1), a Team may not apply to have a player's Salary
    excluded from Team Salary prior to the first anniversary of the date
    of the first Regular Season game that the player is on the Team's
    roster under the Contract in question.
  \item
    The determination of whether a player has suffered a career-ending
    injury or illness shall be made by a physician selected jointly by
    the NBA and the Players Association or, upon agreement of the NBA
    and the Players Association, a Fitness to Play Panel established
    under Article XXII. A player shall be deemed to have suffered a
    career-ending injury or illness if it is determined (i) by a such
    physician or Fitness to Play Panel that the player has an injury or
    illness that (x) prevents him from playing skilled professional
    basketball at an NBA level for the duration of his career, or (y)
    substantially impairs his ability to play skilled professional
    basketball at an NBA level and is of such severity that continuing
    to play professional basketball at an NBA level would subject the
    player to medically unacceptable risk of suffering a
    life-threatening or permanently disabling injury or illness, or (ii)
    by such Fitness to Play Panel that the player has an injury or
    illness that would create a materially elevated risk of death for
    the player under the procedures set forth in Article XXII, Section
    11.
  \item
    Notwithstanding Section 4(h)(1) and (2) above, if after a player's
    Salary is excluded from Team Salary in accordance with this Section
    4(h), the player plays in twenty-five (25) NBA Regular Season and
    playoff games in any Season for any Team, the excluded Salary for
    the Salary Cap Year covering such Season and each subsequent Salary
    Cap Year shall thereupon be included in Team Salary of the Team from
    which the Salary was previously excluded (and if the twenty-fifth
    game played is a playoff game, then the excluded Salary shall be
    included in Team Salary retroactively as of the start of the Team's
    last Regular Season game); provided, however, that the foregoing
    sentence shall not apply in the event a player is determined to have
    suffered a career-ending injury or illness pursuant to Section
    4((h)(2)(ii) above. After a player's Salary for one (1) or more
    Salary Cap Years has been included in Team Salary in accordance with
    this Section 4(h)(3), the Team shall be permitted to re-apply to
    have the player's Salary (for each Salary Cap Year remaining at the
    time of the re-application) excluded from Team Salary in accordance
    with the rules set forth in this Section 4(h) (including the waiting
    period criteria set forth in Section 4(h)(1) above).
  \item
    If a Team applies to have a player's Salary excluded from its Team
    Salary pursuant to this Section 4(h), the player shall cooperate in
    the processing of the application, including by appearing at the
    reasonably scheduled place and time for examination by the
    jointly-selected physician. The player shall not make any
    misrepresentation or fail to disclose any relevant information in
    connection with the processing of the application.
  \item
    Only the Team with which the player was under Contract at the time
    his career-ending injury or illness became known or reasonably
    should have become known shall be permitted to apply to have the
    player's Salary excluded from Team Salary pursuant to this Section
    4(h). A Team may only apply to have a player's Salary excluded from
    its Team Salary pursuant to this Section 4(h) during the term
    covered by the player's Contract. For clarity, if a player's Salary
    is excluded from Team Salary pursuant to this Section 4(h), if, at
    the time of such exclusion, the Team has previously elected to
    stretch any Salary in respect of one or more current or future
    Salary Cap Years pursuant to Section 7(d)(6), such stretched Salary
    shall also be excluded.
  \item
    Notwithstanding anything to the contrary in this Agreement, (i) if a
    Team applies to have a player's Salary excluded from its Team Salary
    pursuant to this Section 4(h) and such application is granted, the
    Team will be prohibited from re-signing or re-acquiring that player
    at any time, and (ii) if a Team makes a request for an Exception to
    replace a Disabled Player pursuant to Section 6(c) below for a
    Salary Cap Year, then, whether such application is granted or
    denied, the Team will be precluded from applying to have that
    player's Salary excluded from its Team Salary pursuant to this
    Section 4(h) for the same Salary Cap Year.
  \end{enumerate}
\item
  \textbf{Summer Contracts.}

  \begin{enumerate}
  \def\labelenumii{(\arabic{enumii})}
  \tightlist
  \item
    Except as provided in Section 4(i)(2) below and subject to Article
    II, Section 15, from July 1 until the day prior to the first day of
    the next Regular Season, a Team may enter into Player Contracts that
    will not be included in Team Salary until the first day of such
    Regular Season (i.e., the player will be deemed not to have any
    Salary until the first day of such Regular Season), provided that
    such Contracts satisfy the requirements of this Section 4(i) (a
    ``Summer Contract''). Except as set forth in the following sentence,
    no Summer Contract may provide for (i) Compensation of any kind that
    is or may be paid or earned prior to the first day of the next
    Regular Season, or (ii) Compensation protection of any kind pursuant
    to Article II, Section 3(g) or 4. The only consideration that may be
    provided to a player signed to a Summer Contract, prior to the start
    of the Regular Season, is per diem, lodging, transportation,
    compensation in accordance with paragraph 3(b) of the Uniform Player
    Contract, and a disability insurance policy covering disabilities
    incurred while such player participates in summer leagues or rookie
    camps for the Team. A Team that has entered into one or more Summer
    Contracts must terminate such Contracts no later than the day prior
    to the first day of a Regular Season, except to the extent the Team
    has Room for such Contracts or is entitled to use the Minimum Player
    Salary Exception.
  \item
    A Team may not enter into a Summer Contract with a Veteran Free
    Agent who last played for the Team unless the Contract is for one
    (1) Season only and provides for no more than the Minimum Player
    Salary applicable to such player.
  \end{enumerate}
\item
  \textbf{Two-Way Contracts.} Two-Way Player Salaries shall be excluded
  from Team Salary. Thus, for example, a Team is not required to have
  Room or an Exception to sign, acquire, or convert a player to a
  Two-Way Contract.
\item
  \textbf{Exhibit 10 Bonus.} Any amounts earned by a player pursuant to
  an Exhibit 10 Bonus shall be excluded from Team Salary.
\item
  \textbf{Team Salary Summaries.}

  \begin{enumerate}
  \def\labelenumii{(\arabic{enumii})}
  \tightlist
  \item
    The NBA shall provide the Players Association with Team Salary
    summaries and a list of current Exceptions twice a month during the
    Regular Season and once every week during the off-season.
  \item
    In the event that the NBA fails to provide the Players Association
    with any Team Salary summary or list of Exceptions as provided for
    in Section 4(l)(1) above, the Players Association shall notify the
    NBA of such failure, and the NBA, upon receipt of such notice, shall
    as soon as reasonably possible, but in no event later than two (2)
    business days following receipt of such notice, provide the Players
    Association with any such summary or list that should have been
    provided pursuant to Section 4(l)(1) above.
  \end{enumerate}
\end{enumerate}

\section{Operation of Salary Cap.}\label{operation-of-salary-cap.}

\begin{enumerate}
\def\labelenumi{(\alph{enumi})}
\tightlist
\item
  \textbf{Basic Rule.} A Team's Team Salary may not exceed the Salary
  Cap at any time unless the Team is using one of the Exceptions set
  forth in Section 6 below.
\item
  \textbf{Room.} Subject to the other provisions of this Agreement,
  including without limitation Article II, Section 7, any Team with Room
  may enter into a Player Contract that calls for a Salary in the first
  Salary Cap Year covered by such Contract that would not exceed the
  Team's then-current Room.
\item
  \textbf{Annual Salary Increases and Decreases.}

  \begin{enumerate}
  \def\labelenumii{(\arabic{enumii})}
  \tightlist
  \item
    The following rules apply to all Player Contracts other than
    Contracts between Qualifying Veteran Free Agents or Early Qualifying
    Veteran Free Agents and their Prior Team:

    \begin{enumerate}
    \def\labelenumiii{(\roman{enumiii})}
    \tightlist
    \item
      For each Salary Cap Year covered by a Player Contract after the
      first Salary Cap Year, the player's: (i) Salary, excluding
      Incentive Compensation, may increase or decrease in relation to
      the previous Salary Cap Year's Salary, excluding Incentive
      Compensation, by no more than five percent (5\%) of the Salary for
      the first Salary Cap Year covered by the Contract; and (ii)
      Regular Salary may increase or decrease in relation to the
      previous Salary Cap Year's Regular Salary by no more than five
      percent (5\%) of the Regular Salary for the first Salary Cap Year
      covered by the Contract.
    \item
      In the event that the first Salary Cap Year covered by a Contract
      provides for Incentive Compensation, the total amount of Likely
      Bonuses in each subsequent Salary Cap Year covered by the Contract
      may increase or decrease by up to five percent (5\%) of the amount
      of Likely Bonuses in the first Salary Cap Year, and the total
      amount of Unlikely Bonuses in each subsequent Salary Cap Year may
      increase or decrease by up to five percent (5\%) of the amount of
      Unlikely Bonuses in the first Salary Cap Year.
    \end{enumerate}
  \item
    The following rules apply to all Player Contracts between Qualifying
    Veteran Free Agents or Early Qualifying Veteran Free Agents and
    their Prior Team (except any such Contracts signed pursuant to
    Section 6(d)(3), Section 6(e)(2), Section 6(f)(3), Section 6(g)(3)
    or Section 8(e)(1) below, which shall be governed by Section 5(c)(1)
    above):

    \begin{enumerate}
    \def\labelenumiii{(\roman{enumiii})}
    \tightlist
    \item
      For each Salary Cap Year covered by a Player Contract after the
      first Salary Cap Year, the player's: (i) Salary, excluding
      Incentive Compensation, may increase or decrease in relation to
      the previous Salary Cap Year's Salary, excluding Incentive
      Compensation, by no more than eight percent (8\%) of the Salary
      for the first Salary Cap Year covered by the Contract; and (ii)
      Regular Salary may increase or decrease in relation to the
      previous Salary Cap Year's Regular Salary by no more than eight
      percent (8\%) of the Regular Salary for the first Salary Cap Year
      covered by the Contract.
    \item
      In the event that the first Salary Cap Year covered by a Contract
      provides for Incentive Compensation, the total amount of Likely
      Bonuses in each subsequent Salary Cap Year covered by the Contract
      may increase or decrease by up to eight percent (8\%) of the
      amount of Likely Bonuses in the first Salary Cap Year, and the
      total amount of Unlikely Bonuses in each subsequent Salary Cap
      Year may increase or decrease by up to eight percent (8\%) of the
      amount of Unlikely Bonuses in the first Salary Cap Year.
    \end{enumerate}
  \item
    The following rules apply to all Extensions other than Extensions
    entered into in connection with a trade pursuant to Section 8(e)(2)
    below:

    \begin{enumerate}
    \def\labelenumiii{(\roman{enumiii})}
    \tightlist
    \item
      For each Salary Cap Year covered by an Extension after the first
      Salary Cap Year covered by the extended term, the player's: (i)
      Salary, excluding Incentive Compensation, may increase or decrease
      in relation to the previous Salary Cap Year's Salary, excluding
      Incentive Compensation, by no more than eight percent (8\%) of the
      Salary for the first Salary Cap Year covered by the extended term
      of the Contract; and (ii) Regular Salary may increase or decrease
      in relation to the previous Salary Cap Year's Regular Salary by no
      more than eight percent (8\%) of the Regular Salary for the first
      Salary Cap Year covered by the Contract.
    \item
      In the event that the first Salary Cap Year covered by the
      extended term of the Contract provides for Incentive Compensation,
      the amount of Likely Bonuses and Unlikely Bonuses in each Salary
      Cap Year covered by the Extension after the first Salary Cap Year
      covered by the extended term may increase or decrease by up to
      eight percent (8\%) of the amount of Likely Bonuses and Unlikely
      Bonuses, respectively, in the first Salary Cap Year covered by the
      extended term.
    \end{enumerate}
  \item
    The following rules apply to Extensions entered into in connection
    with a trade pursuant to Section 8(e)(2) below:

    \begin{enumerate}
    \def\labelenumiii{(\roman{enumiii})}
    \tightlist
    \item
      For each Salary Cap Year covered by an Extension after the first
      Salary Cap Year covered by the extended term, the player's: (i)
      Salary, excluding Incentive Compensation, may increase or decrease
      in relation to the previous Salary Cap Year's Salary, excluding
      Incentive Compensation, by no more than five percent (5\%) of the
      Salary for the first Salary Cap Year covered by the extended term
      of the Contract; and (ii) Regular Salary may increase or decrease
      in relation to the previous Salary Cap Year's Regular Salary by no
      more than five percent (5\%) of the Regular Salary for the first
      Salary Cap Year covered by the Contract.
    \item
      In the event that the last Salary Cap Year covered by the original
      term of the Contract provides for Incentive Compensation, the
      amount of Likely Bonuses and Unlikely Bonuses in each Salary Cap
      Year covered by the Extension after the first Salary Cap Year
      covered by the extended term may increase or decrease by up to
      five percent (5\%) of the amount of Likely Bonuses and Unlikely
      Bonuses, respectively, in the first Salary Cap Year covered by the
      extended term.
    \end{enumerate}
  \item
    For purposes of this Section 5(c) only, the amount of any bonuses
    that a player may receive pursuant to Article II, Sections 3(b)(iii)
    and 3(c) shall be added to the player's Regular Salary and excluded
    from his Incentive Compensation.
  \item
    The foregoing rules set forth above in this Section 5 shall not
    apply to Two-Way Contracts. With respect to a Two-Way Contract with
    a term of two (2) Seasons, the Two-Way NBADL Salary in the second
    Season of such Contract shall equal the Two-Way NBADL Annual Salary
    for the Salary Cap Year encompassing such Season.
  \end{enumerate}
\item
  \textbf{Performance Bonuses.}

  \begin{enumerate}
  \def\labelenumii{(\arabic{enumii})}
  \tightlist
  \item
    Notwithstanding any other provision of this Agreement, no Player
    Contract may provide for Unlikely Bonuses in any Salary Cap Year
    that exceed fifteen percent (15\%) of the player's Regular Salary
    for such Salary Cap Year at the time the Contract is signed;
    provided, however, that: (i) with respect to Extensions, if the
    amount of Unlikely Bonuses in the Salary Cap Year in which the
    Extension is signed exceeds fifteen percent (15\%) of the player's
    Regular Salary for such Salary Cap Year, the Extension may provide
    for up to the same percentage of Unlikely Bonuses in the first year
    of the extended term; and (ii) no Renegotiation may provide for an
    increase in Unlikely Bonuses if, after the Renegotiation, the amount
    of Unlikely Bonuses in respect of any Salary Cap Year covered by the
    renegotiated Contract exceeds fifteen (15\%) of the player's Regular
    Salary for such Salary Cap Year.
  \item
    No Player Contract may provide for any Unlikely Bonus for the first
    Salary Cap Year covered by the Contract that, if included in the
    player's Salary for such Salary Cap Year, would result in the Team's
    Team Salary exceeding the Room under which it is signing the
    Contract. For the sole purpose of determining whether a Team has
    Room for a new Unlikely Bonus, the Team's Room shall be deemed
    reduced by all Unlikely Bonuses in Contracts approved by the
    Commissioner that may be paid to all of the Team's players that
    entered into Player Contracts (including Renegotiations) during that
    Salary Cap Year.
  \end{enumerate}
\item
  \textbf{No Futures Contracts.} Subject to Section 5(e)(4) below, but
  notwithstanding any other provision in this Agreement:

  \begin{enumerate}
  \def\labelenumii{(\arabic{enumii})}
  \item
    Every Player Contract must cover at least the then-current Season
    (or the upcoming Season in the case of a Contract entered into from
    July 1 through the day prior to the first day of the Season).
  \item
    No Team and player may enter into a Player Contract from the
    commencement of the Team's last game of the Regular Season through
    the following June 30. The preceding sentence shall not prohibit a
    Team and player from entering into an amendment to an existing
    Player Contract during such period if such amendment would otherwise
    be permitted under this Agreement.
  \item
    A Player Contract that covers more than one (1) Season must be for a
    consecutive period of Seasons.
  \item
    \begin{enumerate}
    \def\labelenumiii{(\roman{enumiii})}
    \tightlist
    \item
      A player who receives a Required Tender or a Qualifying Offer
      during the month of June may accept such Required Tender or
      Qualifying Offer beginning on the date he receives it.
    \item
      From February 1 through June 30 of any Salary Cap Year, a First
      Round Pick may enter into a Rookie Scale Contract commencing with
      the following Season, provided that as of or at any point
      following the first day of the then-current Regular Season (or the
      preceding Regular Season in the case of a Contract signed from the
      day following the last day of the Regular Season through June 30)
      the player was a party to a player contract with a professional
      basketball team not in the NBA covering such Regular Season.
    \end{enumerate}
  \end{enumerate}
\end{enumerate}

\section{Exceptions to the Salary
Cap.}\label{exceptions-to-the-salary-cap.}

There shall be the following exceptions to the rule that a Team's Team
Salary may not exceed the Salary Cap:

\begin{enumerate}
\def\labelenumi{(\alph{enumi})}
\tightlist
\item
  \textbf{Existing Contracts.} A Team may exceed the Salary Cap to the
  extent of its current contractual commitments, provided that such
  contracts satisfied the provisions of this Agreement when entered into
  or were entered into prior to the effective date of this Agreement in
  accordance with the rules then in effect.
\item
  \textbf{Veteran Free Agent Exception.} Subject to the rules set forth
  in Section 6(m) below, beginning at 12:01 p.m. eastern time on the
  last day of the Moratorium Period following the last Season covered by
  a Veteran Free Agent's Player Contract, such player may enter into a
  new Player Contract with his Prior Team (or, in the case of a player
  selected in an Expansion Draft that year, with the Team that selected
  such player in an Expansion Draft) as follows:

  \begin{enumerate}
  \def\labelenumii{(\arabic{enumii})}
  \item
    If the player is a Qualifying Veteran Free Agent, the new Player
    Contract may provide for Salary and Unlikely Bonuses in the first
    Salary Cap Year totaling up to the maximum amount provided for in
    Article II, Section 7.
  \item
    If the player is a Non-Qualifying Veteran Free Agent, then, subject
    to Article II, Section 7, the new Player Contract may provide in the
    first Salary Cap Year up to the greater of: (i) one hundred twenty
    percent (120\%) of the Regular Salary for the final Salary Cap Year
    of the player's prior Contract, plus one hundred twenty percent
    (120\%) of any Likely Bonuses and Unlikely Bonuses, respectively,
    called for in the final Salary Cap Year covered by the player's
    prior Contract; (ii) Salary plus Unlikely Bonuses totaling one
    hundred twenty percent (120\%) of the then-current Minimum Annual
    Salary applicable to the player; or (iii) in the case of a Contract
    between a Team and its Restricted Free Agent, the Salary and
    Unlikely Bonuses required to be provided in a Qualifying Offer.
    Annual increases and decreases in Salary and Unlikely Bonuses shall
    be governed by Section 5(c)(1) above.
  \item
    \begin{enumerate}
    \def\labelenumiii{(\roman{enumiii})}
    \tightlist
    \item
      If the player is an Early Qualifying Veteran Free Agent, the new
      Player Contract must cover at least two (2) Seasons (not including
      a Season covered by an Option Year) and, subject to Article II,
      Section 7, may provide in the first Salary Cap Year up to the
      greater of: (A) one hundred seventy-five percent (175\%) of the
      Regular Salary for the final Salary Cap Year covered by his prior
      Contract, plus one hundred seventy-five percent (175\%) of any
      Likely Bonuses and Unlikely Bonuses, respectively, called for in
      the final Salary Cap Year covered by the player's prior Contract,
      or (B) Salary plus Unlikely Bonuses totaling an amount equal to
      one hundred five percent (105\%) of the Average Player Salary for
      the prior Salary Cap Year (or if the Audit Report for the prior
      Salary Cap Year has not been completed, one hundred five percent
      (105\%) of the Average Player Salary for the prior Salary Cap Year
      as computed by substituting Estimated Total Salaries (as defined
      in Section 1(i) above) for Total Salaries).
    \item
      Notwithstanding anything to the contrary in Section 5(c)(2) above,
      if an Early Qualifying Veteran Free Agent with two (2) Years of
      Service receives an Offer Sheet in accordance with the provisions
      of Article XI, Section 5(c), the player's Prior Team may use the
      Early Qualifying Veteran Free Agent Exception to match the Offer
      Sheet.
    \end{enumerate}
  \end{enumerate}
\item
  \textbf{Disabled Player Exception.}

  \begin{enumerate}
  \def\labelenumii{(\arabic{enumii})}
  \tightlist
  \item
    Subject to the rules set forth in Section 6(m) below, a Team may, in
    accordance with the rules set forth in this Section 6(c), sign or
    acquire one Replacement Player to replace a player who, as a result
    of a Disabling Injury or Illness (as defined below), is unable to
    render playing services (the ``Disabled Player'').

    \begin{enumerate}
    \def\labelenumiii{(\roman{enumiii})}
    \tightlist
    \item
      An application for a Disabled Player Exception in respect of a
      Salary Cap Year, regardless of when the Disabling Injury or
      Illness occurred, may be made at any time from July 1 through
      January 15 of such Salary Cap Year.
    \item
      If a Team wishes to sign a Replacement Player pursuant to this
      Section 6(c), such Replacement Player's Contract may be for one
      Season and provide Salary and Unlikely Bonuses for the Salary Cap
      Year in which the player is signed totaling up to the lesser of
      (A) fifty percent (50\%) of the Disabled Player's Salary for the
      then-current Salary Cap Year, or (B) an amount equal to
      Non-Taxpayer Mid-Level Salary Exception (as defined in Section
      6(e) below) for such Salary Cap Year.
    \item
      If a Team wishes to acquire a Replacement Player pursuant to this
      Section 6(c), the Replacement Player must have only one Season
      remaining on his Player Contract and the Replacement Player's
      post-assignment Salary for the Salary Cap Year in which the
      Replacement Player is acquired may be up to the lesser of the
      amount described in Section 6(c)(1)(ii)(A) above or the amount
      described in Section 6(c)(1)(ii)(B) above, plus, in either case,
      \$100,000.
    \end{enumerate}
  \item
    For purposes of this Section 6(c), Disabling Injury or Illness means
    any injury or illness that, in the opinion of the physician
    described in Section 6(c)(4) below, makes it substantially more
    likely than not that the player would be unable to play through the
    following June 15.
  \item
    The Exception for a Disabling Injury or Illness shall expire on the
    March 10 following the date the Exception is granted.
  \item
    The determination of whether a player has suffered a Disabling
    Injury or Illness shall be made by a physician designated by the
    NBA, who shall review the relevant medical information and, if the
    physician deems it appropriate, examine the player. The NBA shall
    advise the Players Association of the determination of its physician
    within one (1) business day of such determination. In the event the
    Players Association disputes the NBA physician's determination, the
    parties will immediately refer the matter to a neutral physician (to
    be selected by the parties at the commencement of each Salary Cap
    Year) to review the relevant medical information and, if the neutral
    physician deems it appropriate, examine the player. Within three (3)
    business days of receipt of such information (and examination of the
    player, if requested), the neutral physician shall make a final
    determination, which will be final, binding and unappealable. The
    cost of the NBA physician will be borne by the NBA. The cost of the
    neutral physician will be borne equally and jointly by the NBA and
    the Players Association.
  \item
    If a Team requests an Exception pursuant to this Section 6(c), the
    player with respect to whom the request is made shall cooperate in
    the processing of the request, including by appearing at the
    scheduled place and time for examination by the NBA-appointed
    physician and, if necessary, the neutral physician. The player shall
    not make any misrepresentation or fail to disclose any relevant
    information in connection with the processing of the application.
  \item
    Notwithstanding a Team's receipt of an Exception in respect of a
    Disabled Player pursuant to this Section 6(c), such player, upon
    recovering from his injury or illness, may resume playing for the
    Team. If the player resumes playing for the Team, or is traded,
    prior to the Team's use of its Exception, the Exception shall be
    extinguished.
  \item
    The Disabled Player Exception is available only to the Team with
    which the player was under Contract at the time his Disabling Injury
    or Illness became known or reasonably should have become known. In
    order for a Team to apply for a Disabled Player Exception pursuant
    to this Section 6(c), the Disabled Player must continue to be on the
    Team's roster from the time the Team makes such application through
    the date upon which the Exception is granted.
  \item
    If a Team makes a request for an Exception to replace a Disabled
    Player pursuant to this Section 6(c) and such request is denied, the
    Team shall not be permitted to make any subsequent request for an
    Exception to replace the same player pursuant to this Section 6(c)
    unless ninety (90) days have passed since the first request was
    denied and the Team establishes that the subsequent request is based
    on a new injury or an aggravation of the same injury. If a Team
    makes a request for an Exception to replace a Disabled Player for a
    Season pursuant to this Section 6(c), then, whether such request is
    granted or denied, the Team shall be permitted to renew its request
    for an Exception to replace the Disabled Player for a subsequent
    Season(s) by applying for another Exception in respect of that
    player for such Season in accordance with the rules set forth in
    this Section 6(c).
  \end{enumerate}
\item
  \textbf{Bi-annual Exception.} Subject to the rules set forth in
  Section 6(m) below:

  \begin{enumerate}
  \def\labelenumii{(\arabic{enumii})}
  \tightlist
  \item
    A Team shall be permitted to use the Bi-annual Exception during a
    Salary Cap Year only if (i) the Team's Team Salary at the time the
    Exception is used and at all times thereafter during the Salary Cap
    Year does not exceed the Tax Level for such Salary Cap Year plus the
    Tax Apron Amount (as defined in Section 6(m)(3) below), and (ii) at
    the time the Exception is used, the Team has not already used either
    the Taxpayer Mid-Level Salary Exception or the Mid-Level Salary
    Exception for Room Teams in that same Salary Cap Year. During each
    Salary Cap Year in which a Team is permitted to use the Bi-annual
    Exception, a Team may sign one (1) or more Player Contracts, not to
    exceed two (2) Seasons in length, that, in the aggregate, provide
    for Salaries and Unlikely Bonuses in the first Salary Cap Year
    totaling up to the amounts set forth below:
  \end{enumerate}

  \begin{longtable}[]{@{}ll@{}}
  \toprule
  \begin{minipage}[b]{0.25\columnwidth}\raggedright\strut
  \strut
  \end{minipage} &
  \begin{minipage}[b]{0.64\columnwidth}\raggedright\strut
  Bi-annual Exception\strut
  \end{minipage}\tabularnewline
  \midrule
  \endhead
  \begin{minipage}[t]{0.25\columnwidth}\raggedright\strut
  For the 2017-18 Salary Cap Year:\strut
  \end{minipage} &
  \begin{minipage}[t]{0.64\columnwidth}\raggedright\strut
  \$3.290 million\strut
  \end{minipage}\tabularnewline
  \begin{minipage}[t]{0.25\columnwidth}\raggedright\strut
  For each subsequent Salary Cap Year through 2023-24:\strut
  \end{minipage} &
  \begin{minipage}[t]{0.64\columnwidth}\raggedright\strut
  The preceding Salary Cap Year's Bi-annual Exception amount adjusted by
  applying the percentage increase (or decrease) in the Salary Cap from
  the preceding Salary Cap Year to the current Salary Cap Year\strut
  \end{minipage}\tabularnewline
  \bottomrule
  \end{longtable}

  \begin{enumerate}
  \def\labelenumii{(\arabic{enumii})}
  \setcounter{enumii}{1}
  \tightlist
  \item
    A Team may not use all or any portion of the Bi-annual Exception to
    sign one (1) or more new Player Contracts in any two (2) consecutive
    Salary Cap Years. The prohibition in the preceding sentence against
    using the Bi-annual Exception or any portion thereof in any two (2)
    consecutive Salary Cap Years shall apply to the 2016-17 Salary Cap
    Year (i.e., if a Team used all or any portion of the Bi-annual
    Exception during the 2016-17 Salary Cap Year, that Team shall not be
    permitted to use all or any portion of the Bi-annual Exception
    during the 2017-18 Salary Cap Year).
  \item
    Player Contracts signed pursuant to the Bi-annual Exception covering
    two (2) Seasons may provide for an increase or decrease in Salary
    and Unlikely Bonuses for the second Salary Cap Year in accordance
    with Section 5(c)(1) above.
  \item
    The Bi-annual Exception, if applicable, shall arise on the first day
    of each Salary Cap Year and shall expire on the last day of the
    Team's Regular Season during that Salary Cap Year.
  \end{enumerate}
\item
  \textbf{Non-Taxpayer Mid-Level Salary Exception.} Subject to the rules
  set forth in Section 6(m) below:

  \begin{enumerate}
  \def\labelenumii{(\arabic{enumii})}
  \tightlist
  \item
    A Team shall be permitted to use the Non-Taxpayer Mid- Level Salary
    Exception during a Salary Cap Year only if (i) the Team's Team
    Salary at the time the Exception is used and at all times thereafter
    during the Salary Cap Year does not exceed the Tax Level for such
    Salary Cap Year plus the Tax Apron Amount, and (ii) at the time the
    Exception is used, the Team has not already used either the Taxpayer
    Mid-Level Salary Exception or the Mid-Level Exception for Room Teams
    in that same Salary Cap Year. A Team may use the Non-Taxpayer
    Mid-Level Salary Exception to sign one (1) or more Player Contracts
    during each Salary Cap Year, not to exceed four (4) Seasons in
    length, that, in the aggregate, provide for Salaries and Unlikely
    Bonuses in the first Salary Cap Year totaling up to the amounts set
    forth below:
  \end{enumerate}

  \begin{longtable}[]{@{}ll@{}}
  \toprule
  \begin{minipage}[b]{0.25\columnwidth}\raggedright\strut
  \strut
  \end{minipage} &
  \begin{minipage}[b]{0.64\columnwidth}\raggedright\strut
  Non-Taxpayer Mid-Level Salary Exception\strut
  \end{minipage}\tabularnewline
  \midrule
  \endhead
  \begin{minipage}[t]{0.25\columnwidth}\raggedright\strut
  For the 2017-18 Salary Cap Year:\strut
  \end{minipage} &
  \begin{minipage}[t]{0.64\columnwidth}\raggedright\strut
  \$8.406 million\strut
  \end{minipage}\tabularnewline
  \begin{minipage}[t]{0.25\columnwidth}\raggedright\strut
  For each subsequent Salary Cap Year through 2023-24:\strut
  \end{minipage} &
  \begin{minipage}[t]{0.64\columnwidth}\raggedright\strut
  The preceding Salary Cap Year's Non-Taxpayer Mid-Level Salary
  Exception amount adjusted by applying the percentage increase (or
  decrease) in the Salary Cap from the preceding Salary Cap Year to the
  current Salary Cap Year\strut
  \end{minipage}\tabularnewline
  \bottomrule
  \end{longtable}

  \begin{enumerate}
  \def\labelenumii{(\arabic{enumii})}
  \setcounter{enumii}{1}
  \tightlist
  \item
    Player Contracts signed pursuant to the Non-Taxpayer Mid-Level
    Salary Exception may provide for annual increases and decreases in
    Salary and Unlikely Bonuses in accordance with Section 5(c)(1)
    above.
  \item
    Notwithstanding anything to the contrary in Section 6(e)(2) above,
    if a Veteran Free Agent with one (1) or two (2) Years of Service
    receives an Offer Sheet in accordance with the provisions of Article
    XI, Section 5(c), the player's Prior Team may use the Non-Taxpayer
    Mid-Level Salary Exception to match the Offer Sheet.
  \item
    The Non-Taxpayer Mid-Level Salary Exception shall arise on the first
    day of each Salary Cap Year and shall expire on the last day of the
    Team's Regular Season during that Salary Cap Year.
  \end{enumerate}
\item
  \textbf{Taxpayer Mid-Level Salary Exception.} Subject to the rules set
  forth in Section 6(m) below:

  \begin{enumerate}
  \def\labelenumii{(\arabic{enumii})}
  \tightlist
  \item
    A Team shall be permitted to use the Taxpayer Mid-Level Salary
    Exception in a Salary Cap Year only if (i) the Team's Team Salary
    immediately following the Team's use of such Exception exceeds the
    Tax Level for such Salary Cap Year plus the Tax Apron Amount, and
    (ii) the Team has not already used either the Bi-annual Exception,
    the Non-Taxpayer Mid-Level Salary Exception, or the Mid-Level Salary
    Exception for Room Teams, or acquired a player pursuant to a
    Contract entered into in accordance with Section 8(e)(1), in that
    same Salary Cap Year; once a Team uses the Taxpayer Mid-Level Salary
    Exception during a Salary Cap Year, the Team will be prohibited from
    using either the Bi-annual Exception, the Non-Taxpayer Mid-Level
    Salary Exception, or the Mid-Level Salary Exception for Room Teams,
    or acquiring a player pursuant to a Contract entered into in
    accordance with Section 8(e)(1) at all times thereafter during such
    Salary Cap Year.
  \item
    A Team may use the Taxpayer Mid-Level Salary Exception to sign one
    (1) or more Player Contracts during each Salary Cap Year, not to
    exceed three (3) Seasons in length, that, in the aggregate, provide
    for Salaries and Unlikely Bonuses in the first Salary Cap Year
    totaling up to the amounts set forth below:
  \end{enumerate}

  \begin{longtable}[]{@{}ll@{}}
  \toprule
  \begin{minipage}[b]{0.25\columnwidth}\raggedright\strut
  \strut
  \end{minipage} &
  \begin{minipage}[b]{0.64\columnwidth}\raggedright\strut
  Taxpayer Mid-Level Salary Exception\strut
  \end{minipage}\tabularnewline
  \midrule
  \endhead
  \begin{minipage}[t]{0.25\columnwidth}\raggedright\strut
  For the 2017-18 Salary Cap Year:\strut
  \end{minipage} &
  \begin{minipage}[t]{0.64\columnwidth}\raggedright\strut
  \$5.192 million\strut
  \end{minipage}\tabularnewline
  \begin{minipage}[t]{0.25\columnwidth}\raggedright\strut
  For each subsequent Salary Cap Year through 2023-24:\strut
  \end{minipage} &
  \begin{minipage}[t]{0.64\columnwidth}\raggedright\strut
  The preceding Salary Cap Year's Taxpayer Mid-Level Salary Exception
  amount adjusted by applying the percentage increase (or decrease) in
  the Salary Cap from the preceding Salary Cap Year to the current
  Salary Cap Year\strut
  \end{minipage}\tabularnewline
  \bottomrule
  \end{longtable}

  \begin{enumerate}
  \def\labelenumii{(\arabic{enumii})}
  \setcounter{enumii}{2}
  \tightlist
  \item
    Player Contracts signed pursuant to the Taxpayer Mid-Level Salary
    Exception may provide for annual increases and decreases in Salary
    and Unlikely Bonuses in accordance with Section 5(c)(1) above.
  \item
    The Taxpayer Mid-Level Salary Exception shall arise on the first day
    of each Salary Cap Year and shall expire on the last day of the
    Team's Regular Season during that Salary Cap Year.
  \item
    In the event that a Team uses the Non-Taxpayer Mid-level Salary
    Exception in order to sign one (1) or more new Player Contracts
    during a Salary Cap Year, not to exceed three (3) Seasons in length
    that, in the aggregate, provide for Salaries and Unlikely Bonuses in
    the first Salary Cap Year of the Contract(s) totaling no more than
    the amounts set forth in Section 6(f)(2) above, and, but for the
    Team's use of the Non-Taxpayer Mid-Level Salary Exception as
    described above, the Team otherwise would be permitted to engage in
    a transaction that causes the Team's Team Salary to exceed the Tax
    Level for such Salary Cap Year plus the Tax Apron Amount in
    accordance with the rules set forth in this Article VII, then the
    Team shall be permitted to engage in such transaction, whereupon the
    Team will be deemed to have used the Taxpayer Mid-Level Salary
    Exception instead of the Non-Taxpayer Mid-Level Salary Exception for
    all purposes under this Article VII, and the Team's ability to use
    the Non-Taxpayer Mid-Level Salary Exception during such Salary Cap
    Year shall thereupon be extinguished.
  \end{enumerate}
\item
  \textbf{Mid-Level Salary Exception for Room Teams.} Subject to the
  rules set forth in Section 6(m) below:

  \begin{enumerate}
  \def\labelenumii{(\arabic{enumii})}
  \tightlist
  \item
    In the event (i) a Team's Team Salary at any time during a Salary
    Cap Year is below the Salary Cap for such Salary Cap Year such that
    the Team is not entitled to use the Bi-annual Exception,
    Non-Taxpayer Mid-Level Salary Exception, or Taxpayer Mid-Level
    Salary Exception, and (ii) at the time the Team proposes to use the
    Mid-Level Salary Exception for Room Teams, the Team has not already
    used either the Bi-annual Exception, the Non-Taxpayer Mid-Level
    Salary Exception, or the Taxpayer Mid-Level Salary Exception in that
    same Salary Cap Year, then the Team may at such time use the
    Mid-Level Salary Exception for Room Teams to sign one (1) or more
    Player Contracts, not to exceed two (2) Seasons in length, that, in
    the aggregate, provide for Salaries and Unlikely Bonuses in the
    first Salary Cap Year totaling up to the amounts set forth below:
  \end{enumerate}

  \begin{longtable}[]{@{}ll@{}}
  \toprule
  \begin{minipage}[b]{0.25\columnwidth}\raggedright\strut
  \strut
  \end{minipage} &
  \begin{minipage}[b]{0.64\columnwidth}\raggedright\strut
  Mid-Level Salary Exception for Room Teams\strut
  \end{minipage}\tabularnewline
  \midrule
  \endhead
  \begin{minipage}[t]{0.25\columnwidth}\raggedright\strut
  For the 2017-18 Salary Cap Year:\strut
  \end{minipage} &
  \begin{minipage}[t]{0.64\columnwidth}\raggedright\strut
  \$4.328 million\strut
  \end{minipage}\tabularnewline
  \begin{minipage}[t]{0.25\columnwidth}\raggedright\strut
  For each subsequent Salary Cap Year through 2023-24:\strut
  \end{minipage} &
  \begin{minipage}[t]{0.64\columnwidth}\raggedright\strut
  The preceding Salary Cap Year's Mid-Level Salary Exception for Room
  Teams amount adjusted by applying the percentage increase (or
  decrease) in the Salary Cap from the preceding Salary Cap Year to the
  current Salary Cap Year\strut
  \end{minipage}\tabularnewline
  \bottomrule
  \end{longtable}

  \begin{enumerate}
  \def\labelenumii{(\arabic{enumii})}
  \setcounter{enumii}{1}
  \tightlist
  \item
    Once a Team uses the Mid-Level Salary Exception for Room Teams
    during a Salary Cap Year, the Team will be prohibited from using
    either the Non-Taxpayer Mid-Level Salary Exception, the Taxpayer
    Mid-Level Salary Exception, or the Bi-annual Exception at all times
    thereafter during such Salary Cap Year.
  \item
    Player Contracts signed pursuant to the Mid-Level Salary Exception
    for Room Teams may provide for annual increases and decreases in
    Salary and Unlikely Bonuses in accordance with Section 5(c)(1)
    above.
  \item
    The Mid-Level Salary Exception for Room Teams shall: (i) arise on
    the date upon which the Team's Team Salary falls below the Salary
    Cap for such Salary Cap Year such that the Team is not entitled to
    use the Bi-annual Exception, the Non-Taxpayer Mid-Level Salary
    Exception, and the Taxpayer Mid-Level Salary Exception; and (ii)
    expire on the last day of the Team's Regular Season during that
    Salary Cap Year.
  \end{enumerate}
\item
  \textbf{Rookie Exception.} A Team may enter into a Rookie Scale
  Contract in accordance with Article VIII, Section 1.
\item
  \textbf{Minimum Player Salary Exception.} A Team may sign a player to,
  or acquire by assignment, a Player Contract, not to exceed two (2)
  Seasons in length, that provides for a Salary for the first Season
  equal to the Minimum Player Salary applicable to that player (with no
  bonuses of any kind). A Player Contract signed or acquired pursuant to
  the Minimum Player Salary Exception covering two (2) Seasons must
  provide for a Salary for the second Season equal to the Minimum Player
  Salary applicable to the player for such Season (with no bonuses of
  any kind).
\item
  \textbf{Traded Player Exception.}

  \begin{enumerate}
  \def\labelenumii{(\arabic{enumii})}
  \tightlist
  \item
    Subject to the rules set forth in Section 6(m) below and Section
    6(j)(5) below, a Team may, for a period of one (1) year following
    the date of the trade of a Player Contract to another Team, replace
    the Traded Player with one (1) or more players acquired by
    assignment as follows:

    \begin{enumerate}
    \def\labelenumiii{(\roman{enumiii})}
    \tightlist
    \item
      A Team whose post-assignment Team Salary would exceed the Tax
      Level for the then-current Salary Cap Year may replace a Traded
      Player with one (1) or more Replacement Players whose Player
      Contracts are acquired simultaneously and whose post-assignment
      Salaries for the then-current Salary Cap Year, in the aggregate,
      are no more than an amount equal to one hundred twenty-five
      percent (125\%) of the pre-trade Salary of the Traded Player, plus
      \$100,000. A Team whose post-assignment Team Salary would be equal
      to or less than the Tax Level for the then-current Salary Cap Year
      may replace a Traded Player with one (1) or more Replacement
      Players whose Player Contracts are acquired simultaneously and
      whose post-assignment Salaries for the then-current Salary Cap
      Year, in the aggregate, are no more than an amount equal to the
      greater of: (y) the lesser of: (A) one hundred seventy-five
      (175\%) of the pre-trade Salary of the Traded Player, plus
      \$100,000; or (B) one hundred percent (100\%) of the pre-trade
      Salary of the Traded Player, plus \$5 million; or (z) one hundred
      twenty-five percent (125\%) of the pre-trade Salary of the Traded
      Player, plus \$100,000.
    \item
      If a Team's trade of a player and acquisition of one (1) or more
      Replacement Players do not occur simultaneously, then the
      post-assignment Salary or aggregate Salaries of the Replacement
      Player(s) for the Salary Cap Year in which the Replacement
      Player(s) are acquired may not exceed one hundred percent (100\%)
      of the pre-trade Salary of the Traded Player at the time the
      Traded Player's Contract was traded, plus \$100,000.
    \item
      A Team may aggregate the pre-trade Salaries in two (2) or more
      Player Contracts for the purpose of acquiring in a simultaneous
      trade one (1) or more Replacement Players whose post-trade
      Salaries, in the aggregate, are no more than an amount equal to:
      for Teams whose post-assignment Team Salary would exceed the Tax
      Level for the then-current Salary Cap Year, one hundred
      twenty-five percent (125\%) of the pre-trade aggregated Salaries
      of the Traded Players, plus \$100,000; and for Teams whose
      post-assignment Team Salary would be equal to or less than the Tax
      Level for the then-current Salary Cap Year, the greater of: (y)
      the lesser of: (A) one hundred seventy-five (175\%) of the
      pre-trade aggregated Salaries of the Traded Players, plus
      \$100,000; or (B) one hundred percent (100\%) of the pre-trade
      aggregated Salaries of the Traded Players, plus \$5 million; or
      (z) one hundred twenty-five percent (125\%) of the pre-trade
      aggregated Salaries of the Traded Players, plus \$100,000.
      Notwithstanding the preceding sentence, no Player Contract
      acquired pursuant to an Exception may, for a period of two (2)
      months from the date the Player Contract is acquired, be
      aggregated with any other Contract for purposes of a trade. For
      example, if a player were traded to a Team pursuant to an
      Exception on December 20, 2017, then the player's Contract could
      not be aggregated with any other Contract for purposes of a trade
      until February 20, 2018.
    \end{enumerate}
  \item
    Except as provided in Section 6(j)(3) below, and notwithstanding
    Section 6(m) below, a Team with a Team Salary below the Salary Cap
    may acquire one (1) or more players by assignment whose
    post-assignment Salaries, in the aggregate, are no more than an
    amount equal to the Team's Room plus \$100,000.
  \item
    In lieu of conducting the trade in accordance with Section 6(j)(2)
    above, and notwithstanding Section 6(m) below and subject to Section
    6(j)(5) below, a Team with a Team Salary below the Salary Cap may
    (i) replace a Traded Player with one (1) or more Replacement Players
    whose Player Contracts are acquired simultaneously and whose
    post-trade Salaries for the then-current Season, in the aggregate,
    are no more than an amount equal to the greater of: (w) the lesser
    of: (A) one hundred seventy-five (175\%) of the pre-trade Salary of
    the Traded Player, plus \$100,000; or (B) one hundred percent
    (100\%) of the pre-trade Salary of the Traded Player, plus \$5
    million; or (x) one hundred twenty-five percent (125\%) of the
    pre-trade Salary of the Traded Player, plus \$100,000; or (ii)
    aggregate the pre-trade Salaries in two (2) or more Player Contracts
    for the purpose of acquiring in a simultaneous trade one (1) or more
    Replacement Players whose post-trade Salaries, in the aggregate, are
    no more than an amount equal to the greater of: (y) lesser of: (C)
    one hundred seventy-five (175\%) of the pre-trade aggregated
    Salaries of the Traded Players, plus \$100,000; or (D) the pre-trade
    aggregated Salaries of the Traded Players, plus \$5 million; or (z)
    one hundred twenty-five percent (125\%) of the pre-trade aggregated
    Salaries of the Traded Players, plus \$100,000. Notwithstanding the
    preceding sentence, no Player Contract acquired pursuant to an
    Exception may, for a period of two months from the date the Player
    Contract is acquired, be aggregated with any other Contract for
    purposes of a trade in accordance with this Section 6(j)(3).
  \item
    If a Qualifying Veteran Free Agent or Early Qualifying Veteran Free
    Agent and his prior Team enter into a Player Contract, in accordance
    with Section 6(b)(1) or (3) above, in connection with an agreement
    to trade the Contract in accordance with Section 8(e) below, the
    Team's Team Salary immediately following such Contract signing is
    above the Salary Cap, and the new Contract to be traded provides for
    a Salary for the first Season of such new Contract greater than one
    hundred twenty percent (120\%) of the Salary for the last Season of
    the player's immediately prior Contract (and greater than the
    Minimum Player Salary with no Unlikely Bonuses), then for purposes
    of calculating the assignor Team's Traded Player Exception, the
    player's Salary shall be deemed equal to the greater of (i) the
    Salary for the last Season of his preceding Contract, or (ii) fifty
    percent (50\%) of the Salary for the first Season of his new
    Contract. For purposes of this Section 6(j)(4), if the player's
    immediately prior Contract was a one-year Contract that provided for
    Salary equal to the Minimum Player Salary (with no Unlikely
    Bonuses), the player's prior Salary shall include the portion of the
    Minimum Player Salary, if any, that was reimbursed out of the
    League-wide benefits fund described in Article IV, Section 6(g).
  \item
    With respect to Player Contracts entered into or extended beginning
    with the 2017-18 Season (but in the case of Extensions, only with
    respect to the extended term), for purposes of calculating a Team's
    Traded Player Exception under this Section 6(j):

    \begin{enumerate}
    \def\labelenumiii{(\roman{enumiii})}
    \item
      A Traded Player's Salary shall be deemed reduced by the amount of
      the player's unearned Base Compensation that is not
      fully-protected for lack of skill and injury or illness at the
      time of the trade.
    \item
      For purposes of Section 6(j)(5)(i) above, with respect to the
      assignment of Player Contracts occurring during the period from
      January 10 through the last day of the Regular Season, a Traded
      Player's Base Compensation for such Season shall be deemed
      fully-protected for lack of skill and injury or illness.
    \item
      With respect to the assignment of Player Contracts occurring
      during the period from the day following the last day of a Regular
      Season through June 30 of that Salary Cap Year, a Traded Player's
      Salary will equal the lesser of: (x) the player's Salary for the
      current Salary Cap Year; and (y) the player's Salary for the
      subsequent Salary Cap Year reduced by the amount of the player's
      unearned Base Compensation that is not fully-protected for lack of
      skill and injury or illness for the subsequent Salary Cap Year at
      the time of the trade.

      To illustrate the foregoing, assume that a Team seeks to replace a
      Traded Player whose (i) Base Compensation and Salary for both the
      current and subsequent Seasons is \$8 million, and (ii) Contract
      provides for Base Compensation protection for lack of skill and
      injury or illness equal to \$1 million in both the current and
      subsequent Seasons. If the trade of such Traded Player occurs on:

      \begin{enumerate}
      \def\labelenumiv{(\Alph{enumiv})}
      \setcounter{enumiv}{22}
      \tightlist
      \item
        the day prior to the first day of the current Regular Season,
        the Traded Player's Salary for purposes of calculating the
        Team's Traded Player Exception under this Section 6(j) would be
        \$1 million (\$8 million (the player's Salary) reduced by \$7
        million (the amount of the player's unearned Base Compensation
        that is not fully-protected for lack of skill and injury or
        illness at the time of the trade));
      \item
        after one-quarter of the current Regular Season has elapsed, the
        Traded Player's Salary for purposes of calculating the Team's
        Traded Player Exception under this Section 6(j) would be \$2
        million (\$8 million (the player's Salary) reduced by \$6
        million (\$8 million multiplied by 75\% -- the amount of the
        player's unearned Base Compensation that is not fully-protected
        for lack of skill and injury or illness at the time of the
        trade));
      \item
        on January 10 of the current Regular Season, the Traded Player's
        Salary for purposes of calculating the Team's Traded Player
        Exception under this Section 6(j) would be \$8 million (\$8
        million (the player's Salary) reduced by \$0 (pursuant to
        Section 6(j)(5)(i) above, the deemed amount of the player's
        unearned Base Compensation that is not fully-protected for lack
        of skill and injury or illness at the time of the trade)); and
      \item
        on the day following the last day of the current Regular Season,
        the Traded Player's Salary for purposes of calculating the
        Team's Traded Player Exception under this Section 6(j) would be
        \$1 million (the lesser of: (i) \$8 million (\$8 million (the
        player's Salary for the current Salary Cap Year) reduced by \$0
        (the amount of the player's unearned Base Compensation that is
        not fully-protected for lack of skill and injury or illness at
        the time of the trade for the current Salary Cap Year)), and
        (ii) \$1 million (\$8 million (the player's Salary for the
        subsequent Salary Cap Year) reduced by \$7 million (the amount
        of the player's unearned Base Compensation that is not
        fully-protected for lack of skill and injury or illness at the
        time of the trade for the subsequent Salary Cap Year)).
      \end{enumerate}
    \end{enumerate}
  \item
    The foregoing rules in this Section 6(j) shall not apply to Two-Way
    Players. Accordingly, for example, a Traded Player Exception will
    not arise from trading a Two-Way Player.
  \end{enumerate}
\item
  \textbf{Reinstatement.} If a player has been dismissed and
  disqualified from further association with the NBA and subsequently
  reinstated pursuant to Article XXXIII (Anti-Drug Agreement), the Team
  for which the player last played may enter into a Player Contract with
  such player in accordance with the applicable rules set forth in
  Article XXXIII, Section 12(f) or (g), even if the Team has a Team
  Salary at or above the Salary Cap or such Player Contract causes the
  Team to have a Team Salary above the Salary Cap. If, in accordance
  with the preceding sentence, a Team and a player enter into a Player
  Contract and such Contract covers more than one (1) Season, annual
  increases and decreases in Salary and Unlikely Bonuses shall be
  governed by Section 5(c)(1) above.
\item
  \textbf{Non-Aggregation.} Other than in accordance with Section 6(j)
  above, a Team may not aggregate or combine any of the Exceptions set
  forth above in order to sign or acquire one (1) or more players at
  Salaries greater than that permitted by any one of the Exceptions. If
  a Team has more than one (1) Exception available at the same time, the
  Team shall have the right to choose which Exception it wishes to use
  to sign or acquire a player.
\item
  \textbf{Other Rules.}

  \begin{enumerate}
  \def\labelenumii{(\arabic{enumii})}
  \tightlist
  \item
    A Team shall be entitled to use the Disabled Player Exception,
    Bi-annual Exception, Non-Taxpayer Mid-Level Salary Exception,
    Taxpayer Mid-Level Salary Exception and Traded Player Exception set
    forth in Section 6(c), (d), (e), (f) and (j) above, respectively,
    except as set forth in Section 6(j)(2) and (3) above, only if, at
    the time any such Exception would arise and at all times until it is
    used, the Team's Team Salary, excluding the amount(s) of such
    Exception and any other Exception that would be included in Team
    Salary pursuant to Section 6(m)(2) below, is (i) at or above the
    Salary Cap, or (ii) below the Salary Cap by less than the amount(s)
    of the Team's Exception(s) (excluding the amount of the Taxpayer
    Mid-Level Salary Exception unless the Team is no longer able to use
    the Non-Taxpayer Mid-Level Salary Exception but remains able to use
    the Taxpayer Mid-Level Salary Exception, in which case the amount of
    the Taxpayer Mid-Level Salary Exception shall be included).
  \item
    In the event that when a Disabled Player Exception, Bi-annual
    Exception, Non-Taxpayer Mid-Level Salary Exception (or the Taxpayer
    Mid-Level Salary Exception instead of the Non-Taxpayer Mid-Level
    Salary Exception if the Team is no longer able to use the
    Non-Taxpayer Mid-Level Salary Exception but remains able to use the
    Taxpayer Mid-Level Salary Exception) and/or Traded Player Exception
    arises, the Team's Team Salary is below the Salary Cap (or in the
    event that, prior to the expiration of any such Exceptions, the
    Team's Team Salary falls below the Salary Cap) by less than the
    amount of such Exceptions, then (i) the Team's Team Salary shall
    include, until the Exceptions are actually used or until the Team no
    longer is entitled to use the Exceptions, the amount of the
    Exceptions (or any unused portion of the Exceptions), and (ii) the
    amount by which the Team's Team Salary is less than the Salary Cap
    shall thereby be extinguished. When the Disabled Player Exception is
    used to sign or acquire a player, the Replacement Player's Salary
    for the Season covered by his Contract, instead of the amount of the
    Exception, shall be included in Team Salary. When a Bi-annual
    Exception, Non-Taxpayer Mid-Level Salary Exception or Taxpayer
    Mid-Level Salary Exception is used to sign a player, or when a
    Traded Player Exception is used to acquire a player, the Salary for
    the first Season of the signed or acquired Contract plus any
    then-unused portion of the Exception, instead of the full amount of
    the Exception, shall be included in Team Salary. A Team may at any
    time renounce its rights to use an Exception, in which case the
    Exception (or any unused portion of the Exception) will no longer be
    included in Team Salary.
  \item
    For purposes of Section 6(d), Section 6(e), Section 6(f) and Section
    8(e) below: (i) The Tax Apron Amount for the 2017-18 Salary Cap Year
    shall equal \$6 million and the Tax Apron Amount for each subsequent
    Salary Cap Year shall equal the preceding Salary Cap Year's Tax
    Apron Amount adjusted by applying one-half the percentage increase
    (or decrease) in the Salary Cap from the preceding Salary Cap Year
    to the current Salary Cap Year; and (ii) a Team's Team Salary for
    any Salary Cap Year shall be adjusted as follows:

    \begin{enumerate}
    \def\labelenumiii{(\Alph{enumiii})}
    \tightlist
    \item
      Team Salary shall include all Performance Bonuses excluded from a
      player's Salary under Section 3(d) above;
    \item
      Team Salary shall include the Salary attributable to a Contract
      signed by a Free Agent with zero (0) Years of Service or one (1)
      Year of Service provided for in Section 12(f)(2) below;
    \item
      Team Salary shall include any amount that could be added to the
      Team's Team Salary for such Salary Cap Year pursuant to Section
      4(a)(1)(iii) above;
    \item
      Team Salary shall exclude Free Agent Amounts as described in
      Section 4(a)(2); provided, however, that with respect to any
      Restricted Free Agent, Team Salary shall include the greater of
      (i) the Salary plus Unlikely Bonuses called for in any outstanding
      Qualifying Offer tendered to the player, or (ii) the Salary plus
      Unlikely Bonuses called for in any First Refusal Exercise Notice
      (as defined in Article XI, Section 5(e)) issued with respect to
      such player;
    \item
      Team Salary shall exclude amounts with respect to unsigned First
      Round Picks described in Section 4(a)(4) above, and shall include
      the amount of any outstanding Required Tender to a First Round
      Pick;
    \item
      Team Salary shall exclude the amount of any Salary Cap Exception
      that is deemed included in Team Salary pursuant to Sections
      4(a)(7) and 6(m)(2) above; and
    \item
      Team Salary shall exclude the amount of any incomplete roster cap
      hold amount added to the Team's Salary pursuant to Section 4(f)
      above.
    \end{enumerate}
  \item
    Notwithstanding anything to the contrary in this Agreement, if a
    player is a Veteran Free Agent following the second or third Season
    of his Rookie Scale Contract (where the first Option Year or second
    Option Year (as applicable) to extend such Contract was not
    exercised), then any new Player Contract between the player and the
    Team that signed him to his Rookie Scale Contract (and/or, if such
    Contract was subsequently assigned, any such assignee Team) may
    provide for Regular Salary, Likely Bonuses and Unlikely Bonuses in
    the first Salary Cap Year of up to the Regular Salary, Likely
    Bonuses and Unlikely Bonuses, respectively, that the player would
    have received for such Salary Cap Year had his first or second
    Option Year (as applicable) been exercised. Annual increases and
    decreases in Salary and Unlikely Bonuses shall be governed by
    Section 5(c)(2) above.
  \item
    Beginning on January 10 of each Season, each unused Exception, other
    than the Traded Player Exception, the Minimum Player Salary
    Exception (which is governed by Section 6(i) above and Article I,
    Section 1(ll)) and the Disabled Player Exception, shall be reduced
    daily throughout the end of the Regular Season by the amount of the
    unused Exception as of January 10 multiplied by a fraction, the
    numerator of which is one (1) and the denominator of which is the
    total number of days in such Regular Season.
  \end{enumerate}
\end{enumerate}

\section{Extensions, Renegotiations and Other
Amendments.}\label{extensions-renegotiations-and-other-amendments.}

\begin{enumerate}
\def\labelenumi{(\alph{enumi})}
\tightlist
\item
  \textbf{Veteran Extensions.} No Player Contract, other than a Rookie
  Scale Contract, may be extended except in accordance with the
  following:

  \begin{enumerate}
  \def\labelenumii{(\arabic{enumii})}
  \item
    Subject to the rules set forth in Section 7(a)(2) below: (i) a
    Player Contract covering a term of three (3) or four (4) Seasons
    (including any Option Year) may be extended no sooner than the
    second anniversary of the signing (or, as applicable, the extension)
    of the Contract; and (ii) a Player Contract covering a term of five
    (5) or six (6) Seasons may be extended no sooner than the third
    anniversary of the signing (or, as applicable, the extension) of the
    Contract. A Player Contract covering a term of one (1) or two (2)
    Seasons (including any Option Year) may not be extended. If a player
    and Team seek to enter into an Extension pursuant to this Section
    7(a) (other than a Designated Veteran Player Extension in accordance
    with Section 7(a)(3)(ii) below) more than one (1) year prior to the
    July 1 preceding the first Season covered by the extended term, then
    the Extension may only be negotiated and entered into during the
    off-season (i.e., during the period from July 1 through the day
    prior to the first day of a Regular Season). Notwithstanding the
    foregoing, a Player Contract may be extended pursuant to the
    Designated Veteran Player Extension rules set forth in Article II,
    Section 7 and Section 7(a)(3)(ii) below no sooner than the third
    anniversary of the signing of the Contract, and Designated Veteran
    Player Extensions may only be negotiated and entered into during the
    off-season. For purposes of determining the second or third
    anniversary of the signing of an Extension pursuant to this Section
    7(a)(1), Extensions entered into during the period from October 2
    through the day prior to the first day of the Regular Season (or,
    for Extensions entered into prior to the execution of this
    Agreement, during the period from October 2 through November 2) of a
    Salary Cap Year shall be deemed to have been signed on October 1 of
    such Salary Cap Year.
  \item
    \begin{enumerate}
    \def\labelenumiii{(\roman{enumiii})}
    \tightlist
    \item
      A Player Contract that has been renegotiated to provide for an
      increase in Salary in any Salary Cap Year covered by the Contract
      of more than ten percent (10\%) of the player's Salary prior to
      the Renegotiation, may not subsequently be extended until the
      third anniversary of the signing of such Renegotiation.
    \item
      A Team and a player shall not be permitted to extend any Player
      Contract with a term that has been shortened as a result of the
      player's exercise of an Early Termination Option.
    \item
      Subject to the rules set forth in this Section 7(a): (i) a
      Contract may be extended following the exercise of an Option by a
      player or Team; and (ii) a Contract may be extended following the
      non-exercise of an Option by a player or Team only if:

      \begin{enumerate}
      \def\labelenumiv{(\Alph{enumiv})}
      \tightlist
      \item
        The extended term covers a minimum of two (2) Seasons (excluding
        any new Option Year); and
      \item
        The player's Regular Salary, Likely Bonuses and Unlikely Bonuses
        in the first Salary Cap Year covered by the extended term are no
        less than the Regular Salary, Likely Bonuses and Unlikely
        Bonuses, respectively, that the player would have received for
        such Salary Cap Year had the Option been exercised. In order to
        effectuate an Extension of the types described in this Section
        7(a)(2)(iii), a Team and player may amend a Contract to provide
        simultaneously for the (i) exercise or non-exercise (as
        applicable) of the Option, and (ii) the Extension.
      \end{enumerate}
    \end{enumerate}
  \item
    \begin{enumerate}
    \def\labelenumiii{(\roman{enumiii})}
    \tightlist
    \item
      Subject to Article II, Section 7, a Player Contract extended in
      accordance with this Section 7(a) (other than a Designated Veteran
      Player Extension) may, in the first Salary Cap Year covered by the
      extended term, provide for a Salary, excluding Incentive
      Compensation, of up to the greater of: (A) one hundred twenty
      percent (120\%) of the Regular Salary in the last Salary Cap Year
      covered by the original term of the Contract; or (B) one hundred
      twenty percent (120\%) of the Estimated Average Player Salary for
      the Salary Cap Year in which the Extension is signed (or, if the
      Extension provides for any Incentive Compensation in the first
      Salary Cap Year covered by the extended term, then one hundred
      twenty percent (120\%) of the Estimated Average Player Salary for
      such Salary Cap Year less the amount of such Incentive
      Compensation). In the event that the last Salary Cap Year covered
      by the original term of the Contract provides for Incentive
      Compensation, the first Salary Cap Year covered by the extended
      term may provide for Likely Bonuses and Unlikely Bonuses of up to
      one hundred twenty percent (120\%) of the Likely Bonuses and
      Unlikely Bonuses, respectively, in the last Salary Cap Year
      covered by the original term. Annual increases and decreases in
      Salary and Unlikely Bonuses shall be governed by Section 5(c)(3)
      above.
    \item
      Notwithstanding Section 7(a)(3)(i) above, a Designated Veteran
      Player Extension may provide for a Salary in the first Salary Cap
      year covered by the extended term totaling no more than the
      maximum amount provided for in Article II, Section 7. Annual
      increases and decreases in Salary and Unlikely Bonuses shall be
      governed by Section 5(c)(3) above.
    \item
      Notwithstanding Section 7(a)(3)(i) or (ii) above, an Extension
      entered into in connection with a trade pursuant to Section
      8(e)(2) below may, in the first Salary Cap Year covered by the
      extended term, provide for a Salary, excluding Incentive
      Compensation, of up to one hundred five percent (105\%) of the
      Regular Salary in the last Salary Cap Year covered by the original
      term of the Contract. In the event that the last Salary Cap Year
      covered by the original term of the Contract provides for
      Incentive Compensation, the first Salary Cap Year covered by the
      extended term may provide for Likely Bonuses and Unlikely Bonuses
      of up to one hundred five percent (105\%) of the Likely Bonuses
      and Unlikely Bonuses, respectively, in the last Salary Cap Year
      covered by the original term. Annual increases and decreases in
      Salary and Unlikely Bonuses shall be governed by Section 5(c)(4)
      above.
    \item
      Notwithstanding anything to the contrary in this Agreement, a
      player who will not be a Qualifying Veteran Free Agent at the
      conclusion of his Contract will not be eligible to enter into an
      Extension pursuant to this Section 7(a).
    \end{enumerate}
  \item
    Notwithstanding Section 7(a)(3) above and subject to Article II,
    Section 7, any Player Contract of a player who has played for his
    current Team for at least ten (10) Seasons and whose Salary in the
    last Salary Cap Year covered by the original term of the Contract is
    less than the Salary in the second-to-last Salary Cap Year covered
    by such Contract may, in the first Salary Cap Year covered by an
    extended term, provide for a Salary equal to one hundred seven and
    one-half percent (107.5\%) of the greater of (i) the average of the
    Regular Salaries for each Salary Cap Year covered by the original
    Contract beginning with the Salary Cap Year in which such Contract
    was entered into, or previously extended, as the case may be, or
    (ii) the Regular Salary in the last Salary Cap Year covered by his
    original Contract. In the event that the last Salary Cap Year
    covered by the original term of the Contract provides for Incentive
    Compensation, the first Salary Cap Year covered by the extended term
    may provide for Likely Bonuses and Unlikely Bonuses of up to one
    hundred seven and one-half percent (107.5\%) of the Likely Bonuses
    and Unlikely Bonuses, respectively, in the last Salary Cap Year
    covered by the original term. Annual increases and decreases in
    Salary and Unlikely Bonuses shall be governed by Section 5(c)(3)
    above.
  \end{enumerate}
\item
  \textbf{Rookie Scale Extensions.} No Rookie Scale Contract may be
  extended except in accordance with the following:

  \begin{enumerate}
  \def\labelenumii{(\arabic{enumii})}
  \tightlist
  \item
    A First Round Pick who enters into a Rookie Scale Contract may enter
    into an Extension of such Rookie Scale Contract during the period
    from 12:01 p.m. eastern time on the last day of the Moratorium
    Period through 6:00 p.m. eastern time on the day prior to the first
    day of the Regular Season of the second Option Year provided for in
    such Contract (assuming the Team exercises such Option).
  \item
    An Extension of a Rookie Scale Contract may provide for Salary and
    Unlikely Bonuses in the first Salary Cap Year covered by the
    extended term totaling no more than the maximum amount provided for
    in Article II, Section 7. Annual increases and decreases in Salary
    and Unlikely Bonuses shall be governed by Section 5(c)(3) above.
  \item
    Notwithstanding anything to the contrary in this Agreement, a player
    who will not be a Qualifying Veteran Free Agent at the conclusion of
    his Rookie Scale Contract will not be eligible to enter into an
    Extension of a Rookie Scale Contract pursuant to this Section 7(b).
  \end{enumerate}
\item
  \textbf{Renegotiations.} No Player Contract may be renegotiated except
  in accordance with the following:

  \begin{enumerate}
  \def\labelenumii{(\arabic{enumii})}
  \tightlist
  \item
    Subject to Section 7(c)(2) and (3) below, a Player Contract covering
    a term of four (4) or more Seasons may be renegotiated no sooner
    than the third anniversary of the signing of the Contract.
  \item
    Subject to Section 7(c)(3) below, any Player Contract that has been
    renegotiated in accordance with Section 7(c)(1) above to provide for
    an increase in Salary or Incentive Compensation in any Salary Cap
    Year covered by the Contract of more than five percent (5\%), or
    extended in accordance with Section 7(a) or (b) above, may not
    subsequently be renegotiated until the third anniversary of the
    signing of such Extension or Renegotiation.
  \item
    Assuming Section 7(c) (1) or (2) above are satisfied, a Team with a
    Team Salary below the Salary Cap may renegotiate a Player Contract
    in accordance with the following rules:

    \begin{enumerate}
    \def\labelenumiii{(\roman{enumiii})}
    \tightlist
    \item
      Subject to Article II, Section 7, the Renegotiation may provide
      for additional Regular Salary, Likely Bonuses and/or Unlikely
      Bonuses for the then-current Salary Cap Year covered by the
      Contract (the ``Renegotiation Season'') that, in the aggregate,
      would not exceed the Team's Room at the time of the Renegotiation.
    \item
      Every category (Regular Salary, Likely Bonuses and Unlikely
      Bonuses, respectively) that is increased for the Renegotiation
      Season must also be increased for each of the remaining Seasons of
      the Contract. For each Season of the Contract after the
      Renegotiation Season, the player's additional Regular Salary may
      increase or decrease over the previous Season's additional Regular
      Salary by no more than eight percent (8\%) of the additional
      Regular Salary provided for in the Renegotiation Season. In the
      event that the Renegotiation Season provides for additional
      Incentive Compensation, the amount of additional Likely Bonuses
      and Unlikely Bonuses provided for in each Season after the
      Renegotiation Season may increase or decrease by up to eight
      percent (8\%) of the amount of additional Likely Bonuses and
      Unlikely Bonuses, respectively, provided for in the Renegotiation
      Season.
    \item
      No Renegotiation may contain a signing bonus, unless the
      Renegotiation is accompanied by an Extension and the signing bonus
      would otherwise be permitted under the rules governing the
      inclusion of signing bonuses in Extensions.
    \end{enumerate}
  \item
    In no event may a Team with a Team Salary at or above the Salary Cap
    renegotiate a Player Contract.
  \item
    In no event may a Team and a player renegotiate a Player Contract
    from March 1 through June 30 of any Salary Cap Year.
  \end{enumerate}
\item
  \textbf{Other.}

  \begin{enumerate}
  \def\labelenumii{(\arabic{enumii})}
  \tightlist
  \item
    In no event shall a Team and player negotiate a decrease in Salary
    or in any Incentive Compensation for any Salary Cap Year covered by
    a Player Contract.
  \item
    A Player Contract that is extended pursuant to Section 7(a) above
    may be renegotiated simultaneously, but only if and to the extent
    permitted by the rules set forth in Section 7(c) above.
    Notwithstanding anything to the contrary in this Agreement, if a
    Player Contract is extended pursuant to Section 7(a) above and
    renegotiated simultaneously, then the amount of the player's Salary,
    excluding Incentive Compensation, in the first Salary Cap Year
    covered by the extended term may decrease by no more than forty
    percent (40\%) of the player's Regular Salary (as renegotiated) in
    the last Salary Cap Year covered by the original term. In the event
    that the last Salary Cap Year covered by the original term provides
    for Incentive Compensation and such Incentive Compensation is also
    renegotiated, the amount of Likely Bonuses and Unlikely Bonuses in
    the first Salary Cap Year covered by the extended term may decrease
    by up to forty percent (40\%) of the player's Likely Bonuses and
    Unlikely Bonuses, respectively (as renegotiated), in the last Salary
    Cap Year covered by the original term.
  \item
    In connection with the trade of a Player Contract, the player and
    the assignor Team may agree to amend the Contract to waive all or
    any portion of a trade bonus; provided that, such Contract may not
    be subsequently extended or renegotiated until the later of (i) six
    (6) months from the date of the trade, or (ii) the first date on
    which the Contract could otherwise be extended or renegotiated
    pursuant to this Section 7.
  \item
    In connection with the trade of a Player Contract, notwithstanding
    anything to the contrary in Article XII, Section 2(a), a player and
    the assignor Team may agree upon an amendment to the Contract
    providing for the exercise or non-exercise of an Option contained in
    the Contract by a player or Team (as the case may be), provided that
    the amendment also provides that (i) the player will be traded to
    the assignee Team within forty-eight (48) hours of the execution of
    the amendment, and (ii) such trade and the consummation of such
    trade are conditions precedent to the validity of the amendment.
  \item
    In the event that a Team and a player agree to amend a Player
    Contract to reduce the amount of the player's protected Compensation
    in accordance with Article II, Section 3(n), then: (i) for purposes
    of calculating the player's Salary for the then-current and any
    remaining Salary Cap Year covered by the Contract, notwithstanding
    any stretch of the player's protected Compensation payment schedule,
    the aggregate reduction in the player's protected Compensation shall
    be allocated pro rata over the then-current and each remaining
    Salary Cap Year on the basis of the remaining unearned protected
    Base Compensation in each such Salary Cap Year; and (ii) the Team
    shall not be permitted to sign the player to a new Player Contract
    (or claim the player off of waivers) before the later of: (x) one
    (1) year following the date that the player's Player Contract with
    such Team was terminated; or (y) the July 1 following the last
    Season of such Player Contract.
  \item
    In the event that a Team terminates a Player Contract and the
    payment of the player's protected Compensation for one or more
    Seasons is stretched in accordance with Article II, Section 4(k),
    then the waiving Team may elect to have the player's Salary for the
    then-current and any remaining Salary Cap Years stretched (i.e.,
    re-attributed) for purposes of calculating the Team's Team Salary as
    follows (``Stretched Salary Amounts''):

    \begin{enumerate}
    \def\labelenumiii{(\Alph{enumiii})}
    \tightlist
    \item
      in the event a request for waivers on the player is made during
      the period from September 1 through the following June 30, (i) the
      player's post-termination Salary for the then-current Salary Cap
      Year (after giving effect to the provisions of Section (d)(5)
      above, if applicable) shall remain unchanged, and (ii) the
      player's post-termination Salary for each remaining Salary Cap
      Year (after giving effect to the provisions of Section (d)(5)
      above, if applicable) shall be aggregated and allocated evenly
      over a number of Salary Cap Years equal to twice the number of
      Seasons (including any Player Option Year) remaining on the
      Contract following the Salary Cap Year in which the request for
      waivers occurred (not including the then-current Season or, in the
      case of requests for waivers made from September 1 through the
      first day of a Regular Season, the upcoming Season), plus one (1)
      Season; or
    \item
      in the event a request for waivers on the player is made during
      the period from July 1 through August 31, the player's
      post-termination Salary for the then-current and any remaining
      Salary Cap Years (after giving effect to the provisions of Section
      (d)(5) above, if applicable) shall be aggregated and allocated
      evenly over a number of Salary Cap Years equal to twice the number
      of Seasons (including any Player Option Year) remaining on the
      Contract following the date of the request for waivers (including
      the upcoming Season), plus one (1) Season.\\
      To make an election pursuant to this Section (d)(6), a Team must
      provide the NBA with a written statement electing to stretch the
      player's Salary. Such written statement must be received by the
      NBA during the period commencing immediately following the Team's
      request for waivers on the player and ending one (1) business day
      following the termination of the Contract. The NBA shall provide
      notice of such election to the Players Association by email within
      two (2) business days following the NBA's receipt of such notice.
      Notwithstanding anything to the contrary herein, (i) in no event
      shall a Team be permitted to elect to stretch a waived player's
      Salary if the portion of the Team's Team Salary representing all
      of the Team's waived players (and any other former players) in any
      future Salary Cap Year exceeds or as a result of the proposed
      stretch would exceed fifteen percent (15\%) of the Salary Cap in
      effect during the Salary Cap Year in which the player is waived;
      and (ii) any Team that stretches a player's Salary for Salary Cap
      purposes in accordance with this Section (d)(6) may not re-sign or
      re-acquire the player whose Salary was so stretched prior to the
      July 1 following the end of the last Season of the player's
      Contract (including any Option Year).
    \end{enumerate}
  \item
    In no event shall a Team and player amend a Contract for the purpose
    of terminating or shortening the term of the Contract, except in
    accordance with the NBA waiver procedure or Article XII, Section 2.
  \end{enumerate}
\end{enumerate}

\section{Trade Rules.}\label{trade-rules.}

\begin{enumerate}
\def\labelenumi{(\alph{enumi})}
\item
  A Team shall not be permitted to pay or receive in connection with one
  (1) or more trades occurring during a Salary Cap Year, directly or
  indirectly, more than an aggregate of the amounts set forth below in
  cash across all such trades, including cash received as reimbursement
  for Compensation obligations to players whom the Team is acquiring.

  \begin{longtable}[]{@{}ll@{}}
  \toprule
  \begin{minipage}[b]{0.25\columnwidth}\raggedright\strut
  \strut
  \end{minipage} &
  \begin{minipage}[b]{0.64\columnwidth}\raggedright\strut
  Maximum Annual Cash Limit\strut
  \end{minipage}\tabularnewline
  \midrule
  \endhead
  \begin{minipage}[t]{0.25\columnwidth}\raggedright\strut
  For the 2017-18 Salary Cap Year:\strut
  \end{minipage} &
  \begin{minipage}[t]{0.64\columnwidth}\raggedright\strut
  \$5.1 million\strut
  \end{minipage}\tabularnewline
  \begin{minipage}[t]{0.25\columnwidth}\raggedright\strut
  For each subsequent Salary Cap Year through 2023-24:\strut
  \end{minipage} &
  \begin{minipage}[t]{0.64\columnwidth}\raggedright\strut
  The preceding Salary Cap Year's Maximum Annual Cash Limit amount
  adjusted by applying the percentage increase (or decrease) in the
  Salary Cap from the preceding Salary Cap Year to the current Salary
  Cap Year\strut
  \end{minipage}\tabularnewline
  \bottomrule
  \end{longtable}

  For purposes of this Section 8(a), (i) if a Contract is signed and
  then traded pursuant to Section 8(e)(1) below, and the Contract
  contains a signing bonus, the payment of all or any portion of such
  bonus by the Team that signed the Contract shall be treated as a
  reimbursement of a Compensation obligation of the assignee Team and
  shall be subject to this Section 8(a), and (ii) the amounts paid or
  received by a Team in connection with one (1) or more trades occurring
  during a Salary Cap Year shall not be netted against each other (thus,
  for example, if Team A pays \$5.1 million in connection with one (1)
  trade occurring during the 2017-18 Salary Cap Year and receives \$5.1
  million from another Team in connection with either the same or a
  subsequent trade occurring during the same Salary Cap Year, Team A
  would be unable to either pay or receive any cash in connection with
  any subsequent trades that it makes during that Salary Cap Year).
\item
  A player (other than a Two-Way Player) with a one-year Contract
  (excluding any Option Year) who would be a Qualifying Veteran Free
  Agent or an Early Qualifying Veteran Free Agent upon completing the
  playing services called for under his Contract cannot be traded
  without the player's consent. Should the player consent and be traded,
  then, for purposes of determining whether the player is a Qualifying
  Veteran Free Agent, Early Qualifying Veteran Free Agent or
  Non-Qualifying Veteran Free Agent at the conclusion of the Contract or
  any subsequent Contract between the player and the assignee Team, the
  player shall be considered as having changed Teams by means of signing
  a Contract with the assignee Team as a Free Agent (and not by means of
  trade). For clarity, a player's right to consent to be traded under
  this Section 8(b) shall continue following the initial trade of the
  player's Contract and any proposed subsequent trade of such Contract
  during the term thereof (not including any Option Year).
\item
  A Team cannot trade any player after the NBA trade deadline occurring
  in the last Season of the player's Contract, or after the NBA trade
  deadline occurring in any Season that could be the last Season of the
  player's Contract based upon the exercise or non-exercise of an Option
  or Early Termination Option.
\item
  \begin{enumerate}
  \def\labelenumii{(\roman{enumii})}
  \tightlist
  \item
    No Draft Rookie who signs a Standard NBA Contract or player who
    signs a Two-Way Contract may be traded before thirty (30) days
    following the date on which the Contract is signed.
  \item
    No player who signs a Contract as a Free Agent (or who signs a
    Standard NBA Contract while under a Two-Way Contract) may be traded
    before the later of (A) three (3) months following the date on which
    such Contract was signed or (B) the December 15 of the Salary Cap
    Year in which such Contract was signed; provided, that if a Contract
    is signed in connection with an agreement to trade the Contract in
    accordance with Section 8(e) below, the foregoing rule shall not
    apply to the initial trade but shall instead be applicable if the
    Contract is traded a second time. For the purposes of this rule, a
    Two-Way Contract that is converted to a Standard NBA Contract
    pursuant to such Contract's Standard NBA Contract Conversion Option
    will be deemed to be signed at the date of the conversion.
  \item
    Notwithstanding the rule set forth in Section (d)(ii) above, any
    player who signs a Contract with his prior Team meeting the
    following criteria may not be traded before the later of (x) three
    (3) months following the date on which such Contract was signed or
    (y) the January 15 of the Salary Cap Year in which such Contract was
    signed: the Team Salary of the player's Team is above the Salary Cap
    immediately following the Contract signing and the player is a
    Qualifying Veteran Free Agent or Early Qualifying Veteran Free Agent
    who, in accordance with Section 6(b)(1) or (3) above, enters into a
    new Player Contract with his prior Team that provides for a Salary
    for the first Season of such new Contract greater than one hundred
    twenty percent (120\%) of the Salary for the last Season of the
    player's immediately prior Contract. The rule set forth in this
    Section (d)(iii) shall not apply to a player if his new Contract
    provides for Salary equal to the Minimum Player Salary (with no
    bonuses of any kind). For purposes of the foregoing sentence, if the
    player's immediately prior Contract was a one-year Contract that
    provided for Salary equal to the Minimum Player Salary (with no
    bonuses of any kind), the player's prior Salary shall include the
    portion of the Minimum Player Salary, if any, that was reimbursed
    out of the League-wide benefits fund described in Article IV,
    Section 6(g).
  \end{enumerate}
\item
  \begin{enumerate}
  \def\labelenumii{(\arabic{enumii})}
  \tightlist
  \item
    A Veteran Free Agent and his Prior Team may enter into a Player
    Contract pursuant to an agreement between the Prior Team and another
    Team concerning the signing and subsequent trade of such Contract,
    but only if (i) the Veteran Free Agent finished the prior Season on
    his Prior Team's roster, (ii) the Contract is for at least three (3)
    Seasons (excluding any Option Year) but no more than four (4)
    Seasons in length, (iii) the Contract is not signed pursuant to the
    Non-Taxpayer Mid-Level Salary Exception or the Taxpayer Mid-Level
    Salary Exception, (iv) the first Season of the Contract is fully
    protected for lack of skill, (v) the Contract is entered into prior
    to the first day of the Regular Season, (vi) with respect to any 5th
    Year Eligible Player (as defined in Article II, Section 7) who met
    one of the 5th Year 30\% Max Criteria (as defined in Article II,
    Section 7), the Contract may not provide the player with Salary
    (plus Unlikely Bonuses) in excess of twenty-five percent (25\%) of
    the Salary Cap (as calculated pursuant to Article II, Section 7) in
    effect at the time the Contract is signed, and (vii) the acquiring
    Team has Room for the player's Salary plus any Unlikely Bonuses
    provided for in the first Season of the Contract. Notwithstanding
    anything to the contrary in the preceding sentence, a Team shall not
    be permitted to acquire a player pursuant to a Contract entered into
    in accordance with this Section 8(e)(1) if the Team's Team Salary
    (as calculated pursuant to Section 6(m)(3) above) following the
    proposed transaction would exceed the Tax Level for such Salary Cap
    Year plus the Tax Apron Amount, and if a Team acquires a player
    pursuant to this Section 8(e)(1), then the Team's Team Salary at all
    times thereafter during the Salary Cap Year may not exceed the Tax
    Level for such Salary Cap Year plus the Tax Apron Amount.
  \item
    A player and his Team may amend a Player Contract (including by
    entering into an Extension but not by entering into a Renegotiation)
    pursuant to an agreement between such Team and another Team
    concerning the signing of the amendment and subsequent trade of the
    amended Contract; provided, however, that: (i) no such agreement may
    be made during the period from the last day of the last Regular
    Season covered by the Contract (or the last day of any Regular
    Season that could be the last Regular Season covered by the Contract
    based upon the exercise or non-exercise of an Option or ETO) through
    the following June 30; and (ii) no such Extension entered into
    pursuant to this Section 8(e)(2) may cover more than three (3)
    Seasons from the date the Extension is signed. The Salary and
    Unlikely Bonuses that may be provided in the first year of the
    extended term and annual increases and decreases in Salary and
    Unlikely Bonuses shall be governed by Section 7(a)(3)(iii) and
    Section 5(c)(4) above.
  \item
    A Player Contract or Extension entered into pursuant to Section
    8(e)(1) or (2) above may not contain an Exhibit 6 thereto. However,
    the preceding sentence shall not prohibit the Teams involved in the
    trade from agreeing that the trade (and thus the validity of the
    Player Contract or Extension) will be conditional upon the passage
    of a physical examination to be performed by a physician designated
    by the assignee-Team in accordance with NBA procedures.
  \end{enumerate}
\item
  \begin{enumerate}
  \def\labelenumii{(\roman{enumii})}
  \tightlist
  \item
    In the event a player enters into an Extension pursuant to Section
    7(a) above (other than a Designated Veteran Player Extension
    governed by Section (f)(ii) below) that covers four (4) or more
    Seasons and/or provides for Salary and Unlikely Bonuses or annual
    increases in the player's Salary and Unlikely Bonuses in excess of
    the amounts permissible in connection with Extensions entered in
    connection with an agreement to trade the Contract pursuant to
    Section 8(e)(2) above, the player may not be traded before six (6)
    months following the date on which such Extension was signed. If a
    team acquires a player in a trade, then, for a period of six (6)
    months following the date of the trade, the team may not enter into
    an Extension with the player that provides for four (4) or more
    Seasons and/or provides for Salary and Unlikely Bonuses or annual
    increases in the player's Salary and/or Unlikely Bonuses in excess
    of the amounts permissible in connection with Extensions entered in
    connection with an agreement to trade the Contract pursuant to
    Section 8(e)(2) above.
  \item
    In the event a player enters into a Designated Veteran Player
    Extension pursuant to Section 7(a)(3)(ii) above or a Designated
    Veteran Player Contract pursuant to pursuant to Article II, Section
    7, the player may not be traded before one (1) year following the
    date on which he entered into such Designated Veteran Player
    Extension or Designated Veteran Player Contract.
  \end{enumerate}
\item
  In the event a Rookie Scale Contract is extended pursuant to Section
  7(b) above and a Team proposes to trade such Contract to another Team
  prior to the July 1 immediately following such extension, then, only
  for purposes of determining whether the acquiring Team has Room for
  the Contract, the Salary for the last Season of the original term of
  the Contract shall be deemed to equal the average of the aggregate
  Salaries for such Season and each Season of the extended term.
\item
  If a Team trades a player and the assignee Team subsequently places
  the player on waivers, the assignor Team shall not be permitted to
  sign the player to a new Contract (or claim the player off of waivers)
  before the earlier of: (i) one (1) year following the date all
  conditions to the trade were satisfied; or (ii) the July 1 following
  the last Season of the player's Player Contract.
\item
  Prior to the assignment of any Player Contract, the Team from which
  such Player Contract is to be assigned and the player whose Player
  Contract is to be assigned shall be required to divest themselves, on
  terms mutually agreeable to the player and the Team, of any
  pre-existing financial arrangements between such Team and such player.
  The foregoing shall not apply to Compensation earned by the player
  prior to the assignment or to loans.
\item
  As soon as is practicable following each trade (but in no event later
  than one (1) week from the date of the trade), the NBA shall send to
  the Players Association, by email, a summary of the principal terms of
  the trade; provided, however, that the NBA may omit from such summary
  any terms that the NBA or one (1) or more Teams involved in the trade
  reasonably deem confidential (other than such terms as may be
  necessary to verify the Teams' compliance with Section 8(a) above).
\item
  A ``trade'' of a player under this Agreement shall mean an assignment
  of a Player Contract pursuant to a negotiated exchange between two or
  more Teams following a trade conference call with the NBA league
  office. For clarity, the word ``trade'' shall not include an
  assignment of a player via the NBA's waiver procedures.
\end{enumerate}

\section{Miscellaneous.}\label{miscellaneous.}

\begin{enumerate}
\def\labelenumi{(\alph{enumi})}
\tightlist
\item
  Except where this Agreement states otherwise, for purposes of any rule
  in this Agreement that limits, involves counting, or otherwise relates
  to, the number of Seasons covered by a Contract:

  \begin{enumerate}
  \def\labelenumii{(\arabic{enumii})}
  \tightlist
  \item
    If a Player Contract or Extension is signed after the beginning of a
    Season, the Season in which the Contract or Extension is signed
    shall be counted as one (1) full Season covered by the Contract or
    Extension; and in the case of an Extension that is signed during the
    period from the end of a Season through the immediately following
    June 30, the Season immediately preceding the signing of the
    Extension (i.e., the just-completed Season) shall be counted as one
    (1) full Season covered by the Extension.
  \item
    An Option Year shall be counted as one (1) Season covered by the
    Contract.
  \end{enumerate}
\item
  Except where this Agreement states otherwise, all of the rules in this
  Agreement that limit, affect the calculation of, or otherwise relate
  to, the Compensation or Salary provided for in a Player Contract shall
  apply to Option Years.
\end{enumerate}

\section{Accounting Procedures.}\label{accounting-procedures.}

\begin{enumerate}
\def\labelenumi{(\alph{enumi})}
\item
  \begin{enumerate}
  \def\labelenumii{(\arabic{enumii})}
  \tightlist
  \item
    The NBA and the Players Association shall jointly engage an
    independent auditor (the ``Accountants'') to provide the parties
    with an ``Audit Report'' (and a ``Draft Audit Report,'' and, if
    applicable, an ``Interim Audit Report'' and, if applicable, an
    ``Interim Escrow Audit Report'') setting forth BRI, and Total
    Salaries and Benefits for the immediately preceding Salary Cap Year
    and the information called for by Section 12 below (the ``Escrow
    Information''). The audit reports provided for by this Section
    10(a)(1) are to be prepared in accordance with the provisions and
    definitions contained in this Agreement. The engagement of the
    Accountants shall be deemed to be renewed annually unless they are
    discharged by either party during the period from the submission of
    an Audit Report up to January 1 of the following year. The parties
    agree to share equally the costs incurred by the Accountants in
    preparing the audit reports provided for by this Section 10(a)(1).
  \item
    The Accountants shall submit a ``Draft Audit Report'' for each
    Salary Cap Year to the NBA and the Players Association, along with
    relevant supporting documentation, two (2) weeks prior to the
    scheduled issuance of the final Audit Report.
  \item
    The final Audit Report shall be submitted by the Accountants to the
    parties on or before the last day of the Salary Cap Year under audit
    (i.e., June 30). The audit shall begin as needed to ensure there is
    no reduction in the audit duration compared to the 2011 CBA. The
    Audit Report shall not be deemed final until the parties have
    confirmed in writing their agreement (in a form acceptable to the
    parties) with such Report. The NBA, the Players Association and the
    Teams shall use their best efforts to facilitate the Accountants'
    timely completion of the Audit Report.
  \item
    In the event that, for any reason, the Accountants fail to submit to
    the parties a final Audit Report by June 30, the Accountants shall
    prepare an interim Audit Report (the ``Interim Audit Report'') by
    such date setting forth the Accountants' best estimate of BRI and
    Total Salaries and Benefits for the preceding Salary Cap Year and,
    based upon such best estimates, the Escrow Information. Such Interim
    Audit Report shall include:

    \begin{enumerate}
    \def\labelenumiii{(\roman{enumiii})}
    \tightlist
    \item
      All amounts of BRI and Total Salaries and Benefits (or the
      portions thereof) and all Escrow Information (or the portions
      thereof) for such Salary Cap Year as to which the Accountants have
      completed their review and, by written agreement of the Players
      Association and the NBA (waiving their respective rights to
      dispute such amounts), are not in dispute.
    \item
      With respect to any amounts of BRI or Total Salaries and Benefits
      (or portions thereof) as to which the Accountants have not
      completed their review or which are the subject of a good faith
      dispute between the parties, the NBA's good faith proposal as to
      the proper amount, if any, that should be included in the Audit
      Report.
    \item
      With respect to any items of Escrow Information that are the
      subject of a good faith dispute between the parties, the
      Accountants' good faith determination as to such items, taking
      into account the provisions of Section 10(a)(4)(i) and (ii). As
      soon as practicable after the Interim Audit Report is submitted to
      the parties, the Accountants shall submit the final Audit Report,
      including a description of the differences, if any, from the
      Interim Audit Report. The Audit Report shall not be deemed final
      until the parties have confirmed in writing their agreement (in a
      form acceptable to the parties) with such Report or all disputes
      with respect to such Report have been finally resolved by means of
      the dispute-resolution procedures provided for by this Agreement.
      If, at the conclusion of the Audit Report Challenge Period (as
      defined by Section 12(a)(5) below), the Accountants have not
      submitted or are unable to submit a final Audit Report (because,
      by way of example but not limitation, there are disputes or claims
      that have been asserted pursuant to Article XXXII, Section 9(c)
      and which remain pending), the Accountants shall prepare and
      submit to the parties, within five (5) business days following the
      completion of the Audit Report Challenge Period, an Interim Escrow
      Audit Report that shall include the information set forth in the
      Interim Audit Report as adjusted or amended so as to reflect any
      final determinations made by the System Arbitrator or the Appeals
      Panel (as the case may be) in proceedings commenced pursuant to
      Article XXXII, Section 9(b) and involving disputes or claims with
      respect to such Interim Audit Report. The sole purpose for which
      any Interim Escrow Audit Report is to be used under this Agreement
      is to perform or form the basis for the calculations to be made
      pursuant to Section 12 below.
    \end{enumerate}
  \end{enumerate}
\item
  For purposes of determining BRI, Total Salaries and Benefits and the
  Escrow Information, the Accountants shall perform at least such review
  procedures as shall be agreed upon by the parties. In connection with
  the preparation of Audit Reports for each Salary Cap Year, each Team
  and the NBA shall submit a report to the Accountants, the NBA and the
  Players Association setting forth BRI, Team Salaries and Benefits
  information for such Salary Cap Year, on forms agreed upon by the NBA,
  the Players Association and the Accountants (the ``BRI Reports''). The
  NBA and the Players Association shall agree upon such forms no later
  than April 1 of each Salary Cap Year.
\item
  The Accountants shall review the reasonableness of any estimates of
  revenues or expenses for a Salary Cap Year included in the Teams' and
  the NBA's BRI Reports for such Salary Cap Year and may make such
  adjustments in such estimates as they deem appropriate. To the extent
  the actual amounts of revenues received or expenses incurred for a
  Salary Cap Year differ from such estimates, adjustments shall be made
  in BRI for the following Salary Cap Year in accordance with the
  provisions of Section 10(f) below.
\item
  With respect to deducted expenses, except for Newly-Deductible
  International Expenses, the NBA, League-related entities, Teams and
  Related Parties shall report in BRI Reports only those expenses that
  are reasonable and customary in accordance with the provisions of
  Section 1(a) above. Subject to the terms of Section 1(a)(6) and
  Section 1(a)(14) above, and Section 11 below, all categories of
  expenses deducted in a BRI Report completed by the NBA or a Team shall
  be reviewed by the Accountants, but such categories shall be presumed
  to be reasonable and customary and the amount of the expenses deducted
  by the NBA or a Team that come within such expense categories shall
  also be presumed to be reasonable and customary, unless such
  categories or amounts are found by the Accountants to be either
  unrelated to the revenues involved or grossly excessive.
\item
  The Accountants shall notify designated representatives of the NBA and
  the Players Association: (1) if the Accountants have any questions
  concerning the amounts of revenues or expenses reported by the Teams
  and the NBA or any other information contained in the BRI Reports; or
  (2) if the Accountants propose that any adjustments be made to any
  revenue or expense item or any other information contained in the BRI
  Reports.
\item
  The Accountants shall indicate which amounts included in BRI for a
  Salary Cap Year, if any, represent estimates of revenues or expenses.
  With respect to any such estimated revenues or expenses, the
  Accountants shall, in preparing the Audit Report for the immediately
  succeeding Salary Cap Year (``Subsequent Audit Report''), or the Audit
  Report for the same Salary Cap Year in the event that an Interim Audit
  Report was previously issued for that Salary Cap Year, determine the
  actual revenues and expenses received for the prior Salary Cap Year
  and include as a credit or debit to BRI in such Subsequent Audit
  Report the amount of the aggregate difference, if any, between all
  such estimated revenues and expenses for the prior Salary Cap Year and
  the actual revenues and expenses received for such Salary Cap Year.
\item
  In the event that in the course of preparing an Audit Report for a
  Salary Cap Year the Accountants discover that they committed an error
  in computing BRI in the Audit Reports for either of the two (2)
  previous Salary Cap Years, which error resulted in a material
  understatement or overstatement of BRI for either of such Salary Cap
  Years, and the parties agree that such error was committed and agree
  as to the amount of the resulting understatement or overstatement (or,
  if they do not agree, an error (and the amount of such error) is
  established pursuant to the dispute resolution procedures provided for
  in this Agreement) the amount of such understatement or overstatement
  of BRI shall be added to or subtracted from BRI, as the case may be,
  with interest (at a rate equal to the one (1) year Treasury Bill rate
  as published in The Wall Street Journal on the date of the issuance of
  such Audit Report) accruing from the date of the Audit Report for the
  Salary Cap Year in which such understatement or overstatement occurred
  in equal annual amounts over the then-current and subsequent Salary
  Cap Years. Notwithstanding the foregoing, the parties will jointly
  instruct the Accountants that their audits shall not include
  procedures specifically designed to detect errors committed in prior
  audits.
\item
  In the event that there is an NHL players' strike or owners' lockout
  (``work stoppage'') resulting in the cancellation of all or part of
  any NHL season in any Salary Cap Year, and such work stoppage results
  in a refund being made to luxury suite-holders, premium seat
  license-holders or to purchasers of fixed arena signage and/or naming
  rights in arenas in which both an NBA Team and an NHL team plays its
  home games, then the revenues for luxury suites, premium seat licenses
  and fixed arena signage and/or naming rights in such arenas shall be
  determined as if such refunds were not made. If the work stoppage
  continues for a second year, then the NHL revenues shall be deemed to
  be the amount included for the prior year.
\item
  All disputes with respect to any Interim Audit Report shall be
  resolved exclusively in accordance with the procedures set forth in
  Article XXXII.
\end{enumerate}

\section{Players Association Audit
Rights.}\label{players-association-audit-rights.}

\begin{enumerate}
\def\labelenumi{(\alph{enumi})}
\tightlist
\item
  \textbf{Team Audits.} The Players Association shall have the right as
  part of the annual review of BRI Reports to retain its own accountants
  (the ``Players Association's Accountants''), at its own expense, after
  the submission of each Audit Report under this Agreement, to audit the
  books and records of up to ten (10) NBA teams (of its choosing) (the
  ``First Audit''); provided, however, that such review shall be limited
  to (1) revenue items (including in respect of equity transactions
  subject to Section 1(a)(13) above), and (2) expense items, in each
  case that appear or should have appeared in the BRI Reports. In the
  event that, in the opinion of the Players Association's Accountants,
  such audit indicates misallocations or miscategorizations of revenues
  or expenses (other than with respect to matters that constituted
  Disputed Adjustments in connection with the prior Audit Report)
  resulting in an understatement of BRI, they shall submit to the NBA
  proposed adjustments to BRI consistent with their findings. In the
  event that the NBA disputes such proposed adjustments, such proposed
  adjustments shall be deemed to be ``Disputed Adjustments'' and shall
  be resolved in accordance with the procedures set forth in Article
  XXXII. In addition, in the event that First Audit Disputed Adjustments
  in excess of \$8 million are resolved in favor of the Players
  Association, the Players Association shall then have the right, that
  Season, to have the Players Association's Accountants audit up to an
  additional ten (10) NBA teams for the same Salary Cap Year, in
  accordance with the foregoing procedures (the ``Second Audit''). If,
  as a result of the Second Audit, additional Disputed Adjustments in
  excess of \$8 million are resolved in favor of the Players
  Association, the Players Association shall then have the right, that
  Season, to have the Players Association's Accountants audit all
  remaining NBA Teams for that Salary Cap Year. The amount of any and
  all Disputed Adjustments that are ultimately resolved in favor of the
  Players Association in accordance with this Section 11(a) shall be
  added to BRI in the Salary Cap Year in which such resolution is
  reached.
\item
  \textbf{League Audit.} The Players Association shall have the right as
  part of the annual review of BRI Reports to retain the Players
  Association's Accountants to conduct an audit, at its own expense, of
  the books and records of the NBA, Properties, Media Ventures, and
  other League-related entities associated with generating BRI,
  provided, however, that such audit shall be limited to (1) revenue
  items (including in respect of equity transactions subject to Section
  1(a)(13) above) and (2) expense items, regardless of whether such
  expenses exceed the applicable BRI ratio of expenses to revenues set
  forth in Exhibit D, in each case that appear or should have appeared
  in the BRI Report. In the event that in the opinion of the Players
  Association's Accountants, such audit indicates misallocations or
  miscategorizations of revenues or expenses (other than with respect to
  matters that constituted League Disputed Adjustments in connection
  with the prior Audit Report) resulting in an understatement of BRI,
  they shall submit proposed adjustments to the NBA consistent with
  their findings. In the event that the NBA disputes such proposed
  adjustments, such proposed adjustments shall be deemed to be League
  Disputed Adjustments and resolved in accordance with the procedures
  set forth in Article XXXII. The amount of any and all such League
  Disputed Adjustments that are resolved in the Players Association's
  favor shall be included in BRI in the Salary Cap Year in which such
  resolution is reached. In addition, in the event that any such League
  Disputed Adjustments are resolved in the Players Association's favor,
  the Accountants shall be directed to correct such expense
  misallocations and/or miscategorizations in the remaining Salary Cap
  Years covered by the Agreement.
\item
  \textbf{Confidentiality.} In connection with any audit conducted by
  the Players Association pursuant to this Section 11, the Players
  Association agrees to sign, and to cause its representatives to sign,
  a confidentiality agreement in the form annexed hereto as Exhibit J-1.
  The Players Association also agrees to sign, and to cause its
  representatives to sign, a similar confidentiality agreement with
  respect to information obtained in connection with the Accountants'
  audit pursuant to Section 10 above.
\item
  \textbf{Preceding Salary Cap Year Audit Adjustments.} Notwithstanding
  anything else in this Agreement or any release in the annual BRI
  letter agreement, if upward or downward adjustments are made in
  connection with a Players Association-initiated audit, an adjustment
  to BRI in respect of the same item can also be made for revenues or
  expenses related to the preceding Salary Cap Year, if applicable. For
  example, without limitation, if, based on the audit findings, the
  parties agree that a Team under-reported 2017-18 BRI by one million
  dollars (\$1 million), and that the same error in the same amount
  occurred in 2016-17, then 2018-19 BRI would be adjusted upward by two
  million dollars (\$2 million).
\item
  \textbf{Related Party Access.} The Players Association's Accountants
  shall have access to such portions of a Related Party's books and
  records that the accountants have a well-founded basis to believe have
  a meaningful impact on BRI. For purposes of the foregoing, (i) where a
  team plays in an arena owned or operated by a Related Party, the
  Players Association's Accountants will have access to that Related
  Party arena company's trial balance relating to all revenues and to
  such other portions of the trial balance that the Players
  Association's Accountants have a well-founded basis to believe have a
  meaningful impact on BRI; (ii) for other Related Parties, information
  requests should fit the circumstances to enable the Players
  Association's Accountants to verify the accuracy of BRI amounts (x)
  that cannot reasonably be verified through other means, and (y)
  without accessing financial and business information that there is no
  well-founded basis to believe have a meaningful impact on BRI; and
  (iii) the NBA, Players Association, and the Team will collectively
  consider any request for access to Related Party books and records
  while onsite and make their best efforts to resolve the access issue.
\item
  \textbf{Bilateral Adjustments.} Subject to the deadlines set forth in
  Section 11(g) below, the NBA may propose BRI adjustments with respect
  to any Team audited by the Players Association. The NBA's right to
  propose such adjustments may not adversely affect in any way the time
  and resources available to the Players Association under its audit
  rights. In the event that the Players Association disputes such
  proposed adjustments, such proposed adjustments shall be resolved in
  accordance with the procedures set forth in Article XXXII. The amount
  of any and all such proposed adjustments that are ultimately resolved
  in favor of the NBA in accordance with this Section 11(f), including
  any adjustments made pursuant to Section 11(d) above, shall be
  deducted from BRI in the Salary Cap Year in which such resolution is
  reached.
\item
  \textbf{Timing.} Audits conducted by the Players Association must be
  noticed within ninety (90) days after issuance of the final Audit
  Report in respect of the applicable Salary Cap Year. Any proposed
  adjustments by the Players Association and NBA relating to Team audits
  (and in the case of the Players Association, with respect to any
  League Office audits) will be resolved by April 30 of the following
  calendar year. Each party will provide its proposed adjustments by
  March 25.
\end{enumerate}

\section{Escrow and Tax Arrangement.}\label{escrow-and-tax-arrangement.}

\begin{enumerate}
\def\labelenumi{(\alph{enumi})}
\tightlist
\item
  \textbf{Definitions.} As used in this Agreement, the following terms
  shall have the following meanings:

  \begin{enumerate}
  \def\labelenumii{(\arabic{enumii})}
  \tightlist
  \item
    ``Adjustment Player'' means, with respect to a Salary Cap Year,
    every current or former player included in a Team's Team Salary for
    such Salary Cap Year, every player who is excluded from Team Salary
    pursuant to Section 4(h) above, and every Two-Way Player that is
    paid a Two-Way NBA Salary for such Salary Cap Year.
  \item
    ``Aggregate Compensation Adjustment Amount'' means, with respect to
    a Salary Cap Year, the lesser of (i) the Aggregate Salaries and
    Benefits Adjustment Amount, and (ii) ten percent (10\%) of Total
    Salaries for such Salary Cap Year. For purposes of clause (ii),
    Total Salaries shall be increased by the amount that is included in
    Benefits for such Salary Cap Year pursuant to Article IV, Sections
    6(g)(1) and (3).
  \item
    ``Aggregate Compensation Adjustment Amount Shortfall'' means the
    amount by which the amount received by the NBA from the Escrow Agent
    with respect to a Salary Cap Year pursuant to Section 12(d)(1) below
    is less than the Aggregate Compensation Adjustment Amount for such
    Salary Cap Year.
  \item
    ``Aggregate Salaries and Benefits Adjustment Amount'' means, with
    respect to a Salary Cap Year, the lesser of (i) the Overage (as
    defined in Section 12(a)(14) below) for such Salary Cap Year, and
    (ii) ten percent (10\%) of Total Salaries and Benefits for such
    Salary Cap Year.
  \item
    ``Audit Report Challenge Period'' means the period beginning with
    the date on which an Interim Audit Report is issued by the
    Accountants and ending on the last date by which all challenges
    thereto brought pursuant to Article XXXII, Section 9(b) are
    resolved.
  \item
    ``Deduction Date'' means each of the twelve (12) semi-monthly
    payment dates from November 15 through May 1 provided for under
    paragraph 3 of the Uniform Player Contract.
  \item
    ``Designated Share'' means, with respect to a Salary Cap Year, the
    amount set forth in Section 12(b)(3) below.
  \item
    ``Escrow Agent'' means, for purposes of this Section 12, the escrow
    agent identified in the Salary Escrow Agreement.
  \item
    ``Escrow Amount'' means for an Adjustment Player, with respect to a
    Salary Cap Year, an amount equal to the sum of the player's ``Base
    Escrow Amount'' (as defined below) plus the player's ``Performance
    Bonus Escrow Amount'' (as defined below) plus the player's ``Trade
    Bonus Escrow Amount'' (as defined below). ``Base Escrow Amount''
    means for an Adjustment Player, with respect to a Salary Cap Year,
    an amount equal to ten percent (10\%) of the Adjustment Player's
    Salary for such Salary Cap Year. For purposes of calculating an
    Adjustment Player's Base Escrow Amount in accordance with the
    preceding sentence: (i) a player's Salary shall exclude all
    Performance Bonuses included in Salary under Section 3(d) above;
    (ii) the Salary of a player under a one-year Contract making the
    Minimum Player Salary shall include the portion of such Minimum
    Player Salary that is reimbursed out of the League-wide benefits
    fund described in Article IV, Section 6(g)(2); (iii) the Salary of a
    player under a Rookie Scale Contract whose Compensation under his
    Contract is increased pursuant to Article VIII, Section 5 shall
    include the portion of his Compensation attributable to the Rookie
    Scale Conforming Increases (as described in Article VIII, Section
    5(a)) that is reimbursed out of the League-wide benefits fund
    described in Article IV, Section 6(g)(4); and (iv) with respect to a
    Two-Way Player, such player's Salary shall only include the Two-Way
    NBA Salary portion of his Two-Way Player Salary. ``Performance Bonus
    Escrow Amount'' means for an Adjustment Player, with respect to a
    Salary Cap Year, an amount equal to ten percent (10\%) of the
    Performance Bonuses earned by such player during such Salary Cap
    Year. ``Trade Bonus Escrow Amount'' means for an Adjustment Player
    whose Contract contains a trade bonus, with respect to a Salary Cap
    Year in which the player's Contract is traded during the period
    following the conclusion of the Team's Season through June 30, an
    amount equal to ten percent (10\%) of the portion of the trade bonus
    that is included in Salary for such Salary Cap Year.
  \item
    ``Escrow Schedules'' means the schedules prepared by the NBA with
    respect to each Salary Cap Year setting forth: the Base Escrow
    Amount; when calculable following the conclusion of the applicable
    Regular Season, the Performance Bonus Escrow Amount; when
    calculable, the Trade Bonus Escrow Amount; and the dates on which
    each Adjustment Player's Base Escrow Amount, Performance Bonus
    Escrow Amount and Trade Bonus Escrow Amount are to be deducted from
    the player's Compensation and/or delivered to the Escrow Agent.
  \item
    ``Individual Compensation Adjustment Amount'' means for an
    Adjustment Player, with respect to a Salary Cap Year, the amount
    calculated following the conclusion of the Salary Cap Year by
    multiplying the Aggregate Compensation Adjustment Amount for such
    Salary Cap Year by a fraction, the numerator of which is the
    Adjustment Player's Salary for such Salary Cap Year and the
    denominator of which is the sum of all Adjustment Players' Salaries
    for such Salary Cap Year. For purposes of calculating the fraction
    described in the preceding sentence: (i) a player's Salary shall
    include all Performance Bonuses excluded from Salary under Section
    3(d) above but actually earned by the player during such Salary Cap
    Year, and shall exclude all Performance Bonuses included in Salary
    under Section 3(d) above but not actually earned by the player
    during such Salary Cap Year; (ii) the Salary of a player under a
    one-year Contract making the Minimum Player Salary shall include the
    portion of such Minimum Player Salary that is reimbursed out of the
    League-wide benefits fund described in Article IV, Section 6(g)(2);
    (iii) the Salary of a player under a Rookie Scale Contract whose
    Compensation under his Contract is increased pursuant to Article
    VIII, Section 5 shall include the portion of his Compensation
    attributable to the Rookie Scale Conforming Increases (as described
    in Article VIII, Section 5(a)) that is reimbursed out of the
    League-wide benefits fund described in Article IV, Section 6(g)(4);
    and (iv) with respect to a Two-Way Player, the Salary shall only
    include the Two-Way NBA Salary portion of his Two-Way Player Salary.
  \item
    ``Individual Shortfall Adjustment Amount'' means, with respect to
    each Contract that is amended pursuant to Section 12(e)(1) below,
    the amount that the Compensation otherwise payable in accordance
    with that Contract shall be reduced pursuant to Section 12(e)(2)
    below.
  \item
    ``New Benefit Adjustment Amount'' means an amount equal to the
    difference between the Aggregate Salaries and Benefits Adjustment
    Amount and the Aggregate Compensation Adjustment Amount.
  \item
    ``Overage'' means the amount, if any, by which Total Salaries and
    Benefits for a Salary Cap Year exceed the Designated Share for such
    Salary Cap Year.
  \item
    ``Performance Bonus Deduction Date'' means, with respect to an
    Adjustment Player, the earlier of (i) the date that any Performance
    Bonus earned by the player during the applicable Salary Cap Year is
    paid to the player pursuant to his Contract, or (ii) the day
    following the date of the last game of the NBA Finals occurring
    during such Salary Cap Year.
  \item
    ``Salary Escrow Agreement'' means the escrow agreement in the form
    agreed upon by the parties (or such other form to which the parties
    may agree) to be entered into with the Escrow Agent.
  \item
    ``Tax Level'' means, with respect to a Salary Cap Year, an amount
    determined in accordance with the following:

    \begin{enumerate}
    \def\labelenumiii{(\roman{enumiii})}
    \tightlist
    \item
      Subject to the adjustments set forth in Section 12(a)(17)(vi) or
      (vii) below, the Tax Level shall be an amount determined by the
      following calculation:\\
      STEP 1: Compute fifty-three and fifty-one one hundredths percent
      (53.51\%) of Projected BRI (as defined in Section 1(c) above).\\
      STEP 2: Subtract Projected Benefits (as defined in Article IV,
      Section 9) for such Salary Cap Year from the result in Step 1.\\
      STEP 3: Divide the result in Step 2 by the number of Teams
      scheduled to play in the NBA during such Salary Cap Year
      (excluding Expansion Teams during their first two (2) Salary Cap
      Years in the NBA).\\
    \item
      Notwithstanding Section 12(a)(17)(i) above, in the event that,
      subject to the adjustments set forth in Section 12(a)(17)(vi) or
      (vii) below, Projected BRI for any Salary Cap Year in which one
      (1) or more Expansion Teams is scheduled to play its second
      Season, plus Projected Local Expansion Team BRI for such Salary
      Cap Year, multiplied by the applicable percentage of Projected BRI
      set forth in Section 12(a)(17)(i) above, less Projected benefits
      for such Salary Cap Year (including for the Expansion Team(s)),
      divided by the number of Teams scheduled to play in the NBA during
      such Salary Cap Year (including the Expansion Team(s)), exceeds
      the Tax Level calculated in accordance with Section 12(a)(17)(i)
      above, the Tax Level shall equal the amount calculated pursuant to
      this Section 12(a)(17)(ii).
    \item
      The Tax Level for a Salary Cap Year will be in effect commencing
      on the first day of the Salary Cap Year (i.e., July 1) and shall
      continue through and including the last day of the Salary Cap
      Year.
    \item
      In the event that the Audit Report for a Salary Cap Year has not
      been completed as of the last day of the Moratorium Period
      immediately following the end of such Salary Cap Year and the NBA
      and the Players Association have not reached agreement on
      Projected BRI and Projected Benefits pursuant to Section 1 above
      and Article IV, Section 9 for the Salary Cap Year that commenced
      on the immediately preceding July 1, then the Tax Level for the
      Salary Cap Year that commenced on such immediately preceding July
      1 will be calculated using Interim Projected BRI instead of
      Projected BRI, Estimated BRI instead of BRI and Estimated Benefits
      instead of Benefits for all purposes under this Section 12(a)(17)
      including, without limitation, the adjustments set forth in
      Section 12(a)(17)(vi) or (vii) below. In the event that the
      Interim Audit Report for a Salary Cap Year has not been completed
      as of the last day of the Moratorium Period immediately following
      the end of such Salary Cap Year, and the NBA and Players
      Association have not reached agreement on Projected BRI and
      Projected Benefits pursuant to Section 1 above and Article IV,
      Section 9, then the Tax Level for the Salary Cap Year that
      commenced on the immediately preceding July 1 shall, until such
      Interim Audit Report is completed, be an amount that would have
      been the Tax Level for the preceding Salary Cap Year had Projected
      BRI or Interim Projected BRI, as the case may be, for such
      preceding Salary Cap Year included, with respect to the NBA's
      national broadcast, national telecast or network cable television
      contracts, the rights fees or other non-contingent payments stated
      in such contracts for the Season following the Season covered by
      such preceding Salary Cap Year instead of for the Season covered
      by such preceding Salary Cap Year.
    \item
      In the event that the Tax Level for a Salary Cap Year is
      calculated in accordance with Section 12(a)(17)(iv) above (i.e.,
      is based upon an Interim Audit Report for the prior Salary Cap
      Year) and BRI and Benefits as set forth in the Audit Report for
      the prior Salary Cap Year are different from those in the Interim
      Audit Report such that the Tax Level would have been different
      from that based upon the Interim Audit Report, any such difference
      in the Tax Level shall be debited or credited, as the case may be,
      to the Tax Level for the subsequent Salary Cap Year, except that,
      with respect to the 2023-24 Salary Cap Year (or, in the
      alternative, if either the NBA or the Players Association
      exercises its option to terminate this Agreement following the
      2022-23 Salary Cap Year pursuant to Article XXXIX) any such
      differences shall be debited or credited, as the case may be, to
      the Tax Level for the then-current Salary Cap Year, in all such
      cases with interest (at a rate equal to the one (1) year Treasury
      Bill rate as published in The Wall Street Journal on the date of
      the issuance of the Interim Audit Report).
    \item
      In the event that Total Salaries and Benefits for any Salary Cap
      Year are less than the Designated Share for such Salary Cap Year,
      then the Tax Level for the subsequent Salary Cap Year shall be
      increased by the amount of the shortfall divided by the number of
      Teams in the NBA during such subsequent Salary Cap Year (other
      than Expansion Teams in their first two (2) Salary Cap Years in
      the NBA).
    \item
      In the event that there is an Overage (as defined in Section
      12(a)(14) above) in any Salary Cap Year, then the Tax Level for
      the subsequent Salary Cap Year shall be adjusted as follows:

      \begin{enumerate}
      \def\labelenumiv{(\Alph{enumiv})}
      \tightlist
      \item
        If the amount of the Overage in the Salary Cap Year is equal to
        or less than six percent (6\%) of Total Salaries and Benefits,
        then no adjustments shall be made to the Tax Level for the
        subsequent Salary Cap Year.
      \item
        If the amount of the Overage in the Salary Cap Year equals more
        than six percent (6\%) of Total Salaries and Benefits, then the
        Tax Level for the subsequent Salary Cap Year shall be reduced by
        an amount to be calculated as follows:\\
        STEP 1: Subtract six percent (6\%) of Total Salaries and
        Benefits from the Overage.\\
        STEP 2: If Projected BRI for the subsequent Salary Cap Year does
        not exceed BRI for the Salary Cap Year by more than eight
        percent (8\%) of BRI for the Salary Cap Year or if the amount of
        the Overage described above exceeds nine percent (9\%) of Total
        Salaries and Benefits, then divide the result of Step 1 by the
        number of Teams in the NBA during the subsequent Salary Cap Year
        (other than Expansion Teams in their first two (2) Salary Cap
        Years in the NBA). The result of this calculation is the amount
        of the reduction in the Tax Level for such subsequent Salary Cap
        Year, and no further steps are required.\\
        If Projected BRI for the subsequent Salary Cap Year exceeds BRI
        for the Salary Cap Year by more than eight percent (8\%) of BRI
        for the Salary Cap Year and the amount of the Overage described
        above does not exceed nine percent (9\%) of Total Salaries and
        Benefits, then proceed to Step 3.\\
        STEP 3: Subtract BRI for the Salary Cap Year from Projected BRI
        for the subsequent Salary Cap Year.\\
        STEP 4: Subtract eight percent (8\%) of BRI for the Salary Cap
        Year from the result of Step 3.\\
        STEP 5: Multiply the result of Step 4 by fifty-three and
        fifty-one one hundredths percent (53.51\%).\\
        STEP 6: Subtract the result of Step 5 from the result of Step 1.
        If the result of this step is less than zero, then no
        adjustments shall be made to the Tax Level for the subsequent
        Salary Cap Year, and no further steps are required.\\
        STEP 7: Divide the result of Step 6 by the number of Teams in
        the NBA during such subsequent Salary Cap Year (other than
        Expansion Teams in their first two (2) Salary Cap Years in the
        NBA). The result of this calculation is the amount of the
        reduction in the Tax Level for such subsequent Salary Cap
        Year.\\
        \emph{Example: Assume Total Salaries and Benefits for the
        2019-20 Season are \$4.686 billion and the Designated Share is
        \$4.335 billion resulting in an Overage of \$351 million (which
        equals 7.5\% of Total Salaries and Benefits). Assume Projected
        BRI for the 2020-21 Season (\$8.925 billion) exceeds actual BRI
        for the 2019-20 Season (\$8.500 billion) by 5\%. Assume there
        are 30 Teams in the NBA in 2020-21. The Tax Level for the
        2020-21 Season would be reduced by \$2.3 million ((7.5\% of
        Total Salaries and Benefits (\$351 million) less 6\% of Total
        Salaries and Benefits (\$281 million)) divided by 30).}\\
        \emph{Example: Same Total Salaries and Benefits as in the prior
        example, except assume Projected BRI for the 2020-21 Season
        (\$9.265 billion) exceeds actual BRI for the 2019-20 Season
        (\$8.500 billion) by 9\%. The Tax Level for the 2020-21 Season
        would be reduced by \$0.8 million (\$25 million (the Overage of
        7.5\% of Total Salaries and Benefits (\$351 million) less 6\% of
        Total Salaries and Benefits (\$281 million)), less \$45 million
        (i.e., the difference between (x) Projected BRI for the 2020-21
        Season (\$9.265 billion) less actual BRI for the 2019-20 Season
        (\$8.500 billion) (i.e., \$765 million) and (y) 8\% of actual
        BRI for the 2019-20 Season (i.e., \$680 million) -- a difference
        of \$85 million -- multiplied by 53.51\%), divided by 30).}
      \end{enumerate}
    \end{enumerate}
  \item
    ``Trade Bonus Deduction Date'' means, with respect to an Adjustment
    Player, the earlier of (i) the date that any trade bonus earned by
    the player during the applicable Salary Cap Year is paid to the
    player pursuant to his Contract, or (ii) the June 30 of such Salary
    Cap Year.
  \end{enumerate}
\item
  \textbf{Compensation Adjustment Rules.}

  \begin{enumerate}
  \def\labelenumii{(\arabic{enumii})}
  \tightlist
  \item
    In the event that there is an Overage in any Salary Cap Year, (i)
    the Contracts of all Adjustment Players shall be amended by
    operation of this Agreement, such that the aggregate Compensation
    otherwise payable to all such Adjustment Players with respect to
    such Salary Cap Year shall be reduced by the Aggregate Compensation
    Adjustment Amount, (ii) the New Benefit Amount for such Season as
    provided for by Article IV, Section 8(a)(1) (i.e., the New Benefit
    amount that is presently specified in Article IV, Section 8(a)(1) or
    such lesser New Benefit Amount as is designated by the Players
    Association, in accordance with the provisions of Article IV,
    Section 8(a)(1), by notice in writing to the NBA delivered on or
    before the March 15 prior to the commencement of the Salary Cap Year
    encompassing such Season) shall be reduced by the New Benefit
    Adjustment Amount, and (iii) the Additional Benefit Amount as
    provided for by Article IV, Section 4(d)(1) (i.e., the one percent
    (1\%) of BRI for additional benefits) shall be reduced by any
    remaining Overage after applying the reductions set forth in Section
    12(b)(1)(i)-(ii).
  \item
    Subject to Section 12(d) and (e) below, to effectuate the aggregate
    Compensation reduction provided for in Section 12(b)(1)(i) above,
    the Compensation otherwise payable in accordance with each
    Adjustment Player's Contract shall be reduced by the player's
    respective Individual Compensation Adjustment Amount.
  \item
    The Designated Share for each Salary Cap Year covered by the term of
    this Agreement shall equal fifty percent (50\%) of BRI for such
    Salary Cap Year, provided that the Designated Share for a Salary Cap
    Year shall be increased or decreased in accordance with the
    following: (i) in the event that BRI for a Salary Cap Year exceeds
    the amount of BRI forecasted for such Salary Cap Year as set forth
    below, then the Designated Share for such Salary Cap Year shall
    equal fifty percent (50\%) of the amount of BRI forecasted for such
    Salary Cap Year, plus sixty and one-half percent (60.5\%) of the
    difference between the BRI for such Salary Cap Year and the BRI
    forecasted for such Salary Cap Year; and (ii) in the event that BRI
    forecasted for a Salary Cap Year as set forth below exceeds BRI for
    such Salary Cap Year, then the Designated Share for such Salary Cap
    Year shall equal fifty percent (50\%) of the amount of BRI
    forecasted for such Salary Cap Year, less sixty and one-half percent
    (60.5\%) of the difference between the BRI forecasted for such
    Salary Cap Year and BRI for such Salary Cap Year. Notwithstanding
    anything to the contrary in the foregoing, in no event shall the
    Designated Share for any Salary Cap Year be less than forty-nine
    percent (49\%) of BRI or greater than fifty-one percent (51\%) of
    BRI. To illustrate the foregoing, (X) if BRI for the 2021-22 Salary
    Cap Year were to equal \$6.442 billion, then the Designated Share
    for the 2021-22 Salary Cap Year would equal \$3.2315 billion
    (\$3.171 billion (forecasted BRI of \$6.342 billion multiplied by
    50\%) plus \$60.5 million (60.5\% of \$100 million -- the difference
    between BRI of \$6.442 billion and forecasted BRI of \$6.342
    billion), which would equate to 50.16\% of BRI; (Y) if BRI for the
    2021-22 Salary Cap Year were to equal \$6.242 billion, then the
    Designated Share for the 2021-22 Salary Cap Year would equal
    \$3.1105 billion (\$3.171 billion (forecasted BRI of \$6.342 billion
    multiplied by 50\%) less \$60.5 million (60.5\% of \$100 million --
    the difference between forecasted BRI of \$6.342 billion and BRI of
    \$6.242 billion), which would equate to 49.83\% of BRI; and (Z) if
    BRI for the 2021-22 Salary Cap Year were to equal \$7.084 billion,
    then the Designated Share for the 2021-22 Salary Cap Year would
    equal \$3.6128 billion or 51\% of BRI (since the amount per the
    calculation would exceed 51\% of BRI -- \$3.171 billion (forecasted
    BRI of \$6.342 billion multiplied by 50\%) plus \$448.9 million
    (60.5\% of \$742 million -- the difference between BRI of \$7.084
    billion and forecasted BRI of \$6.342 billion) would equal \$3.6199
    billion or 51.1\% of BRI).
  \end{enumerate}

  \begin{longtable}[]{@{}cc@{}}
  \toprule
  Salary Cap Year & Forecasted BRI\tabularnewline
  \midrule
  \endhead
  2017-18 & \$5.318 billion\tabularnewline
  2018-19 & \$5.557 billion\tabularnewline
  2019-20 & \$5.807 billion\tabularnewline
  2020-21 & \$6.069 billion\tabularnewline
  2021-22 & \$6.342 billion\tabularnewline
  2022-23 & \$6.627 billion\tabularnewline
  2023-24 & \$6.926 billion\tabularnewline
  \bottomrule
  \end{longtable}
\item
  Escrow Procedure.

  \begin{enumerate}
  \def\labelenumii{(\arabic{enumii})}
  \tightlist
  \item
    The following shall apply with respect to each Salary Cap Year
    (subject to Section 12(d) below regarding final reconciliation):

    \begin{enumerate}
    \def\labelenumiii{(\roman{enumiii})}
    \tightlist
    \item
      The Compensation otherwise payable to each Adjustment Player shall
      be reduced by the Escrow Amount applicable to such Adjustment
      Player; and
    \item
      Each Team shall deposit the Escrow Amount with respect to each of
      its Adjustment Players with the Escrow Agent.
    \end{enumerate}
  \item
    Except as set forth in Section 12(c)(4) below: (i) the Base Escrow
    Amount for each Adjustment Player (other than a Two-Way Player)
    shall be collected in twelve (12) equal installments from each of
    the player's semi-monthly Compensation payments on each Deduction
    Date; (ii) the Performance Bonus Escrow Amount for each Adjustment
    Player shall be collected in one (1) installment on the Performance
    Bonus Deduction Date; (iii) the Trade Bonus Escrow Amount for each
    Adjustment Player shall be collected in one (1) installment on the
    Trade Bonus Deduction Date; and (iv) the Base Escrow Amount for each
    Adjustment Player under a Two-Way Contract shall be collected by the
    Two-Way Player's Team by deducting ten percent (10\%) of the portion
    of the player's Two-Way NBA Salary on each of the applicable twelve
    (12) semi-monthly Compensation payments on each Deduction Date.
  \item
    The procedure for deducting and depositing Escrow Amounts shall be
    as follows:

    \begin{enumerate}
    \def\labelenumiii{(\roman{enumiii})}
    \item
      The NBA will prepare and send to the Players Association the
      Escrow Schedules on or before November 8 of each Salary Cap Year,
      and periodically thereafter to reflect any new or adjusted Escrow
      Amounts calculated in accordance with Section 12(c)(4) below.
      Notwithstanding the foregoing, the Escrow Schedules prepared
      during the Regular Season shall not include Base Escrow Amounts in
      respect of Two-Way Players. The Escrow Schedules in respect of the
      May 1 Deduction Date shall be updated to include Base Escrow
      Amounts in respect of Two-Way Players.
    \item
      \begin{enumerate}
      \def\labelenumiv{(\Alph{enumiv})}
      \tightlist
      \item
        Within three (3) business days after each Deduction Date, each
        Team shall deliver to the Escrow Agent, in accordance with the
        Salary Escrow Agreement, the aggregate Base Escrow Amounts that
        the Team is obligated to deduct with respect to such Deduction
        Date for all of its Adjustment Players (other than Two-Way
        Players). Within three (3) business days following the date of
        the last game of the NBA Finals occurring during each Salary Cap
        Year, each Team shall deliver to the Escrow Agent, in accordance
        with the Salary Escrow Agreement, the aggregate Performance
        Bonus Escrow Amounts for all of its Adjustment Players. Within
        three (3) business days following an Adjustment Player's Trade
        Bonus Deduction Date, the player's Team shall deliver to the
        Escrow Agent, in accordance with the Salary Escrow Agreement,
        the player's Trade Bonus Escrow Amount. All amounts received by
        the Escrow Agent shall be invested and disbursed in accordance
        with the provisions of the Salary Escrow Agreement. With respect
        to Two-Way Players, within three (3) business days of the
        January 1, March 1 and May 1 Deduction Dates, each Team shall
        deliver to the Escrow Agent in accordance with the Salary Escrow
        Agreement, the aggregate amount withheld from the Compensation
        of the Team's Two-Way Players for the prior three Deduction
        Dates and current Deduction Date pursuant to Section 12(c)(2)
        above.
      \item
        It is the intention of the NBA and the Players Association that
        the provisions of this Section 12(c) result each Salary Cap Year
        in the delivery to the Escrow Agent, with respect to each
        Adjustment Player, of the player's full Escrow Amount.
        Accordingly, if for any reason the procedures in this Section
        12(c) would result in less than the full Escrow Amount with
        respect to any Adjustment Player being delivered to the Escrow
        Agent, the NBA and the Players Association shall agree on such
        supplemental measures as are necessary to effectuate the
        deduction of the appropriate amount from the player's
        Compensation for delivery to the Escrow Agent.
      \end{enumerate}
    \end{enumerate}
  \item
    After November 8, the NBA shall periodically update the Escrow
    Schedules to add Escrow Amounts for players (other than Two-Way
    Players) who enter into new Player Contracts and to make such
    adjustments as may be necessary to previously-listed Escrow Amounts
    (such as adjustments resulting from earned Performance Bonuses,
    Contract terminations, Renegotiations, etc.). Any portion of a Base
    Escrow Amount that has not been deducted as of the date any such
    updated Schedules are prepared shall be deducted in equal
    installments from each of the remaining semi-monthly Compensation
    payments to be made to the player from November 15 through May 1 of
    the applicable Salary Cap Year. Notwithstanding the foregoing, this
    Section 12(c)(4) shall not apply to the procedures for Two-Way
    Contracts, which are provided for in Section 12(c)(2) and (3) above.
  \item
    Within seven (7) days after receiving any set of Escrow Schedules
    from the NBA, or within seven (7) days after any event that the
    Players Association believes warrants a change in any
    previously-issued Schedules, the Players Association may bring a
    proceeding before the System Arbitrator, in accordance with Article
    XXXII, Section 10, contesting the NBA's calculation of any player's
    Escrow Amount for such Salary Cap Year. Notwithstanding the
    commencement of any such proceeding, each Team shall commence and
    continue remitting to the Escrow Agent the total deductions due with
    respect to each Deduction Date and each Performance Bonus Deduction
    Date as set forth in the Schedules, and in no event shall any Team
    be prohibited from remitting to the Escrow Agent any such deduction
    prior to a final determination in any such proceeding.
  \item
    In the event that the NBA makes a determination in accordance with
    Section 12(c)(4) above, or a final determination is made in a
    proceeding in accordance with Section 12(c)(5) above, that an Escrow
    Amount was erroneously calculated by the NBA, the sole remedy with
    respect to any amounts erroneously deducted from the player's Salary
    shall be to modify, as soon as practicable, the deduction schedule
    applicable to such player so as to reduce, in equal amounts, all
    scheduled future deductions from post-determination payments of
    Compensation until the amount of any prior over-deduction is fully
    off-set; provided, however, that to the extent that reducing the
    player's future deductions would not fully offset the prior
    over-deductions, the parties shall instruct the Escrow Agent to pay
    the player as soon as practicable, with interest, such additional
    amounts as are necessary to fully off-set such over-deductions.
  \end{enumerate}
\item
  \textbf{Reconciliation Procedures.}

  \begin{enumerate}
  \def\labelenumii{(\arabic{enumii})}
  \tightlist
  \item
    In the event of an Overage: (i) the NBA shall be entitled to receive
    from the Escrow Agent, with respect to each Adjustment Player, such
    player's Individual Compensation Adjustment Amount (or, in the event
    that the player's Escrow Amount is less than his Individual
    Compensation Adjustment Amount, a portion of his Individual
    Compensation Adjustment Amount equal to his Escrow Amount); and (ii)
    each Adjustment Player shall be entitled to receive from the Escrow
    Agent the amount, if any, by which the player's Escrow Amount
    exceeds his Individual Compensation Adjustment Amount. In the event
    that there is no Overage, each Adjustment Player shall be entitled
    to receive from the Escrow Agent his entire Escrow Amount.
  \item
    Any interest earned on Escrow Amounts remitted to the Escrow Agent
    shall be allocated among the Adjustment Players, collectively, and
    the NBA in proportion to the percentage of the aggregate Escrow
    Amounts that the Adjustment Players, collectively, and the NBA are
    to receive from the Escrow Agent in accordance with Section 12(d)(1)
    above. The Adjustment Players' collective share of interest shall be
    allocated among the individual players in proportion to the amount
    each player is entitled to receive from the Escrow Agent in
    accordance with Section 12(d)(1) above.
  \item
    The parties shall cause the Accountants to include in the Interim
    Audit Report and the Audit Report (or, if no final Audit Report has
    been submitted at the conclusion of the Audit Report Challenge
    Period, in the Interim Escrow Audit Report) for each Salary Cap Year
    schedules setting forth, with respect to such Salary Cap Year:

    \begin{enumerate}
    \def\labelenumiii{(\roman{enumiii})}
    \tightlist
    \item
      the amount of any Overage;
    \item
      the Aggregate Salaries and Benefits Adjustment Amount, if any;
    \item
      the Aggregate Compensation Adjustment Amount, if any;
    \item
      the New Benefits Adjustment Amount, if any, and the amount of the
      reduction in the Additional Benefit Amount, if any;
    \item
      each Adjustment Player's Individual Compensation Adjustment
      Amount, if any;
    \item
      each Adjustment Player's Escrow Amount, if any, as set forth in
      the Escrow Schedules;
    \item
      a list of all Adjustment Players whose Individual Compensation
      Adjustment Amounts exceed their Escrow Amounts, which list shall
      also include (A) each such player's Individual Compensation
      Adjustment Amount, (B) each such player's Escrow Amount, (C) the
      amount by which each such player's Individual Compensation
      Adjustment Amount exceeds his Escrow Amount, (D) the sum of all
      such players' Escrow Amounts, (E) the sum of all such players'
      Individual Compensation Adjustment Amounts, and (F) the aggregate
      amount by which all such players' Individual Compensation
      Adjustment Amounts exceed their Escrow Amounts;
    \item
      a list of all Adjustment Players whose Individual Compensation
      Adjustment Amounts are equal to or less than their Escrow Amounts,
      which list shall also include (A) each such player's Individual
      Compensation Adjustment Amount, (B) each such player's Escrow
      Amount, (C) the amount, if any, by which each such player's Escrow
      Amount exceeds his Individual Compensation Adjustment Amount, (D)
      the sum of all such players' Escrow Amounts, (E) the sum of all
      such players' Individual Compensation Adjustment Amounts, and (F)
      the aggregate amount by which all such players' Escrow Amounts
      exceed their Individual Compensation Adjustment Amounts;
    \item
      in accordance with the provisions of Section 12(d)(1) and (d)(2)
      above, (A) the percentage of the interest earned on the Escrow
      Amounts to be allocated to the NBA, (B) the percentage of the
      interest earned on the Escrow Amounts to be allocated to the
      Adjustment Players collectively, and (C) the percentage of the
      interest earned on the Escrow Amounts to be allocated to each
      individual Adjustment Player;
    \item
      the Tax Level; and
    \item
      the amount, if any, by which each Team's Team Salary as computed
      in Section 12(f) below exceeds the Tax Level.
    \end{enumerate}
  \item
    In addition to the information described in Section 12(d)(3) above,
    the parties shall cause the Accountants to include in the Audit
    Report (or, if no final Audit Report has been submitted at the
    conclusion of the Audit Report Challenge Period, in the Interim
    Escrow Audit Report) a separate Notice to the NBA and the Players
    Association, in the form attached to the Salary Escrow Agreement,
    setting forth:

    \begin{enumerate}
    \def\labelenumiii{(\roman{enumiii})}
    \tightlist
    \item
      in the space designated in paragraph 1 of the Notice, the sum of
      the amounts described in Section 12(d)(3)(vii)(D) and Section
      12(d)(3)(viii)(E) above, which sum is to be disbursed by the
      Escrow Agent to the NBA;
    \item
      in the space designated in paragraph 2 of the Notice, the amounts
      described in Section 12(d)(3)(viii)(C) above, which amounts are to
      be disbursed by the Escrow Agent to each respective Adjustment
      Player described in Section 12(d)(3)(viii) above;
    \item
      in the space designated in paragraph 3 of the Notice, the
      information described in Section 12(d)(3)(ix)(A) above, which
      information shall be the basis for the Escrow Agent's calculation
      of interest earned on the Escrow Amounts, which interest is to be
      disbursed by the Escrow Agent to the NBA; and
    \item
      in the space designated in paragraph (4) of the Notice, the
      information described in Section 12(d)(3)(ix)(C) above, which
      information shall be the basis for the Escrow Agent's calculation
      of interest earned on the Escrow Amounts, which interest is to be
      disbursed by the Escrow Agent to each Adjustment Player.
    \end{enumerate}
  \item
    No later than seven (7) business days after the earlier of (i) the
    completion of the Audit Report for the prior Salary Cap Year, or
    (ii) the completion of the Audit Report Challenge Period, the NBA
    and/or the Players Association shall deliver to the Escrow Agent the
    completed Notice to the NBA and the Players Association. As soon as
    practicable following receipt of such Notice, the Escrow Agent shall
    disburse the specified sums to the specified payees.
  \item
    Any amounts that the Escrow Agent is obligated to disburse to a
    player pursuant to this Section 12, including the amounts described
    in Section 12(d)(4)(iv) above, shall be reduced by all amounts
    required to be withheld by federal, state, and local authorities,
    which withholdings shall be disbursed by the Escrow Agent to the
    player's Team for remittance to the appropriate authorities. To
    assist the Escrow Agent in disbursing the appropriate amounts to
    each Adjustment Player and his respective Team, each Team, based on
    the information set forth in paragraph 2 and paragraph 4 of the
    Notice to the NBA and the Players Association, shall promptly
    provide the Escrow Agent with a schedule for each of its Adjustment
    Players showing the exact withholding amount to be disbursed to the
    Team for remittance to the appropriate federal, state and local
    authorities. In no circumstance shall the employer's share of FICA,
    FUTA, or any other employer taxes be paid out of the amounts
    deposited in escrow or any interest or earnings thereon. Any such
    obligations shall remain with each player's individual employer.
  \end{enumerate}
\item
  \textbf{Aggregate Compensation Adjustment Amount Shortfalls.}

  \begin{enumerate}
  \def\labelenumii{(\arabic{enumii})}
  \tightlist
  \item
    If, with respect to any Salary Cap Year, there is an Aggregate
    Compensation Adjustment Amount Shortfall, then the Contract (other
    than a Two-Way Contract) of each of the following Salary Cap Year's
    Adjustment Players shall be amended by operation of this Agreement,
    in accordance with Section 12(e)(2) below, such that the aggregate
    Compensation paid to all such players with respect to the Season
    covered by such following Salary Cap Year shall be reduced by the
    Aggregate Compensation Adjustment Amount Shortfall, which reduction
    shall be in addition to the full amount of any reduction for such
    following Salary Cap Year called for in Section 12(b)(1)-(2) above.
  \item
    The Individual Shortfall Adjustment Amount for each Adjustment
    Player whose Contract is amended in accordance with Section 12(e)(1)
    above shall be calculated by multiplying the Aggregate Compensation
    Adjustment Amount Shortfall for the prior Salary Cap Year by a
    fraction, the numerator of which is the player's then-current
    Salary, and the denominator of which is the sum of all such players'
    then-current Salaries. For purposes of calculating the fraction
    described in the preceding sentence, (i) the Salary of a player
    making the Minimum Player Salary shall include the portion of such
    Minimum Player Salary that is reimbursed out of the League-wide
    benefits fund described in Article IV, Section 6(g)(2); and (ii) the
    Salary of a player under a Rookie Scale Contract whose Compensation
    under his Contract is increased pursuant to Article VIII, Section 5
    shall include the portion of his Compensation attributable to the
    Rookie Scale Conforming Increases (as described in Article VIII,
    Section 5(a)) that is reimbursed out of the League-wide benefits
    fund described in Article IV, Section 6(g)(4).
  \item
    The Individual Shortfall Adjustment Amount for each Adjustment
    Player shall be deducted by the player's Team in four (4) equal
    installments from each of the player's first four (4) semi-monthly
    payment dates under paragraph 3 of the Uniform Player Contract
    following delivery to the Escrow Agent of the completed Notice to
    the NBA and the Players Association. All such deductions shall be
    promptly remitted by the Teams to the NBA.
  \end{enumerate}
\item
  \textbf{Team Payments.}

  \begin{enumerate}
  \def\labelenumii{(\arabic{enumii})}
  \tightlist
  \item
    Each Team whose Team Salary exceeds the Tax Level for any Salary Cap
    Year shall be required to pay a tax to the NBA. For each Salary Cap
    Year, the tax shall be calculated: (A) using the rates in Section
    12(f)(1)(i) (``Standard Tax Rates'') for any Team whose Team Salary
    did not exceed the Tax Level in three (3) or more of the four (4)
    Salary Cap Years immediately preceding the current Salary Cap Year;
    and (B) using the rates shown in Section 12(f)(1)(ii) (``Repeater
    Tax Rates'') for any Team whose Team Salary exceeded the Tax Level
    in three (3) or more of the four (4) Salary Cap Years immediately
    preceding the current Salary Cap Year.

    \begin{enumerate}
    \def\labelenumiii{(\roman{enumiii})}
    \tightlist
    \item
      Standard Tax Rates:
    \end{enumerate}

    \begin{longtable}[]{@{}cc@{}}
    \toprule
    \begin{minipage}[b]{0.42\columnwidth}\centering\strut
    Incremental Team Salary Above Tax Level\strut
    \end{minipage} &
    \begin{minipage}[b]{0.52\columnwidth}\centering\strut
    Tax Rate for Increment\strut
    \end{minipage}\tabularnewline
    \midrule
    \endhead
    \begin{minipage}[t]{0.42\columnwidth}\centering\strut
    \$0 -- \$4,999,999\strut
    \end{minipage} &
    \begin{minipage}[t]{0.52\columnwidth}\centering\strut
    \$1.50-for-\$1\strut
    \end{minipage}\tabularnewline
    \begin{minipage}[t]{0.42\columnwidth}\centering\strut
    \$5,000,000 -- \$9,999,999\strut
    \end{minipage} &
    \begin{minipage}[t]{0.52\columnwidth}\centering\strut
    \$1.75-for-\$1\strut
    \end{minipage}\tabularnewline
    \begin{minipage}[t]{0.42\columnwidth}\centering\strut
    \$10,000,000 -- \$14,999,999\strut
    \end{minipage} &
    \begin{minipage}[t]{0.52\columnwidth}\centering\strut
    \$2.50-for-\$1\strut
    \end{minipage}\tabularnewline
    \begin{minipage}[t]{0.42\columnwidth}\centering\strut
    \$15,000,000 -- \$19,999,999\strut
    \end{minipage} &
    \begin{minipage}[t]{0.52\columnwidth}\centering\strut
    \$3.25-for-\$1\strut
    \end{minipage}\tabularnewline
    \begin{minipage}[t]{0.42\columnwidth}\centering\strut
    \$20,000,000 and over\strut
    \end{minipage} &
    \begin{minipage}[t]{0.52\columnwidth}\centering\strut
    Tax rates increase by \$0.50 for each additional \$5,000,000
    increment above the Tax Level (e.g., for Team Salary \$20,000,000 to
    \$24,999,999 above the Tax Level, the Tax rate is \$3.75-for-\$1 for
    that increment).\strut
    \end{minipage}\tabularnewline
    \bottomrule
    \end{longtable}

    \emph{Example: In 2017-18, a Team with Team Salary that exceeds the
    Tax Level by \$11 million would pay a tax of \$18.75 million (i.e.,
    \$5 million times \$1.50, plus \$5 million times \$1.75, plus \$1
    million times \$2.50).}

    \begin{enumerate}
    \def\labelenumiii{(\roman{enumiii})}
    \setcounter{enumiii}{1}
    \tightlist
    \item
      Repeater Tax Rates:
    \end{enumerate}

    \begin{longtable}[]{@{}cc@{}}
    \toprule
    \begin{minipage}[b]{0.42\columnwidth}\centering\strut
    Incremental Team Salary Above Tax Level\strut
    \end{minipage} &
    \begin{minipage}[b]{0.52\columnwidth}\centering\strut
    Tax Rate for Increment\strut
    \end{minipage}\tabularnewline
    \midrule
    \endhead
    \begin{minipage}[t]{0.42\columnwidth}\centering\strut
    \$0 -- \$4,999,999\strut
    \end{minipage} &
    \begin{minipage}[t]{0.52\columnwidth}\centering\strut
    \$2.50-for-\$1\strut
    \end{minipage}\tabularnewline
    \begin{minipage}[t]{0.42\columnwidth}\centering\strut
    \$5,000,000 -- \$9,999,999\strut
    \end{minipage} &
    \begin{minipage}[t]{0.52\columnwidth}\centering\strut
    \$2.75-for-\$1\strut
    \end{minipage}\tabularnewline
    \begin{minipage}[t]{0.42\columnwidth}\centering\strut
    \$10,000,000 -- \$14,999,999\strut
    \end{minipage} &
    \begin{minipage}[t]{0.52\columnwidth}\centering\strut
    \$3.50-for-\$1\strut
    \end{minipage}\tabularnewline
    \begin{minipage}[t]{0.42\columnwidth}\centering\strut
    \$15,000,000 -- \$19,999,999\strut
    \end{minipage} &
    \begin{minipage}[t]{0.52\columnwidth}\centering\strut
    \$4.25-for-\$1\strut
    \end{minipage}\tabularnewline
    \begin{minipage}[t]{0.42\columnwidth}\centering\strut
    \$20,000,000 and over\strut
    \end{minipage} &
    \begin{minipage}[t]{0.52\columnwidth}\centering\strut
    Tax rates increase by \$0.50 for each additional \$5,000,000
    increment above the Tax Level (e.g., for Team Salary \$20,000,000 to
    \$24,999,999 above the Tax Level, the Tax rate is \$4.75-for-\$1 for
    that increment).\strut
    \end{minipage}\tabularnewline
    \bottomrule
    \end{longtable}

    \emph{Example: In 2017-18, a Team (whose Team Salary exceeded the
    Tax Level in all the 2014-15, 2015-16 and 2016-17 Salary Cap Years)
    with Team Salary that exceeds the Tax Level by \$11 million would
    pay a tax of \$29.75 million (i.e., \$5 million times \$2.50, plus
    \$5 million times \$2.75, plus \$1 million times \$3.50).}
  \item
    For purposes of computing the amount of tax a Team owes: (i) a
    Team's Team Salary shall be the sum of: (A) its Team Salary as of
    the start of its last Regular Season game, plus all Performance
    Bonuses excluded from Salary under Section 3(d) above but actually
    earned by the player during such Salary Cap Year, less all
    Performance Bonuses included in Salary under Section 3(d) above but
    not actually earned by the player during such Salary Cap Year; plus
    (B) with respect to any trade that occurs following the conclusion
    of the Team's last Regular Season game, the portion of any trade
    bonus earned by a player that is included in the Team's Team Salary
    for such Salary Cap Year, plus (C) any amount that is added to the
    Team's Team Salary for such Salary Cap Year following the start of
    the Team's last Regular Season game pursuant to Section 4(a)(1)(iii)
    above; and (ii) the Salary attributable to a Contract between a Team
    and a Free Agent with zero (0) Years of Service or one (1) Year of
    Service shall be deemed to be the greater of (x) such Salary or (y)
    the Minimum Player Salary that would be applicable to a player with
    two (2) Years of Service, or in the event such player's Contract is
    terminated during the Regular Season, the Minimum Player Salary that
    would be applicable to a player with two (2) Years of Service,
    reduced pro-rata to reflect the player's post-termination Salary.
  \item
    Each Team that owes a tax shall make the required tax payment to the
    NBA no later than ten (10) business days following the earlier of
    (i) the completion of the Audit Report for the prior Salary Cap
    Year, or (ii) the completion of the Audit Report Challenge Period.
  \item
    For purposes of this Section 12(f) and subject to the provisions of
    Section 12(f)(1) above, Team Salary shall be calculated by the
    Accountants in the same manner as Team Salary is calculated by the
    Accountants for purposes of computing Total Salaries and Benefits in
    the Audit Report.
  \end{enumerate}
\item
  \textbf{Escrow and Tax Proceeds.} All amounts remitted to the NBA by
  the Escrow Agent or NBA Teams pursuant to this Section 12 shall be the
  exclusive property of the NBA, and the use and/or disposition of all
  such amounts, including the allocation or distribution of such amounts
  to one (1) or more NBA Teams, if any, shall be subject only to the
  following limitations:

  \begin{enumerate}
  \def\labelenumii{(\arabic{enumii})}
  \tightlist
  \item
    The Escrow Amounts remitted to the NBA by the Escrow Agent pursuant
    to Section 12(d) above with respect to each Salary Cap Year shall be
    used and/or distributed as follows:

    \begin{enumerate}
    \def\labelenumiii{(\roman{enumiii})}
    \tightlist
    \item
      the NBA may elect to distribute all or a portion of such amounts
      to NBA teams, provided that any such distribution pursuant to this
      Section 12(g)(1)(i) must be made to all Teams in equal shares; and
    \item
      amounts not distributed in accordance with Section 12(g)(1)(i)
      above shall be used for one (1) or more ``League purposes'' (as
      defined in Section 12(g)(3) below) selected by the NBA.
    \end{enumerate}
  \item
    The tax amounts remitted to the NBA by NBA Teams pursuant to Section
    12(f) above for each Salary Cap Year shall be used and/or
    distributed as follows:

    \begin{enumerate}
    \def\labelenumiii{(\roman{enumiii})}
    \tightlist
    \item
      the NBA may elect to distribute up to fifty percent (50\%) of such
      amounts to one (1) or more Teams based in whole or in part on the
      fact that such Team(s) did not owe a tax for such Salary Cap Year
      (e.g., the NBA could elect to distribute fifty percent (50\%) of
      such amounts in equal shares to all non-taxpayers in such Salary
      Cap Year); and
    \item
      amounts not distributed in accordance with Section 12(g)(2)(i)
      above shall be used for one (1) or more ``League purposes'' (as
      defined in Section 12(g)(3) below) selected by the NBA.
    \end{enumerate}
  \item
    For purposes of this Section 12(g), the use of Escrow Amounts or tax
    amounts for a ``League purpose'' shall mean the use of such amounts
    for any purpose, including, but not limited to, the distribution of
    such amounts to one (1) or more Teams; provided, however, that such
    amounts may not be distributed to a Team or expended for the benefit
    or detriment of a Team in a manner that is based, directly or
    indirectly, in whole or in part, on the amount of the Team's Team
    Salary or on whether the Team is a taxpayer. Without limiting the
    foregoing, a team assistance plan adopted by the NBA and funded, in
    whole or in part, with Escrow Amounts and/or tax amounts shall be
    considered a ``League purpose'' if, pursuant to the plan, a Team's
    entitlement to an assistance receipt and/or the amount of such
    receipt is based, in whole or in part, on a profit, loss, and/or
    expense computation determined by the NBA under which the Team is
    credited with a Team Salary no less than the league average;
    provided, however, that in order to qualify as a ``League purpose,''
    such a plan may not otherwise base a Team's entitlement to
    assistance and/or the amount of such assistance on the amount of a
    Team's Team Salary or on whether the Team is a taxpayer.
  \end{enumerate}
\item
  \textbf{Revenue Decline.} If BRI for any Salary Cap Year substantially
  decreases from the prior Salary Cap Year's BRI, and, as a result, the
  players receive more than the Designated Share (as defined in Section
  12(b)(3) above) for such Salary Cap Year, then the NBA and the Players
  Association shall negotiate in good faith to agree upon an adjustment
  to the provisions of this Agreement in a manner reasonably
  satisfactory to the parties to address the issue.
\item
  \textbf{Miscellaneous.}

  \begin{enumerate}
  \def\labelenumii{(\arabic{enumii})}
  \item
    Notwithstanding any other provision of this Agreement, the
    computation of an Adjustment Player's Salary shall for purposes of
    the rules set forth in this Agreement be made without regard to any
    reduction in such player's Compensation made pursuant to this
    Section 12.
  \item
    The NBA shall be permitted to assign to such designee, as the NBA
    may determine, any rights the NBA has to receive amounts from the
    Escrow Agent or NBA Teams pursuant to this Section 12.
  \item
    Consistent with Sections 3(f) and 3(j) above (One-Year Minimum
    Contracts and Existing Rookie Scale Contract Increases), except for
    purposes of calculating the amounts referred to in the definitions
    of Escrow Amount, Individual Compensation Adjustment Amount, and
    Individual Shortfall Adjustment Amount (set forth in Sections
    12(a)(9), (a)(11), and (a)(12) above), and subject to Section
    12(f)(2)(ii) above, (i) the Salary of every player who signs a
    one-year Contract for the Minimum Player Salary applicable to such
    player shall, for all other purposes in this Section 12, be the
    lesser of (x) such Minimum Player Salary, or (v) the portion of such
    Minimum Player Salary that is not reimbursed out of the League-wide
    benefits fund described in Article IV, Section 6(g)(2); and (ii) the
    Salary of every Rookie Scale player whose Compensation under his
    Rookie Scale Contract is increased pursuant to Article VIII, Section
    5, shall equal the player's Salary under his Contract prior to
    application of the existing Rookie Scale Conforming Increases and
    shall not include any portion of the Rookie Scale Conforming
    Increases paid to the player that is reimbursed out of the
    League-wide benefits fund described in Article IV, Section 6(g)(4).
  \item
    \begin{enumerate}
    \def\labelenumiii{(\roman{enumiii})}
    \tightlist
    \item
      For purposes of the computations made by the Accountants pursuant
      to Section 10 above and this Section 12, the Salary of a player
      who is suspended by the NBA -- but not by a Team -- shall be
      reduced (for the Salary Cap Year covering the Season during which
      the player is suspended) by an amount equal to fifty percent
      (50\%) of the suspension-related Compensation amount that is
      collected from the player and retained by the NBA at the time the
      computations of the Accountants are made. Other than as set forth
      in the preceding sentence, the computation of a player's Salary
      under this Agreement shall be made without regard to any reduction
      in Compensation that results from the player's suspension by the
      NBA or his Team.
    \item
      When (A) a player has forfeited a portion of his Compensation for
      a Season as a result of a suspension imposed by the NBA or his
      Team and (B) the player's Compensation is later reduced pursuant
      to Section 12(b)(1) and (2) above, the player shall be entitled to
      a refund of a portion of the Compensation forfeited as a result of
      the suspension. The refund shall be in an amount equal to (x) the
      player's Individual Compensation Adjustment Amount for the Salary
      Cap Year to which the suspension related multiplied by a fraction,
      the numerator of which is the amount of the player's Compensation
      that was forfeited as a result of the suspension and the
      denominator of which is the player's Base Compensation for such
      Season as of the date(s) he served the suspension, less (y) all
      amounts required to be withheld by federal, state, and local
      authorities. For purposes of the calculation required in clause
      (x) above, a player's Individual Compensation Adjustment Amount
      shall be deemed to include only the portion of the player's
      Individual Compensation Adjustment Amount that relates to the
      player's Base Compensation earned under the Player Contract for
      the NBA team that the player was playing for while he was
      suspended. Such refund shall be made to the player within sixty
      (60) days following the Accountants' submission to the NBA and the
      Players Association of a final Audit Report or an Interim Escrow
      Audit Report for the Salary Cap Year covering the Season for which
      the suspension-related Compensation amount is collected.
    \end{enumerate}
  \item
    In the event that the Overage for any Salary Cap Year exceeds the
    total of (i) the Aggregate Salaries and Benefits Adjustment Amount
    for such Salary Cap Year, plus (ii) the Additional Benefit Amount as
    provided for by Article IV, Section 4(d)(1) for such Salary Cap
    Year, the NBA shall not be entitled to reduce player Compensation in
    such Salary Cap Year or any subsequent Salary Cap Year so as to
    recover any amounts in excess of the Aggregate Compensation
    Adjustment Amount.
  \end{enumerate}
\end{enumerate}

\chapter{ROOKIE SCALE}\label{rookie-scale}

\section{Rookie Scale Contracts for First Round
Picks.}\label{rookie-scale-contracts-for-first-round-picks.}

\begin{enumerate}
\def\labelenumi{(\alph{enumi})}
\item
  Each Rookie Scale Contract between a Team and a First Round Pick shall
  cover a period of two (2) Seasons, but shall have an Option in favor
  of the Team for the player's third Season and a second Option in favor
  of the Team for the player's fourth Season. The Option for the
  player's third Season shall be exercisable during the period from the
  day following the last day of the first Season through the immediately
  following October 31. The Option for the player's fourth Season shall
  be exercisable during the period from the day following the last day
  of the second Season through the immediately following October 31.
  (For clarity, consistent with the rule set forth in Article XLII,
  Section 2, if October 31 in any year falls on a Saturday, Sunday or
  Federal Holiday, then the deadline for exercising Options in Rookie
  Scale Contracts shall be deemed to fall on the following business
  day.) Such Options shall be exercisable by notice to the player that
  is either personally delivered to the player or his representative or
  sent by email or pre-paid certified, registered, or overnight mail to
  the last known address of the player or his representative, signed by
  the Team, informing the player that the Team has exercised such
  Option.
\item
  \begin{enumerate}
  \def\labelenumii{(\roman{enumii})}
  \tightlist
  \item
    The Rookie Salary Scale applicable to a First Round Pick is
    determined by the first Season to be covered by the player's Rookie
    Scale Contract. Accordingly, for example, if a player's Rookie Scale
    Contract commences with the 2017-18 Season, the 2017-18 Rookie
    Salary Scale shall apply. Within a particular Rookie Salary Scale, a
    First Round Pick's applicable Rookie Scale Amounts are determined by
    the player's selection number in the NBA Draft. Accordingly, for
    example, the Rookie Scale Amounts applicable to the eighth player
    selected in the first round of the NBA Draft shall be those
    specified in the applicable Rookie Salary Scale for the eighth pick.
    Notwithstanding anything to the contrary in this Section 1(b)(i) or
    in Section 1(b)(ii) below, beginning on January 10 of each Season,
    an unsigned First Round Pick's applicable Rookie Scale Amount for
    such Season shall be reduced daily through the end of the Regular
    Season by the product of the applicable Rookie Scale Amount (as set
    forth in the applicable Rookie Salary Scale) multiplied by a
    fraction, the numerator of which is one (1) and the denominator of
    which is the total number of days in such Regular Season.
  \item
    Notwithstanding Section 1(b)(i) above, if, pursuant to any provision
    of this Agreement or the NBA Constitution and By-Laws, one (1) or
    more Teams is required to forfeit one (1) or more draft picks in the
    first round of a particular NBA Draft, then:

    \begin{enumerate}
    \def\labelenumiii{(\Alph{enumiii})}
    \tightlist
    \item
      the Rookie Salary Scale for the Salary Cap Year immediately
      following such Draft shall be adjusted by removing one (1) or more
      Rookie Scale Amounts from the middle of the Rookie Salary Scale,
      as follows: if one (1) first round pick is forfeited, then the
      Rookie Scale Amounts that would have been applicable to the 15th
      player selected in the first round (absent any forfeiture of
      picks) (hereinafter, the ``15th Pick'') shall be removed from the
      Rookie Salary Scale; if two (2) first round picks are forfeited,
      then the Rookie Scale Amounts applicable to the 15th Pick and the
      pick immediately following the 15th Pick shall be removed from the
      Rookie Salary Scale; if three (3) first round picks are forfeited,
      then the Rookie Scale Amounts applicable to the 15th Pick and the
      picks immediately preceding and immediately following the 15th
      Pick shall be removed from the Rookie Salary Scale; and if more
      than three picks are forfeited, additional Rookie Scale Amounts
      shall be removed from the Rookie Salary Scale in accordance with
      the foregoing procedure; and
    \item
      the Rookie Scale Amounts applicable to players selected in such
      Draft shall be determined by their selection number under the
      Rookie Salary Scale as adjusted by Section 1(b)(ii)(A) above.
      Accordingly, for example, if one First Round Pick were forfeited
      in the first round of the 2018 Draft, the applicable Rookie Scale
      Amounts would remain unchanged for the first 14 picks, and the
      Rookie Scale Amounts applicable to the remaining 15 picks in the
      first round would be the Rookie Scale Amounts that (absent any
      forfeiture of picks) would have been applicable to picks 16
      through 30.
    \end{enumerate}
  \end{enumerate}
\item
  \begin{enumerate}
  \def\labelenumii{(\roman{enumii})}
  \tightlist
  \item
    A Rookie Scale Contract shall provide in each of the two (2) Seasons
    covered by the Contract and the first Option Year at least eighty
    percent (80\%) of the applicable Rookie Scale Amount in Current Base
    Compensation. Components of Salary in excess of eighty percent
    (80\%), if any, are subject to individual negotiation, except that
    (i) in no event may Salary plus Unlikely Bonuses for any Salary Cap
    Year exceed one hundred twenty percent (120\%) of the applicable
    Rookie Scale Amount, and (ii) a Rookie Scale Contract may not
    provide for a signing bonus (except for an ``international player''
    payment in excess of the Excluded International Player Payment
    Amount made in accordance with Article VII, Section 3(e)) or a loan.
    A Rookie Scale Contract may provide for a payment schedule in any
    Season that is more favorable to the player than that called for
    under paragraph 3 of the Uniform Player Contract, subject to the
    other provisions of this Agreement.
  \item
    A Rookie Scale Contract must provide for Compensation protection for
    lack of skill and injury or illness in each of the two (2) Seasons
    covered by the Contract and the first Option Year to the extent of
    not less than eighty percent (80\%) of the applicable Rookie Scale
    Amount. To the extent permitted by Article II, Section 4(l), a Team
    and a First Round Pick may negotiate additional conditions or
    limitations applicable to the player's Base Compensation protection,
    except that lack of skill and injury or illness protection to the
    extent of at least eighty percent (80\%) of the applicable Rookie
    Scale Amount in each of the first two (2) Seasons and the first
    Option Year shall contain no such individually-negotiated additional
    conditions or limitations.
  \item
    The terms and conditions (other than with respect to the payment
    schedule for the player's Base Compensation) that apply to the
    second Option Year shall be unchanged from all terms and conditions
    that applied to the player's first Option Year (including but not
    limited to the percentage of Base Compensation that is protected),
    except that the Salary, (excluding Incentive Compensation), Likely
    Bonuses, and Unlikely Bonuses for the second Option Year shall be
    increased over the Salary (excluding Incentive Compensation), Likely
    Bonuses, and Unlikely Bonuses, respectively, for the first Option
    Year by the applicable percentage specified in the applicable Rookie
    Salary Scale.
  \end{enumerate}
\item
  Notwithstanding any other provision of this Agreement, if a trade of a
  Rookie Scale Contract would, by reason of a trade bonus contained in
  such Contract, cause the player's Salary plus Unlikely Bonuses for the
  Salary Cap Year in which such trade occurs to exceed one hundred
  twenty percent (120\%) of the player's applicable Rookie Scale Amount
  for such Salary Cap Year, such player's trade bonus shall be deemed
  amended to the extent necessary to reduce the player's Salary plus
  Unlikely Bonuses for such Salary Cap Year to one hundred twenty
  percent (120\%) of the applicable Rookie Scale Amount.
\end{enumerate}

\section{Rookie Contracts for Later-Signed First Round
Picks.}\label{rookie-contracts-for-later-signed-first-round-picks.}

Except as provided in Section 3 below, a First Round Pick who does not
sign with the Team that holds his draft rights for any portion of the
three (3) Seasons following the NBA Draft in which he was selected (and
who did not play intercollegiate basketball during such period) may
enter into either (a) a Rookie Scale Contract in accordance with Section
1 above, or (b) if the Team has Room in excess of the applicable
first-year Rookie Scale Amount and subject to the provisions of Article
VII, a Contract covering no fewer than three (3) Seasons (not including
any Option Year) that provides for Base Compensation in the first Season
greater than one hundred twenty percent (120\%) of the applicable
first-year Rookie Scale Amount.

\section{Loss of Draft Rights.}\label{loss-of-draft-rights.}

If for any reason a Team fails to make a Required Tender to a First
Round Pick in accordance with Article X, withdraws a Required Tender to
a First Round Pick in accordance with Article X, or renounces a First
Round Pick in accordance with Article X, or if a First Round Pick
selected in a Subsequent Draft does not sign a Contract for a period of
one (1) year following such Subsequent Draft in accordance with Article
X, then the rules set forth in Sections 1 and 2 above shall not apply,
and such First Round Pick shall become a Rookie Free Agent. In addition,
any Team that fails to make a Required Tender to a First Round Pick,
withdraws a Required Tender to a First Round Pick, renounces a First
Round Pick, or fails to sign within one (1) year a First Round Pick
selected in a Subsequent Draft shall be prohibited from signing such
player until after he has signed a Player Contract with another NBA
Team, and either (a) the player completes the playing services called
for under the Contract, or (b) the Contract is terminated in accordance
with the NBA waiver procedure.

\section{2018-19 and 2019-20 Rookie Salary
Scales.}\label{and-2019-20-rookie-salary-scales.}

The Rookie Scale Amounts for the 2018-19 and 2019-20 Salary Cap Years
shall be calculated in accordance with the following:

\begin{enumerate}
\def\labelenumi{(\alph{enumi})}
\tightlist
\item
  For purposes of this Section 4, the ``Baseline Rookie Scale'' for the
  2017-18 Salary Cap Year shall be the Rookie Salary Scale annexed
  hereto as Exhibit B-2. The Baseline Rookie Scale for the 2018-19 and
  2019-20 Salary Cap Years shall be calculated by taking the Baseline
  Rookie Scale for the prior Salary Cap Year and adjusting the Rookie
  Scale amounts therein by applying the percentage increase (or
  decrease) in the Salary Cap from the preceding Salary Cap Year to the
  current Salary Cap Year (e.g., to determine the 2018-19 Baseline
  Rookie Scale, the percentage increase (or decrease) in the Salary Cap
  from the 2017-18 Salary Cap Year to the 2018-19 Salary Cap Year would
  be applied); provided, however, that the applicable ``4th Year
  Option'' and Qualifying Offer percentages shall remain unchanged.
\item
  The Rookie Salary Scale for the 2018-19 and 2019-20 Salary Cap Years
  shall be calculated by taking the Baseline Rookie Scale for that
  Salary Cap Year and increasing (i) the Rookie Scale amounts therein in
  respect of the 2018-19 Salary Cap Year by thirty percent (30\%), and
  (ii) the Rookie Scale amounts therein in respect of the 2019-20 Salary
  Cap Year and subsequent Salary Cap Years by forty-five percent (45\%).
  (For clarity, the foregoing 45\% increase in Rookie Scale Amounts for
  the 2019-20 Salary Cap Year would be applied in both the 2018-19 and
  2019-20 Rookie Salary Scales.)
\item
  For illustrative purposes, an example of the 2018-19 and 2019-20
  Rookie Salary Scales assuming five percent (5\%) annual growth in the
  Salary Cap is annexed hereto as Exhibit B-3.
\end{enumerate}

\section{Existing Rookie Scale Contract
Increases.}\label{existing-rookie-scale-contract-increases.}

\begin{enumerate}
\def\labelenumi{(\alph{enumi})}
\tightlist
\item
  In respect of any Rookie Scale Contract in effect as of the effective
  date of this Agreement that: (i) was entered into prior to the
  execution of this Agreement and is not terminated on or before June
  30, 2017; and (ii) has a term including the 2016-17, 2017-18, 2018-19,
  and/or 2019-20 Season, such Contract shall be deemed amended on July
  1, 2017 as follows: the remaining Base Compensation, Incentive
  Compensation, and Base Compensation protection shall be increased: (i)
  fifteen percent (15\%) in respect of the 2017-18 Season; (y) thirty
  percent (30\%) in respect of the 2018-19 Season; and (z) forty-five
  percent (45\%) in respect of the 2019-20 Season (collectively the
  ``Rookie Scale Conforming Increases'').
\item
  Rookie Scale Conforming Increases shall be paid by the player's Team
  in accordance with the payment schedule set forth in the player's
  Contract for each applicable Season (i.e., each payment to the player
  under his Contract shall be increased by 15\%, 30\% or 45\%, as
  applicable), and then reimbursed to the Team out of a League-wide fund
  created and maintained by the NBA.
\item
  As set forth in Article IV, Sections 6(g)(3) and 6(d), Rookie Scale
  Conforming Increases and the CBA benefits associated therewith (e.g.,
  the employer's share of payroll taxes) will both be included in the
  calculation of Benefits (and thus, consistent with Article I, Section
  1(www), in the calculation of Total Salaries and Benefits). However,
  subject to Section (d)(ii) below and as set forth in Article VII,
  Sections 3(j) and 4(a), the Rookie Scale Conforming Increases shall be
  excluded from the calculation of an individual player's Salary and
  each Team's Team Salary (and thus will not affect, for example, the
  amount of room, if applicable, a Team has below the Salary Cap, the
  amount of Traded Player Exceptions, etc.).
\item
  With respect to any player whose Contract (before applying the Rookie
  Scale Conforming Increase) provides for Base Compensation in respect
  of the 2017-18 Season (or any future Season) that is less than the
  Minimum Player Salary for the Salary Cap Year encompassing the
  applicable Season pursuant to 2017-18 Minimum Annual Salary Scale:

  \begin{enumerate}
  \def\labelenumii{(\roman{enumii})}
  \tightlist
  \item
    such player shall receive the applicable Rookie Scale Conforming
    Increase and then, if necessary, any additional Base Compensation
    required to reach his applicable Minimum Player Salary;
  \item
    such player shall have his Salary deemed equal to his applicable
    Minimum Player Salary; and
  \item
    notwithstanding Article IV, Section 6(g)(4), the amount that is the
    difference between his Salary under his Rookie Scale Contract prior
    to his Rookie Scale Conforming Increase and his applicable Minimum
    Player Salary shall be paid directly by his Team (and not reimbursed
    out of the League-wide Rookie Scale Conforming Increases fund).
  \end{enumerate}
\end{enumerate}

\chapter{LENGTH OF PLAYER CONTRACTS}\label{length-of-player-contracts}

\section{Maximum Term.}\label{maximum-term.}

Except where a shorter term is expressly provided for elsewhere in this
Agreement, a Player Contract entered into after the effective date of
this Agreement may cover, in the aggregate, up to but no more than four
(4) Seasons from the date such Contract is signed; provided, however,
that (a) a Player Contract between a Qualifying Veteran Free Agent and
his Prior Team may cover, in the aggregate, up to but no more than five
(5) Seasons from the date such Contract is signed, (b) an Extension of a
Rookie Scale Contract with a player (other than a Designated Player
Rookie Scale Extension with a Team's Designated Rookie Scale Player) may
cover, in the aggregate, up to but no more than five (5) Seasons from
the date such Extension is signed, (c) an Extension of a Veteran Player
(other than a Designated Veteran Player Extension with a Team's
Designated Veteran Player) may cover, in the aggregate, up to no more
than five (5) Seasons from the date such Extension is signed, (d) a
Designated Player Rookie Scale Extension with a Team's Designated Rookie
Scale Player must cover six (6) Seasons from the date such Extension is
signed, and (e) a Designated Veteran Player Extension with a Team's
Designated Veteran Player must cover six (6) Seasons from the date such
Extension is signed. For the avoidance of doubt and consistent with
Article VII, Section 9(a)(2), the maximum Contract and Extension lengths
described herein are inclusive of any Option Year contained in a
Contract or Extension.

\section{Computation of Time.}\label{computation-of-time.}

For purposes of Section 1 above and consistent with Article VII, Section
9(a)(1), if a Player Contract or Extension is signed after the beginning
of a Season, the Season in which the Contract or Extension is signed
shall be counted as one (1) full Season covered by the Contract or
Extension; and in the case of an Extension that is signed during the
period from the end of a Season through the immediately following June
30, the Season immediately preceding the signing of the Extension (i.e.,
the just-completed Season) shall be counted as one (1) full Season
covered by the Extension.

\chapter{PLAYER ELIGIBILITY AND NBA
DRAFT}\label{player-eligibility-and-nba-draft}

\section{Player Eligibility.}\label{player-eligibility.}

\begin{enumerate}
\def\labelenumi{(\alph{enumi})}
\tightlist
\item
  No player may sign a Contract or play in the NBA unless he has been
  eligible for selection in at least one (1) NBA Draft. No player shall
  be eligible for selection in more than two (2) NBA Drafts.
\item
  A player shall be eligible for selection in the first NBA Draft with
  respect to which he has satisfied all applicable requirements of
  Section 1(b)(i) below and one of the requirements of Section 1(b)(ii)
  below:

  \begin{enumerate}
  \def\labelenumii{(\roman{enumii})}
  \item
    The player (A) is or will be at least nineteen (19) years of age
    during the calendar year in which the Draft is held, and (B) with
    respect to a player who is not an international player (defined
    below), at least one (1) NBA Season has elapsed since the player's
    graduation from high school (or, if the player did not graduate from
    high school, since the graduation of the class with which the player
    would have graduated had he graduated from high school); and
  \item
    \begin{enumerate}
    \def\labelenumiii{(\Alph{enumiii})}
    \tightlist
    \item
      The player has graduated from a four-year college or university in
      the United States (or is to graduate in the calendar year in which
      the Draft is held) and has no remaining intercollegiate basketball
      eligibility; or
    \item
      The player is attending or previously attended a four-year college
      or university in the United States, his original class in such
      college or university has graduated (or is to graduate in the
      calendar year in which the Draft is held), and he has no remaining
      intercollegiate basketball eligibility; or
    \item
      The player has graduated from high school in the United States,
      did not enroll in a four-year college or university in the United
      States, and four (4) calendar years have elapsed since such
      player's high school graduation; or
    \item
      The player did not graduate from high school in the United States,
      and four (4) calendar years have elapsed since the graduation of
      the class with which the player would have graduated had he
      graduated from high school; or
    \item
      The player has signed a player contract with a ``professional
      basketball team not in the NBA'' (defined below) that is located
      anywhere in the world, and has rendered services under such
      contract prior to the January 1 immediately preceding such Draft;
      or
    \item
      The player has expressed his desire to be selected in the Draft in
      a writing received by the NBA at least sixty (60) days prior to
      such Draft (an ``Early Entry'' player); or
    \item
      If the player is an ``international player'' (defined below), and
      notwithstanding anything contained in subsections (A) through (F)
      above:

      \begin{enumerate}
      \def\labelenumiv{(\arabic{enumiv})}
      \tightlist
      \item
        The player is or will be twenty-two (22) years of age during the
        calendar year of the Draft; or
      \item
        The player has signed a player contract with a ``professional
        basketball team not in the NBA'' (defined below) that is located
        in the United States, and has rendered services under such
        contract prior to the Draft; or
      \item
        The player has expressed his desire to be selected in the Draft
        in a writing received by the NBA at least sixty (60) days prior
        to such Draft (an ``Early Entry'' player).
      \end{enumerate}
    \end{enumerate}
  \end{enumerate}
\item
  For purposes of this Article X, an ``international player'' is a
  player: (i) who has maintained a permanent residence outside of the
  United States for at least the three (3) years prior to the Draft,
  while participating in the game of basketball as an amateur or as a
  professional outside of the United States; (ii) who has never
  previously enrolled in a college or university in the United States;
  and (iii) who did not complete high school in the United States.
\item
  For purposes of this Article X, a ``professional basketball team not
  in the NBA'' means any team that pays money or compensation of any
  kind -- in excess of a stipend for living expenses -- to a basketball
  player for rendering services to such team.
\end{enumerate}

\section{Term and Timing of Draft
Provisions.}\label{term-and-timing-of-draft-provisions.}

An NBA Draft will be held prior to the commencement of each NBA Season
covered by the term of this Agreement and, despite the expiration of the
other terms of this Agreement pursuant to Article XXXIX, prior to the
commencement of the 2024-25 NBA Season (or, if either party exercises
its option to terminate the Agreement pursuant to Article XXXIX, prior
to the commencement of the 2023-24 NBA Season). Each such Draft will be
held prior to the July 10 preceding the commencement of the NBA Season
on a date to be designated by the Commissioner.

\section{Number of Choices.}\label{number-of-choices.}

\begin{enumerate}
\def\labelenumi{(\alph{enumi})}
\tightlist
\item
  The NBA Draft shall consist of two (2) rounds, with each round
  consisting of the same number of selections as there will be Teams in
  the NBA the following Season. Each Team shall be required to exercise
  any and all draft selections in its possession during each round of
  the Draft.
\item
  If, pursuant to any provision of this Agreement or the NBA
  Constitution and By-Laws, any Team is required to forfeit one or more
  draft pick(s) in a particular NBA Draft, the number of players
  selected in the applicable round of the Draft will be reduced by the
  number of such forfeitures. (Thus, for example, if Team A is required
  to forfeit the ninth pick in the first round of the Draft (at a time
  when there are thirty (30) NBA Teams), there will only be twenty-nine
  (29) players selected in the first round of such Draft.) In the event
  the forfeiture relates to one or more first round picks, the Rookie
  Salary Scale will be adjusted as set forth in Article VIII, Section
  1(b)(ii). Other than as specifically agreed to herein, nothing
  contained in this Agreement shall be deemed to be an agreement of the
  Players Association to any provision of the NBA Constitution and
  By-Laws.
\end{enumerate}

\section{Negotiating Rights to Draft
Rookies.}\label{negotiating-rights-to-draft-rookies.}

\begin{enumerate}
\def\labelenumi{(\alph{enumi})}
\tightlist
\item
  A Team that drafts a player shall, during the period from the date of
  such NBA Draft (hereinafter, the ``Initial Draft'') to the date of the
  next Draft (hereinafter, the ``Subsequent Draft''), be the only Team
  with which such player may negotiate or sign a Player Contract,
  provided that, on or before the July 15 immediately following the
  Initial Draft (for a First Round Pick), or in the two (2) weeks before
  the September 5 immediately following the Initial Draft (for a Second
  Round Pick), such Team has made a Required Tender to such player. If a
  Team has made a Required Tender to such a player and the player has
  not signed a Player Contract within the period between the Initial
  Draft and the Subsequent Draft, the Team that drafted the player shall
  lose its exclusive right to negotiate with the player and the player
  will then be eligible for selection in the Subsequent Draft.
\item
  A Team that, in the Subsequent Draft, drafts a player who (i) was
  drafted in the Initial Draft, (ii) received a Required Tender from the
  Team that drafted him in the Initial Draft, and (iii) did not sign a
  Player Contract with such first Team prior to the Subsequent Draft,
  shall, during the period from the date of the Subsequent Draft to the
  date of the next NBA Draft, be the only Team with which such player
  may negotiate or sign a Player Contract, provided such Team has made a
  Required Tender to such player by the dates specified in Section 4(a)
  above. If such player has not signed a Player Contract within the
  period between the Subsequent Draft and the next NBA Draft with the
  Team that drafted him in the Subsequent Draft, that Team shall lose
  its exclusive right, which it obtained in the Subsequent Draft, to
  negotiate with the player, and the player will become a Rookie Free
  Agent as of the date of the next NBA Draft.
\item
  If a player is drafted in an Initial Draft and (i) receives a Required
  Tender, (ii) does not sign a Player Contract with a Team prior to the
  Subsequent Draft, and (iii) is not drafted by any Team in such
  Subsequent Draft, the player will become a Rookie Free Agent
  immediately upon the conclusion of the Subsequent Draft.
\item
  If a Second Round Pick receives and signs a Required Tender and is
  subsequently waived by the Team after signing such tender, then the
  Team that made the Required Tender to the player shall have exclusive
  rights to negotiate with and sign (or convert) the player to a Two-Way
  Contract for the Season covered by the Required Tender.
\item
  If a player is drafted by a Team in either an Initial or Subsequent
  Draft and that Team does not make a Required Tender to such player,
  the player will become a Rookie Free Agent on the July 16 following
  such Draft (for a First Round Pick) or on the September 6 following
  such Draft (for a Second Round Pick).
\item
  A Team may at any time withdraw a Required Tender it has made to a
  player, provided that the player agrees in writing to the withdrawal.
  In the event that a Required Tender is withdrawn, the player shall
  thereupon become a Rookie Free Agent.
\item
  A Team that holds the exclusive rights to negotiate with and sign a
  drafted player may at any time renounce such exclusive rights, except
  that, if the Team has made a Required Tender to the player, a
  renunciation shall not be permitted during the time the player has to
  accept the Required Tender under Article I, Section 1(yy). In order to
  renounce its exclusive rights with respect to a drafted player, a Team
  shall provide the NBA with an express, written statement renouncing
  such exclusive rights. The NBA shall provide a copy of such statement
  to the Players Association within three (3) business days following
  its receipt thereof.
\end{enumerate}

\section{Effect of Contracts with Other Professional
Teams.}\label{effect-of-contracts-with-other-professional-teams.}

If a player is drafted by a Team in either an Initial or Subsequent
Draft and, during a period in which he may negotiate and sign a Player
Contract with only the Team that drafted him, either (x) is a party to a
previously existing player contract with a professional basketball team
not in the NBA that covers all or any part of the NBA Season immediately
following said Initial or Subsequent Draft, or (y) signs such a player
contract (either (x) or (y), a ``Non-NBA Signing''), then the following
rules will apply:

\begin{enumerate}
\def\labelenumi{(\alph{enumi})}
\tightlist
\item
  Subject to Section 5(b) below, the Team that drafts the player shall
  retain the exclusive NBA rights to negotiate with and sign him for the
  period ending one (1) year from the earlier of the following two
  dates: (i) the date the player notifies such Team that he is available
  to sign a Player Contract with such Team immediately, provided that
  such notice will not be effective until the player is under no
  contractual or other legal impediment to sign and play with such Team
  for the then-current Season (if applicable) and any future Season; or
  (ii) the date of the NBA Draft occurring in the twelve-month period
  from September 1 to August 30 in which the player notifies such Team
  of his availability and intention to play in the NBA during the Season
  immediately following said twelve-month period, provided that such
  notice will not be effective until the player is under no contractual
  or other legal impediment to sign and play with such Team for the
  then-current Season (if applicable) and any future Season.
\item
  If, by July 1 of any year, the player notifies the Team that has
  drafted him that by September 1 of such year he will, immediately
  thereafter and for any future Season, be under no contractual or other
  legal impediment to sign and play with such Team, and provided that on
  such September 1 the player is in fact under no such contractual or
  other legal impediment, then, in order to retain the exclusive NBA
  rights to negotiate with and sign the player as provided in Section
  5(a), such Team must make a Required Tender to the player by September
  10 of such year.
\item
  If the player gives the required notice by July 1 of any year, and the
  Team that drafted him fails to make a Required Tender by September 10
  of such year, the player shall thereupon become a Rookie Free Agent.
\item
  If, during the one-year period of exclusive NBA negotiating rights set
  forth in Section 5(a) above, the player signs a player contract with a
  professional basketball team not in the NBA and the player has not
  made a bona fide effort to negotiate a Player Contract with the Team
  possessing his exclusive NBA rights or such bona fide effort is made
  and such Team makes a Required Tender to such player in accordance
  with Section 5(b) above, then such Team shall retain the exclusive NBA
  rights to negotiate with and sign the player for additional one-year
  periods as measured in and in accordance with the provisions of
  Section 5(a) above.
\item
  If, during the one-year period of exclusive NBA negotiating rights set
  forth in subsection (a) above, (i) the player signs a player contract
  with a professional basketball team not in the NBA, (ii) the player
  has made a bona fide effort to negotiate a Player Contract with the
  Team possessing his exclusive NBA rights, and (iii) such Team fails to
  make a Required Tender to such player in accordance with Section 5(b)
  above, then theplayer shall thereupon become a Rookie Free Agent.
\item
  If, during the one-year period of exclusive NBA negotiating rights set
  forth in Section 5(a) above, the Team makes or has made a Required
  Tender to the player and the player does not sign a player contract
  with any professional basketball team, then (i) in the case of a
  player who was previously drafted in an Initial Draft, the next NBA
  Draft following such one-year period shall be deemed the Subsequent
  Draft as to such player, and the rules applicable to a player who is
  subject to a Subsequent Draft will apply, or (ii) in the case of a
  player who was previously drafted in a Subsequent Draft, such player
  shall become a Rookie Free Agent at the end of such one-year period.
\item
  Notice under this Section 5 shall be provided in writing by personal
  delivery or pre-paid certified, registered, or overnight mail sent to
  the Team's principal address or principal office (as then listed in
  the NBA's records), to the attention of the Team's general manager and
  to the League Office (attention: General Counsel).
\end{enumerate}

\section{\texorpdfstring{Application to ``Early Entry''
Players.}{Application to Early Entry Players.}}\label{application-to-early-entry-players.}

If a player who is eligible for the Draft pursuant to Section
1(b)(ii)(F) or (b)(ii)(G)(3) above (an ``Early Entry'' player) is
selected in such Draft by a Team, the following rules apply:

\begin{enumerate}
\def\labelenumi{(\alph{enumi})}
\tightlist
\item
  Subject to Section 5 above, if the player does not thereafter play
  intercollegiate basketball, then the Team that drafted him shall,
  during the period from the date of such Draft to the date of the Draft
  in which the player would, absent his becoming an Early Entry player,
  first have been eligible to be selected, be the only Team with which
  the player may negotiate or sign a Player Contract, provided that such
  Team makes a Required Tender to the player each year by the date
  specified in Section 4(a) above. For purposes hereof, the Draft in
  which such player would, absent his becoming an Early Entry player,
  first have been eligible to be selected, will be deemed the
  ``Subsequent Draft'' as to that player, and the rules applicable to a
  player who has been drafted in a Subsequent Draft will apply. If the
  player, having been selected in a Draft for which he was eligible as
  an Early Entry player, has not signed a Player Contract with the Team
  that drafted him in such Draft following a Required Tender by that
  Team and is not drafted in the Subsequent Draft (as defined in the
  previous sentence), he shall become a Rookie Free Agent.
\item
  Subject to Section 5 above, if the player does thereafter play
  intercollegiate basketball, then the Team that drafted him shall
  retain the exclusive NBA rights to negotiate with and sign the player
  for the period ending one (1) year from the date of the Draft in which
  the player would, absent his becoming an Early Entry player, first
  have been eligible to be selected, provided that such Team makes a
  Required Tender to the player each year by the date specified in
  Section 4(a) above. For purposes hereof, the Draft in which such
  player would, absent his becoming an Early Entry player, first have
  been eligible to be selected, will be deemed the ``Initial Draft'' as
  to that player. The next NBA Draft shall be deemed the ``Subsequent
  Draft'' as to that player, and the rules applicable to a player who
  has been drafted in a Subsequent Draft will apply.
\item
  Notwithstanding anything to the contrary in this Section 6 or in
  Section 5 above, a Non-NBA Signing by an Early Entry player shall
  never shorten the period of time during which such player may
  negotiate and sign a Player Contract only with the Team that drafted
  him.
\end{enumerate}

\section{Assignment of Draft Rights.}\label{assignment-of-draft-rights.}

In the event that the exclusive right to negotiate with a player
obtained in any NBA Draft is assigned by a Team to another Team, in
accordance with NBA procedures, the Team to which such right has been
assigned shall have the same, but no greater, right to negotiate with
and sign such player as is possessed by the Team assigning such right,
and such player shall have the same, but no greater, obligation to the
Team to which such right has been assigned as he had to the Team
assigning such right.

\section{General.}\label{general.-2}

\begin{enumerate}
\def\labelenumi{(\alph{enumi})}
\tightlist
\item
  The placement of a Rookie on the Armed Services List, or on any of the
  other lists described in the NBA By-Laws, or on any other list created
  by the NBA, shall not extend the period of exclusive negotiating
  rights which a Team has to any Draft Rookie beyond the period
  specified in this Agreement.
\item
  Nothing contained herein shall prevent the NBA, in accordance with the
  applicable provisions of the NBA Constitution and By-Laws, from
  prohibiting or otherwise responding to violations by Teams of the
  exclusive NBA rights obtained in any NBA Draft, as set forth or
  referred to in this Article. Other than as specifically agreed to
  herein, nothing contained in this Agreement shall be deemed to be an
  agreement by the Players Association to any provision of the NBA
  Constitution and By-Laws.
\item
  An Early Entry player who is eligible to be selected in the next NBA
  Draft pursuant to Section 1(b)(ii)(F) or (b)(ii)(G)(3) above shall be
  entitled to withdraw from such Draft by providing written notice that
  is received by the NBA ten (10) days prior to such Draft. A player
  shall not be entitled to withdraw from more than two (2) NBA Drafts.
\item
  Any claim by a player that a Contract offered as a Required Tender
  pursuant to this Article X fails to meet one or more of the criteria
  for a Required Tender shall be made by written notice to the Team
  (with copies sent to the NBA and the Players Association), no later
  than ten (10) days after the receipt of such Contract by the Players
  Association. Such notice must set forth the specific changes that the
  player asserts must be made to the offered Contract in order for it to
  constitute a Required Tender. Upon receipt of such notice, if the
  requested changes are necessary to satisfy the requirements of a
  Required Tender, the Team may, within five (5) business days, offer
  the player an amended Contract incorporating the requested changes. If
  the Team offers such an amended Contract, the player and the Players
  Association shall be precluded from asserting that such Contract does
  not constitute a timely and valid Required Tender.
\item
  For purposes of this Article X, any rights afforded to ``a Team that
  drafts a player'' shall also be afforded to any Team to which such
  rights are subsequently assigned.
\end{enumerate}

\section{NBA Draft Combine.}\label{nba-draft-combine.}

The parties shall work together to: (i) identify ways to secure full
player participation in each year's Draft Combine for all invited
players; and (ii) in connection with this effort, improve the overall
player experience at the Combine. As part of this undertaking, the
Players Association will strongly encourage invited players to attend
each year's Draft Combine.

\chapter{FREE AGENCY}\label{free-agency}

\section{General Rules.}\label{general-rules.}

\begin{enumerate}
\def\labelenumi{(\alph{enumi})}
\item
  Subject to the provisions of Article VII, including, but not limited
  to, Article VII, Section 6(b), and subject further to Article II,
  Section 14:

  \begin{enumerate}
  \def\labelenumii{(\roman{enumii})}
  \tightlist
  \item
    an Unrestricted Free Agent is free at any time beginning on the
    first day of the Moratorium Period to negotiate, and free at any
    time after the conclusion of the Moratorium Period to enter into, a
    Player Contract with any Team; and
  \item
    a Restricted Free Agent is free at any time beginning on the first
    day of the Moratorium Period to negotiate a Player Contract with his
    Prior Team and to negotiate and enter into an Offer Sheet (as
    defined in Section 5(b) below) with any Team other than his Prior
    Team, and is free at any time after the conclusion of the Moratorium
    Period to enter into a Player Contract with his Prior Team.
  \end{enumerate}
\item
  No compensation obligation of any kind to another Team shall be
  applicable to any Free Agent. No right of first refusal (``Right of
  First Refusal'') of any kind shall be applicable to any Free Agent
  other than a Restricted Free Agent.
\item
  \begin{enumerate}
  \def\labelenumii{(\roman{enumii})}
  \tightlist
  \item
    For purposes of this Agreement, ``Qualifying Offer'' means an offer
    of a Uniform Player Contract, signed by the Team, that:

    \begin{enumerate}
    \def\labelenumiii{(\arabic{enumiii})}
    \tightlist
    \item
      is either personally delivered to the player or his representative
      or sent by e-mail or pre-paid certified, registered, or overnight
      mail to the last known address of the player or his representative
      (if sent by e-mail with a copy to the Players Association);
    \item
      is for a period of one (1) year;
    \item
      provides for Salary (excluding Incentive Compensation), Likely
      Bonuses, and Unlikely Bonuses in the amounts described in (ii),
      (iii), and (iv) below;
    \item
      provides for one hundred percent (100\%) of the Base Compensation
      to be protected for lack of skill and injury or illness (with no
      individually-negotiated conditions or limitations on such
      protection and no other types of protection); provided, however,
      that Qualifying Offers for players finishing Two-Way Contracts
      shall not be subject to this Section 1(c)(i)(4) and shall instead
      be subject to the rules set forth in Section 1(c)(iii) below; and
    \item
      provides for one hundred percent (100\%) of the Base Compensation
      to be payable in accordance with paragraph 3 of the Uniform Player
      Contract.
    \end{enumerate}
  \item
    For First Round Picks finishing their Rookie Scale Contracts, the
    Salary (excluding Incentive Compensation), Likely Bonuses, and
    Unlikely Bonuses contained in a Qualifying Offer shall be equal to
    the Salary (excluding Incentive Compensation), Likely Bonuses, and
    Unlikely Bonuses, respectively, provided in the fourth Salary Cap
    Year of the Rookie Scale Contract (``Fourth Year Salary'') increased
    by the percentage called for in the Rookie Salary Scale applicable
    to the First Round Pick's Rookie Scale Contract; provided that:

    \begin{enumerate}
    \def\labelenumiii{(\Alph{enumiii})}
    \tightlist
    \item
      For any First Round Pick finishing his Rookie Scale Contract who
      was not selected with one of the first nine (9) picks in the Draft
      and who, (1) during the third and fourth Seasons of his Rookie
      Scale Contract, either started an average of forty-one (41) or
      more Regular Season games per Season or averaged two thousand
      (2,000) or more minutes of playing time per Regular Season, or (2)
      in the fourth Season of his Rookie Scale Contract either started
      forty-one (41) or more Regular Season games or played two thousand
      (2,000) or more minutes (collectively, the ``Starter Criteria''),
      the Qualifying Offer shall instead contain Base Compensation (with
      no bonuses of any kind) equal to the amount of the Qualifying
      Offer applicable to the ninth player selected in the first round
      of the Draft (the ``ninth player'') as called for by the Rookie
      Salary Scale applicable to the First Round Pick's Rookie Scale
      Contract. For purposes of calculating such Qualifying Offer
      amount, the Fourth Year Salary of the ninth player shall be deemed
      to equal one hundred twenty percent (120\%) of the Rookie Scale
      Amount applicable to the ninth player.
    \item
      For any First Round Pick finishing his Rookie Scale Contract who
      was selected with one of the first through fourteenth picks in the
      Draft and who failed to meet the Starter Criteria, the player's
      Qualifying Offer shall contain the lesser of: (x) the Salary
      (excluding Incentive Compensation), Likely Bonuses, and Unlikely
      Bonuses, respectively, provided in the Fourth Year Salary
      increased by the percentage called for in the Rookie Salary Scale
      applicable to the First Round Pick's Rookie Scale Contract; or (y)
      Base Compensation (with no bonuses of any kind) equal to the
      amount of the Qualifying Offer applicable to the fifteenth player
      selected in the first round of the Draft (the ``fifteenth
      player'') as called for by the Rookie Salary Scale applicable to
      the First Round Pick's Rookie Scale Contract. For purposes of
      calculating such Qualifying Offer amount, the Fourth Year Salary
      of the fifteenth player shall be deemed to equal one hundred
      twenty percent (120\%) of the Rookie Scale Amount applicable to
      the fifteenth player.
    \end{enumerate}
  \item
    With respect to Qualifying Offers for players finishing Two-Way
    Contracts:

    \begin{enumerate}
    \def\labelenumiii{(\Alph{enumiii})}
    \tightlist
    \item
      For any player finishing a Two-Way Contract with a term of two (2)
      Seasons or finishing the second of two (2), or third of three (3),
      consecutive Two-Way Contracts with a term of one (1) Season with
      the same Team, the Qualifying Offer shall be an offer of a
      Standard NBA Contract and shall provide for (i) Base Compensation
      in an amount equal to the Minimum Annual Salary applicable to the
      player for the next Salary Cap Year (with no bonuses of any kind),
      and (ii) Base Compensation protection in an amount equal to the
      Two-Way Annual NBADL Salary for the Season covered by the
      Qualifying Offer, protected for lack of skill and injury or
      illness (with no individually-negotiated conditions or limitations
      on such protection and no other types of protection). With respect
      to any Two-Way Contract that is assigned, the assignor Team and
      the assignee Team shall be deemed to be the same Team for the
      purposes of this Section 1(c)(iii)(A).
    \item
      For all other players finishing Two-Way Contracts, the Qualifying
      Offer shall be an offer of a Two-Way Contract and shall provide
      for (i) the Two-Way Player Salary for the next Salary Cap Year,
      and (ii) \$50,000 of the Base Compensation protection for lack of
      skill and injury or illness (with no individually-negotiated
      conditions or limitations on such protection and no other types of
      protection) (``Two-Way Qualifying Offer'').
    \item
      Notwithstanding Sections (1)(c)(iii)(A) and (B) above, for any
      player finishing a Two-Way Contract who is not eligible to enter
      into another Two-Way Contract with the Team pursuant to Article
      II, Section 11(f), the Qualifying Offer (regardless of whether the
      prior Contract was a Two-Way Contract for a term of one (1) or two
      (2) Seasons) shall be an offer of a Standard NBA Contract and
      shall provide for (i) Base Compensation in an amount equal to the
      Minimum Annual Salary applicable to the player for the next Salary
      Cap Year (with no bonuses of any kind), and (ii) Base Compensation
      protection in an amount equal to the Two-Way Annual NBADL Salary
      for the next Salary Cap Year, protected for lack of skill and
      injury or illness (with no individually-negotiated conditions or
      limitations on such protection and no other types of protection).
    \end{enumerate}
  \item
    For all other players subject to a Right of First Refusal in
    accordance with this Article XI, the Salary (excluding Incentive
    Compensation), Likely Bonuses, and Unlikely Bonuses contained in a
    Qualifying Offer shall be one hundred twenty-five percent (125\%) of
    the player's Salary (excluding Incentive Compensation), Likely
    Bonuses, and Unlikely Bonuses, respectively, for the last Salary Cap
    Year covered by the player's prior Contract (the ``Prior Salary
    Qualifying Offer Amount''), provided that if on the July 1
    immediately following the date on which such a Qualifying Offer was
    made, the sum of the Minimum Annual Salary applicable to the player
    (for the Season covered by the Qualifying Offer) plus \$200,000 (the
    ``Minimum-Plus Qualifying Offer Amount'') is greater than the Prior
    Salary Qualifying Offer Amount, then such a Qualifying Offer shall
    be deemed amended to provide for Base Compensation equal to the
    Minimum-Plus Qualifying Offer Amount (with no bonuses of any kind);
    provided that, for any second round pick or undrafted player with
    two (2) or three (3) Years of Service who met the Starter Criteria
    in respect of the prior two (2) Seasons of his Contract(s) (i.e.~who
    either averaged the games started or minutes played amounts
    described in Section 1(c)(ii)(A)(1) above during his prior two (2)
    seasons, or achieved the games started or minutes played amounts
    described in Section 1(c)(ii)(A)(2) above in his prior Season only),
    the Qualifying Offer shall instead contain, if such amount exceeds
    the greater of the amounts described in (W) or (X) above, Base
    Compensation equal to the amount of the Qualifying Offer applicable
    to the twenty-first player selected in the first round of the Draft
    (the ``twenty-first player'') as called for by the Rookie Salary
    Scale applicable to Rookie Scale Contracts finishing in the same
    Season as the last Season of the player's Contract. For purposes of
    calculating such Qualifying Offer amount, the Fourth Year Salary of
    the twenty-first player shall be deemed to equal one hundred percent
    (100\%) of the Rookie Scale Amount applicable to the twenty-first
    player.
  \item
    All other terms and conditions in a Qualifying Offer must be
    unchanged from those that applied to the last year of the player's
    prior Contract to the extent that such terms and conditions are
    allowable amendments under this Agreement at the time the Qualifying
    Offer is made. In addition, a Team shall be permitted to include in
    any Qualifying Offer an Exhibit 6 to the Uniform Player Contract
    requiring that the player, if he signs the Qualifying Offer, pass a
    physical examination to be performed by a physician designated by
    the Team as a condition precedent to the validity of the Contract.
    For purposes of the foregoing, the Starter Criteria shall be
    determined based upon Official NBA statistics.
  \end{enumerate}
\item
  No Team or any of its employees or agents will make a public statement
  that the Team would match any future Offer Sheet for one of the Team's
  players or offer an impending or current Restricted Free Agent a
  particular Player Contract in free agency (e.g., a Contract providing
  for the player's maximum allowable Salary). The foregoing does not
  limit a Team's ability to express its desire to retain an impending or
  current Restricted Free Agent or to make general statements praising
  such a player (e.g., that the player is an important or essential part
  of the Team, that the Team wants or hopes to retain the player's
  services, and other similar statements).
\end{enumerate}

\section{No Individually-Negotiated Right of First
Refusal}\label{no-individually-negotiated-right-of-first-refusal}

\begin{enumerate}
\def\labelenumi{(\alph{enumi})}
\tightlist
\item
  No Player Contract may include any individually-negotiated Right of
  First Refusal or other limitation on player movement following the
  last Salary Cap Year covered by such Player Contract.
\item
  No Right of First Refusal rule, practice, policy, regulation, or
  agreement providing for a Right of First Refusal shall be applied to
  any player as a result of that player's entry into a player contract
  with (or for otherwise playing with) any team in any professional
  basketball league other than the NBA.
\end{enumerate}

\section{Withholding Services.}\label{withholding-services.}

A player who withholds playing services called for by a Player Contract
for more than thirty (30) days after the start of the last Season
covered by his Player Contract shall be deemed not to have
``complet{[}ed{]} his Player Contract by rendering the playing services
called for thereunder.'' Accordingly, such a player shall not be a
Veteran Free Agent and shall not be entitled to negotiate or sign a
Player Contract with any other professional basketball team unless and
until the Team for which the player last played expressly agrees
otherwise.

\section{Qualifying Offers to Make Certain Players Restricted Free
Agents.}\label{qualifying-offers-to-make-certain-players-restricted-free-agents.}

\begin{enumerate}
\def\labelenumi{(\alph{enumi})}
\item
  \begin{enumerate}
  \def\labelenumii{(\roman{enumii})}
  \tightlist
  \item
    From the day following the Season covered by the second Option Year
    of a First Round Pick's Rookie Scale Contract through the
    immediately following June 29, the player's Team may make a
    Qualifying Offer to the player. If such a Qualifying Offer is made,
    then, on the July 1 following such Season, the player shall become a
    Restricted Free Agent, subject to a Right of First Refusal in favor
    of the Team (``ROFR Team''), as set forth in Section 5 below. If
    such a Qualifying Offer is not made, then the player shall become an
    Unrestricted Free Agent on such July 1. If a Team does not timely
    exercise its Option with respect to the first Option Year or second
    Option Year of a player's Rookie Scale Contract in accordance with
    Article VIII, the player shall, following his second or third Season
    (as the case may be) become an Unrestricted Free Agent.
  \item
    A Team that makes a Qualifying Offer to a player following the
    second Option Year of his Rookie Scale Contract may elect
    simultaneously to offer the player an alternative Contract covering
    five (5) Seasons that provides Salary for the first Salary Cap Year
    equal to the Maximum Annual Salary under Article II, Section 7(a),
    with annual increases in Salary equal to eight percent (8.0\%) of
    the Salary for the first Salary Cap Year (a ``Maximum Qualifying
    Offer''). Providing a player with a Maximum Qualifying Offer shall
    have the consequence described in Section 5(b) below. A Maximum
    Qualifying Offer shall be subject to the following:

    \begin{enumerate}
    \def\labelenumiii{(\Alph{enumiii})}
    \tightlist
    \item
      A Maximum Qualifying Offer shall contain only Base Compensation
      and no bonuses of any kind.
    \item
      A Maximum Qualifying Offer shall state that the player's Base
      Compensation for the first Season shall equal ``the Maximum Annual
      Salary applicable to the player in the first Season of the
      Contract,'' and that the Base Compensation in each of the four (4)
      subsequent Seasons shall ``be increased by eight percent (8.0\%)
      of the Base Compensation for the first Season.'' Such a Contract,
      if timely accepted by the player in accordance with Section
      4(a)(ii)(D) below, shall be deemed amended to provide for specific
      Base Compensation for each Season covered by the Contract, based
      on the Maximum Annual Salary applicable to the player in the first
      Season.
    \item
      A Maximum Qualifying Offer cannot contain an Option or ETO, and
      must provide full Base Compensation protection in each Season for
      lack of skill and injury or illness (with no
      individually-negotiated conditions or limitations on such
      protection).
    \item
      The Team's offer of a Maximum Qualifying Offer must remain open
      for the same period that the player's Qualifying Offer remains
      open and cannot be withdrawn, except that if the Team withdraws
      its Qualifying Offer, the Maximum Qualifying Offer shall be deemed
      to be withdrawn simultaneously.
    \item
      A player may accept either his Qualifying Offer or his Maximum
      Qualifying Offer, but not both.
    \end{enumerate}
  \end{enumerate}
\item
  Any Veteran Free Agent (other than a First Round Pick whose first
  Option Year or second Option Year was not exercised) who (i) will have
  three (3) or fewer Years of Service as of the June 30 following the
  end of the last Season covered by his Player Contract, or (ii) is
  completing a Two-Way Contract and was on the NBA Team's Active or
  Inactive List for fifteen (15) or more days of the NBA Regular Season
  in the last Season of the Two-Way Contract, will be a Restricted Free
  Agent if his Prior Team makes a Qualifying Offer to the player at any
  time from the day following such Season through the immediately
  following June 29. If such a Qualifying Offer is made, then, on the
  July 1 following the last Season covered by the player's Player
  Contract, the player shall become a Restricted Free Agent, subject to
  a Right of First Refusal in favor of the ROFR Team, as set forth in
  Section 5 below. If such a Qualifying Offer is not made, then the
  player shall become an Unrestricted Free Agent on such July 1. For
  clarity, a player who is completing a Two-Way Contract but was not on
  the NBA Team's Active or Inactive List for fifteen (15) or more days
  of the NBA Regular Season in the last Season of the Two-Way Contract
  will become an Unrestricted Free Agent on the July 1 following the
  last Season covered by that Two-Way Contract.
\item
  \begin{enumerate}
  \def\labelenumii{(\roman{enumii})}
  \tightlist
  \item
    A player who receives a Qualifying Offer must be given until the
    October 1 following its issuance to accept it. Notwithstanding the
    preceding sentence, a Qualifying Offer may be withdrawn by the Team
    at any time through the July 13 following its issuance. If the
    Qualifying Offer is not withdrawn on or before July 13, it may be
    withdrawn thereafter but only if the player agrees in writing to the
    withdrawal. If a Qualifying Offer is withdrawn, the player shall
    immediately become an Unrestricted Free Agent. If a Qualifying Offer
    is withdrawn on or after July 14, the Team also shall be deemed to
    have renounced the player in accordance with Article VII, Section
    4(g). A player may not accept a Qualifying Offer after the October 1
    following the issuance thereof, unless the Team, prior to October 1,
    extends the date by which the player may accept the Qualifying
    Offer. In order to extend the date by which a player may accept his
    Qualifying Offer, a Team shall provide the player with written
    notice of the extension, which shall be either personally delivered
    to the player or his representative or sent by pre-paid certified,
    registered, or overnight mail to the last known address of the
    player or his representative. In no event may the acceptance date
    for a Qualifying Offer be extended beyond, or may a player accept a
    Qualifying Offer beyond, the March 1 following its issuance.
  \item
    If a Qualifying Offer is neither withdrawn nor accepted and the
    deadline for accepting it passes, the Team's Right of First Refusal
    shall continue, subject to Section 5(a) below.
  \item
    A player who knows that he has a physical disability that would
    render him physically unable to perform the playing services
    required under a Player Contract the following Season may not
    validly accept a Qualifying Offer received under this Section 4 or
    Section 5 below, unless the ROFR Team consents after disclosure of
    such physical disability. Notwithstanding the immediately preceding
    sentence, a player who knows that he has a physical disability that
    would render him physically unable to perform the playing services
    required under a Player Contract the following Season remains
    subject to the ROFR Team's Right of First Refusal.
  \end{enumerate}
\item
  Any claim that a Contract offered as a Qualifying Offer or a Maximum
  Qualifying Offer fails to meet one or more of the criteria for a
  Qualifying Offer or a Maximum Qualifying Offer shall be made by notice
  to the Team, in writing, no later than ten (10) days after a copy of
  the Qualifying Offer or Maximum Qualifying Offer was given by the Team
  or the NBA to the Players Association. Such notice must set forth the
  specific changes that allegedly must be made to the offered Contract
  in order for it to constitute a Qualifying Offer or a Maximum
  Qualifying Offer. Upon receipt of such notice, if the requested
  changes are necessary to satisfy the requirements of a Qualifying
  Offer or a Maximum Qualifying Offer, the Team may, within five (5)
  business days, offer the player an amended Contract incorporating the
  requested changes. If the Team offers such an amended Contract, the
  player and the Players Association shall be precluded from asserting
  that such Contract does not constitute a timely and valid Qualifying
  Offer or Maximum Qualifying Offer.
\end{enumerate}

\section{Restricted Free Agency.}\label{restricted-free-agency.}

\begin{enumerate}
\def\labelenumi{(\alph{enumi})}
\item
  If a Restricted Free Agent does not sign an Offer Sheet with any Team
  by March 1 of the Season for which the Qualifying Offer is made, and
  does not sign a Player Contract with the ROFR Team before that Season
  ends, then his ROFR Team may reassert its Right of First Refusal for
  the following Season by extending another Qualifying Offer (with the
  same terms, including the amount of Salary (excluding Incentive
  Compensation), Likely Bonuses, and Unlikely Bonuses, respectively,
  that were included in the prior Qualifying Offer) on or before the
  next June 29. A ROFR Team may continue to reassert its Right of First
  Refusal by following the foregoing procedure in each subsequent year
  in which that Restricted Free Agent does not sign an Offer Sheet with
  any Team by March 1 of the Season for which the Qualifying Offer is
  made, and does not sign a Player Contract with the ROFR Team before
  that Season ends. In each Season in which a Team reasserts its Right
  of First Refusal by extending another Qualifying Offer in accordance
  with this Section 5(a), the Team may also elect to simultaneously
  provide the player with a Maximum Qualifying Offer (with the same
  terms that were included in the prior Maximum Qualifying Offer). Any
  such Qualifying Offer and Maximum Qualifying Offer shall be governed
  by the provisions of Section 4 above.
\item
  When a Restricted Free Agent receives an offer to sign a Player
  Contract from a Team other than the ROFR Team (the ``New Team''),
  which he desires to accept, he shall give to the ROFR Team a completed
  certificate substantially in the form of Exhibit G annexed hereto (the
  ``Offer Sheet''), signed by the Restricted Free Agent and the New
  Team, which shall have attached to it a Uniform Player Contract
  separately specifying: (i) the ``Principal Terms'' (as defined in
  Section 5(e) below) of the New Team's offer; and (ii) any
  non-Principal Terms of the New Team's offer that the ROFR Team is not
  required to match (as specified in Section 5(e) below) but which would
  be included in the player's Player Contract with the New Team if the
  ROFR Team does not exercise its Right of First Refusal. The player's
  obligation in the foregoing sentence to give to the ROFR Team a
  completed Offer Sheet shall be deemed satisfied if the Offer Sheet is
  given to the ROFR Team by the New Team. The Offer Sheet must be for a
  Player Contract with a term of more than one (1) season (not including
  any Option Year), unless the ROFR team has tendered the player both a
  Qualifying Offer and a Maximum Qualifying Offer, in which case the
  Offer Sheet must be for a Player Contract with a term of more than two
  (2) Seasons (not including any Option Year). The Offer Sheet cannot be
  for a Two-Way Contract. In order to extend an Offer Sheet, the New
  Team must have Room for the player's Player Contract at the time the
  Offer Sheet is signed and must continue to have such Room at all times
  while the Offer Sheet is outstanding.
\item
  The ROFR Team, upon receipt of the Offer Sheet, may exercise its Right
  of First Refusal, which shall have the consequences hereinafter set
  forth below in this Section 5. The ROFR Team may match an Offer Sheet
  by using, as applicable, Room, a Veteran Free Agent Exception set
  forth in Article VII, Section 6(b) or the Minimum Player Salary
  Exception. In order to match an Offer Sheet, the ROFR Team must have,
  as applicable, Room, a Veteran Free Agent Exception or Minimum Player
  Salary Exception in an amount equal or greater to the Salary plus any
  Unlikely Bonuses provided for in the first Salary Cap Year of the
  player's Contract at the time notice of the Team's exercise of its
  Right of First Refusal is given and must continue to have such Room or
  the applicable Exception at all times the First Refusal Exercise
  Notice remains in effect.
\item
  The following rules shall govern the signing of an Offer Sheet by a
  Restricted Free Agent who has one (1) or two (2) Years of Service:

  \begin{enumerate}
  \def\labelenumii{(\roman{enumii})}
  \tightlist
  \item
    Notwithstanding any other provision of this Agreement, no such Offer
    Sheet may provide for Salary plus Unlikely Bonuses in the first
    Salary Cap Year totaling more than the amount of the Non-Taxpayer
    Mid-Level Salary Exception for such Salary Cap Year. Annual
    increases or decreases in Salary and Unlikely Bonuses shall be
    governed by Article VII, Section 5(c)(1).
  \item
    If an Offer Sheet provides for the maximum allowable amount of
    Salary for the first two (2) Salary Cap Years pursuant to Section
    5(d)(i) above, then, subject to Section 5(d)(iii) below, the Offer
    Sheet may provide for Salary for the third Salary Cap Year of up to
    the maximum amount that the player would have been eligible to
    receive for the third Salary Cap Year, absent the restriction in the
    first sentence of Section 5(d)(i) above and had the player's Salary
    for the first two (2) Salary Cap Years been the maximum amount
    permitted under Article II, Section 7(a) and Article VII, Section
    5(d)(1). If the Offer Sheet provides for Salary for the third Salary
    Cap Year in accordance with the foregoing sentence, then, subject to
    Section 5(d)(iii) below, (A) the player's Salary for the fourth
    Salary Cap Year may increase or decrease in relation to the third
    Salary Cap Year's Salary by no more than four and five tenths
    percent (4.5\%) of the Salary for the third Salary Cap Year, (B) the
    Offer Sheet cannot contain bonuses of any kind, and (C) the Offer
    Sheet must provide for one hundred percent (100\%) of the Base
    Compensation in each Season to be protected for lack of skill and
    injury or illness (with no individually-negotiated conditions or
    limitations on such protection).
  \item
    If a Team extends an Offer Sheet in accordance with Section 5(d)(ii)
    above, then, for purposes of determining whether the Team has Room
    for the Offer Sheet, the Salary for the first Salary Cap Year
    covered by the Offer Sheet shall be deemed to equal the average of
    the aggregate Salaries for such Salary Cap Year and each subsequent
    Salary Cap Year covered by the Offer Sheet. If the ROFR Team does
    not exercise its Right of First Refusal, the player's Salary for
    each Salary Cap Year covered by the Contract with the Team that
    extended the Offer Sheet shall be deemed to equal the average of the
    aggregate Salaries for each such Salary Cap Year. If the ROFR Team
    exercises its Right of First Refusal, the player's Salary for each
    Salary Cap Year covered by the Contract with the ROFR Team shall be
    the Salary for such Salary Cap Year as set forth in the Contract.
    Notwithstanding the preceding sentence, if the sum of: (i) the ROFR
    Team's Team Salary at the time it exercises its Right of First
    Refusal; and (ii) the average of the aggregate Salaries for each
    Salary Cap Year of the Offer Sheet is less than or equal to the
    Salary Cap for the then-current Salary Cap Year, then the ROFR Team
    may, in connection with exercising its Right of First Refusal, elect
    to have the player's Salary for each Salary Cap Year covered by the
    Contract equal the average of such aggregate Salaries for each such
    Salary Cap Year. If the ROFR Team wishes to make such an election,
    it must do so by providing the NBA with a written statement on the
    same day that it gives the First Refusal Exercise Notice to the
    Restricted Free Agent and New Team pursuant to Section 5(t) below,
    and the NBA shall provide a copy of this notice to the Players
    Association within one (1) business day following its receipt
    thereof.
  \end{enumerate}
\item
  The Principal Terms of an Offer Sheet are only:

  \begin{enumerate}
  \def\labelenumii{(\roman{enumii})}
  \tightlist
  \item
    the term of the Contract;
  \item
    the fixed and specified Compensation that the New Team will pay or
    lend to the Restricted Free Agent as a signing bonus, Current Base
    Compensation, and/or Deferred Base Compensation in specified
    installments on specified dates;
  \item
    Incentive Compensation; provided, however, that the only elements of
    such Incentive Compensation that shall be included in the Principal
    Terms are the following: (A) bonuses that qualify as Likely Bonuses
    based upon the performance of the Team extending the Offer Sheet and
    the ROFR Team; and (B) Generally Recognized League Honors; and
  \item
    Any allowable amendments to the terms contained in the Uniform
    Player Contract (e.g., Base Compensation protection, a trade bonus,
    etc.).
  \end{enumerate}
\item
  In the event that an Offer Sheet includes an Exhibit 6 requiring that
  the player pass a physical examination to be performed by a physician
  designated by the New Team, the Exhibit 6 language must be replaced
  with the following: ``This Offer Sheet will be deemed invalid and of
  no force and effect (except as described in Article XI, Section 5(l)
  of the CBA) unless the player passes, in the sole discretion of the
  Team (exercised in good faith), a physical examination in accordance
  with Article II, Section 13(h) of the CBA that is (i) conducted within
  two (2) days of the execution of this Offer Sheet, and (ii) the
  results of which are reported by the Team to the player within three
  (3) days of the execution of this Offer Sheet. The player agrees to
  supply complete and truthful information in connection with any such
  examinations.'' The New Team must notify the player and the ROFR Team
  within the three (3)-day period set forth in the Exhibit 6 in the
  Offer Sheet whether the player has passed the physical. In the event
  that the New Team fails to timely provide such notice, the player
  shall be deemed to have passed the physical with the New Team.
\item
  If, within two (2) days from the date it receives an Offer Sheet, the
  ROFR Team gives to the Restricted Free Agent a ``First Refusal
  Exercise Notice'' substantially in the form of Exhibit H annexed
  hereto, then, subject to Section 5(k) below, such Restricted Free
  Agent and the ROFR Team shall be deemed to have entered into a Player
  Contract containing all the Principal Terms (but not any terms other
  than the Principal Terms) included in the Uniform Player Contract
  attached to the Offer Sheet (except that if the Contract contains an
  Exhibit 6, such Exhibit 6 shall be deemed deleted). Such Contract may
  not thereafter be amended in any manner for a period of one (1) year.
\item
  If the ROFR Team does not give the First Refusal Exercise Notice
  within the aforementioned two (2)-day period, or if during such two
  (2)-day period the ROFR Team provides written notice to the player
  that the Team declines to exercise its Right of First Refusal, then
  the player and the New Team shall be deemed to have entered into a
  Player Contract containing all of the terms and conditions included in
  the Uniform Player Contract attached to the Offer Sheet (including, if
  the Offer Sheet contains an Exhibit 6, that the player pass a physical
  examination to be conducted by the Team as a condition precedent to
  the validity of the Contract). Such Contract may not thereafter be
  amended in any manner for a period of one (1) year.
\item
  Notwithstanding anything contained herein to the contrary, for any
  Offer Sheet received by the ROFR Team during the Moratorium Period,
  the aforementioned two (2)-day period shall commence immediately upon
  conclusion of the Moratorium Period. For clarity, if the ROFR Team
  receives an Offer Sheet at any time during a Moratorium Period, the
  ROFR Team shall have until 11:59 p.m. (ET) on July 8 to give the First
  Refusal Exercise Notice.
\item
  After exercising its Right of First Refusal as described in this
  Section 5, the ROFR Team may not trade the Restricted Free Agent for
  one (1) year, without the player's consent. Even with the player's
  consent, for one (1) year, neither the ROFR Team exercising its Right
  of First Refusal nor any other Team may trade the player to the Team
  whose Offer Sheet was matched.
\item
  \begin{enumerate}
  \def\labelenumii{(\roman{enumii})}
  \tightlist
  \item
    Any Team may condition its First Refusal Exercise Notice on the
    player reporting for and passing, in the sole discretion of the Team
    (exercised in good faith), a physical examination to be conducted by
    a physician designated by the Team within two (2) days from its
    exercise of the Right of First Refusal. In connection with the
    physical examination, the player must supply all information
    reasonably requested of him, provide complete and truthful answers
    to all questions posed to him, and submit to all examinations and
    tests requested of him.
  \item
    In the event the player does not submit to the requested physical
    examination within two (2) days of the exercise of the Right of
    First Refusal then, until such time as the player submits to the
    requested physical examination and is notified of the results, the
    ROFR Team's conditional First Refusal Exercise Notice shall remain
    in effect, except that the ROFR Team may elect at any time to
    withdraw its First Refusal Exercise Notice, which shall have the
    effect of invalidating the Offer Sheet and causing the Team that
    issued the Offer Sheet to be prohibited from signing or acquiring
    the player for a period of one (1) year from the date the First
    Refusal Exercise Notice was withdrawn. If the player does not submit
    to the requested physical examination on or before March 1, the
    Offer Sheet shall be deemed invalid and the Team that issued the
    Offer Sheet shall be prohibited from signing or acquiring the player
    for a period of one (1) year from such March 1.
  \item
    In the event the player does not pass the requested physical
    examination, then in its sole discretion, the ROFR Team may withdraw
    its First Refusal Exercise Notice within two (2) days following the
    date upon which such physical examination is conducted. If the First
    Refusal Exercise Notice is withdrawn, the player and the New Team
    shall be deemed to have entered into a Player Contract in accordance
    with the provisions of Section 5(g) above. If the First Refusal
    Exercise Notice is not withdrawn, then the ROFR Team shall be deemed
    to have waived its right to have the player pass a physical
    examination and will be deemed to have entered into a Player
    Contract in accordance with the provisions of Section 5(g) above.
  \end{enumerate}
\item
  In the event that (i) the Offer Sheet includes an Exhibit 6 and the
  New Team determines that the player has not passed the physical, and
  (ii) either (A) the ROFR Team declines to exercise its Right of First
  Refusal, (B) the period for the ROFR Team to exercise its Right of
  First Refusal expires, or (C) the ROFR Team exercises its Right of
  First Refusal conditioned on the player reporting for and passing a
  physical and timely determines that the player has not passed his
  physical and withdraws its First Refusal Exercise Notice pursuant to
  Section 5(k) above, then the ROFR Team must within two (2) days from
  the later of the date (x) that the ROFR Team receives notice from the
  New Team that the player has not passed the physical examination
  administered by the New Team, or (y) on which the ROFR Team timely
  notifies the player that he has not passed the physical, pursuant to
  Section 5(k) above:

  \begin{enumerate}
  \def\labelenumii{(\roman{enumii})}
  \tightlist
  \item
    Elect to continue to possess such rights with respect to the player
    as the ROFR Team possessed at the time of the execution of the Offer
    Sheet, provided that the ROFR Team can only make this election if
    the ROFR Team has not engaged in any transaction since the Offer
    Sheet was given that the ROFR Team would not have been able to
    engage in if the player's Free Agent Amount (or the amount of a
    Qualifying Offer or Maximum Qualifying Offer made to the player, if
    applicable) at the time the Offer Sheet was given had remained
    included in the ROFR Team's Team Salary; or
  \item
    Decline to continue to possess such rights with respect to the
    player as the ROFR Team possessed at the time of the execution of
    the Offer Sheet, in which case any Qualifying Offer given to the
    player by the Team shall be deemed withdrawn pursuant to Section
    4(c)(i) above, and the Team's Right of First Refusal shall be deemed
    relinquished pursuant to Section 5(o) below. If at the time the New
    Team notifies the ROFR Team that the player has not passed the
    physical administered by the New Team, the ROFR Team has not yet
    exercised its Right of First Refusal or has not yet provided written
    notice to the player that the ROFR Team declines to exercise its
    Right of First Refusal, nothing in this Section 5(l) shall prohibit
    a ROFR Team from: (i) exercising its Right of First Refusal; or (ii)
    making one of the elections set forth in Section 5(l)(i) or 5(l)(ii)
    above.
  \end{enumerate}
\item
  A Team shall not be permitted to exercise its Right of First Refusal
  pursuant to an agreement to trade the Player Contract to another Team
  pursuant to Article VII, Section 8(e).
\item
  There may be only one (1) Offer Sheet signed by a Restricted Free
  Agent outstanding at any one time, provided that the Offer Sheet has
  also been signed by a Team. An Offer Sheet, both before and after it
  is given to the ROFR Team, may be revoked or withdrawn only upon the
  written consent of the ROFR Team, the New Team, and the Restricted
  Free Agent. In such event, a Restricted Free Agent shall again be free
  to negotiate and sign an Offer Sheet with any Team, and any Team shall
  again be free to negotiate and sign an Offer Sheet with such
  Restricted Free Agent, subject only to the ROFR Team's renewed Right
  of First Refusal.
\item
  A Team that holds the Right of First Refusal with respect to a
  Restricted Free Agent may relinquish such Right of First Refusal at
  any time except during the period that the player has been given to
  accept a Qualifying Offer. If a Team relinquishes its Right of First
  Refusal with respect to a Restricted Free Agent, the player shall
  immediately become an Unrestricted Free Agent and the Team shall be
  deemed to have renounced the player in accordance with Article VII,
  Section 4(g) hereof. In order to relinquish its Right of First Refusal
  with respect to a Restricted Free Agent, a Team shall provide the NBA
  with a written statement relinquishing such Right of First Refusal.
  The NBA shall provide a copy of such statement to the Players
  Association by email within two (2) business days following its
  receipt thereof.
\item
  An expedited arbitration before the System Arbitrator, whose decision
  shall be final and binding upon all parties, shall be the exclusive
  method for resolving any disputes concerning this Section 5. If a
  dispute arises between the player and either the ROFR Team or the New
  Team, as the case may be, relating to the contents of an Offer Sheet,
  and/or whether the binding agreement is between the Restricted Free
  Agent and the New Team or the Restricted Free Agent and the ROFR Team,
  such dispute shall immediately be submitted to the System Arbitrator,
  who shall resolve such dispute within five (5) days.
\item
  A Restricted Free Agent may not give an Offer Sheet to the ROFR Team
  at any time after the March 1 of the Season for which he has been made
  a Qualifying Offer.
\item
  On the same day as the giving of an Offer Sheet to the ROFR Team, the
  ROFR Team shall cause a copy thereof to be given to the NBA, which
  shall cause a copy thereof to be promptly given to the Players
  Association. On the same day as the giving of a First Refusal Exercise
  Notice to the Restricted Free Agent, the ROFR Team shall cause a copy
  thereof to be given to the New Team, which shall cause a copy thereof
  to be promptly given to the NBA, which shall cause a copy thereof to
  bepromptly given to the Players Association.
\item
  There may be no consideration of any kind given by one Team to another
  Team in exchange for a Team's decision to exercise or not to exercise
  its Right of First Refusal, or in exchange for a Team's decision to
  submit or not to submit an Offer Sheet to a Restricted Free Agent.
\item
  Any Offer Sheet, First Refusal Exercise Notice, or other writing
  required or permitted to be given under this Article XI, shall be
  either by personal delivery, email, or by pre-paid certified,
  registered, or overnight mail addressed as follows:\\
  To any NBA Team: addressed to that Team at the principal address of
  such Team as then listed on the records of the NBA or at the Team's
  principal office, to the attention of the Team's general manager (and
  if by email, then to the general manager's email address with the Team
  and any such other email address as the Team may designate in
  writing);\\
  To the NBA: National Basketball Association, Olympic Tower, 645 Fifth
  Avenue, New York, NY 10022, Attn: General Counsel (and if by email,
  then to the General Counsel's email address with the NBA and any such
  other email address as the NBA may designate in writing);\\
  To the Players Association: National Basketball Players Association,
  1133 Avenue of the Americas, 5th Floor, New York, NY 10036, Attn:
  General Counsel (and if by email, then to the General Counsel's email
  address with the NBPA and any such other email address as the NBPA may
  designate in writing);\\
  To a Restricted Free Agent: (i) for Qualifying Offers and other
  writings relating to Qualifying Offers (e.g., withdrawal of a
  Qualifying Offer), to the last known email address or address of the
  player or his representative; and (ii) for Offer Sheets and other
  writings relating to Offer Sheets (e.g., First Refusal Exercise
  Notice), to his address listed on the Offer Sheet, and, if the
  Restricted Free Agent designates a representative on the Offer Sheet
  and lists such representative's address thereof, then such
  representative's address (and if by email to the player or his
  representative, then with a copy to the General Counsel's email
  address with the NBPA and any such other email address as the NBPA may
  designate in writing).
\item
  Notwithstanding anything contained herein to the contrary:

  \begin{enumerate}
  \def\labelenumii{(\roman{enumii})}
  \tightlist
  \item
    In addition to personal delivery or delivery by pre-paid certified,
    registered, or overnight mail, any Offer Sheet, notice revoking or
    withdrawing an Offer Sheet, First Refusal Exercise Notice, notice
    declining to exercise a Right of First Refusal, notice relinquishing
    a Right of First Refusal, or notice withdrawing a First Refusal
    Exercise Notice (collectively ``Offer Sheet-Related Notices'') may
    be given by e-mail as follows:

    \begin{enumerate}
    \def\labelenumiii{(\arabic{enumiii})}
    \tightlist
    \item
      To any Team: to the attention of each of the Team's specified
      representatives' email address (as set forth in subsection (3)
      below).
    \item
      To the NBA: to the attention of the email address used for that
      purpose under the 2011 CBA or such other email address as agreed
      to by the parties.
    \item
      To the Players Association: to the attention of the email address
      used for that purpose under the 2011 CBA or such other email
      address as agreed by the parties.
    \item
      To a Restricted Free Agent: to his email address listed on the
      Offer Sheet, and, if the Restricted Free Agent designates a
      representative on the Offer Sheet and lists such representative's
      email address thereon, a copy shall be sent to such representative
      at such email address.
    \end{enumerate}
  \item
    Any Offer Sheet-Related Notice given by email must be sent to the
    NBA, the Players Association, the applicable Restricted Free Agent
    (including such Restricted Free Agent's representative if required
    pursuant to Section 5(t) above), the ROFR Team, and the New Team. If
    an Offer Sheet fails to list a player's email address, delivery of
    any Offer Sheet-Related Notice to the player shall be deemed
    satisfied by email delivery to the Players Association.
  \item
    By the June 1 prior to each Salary Cap Year, each Team shall provide
    to the NBA the names and email addresses of three (3)
    representatives designated by the Team who shall be, for such Salary
    Cap Year, (i) the only representatives of the Team permitted to give
    any Offer Sheet-Related Notice on behalf of the Team via the email
    notification procedures set forth herein, and (ii) the required
    recipients of any Offer Sheet-Related Notice sent to the Team via
    the email notification procedures set forth herein. In each Salary
    Cap Year, the NBA shall provide to the Players Association (and all
    Teams) the list of Team representatives (and such representatives'
    email addresses) by June 15.
  \end{enumerate}
\end{enumerate}

Any Offer Sheet, First Refusal Exercise Notice, or other writing
required or permitted to be given under this Article XI that is sent by
email shall be deemed given when sent. For delivery by any other means
allowed by this Article XI, the following shall apply: (i) an Offer
Sheet shall be deemed given only when received by the ROFR Team; (ii) a
First Refusal Exercise Notice shall be deemed given when sent by the
ROFR Team; (iii) a Qualifying Offer, a Maximum Qualifying Offer, an
amended Qualifying Offer (i.e., pursuant to Section 4(d) above), and a
notice of extension of the date by which a Qualifying Offer can be
accepted shall be deemed given when sent by the ROFR Team; and (iv)
other writings required or permitted to be given under this Article XI
(e.g., notice relinquishing a Right of First Refusal, an acceptance of a
Qualifying Offer, a withdrawal of a Qualifying Offer, notice that a
Qualifying Offer fails to meet one or more of the criteria for a
Qualifying Offer, etc.) shall be deemed given only when received by the
party to whom it is addressed.

\chapter{OPTION CLAUSES}\label{option-clauses}

\section{Team Options.}\label{team-options.}

Except as provided by Article VIII, Section 1, a Player Contract shall
not contain any option in favor of the Team, except an Option (as
defined in Article I, Section 1(ss)) that: (i) is specifically
negotiated between a Veteran or a Rookie (other than a First Round Pick)
and a Team; (ii) authorizes the extension of such Contract for no more
than one (1) year beyond the stated term; (iii) is exercisable only
once; and (iv) provides that the Salary (excluding Incentive
Compensation), Likely Bonuses, and Unlikely Bonuses payable with respect
to the Option Year are no less than one hundred percent (100\%) of the
Salary (excluding Incentive Compensation), Likely Bonuses, and Unlikely
Bonuses, respectively, payable with respect to the last year of the
stated term of such Contract and that all other terms and conditions
(other than with respect to the payment schedule for the player's Base
Compensation) in the Option Year shall be unchanged from those that
applied to the last year of the stated term of such Contract (including,
but not limited to, the percentage of Base Compensation that is
protected).

\section{Player Options.}\label{player-options.}

A Player Contract shall not contain any option in favor of the player,
except:

\begin{enumerate}
\def\labelenumi{(\alph{enumi})}
\tightlist
\item
  an Option that: (i) is specifically negotiated between a Veteran or a
  Rookie (other than a First Round Pick) and a Team; (ii) authorizes the
  extension of such Contract for no more than one (1) year beyond the
  stated term; (iii) is exercisable only once; and (iv) provides that
  the Salary (excluding Incentive Compensation), Likely Bonuses, and
  Unlikely Bonuses payable with respect to the Option Year are no less
  than one hundred percent (100\%) of the Salary (excluding Incentive
  Compensation), Likely Bonuses, and Unlikely Bonuses, respectively,
  payable with respect to the last year of the stated term of such
  Contract and that all other terms and conditions (other than with
  respect to the payment schedule for the player's Base Compensation) in
  the Option Year shall be unchanged from those that applied to the last
  year of the stated term of such Contract (including, but not limited
  to, the percentage of Base Compensation that is protected). If a
  Player Contract contains an Option in favor of the player and
  provides, in whole or in part, for Base Compensation protection in the
  Option Year, such Contract must also contain, in Exhibit 2 of the
  Contract under the heading ``Additional Conditions or Limitations,''
  either the language set forth in subsection (A) below or the language
  set forth in subsection (B) below, but not both, and such language
  shall define the respective rights and obligations of the player and
  Team with respect to the subject matter thereof:

  \begin{enumerate}
  \def\labelenumii{(\Alph{enumii})}
  \tightlist
  \item
    ``If this Contract is terminated by Team prior to Player's exercise
    of the Option described in Exhibit 1 of the Contract, then Player
    shall be entitled to benefit from the Base Compensation protection
    provisions of this Exhibit 2 to the same extent as if the exercise
    of the Option by Player had occurred prior to Team's termination of
    the Contract.''
  \item
    ``If this Contract is terminated by Team prior to Player's exercise
    of the Option described in Exhibit 1 of the Contract, then Team
    shall be relieved of any obligation to pay Player any Base
    Compensation with respect to the Option Year.''\\
    No Player Contract that contains the language set forth in
    subsection (B) above may provide for the Option in favor of the
    player to be exercisable earlier than the day following the date of
    the Team's last game of the Season prior to the Option Year; and/or
  \end{enumerate}
\item
  an Early Termination Option (or ``ETO'') (as defined in Article I,
  Section 1(v)), provided that such ETO is exercisable only once and
  takes effect no earlier than the end of the fourth Season of the
  Contract. A Contract that does not provide for an ETO when signed may
  not be amended to provide for an ETO during the original term of the
  Contract. If a Team and a player enter into an Extension (other than
  an Extension of a Rookie Scale Contract), the Contract may not be
  amended to provide for an ETO and any previously-existing ETO must be
  eliminated. If a Team and player enter into an Extension of a Rookie
  Scale Contract, the Contract may simultaneously be amended to provide
  for an ETO, provided that such ETO is exercisable only once and takes
  effect no earlier than the end of the fourth Season of the extended
  term of the Contract.
\end{enumerate}

\section{No Conditional Options.}\label{no-conditional-options.}

If a Contract contains any Option or ETO, the right of the Team or
player to exercise such Option or ETO must be fixed at the time the
Contract (or Extension) is entered into and may not be contingent upon
the satisfaction of any individually-negotiated condition. In the case
of an ETO, the Effective Season of such ETO also must be fixed at the
time the Contract (or Extension) is entered into and may not be
contingent upon the satisfaction of any individually-negotiated
condition.

\section{Exercise Period.}\label{exercise-period.}

Any ETO must be exercised prior to the June 30 immediately prior to the
Effective Season of such ETO. Any Option must be exercised prior to the
June 30 immediately prior to the Season covered by the Option, except
that an Option in favor of a player who would become a Restricted Free
Agent if the Option were not exercised must be exercised prior to the
June 25 immediately prior to the Season covered by such Option.

\section{Option Exercise Notices.}\label{option-exercise-notices.}

The NBA shall provide the Players Association with copies of any Option
or ETO exercise or non-exercise notice received by the NBA within two
(2) business days of the NBA's receipt of such notice from the Team.

\chapter{CIRCUMVENTION}\label{circumvention}

\section{General Prohibitions.}\label{general-prohibitions.}

\begin{enumerate}
\def\labelenumi{(\alph{enumi})}
\tightlist
\item
  It is the intention of the parties that the provisions agreed to
  herein, including, without limitation, those relating to the Salary
  Cap, the Exceptions to the Salary Cap, the scope of Basketball Related
  Income, the Escrow and Tax Arrangement, the Rookie Scale, the Right of
  First Refusal, the Maximum Player Salary, and free agency, be
  interpreted so as to preserve the essential benefits achieved by both
  parties to this Agreement. Neither the Players Association, the NBA,
  nor any Team (or Team Affiliate) or player (or person or entity acting
  with authority on behalf of such player), shall enter into any
  agreement, including, without limitation, any Player Contract
  (including any Renegotiation, Extension, or amendment of a Player
  Contract), or undertake any action or transaction, including, without
  limitation, the assignment or termination of a Player Contract, which
  is, or which includes any term that is, designed to serve the purpose
  of defeating or circumventing the intention of the parties as
  reflected by all of the provisions of this Agreement.
\item
  It shall constitute a violation of Section 1(a) above for a Team (or
  Team Affiliate) to enter into an agreement or understanding with any
  sponsor or business partner or third-party under which such sponsor,
  business partner or third-party pays or agrees to pay compensation for
  basketball services (even if such compensation is ostensibly
  designated as being for non-basketball services) to a player under
  Contract to the Team. Such an agreement with a sponsor or business
  partner or third-party may be inferred where: (i) such compensation
  from the sponsor or business partner or third-party is substantially
  in excess of the fair market value of any services to be rendered by
  the player for such sponsor or business partner or third-party; and
  (ii) the Compensation in the Player Contract between the player and
  the Team is substantially below the fair market value of such
  Contract.
\item
  It shall constitute a violation of Section 1(a) above for a Team (or
  Team Affiliate) to have a financial arrangement with or offer a
  financial inducement to any player (not including retired players) not
  signed to a current Player Contract, except as permitted by this
  Agreement.
\item
  Nothing contained in Section 1(c) above shall interfere with a Team's
  obligation to pay a player Deferred Compensation earned under a prior
  Player Contract.
\end{enumerate}

\section{No Unauthorized Agreements.}\label{no-unauthorized-agreements.}

\begin{enumerate}
\def\labelenumi{(\alph{enumi})}
\tightlist
\item
  At no time shall there be any agreements or transactions of any kind
  (whether disclosed or undisclosed to the NBA), express or implied,oral
  or written, or promises, undertakings, representations, commitments,
  inducements, assurances of intent, or understandings of any kind
  (whether disclosed or undisclosed to the NBA), between a player (or
  any person or entity controlled by, related to, or acting with
  authority on behalf of, such player) and any Team (or Team Affiliate):

  \begin{enumerate}
  \def\labelenumii{(\roman{enumii})}
  \tightlist
  \item
    concerning any future Renegotiation, Extension, or other amendment
    of an existing Player Contract, or entry into a new Player Contract;
    or
  \item
    except as permitted by this Agreement or as set forth in a Uniform
    Player Contract (provided that the Team has not intentionally
    delayed submitting such Uniform Player Contract for approval by the
    NBA), involving compensation or consideration of any kind or
    anything else of value, to be paid, furnished or made available by,
    to, or for the benefit of the player, or any person or entity
    controlled by, related to, or acting with authority on behalf of the
    player; or
  \item
    except as permitted by this Agreement, involving an investment or
    business opportunity to be furnished or made available by, to, or
    for the benefit of the player, or any person or entity controlled
    by, related to, or acting with authority on behalf of the player).
  \end{enumerate}
\item
  In addition to the foregoing, it shall be a violation of this Section
  2 for any Team (or Team Affiliate) or any player (or any person or
  entity controlled by, related to, or acting with authority on behalf
  of, such player) to attempt to enter into or to intentionally solicit
  any agreement, transaction, promise, undertaking, representation,
  commitment, inducement, assurance of intent or understanding that
  would be prohibited by Section 2(a) above.
\item
  A violation of Section 2(a) or 2(b) above may be proven by direct or
  circumstantial evidence, including, but not limited to, evidence that
  a Player Contract or any term or provision thereof cannot rationally
  be explained in the absence of conduct violative of Section 2(a) or
  2(b).
\item
  In any proceeding brought before the System Arbitrator pursuant to
  this Section 2, no adverse inference shall be drawn against the party
  initiating such proceeding because that party, when it first suspected
  or believed that a violation of Section 2 may have occurred, deferred
  the initiation of such proceeding until it had further reason to
  believe that such a violation had occurred.
\item
  A player will not be found to have committed a violation of Section
  2(a)(ii) above if the violation is the Team's intentional delay in
  submitting a Uniform Player Contract to the NBA and this was done
  without the player's knowledge.
\end{enumerate}

\section{Penalties.}\label{penalties.}

\begin{enumerate}
\def\labelenumi{(\alph{enumi})}
\tightlist
\item
  Upon a finding of a violation of Section 1 above by the System
  Arbitrator, but only following the conclusion of any appeal to the
  Appeals Panel, the Commissioner shall be authorized to:

  \begin{enumerate}
  \def\labelenumii{(\roman{enumii})}
  \tightlist
  \item
    impose a fine of up to \$3,000,000 (50\% of which shall be payable
    to the NBA, and 50\% of which shall be payable to the NBPA-Selected
    Charitable Organization (as defined in Article VI, Section 6(a))) on
    any Team found to have committed such violation for the first time;
  \item
    impose a fine of up to \$4,500,000 (50\% of which shall be payable
    to the NBA, and 50\% of which shall be payable to the NBPA-Selected
    Charitable Organization) on any Team found to have committed such
    violation for at least the second time;
  \item
    direct the forfeiture of one First Round Draft Pick;
  \item
    void any Player Contract, or any Renegotiation, Extension, or
    amendment of a Player Contract, between any player and any Team when
    both the player (or any person or entity acting with authority on
    behalf of such player) and the Team (or Team Affiliate) are found to
    have committed such violation; and/or
  \item
    void any other transaction or agreement found to have violated
    Section 1 above.
  \end{enumerate}
\item
  Upon a finding of a violation of Section 2 above by the System
  Arbitrator, but only following the conclusion of any appeal to the
  Appeals Panel, the Commissioner shall be authorized to:

  \begin{enumerate}
  \def\labelenumii{(\roman{enumii})}
  \tightlist
  \item
    impose a fine of up to \$6,000,000 on any Team found to have
    committed such violation (50\% of which shall be payable to the NBA,
    and 50\% of which shall be payable to the NBPA-Selected Charitable
    Organization);
  \item
    direct the forfeiture of draft picks;
  \item
    when both the player (or any person or entity acting with authority
    on behalf of such player) and the Team (or Team Affiliate) are found
    to have committed such violation, (A) void any Player Contract, or
    any Renegotiation, Extension, or amendment of a Player Contract,
    between such player and such Team, (B) impose a fine of up to
    \$250,000, on any player (50\% of which shall be payable to the NBA,
    and 50\% of which shall be payable to the NBPA-Selected Charitable
    Organization), and/or (C) prohibit any future Player Contract, or
    any Renegotiation, Extension, or amendment of a Player Contract,
    between such player and such Team;
  \item
    suspend for up to one (1) year any Team personnel found to have
    willfully engaged in such violation; and/or
  \item
    void any transaction or agreement found to have violated Section 2
    above and direct the disgorgement by the player of anything of value
    received in connection with such transaction or agreement (except
    Compensation received for services already performed pursuant to a
    Player Contract), unless the player establishes by a preponderance
    of the evidence that he was unaware of the violation.
  \end{enumerate}
\item
  In any proceeding before the System Arbitrator in which it is alleged
  that a player agent or other person or entity acting with authority on
  behalf of a player has violated Section 2 above, the System Arbitrator
  shall make a specific determination with respect to such allegation.
  If the System Arbitrator finds such violation and such finding, if
  appealed, is affirmed by the Appeals Panel, the System Arbitrator
  shall refer such finding to the Players Association, which shall
  accept as binding and conclusive the finding(s) of the System
  Arbitrator (or, in the case of an appeal, the Appeals Panel) that a
  violation of Section 2(a) or 2(b) has occurred and shall consider such
  finding(s) as establishing a violation of the Players Association's
  regulations applicable to such person or entity. The Players
  Association represents that it will impose such discipline as is
  appropriate under the circumstances on the person or entity found to
  have violated Section 2 above, and that it will promptly notify the
  NBA of the discipline imposed; provided, however, that in no event
  shall the penalty imposed upon a player agent found to have violated
  Section 2 above be less than a one-year suspension of that player
  agent's certification by the Players Association.
\item
  In addition to the authority conferred on the Commissioner pursuant to
  Sections 3(a) and 3(b) above, the Commissioner shall be authorized to
  impose a fine of up to \$250,000 on any Team found by the Commissioner
  to have violated Section 2 above. Any fine imposed pursuant to this
  Section 3(d) shall not require as a predicate any finding of, or
  proceeding before, the System Arbitrator. In the event the
  Commissioner imposes such a fine, the Players Association has the
  right to de novo review of the Commissioner's finding that a Section 2
  violation occurred under the System Arbitration provisions of Article
  XXXII. With respect to any fine imposed under this Section 3(d), 50\%
  shall be payable to the NBA and 50\% shall be payable to the
  NBPA-Selected Charitable Organization (as defined in Article VI,
  Section 6(a)).
\end{enumerate}

\section{Production of Tax
Materials.}\label{production-of-tax-materials.}

In any proceeding to enforce Section 1 or 2 above, the System Arbitrator
shall have the authority, upon good cause shown, to direct any Team,
Team Affiliate, or player to produce any tax returns or other relevant
tax materials disclosing income figures for the player (non-income
figures may be redacted), or disclosing expense figures by the Team or
Team Affiliate (non-expense figures may be redacted), which materials
shall not be released to the general public or the media and shall be
treated as strictly confidential by all parties.

\section{Transactions with Retired
Players.}\label{transactions-with-retired-players.}

\begin{enumerate}
\def\labelenumi{(\alph{enumi})}
\item
  If (i) a Team or Team Affiliate enters into a transaction after the
  date of this Agreement with a retired player who played for the Team
  within the five-year (5) period preceding such transaction and the
  terms of such transaction provide for the retired player to be paid
  compensation or consideration in excess of \$10,000 or to be provided
  with an investment or business opportunity, and, if (ii) the
  compensation the retired player received from the Team when he was a
  player was substantially below the then fair market value of such
  player's basketball services under the Salary Cap system, then the NBA
  may challenge the transaction, pursuant to the procedures set forth in
  Section 5(b) below, on the ground that: (A) the compensation or
  consideration to the retired player substantially exceeds the then
  fair market value of the services or other consideration provided by
  the retired player in the transaction; or that (B) the amount of the
  retired player's investment or the benefit conferred upon the retired
  player as a result of the investment or business opportunity is not
  commercially reasonable, given the relative risks and rewards of such
  investment.
\item
  \begin{enumerate}
  \def\labelenumii{(\roman{enumii})}
  \tightlist
  \item
    Any challenge under this Section 5 shall be filed in writing with a
    business valuation expert jointly selected by the NBA and the
    Players Association. In the event the parties cannot agree on the
    identity of a business valuation expert, a business valuation expert
    shall be selected in the same manner set forth in Article XXXI,
    Section 6 for the selection of a Grievance Arbitrator in the absence
    of an agreement between the parties. The business valuation expert
    shall conduct a hearing in which the retired player, the Team and/or
    Team Affiliate, the Players Association, and the NBA are afforded
    the opportunity to appear and participate. The NBA shall have the
    burden of proof in the proceeding. The business valuation expert may
    permit discovery of relevant documents necessary to undertake the
    valuation, and shall render a decision within fifteen (15) days
    following the conclusion of the hearing. Within ten (10) days of any
    decision by the business valuation expert, any of the parties may
    file an appeal with the System Arbitrator, who shall conduct a
    hearing and render a decision within twenty (20) days of the filing
    of the appeal. In any such proceeding, the System Arbitrator shall
    apply an ``arbitrary and capricious'' standard of review. There
    shall be no right of further appeal to the Appeals Panel.
  \item
    If the NBA prevails in its challenge under this Section 5, the
    difference between (A) the compensation or consideration received by
    the retired player, or the value of the investment or business
    opportunity received by the retired player (net of any contribution
    by the retired player), and (B) a reasonable estimate of the fair
    market value of the services or other consideration provided by the
    retired player, or a reasonable estimate of the fair market value of
    the investment or business opportunity, in each case as determined
    by the business valuation expert or the System Arbitrator, as the
    case may be, shall be included in the Team's Team Salary, subject to
    the Team's Room and other Salary Cap rules, and further subject to
    any allocation over time that the business valuation expert or
    System Arbitrator determines is appropriate. In the event that any
    amount required to be included in the Team's Team Salary pursuant to
    this subsection exceeds the Team's Room, the challenged transaction
    or arrangement shall be rescinded and of no further force and
    effect.
  \item
    If the NBA prevails in its challenge under this Section 5, and the
    retired player and the Team and/or Team Affiliate renegotiate or
    terminate the transaction, any revised terms of the transaction
    shall be promptly disclosed to the NBA and the Players Association,
    and may, at the request of the NBA, be re-subjected to the
    procedures of this Section 5(b).
  \end{enumerate}
\item
  Any information disclosed to the NBA and the Players Association
  pursuant to the procedures of this Section 5 shall be treated strictly
  confidential, and shall not be released to the general public or the
  media.
\end{enumerate}

\chapter{ANTI-COLLUSION PROVISIONS}\label{anti-collusion-provisions}

\section{No Collusion.}\label{no-collusion.}

Subject to Section 2 below, no NBA Team, its employees or agents, will
enter into any contracts, combinations or conspiracies, express or
implied, with the NBA or any other NBA Team, their employees or agents:
(a) to negotiate or not to negotiate with any Veteran or Rookie; (b) to
submit or not to submit an Offer Sheet to any Restricted Free Agent; (c)
to offer or not to offer a Player Contract to any Free Agent; (d) to
exercise or not to exercise a Right of First Refusal; or (e) concerning
the terms or conditions of employment offered to any Veteran or Rookie.

\section{Non-Collusive Conduct.}\label{non-collusive-conduct.}

The following is a non-exhaustive list of conduct that shall not be
deemed a violation of Section 1 above:

\begin{enumerate}
\def\labelenumi{(\alph{enumi})}
\tightlist
\item
  the formulation and negotiation of collective bargaining proposals;
\item
  agreements between NBA Teams necessary to the assignment of a Player
  Contract of a Veteran or the assignment of the exclusive negotiating
  rights to a Draft Rookie, where such assignment is contingent upon (i)
  the signing by the Veteran of an amendment to an existing Player
  Contract (including, for example, an Extension), or (ii) the signing
  by the Draft Rookie of a new Player Contract; provided, however, that
  if such contingency is fulfilled by the Veteran entering into an
  amended Player Contract (including, for example, an Extension) or the
  Draft Rookie entering into a new Player Contract, this subsection
  shall only apply if the assignment is actually consummated;
\item
  an agreement between NBA Teams concerning the signing of a new Player
  Contract by a Veteran Free Agent with his Prior Team, where such
  agreement is necessary for the subsequent assignment of the new Player
  Contract between the agreeing Teams; provided, however, that this
  Section 2(c) shall apply only if the subsequent assignment is
  consummated, and only if the agreement and the new Player Contract
  comply with the provisions of Article VII, Section 8(e);
\item
  conduct authorized by the terms and conditions of the NBA Draft (as
  set forth in Article X above);
\item
  conduct authorized by any provision of this Agreement or conduct by
  the NBA League Office, undertaken in good faith, that reflects a
  reasonable interpretation of this Agreement or a Player Contract;
\item
  any action taken by the NBA League Office to exclude from the NBA,
  suspend or discipline any player for any reason authorized or
  permitted by any provision of this Agreement (this Section 2(f),
  however, shall not affect any other rights of any player or the
  Players Association to contest such action); or
\item
  any disapproval by the NBA Commissioner of a Player Contract,
  Extension, Renegotiation or other amendment.
\end{enumerate}

\section{Individual Negotiations.}\label{individual-negotiations.}

No NBA Team shall fail or refuse to negotiate with, or enter into a
Player Contract with, any player who is free to negotiate and sign a
Player Contract with any NBA Team, on any of the following grounds:

\begin{enumerate}
\def\labelenumi{(\alph{enumi})}
\tightlist
\item
  that the player has previously been subject to the exclusive
  negotiating rights obtained by another NBA Team in an NBA Draft; or
\item
  that the player has previously refused or failed to enter into a
  Player Contract containing an Option; or
\item
  that the player has become a Restricted Free Agent or an Unrestricted
  Free Agent; or
\item
  that the player is or has been subject to a Right of First Refusal.
\end{enumerate}

The fact that a Team has not negotiated with, made any offers to, or
entered into any Player Contracts with players who are free to negotiate
and sign Player Contracts with any Team, shall not, by itself, be deemed
proof that such Team failed or refused to negotiate with, make any
offers to, or enter into any Player Contracts with any players on any of
the prohibited grounds referred to in this Section 3.

\section{League Disclosures.}\label{league-disclosures.}

The NBA League Office shall not knowingly communicate or disclose,
directly or indirectly, to any NBA Team that another NBA Team has
negotiated with or is negotiating with any Restricted Free Agent, unless
and until an Offer Sheet (as defined in Article XI, Section 5(b)) shall
have been given to the ROFR Team (as defined in Article XI, Section
4(a)), or any Free Agent prior to the execution of a Player Contract
with that player.

\section{Enforcement of Anti-Collusion
Provisions.}\label{enforcement-of-anti-collusion-provisions.}

\begin{enumerate}
\def\labelenumi{(\alph{enumi})}
\tightlist
\item
  Any player, or the Players Association acting on behalf of a player or
  players, may bring an action before the System Arbitrator alleging a
  violation of Section 1 above. Issues of relief and liability shall be
  determined in the same proceeding (including the amount of damages,
  pursuant to Section 9 below, if any). The complaining party will bear
  the burden of demonstrating by a clear preponderance of the evidence
  that the challenged conduct was in violation of Section 1 above and
  caused economic injury to such player(s); provided, however, that the
  Players Association may, in the absence of economic injury to any
  player, bring an action before the System Arbitrator claiming a
  violation of Section 1 above (which must be proved by a clear
  preponderance of the evidence) and seeking only declaratory relief or
  a direction to cease and desist from the challenged conduct.
\item
  The provisions of this Agreement are not intended to create any
  substantive rights in any party, other than as provided for herein.
  This Agreement may be enforced, and any alleged violations may be
  remedied, only as provided for herein.
\end{enumerate}

\section{Satisfaction of Burden of
Proof.}\label{satisfaction-of-burden-of-proof.}

The failure by a Team or Teams to submit Offer Sheets to Restricted Free
Agents, or to make offers or sign Contracts for the playing services of
Free Agents shall not, by itself or in combination only with evidence
about the playing skills of the player(s) not receiving such offers or
contracts, satisfy the burden of proof set forth in Section 5 above.
However, such evidence may support a finding of a violation of Section 1
above, but only in combination with other evidence that either by itself
or in combination with the evidence referred to in the immediately
preceding sentence indicates that the challenged conduct was in
violation of Section 1 above and, except in cases where the Players
Association seeks only declaratory relief or a direction to cease and
desist from the challenged conduct, caused economic injury to such
player(s).

\section{Summary Judgment.}\label{summary-judgment.}

The System Arbitrator may, at any time following the conclusion of any
permitted discovery, determine whether or not the complainant's evidence
is sufficient to raise a genuine issue of material fact capable of
satisfying the standards imposed by Sections 5 and 6 above. If the
System Arbitrator determines that complainant's evidence is not so
sufficient, he shall dismiss the action. In considering the sufficiency
of the complainant's evidence, the System Arbitrator may consider
documentary evidence and affidavits submitted by the parties.

\section{Remedies for Economic
Injury.}\label{remedies-for-economic-injury.}

In the event that an individual player or players, or the Players
Association acting on his or their behalf, successfully proves a
violation of Section 1 above that has caused economic injury, the player
or players determined by the System Arbitrator to have suffered economic
injury as a result of the violation will have the right:

\begin{enumerate}
\def\labelenumi{(\alph{enumi})}
\tightlist
\item
  to terminate his (or their) existing Player Contract(s) at his (or
  their) option (however, such termination shall not take effect until
  the conclusion of a then-ongoing NBA Season, if any). Such right of
  termination shall not arise until the recommendation of the System
  Arbitrator finding a violation is no longer subject to further appeal
  and must be exercised by the player within thirty (30) days therefrom.
  If, at the time the Player Contract is terminated, such player would
  have been an Unrestricted Free Agent pursuant to the provisions of
  this Agreement, he shall immediately become an Unrestricted Free
  Agent. If, at the time the Player Contract is terminated, such player
  would have been a Restricted Free Agent pursuant to the provisions of
  this Agreement, such player shall immediately become a Restricted Free
  Agent upon such termination; however, any such player may choose to
  reinstate his Player Contract at any time up until September 15 of
  that year; and
\item
  to recover damages as described in Section 9 below. However, if the
  player terminates his Player Contract under Section 8(a) above and
  does not reinstate it pursuant thereto, he may not recover damages for
  the period after such termination takes effect. A player who does not
  terminate his contract, or who reinstates it pursuant to Section 8(a)
  above, may recover damages for the entire period of his injury.
\end{enumerate}

\section{Calculation of Damages.}\label{calculation-of-damages.}

Upon any finding of a violation of Section 1 above that has caused
economic injury, compensatory damages (i.e., the amount by which any
player has been injured as a result of such violation) and
non-compensatory damages (i.e., the amount exceeding compensatory
damages) shall be awarded as follows:

\begin{enumerate}
\def\labelenumi{(\alph{enumi})}
\tightlist
\item
  Two (2) times the amount of compensatory damages, in the event that
  all of the Teams found to have violated Section 1 above have committed
  such a violation for the first time. Any Team found to have committed
  such a violation for the first time shall be jointly and severally
  liable for two (2) times the amount of compensatory damages.
\item
  Three (3) times the amount of compensatory damages, in the event that
  any of the Teams found to have violated Section 1 above have committed
  such a violation for the second time during the term of this
  Agreement. In the event that damages are awarded pursuant to this
  Section 9(b): (i) any Team found to have committed such a violation
  for the first time shall be jointly and severally liable for two (2)
  times the amount of compensatory damages; and (ii) any Team found to
  have committed such a violation for the second time during the term of
  this Agreement shall be jointly and severally liable for three (3)
  times the amount of compensatory damages.
\item
  Three (3) times the amount of compensatory damages, plus, for each
  Team found to have violated Section 1 above for at least the third
  time during the term of this Agreement, four million dollars
  (\$4,000,000), in the event that any of the Teams found to have
  violated Section 1 above have committed such violation for at least
  the third time during the term of this Agreement. In the event that
  damages are awarded pursuant to this Section 9(c): (i) any Team found
  to have committed such a violation for the first time shall be jointly
  and severally liable for two (2) times the amount of compensatory
  damages; (ii) any Team found to have committed such a violation for at
  least the second time during the term of this Agreement shall be
  jointly and severally liable for three (3) times the amount of
  compensatory damages; and (iii) any Team found to have committed such
  a violation for at least the third time during the term of this
  Agreement shall, in addition, pay a fine of four million dollars
  (\$4,000,000).
\end{enumerate}

\section{Payment of Damages.}\label{payment-of-damages.}

In the event damages are awarded pursuant to Section 9 above, the amount
of compensatory damages shall be paid to the injured player or players.
The amount of non-compensatory damages, including any fines, shall be
paid to the Players Association, which may use it for any purpose other
than to pay it to any player who has received compensatory damages,
except that any such player may receive some portion of a
non-compensatory damage award as part of a proportional distribution to
Players Association members.

\section{Effect of Damages on Salary
Cap.}\label{effect-of-damages-on-salary-cap.}

In the event damages are awarded pursuant to Section 9 above, the amount
of non-compensatory damages, including any fines, will not be included
in any of the computations described in Article VII above. The amount of
compensatory damages awarded will be included in such computations.

\section{Contribution.}\label{contribution.}

Any Team found liable under Section 1 above shall have the right to seek
contribution from any other Team found liable for the same violation in
a proceeding before the Commissioner who shall determine what
contribution, if any, is fair and equitable. The Commissioner's
determination with regard to contribution shall be final and binding
upon and unappealable by any Team. A contribution determination by the
Commissioner may be appealed by the Players Association to the System
Arbitrator, except that if such a determination involves fewer than four
(4) Teams found to have committed a violation of Section 1 above and
allocates damages equally among the Teams found liable, there shall be
no appeal to the System Arbitrator. In the event of a contribution
determination by the Commissioner, the NBA shall provide the Players
Association with the data and information that the Commissioner used or
relied upon in making his determination. Any contribution determination
appealed by the Players Association to the System Arbitrator shall be
upheld unless it is clearly erroneous.

\section{No Reimbursement.}\label{no-reimbursement.}

Any damages awarded pursuant to Section 9 above must be paid by the
individual Teams found liable and those Teams may not be reimbursed or
indemnified by any other Team or the NBA, except to the extent of any
award of contribution made pursuant to Section 12 above.

\section{Costs.}\label{costs.}

In any action brought for an alleged violation of Section 1 above, the
System Arbitrator shall order the payment of reasonable attorneys' fees
by any party found to have brought such an action or to have asserted a
defense to such an action without any reasonable basis for asserting
such a claim or defense.

\section{Termination of Agreement.}\label{termination-of-agreement.}

The Players Association shall have the right to terminate this Agreement
(pursuant to the procedure set forth in Article XXXIX, Section 3 of this
Agreement), under the following circumstances:

\begin{enumerate}
\def\labelenumi{(\alph{enumi})}
\tightlist
\item
  Where there has been a finding or findings of one (1) or more
  instances of a violation of Section 1 above with respect to any one
  NBA Season during the term of this Agreement which, either
  individually or in total, involved five (5) or more Teams and caused
  injury to five (5) or more players; or
\item
  Where there has been a finding or findings of one (1) or more
  instances of a violation of Section 1 above with respect to any two
  (2) consecutive NBA Seasons during the term of this Agreement which,
  either individually or in total, involved seven (7) or more Teams and
  caused economic injury to seven (7) or more players. For purposes of
  this Section 15(b), a player found to have been injured by a violation
  of Section 1 above in each of two (2) consecutive Seasons shall be
  counted as an additional player injured by such a violation for each
  such NBA Season; or
\item
  Where, in a proceeding brought by the Players Association, it is shown
  by clear and convincing evidence that during the term of this
  Agreement ten (10) or more Teams have engaged in a violation or
  violations of Section 1 above, causing economic injury to one or more
  NBA players. In order to terminate this Agreement pursuant to this
  Section 15(c) and Article XXXIX, Section 3 of this Agreement:

  \begin{enumerate}
  \def\labelenumii{(\roman{enumii})}
  \tightlist
  \item
    the proceeding must be brought by the Players Association; and
  \item
    the NBA and the System Arbitrator must be informed at the outset of
    any such proceeding that the Players Association is proceeding under
    this Section 15(c) for the purpose of establishing its entitlement
    to terminate this Agreement.
  \end{enumerate}
\end{enumerate}

\section{Discovery.}\label{discovery.}

\begin{enumerate}
\def\labelenumi{(\alph{enumi})}
\tightlist
\item
  In any of the actions described in this Article XIV, the System
  Arbitrator shall grant reasonable and expedited discovery upon the
  application of any party where, and to the extent, he or she
  determines it is reasonable to do so. Such discovery may include the
  production of documents and the taking of depositions.
\item
  Notwithstanding Section 16(a) above, the Players Association and the
  NBA shall each have the right to obtain discovery upon request in any
  three (3) proceedings brought under this Article XIV during the term
  of this Agreement. The scope and extent of such discovery shall be
  determined by the System Arbitrator.
\end{enumerate}

\section{Time Limits.}\label{time-limits.}

Any action under Section 1 above must be brought within ninety (90) days
of the time when the player knows or reasonably should have known that
he had a claim, or within ninety (90) days of the start of the NBA
Season in which a violation of Section 1 above is claimed, whichever is
later. In the absence of a System Arbitrator, the complaining party
shall file such claim for breach of this Agreement pursuant to Section
301 of the Labor Management Relations Act in either the U.S. District
Court for the Southern District of New York or the U.S. District Court
for the District of New Jersey. Any party alleged to have violated
Section 1 shall have the right, prior to any proceedings on the merits,
to make an initial motion to dismiss any complaint that does not comply
with the timeliness requirement of this Section 17.

\chapter{CERTIFICATIONS}\label{certifications}

\section{Contract Certification.}\label{contract-certification.}

\begin{enumerate}
\def\labelenumi{(\alph{enumi})}
\tightlist
\item
  Every Player Contract (other than a 10-Day Contract), or any
  Renegotiation, Extension or other amendment of a Player Contract,
  entered into during the term of this Agreement shall be accompanied by
  a certification, sworn to separately by (i) the person who executed
  the Player Contract on behalf of the Team, (ii) the player, and (iii)
  any player agent who negotiated the Contract on behalf of the player,
  under penalties of perjury, that the Player Contract, Renegotiation,
  Extension, or other amendment sets forth all components of a player's
  Compensation from the Team or any Team Affiliate, and that there are
  no agreements or transactions of any kind (whether disclosed or
  undisclosed to the NBA), express or implied, oral or written, or
  promises, undertakings, representations, commitments, inducements,
  assurances of intent, or understandings of any kind (whether disclosed
  or undisclosed to the NBA):

  \begin{enumerate}
  \def\labelenumii{(\roman{enumii})}
  \tightlist
  \item
    concerning any future Renegotiation, Extension, or other amendment
    of an existing Player Contract, or entry into a new Player Contract;
    or
  \item
    except as permitted by this Agreement or contained in such Uniform
    Player Contract, involving compensation or consideration of any kind
    or anything else of value to be paid, furnished or made available
    by, to, or for the benefit of the player, or any person or entity
    controlled by, related to, or acting with authority on behalf of the
    player; or
  \item
    except as permitted by this Agreement, involving an investment or
    business opportunity to be furnished or made available by, to, or
    for the benefit of the player, or any person or entity controlled
    by, related to, or acting with authority on behalf of the player.
  \end{enumerate}
\item
  Prior to the assignment of any Player Contract of a player who is in
  the last Salary Cap Year of the Contract (or the last Salary Cap Year
  before the player or the Team has the right to terminate the
  Contract), the player, the player's agent, and the Team to which such
  Contract is to be assigned shall each submit to the NBA a
  certification, sworn to under penalties of perjury, that other than
  the Player Contract that has been assigned, or as permitted by this
  Agreement, there are no agreements or transactions of any kind
  (whether disclosed or undisclosed to the NBA), express or implied,
  oral or written, or promises, undertakings, representations,
  commitments, inducements, assurances of intent, or understandings of
  any kind (whether disclosed or undisclosed to the NBA), between the
  player (or the player's agent or any person or entity controlled by or
  related to the player) and the Team to which the Player Contract is to
  be assigned or a Team Affiliate of the Team to which the Player
  Contract is to be assigned concerning (i) any future Renegotiation,
  Extension, or other amendment of the Player Contract that has been
  assigned, (ii) any future Player Contract, or (iii) an investment or
  business opportunity or compensation or consideration of any kind or
  anything else of value to be paid, furnished, or made available by,
  to, or for the benefit of the player or any person or entity
  controlled by, related to, or acting with authority on behalf of the
  player.
\item
  If a player, within two (2) years after the assignment of such
  player's Player Contract, enters into a new Player Contract, or any
  Renegotiation, Extension, or other amendment of the Player Contract
  that had been assigned, the Team, the player, and the player's agent
  shall each submit to the NBA a certification, sworn to under penalties
  of perjury, that, at the time of the assignment, other than the Player
  Contract that has been assigned, or as permitted by this Agreement,
  there were no agreements or transactions of any kind (whether
  disclosed or undisclosed to the NBA), express or implied, oral or
  written, or promises, undertakings, representations, commitments,
  inducements, assurances of intent, or understandings of any kind,
  (whether disclosed or undisclosed to the NBA), between the player (or
  the player's agent or any person or entity controlled by or related to
  the player) and the Team to which the Player Contract has been
  assigned or a Team Affiliate of the Team to which the Player Contract
  has been assigned concerning (i) any future Renegotiation, Extension,
  or other amendment of the Player Contract that has been assigned, (ii)
  any future Player Contract, or (iii) an investment or business
  opportunity or compensation or consideration of any kind or anything
  else of value to be paid, furnished, or made available by, to, or for
  the benefit of the player or any person or entity controlled by,
  related to, or acting with authority on behalf of the player. Such
  certification shall be submitted to the NBA no later than sixty (60)
  days following the execution of such new Player Contract, or any
  Renegotiation, Extension, or other amendment of the Player Contract.
\item
  If an agent, player, or Team fails or refuses to provide a
  certification called for under this Article XV, the NBA shall have the
  option, in its sole discretion, to approve or disapprove the
  transaction in question. In the case of a failure or refusal by an
  agent, and whether the transaction in question is approved or
  disapproved, the Players Association shall take appropriate
  disciplinary action against the agent.
\end{enumerate}

\section{End of Season
Certification.}\label{end-of-season-certification.}

\begin{enumerate}
\def\labelenumi{(\alph{enumi})}
\tightlist
\item
  At the conclusion of each NBA Season, a Governor (or Alternate
  Governor) and the executive primarily responsible for basketball
  operations on behalf of the Team shall each submit to the NBA a
  certification, sworn to under penalties of perjury, that the Team has
  not, to the extent of their knowledge after reasonable inquiry, (i)
  violated the terms of Article XIV, Section 1, (ii) violated the terms
  of Article XIII, Section 2, nor (iii) received from the NBA League
  Office any communication disclosing that an NBA Team has negotiated
  with any Free Agent prior to the execution of a Player Contract with
  that player. Upon receipt of each such certification, the NBA shall
  forward a copy of the certification to the Players Association.
\item
  A violation of this Section 2 may be deemed evidence of a violation of
  Article XIV, Section 1 or Article XIII, Section 2.
\end{enumerate}

\section{False Certification.}\label{false-certification.}

Any criminal complaint of perjury filed by the NBA or any Team based
upon a certification required pursuant to Section 1 above shall be
against the player, the player's agent, and the Team official making
such certification.

\chapter{MUTUAL RESERVATION OF
RIGHTS}\label{mutual-reservation-of-rights}

Upon the expiration or termination of this Agreement, no person shall be
deemed to have waived, by reason of the entry into or effectuation of
this Agreement, any other collective bargaining agreement, or any Player
Contract, or any of the terms of any of them, or by reason of any
practice or course of dealing, their respective rights under law with
respect to any issue or their ability to advance any legal argument.

\chapter{PROCEDURE WITH RESPECT TO PLAYING CONDITIONS AT VARIOUS
FACILITIES}\label{procedure-with-respect-to-playing-conditions-at-various-facilities}

\chaptermark{PROCEDURE WITH RESPECT TO PLAYING CONDITIONS AT VARIOUS \ldots}

When a new franchise is granted, or when an existing franchise moves to
another city or a new or different arena, the Players Association shall,
upon request and within a reasonable period of time, have the right to
inspect the facility to be used by such franchise. Similarly, the
Players Association shall, upon reasonable notice to the Team(s)
involved and the NBA, have the right to inspect the training camp and
practice facilities used by such Team(s). If, following such inspection,
the Players Association is of the opinion that the playing conditions at
such facility will endanger the health and safety of NBA players, it
shall promptly notify the Commissioner and the Team involved in writing.
Promptly following the receipt of such notice, representatives of the
Players Association and of the Team(s) involved, and the Commissioner or
his designee shall meet in an effort to resolve the matter. It is agreed
that the failure of the parties to resolve the matter shall not impair
the legally binding effect of this Agreement or create any right, during
the term of this Agreement, to (a) unilaterally implement any provision
concerning such unresolved matter, (b) lockout, or (c) strike. If no
resolution satisfactory to the Players Association, the Team(s) involved
and the Commissioner is reached, the issue of whether the playing
conditions at the facility in question will endanger the health and
safety of NBA players will, without interruption of the schedule or
training camp or practice activities, immediately be submitted to and
determined by the Grievance Arbitrator in accordance with the provisions
of Article XXXI; provided, however, that the Grievance Arbitrator need
not render an award within twenty-four (24) hours of the conclusion of
the hearing, but shall issue his award as expeditiously as possible
under the circumstances.

\chapter{TRAVEL ACCOMMODATIONS, LOCKER ROOM FACILITIES AND
PARKING}\label{travel-accommodations-locker-room-facilities-and-parking}

\section{Hotel Arrangements.}\label{hotel-arrangements.}

\begin{enumerate}
\def\labelenumi{(\alph{enumi})}
\tightlist
\item
  Each Team agrees to use its best efforts to make the following
  arrangements for its players while they are ``on the road'':

  \begin{enumerate}
  \def\labelenumii{(\roman{enumii})}
  \tightlist
  \item
    to have their baggage picked up by porters;
  \item
    to have them stay in first class hotels; and
  \item
    to have extra-long beds available to them in each hotel.\\
    If there is a finding that a Team has committed a willful violation
    of this Section 1(a), the NBA shall impose a \$5,000 fine on such
    Team.
  \end{enumerate}
\item
  When its players are ``on the road,'' each Team shall provide an
  individual hotel room for each player.
\end{enumerate}

\section{First Class Travel.}\label{first-class-travel.}

\begin{enumerate}
\def\labelenumi{(\alph{enumi})}
\tightlist
\item
  Each Team shall provide first class travel accommodations on all trips
  in excess of one (1) hour, except when such accommodations are not
  available; provided, however, that a Team's head coach may fly first
  class in place of a player when eight (8) or more first class seats
  are provided to players. In the event a Team's head coach flies first
  class in place of a player, one (1) player, designated by the Players
  Association, shall be paid the difference between the amount paid by
  such Team for a first class seat on the flight involved and the cost
  of the seat purchased for such designated player on that flight.
\item
  If there is a finding that a Team has committed a willful violation of
  Section 2(a) above, the NBA shall impose a \$5,000 fine on such Team.
\end{enumerate}

\section{Locker Room Facilities.}\label{locker-room-facilities.}

Each Team agrees to provide suitable locker room facilities and to use
its best efforts to stabilize the temperature in locker rooms to make it
consistent with the temperature on playing courts.

\section{Parking Facilities.}\label{parking-facilities.}

Each Team agrees to make parking facilities available to its players
without charge in connection with games and practices conducted at the
facility regularly used by such Team for home games and/or practices.

\section{Hotel Incidentals.}\label{hotel-incidentals.}

In the event that a Player fails or refuses to pay any incidental
charges he has incurred in connection with a hotel room provided to him
by his Team while the Team is ``on the road,'' he shall be subject to
the following discipline: (i) for each of the first two (2) occasions
during the Season -- a maximum fine of \$100; and (ii) for any
subsequent occasion during such Season, such discipline as is reasonable
under the circumstances.

\section{Two-Way Players.}\label{two-way-players.}

The foregoing requirements and obligations set forth in Sections 1, 2
and 5 above shall not apply to any Two-Way Player traveling between his
NBA Team and NBADL team.

\chapter{UNION SECURITY, DUES AND
CHECK-OFF}\label{union-security-dues-and-check-off}

\section{Membership.}\label{membership.}

As a condition of employment commencing with the execution of this
Agreement, for the duration of this Agreement only, and wherever legal:
(a) any active player who is or later becomes a member in good standing
of the Players Association must maintain his membership in good standing
in the Players Association; and (b) any active player (including a
player in the future) who is not a member in good standing of the
Players Association must, on the 30th day following the beginning of his
employment or the 30th day following the execution of this Agreement,
whichever is later, pay, pursuant to Section 2 below or otherwise,
financial core obligations to the Players Association related to
collective bargaining and the administration of collective bargaining
agreements (hereinafter referred to as ``financial core fees'').

\section{Check-off.}\label{check-off.}

Commencing with the execution of this Agreement and for the duration of
this Agreement only, each Team, following its receipt of the requisite
authorization form, will check-off the initiation fee and annual dues,
assessments and financial core fees, as the case may be, in equal
installments from the first four payments made thereafter to the player
pursuant to paragraph 3(a) of the Uniform Player Contract or from such
lesser number of payments made thereafter as provided for by Exhibit 1
to such Contract, for each player for whom a current check-off
authorization has been provided to the Team. The Team will forward the
check-off monies to the Players Association within fourteen (14) days of
each check-off. If the Team fails to do so, interest at 7\% per annum,
payable to the Players Association, shall begin to accrue on such
check-off monies upon the conclusion of such 14-day period.

\section{Enforcement.}\label{enforcement.}

\begin{enumerate}
\def\labelenumi{(\alph{enumi})}
\tightlist
\item
  Upon written notification to the NBA by the Players Association that a
  player has not paid any initiation fee, dues or financial core fees in
  violation of Section 1 above, the NBA will raise the matter for
  discussion with the player and his Team. If there is no resolution of
  the matter within seven (7) days, then the Team will, upon the written
  request of the Players Association, suspend the player without pay,
  wherever legal. Such suspension will continue until the Players
  Association has notified the Team in writing that the suspended player
  has satisfied his obligation as contained in Section 1 above. The
  parties hereby agree that suspension without pay is adopted as a
  substitute for and in lieu of discharge as the penalty for a violation
  of the union security clause of this Agreement and that no player will
  be discharged for a violation of that clause.\\
  A copy of all notices required by this Section 3(a) will be
  simultaneously mailed to the player involved and the NBA.
\item
  The term ``member in good standing'' as used in this Article XIX
  applies only to the payment of dues or any initiation fee and not to
  any other factors involved in union discipline.
\item
  Other than pursuant to Section 2 above, no Team shall pay any
  initiation fees, dues, or financial core fees on behalf of any player.
\end{enumerate}

\section{No Liability.}\label{no-liability.}

Neither the NBA nor any Team shall be liable for any salary, bonus, or
other monetary or non-monetary claims that result from a player being
suspended pursuant to the terms of Section 3 above. The Players
Association indemnifies, saves and holds harmless the NBA and each Team
against any and all claims, demands, suits, or other forms of liability
that may arise, directly or indirectly, in connection with the
enforcement or application of any term or provision of this Article XIX,
including, without limitation, claims relating to any action taken by
the NBA or any Team in reliance upon any written authorization provided
hereunder.

\chapter{SCHEDULING}\label{scheduling}

\section{Training Camp.}\label{training-camp.}

\begin{enumerate}
\def\labelenumi{(\alph{enumi})}
\item
  Veteran Players will not be required to attend training camp earlier
  than 11 a.m. (local time) on the twenty-second day prior to the first
  game of any Regular Season. On such twenty-second day, Veterans may
  only be required to attend a Team dinner and Team meetings,
  participate in photograph and media sessions, and submit to a physical
  examination.
\item
  Notwithstanding Section 1(a) above, if a Veteran Player is under
  contract to a Team that is scheduled during a particular NBA Season to
  participate outside North America in an Exhibition game or a Regular
  Season game during the first ten (10) days of the Regular Season, such
  Veteran Player may be required to attend the training camp conducted
  in advance of that Regular Season by 11 a.m. (local time) on the
  twenty-fifth day prior to the first game of the Regular Season.
\item
  Rookies may be required to attend training camp on a date earlier than
  the date(s) specified in Sections 1(a) and 1(b) above, but no earlier
  than ten (10) days prior to the date that Veterans on such Team are
  required to attend.
\item
  \begin{enumerate}
  \def\labelenumii{(\roman{enumii})}
  \tightlist
  \item
    Team training camps may be held at any location, within or outside
    the United States and Canada. The NBA shall oversee the arrangements
    made with respect to any training camp held outside the United
    States and Canada and the accommodations provided to participating
    players.
  \item
    The NBA shall be required to notify the Players Association of its
    intention to conduct a team training camp outside the United States
    and Canada. Within three (3) business days of its receipt of such
    notification, the Players Association shall have the right to
    disapprove such plans, provided that such disapproval may be based
    solely on a reasonable and well-founded concern that the location of
    such training camp would be unsafe for players.
  \item
    No Team shall hold its training camp outside the United States and
    Canada in any two (2) successive Seasons, it being understood that
    limited practice sessions held in connection with one (1) or more
    exhibition games outside of the United States or Canada shall not be
    considered training camp for the purposes of this Section 1(d)(iii).
  \item
    Players on a Team that holds its training camp outside of the United
    States and Canada shall have at least one (1) day off following the
    travel day during which they travel back to the United States from
    such training camp.
  \end{enumerate}
\item
  \begin{enumerate}
  \def\labelenumii{(\roman{enumii})}
  \tightlist
  \item
    During any six (6) days beginning on the day after the first day of
    training camp and ending on the fourteenth (14th) day of training
    camp (the ``Two-a-Day Period''): (A) a Team shall be permitted to
    conduct no more than two (2) regular practice sessions per day; (B)
    such session(s) may last an aggregate of no longer than 3.5 hours
    (excluding time -- not to exceed 30 minutes -- spent stretching and
    participating in aerobic warm-ups and cool-downs); (C) there must be
    at least a two (2) hour interval between the two (2) practice
    sessions; and (D) if a Team elects to conduct two (2) regular
    practice sessions during a day, one (1) of the two (2) sessions must
    be limited to non-contact activities. For the remainder of training
    camp, a Team shall be permitted to conduct no more than one (1)
    regular practice session per day and such session may last no longer
    than 3.5 hours (excluding time -- not to exceed 30 minutes -- spent
    stretching and participating in aerobic warm-ups and cool-downs);
    provided, however, that any Team that is unable due to international
    travel for pre-season events to conduct two (2) practice sessions
    per day during the Two-a-Day Period may make up any missed practice
    sessions (up to a maximum of two (2)) during the first five (5) days
    upon the Team's return from such international travel.
  \item
    If a Team conducts one (1) or two (2) regular practice sessions
    during a day in accordance with Section 1(e)(i) above, then except
    as provided in clause (A) of Section 1(e)(iii) below, the Team shall
    not, at a separate time during the day, conduct, organize or
    supervise any additional basketball activity on the basketball
    court.
  \item
    Nothing in Section 1(e)(i) and (ii) above shall be construed to
    prohibit a Team, on any day of training camp, from conducting one
    (1) or two (2) regular practice sessions in accordance with Section
    1(e)(i) above, plus:

    \begin{enumerate}
    \def\labelenumiii{(\Alph{enumiii})}
    \tightlist
    \item
      on-court skills development sessions (e.g., pick-and-roll
      situations, shooting, passing, etc.) not involving the playing of
      live defense (i.e., only ``dummy'' defense may be played) and not
      involving the practicing of four-man or five-man offenses or
      defenses; and
    \item
      team-related or training-related activities (including, but not
      limited to, weight training, other conditioning sessions
      (excluding high-impact conditioning drills that are normally
      conducted during regular practice sessions), video sessions,
      meetings, and promotional appearances), so long as such additional
      activities do not include any basketball activity on the
      basketball court that is organized, supervised, or conducted by
      the Team.
    \end{enumerate}
  \end{enumerate}
\end{enumerate}

\section{Exhibition Games.}\label{exhibition-games.}

\begin{enumerate}
\def\labelenumi{(\alph{enumi})}
\tightlist
\item
  Exhibition games prior to any Regular Season shall not exceed six (6)
  (including intra-squad games for which admission is charged), and
  Exhibition games during any Regular Season shall not exceed three (3).
\item
  Exhibition games shall not be played on the three (3) days prior to
  the Team's first Regular Season game in the United States and Canada,
  on the day prior to a Regular Season game, or on the day prior to and
  the day following the All-Star Game.
\end{enumerate}

\section{Regular Season Games.}\label{regular-season-games.}

Each Team agrees that in no event will it play more than eighty-two (82)
Regular Season games.

\section{Location and Scheduling of
Games.}\label{location-and-scheduling-of-games.}

\begin{enumerate}
\def\labelenumi{(\alph{enumi})}
\tightlist
\item
  Exhibition and Regular Season games may be conducted at any location,
  within or outside the United States and Canada. The NBA shall
  supervise the arrangements made with respect to games conducted
  outside the United States and Canada and the accommodations provided
  to participating players.
\item
  Each year the NBA shall establish the schedule of Regular Season and
  Playoff games in its discretion (subject to Article XXIX, Section 5),
  provided that the number of days beginning on the date of the first
  Regular Season game and continuing through the date of the last
  Regular Season game each Season shall equal approximately (177).
  Notwithstanding the foregoing, if any Regular Season or Playoff games
  are cancelled due to one or more events set forth in Article XXXIX,
  Section 5 (e.g., weather or natural disasters) or any other unexpected
  game cancellation (e.g., due to unexpected unavailability of a Team's
  arena or transportation), the NBA may re-schedule any such cancelled
  game(s) in its discretion, after consulting with the Players
  Association.
\item
  Prior to the NBA's public announcement of the Regular Season game
  schedule each year, the NBA shall provide the Players Association with
  an initial draft of such schedule (no later than the date that such
  draft is provided to all NBA teams), and the Players Association shall
  have an opportunity to provide the NBA with comments (within at least
  as many days as NBA teams are given by the NBA to provide such
  comments). The NBA shall consider, but shall have no obligation to
  make any changes in respect of, the Players Association's comments.
  The Players Association shall keep the draft schedule confidential,
  including by maintaining the confidentiality of any differences
  between the final schedule publicly announced by the NBA and the draft
  schedule previously received by the Players Association.
\end{enumerate}

\section{Holidays.}\label{holidays.}

\begin{enumerate}
\def\labelenumi{(\alph{enumi})}
\tightlist
\item
  No Team will be required to play a game on December 25, unless such
  game is to be telecast or cablecast nationally.
\item
  Games scheduled to be played on January 1 and Good Friday shall not
  commence prior to 6 p.m. (local time), unless the Players Association
  consents thereto, which consent shall not be unreasonably withheld.
  The Players Association will, upon request, consent to the earlier
  commencement of two (2) games on Good Friday and four (4) games on
  January 1 if such games are to be broadcast or cablecast nationally,
  and provided that the Teams involved are in the same time zone or
  otherwise in close geographic proximity.
\item
  Teams at home on December 25 and January 1 (each, a ``Holiday'') may,
  but shall not be required to, conduct a practice on either (or both)
  of such Holidays, provided: (i) the Team's players have requested that
  they practice on the Holiday, as communicated to the Team by the
  Team's player representative; and (ii) within seven (7) days before or
  after the Holiday, the Team's players are provided with a ``day off''
  --- i.e., the Team will not conduct any practice, including any
  optional practice, on such date, and the Team will not have a
  scheduled game on such date.
\item
  Teams shall not depart for an away game or series of away games prior
  to 3 p.m. (local time) on December 25 or January 1, unless reasonable
  transportation arrangements for such game or games cannot be made at
  or after 3 p.m. (local time).
\end{enumerate}

\section{All-Star.}\label{all-star.}

No Team that plays a game on the Thursday prior to the All-Star Game
shall play a game on the Tuesday following the All-Star Game or conduct
a practice session prior to such Tuesday at 2 p.m. (local time).

\section{Travel.}\label{travel.}

\begin{enumerate}
\def\labelenumi{(\alph{enumi})}
\tightlist
\item
  The NBA and its Teams shall use their best efforts to devise
  reasonable travel schedules when Team training camps, Exhibition
  games, and Regular Season games are conducted or played outside the
  United States and Canada.
\item
  No Team shall be required to play a scheduled game on the same day
  that such Team has traveled across two (2) time zones, except in
  unusual circumstances and unless the Players Association consents
  thereto, which consent shall not be unreasonably withheld.
\end{enumerate}

\section{Days Off.}\label{days-off.}

\begin{enumerate}
\def\labelenumi{(\alph{enumi})}
\tightlist
\item
  Each Team will provide a minimum of eighteen (18) Days Off during each
  Regular Season for each of its players on dates to be determined by
  the Team. A ``Day Off'' means a calendar day on which a player is not
  required or permitted to participate in any Team directed activities,
  including, but not limited to, games, practices, travel, or
  promotional activities. Without limitation, Days Off shall include
  days that satisfy the foregoing definition and are provided: (i)
  during All-Star Weekend pursuant to Article XXI, Section 4 (only with
  respect to players not participating in All-Star activities); and (ii)
  in locations other than the Team's home city (such as when the Team is
  ``on the road''). Under no circumstances shall a Team pressure or
  coerce a player into providing services for the Team on a player's Day
  Off. Nothing contained herein shall prevent any player on his Day Off
  from voluntarily engaging in individual basketball related activity at
  the Team's facility or elsewhere (including, but not limited to,
  individual activity with Team coaches, trainers, or medical
  personnel).
\item
  A calendar day shall not fail to meet the definition of a Day Off
  because the team is traveling on such day, provided the team lands at
  its destination airport (or, if the team has not flown and is instead
  traveling by train or bus, arrives at the final destination of such
  train or bus) before 1:00 a.m. on such day.
\item
  For a player whose Player Contract is a Rest-of-Season Contract, the
  Team will provide a minimum number of Days Off during the Regular
  Season, rounded up or down to the nearest whole Day Off, calculated by
  multiplying 18 by a fraction, the numerator of which is the number of
  days remaining in the NBA Regular Season as of the date such
  Rest-of-Season Contract is entered into (including the day on which
  the Contract is entered into), and the denominator of which is the
  total number of days of that NBA Regular Season. Teams are not
  required to provide any Day Off to a player whose Player Contract is a
  10-Day Contract or Two-Way Contract.
\item
  For a player whose Player Contract is assigned by one Team to another
  Team during a Regular Season via trade or the NBA's waiver procedure,
  the assignor Team's obligation pursuant to Article XX, Section 8(a)
  shall be deemed satisfied with respect to the player for such Regular
  Season, and the acquiring Team will provide the player a minimum
  number of Days Off during such Regular Season calculated as if the
  player had entered into a Rest-of-Season Contract: (a) in the case of
  a trade, on the date that all conditions to the trade are satisfied;
  or (b) in the case of a waiver claim, on the date that the acquiring
  Team acquires the player's Contract pursuant to the NBA waiver
  procedure.
\item
  Twice during each Season, the NBA will collect from all Teams, and
  provide to the Players Association, a list of the Days Off given to
  each player on such team during the applicable period.
\item
  In the event that any Season does not include at least an eighty-two
  (82) game schedule, the requirements of Sections 8(a)-(d) above shall
  not apply and the NBA and Players Association will negotiate an
  alternate Days Off rule for such Season.
\end{enumerate}

\chapter{NBA ALL-STAR GAME}\label{nba-all-star-game}

\section{Participation.}\label{participation.}

\begin{enumerate}
\def\labelenumi{(\alph{enumi})}
\tightlist
\item
  Any player selected (by any method designated by the NBA) to play in
  an All-Star Game shall be required to:

  \begin{enumerate}
  \def\labelenumii{(\roman{enumii})}
  \tightlist
  \item
    attend and participate in such Game;
  \item
    attend and participate in one (1) All-Star Skills Competition (but
    not including the Slam Dunk Competition) designated by the NBA that
    is conducted during the All-Star Weekend on which such Game is held;
    and
  \item
    attend and participate in every other event conducted in association
    with such All-Star Weekend, including, but not limited to, a
    reasonable number of media sessions, television appearances, and
    promotional appearances.
  \end{enumerate}
\item
  Any player selected (by any method designated by the NBA) to play in a
  Rookie-Sophomore Game (e.g., Rookies vs.~Sophomores, captains-selected
  mix of Rookies and Sophomores on each team, or U.S. players
  vs.~international players) shall be required to:

  \begin{enumerate}
  \def\labelenumii{(\roman{enumii})}
  \tightlist
  \item
    attend and participate in such Game;
  \item
    attend and participate in any All-Star Skills Competition designated
    by the NBA that is conducted during the All-Star Weekend on which
    such Game is held; and
  \item
    attend and participate in every other event conducted in association
    with such All-Star Weekend, including, but not limited to, a
    reasonable number of media sessions, television appearances, and
    promotional appearances.
  \end{enumerate}
\item
  Any player who has not been selected to play in the All-Star Game or
  the Rookie-Sophomore Game, but has been selected (by any method
  designated by the NBA) to participate in an All-Star Skills
  Competition (but not including the Slam Dunk Competition) shall be
  required to attend and participate in such Skills Competition.
  Notwithstanding the foregoing, no player will be required to attend
  and participate in such All-Star Skills Competition for more than two
  (2) consecutive years, unless he is the prior year's winner of such
  All-Star Skills Competition. Any player who, at the request of the
  NBA, voluntarily agrees to participate in the Slam Dunk Competition,
  shall be required to attend and participate in such Slam Dunk
  Competition.
\item
  Nothing in this Article XXI shall preclude a player who is an officer
  or a representative of the Players Association from attending the
  Players Association's annual meeting during All-Star Weekend or
  preclude any player from attending the Players Association's All-Star
  party.
\item
  Notwithstanding anything to the contrary in Section 1(a), (b) or (c)
  above, a player will not be required to participate in a particular
  All-Star Game, Rookie-Sophomore Game, or All-Star Skills Competition
  if he has been excused from participation in the particular event by
  the Commissioner because (i) he has an injury or illness that renders
  him physically unable to participate in such Game or Skills
  Competition, or (ii) for such other reason as the Commissioner may
  determine in his sole discretion. If the player asserts that he should
  be excused from participation in a particular All-Star game or event
  under Section 1(e)(i) above, the Commissioner shall be authorized to
  require the player to submit to a medical examination to be performed
  by a physician designated by the NBA, and the determination of whether
  the player has satisfied Section 1(e)(i) shall be made by such
  physician in his sole discretion. In the event that a player is
  excused from participation in an All-Star game or event under Section
  1(e)(i) above, he shall thereafter remain on his Team's Inactive List
  until he is cleared to return to the Active List by the NBA.
\item
  Any player who is selected to play in an All-Star Game but is excused
  from participation under Section 1(e) above shall not receive the
  All-Star award due to him under Section 2(a) below unless (i) he does
  not play in his Team's last Regular Season game prior to that All-Star
  Game or (ii) he does not play in his Team's first Regular Season game
  following that All-Star Game.
\end{enumerate}

\section{Awards.}\label{awards.}

The awards in connection with the All-Star Game, Rookie-Sophomore Game,
and All-Star Skills Competitions shall, in the aggregate, for each
Season beginning with the 2017-18 Season, be a total of \$600,000
greater than such awards in the aggregate for each Season as set forth
below, to be allocated in a manner agreed upon by the parties:

\begin{enumerate}
\def\labelenumi{(\alph{enumi})}
\tightlist
\item
  For their participation in an All-Star Game, players on the winning
  team shall each receive \$50,000 and players on the losing team shall
  each receive \$25,000.
\item
  For their participation in a Rookie-Sophomore Game, players on the
  winning team shall each receive \$25,000 and players on the losing
  team shall each receive \$10,000.
\item
  For their participation in an All-Star Skills Competition, players
  shall receive the following amounts:
\end{enumerate}

\begin{longtable}[]{@{}lclc@{}}
\toprule
Slam Dunk & & Three-Point Shootout &\tabularnewline
\midrule
\endhead
1st Place: & \$100,000 & 1st Place: & \$50,000\tabularnewline
2nd Place: & \$50,000 & 2nd Place: & \$35,000\tabularnewline
3rd Place: & \$20,000 & 3rd Place: & \$25,000\tabularnewline
4th Place: & \$20,000 & 4th Place: & \$10,000\tabularnewline
& & 5th Place: & \$10,000\tabularnewline
& & 6th Place: & \$10,000\tabularnewline
\bottomrule
\end{longtable}

\begin{longtable}[]{@{}lclc@{}}
\toprule
Skills & & Shooting Stars &\tabularnewline
\midrule
\endhead
1st Place: & \$50,000 & Winning Team: & \$60,000\tabularnewline
2nd Place: & \$35,000 & 2nd Place Team: & \$45,000\tabularnewline
3rd Place: & \$15,000 & 3rd Place Team: & \$24,000\tabularnewline
4th Place: & \$15,000 & 4th Place Team: & \$24,000\tabularnewline
\bottomrule
\end{longtable}

\section{Player Guests.}\label{player-guests.}

Each player who participates in the All-Star Game, Rookie-Sophomore
Game, or any All-Star Skills Competition may invite two (2) guests, who
shall be reimbursed for the cost of round-trip first-class air
transportation between the home city of the Team by which such player is
employed and the site of the All-Star Game, Rookie-Sophomore Game or
All-Star Skills Competition.

\section{Players Not Participating in All-Star
Activities.}\label{players-not-participating-in-all-star-activities.}

Players who do not attend or participate in the All-Star Game,
Rookie-Sophomore Game, an All-Star Skills Competition, or D-League
All-Star activities shall have three (3) days off during the All-Star
Weekend break.

\section{All-Star Skills
Competitions.}\label{all-star-skills-competitions.}

The All-Star Skills Competitions that take place during any All-Star
Weekend shall be selected by the NBA; provided, however, that before
adding any new event to the All-Star Skills Competitions that take place
during any All-Star Weekend (i.e., an event different from any conducted
by the NBA during any All-Star Weekend held prior to the 2017-18
Season), the NBA shall obtain the consent of the Players Association,
which consent shall not be unreasonably withheld. The rule relating to
mandatory participation in Section 1(c) above shall apply only to
current All-Star Skills Competitions (with the exception of the Slam
Dunk Competition), unless the player is the prior year's winner of an
All-Star Skills Competition (with the exception of the Slam Dunk
Competition), and the new event is consented to by the Players
Association under this Section 5.

\chapter{PLAYER HEALTH AND WELLNESS}\label{player-health-and-wellness}

\section{Requirements for Certain Team Player Health
Professionals.}\label{requirements-for-certain-team-player-health-professionals.}

\begin{enumerate}
\def\labelenumi{(\alph{enumi})}
\tightlist
\item
  Each Team agrees to secure the services of at least two (2) physicians
  as team physicians. Beginning with the 2017-18 Season, each individual
  hired for the first time to perform services as a team physician must
  be a duly licensed physician who as of the hiring date: (i) is board
  certified and fellowship trained in his/her field of medical
  expertise; (ii) has at least five (5) years of post-fellowship
  clinical experience; and (iii) has successfully completed a fellowship
  in sports medicine, has a Certification of Added Qualification (CAQ)
  in sports medicine, or has other ``sports medicine'' qualifications as
  the parties may agree.
\item
  Each Team agrees to secure the services of at least one (1) athletic
  trainer to serve as the Head Athletic Trainer and one (1) athletic
  trainer to serve as an Assistant Athletic Trainer on a full-time
  basis. Beginning with the 2017-18 Regular Season: (i) each individual
  hired for the first time to perform services as an athletic trainer
  for a Team must as of the hiring date: (a) be certified by the
  National Athletic Trainers Association (NATA) or the Canadian Athletic
  Therapists Association (CATA) (or a similar organization as the
  parties may agree), and (b) hold a current certification in Basic
  Cardiac Life Support or Basic Trauma Life Support; and (ii) each
  individual hired for the first time to perform services as a Head
  Athletic Trainer for a Team must, as of the hiring date, have at least
  three (3) years of experience as an athletic trainer since he/she
  first received the foregoing NATA/CATA certification.
\item
  Each Team must secure the services of at least one (1) strength and
  conditioning coach on a full-time basis and designate one (1) strength
  and conditioning coach as the Head Strength and Conditioning Coach.
  Beginning with the 2017-18 Regular Season: (i) each individual hired
  for the first time to perform services as a strength and conditioning
  coach for a Team must, as of the hiring date, have a degree from an
  accredited four-year college or university and a certification from
  the National Strength and Conditioning Association (NSCA) (or a
  similar organization as the parties may agree); and (ii) each
  individual hired for the first time to perform services as a Head
  Strength and Conditioning Coach for a Team must, as of the hiring
  date, have at least three (3) years of experience as a strength and
  conditioning coach since he/she first received the foregoing
  certification.
\end{enumerate}

\section{One Surgeon.}\label{one-surgeon.}

Each Team agrees that a player requiring the care and treatment of an
orthopedic surgeon will, so far as practicable, be referred to and
treated by one (1) orthopedic surgeon (rather than several).

\section{NBA Physicians Association.}\label{nba-physicians-association.}

Representatives designated by the Players Association shall participate
in meetings of the NBA Physicians Association for the purpose of
discussing matters related to the medical care and treatment of players.

\section{Disclosure of Medical or Health
Information.}\label{disclosure-of-medical-or-health-information.}

\begin{enumerate}
\def\labelenumi{(\alph{enumi})}
\tightlist
\item
  A Team physician may disclose all relevant medical information
  concerning a player to (i) the General Manager, coaches, and trainers
  of the Team by which such player is employed, (ii) any entity from
  which any such Team seeks to procure, or has procured, an insurance
  policy covering such player's life or any disability, injury, illness,
  or other health condition such player may suffer or sustain, and (iii)
  subject to the terms of Section 4(d) below, the media or public on
  behalf of the Team.
\item
  Should it be requested in connection with the contemplated assignment
  of a player's Uniform Player Contract to one or more NBA Teams, a
  Team's physician may furnish all relevant medical information relating
  to the player to (i) the physicians and General Manager, coaches, and
  trainers of such other Team or Teams, and (ii) any entity from which
  any such other Team seeks to procure, or has procured, an insurance
  policy covering such player's life or any disability, injury, illness
  or other health condition such player may suffer or sustain.
\item
  Should a Team assign a player to the NBADL, such Team's physician may
  furnish all relevant medical information relating to the player to (i)
  the physicians and General Manager, head coaches, and trainers of the
  player's NBADL team, and (ii) any entity from which the Team, the
  NBADL, or the player's NBADL team seeks to procure, or has procured,
  an insurance policy covering such player's life or any disability,
  injury, illness or other health condition such player may suffer or
  sustain. In addition, an NBADL team physician may furnish all relevant
  medical information relating to the player to the physicians and
  General Manager, coaches, and trainers of the player's Team.
\item
  Subject to Section 4(e) below, each Team may make public medical
  information relating to the players in its employ, provided that such
  information relates solely to the reasons why any such player has not
  been or is not rendering services as a player.
\item
  A player or his immediate family (where appropriate) shall have the
  right to approve the terms and timing of any public release of medical
  information relating to any injuries, illnesses or other health
  conditions suffered by that player that are potentially life- or
  career-threatening, or that do not arise from the player's
  participation in NBA games or practices. If a Team or the NBA requests
  such approval and the player or his immediate family (where
  appropriate) does not provide it, then the Team is limited to
  disclosing that an injury, illness or other health condition is
  preventing a player from rendering services to the Team and that the
  anticipated length of the player's absence from rendering services to
  the Team is unknown. Nothing in this Section 4(e) shall limit a Team
  from disclosing medical information related to an injury, illness or
  other health condition with respect to any player who has made medical
  information available publicly that is inconsistent with the written
  opinion of a Team physician.
\item
  A player is entitled access to his own medical records and the Team
  shall use best efforts to provide such information on or before
  forty-eight (48) business hours of a player request.
\end{enumerate}

\section{Draftees.}\label{draftees.}

Prior to any NBA Draft, the NBA and/or its Teams, acting jointly, may
request that persons eligible for such Draft voluntarily submit to the
administration of standardized medical or laboratory tests (other than
tests for controlled substances), and intelligence and/or personality
tests, the results of which shall be made available to any Team upon
request, but which shall be kept confidential from the public and the
media. Any person who submits to the administration of such tests may,
prior to such Draft, be requested to submit voluntarily to an
examination by the physician(s) for an NBA Team(s), but shall not be
requested to undergo any further medical or laboratory test administered
at the request of the NBA and/or its Teams acting jointly unless such
follow-up testing is deemed necessary by an NBA-appointed physician on
the basis of the initial testing results.

\section{Selection of Team Physician and Other Health Care
Providers.}\label{selection-of-team-physician-and-other-health-care-providers.}

Each Team has the sole and exclusive discretion to select any doctors,
hospitals, clinics, health consultants, or other health care providers
(``Health Care Providers'') to examine and/or treat players pursuant to
the terms of this Agreement and the Uniform Player Contract; provided,
however, no Team will engage any such Health Care Provider based
primarily on a sponsorship relationship (or lack thereof) with the Team,
and without considering the Health Care Provider's qualifications
(including, e.g., medical experience and credentials) and the goal of
providing high quality care to all of its players.

\section{Health Screenings.}\label{health-screenings.}

Players shall submit to reasonable screening and baseline testing (e.g.,
pursuant to NBA cardiac and concussion protocols) and, in connection
with such screening and testing, shall accurately and completely answer
all reasonable health questions (including, upon request, providing
accurate and complete medical histories).

\section{Electronic Medical Records.}\label{electronic-medical-records.}

The NBA will use, during the Term, an electronic medical records system
(``EMR'') that will provide a secure, searchable, centralized database
of player health information. To the extent health information
disclosures are permitted by this Agreement (including the Uniform
Player Contract), such disclosures may be made via secure systems within
the EMR. In addition, the EMR will: (i) allow for the NBA (but not the
Teams) to conduct player health and safety reviews; (ii) allow for
authorized academic researchers to access the data (on a de-identified
basis) and conduct studies designed to improve player health and broaden
medical knowledge (provided that the Players Association will be
provided with notice prior to any such access and gives its consent,
such consent not to be unreasonably withheld); and (iii) give players
the ability to easily access their own health information and to grant
access to such information to physicians of their choice both during and
after their careers.

\section{Concussion Policy.}\label{concussion-policy.}

\begin{enumerate}
\def\labelenumi{(\alph{enumi})}
\tightlist
\item
  A concussion policy designed to maximize the neurological health of
  players, which was first developed in conjunction with the NBA
  Physicians Association and implemented beginning with the 2011-12
  Season, shall continue to be in effect during the Term.
\item
  The concussion policy will be reviewed and updated periodically by the
  NBA in conjunction with the NBA Physicians Association in order to
  keep it current and consistent with the evolving science of concussion
  management. Prior to any update to the concussion policy, the NBA
  shall consult with the Players Association.
\end{enumerate}

\section{Second Opinion.}\label{second-opinion.}

\begin{enumerate}
\def\labelenumi{(\alph{enumi})}
\tightlist
\item
  Subject to the additional terms in subsections (b) through (e) below,
  players shall have the right to receive a second medical opinion at
  the Team's expense regarding the course of treatment for an injury,
  illness, or other health condition that either: (i) has prevented the
  player from participating in a Regular Season or playoff game for two
  (2) weeks or more; (ii) in the opinion of a Team physician for the
  player's Team, is more likely than not to prevent the player from
  being able to participate in an NBA game for two (2) weeks or more (or
  during the off-season, from participating in competitive basketball
  without restriction for two weeks or more); (iii) in the opinion of
  the Team physician will not be significantly aggravated by the player
  continuing to participate in NBA games (or during the offseason
  participating in basketball without restriction) when the player
  reasonably believes that continued participation will significantly
  aggravate his injury, illness or condition; (iv) results in direction
  from the Team physician that the player should undergo surgery; or (v)
  results in direction from the Team physician that the player should
  not undergo surgery when the player reasonably believes that surgery
  is necessary for the injury, illness or other health condition. The
  foregoing shall not limit a player's ability to obtain a second
  medical opinion in circumstances other than those set forth in
  Sections 10(a)(i)-(v) above, provided that the Team shall not be
  obligated to pay for or consider any such second opinion.
\item
  The parties will maintain a list (the ``Second Opinion List'') of
  jointly-appointed medical specialists (each a ``Second Opinion
  Physician''), by specialty and by geographic region in the United
  States and Canada, to provide players with the second medical opinions
  described in subsection (a) above. At least two (2) board-certified
  physicians shall be designated as Second Opinion Physicians for each
  specialty in each of the geographic regions.
\item
  Each Second Opinion Physician will be included on the Second Opinion
  List for the duration of this Agreement, unless either the NBA or the
  Players Association has, by December 1 of any year covered by this
  Agreement, provided written notice to the other party that a physician
  should be removed from the Second Opinion List. Such removal shall be
  effective immediately, provided that, unless otherwise agreed by the
  parties, such removal shall not affect any second opinion process
  involving such Physician that has previously been requested by a
  player.
\item
  Prior to obtaining a second opinion, a player shall notify the Team in
  writing of his decision to seek such second opinion, the name of the
  physician who will be performing the evaluation, and the date and
  location of the evaluation. Upon receiving such notice and prior to
  the player's evaluation, the Team will make available to the physician
  relevant medical information regarding the player.
\item
  If, pursuant to subsections (a) through (d) above, a player obtains a
  second opinion from a Second Opinion Physician, the team will pay the
  medical costs associated with the second opinion provided such cost is
  reasonable for the consultation.
\item
  In connection with obtaining a second opinion from a Second Opinion
  Physician pursuant to subsections (a) through (e) above, a player may
  not be absent from the Team for an unreasonable period of time or miss
  any games without authorization of the Team.
\item
  If the Second Opinion Physician provides the Team with a written
  opinion, and the player has otherwise complied with Paragraph 7(h) of
  the UPC, the Team will be required to consider the second opinion in
  connection with diagnosis or treatment. For clarity, nothing in this
  Section 10 shall be construed to alter or limit in any way the rights
  of any Team or the obligation of any player under the CBA or Uniform
  Player Contract, including without limitation pursuant to the
  provisions of paragraph 7 of the Uniform Player Contract.
\end{enumerate}

\section{Fitness-to-Play.}\label{fitness-to-play.}

\begin{enumerate}
\def\labelenumi{(\alph{enumi})}
\item
  The parties shall establish panels of physicians (each a
  ``Fitness-to-Play Panel'') for the purpose of determining, as set
  forth in this Section 11, whether players with potentially
  life-threatening injuries, illnesses or other health conditions are
  medically able and medically fit to practice and play basketball in
  the NBA. Each Fitness-to-Play panel shall consist of one (1) physician
  appointed by the NBA, one (1) physician appointed by the Players
  Association, and one (1) physician appointed by agreement of the first
  two (2) physicians. Each member of each panel shall: (i) be board
  certified and fellowship trained in his/her field of medical
  expertise; (ii) be a specialist in the subject matter of the
  applicable Fitness-to-Play Panel; and (iii) have at least ten (10)
  years of post-fellowship clinical experience. Each panel will operate
  by majority vote, including but not limited to its fitness to play
  determinations. Once appointed, each physician on a Fitness-to-Play
  Panel shall be included on such Panel for the duration of this
  Agreement, unless either the NBA or the Players Association has, by
  December 1 of any year covered by this Agreement, served written
  notice to the other party that a physician has been removed from such
  Panel. A party may not remove the physician that the other party
  appointed to a Fitness-to-Play Panel. In the event that either party
  removes a physician from a Fitness-to-Play Panel pursuant to the
  foregoing, such removal shall be effective immediately, provided that,
  unless otherwise agreed to by the parties, a physician will continue
  to serve on the Fitness-to-Play panel in respect of any determination
  on a player's injury, illness, or medical condition that has been
  referred to the panel but for which the panel has not yet issued its
  written determination.
\item
  On or before July 1, 2017, the NBA and the Players Association shall
  form two panels (each a ``Fitness-to-Play Panel'') with respect to:
  (i) cardiac illnesses and conditions and (ii) blood clots and other
  blood conditions and disorders. The parties shall create such
  additional Fitness-to-Play Panels as are necessary to address other
  types of potentially life-threatening injuries, illnesses or health
  conditions that may arise.
\item
  If the NBA, a Team, or the Players Association has been advised by a
  physician that a player is medically unable and/or medically unfit to
  perform his duties as a professional basketball player as a result of
  a potentially life-threatening injury, illness or other health
  condition and/or that performing such duties would create a materially
  elevated risk of death for the player, then the NBA, a Team, or the
  Players Association may refer the player to a Fitness-to-Play Panel by
  making such a referral in writing to the player and to the NBA, Team,
  and Players Association, as applicable. Once so referred, the player
  will not be permitted to play or practice in the NBA until he is
  cleared to do so by the Panel as set forth below.
\item
  \begin{enumerate}
  \def\labelenumii{(\arabic{enumii})}
  \tightlist
  \item
    Upon the referral described in subsection (c) above, the Panel will
    be provided with all medical information in the player's medical
    file that any member of the Panel deems relevant to the injury,
    illness or other health condition for which the player was referred.
    The Panel will review the player's injury, illness or other health
    condition (which review shall include an in-person examination of
    the player by each member of the Panel unless such member determines
    that an examination by him/her would serve no useful purpose). Upon
    conclusion of its review, the Panel shall provide a report to the
    NBA, the player's Team, and the Players Association setting forth
    its determination and the reasons therefor.
  \item
    The determination to be made by the Panel is whether, in the panel's
    reasonable medical judgment and experience, and having considered
    current medical knowledge and the best available objective evidence:
    (i) the player is medically able and medically fit to perform his
    duties as a professional basketball player; and (ii) performing such
    duties would not create a materially elevated risk of death for the
    player. Where there are authoritative medical guidelines on fitness
    for athletic participation and a particular injury, illness or other
    health condition (e.g., the American Heart Association/American
    College of Cardiology Scientific Statements on Eligibility and
    Disqualification -- Recommendations for Competitive Athletes with
    Cardiovascular Abnormalities), the panel will consider such
    guidelines in making its determination.
  \item
    Subsequent to the player being referred to a Fitness-to Play-Panel,
    and prior to the Panel's review of the player's injury, illness or
    other health condition, the player (on behalf of himself, his heirs
    and assigns) shall be required to sign a release and covenant not to
    sue agreement in the form agreed upon by the parties; provided that
    this agreement shall not apply to any claim of medical malpractice
    against a Team-affiliated physician or any physician retained by the
    NBA/NBPA for the medical evaluation process.
  \end{enumerate}
\item
  In the event that the Fitness-to-Play Panel determines that the player
  is medically able and medically fit to play professional basketball
  pursuant to the standard in subsection (d) above: (i) the player will
  be required to sign an informed consent and assumption of risk
  agreement in the form agreed upon by the parties before he is able to
  play or practice in the NBA; and (ii) upon satisfying the prior
  clause, shall be deemed at that time medically able and fit to play
  basketball in the NBA and permitted to do so.
\item
  If the Fitness-to-Play Panel does not determine that the player is
  medically able and medically fit to play professional basketball
  pursuant to the standard in subsection (d) above, the NBA, a Team, or
  the Players Association may again refer the player to the
  Fitness-to-Play Panel beginning on the later of the first day of the
  Season that begins immediately following the date on which the Panel
  issued its report or nine (9) months after such date. The party making
  such referral must have been advised in writing by a physician that
  there have been materially changed circumstances since the Panel
  issued its report (e.g., medical advances or a material change in the
  player's medical condition) such that the Panel should reconsider its
  determination. If a player is referred under this subsection (f), the
  Fitness-to-Play Panel shall be comprised of the same members that
  reviewed and determined the player's initial referral, provided that
  the physicians on such panel are available.
\item
  Nothing in this Section 11 shall obligate a Team to permit a player to
  play or practice for the Team, even if a Fitness-to-Play Panel
  determines that the player is medically able to do so. If the Team
  disagrees with the Fitness-to-Play Panel's conclusion and refuses to
  permit the player to play and practice with the Team due to the
  injury, illness, or other health condition for which the player was
  referred to the Fitness-to-Play Panel, then the Team will be required,
  within sixty (60) days of the Panel's issuance of its report (or, if
  the report is issued on or between the date that is sixty (60) days
  prior to the date of the NBA trade deadline and May 31, by August 1)
  (the ``Evaluation Period''), to either trade the player, agree to
  amend the player's contract in accordance with Article II, Section
  3(l) of the CBA, waive the player pursuant to paragraph 16 of the
  Uniform Player Contract, or waive the player pursuant to the ``Partial
  Waiver Procedure'' described in Section 11(i) below (a ``Partial
  Waiver''); provided, however, that the foregoing shall not apply to
  any player who is in the last year of his contract (excluding any
  option year) at the time that the panel provides its report to the
  NBA, the player's Team, and the Players Association pursuant to
  Section 11(d)(1) above. During the Evaluation Period, the player,
  shall cooperate with the Team in connection with the Team's efforts to
  evaluate the player's injury, illness or other health condition,
  including by, among other things, in a prompt and diligent manner
  supplying all information requested of him, completing medical forms,
  and submitting to all examinations, tests and workouts requested of
  him by or on behalf of the Team.
\item
  If a player referred to a Fitness-to-Play Panel satisfies the waiting
  period set forth in Article VII, Section 4(h)(1) of the CBA at the
  time of such referral (or any time thereafter prior to the Panel
  issuing its report), then the Team may request that such panel, acting
  by majority vote, also serve as the physician described in Article
  VII, Section 4(h)(2) of the CBA, and accordingly provide in the
  panel's report a determination for the purposes of Article VII,
  Section 4(h) of the CBA.
\item
  In order for an eligible team, pursuant to Section 11(g) above, to
  designate an eligible player's Contract for a Partial Waiver, the team
  must provide written notice of such waiver and designation to the NBA.
  Once a team duly invokes the Partial Waiver Procedure, such procedure
  shall operate as follows:

  \begin{enumerate}
  \def\labelenumii{(\roman{enumii})}
  \tightlist
  \item
    The waiver period shall be the same as the period for other waivers.
  \item
    Any Team other than the Team requesting the waiver may submit either
    a Full Waiver Claim or a Partial Waiver Claim for the player. A
    ``Full Waiver Claim'' is a claim for the full value of the remaining
    term of the Contract pursuant to Section 6 of the NBA By-Laws. A
    ``Partial Waiver Claim'' is a discount bid of a single dollar amount
    (rounded to the nearest dollar) for a portion of the value of the
    remaining term of the Contract. A Partial Waiver Claim can be for
    any amount equal to or greater than the total of the applicable
    Minimum Player Salary for all of the Remaining Protected Years (as
    defined below) of the Contract and less than the total of the full
    Base Compensation provided for in all of the Remaining Protected
    Years of the Contract, provided that a Partial Waiver Claim may
    never be less than the total of the unprotected Base Compensation
    provided for in all of the Remaining Protected Years of the
    Contract. A ``Remaining Protected Year'' means any remaining year of
    the Contract that contains any amount of Base Compensation
    protection that is not contingent on some event occurring on a date
    after the request for waivers; any remaining years of the Contract
    that are not Remaining Protected Years shall hereinafter be referred
    to as ``Remaining Unprotected Years.'' For clarity, any Player
    Option Year in which the Contract includes the language in Article
    XII, Section 2(a)(A) and the Effective Season of (and any subsequent
    year to) an ETO shall be a Remaining Protected Year, and any Player
    Option Year in which the Contract that includes the language in
    Article XII, Section 2(a)(B) and any Team Option Year shall not.
  \item
    In order to submit a Partial Waiver Claim, the Team must have a Team
    Salary below the Salary Cap and room equal to at least the portion
    of the Claiming Team Base Compensation Obligation (as defined in
    subsection (vi)(A) below) plus any Likely Bonuses applicable to the
    first Year of the Remaining Protected Years of the Contract. For
    purposes of the preceding sentence, ``room'' includes room that can
    be unilaterally created by the claiming Team (e.g., via
    renouncements or waivers, but not via trades) and such room must be
    created immediately upon the awarding of the player pursuant to this
    waiver procedure.
  \item
    If at least one (1) Full Waiver Claim is submitted during the waiver
    period, the Contract shall be awarded to the Team submitting a Full
    Waiver Claim that is entitled to the highest order of preference in
    accordance with the waiver procedures set forth in the NBA
    Constitution and By-Laws. If no Full Waiver Claim is submitted and
    at least one (1) Partial Waiver Claim is submitted, the Contract
    shall be awarded to the Team submitting the highest Partial Waiver
    Claim in total dollars (or, if more than one (1) Team submits the
    highest Partial Waiver Claim in total dollars, to the Team
    submitting the highest Partial Waiver Claim in total dollars that is
    entitled to the highest order of preference in accordance with the
    waiver procedures set forth in the NBA Constitution and By-Laws).
  \item
    If there is no Full Waiver Claim or Partial Waiver Claim submitted
    for the Contract during the waiver period, the Contract shall be
    terminated.
  \item
    In the event that the Contract is awarded to a Team (the ``Claiming
    Team'') as the result of a Partial Waiver Claim:

    \begin{enumerate}
    \def\labelenumiii{(\Alph{enumiii})}
    \tightlist
    \item
      The Claiming Team shall be responsible for payment of the player's
      Base Compensation in an amount equal to the total dollar amount of
      the Partial Waiver Claim allocated over the Remaining Protected
      Years of the Contract in proportion to the Base Compensation
      amounts provided for in each Remaining Protected Year of the
      Contract (e.g., if the player has two (2) years remaining on his
      Contract with a \$10 million fully-protected Base Compensation in
      year one and an \$11 million fifty percent (50\%)-protected Base
      Compensation in year two and the winning Partial Waiver Claim was
      for \$6 million, the Claiming Team shall be responsible for \$2.86
      million of the player's Base Compensation in year one and \$3.14
      million in year two) (the ``Claiming Team Base Compensation
      Obligation''). The waiving team shall be responsible for paying
      the total Base Compensation in each Remaining Protected Year of
      the Contract less the Claiming Team Base Compensation Obligation
      for each Remaining Protected Year of the Contract (the ``Waiving
      Team Base Compensation Obligation''). In addition to the Claiming
      Team Base Compensation Obligation, the Claiming Team shall also be
      responsible for the total amount of all other Compensation
      obligations contained in the Contract other than Base Compensation
      (including, but not limited to, the full amount of any Incentive
      Compensation) and the total Base Compensation for any Remaining
      Unprotected Year.
    \item
      The Claiming Team Base Compensation Obligation plus any Likely
      Bonuses applicable to each Remaining Protected Year of the
      Contract and the total Base Compensation plus any Likely Bonuses
      of any Remaining Unprotected Year shall be included in the Team
      Salary of the Claiming Team immediately upon the awarding of the
      player to the Claiming Team pursuant to this waiver procedure.
    \item
      The Claiming Team may not trade a player awarded as a result of a
      Partial Waiver Claim until the July 1 following the award of the
      player's Contract to the Claiming Team pursuant to this waiver
      procedure. If a Claiming Team trades a player awarded as a result
      of a Partial Waiver Claim (after the waiting period set forth in
      the preceding sentence), the acquiring Team: (i) must have room
      (or a Traded Player Exception in accordance with the rules set
      forth in Article VII, Section 6(j)) in an amount equal to at least
      the Claiming Team Base Compensation Obligation plus any Likely
      Bonuses applicable to the then-current Salary Cap Year; and (ii)
      shall thereafter be deemed the Claiming Team for the purposes of
      this Section 11(i).
    \item
      The Claiming Team shall be responsible for making all payments to
      the player (and paying all related payroll taxes) other than
      Compensation due with respect to any Season prior to the waiver.
      The waiving team shall reimburse the Claiming Team for the portion
      of the Waiving Team Base Compensation Obligation applicable to
      each pay period on or before each applicable pay date.
    \end{enumerate}
  \item
    In the event that the Contract is awarded to the Claiming Team as a
    result of a Partial Waiver Claim and the Claiming Team subsequently
    waives the player (a ``Subsequent Waiver'') resulting in the
    termination of the Contract:

    \begin{enumerate}
    \def\labelenumiii{(\Alph{enumiii})}
    \tightlist
    \item
      Without taking into consideration any conditional Base
      Compensation protection triggered after the date of the initial
      request for waivers but before the Subsequent Waiver (hereinafter
      referred to as ``Triggered Base Compensation Protection''), if the
      Contract contains full Base Compensation protection in each of the
      Remaining Protected Years or if the Contract contains no Remaining
      Protected Years, the Claiming Team Base Compensation Obligation
      and the Waiving Team Base Compensation Obligation shall remain
      unchanged.
    \item
      Without taking into consideration any Triggered Base Compensation
      Protection, if the Contract contains partial protection in one (1)
      or more of the Remaining Protected Years, the Claiming Team Base
      Compensation Obligation and Waiving Team Base Compensation
      Obligation for each such year shall be adjusted as follows upon
      the termination of the Contract:

      \begin{enumerate}
      \def\labelenumiv{(\arabic{enumiv})}
      \tightlist
      \item
        The Claiming Team Base Compensation Obligation for any Remaining
        Protected Year that contains only partial Base Compensation
        protection shall be reduced by a number equal to the Claiming
        Team Base Compensation Obligation for that year, divided by the
        total Base Compensation obligation for that year, multiplied by
        the unprotected Base Compensation remaining to be paid that year
        (the ``Adjusted Claiming Team Base Compensation Obligation'').
      \item
        The Waiving Team Base Compensation Obligation for any Remaining
        Protected Year that contains only partial Base Compensation
        protection shall be reduced by a number equal to the Waiving
        Team Base Compensation Obligation for that year, divided by the
        total Base Compensation obligation for that year, multiplied by
        the unprotected Base Compensation remaining to be paid for that
        year.
      \end{enumerate}
    \item
      The full amount of any Triggered Base Compensation Protection
      shall be added to the Adjusted Claiming Team Base Compensation
      Obligation in each remaining year of Contract that contains
      Triggered Base Compensation Protection.
    \end{enumerate}
  \item
    In the event that the Contract is awarded to a Team as a result of a
    Subsequent Waiver, the Team that is awarded the Contract becomes the
    Claiming Team for the purposes of this Section 11(i) (and,
    accordingly, must have room (or a Traded Player Exception) in an
    amount equal to at least the Claiming Team Base Compensation
    Obligation plus any Likely Bonuses applicable to the then-current
    Salary Cap Year).
  \end{enumerate}
\item
  The costs associated with the Fitness-to-Play Panels will be borne
  equally by the NBA and the Players Association, and the Players
  Association's share shall be paid by the NBA and included in Player
  Benefits under Article IV, Section 6(k) of this Agreement.
\end{enumerate}

\section{Player Care Survey.}\label{player-care-survey.}

The NBA and the Players Association will jointly conduct a confidential
player survey once every two years to solicit the players' input and
opinion regarding the adequacy of medical care provided by their
respective medical and training staffs and commission independent
analyses of the results of such surveys. The costs of such surveys and
analyses will be borne equally by the NBA and the Players Association,
and the Players Association's share shall be paid by the NBA and
included in Player Benefits under Article IV, Section 6(k) of this
Agreement.

\section{Wearables.}\label{wearables.}

\begin{enumerate}
\def\labelenumi{(\alph{enumi})}
\tightlist
\item
  Immediately following the execution of this Agreement, the NBA and the
  Players Association shall form a joint advisory committee (the
  ``Wearables Committee'') to review and approve wearable devices for
  use by players. ``Wearables'' shall mean a device worn by an
  individual that measures movement information (such as distance,
  velocity, acceleration, deceleration, jumps, changes of direction, and
  player load calculated from such information and/or height/weight),
  biometric information (such as heart rate, heart rate variability,
  skin temperature, blood oxygen, hydration, lactate, and/or glucose),
  or other health, fitness, and performance information.
\item
  The Wearables Committee shall consist of three (3) representatives
  appointed by the NBA and three (3) representative appointed by the
  Players Association. At least one of the members appointed by each of
  the NBA and the Players Association must have at least three (3) years
  of experience in sports medicine (such as a physician, athletic
  trainer, strength and conditioning coach, or sports scientist) in the
  NBA or with an NCAA Division I collegiate basketball team. Unless
  otherwise agreed by the parties, Committee members may not have an
  ownership or other financial interest in any company that produces or
  sells any wearable device.
\item
  The Wearables Committee shall be responsible for: (i) reviewing all
  requests by Teams, the NBA, or the NBPA to approve a wearable device
  for use by players, with the standard being whether the wearable
  device would be potentially harmful to anyone (including the player)
  if used as intended, and whether the wearable's functionality has been
  validated; and (ii) setting cybersecurity standards for the storage of
  data collected from Wearables.
\item
  The Wearables Committee will jointly retain such experts as it deems
  necessary in order to conduct its work (e.g., to validate a wearable
  device or to set cybersecurity standards), which the parties expect to
  include professionals in areas such as engineering, data science, and
  cybersecurity. The costs of such experts will be borne equally by the
  NBA and the Players Association, and the Players Association's share
  shall be paid by the NBA and included in Player Benefits under Article
  IV, Section 6(k) of this Agreement.
\item
  No Team may request a player to use any Wearable unless such device is
  one of the devices currently in use as set forth in Section 13(f)
  below or the device and the Team's cybersecurity standards have been
  approved by the Committee pursuant to Section 13(c) above.
\item
  Teams may continue to request that, on a voluntary basis, players use
  the following devices: adidas miCoach elite systems, Catapult Sports
  ClearSky and Optimeye systems, Intel Curie systems, STAT Sports Viper
  systems, VERT Wearable Jump Monitors, Zebra wearable tags, and Zephyr
  Bioharness systems. Once the six (6) members of the Wearables
  Committee have been appointed, the Committee will establish
  cybersecurity standards for the storage of wearable data, and within
  ninety (90) days of the Committee's issuance of such standards, each
  Team using one or more Wearable pursuant to this paragraph must
  confirm to the NBA and the Players Association that it is in
  compliance with such standards. By June 30, 2017, the Committee must
  vote on the continued use of the devices set forth in this paragraph
  or determine that additional time is needed to evaluate a particular
  device. If upon evaluation by the Committee, any of the foregoing
  devices are reviewed and are not approved by the Committee, Teams will
  be required to discontinue the use of such Wearables.
\item
  A Team may request a player to use in practice (or otherwise not in a
  game) on a voluntary basis a Wearable that has been approved by the
  Committee. A player may decline to use (or discontinue use of) a
  Wearable at any time. Before a Team could request that a player use an
  approved Wearable, the Team shall be required to provide the player a
  written, confidential explanation of: (i) what the device will
  measure; (ii) what each such measurement means; and (iii) the benefits
  to the player in obtaining such data.
\item
  A player will have full access to all data collected on him from
  approved Wearables. Members of the Team's staff may also have access
  to such data but it can be used only for limited purposes as set forth
  below. Data collected from a Wearable worn at the request of a Team
  may be used for player health and performance purposes and Team
  on-court tactical and strategic purposes only. The data may not be
  considered, used, discussed or referenced for any other purpose such
  as in negotiations regarding a future Player Contract or other Player
  Contract transaction (e.g., a trade or waiver) involving the player.
  In a proceeding brought by the Players Association under the
  procedures set forth in Article XXXI, the Grievance Arbitrator will
  have authority to impose a fine of up to \$250,000 on any Team shown
  to have violated this provision.
\item
  The parties agree to continue to discuss in good faith the use of
  Wearables in games and the commercialization of data from Wearables.
  Pending an agreement between the parties, Wearables may not be used in
  games, and no player data collected from a Wearable worn at the
  request of a Team may be made available to the public in any way or
  used for any commercial purpose.
\end{enumerate}

\chapter{EXHIBITION GAMES AND OFF-SEASON GAMES AND
EVENTS}\label{exhibition-games-and-off-season-games-and-events}

\section{Exhibition Games.}\label{exhibition-games.-1}

Subject to the provisions of paragraph 2 of the Uniform Player Contract,
players shall be required to participate in Exhibition games between an
NBA Team and a non-member of the NBA at any location, within or outside
the United States, subject to the following conditions:

\begin{enumerate}
\def\labelenumi{(\alph{enumi})}
\tightlist
\item
  The NBA shall supervise the arrangements made with respect to
  tournaments or series conducted outside the United States and the
  accommodations provided to NBA players participating in such foreign
  tournaments or series.
\item
  The NBA shall use its best efforts to establish an Exhibition game
  schedule pursuant to which excessive travel will be avoided and
  reasonable periods of time between games will be allotted.
\item
  In any year in which it is played, the annual Basketball Hall of Fame
  Exhibition game shall be considered as one of the six (6) Exhibition
  games prior to the Regular Season referred to in paragraph 2 of the
  Uniform Player Contract.
\end{enumerate}

\section{Inter-squad Scrimmage.}\label{inter-squad-scrimmage.}

In addition to the Exhibition games provided for by paragraph 2 of the
Uniform Player Contract, and during each of the playoff series conducted
during the term of this Agreement, any Team that qualifies for the
playoffs but is not required to participate in the first round thereof
may arrange and require its players to participate in one inter-squad
game or scrimmage with another similarly-situated Team, provided that
such game or scrimmage is not open to members of the general public.

\section{Off-Season Basketball
Events.}\label{off-season-basketball-events.}

\begin{enumerate}
\def\labelenumi{(\alph{enumi})}
\tightlist
\item
  No player may play in any public off-season basketball game, summer
  league, or public exhibition or competition of basketball skills
  (e.g., a slam dunk contest or a ``tour'' organized by an NBA business
  partner) (each a ``Basketball Event'') unless such Basketball Event is
  approved in writing by the NBA for NBA player participation and
  complies with the terms and conditions of this Section 3. The NBA will
  consider an off-season Basketball Event for approval only if a request
  for such approval is submitted in writing to the NBA, and only if the
  arrangements made with respect to any such off-season Basketball Event
  are confirmed in writing to the NBA and satisfy the following
  requirements, in addition to such other reasonable requirements as the
  NBA may impose:

  \begin{enumerate}
  \def\labelenumii{(\roman{enumii})}
  \tightlist
  \item
    General Requirements.

    \begin{enumerate}
    \def\labelenumiii{(\arabic{enumiii})}
    \tightlist
    \item
      The Basketball Event takes place on or after July 1, but in no
      event later than September 15 (or, in the case of a summer league,
      September 1);
    \item
      Prior to the Basketball Event, each participating player receives
      the express written consent of his Team to participate in the
      Basketball Event;
    \item
      The person(s) organizing the Basketball Event obtains disability
      insurance for the benefit of each participating player's Team, in
      an amount acceptable to the NBA (provided, however, that this
      requirement shall not apply to summer leagues); and
    \item
      The names and logos of the NBA and/or any NBA Team are not used or
      referred to in connection with the Basketball Event, unless the
      NBA provides express written authorization for such use.
    \end{enumerate}
  \item
    Additional Charitable Game Requirements. The NBA will consider an
    off-season charitable game for approval only if, in addition to the
    general requirements set forth in Section 3(a)(i) above and such
    other reasonable requirements as the NBA may impose, the
    arrangements made with respect to such charitable game also satisfy
    the following:

    \begin{enumerate}
    \def\labelenumiii{(\arabic{enumiii})}
    \tightlist
    \item
      The Players Association approves the game (which approval shall
      not be unreasonably withheld);
    \item
      All proceeds from the sale of tickets to the game and other
      sources of revenue from the game (e.g., sponsorship revenue) less
      reasonable expenses incurred to conduct the game are used for
      charitable purposes;
    \item
      The game is officiated by NBA referees assigned by the NBA to
      officiate the game. The person or entity organizing the game will
      be responsible for paying the officiating fees and the actual
      expenses incurred for the referees' lodging and transportation to
      and from the referees' homes to the site of the game;
    \item
      There is at least one (1) NBA Team trainer and at least one (1)
      physician present at the game;
    \item
      The name or likeness of an NBA player is not used, or referred to,
      in advertisements or promotions for or related to the game, except
      that if the organizer of the game is an NBA player, such
      organizer-player's name or likeness may be used, or referred to,
      in such advertisements or promotions;
    \item
      Only current or former professional basketball players participate
      in the game;
    \item
      The game is not accompanied by an exhibition or competition of
      basketball skills (such as a slam dunk contest), unless such
      exhibition or competition has been separately approved in writing
      by the NBA and the Players Association;
    \item
      Participating players are not paid or compensated (in excess of
      per diem and actual reasonable expenses incurred in traveling to
      and participating in the game);
    \item
      The organizer guarantees that the game will produce at least
      \$100,000 for charity, and, if directed by the NBA and the Players
      Association, the organizer (or a third party if the organizer
      itself is a charity) posts security for such amount in a form
      satisfactory to the NBA and the Players Association which grants
      the NBA and/or the Players Association the right to sue to recover
      such amount for the benefit of the charity;
    \item
      The game is played in the United States or Canada; and
    \item
      The organizer agrees to provide the NBA and the Players
      Association with an audited statement of revenues and expenses, in
      a form acceptable to the NBA and the Players Association, within
      sixty (60) days following the game.
    \end{enumerate}
  \item
    Additional Summer League Requirements. The NBA will consider an
    off-season summer league for approval only if, in addition to the
    general requirements set forth in Section 3(a)(i) above and such
    other reasonable requirements as the NBA may impose, the
    arrangements made with respect to each summer league game in which
    an NBA player participates also satisfy the following:

    \begin{enumerate}
    \def\labelenumiii{(\arabic{enumiii})}
    \tightlist
    \item
      Participating players are not paid or compensated (except as
      provided under Section 4(c) below);
    \item
      NBA players do not participate in an exhibition or competition of
      basketball skills (such as a slam dunk contest), unless such
      exhibition or competition has been separately approved in writing
      by the NBA;
    \item
      There is at least one (1) trainer or at least one (1) physician or
      other emergency medical personnel present at the game; and
    \item
      The game is played in the United States or Canada.
    \end{enumerate}
  \end{enumerate}
\item
  Notwithstanding any other terms of this Section 3, and without
  limiting the right of the NBA to approve all arrangements of a
  proposed Basketball Event, the NBA may, in its sole discretion,
  require, as a condition of its approval of a Basketball Event (other
  than a charitable game or summer league), that the Basketball Event
  organizer pay an appropriate fee to the NBA prior to the commencement
  of the Basketball Event.
\item
  For purposes of this Section 3, off-season games in which an NBA
  player participates on behalf of his national basketball federation as
  part of an international FIBA competition (e.g., the Olympics and FIBA
  Basketball World Cup), and the preparatory Exhibition games in
  connection therewith, are excluded from the definition of ``Basketball
  Event;'' provided, however, that such exclusion shall not apply to any
  preparatory Exhibition game (other than games involving the U.S.
  national team) played and/or telecast in the United States.
\item
  Notwithstanding anything to the contrary in this Agreement, a Veteran
  Free Agent remains subject to the provisions of this Section 3 until
  the September 1 following the last Season of his Player Contract;
  provided, however, that any such Veteran Free Agent shall be permitted
  to sign a contract with and play in basketball games for a team in a
  professional basketball league other than the NBA beginning on the
  July 1 immediately following such Season (or prior to July 1 if
  approved in writing by the NBA).
\item
  The NBA shall have the exclusive right to (and to authorize third
  parties to) telecast or broadcast by radio any Basketball Event (in
  whole or in part) that is approved for NBA player participation in
  accordance with this Section 3.
\item
  Notwithstanding anything else in this Article XXIII, the NBA, in
  considering and acting upon a request for approval of a summer league,
  charity game, or other Basketball Event, does not consider or apply
  safety requirements for such leagues, games, or events.
\end{enumerate}

\section{Summer Leagues.}\label{summer-leagues.}

\begin{enumerate}
\def\labelenumi{(\alph{enumi})}
\tightlist
\item
  No NBA Team may simultaneously enroll more than four (4) Veterans in
  any summer basketball league during an off-season. For purposes of
  this Section 4(a), the following players are not considered Veterans:

  \begin{enumerate}
  \def\labelenumii{(\arabic{enumii})}
  \tightlist
  \item
    a player who has never signed a Player Contract or whose first
    Player Contract begins with the Season immediately following the
    off-season in which such summer league is to be conducted;
  \item
    a player not under contract to an NBA Team at the time he enrolls in
    such summer league;
  \item
    a player under contract to an NBA Team but who missed twenty-five
    (25) or more of the Team's games during the Regular Season
    immediately preceding such off-season due to injury or illness; and
  \item
    a player who played for a team in the NBA Development League or any
    other U.S.-based professional league during all, or any portion, of
    the Regular Season immediately preceding such off-season.
  \end{enumerate}
\item
  Prior to playing in a summer basketball league, each player who is
  under contract with a Team for the following Season shall be provided
  by his Team, and requested to sign a ``Notice to Veteran Players
  Concerning Summer Leagues'' in the form attached hereto as Exhibit E.
\item
  The only compensation that may be paid by a Team or any person or
  entity affiliated with a Team to a player participating in a summer
  basketball league is a reasonable expense allowance for: (1) meals,
  but no greater than that set forth in Article III, Section 2; (2)
  lodging; and (3) transportation to and from the player's home to the
  site of the summer league, and to and from the site of the player's
  lodging during the summer league to the site of summer-league-related
  activities. In addition, the Team may purchase a disability insurance
  policy for the player covering the term of the applicable summer
  league.
\item
  No Team shall schedule, and no player shall participate in, a summer
  basketball league that is scheduled to extend, or does in fact extend,
  past September 1 of any calendar year.
\end{enumerate}

\chapter{PROHIBITION OF NO-TRADE
CONTRACTS}\label{prohibition-of-no-trade-contracts}

\section{General Limitation.}\label{general-limitation.}

No Player Contract may contain any prohibition or limitation of an NBA
Team's right to assign such Contract to another NBA Team.

\section{Exceptions to General
Limitation.}\label{exceptions-to-general-limitation.}

Notwithstanding the provisions of Section 1 of this Article XXIV:

\begin{enumerate}
\def\labelenumi{(\alph{enumi})}
\tightlist
\item
  A Player Contract may contain (in Exhibit 4 to such Player Contract) a
  provision entitling a Player to earn Compensation if the player's
  Uniform Player Contract is traded (``trade bonus'') subject to the
  following:

  \begin{enumerate}
  \def\labelenumii{(\roman{enumii})}
  \tightlist
  \item
    A trade bonus shall be payable only the first time that the Contract
    is traded; provided, however, that if a Contract is signed in
    connection with an agreement to trade the Contract in accordance
    with Article VII, Section 8(e) and the Contract contains a trade
    bonus, the bonus shall not apply to such initial trade but shall
    instead be payable only the second time the Contract is traded.
  \item
    A trade bonus shall not exceed fifteen percent (15\%) of the Base
    Compensation remaining to be earned by the player pursuant to the
    Contract at the time of the trade (excluding an Option Year if not
    yet exercised).
  \item
    The only allowable amendments to Exhibit 4 to a Uniform Player
    Contract shall be: (A) the specification of the amount of the trade
    bonus to be paid to the player, expressed as either (1) a specified
    percentage of the Base Compensation remaining to be earned under the
    Contract at the time of the trade (excluding an Option Year if not
    yet exercised), or (2) a specified dollar amount not to exceed a
    specified percentage of Base Compensation remaining to be earned
    under the Contract at the time of the trade (excluding an Option
    Year if not yet exercised); and (B) in connection with the extension
    of a Contract that contains a trade bonus, the specification of
    whether the trade bonus shall apply to the extended term of the
    Contract in accordance with Section 2(a)(v) below.
  \item
    A Contract that does not contain a trade bonus when signed cannot be
    amended to add one, except that: (A) if the Contract is extended
    (other than pursuant to an agreement to trade the extended Contract
    in accordance with Article VII, Section 8(e)), the Contract may be
    amended simultaneously to provide for a trade bonus that will be
    payable only the first time that the Contract is traded following
    the signing of the Extension (and not as a result of any subsequent
    trade), and (B) if the Contract is extended pursuant to an agreement
    to trade the extended Contract in accordance with Article VII,
    Section 8(e), the Contract may be amended simultaneously to provide
    for a trade bonus that shall not apply to such initial trade but
    shall instead be payable only if the extended Contract is traded a
    second time (and not as a result of a subsequent trade).
  \item
    If a Contract is extended that contains a trade bonus that has not
    been previously earned, the Contract may be amended simultaneously
    to provide that the trade bonus provision will not be applicable to
    the extended term. In order to so provide, the extension must
    include a replacement Exhibit 4 to the Contract with the same terms
    as the original Exhibit 4, but also providing that ``The foregoing
    trade bonus shall not be applicable with respect to the extended
    term of this Contract.'' To illustrate the foregoing, assume that a
    player and Team agree at the time of signing of an extension that
    the trade bonus contained in the original Contract shall not be
    applicable to the extended term. In such case: (y) if the player is
    first traded under the Contract during the remainder of the original
    term of the Contract (i.e., prior to the first year of the extended
    term), then the player's trade bonus shall be calculated based
    solely on the Base Compensation remaining to be earned by the player
    pursuant to the original term of the Contract (and not on any Base
    Compensation payable to the Player in respect of the extended term);
    and (z) if the player is first traded under the Contract at any time
    during the extended term, then the trade bonus would not apply to
    such initial trade or any subsequent trade of the Contract during
    the extended term.
  \item
    In no event shall a trade bonus in a Contract be payable more than
    once.
  \end{enumerate}
\item
  A Player Contract entered into by a player who has eight (8) or more
  Years of Service in the NBA and who has rendered four (4) or more
  Years of Service for the Team entering into such Contract may contain
  a prohibition or limitation of such Team's right to trade such
  Contract to another NBA Team.
\end{enumerate}

\chapter{LIMITATION ON DEFERRED
COMPENSATION}\label{limitation-on-deferred-compensation}

\section{General Limitation.}\label{general-limitation.-1}

No Uniform Player Contract may provide for Deferred Compensation for any
Season that exceeds twenty-five percent (25\%) of the player's
Compensation for such Season.

\section{Attribution.}\label{attribution.}

All Player Contracts shall specify the Season(s) to which any Deferred
Compensation is attributable.

\chapter{TEAM RULES}\label{team-rules}

\section{Establishment of Team
Rules.}\label{establishment-of-team-rules.}

Each Team may maintain or establish rules with which its players shall
comply at all times, whether on or off the playing floor; provided,
however, that such rules are in writing, are reasonable, and do not
violate the provisions of this Agreement or the Uniform Player Contract.

\section{Notice.}\label{notice.}

Any rule(s) established by a Team pursuant to Section 1 above shall be
provided to the Players Association prior to the distribution of such
rule(s) to that Team's players.

\section{Grievances Challenging Team
Rules.}\label{grievances-challenging-team-rules.}

The Players Association may file a Grievance challenging the
reasonableness of a rule established by a Team pursuant to Section 1
above, and the Team's imposition of discipline on a player for a
violation of such rule, within thirty (30) days from the date upon which
the imposition of such discipline on the player became known or
reasonably should have become known to the player. No ruling by the
Grievance Arbitrator finding a Team rule unreasonable may be applied
retroactively as to any player other than the player on whose behalf the
Grievance was filed.

\chapter{RIGHT OF SET-OFF}\label{right-of-set-off}

\section{Set-off Calculation.}\label{set-off-calculation.}

\begin{enumerate}
\def\labelenumi{(\alph{enumi})}
\tightlist
\item
  When a Team (``First Team'') terminates a Player Contract (``First
  Contract'') in circumstances where the First Team, following the
  termination, continues to be liable for unearned Base Compensation
  (i.e., unearned as of the date of the termination) called for by the
  First Contract (including any unearned Deferred Base Compensation),
  the First Team's liability for such unearned Base Compensation shall
  be reduced pro rata by a portion of the compensation earned by the
  player (for services as a player) from any professional basketball
  team or teams (the ``Subsequent Team(s)'') during each Salary Cap Year
  covered by the term of the First Contract (including, but not limited
  to, compensation earned but not paid during such period). The amount
  of the reduction in the First Team's liability (the ``set-off''
  amount) shall be calculated for each Salary Cap Year covered by the
  term of the First Contract as follows: STEP 1: Calculate the total
  compensation earned by the player (for services as a player) from the
  Subsequent Team(s) during the Salary Cap Year. STEP 2: Subtract from
  the result in Step 1 (i) if the player had zero (0) Years of Service
  at the time the First Contract was terminated, the Minimum Annual
  Salary applicable to such player for the Salary Cap Year in which the
  First Contract was terminated, or (ii) if the player had one (1) or
  more Years of Service at the time the First Contract was terminated,
  the Minimum Annual Salary applicable to a player with one (1) Year of
  Service for the Salary Cap Year in which the First Contract was
  terminated. STEP 3: If the result in Step 2 is zero or a negative
  amount, there is no reduction in the First Team's liability for
  unearned Base Compensation in respect of the relevant Salary Cap Year.
  If the result in Step 2 is a positive amount, the reduction in the
  First Team's liability for unearned Base Compensation in respect of
  the relevant Salary Cap Year shall equal fifty percent (50\%) of such
  amount.\\
  Notwithstanding anything in this Article XXVII, a Team shall not be
  required to enforce its set-off right against a player in respect of
  compensation earned by the player from any non-NBA Subsequent Team(s).
  The First Team may require that the player provide the First Team with
  evidence (such as a copy of the player's new contract) of the
  compensation to be earned by the player in connection with his
  services for any Subsequent Team(s).
\item
  For the purposes of this Article XXVII, (i) a ``professional
  basketball team'' shall mean any team in any country that pays money
  or compensation of any kind to a basketball player for rendering
  services to such team (other than a reasonable stipend limited to
  basic living expenses); and (ii) ``compensation'' earned by a player
  shall include all forms of compensation (including, without
  limitation, any non-cash compensation) other than benefits comparable
  to the type of benefits (e.g., medical and dental insurance) provided
  to an NBA player in accordance with Article IV above, travel and
  moving expenses, and any car and housing provided temporarily by a
  professional basketball team to the player during the period of time
  for which the player renders services to such team. Notwithstanding
  anything to the contrary in this Article XXVII, when a player receives
  compensation from a non-NBA Subsequent Team on a net-of-tax basis,
  then for purposes of calculating the amount of set-off to which the
  NBA Team is entitled pursuant to this Article XXVII, such compensation
  from the non-NBA Subsequent Team shall be deemed to equal the
  net-of-tax compensation divided by 0.65 (reflecting a deemed
  thirty-five percent (35\%) tax rate); provided, however, that such
  adjustment to the player's compensation from the non-NBA Subsequent
  Team shall not be made, or shall be modified accordingly, if the
  player can establish that taxes in respect of the player's
  compensation calculated under this provision were not paid, or exceed
  the actual amount paid, by the player's non-NBA Subsequent Team.
\item
  Without limiting any other rights the First Team has, in the event a
  player's compensation is reduced pursuant to this Article XXVII and
  the Team is unable to effect all or a portion of the reduction through
  payroll deductions, the NBA shall have the right to direct any
  Subsequent Team that is an NBA Team to withhold any unrecouped amounts
  from the player's compensation under his new Uniform Player Contract
  and remit such amounts to the First Team. To the extent such remedy is
  insufficient to effect a full recoupment of the set-off amount, the
  NBA and Players Association shall negotiate in good faith to agree on
  such supplemental measures as are appropriate to effect such
  recoupment.
\end{enumerate}

\section{Successive Terminations.}\label{successive-terminations.}

In the event of successive terminations by NBA Teams of Player Contracts
involving the same player, the Team first to terminate shall be entitled
to the right of set-off provided for by this Article XXVII until its
compensation liability has been eliminated in its entirety, and the
right of set-off shall then pass in order to the Team(s) terminating any
subsequent Contract(s).

\section{Deferred Compensation.}\label{deferred-compensation.}

In calculating the amount of set-off to which a Team may be entitled
pursuant to this Article XXVII, the unearned Deferred Compensation
payable to a player for or with respect to a period covered by the
terminated Contract shall be discounted on an annual basis by a
percentage equal to the prime rate reported in the ``Money Rates''
column or any successor column of The Wall Street Journal and in effect
at the time the agreement providing for such Deferred Compensation was
made.

\section{Waiver of Set-off Right.}\label{waiver-of-set-off-right.}

A Team and a player may agree in an amendment to an already-existing
Player Contract to modify or eliminate the set-off right provided in
this Article XXVII, but only pursuant to and to the extent allowed by
Article II, Section 3(l).

\section{Stretched Protected Salary.}\label{stretched-protected-salary.}

\begin{enumerate}
\def\labelenumi{(\alph{enumi})}
\tightlist
\item
  In the event (i) a Team terminates a Player Contract and the payment
  of the player's protected Compensation for any remaining Salary Cap
  Year(s) under the First Contract is stretched in accordance with
  Article II, Section 4(k) (the ``mandatory stretch provision''), and
  (ii) the player subsequently earns compensation from another
  professional basketball team triggering a right of set-off under this
  Article XXVII, the amount of set-off to which the First Team may be
  entitled shall be calculated based on the unearned Base Compensation
  in respect of each Salary Cap Year covered by the term of the First
  Contract as provided in such Contract (and not with regard to how such
  protected Base Compensation amounts are payable to the player pursuant
  to the mandatory stretch provision). The set-off amount in respect of
  each remaining Salary Cap Year(s) under the First Contract in which
  the related unearned Base Compensation is stretched in accordance with
  the mandatory stretch provision shall be allocated such that each of
  the player's stretched protected Compensation payments in respect of
  the applicable Salary Cap Year are reduced on an equal basis over the
  applicable stretch period (i.e., for the first Salary Cap Year with
  respect to which a player's protected Compensation is stretched, over
  the entire stretch period, and for any subsequent Salary Cap Years,
  over the remaining stretch period). In no event shall a Team be
  entitled to set-off under a First Contract in respect of compensation
  earned by a player (for services as a player) from a Subsequent
  Team(s) during a Salary Cap Year occurring after the term of the First
  Contract.
\item
  In the event a player's protected Compensation for any remaining
  Salary Cap Year(s) under a First Contract is stretched for cash
  purposes in accordance with the mandatory stretch provision and the
  First Team also elects to stretch the player's Salary under the First
  Contract for Salary Cap purposes in accordance with Article VII,
  Section 7(d)(5), then the set-off amount in respect of each remaining
  Salary Cap Year(s) covered by the term of the First Contract that is
  stretched for Salary Cap purposes in accordance with Article VII,
  Section 7(d)(5) shall be allocated equally to reduce the player's
  re-attributed Salary amounts over the applicable stretch period in the
  manner described in Section 5(a) above.
\item
  The following examples are for clarity:

  \begin{enumerate}
  \def\labelenumii{(\Alph{enumii})}
  \setcounter{enumii}{23}
  \tightlist
  \item
    Assume a player has protected Compensation of \$3 million in respect
    of the 2017-18 Season and is being paid by the First Team at a rate
    of \$1 million over three Seasons in accordance with the mandatory
    stretch provision and the amount of set-off to which the First Team
    is entitled under this Article XXVII with respect to the 2017-18
    Season is \$600,000, then the \$600,000 set-off amount would be
    allocated to each of the three (3) Seasons at a rate of \$200,000
    per Season and the \$200,000 set-off for each Season would be
    deducted in equal amounts from each of the player's protected
    Compensation payments during such year.
  \item
    Assume: (i) a player has remaining protected Compensation of \$9
    million (\$3 million each for the 2017-18, 2018-19 and 2019-20
    Seasons, respectively); (ii) the First Team requested waivers on the
    player on September 5, 2017 and his Contract was terminated on
    September 7, 2017; (iii) the player signed a player contract with
    Subsequent Team A that provides for a term covering the 2017-18
    through 2018-19 Seasons; and (iv) the set-off amount to which the
    First Team is entitled under this Article XXVII in respect of the
    player's contract with Subsequent Team A is as follows: \$600,000 in
    respect of the 2017-18 Salary Cap Year and \$600,000 in respect of
    the 2018-19 Salary Cap Year. (There would be no set-off amount under
    the First Contract in respect of the 2019-20 Salary Cap Year given
    these facts because the term of the contract with Subsequent Team A
    does not cover the 2019-20 Season.) Under these facts: (i) with
    respect to the 2017-18 Season, the player's \$3 million of protected
    Compensation under the First Contract would be reduced by the
    applicable \$600,000 set-off amount and his reduced protected
    Compensation amount of \$2.4 million would be payable in accordance
    with the payment schedule set forth in the First Contract; (ii) with
    respect to the 2018-19 Season, the player's remaining protected
    Compensation of \$6 million in respect of the 2018-19 and 2019-20
    Seasons would be paid by the First Team at a rate of \$1.2 million
    over five Seasons in accordance with the mandatory stretch
    provision, and his aggregate remaining protected Compensation would
    be reduced by the applicable \$600,000 set-off amount such that each
    of the player's stretched protected Compensation payments in respect
    of the 2018-19 Season are reduced on an equal basis over the
    five-year stretch period (i.e., \$120,000 per year over the 2018-19
    through 2022-23 Salary Cap Years).
  \item
    Assume the same facts as in clause (Y) above and that on October 1,
    2019, the player signed a player contract with Subsequent Team B
    covering the 2019-20 Season and the set-off amount to which the
    First Team is entitled under this Article XXVII is \$500,000. In
    such case, the player's aggregate remaining protected Compensation
    would be further reduced by the additional \$500,000 set-off amount
    such that the player's remaining stretched protected Compensation
    payments in respect of the 2018-19 and 2019-20 Seasons are reduced
    on an equal basis over the remaining four-year stretch period (i.e.,
    \$125,000 per year over the 2019-20 through 2022-23 Salary Cap
    Years).
  \end{enumerate}
\end{enumerate}

\chapter{MEDIA RIGHTS}\label{media-rights}

\section{League Rights.}\label{league-rights.}

The Players Association agrees that the NBA, all League- related
entities (including, but not limited to, NBA Properties, Inc. and NBA
Media Ventures, LLC) that generate BRI, and NBA Teams have the right
during and after the term of this Agreement to use, exhibit, distribute,
or license any performance by the players, under this Agreement or the
Uniform Player Contract, in any or all media, formats or forms of
exhibition and distribution, whether analog, digital or other, now known
or hereafter developed, including, but not limited to, print, tape,
disc, computer file, radio, television, motion pictures, other
audio-visual and audio works, Internet, broadband platforms, mobile
platforms, applications, and other distributions platforms
(collectively, ``Media'').

\section{No Suit.}\label{no-suit.}

The Players Association, for itself and present and future NBA players,
covenants not to sue (or finance any suit against) the NBA, all League
related entities (including NBA Properties, Inc. and NBA Media Ventures,
LLC) that generate BRI, and all NBA Teams, or, any of their respective
past, present and future owners (direct and indirect) acting in their
capacity as owners of any of the foregoing entities, officers,
directors, trustees, employees, agents, attorneys, licensees,
successors, heirs, administrators, executors and assigns, with respect
to the use, exhibition, distribution, or license, in any or all Media,
of any performances by any player rendered under this Agreement or prior
collective bargaining agreements, or under Player Contracts made
pursuant thereto; provided, however, that this Section 2 shall not apply
to any Endorsement, as defined in Section 3 below, any Unauthorized
Sponsor Promotion, as defined in Paragraph 14(c) of the Uniform Player
Contract, or any action of the Players Association pursuant to Section 3
(f) below.

\section{Unauthorized Endorsement/Sponsor
Promotion.}\label{unauthorized-endorsementsponsor-promotion.}

\begin{enumerate}
\def\labelenumi{(\alph{enumi})}
\tightlist
\item
  Section 1 above does not confer any right or authority for the NBA,
  any League related entity or any NBA Team to (i) use, or authorize any
  third party to use, any performance by a player in any way that
  constitutes an unauthorized endorsement by such player of a third
  party brand, product or service (``Endorsement''), or (ii) authorize
  any third party to use any performance by a player in any way that
  constitutes an Unauthorized Sponsor Promotion as defined in Paragraph
  14(c) of the Uniform Player Contract.
\item
  For purposes of clarity, and without limitation: (i) it shall not be
  an Endorsement for the NBA, a League-related entity or an NBA Team to
  use, or authorize others to use, including, without limitation, in
  third party advertising and promotional materials, footage and
  photographs of a player's participation in NBA games or other NBA
  events that do not unduly focus on, feature, or highlight, such player
  in a manner that leads the reasonable consumer to believe that such
  player is a spokesman for, or promoter of, a third-party commercial
  product or service; provided that the preceding sentence is
  independent of and is not relevant to determining whether a use is or
  is not an Unauthorized Sponsor Promotion; and (ii) any use of a
  player's Player Attributes that has been expressly authorized by the
  player (not including the Uniform Player Contract) shall not be an
  unauthorized Endorsement or an Unauthorized Sponsor Promotion.
\item
  Any dispute regarding whether a use of any performance by a player is
  or is not an Unauthorized Sponsor Promotion shall be determined by the
  expedited System Arbitration process described in Paragraph 14(d) of
  the Uniform Player Contract.
\item
  For purposes of clarity, nothing in this Agreement or the Uniform
  Player Contract shall limit the rights of the NBA, all League-related
  entities that generate BRI, and NBA Teams to provide, and authorize
  others to provide, advertising and promotional opportunities within
  NBA games or NBA or Team events and NBA-related or Team- related
  content; it being understood that nothing in this sentence is intended
  to authorize the NBA, any League-related entity or any NBA Team to
  use, or authorize any third party to use, any Player Attributes in any
  way (w) that constitutes an unauthorized Endorsement, (x) in the
  creative elements incorporated into such advertising executions that
  constitute an Unauthorized Sponsor Promotion, or (y) in the creative
  elements in promotional opportunities that are not Promotional
  Enhancements that are Unauthorized Sponsor Promotion. For purposes of
  the foregoing, examples of ``advertising'' include 30-second
  commercials, video pre-rolls and courtside signage.
\item
  Nothing in Section 3(d)(x) or (y) above shall limit the right of a
  telecaster or distributor of NBA games, NBA or Team events or
  NBA-related or Team-related content to use, or authorize others to
  use, third party Promotional Enhancements in telecasts or other
  distribution of such games, events or content in accordance with this
  Section 3(e). For purposes of this Section 3(e), ``Promotional
  Enhancements'' means: (i) virtual images, graphics and/or text that
  are superimposed on the video and/or audio depiction of the NBA game,
  NBA or Team event or NBA-related or Team-related content; (ii)
  non-virtual signage or other physical displays otherwise visible in
  the telecast or other distribution of the NBA game, NBA or Team event
  or NBA-related or Team-related content (for purposes of clarity,
  clauses (i) and (ii) above do not include still images except in game
  and program telecasts); and (iii) other promotional opportunities for
  Current Telecasters (as defined in Paragraph 14(d) of the Uniform
  Player Contract) as currently permitted under the contracts referenced
  in Paragraph 14(d) of the Uniform Player Contract for the terms
  thereof. Examples of Promotional Enhancements include branded
  backboard slide- outs, branded feature trackers, sponsored starting
  lineups, branded virtual lineups, virtual courtside signage, virtual
  court signage, branded statistical presentations, studio show
  backdrops, branded halftime desk signage, a sponsored ``Top Plays''
  feature and a sponsored ``audio drop-in'' mention. Creative elements
  incorporated into virtual signage Promotional Enhancements are not
  authorized under this Section if they otherwise are Unauthorized
  Sponsor Promotion.
\item
  Notwithstanding the foregoing, in addition to any other rights the
  Players Association may have, (A) if a telecaster or other distributor
  of NBA games, NBA or Team events or NBA-related or Team-related
  content uses, or authorizes others to use, Player Attributes in
  creative elements within promotional opportunities in telecasts or
  other distribution of such games, events or content (i) in a manner
  that (x) is not covered by Section 3(e)(iii) above, and (y) unduly
  promotes the products or services of a sponsor, and (ii) the promotion
  of the sponsor's products or services within such promotional
  opportunity is more prominent than the NBA content, to which it
  relates, taken as a whole, then the Players Association shall notify
  the NBA in writing, and (B) the NBA shall have a period of fifteen
  (15) days to cause the telecaster or distributor to cease or modify
  such creative elements (``Cure''). If the NBA fails to Cure pursuant
  to the preceding sentence, then the Players Association may sue the
  NBA for any resulting damages to the Players Association's commercial
  group licensing business, with the NBA responsible for the violation
  and such damages even if the NBA did not authorize such promotional
  opportunity.
\item
  For purposes of clarity, nothing contained in this Article XXVIII or
  in Paragraph 14 of the Uniform Player Contract shall prohibit the
  inclusion of a sponsor's name and/or logo on a jersey patch, and any
  depiction of a player wearing a jersey that includes such a jersey
  patch shall not, by reason of the jersey patch alone, constitute an
  unauthorized Endorsement, an Unauthorized Sponsor Promotion, or a
  violation of Section 3(f) above.
\end{enumerate}

\section{Reservation of Rights.}\label{reservation-of-rights.}

The Players Association expressly reserves its rights to bargain
collectively on the subject described in Section 1 above at the
expiration of this Agreement. Such reservation shall not, however,
preclude the NBA from contending that the subject described in Section 1
above is not a mandatory subject of collective bargaining. The right of
the NBA, League related entities, and NBA Teams described in Section 1
above is in addition to, and shall not limit nor be deemed to limit,
derogate from or otherwise prejudice, any and all rights that any one or
all of them have heretofore possessed or enjoyed, do now possess or
enjoy or may hereafter possess or enjoy.

\chapter{MISCELLANEOUS}\label{miscellaneous}

\section{Active Roster Size.}\label{active-roster-size.}

Each Team agrees to have twelve (12) or thirteen (13) players on its
Active List and to have a minimum of eight (8) players on the bench for
all Regular Season games. Notwithstanding the foregoing, any Team may
from time to time as appropriate, but for no more than two (2)
consecutive weeks at a time during the Regular Season, have eleven (11)
players on its Active List. During the period from the day following the
last day of the Regular Season (or, for Teams that qualify for the
playoffs, the day following the Team's last playoff game), until the day
before the first day of the following Regular Season, the maximum number
of players (including Two-Way Players) that a Team may carry on its
Active List shall be twenty (20). Players on the Inactive List shall be
transferred to the Active List on the day following the last day of the
Regular Season (or, for Teams that qualify for the playoffs, the day
following the Team's last playoff game).

\section{Inactive Roster.}\label{inactive-roster.}

Each Team agrees to have at least two (2) players on its Inactive List
for all Regular Season games. Notwithstanding the foregoing, (i) any
Team that has eleven (11) or twelve (12) players on its Active List may
from time to time as appropriate, but for no more than two (2)
consecutive weeks at any time during the Regular Season, have one (1)
player on its Inactive List, and (ii) any Team that has thirteen (13)
players on its Active List may have one (1) player on its Inactive List,
and may, from time to time as appropriate, but for no more than two (2)
consecutive weeks at any time during the Regular Season, have zero (0)
players on its Inactive List. For each Two-Way Player that a Team places
on the Active List or Inactive List, the minimum Inactive List
requirements set forth in this Section 2 shall be increased by one (1)
for that Team.

\section{Two-Way Roster.}\label{two-way-roster.}

\begin{enumerate}
\def\labelenumi{(\alph{enumi})}
\tightlist
\item
  A Two-Way Player shall be placed on his Team's (i) Active List or
  Inactive List (as applicable) while the Two-Way Player is providing
  services to the NBA Team, and (ii) Two-Way List while the Two-Way
  Player is providing services to an NBADL team (or, during a period
  prior to the start of NBADL training camp or after the completion of
  the NBADL playoffs, while the player is not providing services to the
  NBA Team).
\item
  The Two-Way List shall only exist between the first day of the NBA
  Regular Season and the last day of the NBA Regular Season (or, for
  Teams that qualify for the playoffs, the day of the Team's last
  playoff game). For Teams that do not qualify for the playoffs, all
  players on the Two-Way List of any such Team shall be transferred to
  the Active List on the day following the last day of the NBA Regular
  Season. For Teams that qualify for the playoffs, all players on the
  Two-Way List of any such Team shall be transferred to the Active List
  on the day following the Team's last playoff game.
\item
  A Two-Way Player is not eligible to be designated on an NBA Team's
  playoff roster or participate in NBA playoff games, but is permitted
  to travel and practice with the Team and remain on the Team's Inactive
  List during the NBA playoffs; provided, however, that subject to
  Section 4 below, a player who was previously a Two-Way Player but who,
  prior to the start of the Team's last Regular Season game, either
  signs a Standard NBA Contract in accordance with Article II, Section
  11(h) or has his Two-Way Contract converted by the Team to a Standard
  NBA Contract pursuant to Article II, Section 11(g), is eligible to be
  designated on an NBA Team's playoff roster and participate in playoff
  games.
\end{enumerate}

\section{Playoff Eligibility Waiver
Deadline.}\label{playoff-eligibility-waiver-deadline.}

Any player (including any Two-Way Player) with respect to whom a request
for waivers has been made after March 1 is not eligible to participate
in playoff games during the then-current Season unless the player has
been acquired by a Team whose Active List is reduced to eight (8)
players due to injury or illness. For clarity, the March 1 deadline
occurs at the end of the day on March 1. Accordingly, if a waiver
request is made in respect of a player any time on or before March 1 at
11:59 p.m. eastern time, that player will remain eligible for the
playoffs.

\section{Minimum League-Wide Roster.}\label{minimum-league-wide-roster.}

Beginning with the 2017-18 Season, if for two consecutive Regular
Seasons (e.g., 2017-18 and 2018-19 or 2018-19 and 2019-20, but not
2017-18 and 2019-20), NBA Teams in the aggregate employ an average of
less than fourteen and one-half (14.5) players (excluding Two-Way
Players) per Team, then for each Regular Season covered by this
Agreement that follows such consecutive two-year period, the minimum
Inactive List requirements set forth in the first two sentences of
Section 2 above shall be increased by one (1) (``League-Wide Roster
Increase''), so that Teams would for the remaining Seasons of the term
of this Agreement be required to employ fifteen (15) players per team
(excluding Two-Way Players) except as permitted in accordance with
Sections 1 and 2 above. The foregoing rule shall be measured following
each Regular Season as follows: STEP 1: For each player signed to a
Player Contract (including a Rest-of-Season or 10-Day Contract, but
excluding a Two-Way Contract) during a Regular Season, determine the
number of days during such Regular Season that such player was carried
on his Team's Active List or Inactive List (hereinafter ``Duty Days'').
STEP 2: Determine the total Duty Days for all players for such Regular
Season by adding together the results for each player from Step 1. STEP
3: Multiply (x) the number of NBA Teams that played games during the
applicable Regular Season, by (y) 2,480 days. STEP 4: If, for two
consecutive Regular Seasons beginning with the 2017-18 Season, the
result in Step 2 above is less than the result in Step 3 above, then the
League-Wide Roster Increase will be triggered.

\section{Playing Rules and
Officiating.}\label{playing-rules-and-officiating.}

\begin{enumerate}
\def\labelenumi{(\alph{enumi})}
\tightlist
\item
  Up to four (4) representatives of the Players Association, three (3)
  of whom shall be active or recently retired players selected by the
  Players Association, shall be permitted to attend the meetings of and
  have a vote on the NBA Competition Committee with respect to issues
  relating to the NBA playing rules and officiating.
\item
  The Players Association may, on behalf of the players, submit to the
  Commissioner monthly reports as to the conduct of referees, including
  identifying individual referees by name. The NBA will consider, but is
  not required to act, on such reports.
\item
  Upon a request from the Players Association, representatives of the
  NBA Basketball Operations and Referee Operations Departments shall
  meet annually with the Player Association and/or players to discuss
  issues relating to NBA playing rules and officiating. The NBA will
  request that representatives from the National Basketball Referees
  Association, including current referees, attend any such meeting.
\end{enumerate}

\section{Playoffs.}\label{playoffs.}

\begin{enumerate}
\def\labelenumi{(\alph{enumi})}
\tightlist
\item
  The number of Teams participating in the playoffs shall equal sixteen
  (16). Notwithstanding the foregoing, the NBA shall have the right to
  increase the number of Teams participating in the playoffs.
\item
  Each round of the playoffs shall be played in a best-of-seven-games
  format.
\end{enumerate}

\section{Game Tickets.}\label{game-tickets.}

\begin{enumerate}
\def\labelenumi{(\alph{enumi})}
\tightlist
\item
  In the event that a Team provides complimentary tickets to its
  players, the Team may provide up to four (4) tickets per home game and
  up to two (2) tickets per road game. Teams may sell additional tickets
  to players, provided that such sales shall be no less than the season
  ticket holder prices for the applicable game. Seat locations for
  complimentary tickets provided by a Team under this Section 8 must be
  in the lower bowl of the arena and may not be on the floor (i.e., in
  front of the risers or permanent bowl seating or inside the dashers)
  or in a luxury suite (i.e., a private, enclosed area that is separate
  from the arena bowl, including, but not limited to, traditional
  enclosed suites, event level (bunker) suites, and party suites).
\item
  In the event that a Team provides complimentary tickets to its players
  for road games, each player on the roster who travels with the Team
  shall be provided the same number of tickets (i.e., either zero (0),
  one (1) or two (2).
\item
  Teams are prohibited from providing tickets to players on other Teams,
  and players are only permitted to accept tickets from their own Team.
\item
  Any player found to be re-selling complimentary or reduced-price
  tickets will be prohibited from subsequently receiving such tickets
  from his Team.
\item
  In the event that a Team provides home-game tickets to its players,
  seat locations must be allocated to players based on seniority, with
  the most senior players (based on years of NBA service) receiving the
  most favorable seat locations.
\item
  NBA Teams shall provide four (4) tickets to authorized representatives
  of the Players Association to any home game at box office prices,
  provided notice of such request is given at least forty-eight (48)
  hours before the game.
\item
  Each Team agrees to provide retired players with three (3) or more
  years of NBA service with the opportunity to purchase two (2) tickets
  at box office prices to its NBA home games, and to hold such tickets
  for such players, provided tickets are available and the retired
  players provide the Team with forty-eight (48) hours advance notice of
  their desire for such tickets.
\end{enumerate}

\section{League Pass.}\label{league-pass.}

Any player who is under a Uniform Player Contract, with the exception of
10-Day Contracts or Two-Way Contracts, shall receive a free League Pass
Broadband account in each Season of his Player Contract.

\section{Release for Fighting.}\label{release-for-fighting.}

Each NBA Team (hereinafter ``such Team'') hereby releases and waives
every claim it may have against any player employed by other NBA Teams
for injuries sustained by any player in the employ of such Team which
arise out of, or in connection with, any fighting or other form of
violent and/or unsportsmanlike conduct during the course of any
Exhibition, Regular Season, and/or Playoff game.

\section{Limitation on Player
Ownership.}\label{limitation-on-player-ownership.}

During the term of this Agreement, no NBA player may acquire or hold a
direct or indirect interest in the ownership of any NBA Team or in any
company or entity, whether privately or publicly owned, that owns any
interest in any NBA Team; provided, however, that any player may have an
ownership of publicly-traded securities constituting less than five
percent (5\%) of the ownership interests in a company or entity that
directly or indirectly owns an NBA Team.

\section{Nondisclosure.}\label{nondisclosure.}

The parties agree that (a) the economic terms of any individual Uniform
Player Contract entered into by a Team and a player, and (b) any
information contained in, or disclosed to the Players Association in
connection with an Audit Report, Draft Audit Report, Interim Audit
Report, Interim Escrow Audit Report, BRI Report, Escrow Schedule, or
Notice to Escrow Agent, shall not be disclosed to the media by (i) the
NBA, its Teams, or their respective employees, or (ii) the Players
Association, NBA players, or their respective employees, agents, or
representatives.

\section{Implementation of
Agreement.}\label{implementation-of-agreement.}

\begin{enumerate}
\def\labelenumi{(\alph{enumi})}
\tightlist
\item
  The NBA and the Players Association will use their respective best
  efforts to have NBA Teams and NBA players comply with the terms and
  provisions of this Agreement.
\item
  The NBA and the Players Association shall use their respective best
  efforts and take all reasonable steps to cooperate to defend the
  enforceability of this Agreement against any challenge thereto.
\end{enumerate}

\section{Additional Canadian
Provisions.}\label{additional-canadian-provisions.}

\begin{enumerate}
\def\labelenumi{(\alph{enumi})}
\tightlist
\item
  The bases upon which a player may be disciplined or discharged or a
  Player Contract terminated, as set forth in this Agreement and/or in
  the Uniform Player Contract, shall constitute just and reasonable
  cause within the meaning of any applicable Canadian law or statute
  (federal or provincial) and, to the extent this Agreement or the
  Uniform Player Contract provides specific penalties for such conduct,
  those penalties shall apply.
\item
  During the term of this Agreement, the NBA and Players Association
  shall consult regularly about issues relating to the workplace which
  affect the parties or any player bound by this Agreement.
\item
  If and to the extent Sections 48 and 49 of the Ontario Labour
  Relations Act are or may be found applicable to this Agreement, the
  parties agree that the provisions thereof shall apply only to disputes
  between the Toronto Raptors and players for the Toronto Raptors.
  Furthermore, the parties agree and acknowledge that any termination
  and severance benefits provided to players pursuant to this Agreement
  (including the provisions of Player Contracts that provide, in certain
  circumstances, for the continued payment of Salary to a player
  following the termination of a Player Contract) constitute and/or
  shall be deemed to constitute a greater right or benefit to the Player
  pursuant to Section 5(2) of the Employment Standards Act, 2000
  (Ontario) and the provisions of Sections 54-66 of such Act do not
  apply.
\item
  The parties acknowledge and agree that a player employed by an NBA
  Team pursuant to the provisions of a Uniform Player Contract, a 10-Day
  Contract, a Rest-of-Season Contract, or a Two-Way Contract is and/or
  shall be deemed to be an ``employee hired on the basis that his
  employment is to terminate on the expiry of a definite term or the
  completion of a specific task'' within the meaning of paragraph 1 of
  Section 2(1) of Ontario Regulation 288/01 under the Ontario Employment
  Standards Act, 2000, so as to render inapplicable to NBA players the
  provisions of Sections 54-62 of such Act.
\item
  The parties acknowledge and agree that the severance benefits provided
  to players pursuant to this Agreement (including the provisions of
  Player Contracts that provide, in certain circumstances, for the
  continued payment of Salary to a player following the termination of a
  Player Contract) constitute and/or shall be deemed to constitute a
  settlement binding on the player within the meaning of Section 6 of
  the Ontario Employment Standards Act, 2000, and/or ``an amount paid to
  an employee for loss of employment under a provision of an employment
  contract based upon length of employment, length of service or
  seniority'' within the meaning of paragraph 2 of Section 65(8) of the
  Ontario Employment Standards Act, 2000, so as to render inapplicable
  to NBA players the provisions of Sections 63-66 of such Act.
\item
  Upon the NBA's request, the Players Association shall cooperate with
  the NBA in a reasonable manner in connection with any effort the NBA
  may make to seek an exemption from any Canadian (federal or
  provincial) law or regulation affecting the employment relationship
  that is inconsistent with the provisions of this Agreement or any
  other agreement between the Players Association and the NBA (or NBA
  Properties) or between any player and any NBA Team.
\item
  All players employed by NBA Teams shall be paid in U.S. dollars,
  regardless of where such Teams are located.
\end{enumerate}

\section{Gate Reports.}\label{gate-reports.}

The NBA shall provide the Players Association with reports regarding
each Team's gate receipts and paid attendance (including season ticket
sale summaries) as of the date two (2) weeks prior to the date of each
report. The reports shall be provided on or before the following dates
each Season: December 31; February 28; April 30; and July 31; provided,
however, that with respect to season ticket sale summaries, the NBA
shall not provide a provide a report on or before December 31 and shall
instead provide a report on or before September 30.

\section{League-Wide Public Service
Campaigns.}\label{league-wide-public-service-campaigns.}

The NBA will notify the Players Association of any league-wide public
service campaign to be implemented by the NBA at least two (2) weeks
before any player is requested to appear on behalf of such campaign.

\chapter{NO-STRIKE AND NO-LOCKOUT PROVISIONS AND OTHER
UNDERTAKINGS}\label{no-strike-and-no-lockout-provisions-and-other-undertakings}

\section{No Strike.}\label{no-strike.}

During the term of this Agreement, neither the Players Association nor
its members shall engage in any strikes, cessations or stoppages of
work, or any other similar interference with the operations of the NBA
or any of its Teams. Notwithstanding the foregoing, nothing in this
Section 1 shall impair the rights accorded the Players Association by
Article XXXIX, Section 3 (Termination by Players
Association/Anti-Collusion) or Section 6 (Mutual Right of Termination).

\section{No Lockout.}\label{no-lockout.}

During the term of this Agreement, neither the NBA nor its Teams shall
engage in any lockouts, cessations or stoppages of work or any other
similar interference with the employment of NBA players by NBA Teams.
Notwithstanding the foregoing, nothing in this Section 2 shall impair
the rights accorded the NBA by Article XXXIX, Section 4 (Termination by
NBA/National TV Revenues), Section 5 (Termination by NBA/Force Majeure),
or Section 6 (Mutual Right of Termination).

\section{No Breach of Player
Contracts.}\label{no-breach-of-player-contracts.}

The Players Association agrees that it will not engage in any concerted
activities to breach, induce the breach of, or threaten to breach or
induce the breach of, any Player Contract.

\section{Best Efforts of Players
Association.}\label{best-efforts-of-players-association.}

The Players Association will use its best efforts: (a) to prevent each
player from rendering, or threatening to render, services as a
professional basketball player for another professional basketball team
during the term of a Player Contract between such player and the Team
for which he plays (except as said Player Contract may be assigned,
sold, or transferred in accordance with the provisions of such Player
Contract or this Agreement); (b) to prevent each player from refusing,
or threatening to refuse, to participate in any scheduled Exhibition
game, Regular Season game, All-Star Game, Rookie-Sophomore Game,
All-Star Skills Competition, or Playoff game; (c) to prevent each player
from refusing, or threatening to refuse, to report, within the time
required, to a team in the NBA Development League (``NBADL'') when the
player has been assigned to or is providing D-League Two-Way Service
with an NBADL Team in accordance with the provisions of this Agreement,
and to prevent each such player from refusing, or threatening to refuse,
to participate in any scheduled NBADL game; (d) to prevent each player
from otherwise breaching, or threatening to breach, his Player Contract;
and (e) to prevent each player from making any demand upon the NBA or
any of its Teams, including, but not limited to, a demand (accompanied
by threats that the player will render services as a professional
basketball player for another professional basketball team during the
term of his Player Contract) that such Player Contract be renegotiated
during the term thereof; provided, however, that this provision is not
intended to prevent any player from entering into negotiations with a
Team, in accordance with Article VII, with respect to the compensation
to be paid to said player for the Season(s) following the last playing
Season covered by any Player Contract, or renewal or extension thereof.

\section{No Discrimination.}\label{no-discrimination.}

Neither the NBA, any Team nor the Players Association shall discriminate
in the interpretation or application of this Agreement against or in
favor of any Player because of religion, race, national origin, sexual
orientation or activity or lack of activity on behalf of the Players
Association.

\chapter{GRIEVANCE AND ARBITRATION PROCEDURE AND SPECIAL PROCEDURES WITH
RESPECT TO DISPUTES INVOLVING PLAYER
DISCIPLINE}\label{grievance-and-arbitration-procedure-and-special-procedures-with-respect-to-disputes-involving-player-discipline}

\chaptermark{GRIEVANCE AND ARBITRATION PROCEDURE AND SPECIAL \ldots}

\section{Scope.}\label{scope.}

\begin{enumerate}
\def\labelenumi{(\alph{enumi})}
\item
  \begin{enumerate}
  \def\labelenumii{(\roman{enumii})}
  \tightlist
  \item
    Except as provided otherwise by this Agreement or by paragraph 9 of
    the Uniform Player Contract, the Grievance Arbitrator shall have
    exclusive jurisdiction to determine, in accordance with procedures
    set forth in this Article XXXI, any and all disputes involving the
    interpretation or application of, or compliance with, the provisions
    of this Agreement or the provisions of a Player Contract, including
    any dispute concerning the validity of a Player Contract or any
    dispute arising under the Joint NBA/NBPA Policy on Domestic
    Violence, Sexual Assault, and Child Abuse. Any such dispute subject
    to the exclusive jurisdiction of the Grievance Arbitrator shall
    hereinafter be referred to as a ``Grievance.''
  \item
    The Grievance Arbitrator shall also have jurisdiction to resolve
    disputes arising under the Agreement and Declaration of Trust
    Establishing the National Basketball Players Association/National
    Basketball Association Supplemental Benefit Plan, the Agreement and
    Declaration of Trust Establishing the National Basketball Players
    Association/National Basketball Association Labor-Management
    Cooperation and Education Trust, and any agreement and declaration
    of trust establishing the New VEBA provided in Article IV, Section 4
    in accordance with the provisions of such agreements and
    declarations of trust. In connection with the resolution of such
    disputes, to the extent there is any conflict between the provisions
    of such agreements and declarations of trust and the provisions of
    this Agreement, the provisions of such agreements and declarations
    of trust shall control.
  \end{enumerate}
\item
  Notwithstanding the provisions of Section 1(a) above:

  \begin{enumerate}
  \def\labelenumii{(\roman{enumii})}
  \tightlist
  \item
    Disputes arising under Articles I, II, VII, VIII, X, XI, XII, XIII,
    XIV, XV, XVI, XXXVII, XXXIX, and XL, as well as disputes arising
    under Article XXVIII and Paragraph 14 of the Uniform Player Contract
    regarding an Unauthorized Sponsor Promotion (as that term is defined
    in Paragraph 14(c) of the Uniform Player contract) shall (except as
    otherwise specifically provided by Article VII, Section 3(d)(5)) be
    determined by the System Arbitrator provided for in Article XXXII;
    and
  \item
    Disputes involving (A) a fine or suspension imposed upon a player by
    the Commissioner (or his designee) for conduct on the playing court
    or in-game conduct involving another player (as those terms are
    defined in Section 9(c) below), or (B) action taken by the
    Commissioner (or his designee) concerning the preservation of the
    integrity of, or maintenance of public confidence in, the game of
    basketball, shall be resolved in accordance with the provisions set
    forth in Section 9 below.
  \end{enumerate}
\end{enumerate}

\section{Initiation.}\label{initiation.}

\begin{enumerate}
\def\labelenumi{(\alph{enumi})}
\tightlist
\item
  Grievances may be initiated, as set forth below, by a player, a Team,
  the NBA, or the Players Association, except that the Players
  Association may not initiate a Grievance involving player discipline
  without the approval of the player(s) concerned.
\item
  No party may initiate a Grievance until and unless it has first
  discussed the matter with the party or parties against whom the
  Grievance is to be initiated in an attempt to settle it.
\item
  A Grievance must be initiated, in accordance with the provisions of
  Section 2(d) below, within thirty (30) days from the date of the
  occurrence upon which the Grievance is based, or within thirty (30)
  days from the date upon which the facts of the matter became known or
  reasonably should have become known to the party initiating the
  Grievance, whichever is later.
\item
  Subject to the provisions of Sections 2(a)-(c) above: (i) a player or
  the Players Association may initiate a Grievance (A) against the NBA
  by filing written notice thereof with the NBA, and (B) against a Team,
  by filing written notice thereof with the Team and the NBA; (ii) a
  Team may initiate a Grievance by filing written notice thereof with
  the Players Association and furnishing copies of such notice to the
  player(s) involved and to the NBA; and (iii) the NBA may initiate a
  Grievance by filing written notice thereof with the Players
  Association and furnishing copies of such notice to the player(s) and
  Team(s) involved. Any such notice shall expressly state that the party
  is initiating a Grievance pursuant to Article XXXI, Section 2.
\end{enumerate}

\section{Pre-Hearing Motions.}\label{pre-hearing-motions.}

\begin{enumerate}
\def\labelenumi{(\alph{enumi})}
\tightlist
\item
  A party to a Grievance may file a pre-hearing motion with the
  Grievance Arbitrator under this Section 3 if that party is seeking to
  have the Grievance dismissed (i) because the Grievance Arbitrator does
  not have jurisdiction to hear the matter under Section 1 above, or
  (ii) for the opposing party's failure to properly initiate a Grievance
  or file the Grievance on a timely basis pursuant to Section 2 above.
\item
  Upon the filing of a motion under Section 3(a) above, the parties will
  schedule a conference call with the Grievance Arbitrator for the
  purposes of setting a schedule for the motion, including a date for
  the opposing party's opposition brief and a date for oral argument
  before the Grievance Arbitrator. Oral argument under this Section 3(b)
  shall be conducted by teleconference.
\item
  The opposing party may request a factual hearing on the motion in its
  opposition brief, but cannot request a factual hearing on the
  underlying merits of the Grievance. If the Grievance Arbitrator grants
  the request for a factual hearing, the hearing shall comply with the
  requirements of Sections 4, 5 and 6 below.
\item
  The Grievance Arbitrator shall render a decision on the motion
  (including any appropriate award) as soon as practicable and the
  decision shall be accompanied by a written opinion, or, if both the
  NBA and the Players Association agree, the written opinion may follow
  within a reasonable time thereafter. In no event shall the award and
  written opinion be issued more than thirty (30) days following the
  date of the oral argument or, where applicable, following the date
  designated by the Grievance Arbitrator for the submission of
  post-argument briefs. If the decision is dispositive, the award shall
  constitute full, final and complete disposition of the Grievance, and
  shall be binding upon the player(s) and Team(s) involved and the
  parties to this Agreement.
\item
  The procedure set forth in this Section 3 shall not be applicable to
  disputes with respect to which the Expedited Procedure set forth in
  Section 13 is properly invoked by either the NBA or the Players
  Association; provided, however, that this Section 3(e) shall not
  preclude any party from asserting, in a proceeding to which such
  Expedited Procedure applies, that the Grievance should be dismissed
  (i) because the Grievance Arbitrator does not have jurisdiction to
  hear the matter under Section 1 above, or (ii) for the opposing
  party's failure to properly initiate a Grievance or file the Grievance
  on a timely basis pursuant to Section 2 above.
\item
  If a pre-hearing motion to dismiss is denied, the NBA and the Players
  Association shall schedule a hearing promptly with respect to the
  merits of the Grievance involved.
\end{enumerate}

\section{Hearings.}\label{hearings.}

\begin{enumerate}
\def\labelenumi{(\alph{enumi})}
\tightlist
\item
  Upon at least thirty (30) days' written notice to the other side, the
  NBA and the Players Association may arrange to have a hearing
  scheduled on a date that is mutually convenient to the parties to the
  dispute, the NBA, the Players Association, and the Grievance
  Arbitrator; provided, however, that if the NBA and the Players
  Association cannot agree on a hearing date, the Grievance Arbitrator
  shall set a reasonable hearing date that follows the expiration of the
  thirty-day notice period. Only the NBA and the Players Association may
  schedule or postpone hearings before the Grievance Arbitrator.
\item
  Notwithstanding the provisions of Section 4(a) above, during each
  Salary Cap Year covered by this Agreement, the Players Association and
  the NBA shall each have the right, upon a showing of need, to have two
  (2) Grievances scheduled for hearing on or after the tenth day
  following service of the notice provided for by Section 4(a) above.
  The provisions of this Section 4(b) shall not limit or otherwise
  affect the rights of the NBA or the Players Association pursuant to
  Section 13 below.
\item
  If a Grievance is scheduled for hearing under this Article XXXI, and
  the hearing date is thereafter postponed at the request of either the
  NBA or the Players Association, the postponement fee (if any) of the
  Grievance Arbitrator will be borne by the party requesting the
  postponement, unless that party objects and the Grievance Arbitrator
  finds that the request for such postponement was for good cause.
  Should good cause be found, the parties will share any postponement
  fee equally.
\item
  In any Grievance matter, neither the NBA nor the Players Association
  may request or be granted more than one (1) postponement of a hearing
  previously scheduled under this Article XXXI. If a party which has
  been granted a postponement of a hearing fails to attend a
  subsequently scheduled hearing in the same Grievance matter, the
  Grievance shall be resolved against that party.
\item
  If (i) a hearing of a Grievance is not scheduled to take place within
  one (1) year from the initiation of the Grievance, or (ii) in the
  circumstance where the initial date set for the hearing has been
  postponed, if a second hearing in that Grievance is not scheduled to
  take place within two (2) years from the initiation of the Grievance,
  then the Grievance shall, upon written notice to the party or parties
  filing such Grievance, be deemed to have been dismissed with prejudice
  as of the thirtieth (30th) day following the delivery of such notice
  without the need for a hearing or for any action to be taken or
  decision to be issued by the Grievance Arbitrator, unless, upon
  written application made by the party or parties filing such Grievance
  within such thirty-day period, the Grievance Arbitrator determines
  that dismissal of the Grievance without prejudice would be unjust.
\item
  For purposes of computing time under this Section 4, the time shall be
  tolled during any period when there is no Grievance Arbitrator or when
  the grieving party has been unable to schedule a hearing (after making
  efforts to do so) because the Grievance Arbitrator is unavailable.
\item
  Hearings before the Grievance Arbitrator shall be held in New York
  (alternating between the NBA and Players Association offices). All
  such hearings shall be conducted in accordance with the Labor
  Arbitration Rules of the American Arbitration Association; provided,
  however, that in the event of any conflict between such Rules and the
  provisions of this Agreement, the provisions of this Agreement shall
  control.
\end{enumerate}

\section{Procedure.}\label{procedure.}

\begin{enumerate}
\def\labelenumi{(\alph{enumi})}
\tightlist
\item
  Not later than seven (7) days prior to the hearing, the parties shall
  submit to the Grievance Arbitrator a joint statement of the issue(s)
  in dispute. If the parties cannot agree on such a joint statement,
  each party may submit to the Grievance Arbitrator a separate statement
  setting forth the disputed issue(s), and such separate statement shall
  be delivered to the other party or parties at the same time it is
  submitted to the Grievance Arbitrator.
\item
  During each Salary Cap Year covered by this Agreement, the NBA and the
  Players Association shall each be entitled, as a matter of right, in
  connection with two (2) proceedings brought pursuant to this Article
  XXXI, to the discovery, in advance of a hearing, of non-privileged
  documents from any adverse party (or parties) in such proceeding. The
  party (or parties) to whom a request for document discovery is made
  shall have the obligation to produce only documents that are directly
  relevant and material to the core issue(s) in dispute, and shall not
  be obligated to produce documents merely because the production of
  such documents would be reasonably calculated to lead to the discovery
  of relevant or admissible evidence.
\item
  Not later than three (3) business days prior to the hearing, the
  parties shall exchange witness lists, relevant documents and other
  evidentiary materials, and citations of legal authorities that each
  side intends to rely on in its affirmative case. Absent a showing of
  good cause, no party may proffer, refer to, or rely on the testimony
  of any witness, any document or other evidentiary material in its
  affirmative case that has not been identified to the other side as
  required by this subsection.
\item
  The Grievance Arbitrator shall grant the request of any party to file
  a pre-hearing and/or post-hearing brief, unless an opposing party
  demonstrates that the filing of briefs is unreasonable in the
  circumstances. If the Grievance Arbitrator grants a request to file
  pre-hearing briefs, such briefs shall be served on the adverse party
  (or parties) and filed with the Grievance Arbitrator not later than
  three (3) business days prior to the hearing. No pre-hearing brief
  shall exceed ten (10) pages in length, and the rules applicable in the
  United States District Court for the Southern District of New York
  with respect to the calculation of pages, the size of font, margins
  and the like shall apply. If the Grievance Arbitrator grants a request
  to file post-hearing briefs, such briefs shall be served on the
  adverse party (or parties) and filed with the Grievance Arbitrator not
  later than seven (7) calendar days after the conclusion of the hearing
  (unless the parties otherwise agree).
\end{enumerate}

\section{Arbitrator's Decision and
Award.}\label{arbitrators-decision-and-award.}

\begin{enumerate}
\def\labelenumi{(\alph{enumi})}
\tightlist
\item
  Except as set forth in Section 13 below, the Grievance Arbitrator
  shall render an award as soon as practicable. The award shall be
  accompanied by a written opinion, or, if both the NBA and the Players
  Association agree, the written opinion may follow within a reasonable
  time thereafter. In no event shall the award and written opinion be
  issued more than thirty (30) days following the conclusion of a
  Grievance hearing (or, where applicable, following the date designated
  by the Grievance Arbitrator for the submission of post-hearing
  briefs). The award shall constitute full, final and complete
  disposition of the Grievance, and shall be binding upon the player(s)
  and Team(s) involved and the parties to this Agreement.
\item
  In addition to such other limitations as may be imposed on him/her by
  this Agreement, the Grievance Arbitrator shall have jurisdiction and
  authority only to: (i) interpret, apply, or determine compliance with
  the provisions of this Agreement; (ii) interpret, apply or determine
  compliance with the provisions of Player Contracts; (iii) determine
  the validity of Player Contracts; (iv) award damages in connection
  with a proceeding provided for in Section 12 below; (v) award
  declaratory relief in connection with a proceeding initiated by a Team
  to determine whether such Team may properly terminate a Player
  Contract pursuant to paragraph 16(a) of such Contract, and what, if
  any, liability such Team would incur as a result of such termination;
  and (vi) resolve disputes arising under Article VII, Section 3(d)(5),
  Article XXII, Section 5, Article XXVI, and Article XXXIII in the
  manner set forth therein. Notwithstanding the foregoing or any other
  provision of this Agreement or the Uniform Player Contract, the
  Grievance Arbitrator shall not have jurisdiction or authority to add
  to, detract from, or alter in any way the provisions of this Agreement
  (including the provisions of this Section 6(b)) or any Player
  Contract. Nor, in the absence of agreement by the NBA and the Players
  Association, shall the Grievance Arbitrator have jurisdiction or
  authority to resolve questions of substantive, as opposed to
  procedural, arbitrability. Questions of substantive arbitrability
  shall include the question of whether an arbitrator provided for by
  the terms of this Agreement, as opposed to the Commissioner (or his
  designee), has jurisdiction to hear or resolve a particular dispute
  and such questions shall be determined in a judicial proceeding to be
  venued in the United States District Court for the Southern District
  of New York.
\end{enumerate}

\section{Appointment and Replacement of Grievance
Arbitrator.}\label{appointment-and-replacement-of-grievance-arbitrator.}

\begin{enumerate}
\def\labelenumi{(\alph{enumi})}
\tightlist
\item
  The parties to this Agreement shall agree upon the appointment of a
  Grievance Arbitrator, who shall serve for the duration of this
  Agreement; provided, however, that as of September 1, 2017, and as of
  each successive September 1, either of the parties to this Agreement
  may discharge the Grievance Arbitrator by serving written notice upon
  him/her and upon the other party to this Agreement during the period
  July 27 through August 1 immediately preceding each such September 1;
  and provided, further, that as of the April 30 of the last Season
  covered by this Agreement (or any extension thereof), either of the
  parties may discharge the Grievance Arbitrator by serving written
  notice upon him/her and upon the other party to this Agreement during
  the period March 26 through March 31 immediately preceding such April
  30. A Grievance Arbitrator as to whom a notice of discharge has been
  served shall continue to have jurisdiction only with respect to (i)
  Grievances as to which a hearing has been commenced or scheduled for a
  date certain and (ii) Grievances filed within the thirty (30) day
  period preceding the service of a notice of discharge; provided,
  however, that a hearing with respect to Grievances referred to in this
  Section 7(a)(ii) must commence no later than thirty (30) days
  following the effective date of the Grievance Arbitrator's discharge.
\item
  If the Grievance Arbitrator is discharged (or resigns), the parties
  shall agree upon a successor Grievance Arbitrator. In the absence of
  such agreement, the parties shall jointly request the International
  Institute for Conflict Prevention and Resolution (the ``CPR
  Institute'') (or such other organization(s) as the parties may agree
  upon) to submit to the parties a list of eleven (11) attorneys, none
  of whom shall have, nor whose firm shall have, represented within the
  past five (5) years any professional athletes; agents or other
  representatives of professional athletes; labor organizations
  representing athletes; sports leagues, governing bodies, or their
  affiliates; sports teams or their affiliates; or owners in any
  professional sport. If, within seven (7) days from the receipt of such
  list, the parties fail to agree upon the selection of a Grievance
  Arbitrator from among the names on such list, they shall return that
  list, with up to five (5) names deleted therefrom by each party, to
  the CPR Institute (or such other organization as the parties may have
  agreed upon), and the CPR Institute (or such other organization) shall
  choose a new Grievance Arbitrator from the names remaining on such
  list.
\end{enumerate}

\section{Injury Grievances.}\label{injury-grievances.}

\begin{enumerate}
\def\labelenumi{(\alph{enumi})}
\tightlist
\item
  If a party to a dispute arising under paragraphs 7, 16(a)(iii), 16(b),
  or 16(c) of a Uniform Player Contract so elects, the NBA and the
  Players Association shall agree upon a neutral physician or (in the
  absence of such agreement) jointly request that the President of the
  American College of Orthopedic Surgeons (or such other similar
  organization as the NBA and the Players Association agree may be most
  appropriate to the issues in dispute) designate a physician who has no
  relationship with any party covered by this Agreement who shall, for
  purposes of the dispute, serve as an independent medical expert and
  consultant to the Grievance Arbitrator. Such independent medical
  expert shall conduct a physical examination of the player; review such
  medical records and reports relating to the player that bear on the
  issues in dispute; and prepare a written report of the player's
  medical condition, which report shall address any specific medical
  questions submitted to the independent medical expert by joint
  agreement of the parties or by the Grievance Arbitrator. Any reports,
  opinions, or conclusions of the independent medical expert shall be
  provided in writing to the parties in advance of any hearing scheduled
  pursuant to Section 4 above. The opinions and conclusions of the
  independent medical expert shall be accorded such weight as the
  Grievance Arbitrator deems appropriate. The fees and costs of the
  independent medical expert shall be borne equally by both sides.
\item
  During the course of any arbitration proceeding, the Grievance
  Arbitrator may, by appropriate process, require any person (including,
  but not limited to, a Team and a Team physician, and a player and any
  physician consulted by such player) to provide to the player or that
  player's Team, as the case may be, all medical information in the
  possession of any such person relating to the subject matter of the
  arbitration.
\end{enumerate}

\section{Special Procedures with Respect to Player
Discipline.}\label{special-procedures-with-respect-to-player-discipline.}

\begin{enumerate}
\def\labelenumi{(\alph{enumi})}
\item
  A dispute involving (i) a fine of \$50,000 or less or a suspension of
  twelve (12) games or less (or both such fine and suspension) imposed
  upon a player by the Commissioner (or his designee) for (x) conduct on
  the playing court (as defined in Section 9(c)(i) below) or (y) for
  in-game conduct involving another player (as defined in Section
  9(c)(ii) below), or (ii) action taken by the Commissioner (or his
  designee) (A) concerning the preservation of the integrity of, or the
  maintenance of public confidence in, the game of basketball and (B)
  resulting in a financial impact on the player of \$50,000 or less,
  shall not give rise to a Grievance, shall not be subject to a hearing
  before, or resolution by, the Grievance Arbitrator, and shall not be
  determined by arbitration; but instead shall be processed exclusively
  as follows:

  \begin{enumerate}
  \def\labelenumii{(\arabic{enumii})}
  \tightlist
  \item
    Within twenty (20) days following written notification of the action
    taken by the Commissioner (or his designee), the Players Association
    (with the approval of the player involved) may appeal in writing to
    the Commissioner.
  \item
    Upon the written request of the Players Association, the
    Commissioner shall designate a time and place for hearing as soon as
    is reasonably practicable following his receipt of the notice of
    appeal.
  \item
    As soon as reasonably practicable, but not later than twenty (20)
    days, following the conclusion of such hearing, the Commissioner
    shall render a written decision, which decision shall, absent
    further proceedings pursuant to Section 9(a)(5) below, constitute
    full, final and complete disposition of the dispute, and shall be
    binding upon the player(s) and Team(s) involved and the parties to
    this Agreement.
  \item
    In the event such appeal involves a fine and/or suspension imposed
    by the Commissioner's designee, the Commissioner, as a consequence
    of such appeal and hearing, shall have authority only to affirm or
    reduce such fine and/or suspension, and shall not have authority to
    increase such fine and/or suspension.
  \item
    If a dispute under Section 9(a)(i)(y) above is not resolved in a
    manner satisfactory to the player as a result of the procedures set
    forth in Section 9(a)(1)-(4) above, then the Players Association may
    (with the approval of such player) seek review of the financial
    impact of the Commissioner's decision by filing a written request
    for such review with the Player Discipline Arbitrator (as provided
    for below) within ten (10) days following the issuance of such
    decision, and the following procedures shall apply:

    \begin{enumerate}
    \def\labelenumiii{(\alph{enumiii})}
    \tightlist
    \item
      Following receipt of the written request for review, the Player
      Discipline Arbitrator shall schedule a meeting with the player,
      the Players Association, and the NBA (and such representatives as
      each may designate), shall review the relevant facts and
      circumstances, and shall issue a decision affirming or reducing
      the financial penalty imposed by the Commissioner. All such
      meetings shall be in person, shall be held in New York
      (alternating between the NBA and Players Association offices), and
      shall be conducted during the month of September following the
      conclusion of the Season in which the in-game conduct involving
      another player occurred.
    \item
      In reviewing the fine and/or suspension imposed upon the player by
      the Commissioner, the Player Discipline Arbitrator shall have
      authority only to affirm or reduce the financial penalty
      associated with such fine and/or suspension (including lost
      salary). The Player Discipline Arbitrator shall have no authority
      to review financial penalties automatically imposed as a result of
      technical fouls, ejections, or the violation of other similar NBA
      rules that result in the imposition of an automatic penalty (such
      as the ``leaving the bench'' rule). The review by the Player
      Discipline Arbitrator shall be de novo.
    \item
      The decision of the Player Discipline Arbitrator shall constitute
      full, final and complete disposition of the dispute, and shall be
      binding upon the player(s) and Team(s) involved and the parties to
      this Agreement. The Player Discipline Arbitrator shall make no
      public comment regarding the matter.
    \item
      The Player Discipline Arbitrator shall be selected by agreement
      between the NBA and the Players Association, and shall be (i) a
      person with experience in professional basketball (such as a
      former NBA coach, general manager, or player) or (ii) an attorney
      with experience as a private arbitrator and/or mediator. In the
      event that the NBA and the Players Association cannot agree on the
      identity of the Player Discipline Arbitrator, each party shall
      simultaneously serve upon the other a list of the names of five
      (5) individuals meeting the criteria set forth in this Section
      9(a)(5)(d) and shall alternate in striking names from such list
      until only one (1) such name remains; and the individual whose
      name remains on the list shall be selected as the Player
      Discipline Arbitrator. (A coin-flip or such other procedure as
      agreed upon by the NBA and the Players Association shall determine
      which of such parties shall exercise the first strike.)
    \item
      The Player Discipline Arbitrator shall serve for the duration of
      this Agreement; provided, however, that as of January 1, 2018, and
      as of each successive January 1, either of the parties to this
      Agreement may discharge the Player Discipline Arbitrator by
      serving written notice upon him and upon the other party to this
      Agreement during the period from November 1 through December 1
      immediately preceding each such January 1.
    \item
      If the Player Discipline Arbitrator is discharged (or resigns),
      the parties shall select a successor Player Discipline Arbitrator
      in accordance with the procedures set forth in Section 9(a)(5)(d)
      above.
    \end{enumerate}
  \end{enumerate}
\item
  A dispute involving (i) a fine of more than \$50,000 and/or a
  suspension of more than twelve (12) games that is imposed upon a
  player by the Commissioner (or his designee) for conduct on the
  playing court, or (ii) an action taken by the Commissioner (or his
  designee) that (A) concerns the preservation of the integrity of, or
  the maintenance of public confidence in, the game of basketball and
  (B) results in a financial impact on the player of more than \$50,000,
  shall be processed and determined in the same manner as a Grievance
  under Sections 2-7 above; provided, however, that the Grievance
  Arbitrator shall apply an ``arbitrary and capricious'' standard of
  review.
\item
  \begin{enumerate}
  \def\labelenumii{(\roman{enumii})}
  \tightlist
  \item
    As used in this Agreement, ``conduct on the playing court'' shall
    mean conduct in any area within an arena (including, but not limited
    to, locker rooms, vomitories, loading docks, and other back-of-house
    and underground areas, including those used by television production
    and other vehicles), at, during or in connection with an NBA
    Exhibition, All-Star, Regular Season or Playoff game. (By way of
    example and not limitation, conduct ``at'' and/or ``in connection
    with'' an NBA game shall include conduct engaged in by a player
    within an arena from the time the player arrives at the arena for an
    NBA game until the time the player has left the premises of the
    arena following the conclusion of such game.) Conduct engaged in by
    a player outside an arena such as, for example, in a parking lot
    adjacent to an arena, shall not constitute ``conduct on the playing
    court.''
  \item
    As used in this Agreement, ``in-game conduct involving another
    player'' shall mean conduct occurring during the course of an NBA
    Exhibition, All-Star, Regular Season or Playoff Game that is
    exclusively between or among players (and not, for example,
    involving in any manner a referee, fan, or coach) and that takes
    place on or adjacent to the playing floor (including the area of the
    benches), and shall include, but not be limited to, fights,
    altercations, flagrant fouls, and other similar conduct.
  \end{enumerate}
\item
  In the event a matter filed as a Grievance in accordance with the
  provisions of this Article XXXI gives rise to issues involving the
  integrity of, or public confidence in, the game of basketball, and the
  financial impact on the player of the action being grieved is \$50,000
  or less, the Commissioner may, at any stage of its processing, order
  that the matter be withdrawn from such processing and thereafter be
  processed in accordance with the appeal procedure provided in Sections
  9(a)(1)-(4) above.
\end{enumerate}

\section{Procedure with Respect to Fine and Suspension
Amounts.}\label{procedure-with-respect-to-fine-and-suspension-amounts.}

In the event that a Grievance or an appeal challenging a Commissioner or
Team-imposed fine and/or suspension is filed in accordance with this
Article XXXI, the amount of any fine or salary lost by virtue of the
suspension shall be deposited in a separate interest-bearing account
maintained for such fines or suspension-related amounts. The NBA shall
provide written notice to the Players Association of the date and amount
of each deposit made pursuant to this Section 10 by delivering to the
Players Association monthly statements reflecting the investment
activity in such account. In the absence of agreement between the NBA
and the Players Association, the Grievance Arbitrator (in resolving a
Grievance, and in a manner consistent with his determination of such
Grievance), or the Commissioner (or his designee) (in resolving an
appeal, and in a manner consistent with his determination of such
appeal), or the Player Discipline Arbitrator (in connection with his
review of a decision by the Commissioner, and in a manner consistent
with his determination following such review) shall determine the amount
of the deposited funds to be payable to the player, the Team, or the
NBA, and any interest earned on such deposit shall be allocated to the
parties in proportion thereto.

\section{Disputes with Respect to the Terms of a Player
Contract.}\label{disputes-with-respect-to-the-terms-of-a-player-contract.}

\begin{enumerate}
\def\labelenumi{(\alph{enumi})}
\tightlist
\item
  If either the NBA or the Players Association asserts that a term or
  provision of a Player Contract is not permitted by this Agreement,
  either may have the dispute involving such Contract term or provision
  resolved by initiating a Grievance. If such a Grievance is initiated
  by the NBA, the thirty-day time period referred to in Section 2(c)
  above shall commence with the date upon which the NBA received the
  Player Contract (or amendment thereto) containing the disputed term or
  provision. If such a Grievance is initiated by the Players
  Association, the thirty-day time period referred to in Section 2(c)
  above shall commence with the date upon which the Player Contract (or
  amendment thereto) containing the disputed term or provision was first
  made available for inspection by the Players Association.
\item
  If, as a result of the Grievance and Arbitration procedure, a Player
  Contract is found to contain a term or provision that is not permitted
  by this Agreement, then (i) such term or provision shall be deleted
  from the Player Contract and have no force or effect, and the Player
  Contract shall in all other respects remain valid and binding upon the
  parties thereto, and (ii) if the Team and the player agree to reform
  or revise the Player Contract within thirty (30) days of the Grievance
  Arbitrator's decision, such reformation or revision shall be exempted
  from the rules governing Renegotiations contained in Article VII,
  Section 7(c).
\item
  Nothing set forth above shall affect in any manner the Commissioner's
  authority with respect to the approval or disapproval of Player
  Contracts pursuant to paragraph 11 of the Uniform Player Contract; and
  the fact that the Commissioner has approved or not disapproved a
  Player Contract containing a term or provision not permitted by this
  Agreement shall not be referred to in the course of the Grievance and
  Arbitration procedure and shall not be considered in any manner or for
  any purpose by the Grievance Arbitrator in connection with a dispute
  concerning that Player Contract.
\end{enumerate}

\section{Disputes with Respect to Players Under Contract Who Withhold
Playing
Services.}\label{disputes-with-respect-to-players-under-contract-who-withhold-playing-services.}

In addition to any other rights a Team may have under contract or law,
including those under paragraph 9 of a Uniform Player Contract, a Team
may recover damages in a proceeding before the Grievance Arbitrator when
a player who is party to a currently effective Player Contract fails or
refuses to render the services called for under the Player Contract. In
any such proceeding, where the Grievance Arbitrator determines that
damages are continuing to accrue at the time of the hearing, the
Arbitrator shall award such damages (if any) as the Team has by then
sustained, and the hearing shall remain open to enable the submission of
proof on the issue of continuing damages.

\section{Expedited Procedure.}\label{expedited-procedure.}

\begin{enumerate}
\def\labelenumi{(\alph{enumi})}
\tightlist
\item
  Notwithstanding the foregoing, in the event of a dispute arising under
  Article XVII, Article XXX, or Article XXXI, Section 12 of this
  Agreement, or under paragraph 15 of a Uniform Player Contract (but
  only insofar as such paragraph provides), or in the event of an
  alleged breach by a player of paragraph 9 of a Uniform Player
  Contract, the NBA or the Players Association may request that such
  dispute or alleged breach be referred immediately to the Grievance
  Arbitrator. In any such case, the dispute or alleged breach shall be
  asserted by notice in writing or by facsimile given to the other party
  or parties, the NBA, the Players Association, and the Grievance
  Arbitrator.
\item
  The Grievance Arbitrator shall convene a hearing with respect to such
  dispute or alleged breach at the earliest possible time, but in no
  event later than twenty-four (24) hours following his receipt of such
  notice. If the Grievance Arbitrator is not immediately available and
  the parties are unable to agree upon another arbitrator to hear and
  resolve such dispute, the parties shall select an arbitrator in
  accordance with the procedures set forth in Section 7(b) above.
\item
  The award, which shall be issued not later than twenty-four (24) hours
  after the conclusion of the hearing, shall be in writing and may be
  issued with or without opinion. If any party desires an opinion, one
  shall be issued but its issuance shall not delay compliance with or
  enforcement of the award. The award shall constitute full, final and
  complete disposition of the dispute or alleged breach, and shall be
  binding upon the player(s) and Team(s) involved and the parties to
  this Agreement.
\item
  The failure of any party to attend the hearing as scheduled shall not
  delay the hearing, and the Grievance Arbitrator (or an arbitrator
  selected in accordance with the procedures set forth in Section 7(b)
  above, as the case may be) shall be authorized to proceed to take
  evidence and issue an award as though such party were present.
\end{enumerate}

\section{Threshold Amount for Certain
Grievances.}\label{threshold-amount-for-certain-grievances.}

A dispute concerning a fine or suspension (or a combination thereof)
imposed by a Team may be heard and resolved by the Grievance Arbitrator
only if it results in a financial impact on the player of more than
\$5,000. A dispute concerning a fine or suspension (or a combination
thereof) imposed by the Commissioner (or his designee) other than for
conduct on the playing court (as defined in Section 9(c) above) may be
heard and resolved by the Grievance Arbitrator only if it results in a
financial impact on the player of more than \$50,000.

\section{Miscellaneous.}\label{miscellaneous.-1}

\begin{enumerate}
\def\labelenumi{(\alph{enumi})}
\tightlist
\item
  Each of the time limits set forth herein may be extended by mutual
  agreement of the parties involved.
\item
  In any meeting or hearing provided for by this Article XXXI, a player
  may be accompanied by a representative of the Players Association who
  may participate in such meeting or hearing and represent the player.
  In any such meeting or hearing, the NBA and any Team involved may
  attend and be accompanied by a representative who may participate in
  such meeting or hearing and represent the NBA and any such Team.
\item
  The parties recognize that a player may be subjected to disciplinary
  action for just cause by his Team or by the Commissioner (or his
  designee). Therefore, in Grievances regarding discipline, the issue to
  be resolved shall be whether there has been just cause for the penalty
  imposed. Notwithstanding the foregoing, in all proceedings pursuant to
  Section 9(b) above, the Grievance Arbitrator shall apply an
  ``arbitrary and capricious'' standard of review as set forth in that
  Section.
\item
  Nothing contained herein shall excuse a player from prompt compliance
  with any discipline imposed upon him. If discipline imposed upon a
  player is determined to be improper by a final disposition under this
  Article XXXI, the player shall promptly be made whole.
\item
  Nothing contained in this Article XXXI shall be deemed to limit or
  impair the right of the NBA or any Team to impose discipline upon a
  player(s) or to take any other action not inconsistent with the
  provisions of a Player Contract or this Agreement.
\item
  Subject to Section 4(c) above, all costs of arbitration, including the
  fees and expenses of the Grievance Arbitrator, and all costs of the
  proceedings before the Player Discipline Arbitrator (including the
  fees and expenses of the Player Discipline Arbitrator) shall be borne
  equally by the parties thereto; but each party shall bear the cost of
  its own witnesses, counsel, and the like.
\item
  A Team shall not be required to terminate a Player Contract under the
  NBA waiver procedure as a condition precedent to the filing of a
  Grievance with respect to such Player Contract. To the extent that the
  decision of the Impartial Arbitrator in In re: Otis Birdsong, Dec. No.
  87-2, May 14, 1987, is inconsistent with the foregoing, it is hereby
  overruled.
\item
  In a proceeding involving the interpretation of a Player Contract, no
  Uniform Player Contract (whether signed during the term of this
  Agreement or during the term of any prior collective bargaining
  agreement between the parties), or amendment thereto, other than the
  Player Contract or amendment that is the subject of dispute shall be
  admissible as evidence of the meaning of, or of the parties'
  intentions with respect to, any individually-negotiated terms or
  provisions in the Player Contract or amendment that is the subject of
  dispute.
\end{enumerate}

\chapter{SYSTEM ARBITRATION}\label{system-arbitration}

\section{Jurisdiction and Authority.}\label{jurisdiction-and-authority.}

The NBA and the Players Association shall agree upon a System
Arbitrator, who shall have exclusive jurisdiction to determine any and
all disputes arising under Articles I, II, VII (except as otherwise
specifically provided by Article VII, Section 3(d)(5)), VIII, X, XI,
XII, XIII, XIV, XV, XVI, XXXVII, XXXIX, and XL of this Agreement, any
and all disputes arising under Article XXVIII and Paragraph 14 of the
Uniform Player Contract regarding an Unauthorized Sponsor Promotion (as
that term is defined in Paragraph 14(c) of the Uniform Player contract),
and those disputes made subject to his jurisdiction by Sections 9 and 10
of this Article. In addition, in the event of a disagreement between the
NBA and the Players Association, the System Arbitrator shall have
exclusive jurisdiction to determine whether the System Arbitrator, the
Grievance Arbitrator or some other arbitrator provided for by the
provisions of this Agreement has jurisdiction to hear or resolve a
particular dispute.

\section{Initiation.}\label{initiation.-1}

\begin{enumerate}
\def\labelenumi{(\alph{enumi})}
\tightlist
\item
  Subject to Article XIV, Section 5, System Arbitrations may be
  initiated, as set forth below, only by the NBA or the Players
  Association.
\item
  No party may initiate a System Arbitration until and unless it has
  first discussed the matter with the other party in an attempt to
  settle it.
\item
  A System Arbitration must be initiated within three (3) years from the
  date of the act or omission upon which the System Arbitration is
  based, or within three (3) years from the date upon which such act or
  omission became known or reasonably should have become known to the
  party initiating the System Arbitration, whichever is later.
\item
  Either the NBA or the Players Association may initiate a System
  Arbitration by serving a written notice thereof on the other party,
  with a copy of such written notice to be filed with the System
  Arbitrator.
\end{enumerate}

\section{Hearings.}\label{hearings.-1}

\begin{enumerate}
\def\labelenumi{(\alph{enumi})}
\tightlist
\item
  The System Arbitrator shall hold hearings on alleged violations of the
  Articles set forth in Section 1 above. Except as otherwise provided in
  Article XI, Section 5(l) and Sections 9 and 10 below, awards issued by
  the System Arbitrator shall be subject to review by the Appeals Panel,
  in the manner and in accordance with the procedures set forth in
  Sections 3 and 8 of this Article XXXII.
\item
  The System Arbitrator shall make findings of fact and award
  appropriate relief including, without limitation, damages, injunctive
  relief and specific performance; provided, however, that the System
  Arbitrator shall not have the authority to impose an award of punitive
  damages on any party. The System Arbitrator shall render an award as
  soon as practicable, and the award shall be accompanied by a written
  opinion. Notwithstanding the foregoing, if the System Arbitrator
  determines that expedition so requires, he/she shall accompany the
  award with a written summary of the grounds upon which the award is
  based, and a full written opinion may follow within a reasonable time
  thereafter. In no event shall the award and written opinion be issued
  more than thirty (30) days following the date upon which the record of
  a System Arbitration proceeding is closed (or, where applicable, the
  date designated by the System Arbitrator for the submission of
  post-hearing briefs).
\item
  The System Arbitrator shall have authority to order the production of
  documents, the conduct of pre-hearing depositions, and the attendance
  of witnesses at the hearing with respect to the NBA and the Players
  Association, and/or any player or Team. The System Arbitrator shall
  have the authority to compel the attendance of witnesses and the
  production of documents at any hearing within the jurisdiction of the
  System Arbitrator in accordance with the New York C.P.L.R.
\item
  An award of the System Arbitrator shall upon its issuance constitute
  the full, final and complete disposition of the dispute, shall be
  binding upon the parties to this Agreement and upon any player(s) or
  Team(s) involved, and shall be followed by them unless (in cases where
  this Agreement provides for an appeal to the Appeals Panel) a notice
  of appeal is served by the appealing party upon the responding party
  and filed with the System Arbitrator within ten (10) days of the date
  of the award of the System Arbitrator appealed from. If and when an
  award of the System Arbitrator is reversed or modified by the Appeals
  Panel, the effect of such reversal or modification shall be deemed by
  the parties to be retroactive to the time of issuance of the award of
  the System Arbitrator. The parties may seek appropriate relief to
  effectuate and enforce this provision.
\item
  The System Arbitrator shall not have jurisdiction or authority to add
  to, detract from, or alter in any way the provisions of this Agreement
  or any Player Contract. Nor, except for the authority conferred upon
  him/her by the second sentence of Section 1 above (or unless the NBA
  and the Players Association otherwise agree), shall the System
  Arbitrator have jurisdiction or authority to resolve questions of
  substantive, as opposed to procedural, arbitrability (which shall
  include the question of whether an arbitrator provided for by the
  terms of this Agreement, as opposed to the Commissioner (or his/her
  designee), has jurisdiction to hear or resolve a particular dispute),
  which shall be determined in a judicial proceeding to be venued in the
  United States District Court for the Southern District of New York.
\end{enumerate}

\section{Costs Relating to System
Arbitration.}\label{costs-relating-to-system-arbitration.}

\begin{enumerate}
\def\labelenumi{(\alph{enumi})}
\tightlist
\item
  The compensation of the System Arbitrator and the costs and expenses
  incurred in connection with any proceeding brought before the System
  Arbitrator shall be borne equally by the parties to this Agreement;
  provided, however, that each participant in such proceeding shall bear
  its own attorneys' fees and litigation costs.
\item
  Notwithstanding the provisions of Section 4(a) above, if a matter is
  scheduled for hearing under this Article XXXII, and the hearing date
  is thereafter postponed at the request of either the NBA or the
  Players Association, the postponement fee (if any) of the System
  Arbitrator will be borne by the party requesting the postponement
  unless that party objects and the System Arbitrator finds that the
  request for such postponement was for good cause. Should good cause be
  found, the parties will share any postponement fee equally.
\end{enumerate}

\section{Procedure for System
Arbitration.}\label{procedure-for-system-arbitration.}

All matters before the System Arbitrator shall be heard and determined
in an expedited manner, provided that such expedition is reasonable
under the circumstances. A proceeding may be commenced upon seventy-two
(72) hours' written notice (or upon shorter notice if ordered by the
System Arbitrator) served upon the party against whom the proceeding is
brought and filed with the System Arbitrator. All such notices and all
orders and notices issued and directed by the System Arbitrator shall be
served on the NBA, counsel for the NBA, the Players Association, counsel
for the Players Association, and any counsel appearing for individual
NBA players or individual NBA Teams. In any proceeding commenced
pursuant to Article XIV, Section 5, the Players Association (on its own
behalf and/or on behalf of a player) and the NBA (on its own behalf
and/or on behalf of a Team) shall have the right to participate.

\section{Selection of System
Arbitrator.}\label{selection-of-system-arbitrator.}

\begin{enumerate}
\def\labelenumi{(\alph{enumi})}
\tightlist
\item
  In the event that the Players Association and the NBA cannot agree on
  the identity of a System Arbitrator, the parties shall jointly request
  the International Institute for Conflict Prevention and Resolution
  (the ``CPR Institute'') (or such other organization(s) as the parties
  may have agreed upon) to submit to the parties a list of eleven (11)
  attorneys, none of whom shall have, nor whose firm shall have,
  represented within the past five (5) years any professional athletes;
  agents or other representatives of professional athletes; labor
  organizations representing athletes; sports leagues, governing bodies,
  or their affiliates; sports teams or their affiliates; or owners in
  any professional sport. If the parties cannot within seven (7) days
  from the receipt of such list agree to the identity of the System
  Arbitrator from among the names on such list, they shall return said
  list, with up to five (5) names deleted therefrom by each party, to
  the CPR Institute (or such other organization as the parties may have
  agreed upon), which shall choose from the remaining name(s) on the
  list the identity of the System Arbitrator.
\item
  Effective July 1, 2017, the System Arbitrator selected by the parties
  shall serve for continually renewing two-year terms unless notice of
  termination is given either by the NBA or by the Players Association.
  Notice of termination of the System Arbitrator shall be given to the
  other party, and to the System Arbitrator, during the period May 10
  through May 15 immediately preceding the end of any term. Following
  the giving of such notice, a new System Arbitrator shall be selected
  in accordance with the procedures set forth in Section 6(a) above. A
  System Arbitrator as to whom a notice of termination has been given
  shall continue to have jurisdiction only with respect to (i) System
  Arbitrations in which a hearing has been commenced or scheduled for a
  date certain, and (ii) System Arbitrations initiated (in accordance
  with the provisions of Section 2 above) within the thirty (30) day
  period preceding the service of the notice of termination; provided,
  however, that a hearing with respect to System Arbitrations referred
  to in this subsection (ii) must commence no later than thirty (30)
  days following the end of a System Arbitrator's term.
\end{enumerate}

\section{Selection of Appeals Panel.}\label{selection-of-appeals-panel.}

\begin{enumerate}
\def\labelenumi{(\alph{enumi})}
\tightlist
\item
  There shall be a three-member Appeals Panel for each appeal noticed
  from an award of the System Arbitrator. In the event the Players
  Association and the NBA cannot agree upon the members of such a panel,
  the parties will jointly request the CPR Institute (or such other
  organization(s) as the parties may agree) to submit to the parties a
  list of fifteen (15) attorneys (none of whom shall have, nor whose
  firm shall have, represented within the past five (5) years any
  professional athletes; agents or other representatives of professional
  athletes; labor organizations representing athletes; sports leagues,
  governing bodies, or their affiliates; sports teams or their
  affiliates; or owners in any professional sport). If the parties
  cannot within seven (7) days from the receipt of such list agree to
  the identity of the Appeals Panel from among the names on such list,
  they shall meet and alternate striking one (1) name at a time from the
  list until three (3) names on the list remain. The three (3) remaining
  names on the list shall comprise the Appeals Panel.
\item
  Effective July 1, 2017, the members of the Appeals Panel selected by
  the parties shall serve for continually renewing two-year terms unless
  notice of termination is given either by the NBA or by the Players
  Association. On or before June 30, 2018, and on or before each other
  successive June 30 during the term of this Agreement, either party may
  discharge one or more members from the Appeals Panel by serving notice
  of termination on him/her on or before that date and upon the other
  party to this Agreement, and the discharge shall be effective as of
  such June 30. A discharged Appeals Panel member may participate in
  decisions rendered by the Appeals Panel in all cases previously heard
  to closure of the record, but not participate in the consideration or
  decision of any other cases. If a member of the Appeals Panel is not
  discharged as provided above, the member's term will automatically be
  renewed for an additional year. The compensation of the members of the
  Appeals Panel and the costs of proceedings before the Appeals Panel
  shall be borne equally by the parties to this Agreement; provided,
  however, that each participant in an Appeals Panel proceeding shall
  bear its own attorneys' fees and litigation costs.
\end{enumerate}

\section{Procedure Relating to Appeals of Determination by the System
Arbitrator.}\label{procedure-relating-to-appeals-of-determination-by-the-system-arbitrator.}

\begin{enumerate}
\def\labelenumi{(\alph{enumi})}
\tightlist
\item
  Any party seeking to appeal (in whole or in part) an award of the
  System Arbitrator must serve on the other party and file with the
  System Arbitrator a notice of appeal, within ten (10) days of the date
  of the award appealed from. The timely service and filing of a notice
  of appeal shall automatically stay the award of the System Arbitrator
  pending resolution by the Appeals Panel.
\item
  Following the timely service and filing of a notice of appeal, the NBA
  and the Players Association shall attempt to agree upon a briefing
  schedule. In the absence of such agreement, the briefing schedule
  shall be set by the Appeals Panel; provided, however, that any party
  seeking to appeal (in whole or in part) from an award of the System
  Arbitrator shall be afforded no less than fifteen (15) and no more
  than twenty-five (25) days from the date of the issuance of such
  award, or the date of the issuance of the System Arbitrator's written
  opinion, or the date upon which the members of the Appeals Panel have
  been selected in accordance with the provisions of Section 7 above,
  whichever is latest, to serve on the opposing party and file with the
  Appeals Panel its brief in support thereof; and provided further that
  the responding party or parties shall be afforded the same aggregate
  amount of time to serve and file its or their responding brief(s). The
  Appeals Panel shall schedule oral argument on the appeal(s) no less
  than five (5) and no more than ten (10) days following the service and
  filing of the responding brief(s), and shall issue a written decision
  within thirty (30) days from the date of argument.
\item
  The Appeals Panel shall review the findings of fact and conclusions of
  law made by the System Arbitrator using the standards of review
  employed by the United States Court of Appeals for the Second Circuit.
  The decision of the Appeals Panel shall constitute full, final, and
  complete disposition of the dispute, and shall be binding upon the
  parties to this Agreement and upon any player(s) or Team(s) involved.
\end{enumerate}

\section{Special Procedure for Disputes with Respect to Interim Audit
Reports.}\label{special-procedure-for-disputes-with-respect-to-interim-audit-reports.}

\begin{enumerate}
\def\labelenumi{(\alph{enumi})}
\tightlist
\item
  Notwithstanding any of the other provisions of this Agreement, at the
  request of either the NBA or the Players Association, and irrespective
  of which party may commence the proceeding, the procedures set forth
  in this Section 9 shall apply to the resolution of any disputes with
  respect to an Interim Audit Report (as defined in Article VII, Section
  10(a) above), including but not limited to disputes concerning any
  Escrow Information set forth in an Interim Audit Report. If in
  connection with such disputes, there is any conflict between the
  procedures set forth in this Section 9 and those set forth elsewhere
  in this Agreement, the procedures set forth in this Section shall
  control.
\item
  A proceeding before the System Arbitrator shall be commenced, in the
  manner provided for by Sections 2(d) and 5 above, no more than thirty
  (30) days following the delivery by the Accountants (as defined in
  Article VII, Section 10(a) above) of the Interim Audit Report with
  respect to any dispute or claim concerning (i) the amount(s) of BRI or
  Total Salaries (or portions thereof) as to which the Accountants have
  completed their review and which is the subject of a good faith
  dispute between the parties, (ii) the amount(s) of BRI or Total
  Salaries (or portions thereof) as to which the Accountants have not
  completed their review and with respect to which the parties have a
  good faith disagreement, (iii) such Escrow Information (as defined in
  Article VII, Section 10(a) above) as is included in the Interim Audit
  Report as to which the parties have a good faith disagreement, and/or
  (iv) all other disputes (including but not limited to disputes over
  the amounts and includability of any revenues or expenses included or
  excluded from the Interim Audit Report) of which the parties were
  aware or reasonably should have been aware, at the time the proceeding
  was commenced, based upon the contents of the BRI Reports, the Draft
  Audit Report or Interim Audit Report or other documents or writings
  provided to the parties by the Accountants in connection with their
  BRI audit.
\item
  A party's failure to commence a proceeding before the System
  Arbitrator within the thirty-day (30) period provided for by Section
  9(b) above with respect to the disputes or claims enumerated therein
  shall forever bar that party from asserting or seeking relief of any
  kind for any such dispute or claim; provided, however, that the
  provisions of Section 9(b) above and this Section 9(c) shall not bar a
  party from commencing a proceeding before the System Arbitrator and
  seeking appropriate relief, subject to the limitations imposed by
  Section 2 above:

  \begin{enumerate}
  \def\labelenumii{(\roman{enumii})}
  \tightlist
  \item
    With respect to a dispute or claim concerning an Interim Audit
    Report as to which such party was not aware or reasonably should not
    have been aware, based upon the materials referred to in Section
    9(b) above, during the thirty-day (30) period following the delivery
    of such Interim Audit Report; or
  \item
    With respect to any dispute or claim relating to a subsequent Salary
    Cap Year, including, but not limited to, any dispute concerning the
    includability or non-includability in BRI of a category or type of
    revenue or the allowance or disallowance of a category or type of
    expense, without regard to whether, based upon the materials
    referred to in Section 9(b) above (other than a BRI Report, Draft
    Audit Report or Interim Audit Report), the party was or reasonably
    should have been aware of such dispute or claim during the
    thirty-day (30) period following the delivery of such Interim Audit
    Report.
  \item
    Subject to Section 9(c)(ii) above, no determination made by the
    System Arbitrator or the Appeals Panel (as the case may be) in a
    proceeding commenced pursuant to Section 9(c)(i) or (ii) above shall
    affect any calculations made pursuant to Article VII, Section 12.
  \end{enumerate}
\item
  Where a hearing before the System Arbitrator is provided for by this
  Section 9, such hearing shall be conducted within fifteen (15) days
  from the commencement of the proceeding, and the System Arbitrator
  shall render an award and issue a written decision as soon as
  possible, but in no event later than fifteen (15) days following the
  close of the hearing. Where a right to appeal from the System
  Arbitrator's award is provided for by this Section 9, any party
  seeking to appeal (in whole or in part) from such an award shall serve
  and file a notice of appeal therefrom within five (5) days from the
  date of such award and shall serve and file its brief in support of
  such appeal within fifteen (15) days from the date of the System
  Arbitrator's award or within five (5) days from the date upon which
  the members of the Appeals Panel have been selected, whichever is
  later. The party opposing such appeal shall serve and file its brief
  in opposition within ten (10) days following its receipt of the brief
  in support of the appeal. The Appeals Panel shall schedule oral
  argument at its discretion, but shall issue a written decision within
  twenty (20) days following its receipt of the brief from the party
  opposing the appeal.
\item
  Any dispute concerning the amounts (as opposed to the includability)
  of any revenues or expenses to be included in an Interim Audit Report
  (hereinafter referred to as ``Disputed Adjustments'') shall, whenever
  such Disputed Adjustments for all Teams are adverse to the party
  asserting the dispute in an aggregate amount of less than \$10 million
  for any Season covered by this Agreement, be resolved by the
  Accountants; and the determination of the Accountants shall constitute
  full, final and complete disposition of the dispute and shall be
  binding upon the parties to this Agreement. Notwithstanding the
  foregoing, any Disputed Adjustments that involve the interpretation,
  validity or application of this Agreement shall be resolved by the
  System Arbitrator and shall be appealable to the Appeals Panel in
  accordance with the provisions of Section 9(d) above.
\item
  If the Disputed Adjustments for all Teams are adverse to the party
  asserting the dispute in an aggregate amount of \$10 million or more
  but less than \$15 million for any Season covered by this Agreement,
  the determination of the System Arbitrator shall constitute full,
  final and complete disposition of the dispute and shall be binding
  upon the parties to this Agreement, and there shall be no appeal to
  the Appeals Panel. Notwithstanding the foregoing, any Disputed
  Adjustments that involve the interpretation, validity or application
  of this Agreement shall be resolved by the System Arbitrator and shall
  be appealable to the Appeals Panel in accordance with the provisions
  of Section 9(d) above.
\item
  If the Disputed Adjustments for all Teams are adverse to the party
  asserting the dispute in an aggregate amount of \$10 million or more
  but less than \$15 million for any Season covered by this Agreement,
  and if the party asserting such dispute does not prevail before the
  System Arbitrator, then that party shall pay all of the fees and
  expenses of the System Arbitrator and the reasonable costs and
  expenses, including attorneys' fees, of the other party for its
  defense of the proceeding; provided, however, that if each party has
  asserted a dispute upon which it has not prevailed, all such fees,
  expenses and costs shall be borne in the manner provided for by
  Section 4 above.
\item
  All other disputes involving an Interim Audit Report (including but
  not limited to disputes over the amounts and includability of any
  revenues or expenses to be included in such Reports) and the Escrow
  Information shall be resolved by the System Arbitrator and shall be
  appealable to the Appeals Panel in accordance with the provisions of
  Section 9(d) above.
\end{enumerate}

\section{Special Procedure for Disputes with Respect to the Escrow
Schedules.}\label{special-procedure-for-disputes-with-respect-to-the-escrow-schedules.}

\begin{enumerate}
\def\labelenumi{(\alph{enumi})}
\tightlist
\item
  Notwithstanding any of the other provisions of this Agreement, the
  procedures set forth in this Section 10 shall apply to the resolution
  of any disputes with respect to the Escrow Schedules described in
  Article VII, Section 12. If in connection with such disputes, there is
  any conflict between the procedures set forth in this Section 10 and
  those set forth elsewhere in this Agreement, the procedures set forth
  in this Section shall control.
\item
  In the event of any dispute with respect to the Escrow Schedules, the
  proceeding before the System Arbitrator shall be commenced, in the
  manner provided for by Sections 2(d) and 5 above, no more than seven
  (7) days following the transmittal to the Players Association of any
  of such schedules.
\item
  The hearing before the System Arbitrator with respect to a dispute
  concerning the Escrow Schedules shall be conducted within ten (10)
  days following the commencement of the proceeding and the briefs of
  the parties, if any, shall be filed before the opening of the hearing
  on a date or dates set by the System Arbitrator. The hearing shall be
  conducted on an expedited basis and, unless the parties otherwise
  agree or a party demonstrates that such limitation will result in
  undue prejudice, will not last longer than two (2) full days.
\item
  If in connection with the Escrow Schedules, there is a dispute between
  the NBA and the Players Association and the amount in controversy is
  \$5 million or less, the determination of the System Arbitrator shall
  constitute full, final and complete disposition of the dispute and
  shall be binding upon the parties to this Agreement, and there shall
  be no appeal to the Appeals Panel. If with respect to such dispute the
  amount in controversy is more than \$5 million, either party may
  appeal a determination of the System Arbitrator to the Appeals Panel.
\item
  In connection with any dispute concerning the Escrow Schedules, the
  System Arbitrator shall render an award and issue a written decision
  as soon as possible, but in no event later than ten (10) days
  following the close of the hearing. When the award is issued, the
  System Arbitrator shall set forth the basis therefore either in a
  written opinion or orally at a conference with the parties (which
  conference may be conducted by telephone) of which a stenographic
  record shall be made. Any party seeking to appeal (in whole or in
  part) from an award of the System Arbitrator rendered pursuant to
  Section 10(d) above shall serve and file a notice of appeal therefrom
  within two (2) business days from the date of such award. The party
  seeking to appeal shall serve and file its brief in support of such
  appeal within ten (10) days from the date of the System Arbitrator's
  award or within three (3) days from the date upon which the members of
  the Appeals Panel have been selected, whichever is later. The party
  opposing such appeal shall serve and file its brief in opposition
  within ten (10) days following its receipt of the brief in support of
  the appeal. The Appeals Panel shall schedule oral argument at its
  discretion, but shall issue a written decision within twenty (20) days
  following its receipt of the brief from the party opposing the appeal.
\end{enumerate}

\chapter{ANTI-DRUG PROGRAM}\label{anti-drug-program}

\section{Definitions.}\label{definitions.-2}

As used in this Article XXXIII, the following terms shall have the
following meanings:

\begin{enumerate}
\def\labelenumi{(\alph{enumi})}
\tightlist
\item
  ``Authorization for Testing'' shall mean a notice issued by the
  Independent Expert pursuant to the provisions of Section 5 below in
  the form annexed hereto as Exhibit I-1 to this Agreement.
\item
  ``Come Forward Voluntarily'' shall mean that a player has directly
  communicated to the Medical Director his desire to enter the Program
  and seek treatment for a problem involving the use of a Drug of Abuse
  or Marijuana. Such communication may be facilitated by a
  representative of the NBA or the Players Association (e.g., by
  arranging a conference call among the player, the Medical Director,
  and such representative in which this communication occurs). A player
  may not Come Forward Voluntarily if, prior to his direct communication
  to the Medical Director, he has been notified that his most recent
  drug test was positive for a Drug of Abuse or Marijuana. A player may
  not Come Forward Voluntarily for the use of a SPED.
\item
  ``Counselors'' shall mean the persons selected by the Medical Director
  to provide counseling and other treatment to players in the Program.
\item
  ``Diuretics'' shall mean any of the substances listed as diuretics on
  Exhibit I-2 to this Agreement.
\item
  ``Drugs of Abuse'' shall mean any of the substances listed as drugs of
  abuse on Exhibit I-2 to this Agreement.
\item
  ``Drugs of Abuse Program'' shall mean (i) the testing program for
  Drugs of Abuse set forth in this Article XXXIII, and (ii) the
  education, treatment, and counseling program for Drugs of Abuse
  established by the Medical Director (after consultation with the NBA
  and the Players Association), which may contain such
  elements---including, but not limited to, urine, blood, breath, or
  other testing for Prohibited Substances other than SPEDs---as may be
  determined by the Medical Director in his or her professional
  judgment.
\item
  ``First-Year Player'' shall mean a player under Contract to an NBA
  Team who, prior to the then-current Season, has not been on the roster
  of an NBA Team following the first game of a Regular Season.
\item
  ``HGH Blood Testing'' shall mean the collection and testing of blood
  samples for Human Growth Hormone.
\item
  ``In-Patient Facility'' shall mean such treatment center or other
  facility as may be selected by the Medical Director and agreed upon by
  the NBA and the Players Association.
\item
  ``Independent Expert'' or ``Expert'' shall mean the person selected by
  the NBA and the Players Association in accordance with Section 2(c)
  below.
\item
  ``Marijuana Program'' shall mean (i) the testing program for marijuana
  set forth in this Article XXXIII, and (ii) the education, treatment,
  and counseling program for marijuana established by the Medical
  Director (after consultation with the NBA and the Players
  Association), which may contain such elements---including, but not
  limited to, urine, blood, breath, or other testing for Prohibited
  Substances other than SPEDs---as may be determined by the Medical
  Director in his or her professional judgment.
\item
  ``Medical Director'' shall mean the person selected by the NBA and the
  Players Association in accordance with Section 2(a) below.
\item
  ``Off-Season'' shall mean, for any given player, the day after the
  last game of that player's Team's Season and ending the day before the
  first day of that player's Team training camp.
\item
  ``Prohibited Substance'' shall mean any of the substances listed on
  Exhibit I-2 to this Agreement and any other substance added to such
  Exhibit under the provisions of Section 16 below.
\item
  ``Program'' shall mean this Anti-Drug Program, and shall include the
  Drugs of Abuse Program, the Marijuana Program, and the SPED Program.
\item
  ``Prohibited Substances Committee'' shall mean the committee selected
  by the NBA and the Players Association in accordance with Section 2(e)
  below.
\item
  ``SPED'' shall mean any of the steroids, performance-enhancing drugs
  and masking agents listed on Exhibit I-2 to this Agreement.
\item
  ``SPED Medical Director'' shall mean the person selected by the NBA
  and the Players Association in accordance with Section 2(b) below.
\item
  ``SPED Program'' shall mean the (i) testing program for SPEDs set
  forth in this Article XXXIII, and (ii) the education, treatment, and
  counseling program for SPEDs established by the SPED Medical Director
  (after consultation with the NBA and the Players Association), which
  may contain such elements---including, but not limited to, urine,
  blood, breath or other testing for SPEDs and Diuretics (but not for
  any other Prohibited Substance)---as may be determined by the SPED
  Medical Director in his or her professional judgment.
\item
  ``Tender'' shall mean an offer of a Uniform Player Contract, signed by
  the Team, that is either personally delivered to the player or his
  representative or sent by prepaid certified, registered, or overnight
  mail to the last known address of the player or his representative.
\item
  ``Veteran Player'' shall mean any player who is not a First-Year
  Player.
\end{enumerate}

\section{Administration.}\label{administration.}

\begin{enumerate}
\def\labelenumi{(\alph{enumi})}
\item
  The NBA and the Players Association shall jointly select a Medical
  Director who shall be a person experienced in the field of testing and
  treatment for substance abuse. The Medical Director shall have the
  responsibility, among other duties, for selecting and supervising the
  Counselors and other personnel necessary for the effective
  implementation of the Drugs of Abuse and Marijuana Programs, for
  making medical review determinations for Prohibited Substances other
  than SPEDs, for evaluating and treating players subject to such
  Programs, and for otherwise managing and overseeing such Programs,
  subject to the control of the NBA and the Players Association. To the
  extent practicable, the Medical Director shall select qualified
  retired NBA players to serve as Counselors.
\item
  The NBA and the Players Association shall jointly select a SPED
  Medical Director who shall be a medical doctor, preferably
  specializing in internal or sports medicine, with experience in the
  field of testing and treatment for steroids and performance-enhancing
  drugs. The SPED Medical Director shall have the responsibility, among
  other duties, for making medical review determinations for SPEDs, for
  evaluating and treating players subject to the SPED Program, and for
  otherwise managing and overseeing the SPED Program, subject to the
  control of the NBA and the Players Association. The Medical Director
  shall continue to perform the responsibilities of the SPED Medical
  Director until the first SPED Medical Director is selected pursuant to
  this Agreement.
\item
  The NBA and the Players Association shall jointly select an
  Independent Expert who shall be a person experienced in the field of
  substance abuse detection and enforcement and shall be authorized to
  issue Authorizations for Testing in accordance with Section 5 below.
\item
  The Medical Director, the SPED Medical Director and the Independent
  Expert shall all serve for the duration of this Agreement, unless
  either the NBA or the Players Association has, by September 1 of any
  year covered by this Agreement, served written notice of discharge
  upon the other party and, as appropriate, the Medical Director, SPED
  Medical Director and/or the Independent Expert. Such notice of
  discharge shall be effective as of the immediately following September
  30; provided, however, that if the parties do not reach agreement by
  such September 30 as to who shall serve thereafter as the Medical
  Director, SPED Medical Director and/or the Independent Expert, as the
  case may be, each party shall, by the immediately following October
  15, appoint a person who shall have no relationship to or affiliation
  with that party. Such persons shall then have until the immediately
  following December 1 to agree on the appointment of a new Medical
  Director, SPED Medical Director and/or Independent Expert. Until a new
  Medical Director, SPED Medical Director and/or Independent Expert has
  been appointed, the previous Medical Director, SPED Medical Director
  and/or Independent Expert shall continue to serve.
\item
  \begin{enumerate}
  \def\labelenumii{(\roman{enumii})}
  \tightlist
  \item
    The NBA and the Players Association shall form a Prohibited
    Substance Committee, which shall be comprised of one (1)
    representative from the NBA, one (1) representative from the Players
    Association, and three (3) individuals jointly selected by the NBA
    and the Players Association who shall be experts in the field of
    testing and treatment for drugs of abuse and performance-enhancing
    substances. The members of this Committee shall serve for the
    duration of the Agreement.
  \item
    The members of the Prohibited Substances Committee shall meet
    (either in person or by conference call) at least once each Season
    and once each off-season (the ``Annual Meetings''). The Annual
    Meetings shall be scheduled by the NBA after consultation with the
    Players Association. At the Annual Meetings, the Committee shall
    review the Program's list of Prohibited Substances, and discuss
    general anti-doping issues (including, but not limited to, advances
    in drug testing science and technology, and modifications to
    relevant anti-doping policies of other sports organizations). The
    Committee shall also make recommendations to the NBA and NBPA for
    changes to the list of Prohibited Substances (including the
    determination of laboratory analysis cutoff levels).
  \item
    As of September 1, 2017, and as of each successive September 1,
    either of the parties to this Agreement may discharge any
    jointly-selected member of the Prohibited Substances Committee by
    serving thirty (30) days' prior notice upon that person and upon the
    other party to this Agreement. In the case of such discharge, or in
    the event a Committee member resigns, and if the parties are unable
    to agree on a replacement Committee member within thirty (30) days,
    then the parties shall request a list of seven (7) names of
    potential replacements prepared by the Medical Director and any
    remaining jointly-selected Committee members, and, within seven (7)
    days, shall select the necessary replacement by alternately striking
    names from the list until only one (1) remains.
  \end{enumerate}
\item
  Unless specifically stated otherwise in this Article XXXIII, all costs
  of the Program in excess of those covered by the NBA Players Group
  Health Plan, including the fees and expenses of the Medical Director,
  the SPED Medical Director, the Independent Expert, and the Prohibited
  Substances Committee shall be shared equally by the NBA and Players
  Association. The Players Association's share shall be paid by the NBA
  and included in Player Benefits under Article IV, Section 6(f) of this
  Agreement. The fees and expenses incurred by the NBA in conducting
  testing pursuant to Sections 5, 6 and 15 below shall be borne by the
  NBA.
\item
  Any and all disputes arising under this Article XXXIII shall be
  resolved in accordance with Article XXXI, Sections 2-7 and 15 of this
  Agreement; provided, however, that in any challenge to a decision,
  recommendation, or other conduct of the Medical Director, SPED Medical
  Director, Independent Expert, or Prohibited Substances Committee, or
  in any challenge to an action or process over which the Medical
  Director or the SPED Medical Director has supervision, the Grievance
  Arbitrator shall apply an ``arbitrary and capricious'' standard of
  review; and provided further that nothing in this Section 2(g) shall
  limit or otherwise affect paragraph 19 of the Uniform Player Contract.
\end{enumerate}

\section{Confidentiality.}\label{confidentiality.}

\begin{enumerate}
\def\labelenumi{(\alph{enumi})}
\tightlist
\item
  Other than as reasonably required in connection with the suspension or
  disqualification of a player, the NBA, the Teams, and the Players
  Association, and all of their members, affiliates, agents,
  consultants, and employees, are prohibited from publicly disclosing
  information about the diagnosis, treatment, prognosis, test results,
  compliance, or the fact of participation of a player in the Program
  (``Program Information''). If a player is suspended or disqualified
  for conduct involving a Drug of Abuse, Diuretic, or marijuana, the NBA
  shall not publicly disclose the particular Prohibited Substance
  involved, absent the agreement of the Players Association or the prior
  disclosure of such information by the player (or by a person
  authorized by the player to disclose such information). If a player is
  suspended or disqualified for conduct involving a SPED, the particular
  SPED shall be publicly disclosed along with the announcement of the
  applicable penalty.
\item
  The Medical Director, the SPED Medical Director, and the Counselors,
  and all of their affiliates, agents, consultants, and employees, are
  prohibited from publicly disclosing Program Information; provided,
  however, that the Medical Director and the SPED Medical Director shall
  not be prohibited from disclosing such information to the NBA and the
  Players Association.
\item
  The Independent Expert is prohibited from publicly disclosing any
  information supplied to him by the NBA or the Players Association
  pursuant to Section 5 below.
\item
  Members of the Prohibited Substances Committee are prohibited from
  publicly disclosing any information obtained by them in connection
  with their duties as Committee members. If a jointly-selected member
  of the Committee violates this Section 3(d), he shall be immediately
  discharged from the Committee.
\item
  Any Program Information that is publicly disclosed (i) under Section
  3(a) above, (ii) by the player, (iii) with the player's authorization,
  or (iv) through release by sources other than the NBA, NBA Teams, the
  Players Association, the Medical Director, the Counselors, the SPED
  Medical Director, the Independent Expert, or the Prohibited Substances
  Committee, or any of their members, affiliates, agents, consultants,
  and employees, will, after such disclosure, no longer be subject to
  the confidentiality provisions of this Section 3.
\item
  Other than as reasonably required by the Reasonable Cause Testing
  procedure set forth in Section 5 below, neither the NBA nor the
  Players Association shall divulge to any other person or entity
  (including their respective members, affiliates, agents, consultants,
  employees, and the player and Team involved):

  \begin{enumerate}
  \def\labelenumii{(\roman{enumii})}
  \tightlist
  \item
    that it has received information regarding the use, possession, or
    distribution of a Prohibited Substance by a player;
  \item
    that it is considering requesting, has requested, or has had a
    conference with the Independent Expert concerning the suspected use,
    possession, or distribution of a Prohibited Substance by a player;
  \item
    any information disclosed to the Independent Expert; or
  \item
    the results of any conference with the Independent Expert.
  \end{enumerate}
\item
  Notwithstanding anything to the contrary contained in Section 3(a)-(f)
  above, the NBA and the Players Association shall promptly advise and
  make available to each other all information either of them may have
  in their possession, custody, or control that provides cause to
  believe that a player is engaged in the use, possession, or
  distribution of a Prohibited Substance.
\item
  Nothing contained in this Section 3 shall prohibit a Team from
  providing to the NBA information concerning whether a player is
  engaged in the use, possession, or distribution of a Prohibited
  Substance. For clarity, this Section 3(h) does not permit a Team to
  provide information to the NBA in violation of Section 17(d) below.
\end{enumerate}

\section{Testing.}\label{testing.}

\begin{enumerate}
\def\labelenumi{(\alph{enumi})}
\tightlist
\item
  Testing conducted pursuant to this Article XXXIII, whether by the NBA,
  the Medical Director or the SPED Medical Director, shall be conducted
  in compliance with scientifically accepted analytical techniques. Such
  testing shall also comply with Section 4(b) below, the collection
  procedures described in Exhibit I-3 (for urine collections) and
  Exhibit I-4 (for blood collections) to this Agreement, and such
  additional procedures and protocols as may be established by the NBA
  (after consultation with the Players Association) or the Medical
  Director or the SPED Medical Director, as applicable (after
  consultation with the NBA and the Players Association). The NBA and
  the Medical Director or the SPED Medical Director, as applicable
  (after consultation with the Players Association), are authorized to
  retain such consultants and support services as are necessary and
  appropriate to administer and conduct such testing.
\item
  If a player is selected for random drug testing pursuant to Section 6
  below on a day he is scheduled to play a game, the following
  additional procedures will apply: (i) any blood testing must occur
  after the game; and (ii) for urine testing of a visiting team
  scheduled at game-day shoot-arounds, tests will be scheduled to occur
  before the shoot-around for that team commences, and for any tests
  that are not completed by the time the visiting team bus is scheduled
  to leave the arena or practice facility after the shoot-around is
  completed, the team will provide alternate transportation to the team
  hotel for any player that must remain at the arena or practice
  facility to complete the testing process and will ensure that a Team
  staff member remains with the affected player(s) and accompanies him
  or them back to the Team's hotel.
\item
  All tests conducted pursuant to this Article XXXIII shall be analyzed
  by laboratories selected by the NBA and the Players Association, and
  certified by the World Anti-Doping Agency or the Substance Abuse and
  Mental Health Services Administration (SAMHSA).
\item
  Any test conducted pursuant to this Article XXXIII will be considered
  ``positive'' for a Prohibited Substance under the following
  circumstances:

  \begin{enumerate}
  \def\labelenumii{(\roman{enumii})}
  \tightlist
  \item
    If the test is for a Prohibited Substance other than a SPED or
    Diuretic and it is confirmed by laboratory analysis at the levels
    set forth in Exhibit I-5.
  \item
    If the test is for a SPED, and it is confirmed by laboratory
    analysis at the levels set forth in Exhibit I-6.
  \item
    If a player refuses to submit to a test or cooperate fully with the
    testing process, without a reasonable explanation satisfactory to
    the Medical Director or the SPED Medical Director (for testing under
    the SPED Program only); provided, however, that the NBA will use its
    best efforts (A) to have the drug testing collectors immediately
    notify the NBA when any player refuses to submit to a test or
    cooperate fully with the testing process, and (B) to provide such
    information to the Players Association as soon as possible
    thereafter; and provided, further, that (C) following any player's
    refusal to submit to a test or failure to cooperate fully with the
    testing process, the drug testing collector shall wait ninety (90)
    minutes at the collection site, and (D) if the player submits to the
    test and cooperates fully with the testing process within such
    additional time, then his earlier refusal or failure to cooperate
    shall be excused and the test shall not be deemed positive under
    this Section 4(d).
  \item
    If the player fails to submit to a scheduled test, without a
    reasonable explanation satisfactory to the Medical Director or SPED
    Medical Director (for testing under the SPED Program only).
  \item
    If the player attempts to substitute, dilute, or adulterate a
    specimen sample or in any other manner alter a test result (other
    than by testing positive for a Diuretic).
  \item
    If the test is positive for a Diuretic, and it is confirmed by
    laboratory analysis at any detectable level.
  \end{enumerate}
\item
  The NBA shall promptly notify the Players Association of any positive
  test conducted by the NBA, and shall thereafter notify the player. The
  Medical Director or the SPED Medical Director (as applicable) shall
  promptly notify the player of any positive test conducted by the
  Medical Director or SPED Medical Director (as applicable); provided,
  however, that if the positive test will result in a penalty to be
  imposed on the player, the Medical Director or SPED Medical Director
  (as applicable) shall notify the NBA and the Players Association of
  the positive test result and the NBA shall thereafter notify the
  player of such result and such penalty.
\item
  Any player who is notified of a positive test pursuant to Section 4(e)
  above may, within five (5) business days of such notification, inform
  the NBA and the Players Association that he requests testing of the
  split or ``B'' sample of his specimen. The test of the ``B'' sample
  will be performed at a laboratory other than the laboratory that
  performed the test on the original or ``A'' sample. Any such test
  shall be subject to the provisions of this Section 4 and shall be sent
  to the laboratory for testing within ten (10) business days of the
  player's request.
\item
  Any positive test pursuant to Section 4(d)(i) or (vi) above shall be
  reviewed by the Medical Director. Any positive test pursuant to
  Section 4(d)(ii) shall be reviewed by the SPED Medical Director. If
  the Medical Director or SPED Medical Director (as applicable)
  determines, in his professional judgment, that there is a valid
  alternative medical explanation for such positive test result, then
  the test shall be deemed negative.
\item
  If the test result for any player is reported by the laboratory as
  ``invalid'' or ``endogenous steroids abnormally low,'' the NBA shall
  promptly notify the Players Association, and shall thereafter notify
  the player. In the event of such a test result, the player shall be
  required to submit to another test on a date determined by the NBA
  that is not more than thirty (30) days after the date of the original
  test (the ``Re-Test''). If the Re-Test results in (i) a positive test
  for a Drug of Abuse or a positive test under Section 4(d)(iii), (iv)
  or (v) above, the player shall immediately be dismissed and
  disqualified from any association with the NBA or its Teams in
  accordance with the provisions of Section 11(a) below; (ii) a positive
  test for marijuana, the player shall suffer the applicable
  consequences set forth in Section 8 below; (iii) a positive test for a
  SPED, the player shall suffer the applicable consequences set forth in
  Section 9 below; or (iv) a positive test for a Diuretic, the player
  shall be deemed to have tested positive test for a SPED and shall
  suffer the applicable consequences set forth in Section 9 below. The
  original test will not be counted towards the number of tests to be
  administered to that player for that Season under Section 6 (Random
  Testing) below.
\end{enumerate}

\section{Reasonable Cause Testing or
Hearing.}\label{reasonable-cause-testing-or-hearing.}

\begin{enumerate}
\def\labelenumi{(\alph{enumi})}
\tightlist
\item
  In the event that either the NBA or the Players Association has
  information that gives it reasonable cause to believe that a player is
  engaged in the use, possession, or distribution of a Prohibited
  Substance, including information that a First-Year Player may have
  been engaged in such conduct during the period beginning three (3)
  months prior to his entry into the NBA, such party shall request a
  conference with the other party and the Independent Expert, which
  shall be held within twenty-four (24) hours or as soon thereafter as
  the Expert is available. Upon hearing the information presented, the
  Independent Expert shall immediately decide whether there is
  reasonable cause to believe that the player in question has been
  engaged in the use, possession, or distribution of a Prohibited
  Substance. If the Independent Expert decides that such reasonable
  cause exists, the Expert shall thereupon issue an Authorization for
  Testing with respect to such player.
\item
  In evaluating the information presented to him, the Independent Expert
  shall use his independent judgment based upon his experience in
  substance abuse detection and enforcement. The parties acknowledge
  that the type of information to be presented to the Independent Expert
  is likely to consist of reports of conversations with third parties of
  the type generally considered by law enforcement authorities to be
  reliable sources, and that such sources might not otherwise come
  forward if their identities were to become known. Accordingly, neither
  the NBA nor the Players Association shall be required to divulge to
  each other or to the Independent Expert the names (or other
  identifying characteristics) of their sources of information regarding
  the use, possession, or distribution of a Prohibited Substance, and
  the absence of such identification of sources, standing alone, shall
  not constitute a basis for the Expert to refuse to issue an
  Authorization for Testing with respect to a player. In conferences
  with the Independent Expert, the player involved shall not be
  identified by name until such time as the Expert has determined to
  issue an Authorization for Testing with respect to such player in the
  form set forth in Exhibit I-1 to this Agreement.
\item
  Immediately upon the Independent Expert's issuance of an Authorization
  for Testing with respect to a particular player, the NBA shall arrange
  for such player to undergo testing for Drugs of Abuse (if the
  Authorization for Testing was based on information regarding the use,
  possession, or distribution of a Drug of Abuse), for marijuana (if the
  authorization for Testing was based on information regarding the
  player's use, possession, or distribution of marijuana), or for SPEDs
  (if the Authorization for Testing was based on information regarding
  the player's use, possession, or distribution of a SPED) no more than
  four (4) times during the six-week period commencing with the issuance
  of the Authorization for Testing. Such testing may be administered at
  any time, in the discretion of the NBA, without prior notice to the
  player.
\item
  In the event that the player tests positive for a Drug of Abuse
  pursuant to this Section 5, or tests positive pursuant to Section
  4(d)(iii), (iv) or (v) above in connection with testing conducted
  pursuant to this Section 5, he shall immediately be dismissed and
  disqualified from any association with the NBA or any of its Teams in
  accordance with the provisions of Section 11(a) below. If the player
  tests positive for marijuana or a SPED pursuant to this Section 5, he
  shall enter the Program and suffer the applicable consequences set
  forth in Sections 8 or 9 below, as the case may be. If the player
  tests positive for a Diuretic, he shall suffer the applicable
  consequences of a positive test for the Prohibited Substance for which
  the Authorization for Testing was issued.
\item
  In the event that either the NBA or the Players Association determines
  that there is sufficient evidence to demonstrate that, within the
  previous year, a player has engaged in the use, possession, or
  distribution of a Prohibited Substance, or has received treatment for
  use of a Prohibited Substance other than in accordance with the terms
  of this Article XXXIII, it may, in lieu of requesting the testing
  procedure set forth in Section 5(a)-(d) above, request a hearing on
  the matter before the Grievance Arbitrator. If the Grievance
  Arbitrator concludes that, within the previous year, the Player has
  used, possessed, or distributed a Prohibited Substance, or has
  received treatment other than in accordance with the terms of this
  Article XXXIII, the player shall immediately be dismissed and
  disqualified from any association with the NBA or any of its Teams in
  accordance with the provisions of Section 11(a) below, notwithstanding
  the fact that the player has not undergone the testing procedure set
  forth in this Section 5; provided, however, that if the Grievance
  Arbitrator concludes that the player has used or possessed marijuana
  or a SPED, he shall enter the Program and suffer the applicable
  consequences set forth in Sections 8 or 9 below, as the case may be.
\end{enumerate}

\section{Random Testing.}\label{random-testing.}

\begin{enumerate}
\def\labelenumi{(\alph{enumi})}
\item
  In addition to the testing procedures set forth in Section 5 above, a
  player shall be required to undergo testing for Prohibited Substances
  at any time, without prior notice to the player, no more than four (4)
  times each Season and no more than two (2) times during each
  Off-Season. For purposes of this Section 6, the last day of a Season
  for a player shall be the day before that player's Off-Season begins.
  During each Season, the NBA will conduct no more than 1,525 total
  tests. During the Off-Season, the NBA will conduct no more than 600
  total tests. The scheduling of testing and collection of urine samples
  will be conducted according to a random player selection procedure by
  a third-party organization, and neither the NBA, the Players
  Association, any Team or any player will have any involvement in
  selecting the players to be tested or will receive prior notice of the
  testing schedule; provided, however, that it shall not be a violation
  of the foregoing for the third-party organization (or a specimen
  collector for the same) to provide advance notice of a scheduled
  collection to an NBA Team Security Representative, so long as such
  notice does not identify the player(s) who will be tested and seeks
  merely to facilitate access of the collector to the testing location.
  Urine samples collected during the Season will be tested for all
  Prohibited Substances; urine samples collected during the Off-Season
  will be tested for SPEDs and Diuretics only and may not under any
  circumstances be tested with respect to any other Prohibited
  Substances.
\item
  \begin{enumerate}
  \def\labelenumii{(\roman{enumii})}
  \tightlist
  \item
    In the event that a First-Year Player tests positive for a Drug of
    Abuse pursuant to this Section 6, he shall immediately be dismissed
    and disqualified from any association with the NBA or its Teams for
    a period of one (1) year, his Player Contract shall be rendered null
    and void and of no further force or effect (subject to the
    provisions of paragraph 8 of the Uniform Player Contract), and he
    shall enter Stage 1 of the Drugs of Abuse Program. Such dismissal
    and disqualification shall be mandatory and may not be rescinded or
    reduced by the player's Team or the NBA; provided, however, that
    such dismissal and disqualification, may be reduced or rescinded by
    the Grievance Arbitrator in accordance with Section 19 below.
  \item
    During any period while a First-Year Player is dismissed and
    disqualified from the NBA under Section 6(b)(i) above, and so long
    as such player is in compliance with his in- patient or aftercare
    obligations under the Program (as determined by the Medical
    Director), he shall receive from his Team a reasonable and necessary
    living expense stipend to be agreed upon by the NBA and the Players
    Association which (A) shall not exceed twenty-five percent (25\%) of
    the Salary that the player would otherwise have been entitled to
    earn for the period of his dismissal and disqualification and (B)
    shall not be payable for more than one (1) year from the date of
    such dismissal and disqualification.
  \item
    Any First-Year Player who tests positive for marijuana or a SPED
    pursuant to this Section 6, shall suffer the applicable consequences
    set forth in Sections 8 or 9 below, as the case may be. Any
    First-Year Player who tests positive for a Diuretic pursuant to this
    Section 6 shall be deemed to have tested positive for a SPED and
    shall suffer the applicable consequences set forth in Section 9
    below.
  \end{enumerate}
\item
  In the event that a Veteran Player tests positive for a Drug of Abuse
  pursuant to this Section 6, he shall immediately be dismissed and
  disqualified from any association with the NBA or any of its Teams in
  accordance with the provisions of Section 11(a) below; provided,
  however, that such dismissal and disqualification may be reduced or
  rescinded by the Grievance Arbitrator in accordance with Section 19
  below. If the player tests positive for marijuana or a SPED pursuant
  to this Section 6, he shall enter the Program and suffer the
  applicable consequences set forth in Sections 8 or 9 below, as the
  case may be. If the player tests positive for a Diuretic pursuant to
  this Section 6, he shall be deemed to have tested positive test for a
  SPED and shall suffer the applicable consequences set forth in Section
  9 below.
\item
  In the event that any player tests ``positive'' pursuant to Section
  4(d)(iii), (iv) or (v) above in connection with testing conducted
  pursuant to this Section 6, that positive test result shall be
  considered a positive test result for a Drug of Abuse, and the player
  shall immediately be dismissed and disqualified from any association
  with the NBA or any of its Teams in accordance with the provisions of
  Section 11(a) below.
\end{enumerate}

\section{Drugs of Abuse Program.}\label{drugs-of-abuse-program.}

\begin{enumerate}
\def\labelenumi{(\alph{enumi})}
\tightlist
\item
  Voluntary Entry.

  \begin{enumerate}
  \def\labelenumii{(\roman{enumii})}
  \tightlist
  \item
    A player may enter the Drugs of Abuse Program voluntarily at any
    time by Coming Forward Voluntarily for a problem involving the use
    of a Drug of Abuse; provided, however, that a player may not Come
    Forward Voluntarily (A) until he has been selected in an NBA Draft
    or invited to an NBA training camp; (B) during any period in which
    an Authorization for Testing as to that player remains in effect
    pursuant to Section 5 above; (C) during any period in which he
    remains subject to in-patient or aftercare treatment in Stage 1 of
    the Drugs of Abuse Program; or (D) after he has reached Stage 2 of
    the Drugs of Abuse Program.
  \item
    If a player who has not previously entered the Drugs of Abuse
    Program Comes Forward Voluntarily for a problem involving the use of
    a Drug of Abuse, he shall enter Stage 1 of the Drugs of Abuse
    Program.
  \item
    If a player who has not previously entered Stage 2 of the Drugs of
    Abuse Program, but who has been notified by the Medical Director
    that he has successfully completed Stage 1 of that Program, Comes
    Forward Voluntarily for a problem involving the use of a Drug of
    Abuse, he shall enter Stage 2 of the Drugs of Abuse Program.
  \item
    No penalty of any kind will be imposed on a player as a result of
    having Come Forward Voluntarily for a problem involving the use of a
    Drug of Abuse. The foregoing sentence shall not preclude the
    imposition of a penalty under Section 7(c)(iv) below as a result of
    the player's entering Stage 2 of the Drugs of Abuse Program, or any
    penalty called for by this Article XXXIII as a result of conduct by
    the player that occurs after he has Come Forward Voluntarily.
  \end{enumerate}
\item
  Stage 1.

  \begin{enumerate}
  \def\labelenumii{(\roman{enumii})}
  \tightlist
  \item
    Any player who has entered Stage 1 of the Drugs of Abuse Program
    shall be required to submit to an evaluation by the Medical
    Director, provide (or cause to be provided) to the Medical Director
    such relevant medical and treatment records as the Medical Director
    may request, and commence the treatment and testing program
    prescribed by the Medical Director.
  \item
    If a player, within ten (10) days of the date on which he was
    notified that he had entered Stage 1 of the Drugs of Abuse Program
    and without a reasonable excuse, fails to comply (in the
    professional judgment of the Medical Director) with any of the
    obligations set forth in Section 7(b)(i) above, he shall be
    suspended until such time as the Medical Director determines that he
    has fully complied with Section 7(b)(i) above. If such noncompliance
    continues without a reasonable excuse (in the professional judgment
    of the Medical Director) for thirty (30) days from the date on which
    the player was notified that he had entered Stage 1 of the Drugs of
    Abuse Program, the player shall, following notice of the player's
    non-compliance by the Medical Director to the NBA and then by the
    NBA to the player's Team (notwithstanding the provisions of Section
    3 above), (A) advance to Stage 2 of the Drugs of Abuse Program, or
    (B) the player's Team may, notwithstanding any term or provision in
    or amendment to the player's Uniform Player Contract, elect to
    terminate such Contract without any further obligation to pay
    Compensation, except to pay the Compensation (either Current or
    Deferred) that may have been earned by the player to the date of
    termination.
  \item
    Except as provided in this Article XXXIII, no penalty of any kind
    will be imposed on a player while he is in Stage 1 of the Drugs of
    Abuse Program and, provided he complies with the terms of his
    prescribed treatment, he will continue to receive his Compensation
    during the term of his treatment for a period of up to three (3)
    months of care in an In-Patient Facility and such aftercare as may
    be required by the Medical Director.
  \end{enumerate}
\item
  Stage 2.

  \begin{enumerate}
  \def\labelenumii{(\roman{enumii})}
  \tightlist
  \item
    Any player who has entered Stage 2 of the Drugs of Abuse Program
    shall be required to submit to an evaluation by the Medical
    Director, provide (or cause to be provided) to the Medical Director
    such relevant medical and treatment records as the Medical Director
    may request, and commence the treatment and testing program
    prescribed by the Medical Director.
  \item
    If a player, within thirty (30) days of the date on which he was
    notified that he had entered Stage 2 of the Drugs of Abuse Program
    and without a reasonable excuse, fails to comply (in the
    professional judgment of the Medical Director) with any of the
    obligations set forth in Section 7(c)(i) above, he shall immediately
    be dismissed and disqualified from any association with the NBA or
    any of its Teams in accordance with the provisions of Section 11(a)
    below.
  \item
    A player in Stage 2 of the Drugs of Abuse Program shall be suspended
    during the period of his in-patient treatment and for at least the
    first six (6) months of his aftercare treatment. The player shall
    remain suspended during any subsequent period in which he is
    undergoing treatment that, in the professional judgment of the
    Medical Director, prevents him from rendering the playing services
    called for by his Uniform Player Contract.
  \item
    Any subsequent use, possession, or distribution of a Drug of Abuse
    by a player in Stage 2, even if voluntarily disclosed, or any
    conduct by a player in Stage 2 that results in his advancing one (1)
    Stage in the Drugs of Abuse Program, shall result in the player
    being immediately dismissed and disqualified from any association
    with the NBA or any of its Teams in accordance with the provisions
    of Section 11(a) below.
  \end{enumerate}
\item
  Treatment and Testing Program. A player who enters the Drugs of Abuse
  Program shall be required to comply with such in-patient and aftercare
  program as may be prescribed and supplemented from time to time by the
  Medical Director. Such program may include random testing for
  Prohibited Substances other than SPEDs, and for alcohol, and such
  non-testing elements as may be determined in the professional judgment
  of the Medical Director.
\end{enumerate}

\section{Marijuana Program.}\label{marijuana-program.}

\begin{enumerate}
\def\labelenumi{(\alph{enumi})}
\tightlist
\item
  Voluntary Entry.

  \begin{enumerate}
  \def\labelenumii{(\roman{enumii})}
  \tightlist
  \item
    A player may enter the Marijuana Program voluntarily at any time by
    Coming Forward Voluntarily; provided, however, that a player may not
    Come Forward Voluntarily for a problem involving the use of
    marijuana (A) until he has been selected in an NBA Draft or invited
    to an NBA training camp; (B) during any period in which an
    Authorization for Testing as to that player remains in effect
    pursuant to Section 5 above; or (C) during any period in which he
    remains subject to in-patient or aftercare treatment in the
    Marijuana Program.
  \item
    If a player who has not previously entered the Marijuana Program, or
    a player who has been notified by the Medical Director that he has
    successfully completed that Program, Comes Forward Voluntarily for a
    problem involving the use of marijuana, he shall enter the Marijuana
    Program.
  \item
    No penalty of any kind will be imposed on a player as a result of
    having Come Forward Voluntarily for a problem involving the use of
    marijuana. The foregoing sentence shall not preclude the imposition
    of any penalty called for by this Article XXXIII as a result of
    conduct by the player that occurs after he has Come Forward
    Voluntarily.
  \end{enumerate}
\item
  Treatment.

  \begin{enumerate}
  \def\labelenumii{(\roman{enumii})}
  \tightlist
  \item
    A player who enters the Marijuana Program shall be required to
    submit to an evaluation by the Medical Director, provide (or cause
    to be provided) to the Medical Director such relevant medical and
    treatment records as the Medical Director may request, and commence
    the treatment and testing program prescribed by the Medical
    Director. Such program may include random testing for Prohibited
    Substances other than SPEDs, and for alcohol, and such non-testing
    elements as may be determined in the professional judgment of the
    Medical Director.
  \item
    If a player, within five (5) days of the date on which he was
    notified that he had entered the Marijuana Program and without a
    reasonable excuse, fails to comply (in the professional judgment of
    the Medical Director) with any of the obligations set forth in the
    first sentence of Section 8(b)(i) above, he shall be fined \$10,000;
    if the player thereafter fails to comply, without a reasonable
    excuse, with such obligations (in the professional judgment of the
    Medical Director) within eight (8) days of such notification, he
    shall be fined an additional \$10,000; and for each additional day
    beyond the 8th day that the player, without a reasonable excuse,
    fails to comply with such obligations (in the professional judgment
    of the Medical Director), he shall be fined an additional \$10,000.
    The total amount of such fines may not exceed the player's total
    Compensation.
  \end{enumerate}
\item
  Penalties. Any player who (i) tests positive for marijuana pursuant to
  Section 5 (Reasonable Cause Testing), Section 6 (Random Testing), or
  Section 15 (Additional Bases for Testing), (ii) is adjudged by the
  Grievance Arbitrator pursuant to Section 5(e) above to have used or
  possessed marijuana, or (iii) has been convicted of (including a plea
  of guilty, no contest or nolo contendere to) the use or possession of
  marijuana in violation of the law, shall suffer the following
  penalties:

  \begin{enumerate}
  \def\labelenumii{(\Alph{enumii})}
  \tightlist
  \item
    For the first such violation, the player shall be required to enter
    the Marijuana Program;
  \item
    For the second such violation, the player shall be fined \$25,000
    and, if the player is not then subject to in-patient or aftercare
    treatment in the Marijuana Program, be required to enter the
    Marijuana Program;
  \item
    For the third such violation, the player shall be suspended for five
    (5) games and, if the player is not then subject to in-patient or
    aftercare treatment in the Marijuana Program, be required to enter
    the Marijuana Program; and
  \item
    For any subsequent violation, the player shall be suspended for five
    (5) games longer than his immediately-preceding suspension for
    violating the Marijuana Program and, if the player is not then
    subject to in-patient or aftercare treatment in the Marijuana
    Program, be required to enter the Marijuana Program.
  \end{enumerate}
\end{enumerate}

\section{Steroids and Performance-Enhancing Drugs
Program.}\label{steroids-and-performance-enhancing-drugs-program.}

\begin{enumerate}
\def\labelenumi{(\alph{enumi})}
\tightlist
\item
  Treatment.

  \begin{enumerate}
  \def\labelenumii{(\roman{enumii})}
  \tightlist
  \item
    A player who enters the SPED Program shall be required to submit to
    an evaluation by the SPED Medical Director, provide (or cause to be
    provided) to the SPED Medical Director such relevant medical and
    treatment records as the SPED Medical Director may request, and
    commence the treatment and testing program prescribed by the SPED
    Medical Director. Such program may include random testing for SPEDs
    and Diuretics and such non-testing elements as may be determined in
    the professional judgment of the SPED Medical Director.
  \item
    If a player, within five (5) days of the date on which he was
    notified that he had entered the SPED Program and without a
    reasonable excuse, fails to comply (in the professional judgment of
    the SPED Medical Director) with any of the obligations set forth in
    the first sentence of Section 9(a)(i) above, he shall be fined
    \$10,000; if the player, without a reasonable excuse, thereafter
    fails to comply with such obligations (in the professional judgment
    of the SPED Medical Director) within eight (8) days of such
    notification, he shall be fined an additional \$10,000; and for each
    additional day beyond the 8th day that the player, without a
    reasonable excuse, fails to comply with such obligations (in the
    professional judgment of the SPED Medical Director), he shall be
    fined an additional \$10,000. The total amount of such fines shall
    not exceed the player's total Compensation.
  \end{enumerate}
\item
  Penalties. Any player who (i) tests positive for a SPED pursuant to
  Section 5 (Reasonable Cause Testing), Section 6 (Random Testing), or
  Section 15 (Additional Bases for Testing), or (ii) is adjudged by the
  Grievance Arbitrator pursuant to Section 5(e) above to have used or
  possessed a SPED, shall suffer the following penalties:

  \begin{enumerate}
  \def\labelenumii{(\Alph{enumii})}
  \tightlist
  \item
    For the first such violation, the player shall be suspended for
    twenty -five (25) games and required to enter the SPED Program;
  \item
    for the second such violation, the player shall be suspended for
    fifty-five (55) games and, if the player is not then subject to
    in-patient or aftercare treatment in the SPED Program, be required
    to enter the SPED Program; and
  \item
    for the third such violation, the player shall be immediately
    dismissed and disqualified from any association with the NBA or any
    of its Teams in accordance with the provisions of Section 11(a)
    below.
  \end{enumerate}
\item
  The penalties set forth in Section 9(b)(i) above and Section 9(b)(ii)
  above with respect to a player's use of a SPED may be reduced or
  rescinded by the Grievance Arbitrator in accordance with Section 19
  below.
\end{enumerate}

\section{Noncompliance with
Treatment.}\label{noncompliance-with-treatment.}

\begin{enumerate}
\def\labelenumi{(\alph{enumi})}
\tightlist
\item
  Drugs of Abuse.

  \begin{enumerate}
  \def\labelenumii{(\roman{enumii})}
  \tightlist
  \item
    Any player who, after entering Stage 1 or Stage 2 of the Drugs of
    Abuse Program, fails to comply with his treatment or his aftercare
    program as prescribed and determined by the Medical Director, shall
    be suspended. Such suspension shall continue until the player has,
    in the professional judgment of the Medical Director, resumed full
    compliance with his treatment program.
  \item
    Notwithstanding Section 10(a)(i) above, any player who in the
    professional judgment of the Medical Director, after entering Stage
    1 or Stage 2 of the Drugs of Abuse Program, fails to comply with his
    treatment program through (A) a pattern of behavior that
    demonstrates a mindful disregard for his treatment responsibilities,
    or (B) a positive test for a Prohibited Substance other than a SPED
    that is not clinically expected by the Medical Director, shall
    suffer the following penalties:

    \begin{enumerate}
    \def\labelenumiii{(\arabic{enumiii})}
    \tightlist
    \item
      if the player is in Stage 1 of the Drugs of Abuse Program, he
      shall advance to Stage 2 and be suspended until, in the
      professional judgment of the Medical Director, he has resumed full
      compliance with his treatment program; or
    \item
      if the player already is in Stage 2 of the Drugs of Abuse Program,
      he shall immediately be dismissed and disqualified from any
      association with the NBA or any of its Teams in accordance with
      the provisions of Section 11(a) below.
    \end{enumerate}
  \end{enumerate}
\item
  Marijuana.

  \begin{enumerate}
  \def\labelenumii{(\roman{enumii})}
  \tightlist
  \item
    Any player who, after entering the Marijuana Program, fails to
    comply (without a reasonable excuse) with his treatment program as
    prescribed and determined by the Medical Director, shall be fined
    \$5,000 for each day that he fails to comply. Such fines shall
    continue until the player has, in the professional judgment of the
    Medical Director, resumed full compliance with his treatment
    program. The total amount of such fines shall not exceed the
    player's total Compensation.
  \item
    Notwithstanding Section 10(b)(i) above, any player who, after
    entering the Marijuana Program, fails to comply with his treatment
    program as prescribed and determined by the Medical Director through
    (A) a pattern of behavior that demonstrates a mindful disregard for
    his treatment responsibilities, or (B) a positive test for marijuana
    that is not clinically expected by the Medical Director, shall
    suffer the following penalties:

    \begin{enumerate}
    \def\labelenumiii{(\arabic{enumiii})}
    \tightlist
    \item
      if the player has not previously been fined \$25,000 under Section
      8(c) above or this Section 10(b)(ii), a fine of \$25,000;
    \item
      if the player has previously been fined \$25,000 under Section
      8(c) above or this Section 10(b)(ii), a suspension of five (5)
      games; or
    \item
      if the player has previously been suspended for five (5) or more
      games under Section 8(c) above or this Section 10(b)(ii), a
      suspension that is at least five (5) games longer than his
      immediately-preceding suspension and that shall continue until, in
      the professional judgment of the Medical Director, the player
      resumes full compliance with his treatment program.
    \end{enumerate}
  \item
    In addition to any consequence to the player under Section 10(b)(ii)
    above, any player who has entered the Marijuana Program but not the
    Drugs of Abuse Program, and tests positive for a Drug of Abuse in
    any test conducted by the Medical Director, shall enter Stage 1 of
    the Drugs of Abuse Program.
  \end{enumerate}
\item
  SPEDs.

  \begin{enumerate}
  \def\labelenumii{(\roman{enumii})}
  \tightlist
  \item
    Any player who, after entering the SPED Program, fails to comply
    (without a reasonable excuse) with his treatment program as
    prescribed and determined by the SPED Medical Director, shall be
    fined \$5,000 per day for each day that he fails to comply. Such
    fines shall continue until the player has, in the professional
    judgment of the SPED Medical Director, resumed full compliance with
    his treatment program. The total amount of such fines shall not
    exceed the player's total Compensation.
  \item
    Notwithstanding Section 10(c)(i) above, any player who, after
    entering the SPED Program, fails to comply with his treatment
    program as prescribed and determined by the SPED Medical Director
    through (A) a pattern of behavior that demonstrates a mindful
    disregard for his treatment responsibilities, or (B) a positive test
    for a SPED that is not clinically expected by the SPED Medical
    Director, shall suffer the following penalties:

    \begin{enumerate}
    \def\labelenumiii{(\arabic{enumiii})}
    \tightlist
    \item
      if the player has not previously been suspended for twenty-five
      (25) games under Section 9(b) above or this Section 10(c)(ii), a
      suspension of twenty-five (25) games;
    \item
      if the player has previously been suspended for twenty-five (25)
      games under Section 9(b) above or this Section 10(c)(ii), a
      suspension of fifty-five (55) games; or
    \item
      if the player has been previously suspended for fifty-five (55)
      games under Section 9(b) above or this Section 10(c)(ii), the
      player shall be immediately dismissed and disqualified from any
      association with the NBA or any of its Teams in accordance with
      the provisions of Section 11(a) below.
    \end{enumerate}
  \end{enumerate}
\item
  Directed Testing. Any player who, after entering the Program, and
  without a reasonable explanation satisfactory to the Medical Director,
  (i) fails to appear for any of his Team's scheduled games, or (ii)
  misses, during any consecutive seven-day (7) period, any two (2)
  airplane flights on which his team is scheduled to travel, any two (2)
  Team practices, or a combination of any one (1) practice and any one
  (1) Team flight, shall immediately submit to a urine test to be
  conducted by the NBA. If any test conducted pursuant to this Section
  10(d) is positive: (x) for a Drug of Abuse or pursuant to Section
  4(d)(iii), (iv) or (v) above (for a player in the Drugs of Abuse
  Program), then the player shall suffer the applicable consequence set
  forth in Section 10(a)(ii) above; (y) for marijuana or pursuant to
  Section 4(d)(iii), (iv) or (v) above (for a player in the Marijuana
  Program), then the player shall suffer the applicable consequence set
  forth in Section 10(b)(ii) above; or (z) for a SPED or pursuant to
  Section 4(d)(iii), (iv) or (v) above (for a player in the SPED
  Program), then the player will suffer the applicable consequence set
  forth in Section 10(c)(ii) above. If any test conducted pursuant to
  this Section 10(d) is positive for a Diuretic, then the player shall
  suffer the applicable consequences of a positive test for the
  Prohibited Substance for which he entered the Program.
\end{enumerate}

\section{Dismissal and
Disqualification.}\label{dismissal-and-disqualification.}

\begin{enumerate}
\def\labelenumi{(\alph{enumi})}
\tightlist
\item
  A player who, under the terms of this Agreement, is ``dismissed and
  disqualified from any association with the NBA or any of its Teams in
  accordance with the provisions of Section 11(a)'' shall, without
  exception, immediately be so dismissed and disqualified for a period
  of not less than two (2) years, and such player's Player Contract
  shall be rendered null and void and of no further force or effect
  (subject to the provisions of paragraph 8 of the Uniform Player
  Contract). Such dismissal and disqualification shall be mandatory and
  may not be rescinded or reduced by the player's Team or the NBA.
\item
  In addition to any other provision of this Agreement requiring that a
  player be dismissed and disqualified from any association with the NBA
  or any of its Teams in accordance with the provisions of Section 11(a)
  above, a player will also be dismissed and disqualified under Section
  11(a) above if he is convicted of (including a plea of guilty, no
  contest, or nolo contendere to) a crime involving the use, possession
  or distribution of a Prohibited Substance other than marijuana or a
  felony involving the distribution of marijuana.
\end{enumerate}

\section{Reinstatement.}\label{reinstatement.}

\begin{enumerate}
\def\labelenumi{(\alph{enumi})}
\item
  After a period of at least two (2) years from the time of a player's
  dismissal and disqualification under Section 11(a) above, and after a
  period of at least one (1) year from the date of a First-Year Player's
  dismissal and disqualification under Section 6(b) above, such player
  may apply for reinstatement as a player in the NBA. However, such
  player shall have no right to reinstatement under any circumstance and
  the reinstatement shall be granted only with the prior approval of
  both the NBA and the Players Association, which shall not be
  unreasonably withheld. The approval of the NBA and the Players
  Association shall rest in their absolute and sole discretion, and
  their decision shall be final, binding, and unappealable. Among the
  factors that may be considered by the NBA and the Players Association
  in determining whether to grant reinstatement are (without
  limitation): the circumstances surrounding the player's dismissal and
  disqualification; whether the player has satisfactorily completed a
  treatment and rehabilitation program; the player's conduct since his
  dismissal, including the extent to which the player has since
  comported himself as a suitable role model for youth; and whether the
  player is judged to possess the requisite qualities of good character
  and morality.
\item
  For a First-Year Player, the NBA and the Players Association will
  consider an application for reinstatement only if the player has, in
  the opinion of the Medical Director or the SPED Medical Director (as
  applicable), successfully completed any in-patient treatment and/or
  aftercare prescribed by the Medical Director or the SPED Medical
  Director (as applicable). For a Veteran Player who was dismissed and
  disqualified under Section 11(a) above in connection with a Drug of
  Abuse, the NBA and the Players Association will consider any
  application for reinstatement only if the player can demonstrate, by
  proof of random urine testing acceptable to the Medical Director
  (conducted on at least a weekly basis), that he has not tested
  positive (i) for a Drug of Abuse or Marijuana within the twelve (12)
  months prior to the submission of his application for reinstatement
  and during any period while his application is being reviewed, and
  (ii) if the Medical Director deems it necessary in his or her
  professional judgment, for alcohol for the six (6) months prior to the
  submission of his application for reinstatement and during any period
  while his application is being reviewed. For a Veteran Player who was
  dismissed and disqualified under Section 11(a) above in connection
  with a SPED, the NBA and the Players Association will consider any
  application for reinstatement only if the player can demonstrate, by
  proof of random urine and/or blood testing acceptable to the SPED
  Medical Director (conducted on at least a weekly basis), that he has
  not tested positive for a SPED within the twelve (12) months prior to
  the submission of his application for reinstatement and during any
  period while his application is being reviewed.
\item
  The granting of an application for reinstatement may be conditioned
  upon random testing of the player or such other terms as may be agreed
  upon by the NBA and the Players Association, whether or not such terms
  are contemplated by the terms of this Article XXXIII.
\item
  Any player who has been reinstated pursuant to this Section 12 and is
  subsequently dismissed and disqualified from any association with the
  NBA or any of its Teams in accordance with the provisions of Section
  11(a) above shall therefore be ineligible for reinstatement pursuant
  to this Section 12.
\item
  In the event that the application for reinstatement of a First-Year
  Player dismissed and disqualified pursuant to Section 6(b) above is
  approved, such player, by reason of his Player Contract having been
  rendered null and void pursuant to Section 6(b) above, shall be deemed
  not to have completed his Player Contract by rendering the playing
  services called for thereunder. Accordingly, such player shall not be
  a Free Agent and shall not be entitled to negotiate or sign a Player
  Contract with any NBA Team, except as specifically provided in this
  Section 12.
\item
  \begin{enumerate}
  \def\labelenumii{(\roman{enumii})}
  \tightlist
  \item
    A First-Year Player who has been reinstated pursuant to this Section
    12 shall, immediately upon such reinstatement, notify the Team to
    which he was under contract at the time of his dismissal and
    disqualification (the ``previous Team''). Upon receipt of such
    notification, and subject to Section 12(f)(ii) below, the previous
    Team shall then have thirty (30) days in which to make a Tender to
    the player with a stated term of at least one (1) full NBA Season
    (or, in the event that the Tender is made during a Season, of at
    least the remainder of that Season) and calling for at least the
    Minimum Player Salary then applicable to that player but not more
    than the Salary provided for in Section 12(f)(iii) below. If the
    previous Team makes such a Tender, it shall, for a period of one (1)
    year from the date of the Tender, be the only NBA Team with which
    the player may negotiate and sign a Player Contract. If the player
    does not sign a Player Contract with the previous Team within the
    year following such Tender, the player shall thereupon be deemed a
    Restricted Free Agent, subject to a Right of First Refusal. If the
    previous Team fails to make a Tender, the player shall become an
    Unrestricted Free Agent.
  \item
    Notwithstanding anything to the contrary in Section 12(f)(i) above,
    the 30-day period for the previous Team to make a Tender shall be
    tolled if (A) on the date the player serves the notice required by
    Section 12(f)(i), he is under contract to a professional basketball
    team not in the NBA, or (B) the player signs a contract with a
    professional basketball team not in the NBA at any point after the
    date on which the player serves the notice required by Section
    12(f)(i) and before the date on which the previous Team makes a
    Tender. If the 30-day period for making a Tender is tolled pursuant
    to the preceding sentence, the period shall remain tolled until the
    date on which the player notifies the Team that he is immediately
    available to sign and begin rendering playing services under a
    Player Contract with such Team, provided that such notice will not
    be effective until the player is under no contractual or other legal
    impediment to sign with and begin rendering playing services for
    such team.
  \item
    A First-Year Player who is reinstated pursuant to this Section 12
    may enter into a Player Contract with his previous Team that
    provides for a Salary and Unlikely Bonuses for the first Season of
    up to the Player's Salary and Unlikely Bonuses, respectively, for
    the Salary Cap Year in which he was dismissed and disqualified
    (reduced on a pro rata basis if the first Season of the new Contract
    is a partial Season), even if the Team has a Team Salary at or above
    the Salary Cap or such Player Contract causes the Team to have a
    Team Salary above the Salary Cap. If the player and the previous
    Team enter into such Player Contract and such Contract covers more
    than one Season, increases and decreases in Salary for Seasons
    following the first Season shall be governed by Article VII, Section
    5(c)(1); provided, however, that if the player who is reinstated was
    dismissed and disqualified during the term of his Rookie Scale
    Contract, then (A) the number of Seasons in the player's new
    Contract may not exceed two (2) Seasons plus two (2) Option Years in
    favor of the Team, and the Salary and Unlikely Bonuses called for in
    any Season of the player's new Contract, including any Option Year,
    may not exceed the Salary and Unlikely Bonuses called for during the
    corresponding Season of his Rookie Scale Contract, and (B) if the
    new Contract contains terms identical to those contained in the
    remaining Seasons of the Player's Rookie Scale Contract at the time
    he was dismissed and disqualified, and the Team exercises all Option
    Year(s) available under the new Contract, then the player's Team
    shall retain the same rights with respect to such new Contract as it
    would have retained under Article XI following the completion of the
    player's Rookie Scale Contract.
  \end{enumerate}
\item
  \begin{enumerate}
  \def\labelenumii{(\roman{enumii})}
  \tightlist
  \item
    A Veteran Player who has been reinstated pursuant to this Section 12
    shall, immediately upon such reinstatement, notify the Team to which
    he was under contract at the time of his dismissal and
    disqualification (the ``previous Team''). Upon receipt of such
    notification, and subject to Section 12(g)(ii) below, the previous
    Team shall then have thirty (30) days in which to make a Tender to
    the player with a stated term of at least one (1) full NBA Season
    (or, in the event the Tender is made during a Season, of at least
    the rest of that Season) and calling for a Salary in the first
    Season covered by the Tender at least equal to the lesser of (A) the
    player's Salary for the Salary Cap Year in which he was dismissed
    and disqualified, or (B) the Estimated Average Player Salary during
    the then-current Season, in either case reduced on a pro rata basis
    if the first Season covered by the Tender is a partial Season, but
    not greater than the Salary provided in Section 12(g)(iii) below. If
    the previous Team makes such a Tender, it shall, for a period of one
    (1) year from the date of the Tender, be the only NBA Team with
    which the player may negotiate and sign a Player Contract. If the
    player does not sign a Player Contract with the previous Team within
    the year following such Tender, then the player shall thereupon be
    deemed a Restricted or an Unrestricted Free Agent, in accordance
    with the provisions of Article XI. If the previous Team fails to
    make a Required Tender, the player shall become an Unrestricted Free
    Agent.
  \item
    Notwithstanding anything to the contrary in Section 12(g)(i) above,
    the 30-day period for the previous Team to make a Tender shall be
    tolled if (A) on the date the player serves the notice required by
    Section 12(g)(i), he is under contract to a professional basketball
    team not in the NBA, or (B) the player signs a contract with a
    professional basketball team not in the NBA at any point after the
    date on which he serves the notice required by Section 12(g)(i) and
    before the date on which the previous Team makes a Tender. If the
    30-day period for making a Tender is tolled pursuant to the
    preceding sentence, the period shall remain tolled until the date on
    which the player notifies the Team that he is available to sign a
    Player Contract with and begin rendering playing services for such
    Team immediately, provided that such notice will not be effective
    until the player is under no contractual or other legal impediment
    to sign with and begin rendering playing services for such Team.
  \item
    A Veteran Player who is reinstated pursuant to this Section 12 may
    enter into a Player Contract with his previous Team that provides
    for a Salary and Unlikely Bonuses for the first Season of up to the
    player's Salary and Unlikely Bonuses, respectively, for the Salary
    Cap Year in which he was dismissed and disqualified (reduced on a
    pro rata basis if the first Season of the new Contract is a partial
    Season), even if the Team has a Team Salary at or above the Salary
    Cap or such Player Contract causes the Team to have a Team Salary
    above the Salary Cap. If the player and the previous Team enter into
    such Player Contract and such Contract covers more than one (1)
    Season, increases and decreases in Salary for Seasons following the
    first Season shall be governed by Article VII, Section 5(c)(i);
    provided, however, that if the player who is reinstated was
    dismissed and disqualified during the term of his Rookie Scale
    Contract, then (A) the number of Seasons in the Player's new
    Contract may not exceed the number of Seasons (including the Option
    Year in favor of the Team) that remained under the player's Rookie
    Scale Contract at the time he was dismissed and disqualified, and
    the Salary called for in any Season of the Player's new Contract
    (including any Option Year), may not exceed the Salary called for
    during the corresponding Season of his Rookie Scale Contract, and
    (B) if the new Contract contains terms identical to those contained
    in the remaining Seasons of the player's Rookie Scale Contract at
    the time he was dismissed and disqualified, and the player's Team
    ultimately exercises the Option Year available under the new
    Contract, then such Team shall retain the same rights with respect
    to such new Contract as it would have retained under Article XI
    following the completion of the player's Rookie Scale Contract.
  \end{enumerate}
\end{enumerate}

\section{Exclusivity of the Program.}\label{exclusivity-of-the-program.}

\begin{enumerate}
\def\labelenumi{(\alph{enumi})}
\tightlist
\item
  Except as expressly provided in this Article XXXIII, there shall be no
  other screening or testing for Prohibited Substances conducted by the
  NBA or any Team, and no player may undergo such screening or testing;
  provided, however, that, in a medical emergency, team physicians may
  test players solely for diagnostic purposes in order to provide
  satisfactory medical care. The results of any diagnostic drug testing
  conducted pursuant to the preceding sentence shall not be used for any
  other purpose by the player's Team or the NBA. If any Team is found to
  have tested a player in violation of this Section 13, the NBA will
  impose a substantial fine not to exceed \$750,000 upon such Team
  pursuant to the NBA's Constitution and By-Laws.
\item
  The penalties set forth in this Article XXXIII shall be the exclusive
  penalties to be imposed upon a player for the use, possession or
  distribution of a Prohibited Substance.
\item
  No Uniform Player Contract entered into after the date hereof shall
  include any term or provision that modifies, contradicts, changes, or
  is inconsistent with paragraph 8 of such Contract (including any
  condition or limitation on salary protection other than the standard
  conditions or limitations specifically provided for in Article II,
  Section 4) or provides for the testing of a player for illegal
  substances. Any term or provision of a currently effective Uniform
  Player Contract that is inconsistent with paragraph 8 of such Contract
  shall be deemed null and void only to the extent of the inconsistency.
\end{enumerate}

\section{Random HGH Blood Testing.}\label{random-hgh-blood-testing.}

\begin{enumerate}
\def\labelenumi{(\alph{enumi})}
\tightlist
\item
  In addition to the testing procedures set forth in Section 5 above, a
  player shall be required to undergo HGH Blood Testing at any time,
  without prior notice to the player, no more than two (2) times each
  Season and no more than one (1) time during each Off-Season. For
  purposes of this Section 14, the last day of a Season for a player
  shall be the day before that player's Off-Season begins. The
  scheduling of testing and collection of blood samples will be
  conducted according to a random player selection procedure by a
  third-party organization, and neither the NBA, the Players
  Association, any Team or any player will have any involvement in
  selecting the players to be tested or will receive prior notice of the
  testing schedule; provided, however, that it shall not be a violation
  of the foregoing for the third-party organization (or a specimen
  collector for the same) to provideadvance notice of a scheduled
  collection to an NBA Team Security Representative, so long as such
  notice does not identify the player(s) who will be tested and seeks
  merely to facilitate access of the collector to the testing location.
  HGH Blood Testing may also take place under Section 5 (Reasonable
  Cause Testing) and Section 9 (Steroids and Performance-Enhancing Drugs
  Program) above, and Section 15 (Additional Bases for Testing) below.
  (For clarity, the number of random blood tests for Human Growth
  Hormone pursuant to this Section 14 shall be in addition to the number
  of random urine tests for other Prohibited Substances called for in
  Section 6 above.) HGH Blood Testing may occur at the same time that
  players undergo random urine tests for other SPEDs, subject to the
  procedures governing game-day blood testing set forth in Section 4(b)
  above and Exhibit I-4 to this Agreement.
\item
  In the event that a player tests positive for a Human Growth Hormone
  pursuant to this Section 14, he shall enter the SPED Program and
  suffer the consequences set forth in Section 9 above.
\item
  The isoform test for HGH blood testing will be used with corresponding
  decision limits issued by the World Anti-Doping Agency in June of 2014
  (the ``WADA Decision Limits'') for positive test results. (The WADA
  Decision Limits are 1.84 for kit 1 and 1.91 for kit 2 for male
  athletes.)
\end{enumerate}

\section{Additional Bases for
Testing.}\label{additional-bases-for-testing.}

\begin{enumerate}
\def\labelenumi{(\alph{enumi})}
\tightlist
\item
  Any player who seeks treatment outside the Program for a problem
  involving a Prohibited Substance shall, as directed by the NBA (after
  notice to the Players Association), submit himself to an evaluation by
  the Medical Director or SPED Medical Director (as applicable) and
  provide (or cause to be provided) to the Medical Director or SPED
  Medical Director (as applicable) such medical and treatment records as
  the Medical Director or SPED Medical Director (as applicable) may
  request. The Medical Director or SPED Medical Director (as applicable)
  may, in his or her professional judgment, also require such a player,
  without prior notice, to submit to testing for Prohibited Substances,
  provided that the frequency of such testing shall not exceed three (3)
  times per week and the duration of such testing shall not exceed one
  (1) year from the date of the player's initial evaluation by the
  Medical Director or SPED Medical Director (as applicable).
\item
  Any player who is subject to in-patient or aftercare treatment in the
  Program and is formally charged with ``driving while intoxicated,''
  ``driving under the influence of alcohol,'' or any other crime or
  offense involving suspected alcohol or illegal substance use shall,
  provided that the NBA has advised the Players Association, be required
  to submit to a urine test, to be conducted by the NBA, within seven
  (7) days of being so charged.
\item
  If, pursuant to Section 15(a) above, a player (i) tests positive for a
  Drug of Abuse; (ii) tests positive pursuant to Section 4(d)(iii), (iv)
  or (v) above; or (iii) refuses or fails to submit to an evaluation or
  provide (or cause to be provided) the information requested by the
  Medical Director, but does not Come Forward Voluntarily within sixty
  (60) days of being requested to do so by the NBA (with notice to the
  Players Association), or if, pursuant to Section 15(b) above, a player
  tests positive for a Drug of Abuse, then, in either case the player
  shall advance two stages in the Drugs of Abuse Program---i.e., the
  player shall enter Stage 2 of the Drugs of Abuse Program (if the
  player had not previously entered Stage 1 of such Program), and the
  player shall be dismissed and disqualified from any association with
  the NBA or any of its Teams in accordance with the provisions of
  Section 11(a) above (if the player had previously entered Stage 1 or
  Stage 2 of such Program).
\item
  If, pursuant to Section 15(a) or (b) above, a player tests positive
  for marijuana or a SPED, he shall suffer the applicable consequences
  set forth in Sections 8 or 9 above, as the case may be. If, pursuant
  to Section 15(a) or (b) above, a player tests positive for a Diuretic,
  he shall be deemed to have tested positive for a SPED and shall suffer
  the applicable consequences set forth in Section 9 above.
\item
  If a player is or, within the previous six (6) months, (i) has been in
  possession of any device or product used or designed for substituting,
  diluting, or adulterating a specimen sample, or (ii) has been subject
  to a finding by another sports league or anti-doping organization that
  he has substituted, diluted or adulterated a specimen sample and that
  finding has not been overturned on appeal, that player shall be
  required to undergo testing for Prohibited Substances no more than
  four (4) times during the six-week period following his notification
  by the NBA of the commencement of such testing. If the player (i)
  tests positive for a Drug of Abuse or (ii) tests positive pursuant to
  Section 4(d)(iii), (iv) or (v) above, he shall be dismissed and
  disqualified from any association with the NBA or any of its Teams in
  accordance with the provisions of Section 11(a) above. If the player
  tests positive for marijuana or a SPED, he shall suffer the applicable
  consequences set forth in Sections 8 or 9 above, as the case may be.
  If the player tests positive for a Diuretic, he shall be deemed to
  have tested positive for a SPED and shall suffer the applicable
  consequences set forth in Section 9 above. A player who tests positive
  for a Drug of Abuse or a SPED pursuant to this Section 15(e) may have
  his dismissal and disqualification or other penalty reduced or
  rescinded by the Grievance Arbitrator in accordance with Section 19
  below.
\item
  Nothing in this Section 15 shall limit or otherwise affect any of the
  provisions of Section 5 (Reasonable Cause Testing).
\end{enumerate}

\section{Additional Prohibited Substances and Testing
Methods.}\label{additional-prohibited-substances-and-testing-methods.}

\begin{enumerate}
\def\labelenumi{(\alph{enumi})}
\tightlist
\item
  Any steroid or performance-enhancing drug that is declared illegal
  during the term of this Agreement will automatically be added to the
  list of Prohibited Substances as a SPED.
\item
  At any time during the term of this Agreement, either the NBA or the
  Players Association may convene a meeting of the Prohibited Substances
  Committee to request that a substance or substances be added to the
  list of Prohibited Substances set forth on Exhibit I-2 to this
  Agreement. Any such addition of a Prohibited Substance may only
  include a substance that is or is reasonably likely to be physically
  harmful to Players and is or is reasonably likely to be improperly
  performance-enhancing. The determination of the Committee to add to
  the list of Prohibited Substances shall be made by a majority vote of
  all five (5) Committee members, and shall be final, binding, and
  unappealable.
\item
  Players will receive notice of any addition to the list of Prohibited
  Substances six (6) months prior to the date on which such addition
  becomes effective under this Article XXXIII.
\item
  At any time during the term of this Agreement, either the NBA or the
  Players Association may convene a meeting of the Prohibited Substances
  Committee to request that a testing method be added to the Program.
  Pursuant to this Section 16(d), the Prohibited Substances Committee
  shall have the authority to: (i) determine what testing methods will
  be used to detect newly added Prohibited Substances under the Program,
  if such Prohibited Substances are detected by methods not currently
  used by the Program's laboratories; and (ii) approve the use of new
  testing methods for current Prohibited Substances when such methods
  have been developed or validated during the term of this Agreement;
  provided, however, that the Prohibited Substances Committee shall not
  have the authority to add a testing method that would require a change
  to the manner in which specimens are collected from players (such as a
  change from urine collections to blood collections). Any determination
  of the Committee pursuant to this Section 16(d) shall be made by a
  majority vote of all five (5) Committee members, and shall be final,
  binding, and unappealable.
\end{enumerate}

\section{Prescriptions Under the Anti-Drug
Program.}\label{prescriptions-under-the-anti-drug-program.}

\begin{enumerate}
\def\labelenumi{(\alph{enumi})}
\tightlist
\item
  Notwithstanding the confidentiality provisions of Section 3 of this
  Article XXXIII, before any player is prescribed a drug or substance
  (whether or not it is a Prohibited Substance) as part of his treatment
  in the Program, the Medical Director or SPED Medical Director (as
  applicable) will notify the designated physician of the player's team
  of the name of the drug or substance (the ``Proposed Substance''), the
  medical justification for the prescription of the Proposed Substance,
  and the name of the prescribing physician.
\item
  If the designated physician of the player's team advises the Medical
  Director or SPED Medical Director (as applicable) -- at that time or
  at any time thereafter -- that the Proposed Substance would create a
  possible adverse reaction with another prescription substance that the
  player is being administered, a discussion will be held among the
  Medical Director or SPED Medical Director (as applicable), the
  prescribing physician and the designated team physician with respect
  to modifying one or both of the prescriptions so as to avoid the
  potential adverse reaction.
\item
  If the Medical Director or SPED Medical Director (as applicable)
  becomes aware that a player has been traded to or signed with another
  team after notification has been made to a designated team physician
  under Section 17(a) above, the Medical Director or SPED Medical
  Director (as applicable) is required to make the same notification to
  the designated team physician of the player's new team and to have the
  discussion required by Section 17(b) above.
\item
  A team physician who receives a notification from the Medical Director
  or SPED Medical Director (as applicable) under this Section 17 may
  only disclose the prescription for the Proposed Substance to other
  members of the team medical staff who are required to be advised of
  the prescription in order to ensure that the player is receiving
  proper medical care from the team's medical staff, and to no other
  person.
\end{enumerate}

\section{Longitudinal Profile for Exogenous
Testosterone.}\label{longitudinal-profile-for-exogenous-testosterone.}

\begin{enumerate}
\def\labelenumi{(\alph{enumi})}
\tightlist
\item
  A longitudinal profile for exogenous testosterone will be established
  for each player (the ``Longitudinal Profile''). The sole purpose of
  the Longitudinal Profile is to assist the laboratory selected by the
  parties to perform the analysis of primary specimens for urine tests
  under the Program (the ``Laboratory'') in determining which specimens
  shall be subjected to carbon isotope ratio mass spectrometry
  (``IRMS'') analysis.
\item
  Players' Longitudinal Profiles will be created pursuant to the
  protocol set forth in Exhibit I-7. The three (3) tests used to create
  the Longitudinal Profiles will be random tests conducted under Section
  6 above, and the creation of the Longitudinal Profiles will not
  require any player to undergo any testing in addition to the required
  random testing set forth in Section 6 above.
\item
  Once a player's Longitudinal Profile is established, the director of
  the Laboratory will consider the player's Baseline Values (as defined
  in Exhibit I-7) in comparison to the normalized Testosterone
  concentration, normalized Epitestosterone concentration or the
  corresponding Baseline Testosterone/Epitestosterone ratio
  (collectively, the ``Specimen Values'') of the player's subsequent
  urine specimens and will determine, in his or her discretion, whether
  to conduct an IRMS analysis on a urine specimen. In addition, the
  Laboratory will randomly select urine specimens for IRMS analysis to
  ensure that such analysis is conducted on at least one (1) urine
  specimen from every player during each year covered by the Program
  (i.e., from October 1 through September 30). The decision regarding
  whether to conduct IRMS analysis on a urine specimen for any other
  reason will remain in the discretion of the director of the
  Laboratory.
\end{enumerate}

\section{No Significant Fault or Negligence By
Player.}\label{no-significant-fault-or-negligence-by-player.}

\begin{enumerate}
\def\labelenumi{(\alph{enumi})}
\tightlist
\item
  If a player proves by clear and convincing evidence that he bears no
  significant fault or negligence for the presence of a Drug of Abuse or
  a SPED in his test result, the Grievance Arbitrator may, in a
  proceeding brought under Article XXXI of this Agreement, reduce or
  rescind the penalty otherwise applicable under this Article XXXIII.
  Such reduction or rescission (if any) will be determined at the
  discretion of the Grievance Arbitrator.
\item
  For purposes of this Section 19, ``no significant fault or
  negligence'' means the unusual circumstance in which the Player did
  not know or suspect, and could not reasonably have known or suspected,
  even with the exercise of considerable caution and diligence, that he
  was taking, ingesting, applying, or otherwise using the Drug of Abuse
  or SPED. To show that he bears no significant fault or negligence, the
  Player must also establish how the Drug of Abuse or SPED entered his
  system. A Player cannot satisfy his burden by merely denying that he
  intentionally used the Drug of Abuse or SPED.
\end{enumerate}

\chapter{RECOGNITION CLAUSE}\label{recognition-clause}

The NBA recognizes the Players Association as the exclusive collective
bargaining representative of all persons who are employed by NBA Teams
as professional basketball players and/or who may become so employed
during the term of any collective bargaining agreement between the
parties or any extension thereof: (a) all persons who are employed by
NBA Teams as professional basketball players; (b) all persons who have
been previously employed by an NBA Team as professional basketball
players who are seeking employment with an NBA Team as a professional
basketball player; (c) all rookie players selected in each year's NBA
Draft; and (d) all undrafted rookie players seeking employment with an
NBA Team as a professional basketball player. The Players Association
warrants that it is duly empowered to enter into this Agreement for and
on behalf of such persons. The NBA and the Players Association agree
that, notwithstanding the foregoing, such persons and NBA Teams may, on
an individual basis, bargain with respect to and agree upon the
provisions of Player Contracts, but only as and to the extent permitted
by this Agreement.

\chapter{SAVINGS CLAUSE}\label{savings-clause}

In the event that any provision hereof is found to be inconsistent with
the Internal Revenue Code of 1986, as amended (or the rules and
regulations issued thereunder (the ``Code'')), the National Labor
Relations Act, any other federal, state, provincial, or local statute or
ordinance, or the rules and regulations of any other government agency,
or is determined to have an adverse effect upon the right of the NBA (or
any successor entity) to a tax exemption under Section 501(c)(6) of the
Code (or any successor section of like import), then the parties hereto
agree to make such changes as are necessary to avoid such inconsistency
or to obtain or maintain such exemption retaining, to the extent
possible, the intention of such provision.

\chapter{PLAYER AGENTS}\label{player-agents}

\section{Approval of Player
Contracts.}\label{approval-of-player-contracts.}

The NBA shall not approve any Player Contract between a player and a
Team unless such player (a) is represented in the negotiations with
respect to such Player Contract by an agent or representative duly
certified by the Players Association in accordance with the Players
Association's Regulations Governing Player Agents and authorized to
represent him, or (b) acts on his own behalf in negotiating such Player
Contract.

\section{Fines.}\label{fines.}

The NBA shall impose a fine of \$50,000 upon any Team that negotiates a
Player Contract with an agent or representative not certified by the
Players Association in accordance with the Players Association's
Regulations Governing Player Agents if, at the time of such
negotiations, such Team either (a) knows that such agent or
representative has not been so certified, or (b) fails to make
reasonable inquiry of the NBA as to whether such agent or representative
has been so certified. Notwithstanding the preceding sentence, in no
event shall any Team be subject to a fine if the Team negotiates a
Player Contract with an agent or representative designated as the
player's authorized agent on the then- current agent list provided by
the Players Association to the NBA in accordance with Section 5 below.

\section{Prohibition on Players as
Agents.}\label{prohibition-on-players-as-agents.}

For purposes of negotiating the terms of a Uniform Player Contract or
otherwise dealing with a Team over any matter, players are prohibited
from (a) representing other current or prospective NBA players as an
agent certified under the Players Association's Regulations Governing
Player Agents, or (b) holding an equity interest or position in a
business entity that represents other current or prospective NBA players
as an agent certified under the Players Association's Regulations
Governing Player Agents.

\section{Indemnity.}\label{indemnity.}

The Players Association agrees to indemnify and hold harmless the NBA,
its Teams and each of its and their respective past, present and future
owners (direct and indirect) acting in their capacity as Team owners,
officers, directors, trustees, employees, successors, agents, attorneys,
heirs, administrators, executors and assigns, from any and all claims of
any kind arising from or relating to (a) the Players Association's
Regulations Governing Player Agents, and (b) the provisions of this
Article, including, without limitation, any judgments, costs and
settlements, provided that the Players Association is immediately
notified of such claim in writing (and, in no event later than five (5)
days from the receipt thereof), is given the opportunity to assume the
defense thereof, and the NBA and/or its Teams (whichever is sued) use
their best efforts to defend such claim, and do not admit liability with
respect to and do not settle such claim without the prior written
consent of the Players Association.

\section{Agent Lists.}\label{agent-lists.}

The Players Association agrees to provide the NBA League Office with a
list of (a) all agents certified under the Players Association's
Regulations Governing Player Agents, and (b) the players represented by
each such agent. Such list shall be updated once every two (2) weeks
from the day after the NBA Finals to the first day of the next
succeeding Regular Season and shall be updated once every month at all
other times.

\section{Confirmation by the Players
Association.}\label{confirmation-by-the-players-association.}

If the NBA has reason to believe that the agent representing a player in
Contract negotiations is not a certified agent or is not the agent
authorized to represent the player, then the NBA may, at its election,
request in writing from the Players Association confirmation as to
whether the agent who represented the player in the Contract
negotiations is in fact the player's certified representative. If within
three (3) business days of the date the Players Association receives
such written request, the NBA does not receive a written response from
the Players Association stating that the agent who represented the
player is not the player's certified representative, then the NBA shall
be free to act as if the agent is the player's confirmed certified
representative.

\chapter{PLAYER APPEARANCES/UNIFORM}\label{player-appearancesuniform}

\section{Player Appearances.}\label{player-appearances.}

A player may, during each Contract year covered by a Player Contract to
which he is a party, be required to make up to two (2) appearances at
the request of NBA Properties, Inc. in accordance with paragraph 13(d)
of a Uniform Player Contract and Article II, Section 8. Any appearance
that a player is required to make shall comply with the terms of Article
II, Section 8, and when a player makes an appearance in accordance with
this Section, he shall be paid at least \$3,500. When a player fails,
without reasonable excuse, to appear or reasonably to cooperate during
an appearance at any of the licensing appearances referred to in this
Section, he may be fined for each failure in an amount up to \$20,000.

\section{Uniform.}\label{uniform.}

\begin{enumerate}
\def\labelenumi{(\alph{enumi})}
\tightlist
\item
  During any NBA game or practice, including warm-up periods and going
  to and from the locker room to the playing floor, a player shall wear
  only the Uniform as supplied by his Team. For purposes of the
  preceding sentence only, ``Uniform'' means all clothing and other
  items (such as kneepads, wristbands and headbands, but not including
  Sneakers) worn by a player during an NBA game or practice.
  ``Sneakers'' means athletic shoes of the type worn by players while
  playing an NBA game.
\item
  Other than as may be incorporated into his Uniform and the
  manufacturer's identification incorporated into his Sneakers, a player
  may not, during any NBA game, display any commercial, promotional, or
  charitable name, mark, logo or other identification, including but not
  limited to on his body, in his hair, or otherwise.
\end{enumerate}

\chapter{INTEGRATION, ENTIRE AGREEMENT, INTERPRETATION, AND CHOICE OF
LAW}\label{integration-entire-agreement-interpretation-and-choice-of-law}

\chaptermark{INTEGRATION, ENTIRE AGREEMENT, INTERPRETATION, AND \ldots}

\section{Integration, Entire
Agreement.}\label{integration-entire-agreement.}

This Agreement, together with the exhibits hereto, and all letter
agreements executed contemporaneously herewith, constitutes the entire
understanding between the parties and all understandings, conversations
and communications, proposals, and counterproposals, oral and written
(including any draft of this Agreement) between the Members of the NBA
and the Players Association, or on behalf of them, are merged into and
superseded by this Agreement and shall be of no force or effect, except
as expressly provided herein. No such understandings, conversations,
communications, proposals, counterproposals or drafts shall be referred
to in any proceeding by the parties. Further, no understanding contained
in this Agreement shall be modified, altered or amended, except by a
writing signed by the party against whom enforcement is sought.

\section{Interpretation.}\label{interpretation.}

\begin{enumerate}
\def\labelenumi{(\alph{enumi})}
\tightlist
\item
  The NBA and Players Association recognize and acknowledge that there
  are and may continue to be (i) a collective bargaining relationship
  between WNBA, LLC (``WNBA'') and the Women's National Basketball
  Players Association (``WNBPA''), and (ii) a business arrangement
  between the NBA Development League (``NBADL'') and the Players
  Association, each of which is separate and distinct from the
  collective bargaining relationship between the NBA and the Players
  Association.
\item
  The NBA and the Players Association agree that this Agreement shall be
  interpreted without reference: (i) to any past, present or future
  WNBA/WNBPA collective bargaining agreement (or to any other past,
  present or future agreement between the WNBA or WNBA Enterprises, LLC,
  on the one hand, and the WNBPA on the other) or to any past, present,
  or future Standard Player Contract, Team Marketing and Promotional
  Agreement, or WNBA Marketing and Promotional Agreement (collectively,
  ``WNBA Agreements''); (ii) to any past, present or future agreement
  between the NBADL and the Players Association; (iii) to any of the
  provisions of such agreements or contracts; (iv) to the fact that a
  subject was not or is not covered by or included in any such
  agreements or contracts; and/or (v) to any judicial, arbitral, or
  administrative decision interpreting any of such agreements or
  contracts.
\item
  The parties agree that they will make no reference to any of the WNBA
  Agreements, NBADL/Players Association agreements, contracts or
  decisions referred to in Section 2(b) above, or to the fact that a
  particular provision was not or is not included in any such agreement
  or contract, or to any practice or policy of the WNBA (or WNBA
  Enterprises, LLC), the NBADL, or the WNBPA, in any arbitral, judicial,
  administrative, or other proceeding concerning the interpretation or
  enforcement of this Agreement, including, without limitation, a
  proceeding brought under Articles XXXI or XXXII of this Agreement. The
  parties further agree that no such agreement, contract, provision (or
  absence of provisions), decision, practice, or policy may be relied
  upon by any decision maker in such proceedings.
\end{enumerate}

\section{Choice of Law.}\label{choice-of-law.}

This Agreement (including the Uniform Player Contract and all other
Exhibits to this Agreement) is made under and shall be governed by the
internal law of the State of New York, except where federal law may
govern.

\chapter{TERM OF AGREEMENT}\label{term-of-agreement}

\section{Effective Date and Expiration
Date.}\label{effective-date-and-expiration-date.}

This Agreement shall be effective from July 1, 2017 (except with respect
to provisions that the parties have specifically agreed herein will
commence earlier) and, unless terminated pursuant to the provisions of
this Article XXXIX, shall continue in full force and effect through June
30, 2024.

\section{Mutual Options to Terminate Following Sixth
Season.}\label{mutual-options-to-terminate-following-sixth-season.}

The NBA and the Players Association shall each have the option to
terminate this Agreement on June 30, 2023 by serving written notice of
its exercise of such option on the other party on or before December 15,
2022.

\section{Termination by Players
Association/Anti-Collusion.}\label{termination-by-players-associationanti-collusion.}

\begin{enumerate}
\def\labelenumi{(\alph{enumi})}
\tightlist
\item
  In the event the conditions of Article XIV, Section 15 are satisfied,
  the Players Association shall have the right to terminate this
  Agreement by serving written notice of its exercise of such right
  within thirty (30) days after the System Arbitrator's report finding
  the requisite conditions (pursuant to Article XIV, Section 15) becomes
  final and any appeals therefrom have been exhausted or, in the absence
  of a System Arbitrator, by serving such written notice upon the NBA
  within thirty (30) days after any decision by a court finding the
  requisite conditions (pursuant to Article XIV, Section 15). In the
  latter situation, if the finding of the court is reversed on appeal,
  the Agreement shall be immediately reinstated and both parties reserve
  their rights with respect to any conduct by the other party during the
  period from the date of service of the termination notice to the date
  upon which the Agreement was reinstated.
\item
  If the Players Association exercises the right accorded it by Section
  3(a) above, this Agreement shall terminate as of the June 30
  immediately following the service of the termination notice.
\end{enumerate}

\section{Termination by NBA/National TV
Revenues.}\label{termination-by-nbanational-tv-revenues.}

\begin{enumerate}
\def\labelenumi{(\alph{enumi})}
\tightlist
\item
  For the purposes of this provision: (i) ``National TV Revenues'' shall
  mean the rights fees or other non-contingent payments stated in the
  NBA's third-party national broadcast network (e.g., ABC) and cable
  network (e.g., TNT or ESPN) television agreements (each, a ``National
  TV Agreement''); and (ii) ``Other Media Income'' shall mean the
  aggregate net income earned by any League-related entity (as defined
  in Article VII, Section 1(a)(1)) (but excluding net income
  attributable to ownership interests in any such League-related entity
  that is not owned by the NBA, NBA Properties, Inc., NBA Media
  Ventures, LLC and/or a group of NBA Teams) or by the NBA on behalf of
  the Teams from agreements that provide for the transmission of live
  (or delayed) NBA games, on a domestic or international basis, by means
  of television, radio, internet and any other mode of delivery
  referenced in Article VII, Section 1(a)(1)(ii), net of reasonable and
  customary expenses related thereto.
\item
  If, during the term of this Agreement, (i) the sum of the average
  annual National TV Revenues provided for under the Successor
  Agreements (as defined in Article VII, Section 1(c)(2)), plus 104.5\%
  of Other Media Income for the most recent Salary Cap Year, will be at
  least 35\% less than (ii) the sum of the average annual National TV
  Revenues provided for under the NBA/ABC and NBA/TBS Agreements, plus
  Other Media Income for the 2016-17 Salary Cap Year, the NBA shall have
  the right to terminate this Agreement effective as of the June 30
  immediately preceding the first Season covered by the Successor
  Agreements, by providing written notice of such termination to the
  Players Association at least sixty (60) days prior to such June 30.
  During the period following delivery of such written notice of
  termination, the NBA and the Players Association shall engage in good
  faith negotiations for the purpose of entering into a successor
  agreement and the provisions of Article XXX shall remain in full force
  and effect.
\end{enumerate}

\section{Termination by NBA/Force
Majeure.}\label{termination-by-nbaforce-majeure.}

\begin{enumerate}
\def\labelenumi{(\alph{enumi})}
\tightlist
\item
  ``Force Majeure Event'' shall mean the occurrence of any of the
  following events or conditions, provided that such event or condition
  either (i) makes it impossible for the NBA to perform its obligations
  under this Agreement, or (ii) frustrates the underlying purpose of
  this Agreement, or (iii) makes it economically impracticable for the
  NBA to perform its obligations under this Agreement: wars or war-like
  action (whether actual or threatened and whether conventional or
  other, including, but not limited to, chemical or biological wars or
  war-like action); sabotage, terrorism or threats of sabotage or
  terrorism; explosions; epidemics; weather or natural disasters,
  including, but not limited to, fires, floods, droughts, hurricanes,
  tornados, storms or earthquakes; and any governmental order or action
  (civil or military); provided, however, that none of the foregoing
  enumerated events or conditions is within the reasonable control of
  the NBA or an NBA Team.
\item
  In addition to any other rights a Team or the NBA may have by contract
  or by law, if a Force Majeure Event occurs and, as a result, one or
  more Teams are unable to play one or more games (whether Exhibition,
  Regular Season, or Playoff games), then, for each missed Exhibition,
  Regular Season, or Playoff game during such period (the ``Force
  Majeure Period'') that was not rescheduled and replayed, the
  Compensation payable to each player who was on the roster of a Team
  that was unable to play one or more games during the Force Majeure
  Period shall be reduced by 1/92.6th of the player's Compensation for
  the Season(s) covering the Force Majeure Period. For purposes of the
  foregoing calculation, and notwithstanding the actual number of games
  that any Team played, was scheduled to play, or could have played
  during the Seasons(s) affected by the Force Majeure Event, each Team
  shall be deemed to play five (5) Exhibition games, eighty-two (82)
  Regular Season games, and 5.6 Playoff games during each such Season.
\item
  In the event that Section 5(b) above applies, the applicable
  Compensation reduction from each player shall be withheld by the
  player's Team from the first Compensation payment (or payments, if the
  first such payment is insufficient to satisfy the reduction) that is
  (or are) due or to become due to such player following the
  commencement of the Force Majeure Period (whether under the Player
  Contract that was in existence at the commencement of the Force
  Majeure Period or any subsequent Player Contract between the player
  and the Team). If such Compensation payment (or payments) is (or are)
  insufficient to cover the Compensation reduction required by Section
  5(b) above, then either (i) the player shall promptly pay the
  difference directly to the Team (``old Team''), or (ii) if he
  subsequently enters into a Player Contract with, or is traded to,
  another NBA Team (``new Team''), such difference shall be withheld
  from the first available Compensation payment (or payments, if the
  first such payment is insufficient to satisfy the remaining reduction)
  that is (or are) due to the player from the new Team and shall be
  remitted by the new Team to the old Team.
\item
  Upon the occurrence of a Force Majeure Event satisfying the terms of
  Section 5(a) above, the NBA shall have the right to terminate this
  Agreement as of the sixtieth (60th) day following delivery to the
  Players Association of a written notice of termination, which must be
  delivered to the Players Association within sixty (60) days of the
  Force Majeure Event. During the sixty-day period following delivery of
  such written notice of termination, the NBA and the Players
  Association shall engage in good faith negotiations for the purpose of
  entering into a successor agreement, and during such period the
  provisions of Article XXX shall remain in full force and effect.
\end{enumerate}

\section{Mutual Right of
Termination.}\label{mutual-right-of-termination.}

If at any time during the term of this Agreement any provision contained
in Article VII, X, XI and XIV of this Agreement is enjoined, vacated,
declared null and void or is rendered unenforceable by any court of
competent jurisdiction, then either the NBA or the Players Association
shall have the right to terminate this Agreement by serving upon the
other party written notice of termination at least sixty (60) days prior
to the effective date of such termination.

\section{No Obligation to Terminate; No
Waiver.}\label{no-obligation-to-terminate-no-waiver.}

The grant to either party of a right or option to terminate pursuant to
the provisions of this Article XXXIX shall not carry with it the
obligation to exercise that right or option; and the failure of the NBA
or the Players Association to exercise any right or option to terminate
this Agreement with respect to any playing Season in accordance with
this Article XXXIX shall not be deemed a waiver of or in any way impair
or prejudice the NBA or the Players Association's right or option, if
any, to terminate this Agreement in accordance with this Article XXXIX
with respect to any succeeding Season.

\chapter{EXPANSION AND CONTRACTION}\label{expansion-and-contraction}

\section{Expansion.}\label{expansion.}

The NBA may determine during the term of this Agreement to expand the
number of Teams and to have existing Teams make available for assignment
to any such Expansion Teams the Player Contracts of a certain number of
Veterans under substantially the same terms and in substantially the
same manner that Player Contracts were made available to the Charlotte
expansion Team pursuant to the 1999 NBA/NBPA Collective Bargaining
Agreement; provided, however, that any change shall be subject to the
approval of the Players Association, which shall not be unreasonably
withheld.

\section{Contraction.}\label{contraction.}

If, during the term of this Agreement, the NBA decides to contract the
number of Teams, (a) the NBA shall provide written notice of such
decision to the Players Association, and (b) the NBA and the Players
Association shall negotiate and agree upon the effects of such decision
on the players and the procedures to be followed in connection
therewith.

\chapter{NBA DEVELOPMENT LEAGUE}\label{nba-development-league}

\section{NBADL Work Assignments.}\label{nbadl-work-assignments.}

\begin{enumerate}
\def\labelenumi{(\alph{enumi})}
\tightlist
\item
  An NBA Team may at any time assign a player (other than a Two-Way
  Player) on its Active List or Inactive List to an NBADL team, provided
  that the player (i) has either zero (0), one (1) or two (2) Years of
  Service at the time of the assignment, or (ii) has more than two (2)
  years of service at the time of the assignment and the player and the
  Players Association consent to such assignment in writing. Upon such
  assignment (``NBADL Work Assignment''), the player will be placed on
  the NBA Team's Inactive List, and shall (i) report to the NBADL team
  (and render for the NBADL team such services as the player is required
  to render for the NBA Team under his Uniform Player Contract and this
  Agreement), and (ii) at the direction of the NBA Team, subsequently
  return and report to, and resume the performance of services for, the
  NBA Team. An NBADL Work Assignment commences when the player reports
  in-person to the NBADL team, and ends either when the player, upon
  being recalled, reports back to his NBA Team or when the NBADL season
  concludes.
\item
  There shall be no limit on the number of NBADL Work Assignments given
  to a player. No NBA Team shall issue an NBADL Work Assignment for the
  purpose of disciplining a player for misconduct or retaliating against
  a player for exercising any right that he has under this Agreement or
  the Uniform Player Contract.
\item
  The NBA may establish reasonable rules regarding the assignment and
  recall of players to the NBADL provided that such rules do not violate
  the provisions of this Article XLI.
\end{enumerate}

\section{Reporting Requirements for NBADL Work
Assignments.}\label{reporting-requirements-for-nbadl-work-assignments.}

\begin{enumerate}
\def\labelenumi{(\alph{enumi})}
\tightlist
\item
  In order to initiate an NBADL Work Assignment or terminate such
  assignment and recall the player, the NBA Team shall provide the
  player, the NBA, and the Players Association with written notice. The
  player shall report to the NBADL team or NBA Team, whichever is
  applicable, within forty-eight (48) hours after such notice is
  received by the player.
\item
  If the player, without a reasonable excuse, does not report to the
  NBADL team or NBA Team, whichever is applicable, within the time
  provided in Section 2(a) above, the player may be fined and/or
  suspended without pay by the NBA Team until such time as he reports.
  In addition, such failure to report, without a reasonable excuse,
  shall constitute conduct prejudicial to the NBA under Article 35(d) of
  the NBA Constitution, subject however to the One Penalty requirement
  set forth in Article VI, Section 10.
\end{enumerate}

\section{Travel and Relocation
Expenses.}\label{travel-and-relocation-expenses.}

A player's NBA Team shall be obligated to reimburse the player for his
ordinary and reasonable expenses incurred in (a) traveling to and, when
recalled, from the NBADL team to begin and/or end any NBADL Work
Assignment or period of service on the Two-Way List (``D-League Two-Way
Service''), and (b) relocating to and, if recalled, from the NBADL
team's home location to begin and/or end any NBADL Work Assignment or
D-League Two-Way Service, that extends beyond a period of thirty (30)
days. During any NBADL Work Assignment or D-League Two-Way Service, the
player will be provided with housing or a housing subsidy in accordance
with the NBADL housing policy.

\section{Terms of NBADL Work Assignment and D-League Two-Way
Service.}\label{terms-of-nbadl-work-assignment-and-d-league-two-way-service.}

\begin{enumerate}
\def\labelenumi{(\alph{enumi})}
\tightlist
\item
  General Terms. During or in connection with any NBADL Work Assignment
  or D-League Two-Way Service, and except as expressly set forth in, or
  limited or modified by, this Article XLI, a player shall (i) accept
  and be subject to the work requirements and conditions applicable to
  NBADL players (as such requirements and conditions may change from
  time to time), and (ii) continue to be subject to the terms and
  obligations and entitled to the benefits and rights (including,
  without limitation, Years of Service and free agency rights) of his
  Uniform Player Contract and this Agreement.
\item
  Compensation and Benefits.

  \begin{enumerate}
  \def\labelenumii{(\roman{enumii})}
  \tightlist
  \item
    During or in connection with any NBADL Work Assignment or D-League
    Two-Way Service, a player (A) shall continue to receive the
    Compensation called for by his Uniform Player Contract, and (B)
    shall not receive (or accept) any compensation of any kind from the
    NBADL or any NBADL team other than as expressly set forth in this
    Article XLI. The player's performance in the NBADL shall not be
    considered for purposes of any Incentive Compensation contained in
    his Uniform Player Contract.
  \item
    Any Compensation protection provided to a player in his Uniform
    Player Contract shall remain in effect during an NBADL Work
    Assignment or D-League Two-Way Service. For purposes of Article II,
    Section 4, an injury sustained while participating in a basketball
    practice or game for an NBADL team shall be deemed an injury
    sustained while participating in a basketball practice or game for
    the NBA Team.
  \item
    During or in connection with any NBADL Work Assignment, a player (A)
    shall continue to be eligible to receive the benefits set forth in
    Article IV of this Agreement to the extent that such player would
    have been eligible to receive such benefits under this Agreement
    absent the NBADL Work Assignment, and (B) shall not be eligible to
    receive (and shall not accept) any benefits from the NBADL or any
    NBADL team, unless expressly set forth in this Article XLI.
  \item
    To the extent necessary, any plans and/or policies described in
    Article IV of this Agreement shall be amended to implement the
    provisions of Section 4(b)(iii) of this Article.
  \end{enumerate}
\item
  Meal Expense. While on the road with his NBADL team, a player:

  \begin{enumerate}
  \def\labelenumii{(\roman{enumii})}
  \tightlist
  \item
    On an NBADL Work Assignment, (i) shall receive the meal expense
    allowance applicable to NBA players, in accordance with the terms of
    Article III, Section 2 of this Agreement, and (ii) shall not receive
    (or accept) any meal expense or per diem from the NBADL or any NBADL
    team.
  \item
    Providing D-League Two-Way Service shall receive the meal expense
    allowance applicable to NBADL players.
  \end{enumerate}
\item
  Travel Accommodations. During an NBADL Work Assignment or D-League
  Two-Way Service, the player shall be provided with the same travel
  accommodations (including, but not limited to, transportation and
  hotel arrangements for ``road'' games) that are provided to NBADL
  players pursuant to applicable NBADL policies, except that: (i) a
  player on an NBADL Work Assignment shall not be required to share a
  hotel room; and (ii) a player on an NBADL Work Assignment shall be
  permitted to fly first class when traveling by air with his NBADL team
  to road games to the extent first class seats are available on his
  NBADL team's flight.
\item
  Conduct and Discipline.

  \begin{enumerate}
  \def\labelenumii{(\roman{enumii})}
  \tightlist
  \item
    During any NBADL Work Assignment or D-League Two-Way Service, the
    player will: (A) observe and comply with all rules and policies of
    the NBADL or his NBADL team at all times, whether on or off the
    playing floor; (B) give his best services, as well as his loyalty,
    to the NBADL team; (C) be neatly and fully attired in public; (D)
    conduct himself on and off the court according to the highest
    standards of honesty, citizenship, and sportsmanship; and (E) not do
    anything that, in the opinion of the Commissioner of the NBA, is
    materially detrimental or materially prejudicial to the best
    interests of the NBA Team, the NBA, the NBADL or the NBADL team.
  \item
    During or in connection with any NBADL Work Assignment or D-League
    Two-Way Service, the NBADL, the player's NBADL team, the NBA and the
    player's NBA Team may impose a fine and/or suspension on the player
    for the violation of NBADL or NBADL team rules or policies or for
    any conduct impairing the faithful and thorough discharge of the
    duties incumbent upon the player. Any disciplinary action taken by
    the NBA or an NBA Team in response to any act or conduct of a player
    during an NBADL Work Assignment or D-League Two-Way Service will
    supersede disciplinary action taken by the NBADL or any NBADL team
    in response to such act or conduct. Further, with respect to
    discipline imposed by the NBA and/or the NBA Team, the One Penalty
    rule set forth in Article VI, Section 10 of this Agreement shall
    apply. The amount of any such fine and/or suspension that may be
    imposed by the NBA or an NBA Team shall be governed by the terms of
    this Agreement and the Uniform Player Contract and shall not be
    limited by any NBADL rules, policies, practices, procedures, or fine
    schedules.
  \item
    All players on NBADL Work Assignments and all Two-Way Players
    providing D-League Two-Way Service shall be subject to the Joint
    NBA/NBPA Policy on Domestic Violence, Sexual Assault, and Child
    Abuse set forth as Exhibit F to this Agreement. Any evaluation,
    counseling, treatment, and/or discipline of such players for
    engaging in acts covered by this Policy shall be governed
    exclusively by the terms of the Policy. In the event any such player
    engages in other off-court conduct that is prohibited by both NBA
    and NBADL rules, NBA rules shall apply.
  \item
    When a player on an NBADL Work Assignment is suspended by his Team,
    the NBADL team to which he has been assigned, the NBA or the NBADL,
    such player's Base Compensation for the year of the Contract during
    which such suspension occurs shall be reduced by (i) 1/145th of the
    player's Base Compensation for each missed NBADL exhibition, regular
    season or playoff game for any suspension of less than twenty (20)
    games, and (ii) by 1/110th of the player's Base Compensation for
    each missed NBADL exhibition, regular season or playoff game for any
    suspension of twenty (20) games or more (including any indefinite
    suspension of a player that persists for twenty (20) games or more,
    or consecutive suspensions for continuing acts or conduct that
    persist for twenty (20) games or more). The player must remain on
    the NBA Inactive List during the term of the suspension but may be
    recalled at the option of the Team; provided, however, that the
    player may not play in NBA games during the term of the suspension.
  \item
    Notwithstanding anything to the contrary in Article VI, when a
    Two-Way Player is suspended by his Team, his NBADL team, the NBA or
    the NBADL, the following shall apply: (A) if the Two-Way Player was
    on the Two-Way list when the actions that led to the suspension
    occurred, such player's Base Compensation shall be reduced by (i)
    1/88th of the player's Two-Way NBADL Salary for any suspension of
    less than twelve (12) games, and (ii) 1/67th of the player's Two-Way
    NBADL Salary for any suspension of twelve (12) games or more
    (including any indefinite suspension of a player that persists for
    twelve (12) games or more, or consecutive suspensions for continuing
    acts or conduct that persist for twelve (12) games or more) (the
    ``D-League Reduction Rate'') for each missed NBADL exhibition,
    regular season or playoff game, and such player must be maintained
    on the Team's Two- Way List for the full term of the suspension; (B)
    if the Two-Way Player was on the Active List when the actions that
    led to the suspension occurred and the player was suspended by the
    NBA, such player's Base Compensation shall be reduced by (i) 1/145th
    of the player's Two-Way NBA Salary for any suspension of less than
    twenty (20) games, and (ii) 1/110th of the player's Two-Way NBA
    Salary for any suspension of twenty (20) games or more (including
    any indefinite suspension of a player that persists for twenty (20)
    games or more, or consecutive suspensions for continuing acts or
    conduct that persist for twenty (20) games or more) (the ``NBA
    Reduction Rate'') for each missed NBA exhibition, regular season or
    playoff game, and such player must be maintained on the Team's
    Active List during the full term of the suspension, except if the
    suspension is for more than five (5) games, in which case the player
    must be transferred to the Team's Two-Way List following the fifth
    game of the suspension (at which point he would begin to accrue his
    Two-Way NBADL Salary and his Base Compensation would be reduced at
    the D-League Reduction Rate for each missed NBA exhibition, regular
    season or playoff game); (C) if the Two-Way Player was on the
    Inactive List when the actions that led to the suspension occurred
    and the suspension was by the NBA, such player must be transferred
    to the Two-Way List during the full term of the suspension and such
    player's Base Compensation shall be reduced by the D-League
    Reduction Rate for every missed NBA exhibition, regular season or
    playoff game; (D) if such player was on the Active or Inactive List
    when the actions that led to the suspension occurred and the
    suspension was by his Team, such player may be placed on the Two-Way
    List for all or part of the term of the suspension and such player's
    Base Compensation shall be reduced by the NBA Reduction Rate for
    each missed NBA exhibition, regular season or playoff game while the
    Two-Way Player is on the Active or Inactive List and by the D-League
    Reduction Rate for each missed NBA exhibition, regular season or
    playoff game while the Two-Way Player is on the Two-Way List.
  \item
    A fine or suspension imposed by the NBADL or NBADL team in
    connection with a player's NBADL Work Assignment or D-League Two-Way
    Service may be heard and resolved by the Grievance Arbitrator
    pursuant to Article XXXI of this Agreement only if it results in a
    financial impact to the player of more than \$5,000. For purposes of
    paragraph 16(a)(ii) of a player's Uniform Player Contract, during or
    in connection with any NBADL Work Assignment or D-League Two-Way
    Service, (A) the terms ``any official or employee of the Team or the
    NBA (other than another player)'' will be construed to include,
    without limitation, any official or employee of the NBADL or the
    player's NBADL team (other than another player), and (B) the terms
    ``any NBA game or event'' will be construed to include, without
    limitation, any NBADL game or event.
  \end{enumerate}
\item
  Medical Treatment and Physical Condition.

  \begin{enumerate}
  \def\labelenumii{(\roman{enumii})}
  \tightlist
  \item
    The NBADL and/or NBADL team may make public medical information
    about a player on an NBADL Work Assignment or D-League Two-Way
    Service to the same extent as an NBA Team would be able to, pursuant
    to Article XXII, Section 4.
  \item
    For purposes of paragraphs 7, 16(a)(iii), 16(b), and 16(c) of the
    player's Uniform Player Contract, the terms ``basketball practice or
    game played for the Team'' or ``playing for the Team'' will be
    construed to include, without limitation, any practice or game
    played in the NBADL during an NBADL Work Assignment or D-League
    Two-Way Service.
  \end{enumerate}
\item
  Prohibited Substances. During any NBADL Work Assignment or D-League
  Two-Way Service, the player (i) shall be subject to Article XXXIII
  (Anti-Drug Program) of this Agreement and paragraph 8 of the Uniform
  Player Contract, and (ii) shall not be subject to any anti-drug
  program maintained by the NBADL.
\item
  Player Attributes and Performances. Notwithstanding anything to the
  contrary in this Agreement or the Uniform Player Contract, with
  respect to any player who serves or has served on an NBADL Work
  Assignment or provides or has provided D-League Two-Way Service:

  \begin{enumerate}
  \def\labelenumii{(\roman{enumii})}
  \tightlist
  \item
    The NBA and its related entities (including, without limitation, NBA
    Teams), and the NBADL and its related entities (including, without
    limitation, NBADL teams), shall have the right to use, and to
    license others to use, such player's Player Attributes (as defined
    in Paragraph 14(a) of the Uniform Player Contract) in connection
    with any advertising, marketing, or collateral materials or
    marketing programs conducted by the NBADL or any NBADL team that is
    intended to promote (1) any game in which an NBADL team participates
    or any NBADL game telecast or broadcast (including NBADL pre-season,
    exhibition, regular season, or playoff games), (2) the NBADL, its
    teams or its players, or (3) the sport of basketball.
  \item
    The NBA and its related entities (including, without limitation, NBA
    Teams), and the NBADL and its related entities (including, without
    limitation, NBADL teams), shall have the right to use, and to
    license others to use, any performance of such player in connection
    with any form of broadcast or telecast, including over-the-air
    television, cable television, pay television, direct broadcast
    satellite television, and any form of cassette, cartridge, disk
    system, or other means of distribution known or unknown. The
    foregoing does not confer any right or authority for the NBA and its
    related entities (including, without limitation, NBA Teams), and/or
    the NBADL and its related entities (including, without limitation,
    NBADL teams), to use or authorize others to use the Player's Player
    Attributes in a manner that constitutes an unauthorized Endorsement
    or an Unauthorized Sponsor Promotion (as such terms are defined and
    clarified in Article XXVIII of this Agreement and Paragraph 14 of
    the Uniform Player Contract). For purposes of clarity and without
    limitation, any use of a player's Player Attributes that has been
    expressly authorized by the player (not including the Uniform Player
    Contract) shall not be an unauthorized Endorsement or an
    Unauthorized Sponsor Promotion. For the purposes of this Section
    4(h), references to the NBA and NBA Teams in Article XXVIII, Section
    3 (Unauthorized Endorsement/Sponsor Promotion) shall apply to the
    NBADL and NBADL teams.
  \end{enumerate}
\item
  Promotional Activities. In connection with a player's NBADL Work
  Assignment or D-League Two-Way Service, the rights accorded to the NBA
  and his NBA Team under paragraph 13(a) of the Uniform Player Contract
  shall extend, without limitation, to the NBADL and his NBADL team, and
  any promotional appearances such player is required to make during an
  NBADL Work Assignment or D-League Two-Way Service shall count against
  the appearances the player is obligated to provide to the NBA and his
  NBA Team under Article II, Section 8; provided, however, that such
  player will be required to provide two (2) additional promotional
  appearances while on NBADL Work Assignment or D-League Two-Way Service
  each Season to the NBADL or his NBADL team.
\end{enumerate}

\section{Miscellaneous.}\label{miscellaneous.-2}

\begin{enumerate}
\def\labelenumi{(\alph{enumi})}
\tightlist
\item
  With respect to the duties and obligations of players under paragraph
  5 of the Uniform Player Contract (relating to Article 35 of the NBA
  Constitution) during or in connection with any NBADL Work Assignment
  or D-League Two-Way Service:

  \begin{enumerate}
  \def\labelenumii{(\roman{enumii})}
  \tightlist
  \item
    the terms ``game'' or ``games'' in Article 35(b) and (c) of the
    Constitution will be construed to include, without limitation, any
    game played by an NBADL team;
  \item
    the term ``basketball'' or ``game of basketball'' in Article 35(c)
    and (d) of the Constitution will be construed to include, without
    limitation, the NBADL or any of its teams;
  \item
    the prohibition concerning wagering in Article 35(f) of the
    Constitution will extend, without limitation, to any game played by
    an NBADL team; and
  \item
    the Commissioner's authority to act pursuant to paragraph 5(e) of
    the Uniform Player Contract will extend, without limitation, to any
    game played by an NBADL team.
  \end{enumerate}
\item
  A player shall not directly or indirectly own or hold any interest in
  the NBADL or any NBADL team unless authorized by the NBA.
\item
  At the conclusion of each Season covered by this Agreement, the NBA
  and the Players Association shall meet to discuss issues concerning
  the operation of this Article XLI.
\end{enumerate}

\section{Career Opportunities for Former NBA
Players.}\label{career-opportunities-for-former-nba-players.}

\begin{enumerate}
\def\labelenumi{(\alph{enumi})}
\tightlist
\item
  The NBA and/or NBADL will operate an apprenticeship program in the
  NBA/NBADL League Office and/or on NBADL team coaching staffs to
  provide business and basketball operations immersion training for
  former NBA players. Each session will last for approximately three (3)
  months and include basketball operations, community relations, sales
  and marketing, and/or team coaching rotations. There will be two (2)
  sessions held annually, and each session will include up to two (2)
  former NBA players (based on player interest and, with respect to
  NBADL team coaching apprenticeships, availability of NBADL teams
  willing to participate). Participating former players in the League
  Office program will receive a monthly stipend to be agreed upon by the
  NBA and the Players Association. Participating former players in the
  NBADL team coaching staff program will receive a monthly stipend to be
  agreed upon by the NBA and the Players Association, and housing or a
  housing subsidy in accordance with the NBADL housing policy.
\item
  The NBA and/or NBADL will operate an NBADL assistant coaching program
  to complement the existing NBA program and provide coaching training
  and experience for former NBA players. Up to fourteen (14) total
  coaching spots will be made available each year at the D-League
  National Tryout and Elite Mini Camp. Top performing coaches at such
  events will have the opportunity to take part in other key basketball
  operations events such as the Portsmouth Invitational Tournament and
  the NBA Draft Combine. Participating former players will receive
  reimbursement for all reasonable expenses associated with
  participating in the coaching program.
\item
  The following programs will be created for former NBA players to have
  access to information about job opportunities in the NBADL:

  \begin{enumerate}
  \def\labelenumii{(\roman{enumii})}
  \tightlist
  \item
    A database will be set up so that all NBADL and NBADL team job
    openings can be shared with former NBA players who have expressed
    interest in such positions.
  \item
    NBADL teams will attend an annual job fair held in connection with
    an NBA or NBADL event (e.g., Draft Combine/D-League Elite Mini Camp
    or D-League Showcase) to facilitate discussions between NBADL team
    executives and former NBA players. The NBA will use reasonable
    efforts to ensure that a representative from each NBADL team attends
    each job fair.
  \end{enumerate}
\end{enumerate}

\chapter{OTHER}\label{other}

\section{Headings and Organization.}\label{headings-and-organization.}

The headings and organization of this Agreement are solely for the
convenience of the parties, and shall not be deemed part of, or
considered in construing or interpreting, this Agreement.

\section{Time Periods.}\label{time-periods.}

Unless specifically stated otherwise, the specification of any time
period in this Agreement shall include any non-business days within such
period, except that any deadline falling on a Saturday, Sunday, or
Federal Holiday shall be deemed to fall on the following business day.

\section{Exhibits.}\label{exhibits.}

All of the Exhibits hereto are an integral part of this Agreement and of
the agreement of the parties thereto.

NATIONAL BASKETBALL ASSOCIATION

By:\\
\_\_\_\_\_\_\_\_\_\_\_\_\_\_\_\_\_\_\_\_\_\_\_\_\_\_\_\_\_\\
Adam Silver\\
Commissioner

NATIONAL BASKETBALL PLAYERS ASSOCIATION

By:\\
\_\_\_\_\_\_\_\_\_\_\_\_\_\_\_\_\_\_\_\_\_\_\_\_\_\_\_\_\_\\
Michele Roberts\\
Executive Director

\appendix


\chapter{NATIONAL BASKETBALL ASSOCIATION UNIFORM PLAYER
CONTRACT}\label{national-basketball-association-uniform-player-contract}

THIS AGREEMENT made this \_\_\_\_\_\_ day of
\_\_\_\_\_\_\_\_\_\_\_\_\_\_\_\_\_\_\_, is by and between
\_\_\_\_\_\_\_\_\_\_\_\_\_\_\_\_\_\_\_\_\_\_\_\_ (hereinafter called the
``Team''), a member of the National Basketball Association (hereinafter
called the ``NBA'' or ``League'') and
\_\_\_\_\_\_\_\_\_\_\_\_\_\_\_\_\_\_\_, an individual whose address is
shown below (hereinafter called the ``Player''). In consideration of the
mutual promises hereinafter contained, the parties hereto promise and
agree as follows:

\begin{enumerate}
\def\labelenumi{\arabic{enumi}.}
\item
  \textbf{TERM.}

  The Team hereby employs the Player as a skilled basketball player for
  a term of \_\_\_\_ year(s) from the 1st day of September \_\_\_\_.
\item
  \textbf{SERVICES.}

  The services to be rendered by the Player pursuant to this Contract
  shall include: (a) training camp, (b) practices, meetings, workouts,
  and skill or conditioning sessions conducted by the Team during the
  Season, (c) games scheduled for the Team during any Regular Season,
  (d) Exhibition games scheduled by the Team or the League during and
  prior to any Regular Season, (e) if the Player is invited to
  participate, the NBA's All-Star Game (including the Rookie-Sophomore
  Game) and every event conducted in association with such All-Star
  Game, but only in accordance with Article XXI of the Collective
  Bargaining Agreement currently in effect between the NBA and the
  National Basketball Players Association (hereinafter the ``CBA''), (f)
  Playoff games scheduled by the League subsequent to any Regular
  Season, (g) promotional and commercial activities of the Team and the
  League as set forth in this Contract and the CBA, (h) any NBADL Work
  Assignment in accordance with Article XLI of the CBA, and (i) any
  service in the NBADL pursuant to a Two-Way Contract.
\item
  \textbf{COMPENSATION.}
\end{enumerate}

\begin{enumerate}
\def\labelenumi{(\alph{enumi})}
\item
  Subject to paragraph 3(b) below, the Team agrees to pay the Player for
  rendering the services and performing the obligations described herein
  the Compensation described in Exhibit 1, Exhibit 1A, Exhibit 1B, or
  Exhibit 10 hereto, as applicable (less all amounts required to be
  withheld by any governmental authority, and exclusive of any amount(s)
  which the Player shall be entitled to receive from the Player Playoff
  Pool). For Standard NBA Contracts, unless otherwise provided in
  Exhibit 1 or Exhibit 1A, such Compensation shall be paid in
  twenty-four (24) equal semi-monthly payments beginning with the first
  of said payments on November 15th of each year covered by this
  Contract (``contract year'') and continuing with such payments on the
  first and fifteenth of each month until said Compensation is paid in
  full. For Two-Way Contracts, Compensation shall be paid as follows:
  (i) the Player's Two-Way NBADL Salary shall be paid in twenty-four
  (24) equal semi-monthly payments beginning with the first of said
  payments on November 15th of each contract year and continuing with
  such payments on the first and fifteenth of each month until said
  Compensation is paid in full (each such payment date, a ``Semi-Monthly
  Payment Date''); (ii) for each NBA Day of Service that the Player
  accrues prior to the first Semi-Monthly Payment Date and between each
  subsequent Semi-Monthly Payment Date (each such period, an ``NBA Day
  of Service Payment Period''), the Player shall be paid a payment equal
  to (x) the Two-Way NBA Salary daily rate, less (y) the Two-Way NBADL
  Salary daily rate, multiplied by (z) the number of NBA Days of Service
  that the Player accrues during such NBA Day of Service Payment Period,
  with such payment, if applicable, made on the Semi-Monthly Payment
  Date two weeks after the completion of the NBA Day of Service Payment
  Period.
\item
  The Team agrees to pay the Player \$2,000 per week, pro rata, less all
  amounts required to be withheld by any governmental authority, for
  each week (up to a maximum of four (4) weeks for Veterans and up to a
  maximum of five (5) weeks for Rookies) prior to the Team's first
  Regular Season game that the Player is in attendance at NBA training
  camp or Exhibition games; provided, however, that no such payments
  shall be made if, prior to the date on which he is required to attend
  training camp, the Player has been paid \$10,000 or more in
  Compensation with respect to the NBA Season scheduled to commence
  immediately following such training camp. Any Compensation paid by the
  Team pursuant to this subparagraph shall be considered an advance
  against any Compensation owed to the Player pursuant to paragraph 3(a)
  above, and the first scheduled payment of such Compensation (or such
  subsequent payments, if the first scheduled payment is not sufficient)
  shall be reduced by the amount of such advance; except that in the
  case of Two-Way Players, any Compensation paid by the Team pursuant to
  this subparagraph shall be considered an advance against such player's
  Two-Way NBA Salary only, and the first scheduled payment of such
  Two-Way NBA Salary (or such subsequent payments, if the first
  scheduled payment is not sufficient) shall be reduced by the amount of
  such advance.
\item
  The Team will not pay and the Player will not accept any bonus or
  anything of value on account of the Team's winning any particular NBA
  game or series of games or attaining a certain position in the
  standings of the League as of a certain date, other than the final
  standing of the Team.
\end{enumerate}

\begin{enumerate}
\def\labelenumi{\arabic{enumi}.}
\setcounter{enumi}{3}
\item
  \textbf{EXPENSES.}

  The Team agrees to pay all proper and necessary expenses of the
  Player, including the reasonable lodging expenses of the Player while
  playing for the Team ``on the road'' and during the NBA training camp
  period (defined for this paragraph only to mean the period from the
  first day of training camp through the day of the Team's first
  Exhibition game) for as long as the Player is not then living at home.
  The Player, while ``on the road'' (and during the NBA training camp
  period, only if the Player is not then living at home and the Team
  does not pay for meals directly), shall be paid a meal expense
  allowance as set forth in the CBA. No deductions from such meal
  expense allowance shall be made for meals served on an airplane.
  During the NBA training camp period (and only if the Player is not
  then living at home and the Team does not pay for meals directly), the
  meal expense allowance shall be paid in weekly installments commencing
  with the first week of training camp. For the purposes of this
  paragraph, the Player shall be considered to be ``on the road'' from
  the time the Team leaves its home city until the time the Team arrives
  back at its home city.
\item
  \textbf{CONDUCT.}
\end{enumerate}

\begin{enumerate}
\def\labelenumi{(\alph{enumi})}
\item
  The Player agrees to observe and comply with all Team rules, as
  maintained or promulgated in accordance with the CBA, at all times
  whether on or off the playing floor. Subject to the provisions of the
  CBA, such rules shall be part of this Contract as fully as if herein
  written and shall be binding upon the Player.
\item
  The Player agrees: (i) to give his best services, as well as his
  loyalty, to the Team, and to play basketball only for the Team and its
  assignees; (ii) to be neatly and fully attired in public; (iii) to
  conduct himself on and off the court according to the highest
  standards of honesty, citizenship, and sportsmanship; and (iv) not to
  do anything that is materially detrimental or materially prejudicial
  to the best interests of the Team or the League.
\item
  For any violation of Team rules, any breach of any provision of this
  Contract, or for any conduct impairing the faithful and thorough
  discharge of the duties incumbent upon the Player, the Team may
  reasonably impose fines and/or suspensions on the Player in accordance
  with the terms of the CBA.
\item
  The Player agrees to be bound by Article 35 of the NBA Constitution, a
  copy of which, as in effect on the date of this Contract, is attached
  hereto. The Player acknowledges that the Commissioner is empowered to
  impose fines upon and/or suspend the Player for causes and in the
  manner provided in such Article, provided that such fines and/or
  suspensions are consistent with the terms of the CBA.
\item
  The Player agrees that if the Commissioner, in his sole judgment,
  shall find that the Player has bet, or has offered or attempted to
  bet, money or anything of value on the outcome of any game
  participated in by any Team or NBADL team, the Commissioner shall have
  the power in his sole discretion to suspend the Player indefinitely or
  to expel him as a player for any Team, and the Commissioner's finding
  and decision shall be final, binding, conclusive, and unappealable.
\item
  The Player agrees that he will not, during the term of this Contract,
  directly or indirectly, entice, induce, or persuade, or attempt to
  entice, induce, or persuade, any player or coach who is under contract
  to any NBA Team to enter into negotiations for or relating to his
  services as a basketball player or coach, nor shall he negotiate for
  or contract for such services, except with the prior written consent
  of such Team. Breach of this subparagraph, in addition to the remedies
  available to the Team, shall be punishable by fine and/or suspension
  to be imposed by the Commissioner.
\item
  When the Player is fined and/or suspended by the Team or the NBA, he
  shall be given notice in writing (with a copy to the Players
  Association), stating the amount of the fine or the duration of the
  suspension and the reasons therefor.
\end{enumerate}

\begin{enumerate}
\def\labelenumi{\arabic{enumi}.}
\setcounter{enumi}{5}
\tightlist
\item
  \textbf{WITHHOLDING.}
\end{enumerate}

\begin{enumerate}
\def\labelenumi{(\alph{enumi})}
\item
  In the event the Player is fined and/or suspended by the Team or the
  NBA (or, as applicable, the NBADL or an NBADL team), the Team shall
  withhold the amount of the fine or, in the case of a suspension, the
  amount provided in Article VI of the CBA (or, as applicable, Article
  XLI) from any Current Base Compensation due or to become due to the
  Player with respect to the contract year in which the conduct
  resulting in the fine and/or the suspension occurred (or a subsequent
  contract year if the Player has received all Current Base Compensation
  due to him for the then current contract year). If, at the time the
  Player is fined and/or suspended, the Current Base Compensation
  remaining to be paid to the Player under this Contract is not
  sufficient to cover such fine and/or suspension, then the Player
  agrees promptly to pay the amount directly to the Team. In no case
  shall the Player permit any such fine and/or suspension to be paid on
  his behalf by anyone other than himself.
\item
  Any Current Base Compensation withheld from or paid by the Player
  pursuant to this paragraph 6 shall be retained by the Team or the
  League, as the case may be, unless the Player contests the fine and/or
  suspension by initiating a timely Grievance in accordance with the
  provisions of the CBA. If such Grievance is initiated and it satisfies
  Article XXXI, Section 14 of the CBA, the amount withheld from the
  Player shall be placed in an interest-bearing account, pursuant to
  Article XXXI, Section 10 of the CBA, pending the resolution of the
  Grievance.
\end{enumerate}

\begin{enumerate}
\def\labelenumi{\arabic{enumi}.}
\setcounter{enumi}{6}
\tightlist
\item
  \textbf{PHYSICAL CONDITION.}
\end{enumerate}

\begin{enumerate}
\def\labelenumi{(\alph{enumi})}
\item
  The Player agrees to report at the time and place fixed by the Team in
  good physical condition and to keep himself throughout each NBA Season
  in good physical condition.
\item
  If the Player, in the judgment of the Team's physician, is not in good
  physical condition at the date of his first scheduled game for the
  Team, or if, at the beginning of or during any Season, he fails to
  remain in good physical condition (unless such condition results
  directly from an injury sustained by the Player as a direct result of
  participating in any basketball practice or game played for the Team
  during such Season), so as to render the Player, in the judgment of
  the Team's physician, unfit to play skilled basketball, the Team shall
  have the right to suspend such Player until such time as, in the
  judgment of the Team's physician, the Player is in sufficiently good
  physical condition to play skilled basketball. In the event of such
  suspension, the Base Compensation payable to the Player for any Season
  during such suspension shall be reduced in the same proportion as the
  length of the period during which, in the judgment of the Team's
  physician, the Player is unfit to play skilled basketball, bears to
  the length of such Season. Nothing in this subparagraph shall
  authorize the Team to suspend the Player solely because the Player is
  injured or ill.
\item
  If, during the term of this Contract, the Player is injured as a
  direct result of participating in any basketball practice or game
  played for the Team, the Team will pay the Player's reasonable
  hospitalization and medical expenses (including doctor's bills),
  provided that the hospital and doctor are selected by the Team, that
  the Team shall be obligated to pay only those expenses incurred as a
  direct result of medical treatment caused solely by and relating
  directly to the injury sustained by the Player. The Team will also pay
  costs associated with a second opinion in accordance with Article
  XXII, Section 10 of the CBA. Subject to the provisions set forth in
  Exhibit 3, if in the judgment of the Team's physician, the Player's
  injuries resulted directly from playing for the Team and render him
  unfit to play skilled basketball, then, so long as such unfitness
  continues, but in no event after the Player has received his full Base
  Compensation for the Season in which the injury was sustained, the
  Team shall pay to the Player the Base Compensation prescribed in
  Exhibit 1 to this Contract for such Season (or in the case of a
  Two-Way Contract, so long as such unfitness continues but in no event
  after the Two-Way Player has received his Two-Way Annual NBADL Salary
  for such NBADL Regular Season (prorated as necessary if the Two-Way
  Contract was entered into after the start of the NBADL Regular Season)
  plus (i) any Two-Way NBA Salary earned by such Two-Way Player during
  such NBA Regular Season prior to the date of such unfitness, less (ii)
  such Two-Way Player's Two-Way NBADL Salary covering the number of NBA
  Days of Service accrued by such Two-Way Player during such NBA Regular
  Season prior to the date of such unfitness). The Team's obligations
  hereunder shall be reduced by (x) any workers' compensation benefits,
  which, to the extent permitted by law, the Player hereby assigns to
  the Team, and (y) any insurance provided for by the Team whether paid
  or payable to the Player.
\item
  The Player agrees to provide to the Team's coach, trainer, or
  physician prompt notice of any injury, illness, or medical condition
  suffered by him that is likely to affect adversely the Player's
  ability to render the services required under this Contract, including
  the time, place, cause, and nature of such injury, illness, or
  condition.
\item
  Should the Player suffer an injury, illness, or medical condition, he
  will submit himself to a medical examination, appropriate medical
  treatment by a physician designated by the Team, and such
  rehabilitation activities as such physician may specify. Such
  examination when made at the request of the Team shall be at its
  expense, unless made necessary by some act or conduct of the Player
  contrary to the terms of this Contract.
\item
  The Player agrees (i) to submit to a physical examination at the
  commencement and conclusion of each contract year hereunder, and at
  such other times as reasonably determined by the Team to be medically
  necessary, and (ii) at the commencement of this Contract, and upon the
  request of the Team, to provide a complete prior medical history.
\item
  The Player agrees to supply complete and truthful information in
  connection with any medical examinations or requests for medical
  information authorized by this Contract.
\item
  \begin{enumerate}
  \def\labelenumii{(\roman{enumii})}
  \tightlist
  \item
    A Player who consults or is treated by a physician (including a
    psychiatrist) or a professional providing non-mental health related
    medical services (e.g., chiropractor, physical therapist) other than
    a physician or other professional designated by the Team shall give
    notice of such consultation or treatment to the Team and shall
    provide the Team with all information it may request concerning any
    condition that in the judgment of the Team's physician may affect
    the Player's ability to play skilled basketball.
  \item
    A Player who engages in five (5) or more training or workout
    sessions with a trainer, performance coach, strength and
    conditioning coach, or any other similar coach or trainer other than
    at the direction of the Team (each a ``Third-Party Trainer''), shall
    give notice of such training or workout to the Team prior to the
    first such training or work out session, provided that if the player
    does not initially plan to continue working with any such
    Third-Party Trainer for five (5) or more sessions, such notice must
    be provided no later than prior to the fifth such session. This
    notice requirement shall not apply to workouts or training that
    exclusively involve jogging, road bicycling, swimming, yoga, Pilates
    and/or dance; and the Player's failure to comply with such notice
    requirement shall not itself constitute a material breach of this
    Contract. For clarity with respect to counting multi-day training or
    workout sessions under this paragraph, any such session(s) shall be
    counted to equal the number of days on which such training or
    workouts occurred. Subject to the Team's other rights, and the
    player's other obligations, under the CBA and this Contract,
    including, for example, the player's obligations under this
    Paragraph 7 to report in good physical condition and to submit to
    treatment and rehabilitation specified by a physician designated by
    the Team, a player will have the right in the off-season to work out
    with one or more Third-Party Trainers of his choosing and may not be
    disciplined for exercising that right.
  \end{enumerate}
\item
  If and to the extent necessary to enable or facilitate the disclosure
  of medical information as provided for by this Contract or Article
  XXII or XXXIII of the CBA, the Player shall execute such individual
  authorization(s) as may be requested by the Team or the Medical
  Director of the Anti-Drug Program or as may be required by health care
  providers who examine or treat the Player.
\end{enumerate}

\begin{enumerate}
\def\labelenumi{\arabic{enumi}.}
\setcounter{enumi}{7}
\item
  \textbf{PROHIBITED SUBSTANCES/DOMESTIC VIOLENCE.}

  The Player acknowledges that this Contract may be terminated in
  accordance with the express provisions of (i) Article XXXIII
  (Anti-Drug Program) of the CBA or (ii) the Joint NBA/NBPA Policy on
  Domestic Violence, Sexual Assault, and Child Abuse, and that any such
  termination will result in the Player's immediate dismissal and
  disqualification from any employment by the NBA and any of its Teams.
  Notwithstanding any terms or provisions of this Contract (including
  any amendments hereto), in the event of such termination, all
  obligations of the Team, including obligations to pay Compensation,
  shall cease, except the obligation of the Team to pay the Player's
  earned Compensation (whether Current or Deferred) to the date of
  termination.
\item
  \textbf{UNIQUE SKILLS.}

  The Player represents and agrees that he has extraordinary and unique
  skill and ability as a basketball player, that the services to be
  rendered by him hereunder cannot be replaced or the loss thereof
  adequately compensated for in money damages, and that any breach by
  the Player of this Contract will cause irreparable injury to the Team,
  and to its assignees. Therefore, it is agreed that in the event it is
  alleged by the Team that the Player is playing, attempting or
  threatening to play, or negotiating for the purpose of playing, during
  the term of this Contract, for any other person, firm, entity, or
  organization, the Team and its assignees (in addition to any other
  remedies that may be available to them judicially or by way of
  arbitration) shall have the right to obtain from any court or
  arbitrator having jurisdiction such equitable relief as may be
  appropriate, including a decree enjoining the Player from any further
  such breach of this Contract, and enjoining the Player from playing
  basketball for any other person, firm, entity, or organization during
  the term of this Contract. The Player agrees that this right may be
  enforced by the Team or the NBA. In any suit, action, or arbitration
  proceeding brought to obtain such equitable relief, the Player does
  hereby waive his right, if any, to trial by jury, and does hereby
  waive his right, if any, to interpose any counterclaim or set-off for
  any cause whatever.
\item
  \textbf{ASSIGNMENT.}
\end{enumerate}

\begin{enumerate}
\def\labelenumi{(\alph{enumi})}
\tightlist
\item
  The Team shall have the right to assign this Contract to any other NBA
  Team, and the Player agrees to accept such assignment and to
  faithfully perform and carry out this Contract with the same force and
  effect as if it had been entered into by the Player with the assignee
  Team instead of with the Team.
\item
  In the event that this Contract is assigned to any other NBA Team, all
  reasonable expenses incurred by the Player in moving himself and his
  family to the home territory of the Team to which such assignment is
  made, as a result thereof, shall be paid by the assignee Team.
\item
  In the event that this Contract is assigned to another NBA Team, the
  Player (or his agent) shall forthwith be provided notice of such
  assignment by phone or email. With respect to an assignment by trade,
  notice of the trade must be provided to the Player (or his agent) by
  phone or email either before conclusion of the trade call with the NBA
  or as soon as possible after the conclusion of the trade call (but in
  no event may such notification be made more than one (1) hour after
  the conclusion of the trade call or less than one (1) hour prior to
  the public announcement of the assignment). The Player shall report to
  the assignee Team within forty-eight (48) hours after said notice has
  been received (if the assignment is made during a Season), within one
  (1) week after said notice has been received (if the assignment is
  made between Seasons), or within such longer time for reporting as may
  be specified in said notice. The NBA shall also notify the Players
  Association of any such assignment as soon as practicable but in no
  event later than one (1) business day after such assignment occurs.
  The Player further agrees that, immediately upon reporting to the
  assignee Team, he will submit upon request to a physical examination
  conducted by a physician designated by the assignee Team.
\item
  If the Player, without a reasonable excuse, does not report to the
  Team to which this Contract has been assigned within the time provided
  in subsection (c) above, then (i) upon consummation of the assignment,
  the Player may be disciplined by the assignee Team or, if the
  assignment is not consummated or is voided as a result of the Player's
  failure to so report, by the assignor Team, and (ii) such conduct
  shall constitute conduct prejudicial to the NBA under Article 35(d) of
  the NBA Constitution, and shall therefore subject the Player to
  discipline from the NBA in accordance with such Article.
\end{enumerate}

\begin{enumerate}
\def\labelenumi{\arabic{enumi}.}
\setcounter{enumi}{10}
\tightlist
\item
  \textbf{VALIDITY AND FILING.}
\end{enumerate}

\begin{enumerate}
\def\labelenumi{(\alph{enumi})}
\tightlist
\item
  This Contract shall be valid and binding upon the Team and the Player
  immediately upon its execution.
\item
  The Team agrees to file a copy of this Contract, and/or any
  amendment(s) thereto, with and as directed by the Commissioner of the
  NBA as soon as practicable by email, but in no event may such filing
  be made more than forty-eight (48) hours after the execution of this
  Contract and/or amendment(s).
\item
  If pursuant to the NBA Constitution and By-Laws or the CBA, the
  Commissioner disapproves this Contract (or any amendment(s) thereto)
  within ten (10) days from the first business day following the day on
  which this Contract (or amendment) is first received, as directed, in
  his office, this Contract (or amendment) shall thereupon terminate and
  be of no further force or effect and the Team and the Player shall
  thereupon be relieved of their respective rights and liabilities
  thereunder, provided that such ten (10) day period shall be fifteen
  (15) days for any Contract (or amendment) so received during the
  period each year from July 1 through the date that is fourteen (14)
  days following the last day of the Moratorium Period. If the
  Commissioner's disapproval is subsequently overturned in any
  proceeding brought under the arbitration provisions of the CBA
  (including any appeals), the Contract shall again be valid and binding
  upon the Team and the Player, and the Commissioner shall be afforded
  another ten-day period to disapprove the Contract (based on the Team's
  Room at the time the Commissioner's disapproval is overturned) as set
  forth in the foregoing sentence. The NBA will inform the Players
  Association if the Commissioner disapproves this Contract (or any
  amendment(s) thereto) no later than one (1) day following the date of
  such disapproval.
\end{enumerate}

\begin{enumerate}
\def\labelenumi{\arabic{enumi}.}
\setcounter{enumi}{11}
\item
  \textbf{PROHIBITED ACTIVITIES.}

  The Player and the Team acknowledge and agree that the Player's
  participation in certain other activities may impair or destroy his
  ability and skill as a basketball player, and the Player's
  participation in any game or exhibition of basketball other than at
  the request of the Team may result in injury to him. Accordingly, the
  Player agrees that he will not, without the written consent of the
  Team, engage in any activity that a reasonable person would recognize
  as involving or exposing the participant to a substantial risk of
  bodily injury including, but not limited to: (i) sky-diving, hang
  gliding, snow skiing, rock or mountain climbing (as distinguished from
  hiking), water or jet skiing, whitewater rafting, rappelling, bungee
  jumping, trampoline jumping, and mountain biking; (ii) any fighting,
  boxing, or wrestling; (iii) using fireworks or participating in any
  activity involving firearms or other weapons; (iv) riding on electric
  scooters or hoverboards; (v) driving or riding on a motorcycle or
  moped or four-wheeling/off-roading of any kind; (vi) riding in or on
  any motorized vehicle in any kind of race or racing contest; (vii)
  operating an aircraft of any kind; (viii) engaging in any other
  activity excluded or prohibited by or under any insurance policy which
  the Team procures against the injury, illness or disability to or of
  the Player, or death of the Player, for which the Player has received
  written notice from the Team prior to the execution of this Contract;
  or (ix) participating in any game or exhibition of basketball,
  football, baseball, hockey, lacrosse, or other team sport or
  competition. If the Player violates this Paragraph 12, he shall be
  subject to discipline imposed by the Team and/or the Commissioner of
  the NBA. Nothing contained herein shall be intended to require the
  Player to obtain the written consent of the Team in order to enable
  the Player to participate in, as an amateur, the sports of golf,
  tennis, handball, swimming, hiking, softball, volleyball, and other
  similar sports that a reasonable person would not recognize as
  involving or exposing the participant to a substantial risk of bodily
  injury.
\item
  \textbf{PROMOTIONAL ACTIVITIES.}
\end{enumerate}

\begin{enumerate}
\def\labelenumi{(\alph{enumi})}
\tightlist
\item
  The Player agrees to allow the Team, the NBA, or any League-related
  entity to take pictures of the Player, alone or together with others,
  for still photographs, motion pictures, television, or other Media (as
  such term is defined in Article XXVIII of the CBA), at such reasonable
  times as the Team, the NBA or the League-related entity may designate.
  No matter by whom taken, such images may be used in any manner desired
  by either the Team, the NBA, or the League-related entity for
  publicity or promotional purposes for Teams or the NBA. The rights in
  any such images taken by the Team, the NBA, or the League-related
  entity shall belong to the Team, the NBA, or the League-related
  entity, as their interests may appear.
\item
  The Player agrees that, during any year of this Contract, he will not
  make public appearances, participate in radio or television programs,
  permit his picture to be taken, write or sponsor newspaper or magazine
  articles, or sponsor commercial products without the written consent
  of the Team, which shall not be withheld except in the reasonable
  interests of the Team or the NBA. The foregoing shall be interpreted
  in accordance with the decision in Portland Trail Blazers v. Darnell
  Valentine and Jim Paxson, Decision 86-2 (August 13, 1986).
\item
  Upon request, the Player shall consent to and make himself available
  for interviews by representatives of the media conducted at reasonable
  times.
\item
  In addition to the foregoing, and subject to the conditions and
  limitations set forth in Article II, Section 8 of the CBA, the Player
  agrees to participate, upon request, in all other reasonable
  promotional activities of the Team, the NBA, and any League-related
  entity. For each such promotional appearance made on behalf of a
  commercial sponsor of the Team, the Team agrees to pay the Player
  \$3,500 subject to Article II, Section 8 of the CBA, or, if the Team
  agrees, such higher amount that is consistent with the Team's past
  practice and not otherwise unreasonable.
\end{enumerate}

\begin{enumerate}
\def\labelenumi{\arabic{enumi}.}
\setcounter{enumi}{13}
\tightlist
\item
  \textbf{LEAGUE PROMOTION.}
\end{enumerate}

\begin{enumerate}
\def\labelenumi{(\alph{enumi})}
\tightlist
\item
  The NBA, all League-related entities, and the Teams may use, and may
  authorize others to use, in League Promotions, the Player's name,
  nickname, picture, portrait, likeness, signature, voice, caricature,
  biographical information, or other identifiable feature (collectively,
  ``Player Attributes''). The NBA, all League-related entities, and the
  Teams shall be entitled to use the Player's Player Attributes
  individually pursuant to the preceding sentence and may, but shall not
  be required to, use the Player's Player Attributes in a group or as
  one of multiple players. As used herein, ``League Promotion'' shall
  mean any and all uses intended to publicize, promote or market
  (including in any and all Media) (i) the NBA, any League-related
  entity that generates BRI (as defined in Article VII of the CBA), any
  Team, or any Player, (ii) any game in which a Team participates
  (including a Pre- Season, Exhibition, Regular Season, and Playoff
  game), including the sale of tickets to any such game, (iii) any
  telecast or other exhibition or distribution of (x) any such game or
  (y) any NBA-related or Team-related program or content, (iv) any NBA
  or Team facility, platform, or event, including the sale of tickets to
  any such event, or public service activity conducted by the NBA, a
  League-related entity that generates BRI, or a Team, or (v) the sport
  of basketball. For purposes of clarity, the foregoing rights of the
  NBA, League-related entities, and the Teams include the right and
  authority to use, and to authorize others to use, after the term of
  this Contract, any Player Attributes fixed in a tangible medium (e.g.,
  filmed, photographed, recorded or otherwise captured) during the term
  of this Contract solely for the purposes described herein.
\item
  Paragraph 14(a) above does not confer any right or authority to (i)
  use the Player's Player Attributes in a manner that constitutes an
  unauthorized Endorsement (as such term is defined and clarified in
  Article XXVIII of the CBA); (ii) use or authorize others to use the
  Player's Player Attributes (including in any program, content,
  platform, facility or event) in a manner that constitutes an
  Unauthorized Sponsor Promotion (as such term is defined and clarified
  in Paragraph 14(c) below); or (iii) authorize others (including any
  NBA sponsor or Team sponsor) to use the Player's Player Attributes on
  any product, product packaging, service or service- related materials
  sold or distributed by any third party, or any associated premiums.
\item
  An ``Unauthorized Sponsor Promotion'' shall mean a use of the Player's
  Player Attributes by a third party, or anyone on the third party's
  behalf (including, without limitation, the NBA, any League-related
  entity or any NBA Team), to promote, market, or advertise the third
  party's product, service, or brand; provided, however, the term
  Unauthorized Sponsor Promotion does not include the use of the
  Player's Player Attributes (i) by, or on behalf of, a telecaster or
  distributor of NBA games to promote the telecast or distribution of
  such NBA game, the fact that such third party is the telecaster or
  distributor of NBA content (e.g., an advertisement promoting MSG as
  the ``Home for New York Sports'' that includes a photograph of a
  Knicks player; or an ESPN advertisement promoting ESPN as the
  ``Worldwide Leader in Sports'' that includes footage of NBA players),
  or other sports-related programming of the telecaster or distributor
  (but not related parties of the telecaster or distributor -- e.g., the
  Player's Player Attributes may be used to promote an e-commerce
  company's video service that carries games and may carry other sports
  content, but may not be used to promote other products or services of
  the e-commerce company), (ii) by, or on behalf of, a telecaster or
  distributor of NBA programs or content to promote such NBA programs or
  content, (iii) by a third party, or anyone on the third party's
  behalf, for use in the promotion of the sale of tickets to an NBA game
  or event, or the sale of player-identified merchandise, (iv) by a
  third party when jointly licensed by the Players Association, or (v)
  by, or on behalf of, a third party to promote, market or advertise the
  third party's product, service or brand as part of a League Promotion
  or a promotional opportunity under Article XXVIII, Section 3(d)(y) of
  the CBA unless the execution (e.g., television advertisement, print
  ad, web ad) includes (x) more than (A) the third party's brand name
  and/or logo (either or both) (which use may not be persistent within
  such execution), provided that it shall not be considered persistent
  use of a third party's brand name and/or logo when used in conjunction
  with reference to the name and/or logo of the subject of such League
  Promotion for which the third party is a title or presenting sponsor
  (e.g., title sponsorship of the Slam Dunk Contest or a pre-game show)
  and (B) the subordinate and incidental promotion of the third party's
  products and services (e.g., not a call to action for a specific
  product or service), or (y) more than the subordinate and incidental
  promotion of the third party's products and services (it being
  understood that this clause (y) does not apply to an execution that
  includes the third party's brand name and/or logo, but clause (x)
  above does apply).
\item
  In addition to Paragraph 14(c)(i) above, solely for the current term
  of the contracts in effect as of the execution date of the CBA,
  between the NBA, any League-related entity, or any NBA Team, on the
  one hand, and ABC/ESPN, TNT, or any regional, local or international
  telecasters or distributors of NBA games, on the other hand (such as
  MSG, FoxSports Ohio, or Tencent, Inc.) (each, a ``Current
  Telecaster''), it shall not be an Unauthorized Sponsor Promotion for a
  Current Telecaster to use Player Attributes to promote (I) itself and
  its sports programming or its other sports content and (II) to the
  extent currently authorized by those contracts, its non-sports
  programming and content; the term Current Telecaster does not include
  related parties of the Current Telecaster. It shall be an Unauthorized
  Sponsor Promotion for the NBA, any League- related entity or any NBA
  Team to use the Player's Player Attributes as described in
  subparagraph (c), where such use (A) promotes the products, services
  or brands of a third party that does not generate BRI, and (B) is not
  jointly licensed with the Players Association. Any dispute regarding
  whether a use of Player Attributes is or is not an Unauthorized
  Sponsor Promotion shall be determined by the System Arbitrator on an
  expedited basis, as soon as possible following a hearing conducted
  within seventy-two (72) hours after commencement of the proceeding.
\item
  The Player does not and will not contest during or after the term of
  this Contract, and the Player hereby acknowledges, the exclusive
  rights of the NBA, all League-related entities that generate BRI and
  the Teams (i) to telecast, or otherwise distribute, transmit, exhibit
  or perform, on a live, delayed, or archived basis, in any and all
  Media, any performance by the Player under this Contract or the CBA
  (including in NBA games or any excerpts thereof) and (ii) to produce,
  license, offer for sale, sell, market, or otherwise, exhibit,
  distribute, transmit or perform (or authorize a third party to do any
  of the foregoing), on a live, delayed, or archived basis, any such
  performance in any and all Media, including, but not limited to, as
  part of programming or a content offering or in packaged or other
  electronic or digital media. The foregoing does not confer any right
  or authority to use the Player's Player Attributes in a manner that
  constitutes an unauthorized Endorsement or Unauthorized Sponsor
  Promotion (as such terms are defined and clarified in Article XXVIII
  of the CBA and Paragraph 14(c) above) or any right which would violate
  Article XXVIII, Section 3(f) of the CBA. For purposes of clarity and
  without limitation, any use of a Player's Player Attributes that has
  been expressly authorized by the Player (not including in this
  Contract) shall not be an unauthorized Endorsement or an Unauthorized
  Sponsor Promotion.
\end{enumerate}

\begin{enumerate}
\def\labelenumi{\arabic{enumi}.}
\setcounter{enumi}{14}
\item
  \textbf{TEAM DEFAULT.}

  In the event of an alleged default by the Team in the payments to the
  Player provided for by this Contract, or in the event of an alleged
  failure by the Team to perform any other material obligation that it
  has agreed to perform hereunder, the Player shall notify both the Team
  and the League in writing of the facts constituting such alleged
  default or alleged failure. If neither the Team nor the League shall
  cause such alleged default or alleged failure to be remedied within
  five (5) days after receipt of such written notice, the Players
  Association shall, on behalf of the Player, have the right to request
  that the dispute concerning such alleged default or alleged failure be
  referred immediately to the Grievance Arbitrator in accordance with
  the provisions of the CBA. If, as a result of such arbitration, an
  award issues in favor of the Player, and if neither the Team nor the
  League complies with such award within ten (10) days after the service
  thereof, the Player shall have the right, by a further written notice
  to the Team and the League, to terminate this Contract.
\item
  \textbf{TERMINATION.}
\end{enumerate}

\begin{enumerate}
\def\labelenumi{(\alph{enumi})}
\tightlist
\item
  The Team may terminate this Contract upon written notice to the Player
  if the Player shall:

  \begin{enumerate}
  \def\labelenumii{(\roman{enumii})}
  \tightlist
  \item
    at any time, fail, refuse, or neglect to conform his personal
    conduct to standards of good citizenship, good moral character
    (defined here to mean not engaging in acts of moral turpitude,
    whether or not such acts would constitute a crime), and good
    sportsmanship, to keep himself in first class physical condition, or
    to obey the Team's training rules;
  \item
    at any time commit a significant and inexcusable physical attack
    against any official or employee of the Team or the NBA (other than
    another player), or any person in attendance at any NBA game or
    event, considering the totality of the circumstances, including (but
    not limited to) the degree of provocation (if any) that may have led
    to the attack, the nature and scope of the attack, the Player's
    state of mind at the time of the attack, and the extent of any
    injury resulting from the attack;
  \item
    at any time, fail, in the sole opinion of the Team's management, to
    exhibit sufficient skill or competitive ability to qualify to
    continue as a member of the Team; provided, however, (A) that if
    this Contract is terminated by the Team, in accordance with the
    provisions of this subparagraph, prior to January 10 (or, in the
    case of a Two-Way Contract, prior to January 20) of any Season, and
    the Player, at the time of such termination, is unfit to play
    skilled basketball as the result of an injury resulting directly
    from his playing for the Team, the Player shall (subject to the
    provisions set forth in Exhibit 3) continue to receive his full Base
    Compensation, or, in the case of a Two-Way Contract, his full
    Two-Way NBADL Salary plus any Two-Way NBA Salary that has been
    earned by the Player), less all workers' compensation benefits
    (which, to the extent permitted by law, and if not deducted from the
    Player's Compensation by the Team, the Player hereby assigns to the
    Team) and any insurance provided for by the Team paid or payable to
    the Player by reason of said injury, until such time as the Player
    is fit to play skilled basketball, but not beyond the Season during
    which such termination occurred; and provided, further, (B) that if
    this Contract is terminated by the Team, in accordance with the
    provisions of this subparagraph, during the period from the January
    10 (or, in the case of a Two-Way Contract, from the January 20) of
    any Season through the end of such Season, the Player shall be
    entitled to receive his full Base Compensation for said Season (or,
    in the case of a Two-Way Contract, his Two-Way Annual NBADL Salary
    for such NBADL Regular Season (prorated as necessary if the Two-Way
    Contract was entered into after the start of the NBADL Regular
    Season) plus (i) any Two-Way NBA Salary earned by such Two-Way
    Player during such NBA Regular Season prior to the date of
    termination, less (ii) such Two-Way Player's Two-Way NBADL Salary
    covering the number of NBA Days of Service accrued by such Two-Way
    Player during such NBA Regular Season prior to the date of
    termination); or
  \item
    at any time, fail, refuse, or neglect to render his services
    hereunder or in any other manner materially breach this Contract.
  \end{enumerate}
\item
  If this Contract is terminated by the Team by reason of the Player's
  failure to render his services hereunder due to disability caused by
  an injury to the Player resulting directly from his playing for the
  Team and rendering him unfit to play skilled basketball, and notice of
  such injury is given by the Player as provided herein, the Player
  shall (subject to the provisions set forth in Exhibit 3) be entitled
  to receive his full Base Compensation for the Season in which the
  injury was sustained (or, in the case of a Two-Way Contract, his
  Two-Way Annual NBADL Salary for such NBADL Regular Season (prorated as
  necessary if the Two-Way Contract was entered into after the start of
  the NBADL Regular Season) plus (i) any Two-Way NBA Salary earned by
  such Two-Way Player during such NBA Regular Season prior to the date
  of termination, less (ii) such Two-Way Player's Two-Way NBADL Salary
  covering the number of NBA Days of Service accrued by such Two-Way
  Player during such NBA Regular Season prior to the date of
  termination), less all workers' compensation benefits (which, to the
  extent permitted by law, and if not deducted from the Player's
  Compensation by the Team, the Player hereby assigns to the Team) and
  any insurance provided for by the Team paid or payable to the Player
  by reason of said injury.
\item
  Notwithstanding the provisions of paragraph 16(b) above, if this
  Contract is terminated by the Team prior to the first game of a
  Regular Season by reason of the Player's failure to render his
  services hereunder due to an injury or condition sustained or suffered
  during a preceding Season, or after such Season but prior to the
  Player's participation in any basketball practice or game played for
  the Team, payment by the Team of any Compensation earned through the
  date of termination under paragraph 3(b) above, payment of the
  Player's board, lodging, and expense allowance during the training
  camp period, payment of the reasonable traveling expenses of the
  Player to his home city, and the expert training and coaching provided
  by the Team to the Player during the training season shall be full
  payment to the Player.
\item
  If this Contract is terminated by the Team during the period
  designated by the Team for attendance at NBA training camp, payment by
  the Team of any Compensation earned through the date of termination
  under paragraph 3(b) above, payment of the Player's board, lodging,
  and expense allowance during such period to the date of termination,
  payment of the reasonable traveling expenses of the Player to his home
  city, and the expert training and coaching provided by the Team to the
  Player during the training season shall be full payment to the Player.
\item
  If this Contract is terminated by the Team after the first game of a
  Regular Season, except in the case provided for in subparagraphs
  (a)(iii) and (b) of this paragraph 16, (A) with respect to a Standard
  NBA Contract, the Player shall be entitled to receive as full payment
  hereunder a sum of money which, when added to the salary which he has
  already received during such Season, will represent the same
  proportionate amount of the annual sum set forth in Exhibit 1 or
  Exhibit 1A hereto as the number of days of such Regular Season then
  past bears to the total number of days of such Regular Season, plus
  the reasonable traveling expenses of the Player to his home, and (B)
  with respect to a Two-Way Contract, the Player shall be entitled to
  receive as full payment hereunder a sum of money which, when added to
  the salary which he has already received during such Season, shall
  equal the sum of the Player's Two-Way NBA Salary (reflecting the
  number of NBA Days of Service the Player has accrued up until the date
  of termination) and the Player's Two-Way NBADL Salary (reflecting the
  number of NBADL Days of Service the Player has accrued up until the
  date of termination), plus the reasonable traveling expenses of the
  Player to his home.
\item
  If the Team proposes to terminate this Contract in accordance with
  subparagraph (a) of this paragraph 16, it must first comply with the
  following waiver procedure:

  \begin{enumerate}
  \def\labelenumii{(\roman{enumii})}
  \tightlist
  \item
    The Team shall request the NBA Commissioner to request waivers from
    all other clubs. Such waiver request may not be withdrawn.
  \item
    Upon receipt of the waiver request, any other NBA Team may claim
    assignment of this Contract at such waiver price as may be fixed by
    the League, the priority of claims to be determined in accordance
    with the NBA Constitution and By-Laws.
  \item
    If this Contract is so claimed, the Team agrees that it shall, upon
    the assignment of this Contract to the claiming Team, notify the
    Player of such assignment as provided in paragraph 10(c) hereof, and
    the Player agrees he shall report to the assignee Team as provided
    in said paragraph 10(c).
  \item
    If the Contract is not claimed prior to the expiration of the waiver
    period, it shall terminate and the Team shall promptly deliver
    written notice of termination to the Player.
  \item
    The NBA shall promptly notify the Players Association of the
    disposition of any waiver request.
  \item
    To the extent not inconsistent with the foregoing provisions of this
    subparagraph (f), the waiver procedures set forth in the NBA
    Constitution and By-Laws, a copy of which, as in effect on the date
    of this Contract, is attached hereto, shall govern.
  \end{enumerate}
\item
  Upon any termination of this Contract by the Player, all obligations
  of the Team to pay Compensation shall cease on the date of
  termination, except the obligation of the Team to pay the Player's
  Compensation to said date.
\end{enumerate}

\begin{enumerate}
\def\labelenumi{\arabic{enumi}.}
\setcounter{enumi}{16}
\item
  \textbf{DISPUTES.}

  In the event of any dispute arising between the Player and the Team
  relating to any matter arising under this Contract, or concerning the
  performance or interpretation thereof (except for a dispute arising
  under paragraph 9 hereof or as provided in paragraph 14 above), such
  dispute shall be resolved in accordance with the Grievance and
  Arbitration Procedure set forth in Article XXXI of the CBA.
\item
  \textbf{PLAYER NOT A MEMBER.}

  Nothing contained in this Contract or in any provision of the NBA
  Constitution and By-Laws shall be construed to constitute the Player a
  member of the NBA or to confer upon him any of the rights or
  privileges of a member thereof.
\item
  \textbf{RELEASE.}

  The Player hereby releases and waives any and all claims he may have,
  or that may arise during the term of this Contract, against (a) the
  NBA and its related entities, the NBADL and its related entities, and
  every member of the NBA or the NBADL, and every director, officer,
  owner, stockholder, trustee, partner, and employee of the NBA, NBADL
  and their respective related entities and/or any member of the NBA or
  NBADL and their related entities (excluding persons employed as
  players by any such member), and (b) any person retained by the NBA
  and/or the Players Association in connection with the NBA/NBPA
  Anti-Drug Program, the Grievance Arbitrator, the System Arbitrator,
  and any other arbitrator or expert retained by the NBA and/or the
  Players Association under the terms of the CBA, in both cases (a) and
  (b) above, arising out of, or in connection with, and whether or not
  by negligence, (i) any injury that is subject to the provisions of
  paragraph 7 hereof, (ii) any fighting or other form of violent and/or
  unsportsmanlike conduct occurring during the course of any practice,
  any NBADL game, and/or any NBA Exhibition, Regular Season, and/or
  Playoff game (in all cases on or adjacent to the playing floor or in
  or adjacent to any facility used for such practices or games), (iii)
  the testing procedures or the imposition of any penalties set forth in
  paragraph 8 hereof and in the NBA/NBPA Anti-Drug Program, or (iv) any
  injury suffered in the course of his employment as to which he has or
  would have a claim for workers' compensation benefits. The foregoing
  shall not apply to any claim of medical malpractice against a
  Team-affiliated physician or other medical personnel.
\item
  \textbf{ENTIRE AGREEMENT.}

  This Contract (including any Exhibits hereto) contains the entire
  agreement between the parties and, except as provided in the CBA, sets
  forth all components of the Player's Compensation from the Team or any
  Team Affiliate, and there are no other agreements or transactions of
  any kind (whether disclosed or undisclosed to the NBA), express or
  implied, oral or written, or promises, undertakings, representations,
  commitments, inducements, assurances of intent, or understandings of
  any kind (whether disclosed or undisclosed to the NBA) (a) concerning
  any future Renegotiation, Extension, or other amendment of this
  Contract or the entry into any new Player Contract, or (b) involving
  compensation or consideration of any kind (including, without
  limitation, an investment or business opportunity) to be paid,
  furnished, or made available to the Player, or any person or entity
  controlled by, related to, or acting with authority on behalf of the
  Player, by the Team or any Team Affiliate.
\end{enumerate}

\newpage

\textbf{\emph{EXAMINE THIS CONTRACT CAREFULLY BEFORE SIGNING IT.}}

THIS CONTRACT INCLUDES EXHIBITS \_\_\_\_\_\_\_\_\_\_, WHICH ARE ATTACHED
HERETO AND MADE A PART HEREOF.

IN WITNESS WHEREOF the Player has hereunto signed his name and the Team
has caused this Contract to be executed by its duly authorized officer.

\begin{longtable}[]{@{}ll@{}}
\toprule
Dated: \_\_\_\_\_\_\_\_\_\_\_\_\_\_\_\_\_\_\_\_\_ & By:
\_\_\_\_\_\_\_\_\_\_\_\_\_\_\_\_\_\_\_\_\_\_\_\_\_\_\_\_\tabularnewline
& Title:
\_\_\_\_\_\_\_\_\_\_\_\_\_\_\_\_\_\_\_\_\_\_\_\_\_\_\_\_\tabularnewline
& Team:
\_\_\_\_\_\_\_\_\_\_\_\_\_\_\_\_\_\_\_\_\_\_\_\_\_\_\_\_\tabularnewline
&\tabularnewline
Dated: \_\_\_\_\_\_\_\_\_\_\_\_\_\_\_\_\_\_\_\_\_ & By:
\_\_\_\_\_\_\_\_\_\_\_\_\_\_\_\_\_\_\_\_\_\_\_\_\_\_\_\_\tabularnewline
& Player:
\_\_\_\_\_\_\_\_\_\_\_\_\_\_\_\_\_\_\_\_\_\_\_\_\_\_\_\_\tabularnewline
& Player's Address:\tabularnewline
&
\_\_\_\_\_\_\_\_\_\_\_\_\_\_\_\_\_\_\_\_\_\_\_\_\_\_\_\_\_\_\_\_\_\_\_\_\tabularnewline
&
\_\_\_\_\_\_\_\_\_\_\_\_\_\_\_\_\_\_\_\_\_\_\_\_\_\_\_\_\_\_\_\_\_\_\_\_\tabularnewline
\bottomrule
\end{longtable}

\newpage

\subsection{EXCERPT FROM NBA
CONSTITUTION}\label{excerpt-from-nba-constitution}

\subsubsection{MISCONDUCT}\label{misconduct}

\begin{enumerate}
\def\labelenumi{\arabic{enumi}.}
\setcounter{enumi}{34}
\tightlist
\item
  The provisions of this Article 35 shall govern all Players in the
  Association, hereinafter referred to as ``Players.''

  \begin{enumerate}
  \def\labelenumii{(\alph{enumii})}
  \tightlist
  \item
    Each Member shall provide and require in every contract with any of
    its Players that they shall be bound and governed by the provisions
    of this Article. Each Member, at the direction of the Board of
    Governors or the Commissioner, as the case may be, shall take such
    action as the Board or the Commissioner may direct in order to
    effectuate the purposes of this Article.
  \item
    The Commissioner shall direct the dismissal and perpetual
    disqualification from any further association with the Association
    or any of its Members, of any Player found by the Commissioner after
    a hearing to have been guilty of offering, agreeing, conspiring,
    aiding or attempting to cause any game of basketball to result
    otherwise than on its merits.
  \item
    If in the opinion of the Commissioner any act or conduct of a Player
    at or during an Exhibition, Regular Season, or Playoff game has been
    prejudicial to or against the best interests of the Association or
    the game of basketball, the Commissioner shall impose upon such
    Player a fine not exceeding \$50,000, or may order for a time the
    suspension of any such Player from any connection or duties with
    Exhibition, Regular Season, or Playoff games, or he may order both
    such fine and suspension.
  \item
    The Commissioner shall have the power to suspend for a definite or
    indefinite period, or to impose a fine not exceeding \$50,000, or
    inflict both such suspension and fine upon any Player who, in his
    opinion, (i) shall have made or caused to be made any statement
    having, or that was designed to have, an effect prejudicial or
    detrimental to the best interests of basketball or of the
    Association or of a Member, or (ii) shall have been guilty of
    conduct that does not conform to standards of morality or fair play,
    that does not comply at all times with all federal, state, and local
    laws, or that is prejudicial or detrimental to the Association.
  \item
    Any Player who, directly or indirectly, entices, induces, persuades
    or attempts to entice, induce, or persuade any Player, Coach,
    Trainer, General Manager or any other person who is under contract
    to any other Member of the Association to enter into negotiations
    for or relating to his services or negotiates or contracts for such
    services shall, on being charged with such tampering, be given an
    opportunity to answer such charges after due notice and the
    Commissioner shall have the power to decide whether or not the
    charges have been sustained; in the event his decision is that the
    charges have been sustained, then the Commissioner shall have the
    power to suspend such Player for a definite or indefinite period, or
    to impose a fine not exceeding \$50,000, or inflict both such
    suspension and fine upon any such Player.
  \item
    Any Player who, directly or indirectly, wagers money or anything of
    value on the outcome of any game played by a Team in the league
    operated by the Association shall, on being charged with such
    wagering, be given an opportunity to answer such charges after due
    notice, and the decision of the Commissioner shall be final, binding
    and conclusive and unappealable. The penalty for such offense shall
    be within the absolute and sole discretion of the Commissioner and
    may include a fine, suspension, expulsion and/or perpetual
    disqualification from further association with the Association or
    any of its Members.
  \item
    Except for a penalty imposed under Paragraph (f) of this Article 35:
    (i) any challenge by a Team to the decisions and acts of the
    Commissioner pursuant to Article 35 shall be appealable to the Board
    of Governors, who shall determine such appeals in accordance with
    such rules and regulations as may be adopted by the Board in its
    absolute and sole discretion, and (ii) any challenge by a Player to
    the decisions or acts of the Commissioner pursuant to Article 35
    shall be governed by the provisions of Article XXXI of the NBA/NBPA
    Collective Bargaining Agreement then in effect.
  \end{enumerate}
\end{enumerate}

\newpage

\subsection{EXCERPT FROM NBA BY-LAWS}\label{excerpt-from-nba-by-laws}

5.01. \emph{Waiver Right.} Except for sales and trading between Members
in accordance with these By-Laws, no Member shall sell, option, or
otherwise assign the contract with, right to the services of, or right
to negotiate with, a Player without complying with the waiver procedure
prescribed by this Constitution and By-Laws.

5.02. \emph{Waiver Price.} The waiver price shall be \$1,000 per Player.

5.03. \emph{Waiver Procedure.} A Member desiring to secure waivers on a
Player shall notify the Commissioner or the Commissioner's designee, who
shall, on behalf of such Member, immediately notify all other Members of
the waiver request. Such Player shall be assumed to have been waived
unless a Member shall notify the Commissioner or the Commissioner's
designee in accordance with Section 5.04 of a claim to the rights to
such Player. Once a Member has notified the Commissioner or the
Commissioner's designee of its desire to secure waivers on a Player,
such notice may not be withdrawn. A Player remains the financial
responsibility of the Member placing him on waivers until the waiver
period set by the Commissioner or the Commissioner's designee has
expired.

5.04. \emph{Waiver Period.} If the Commissioner or the Commissioner's
designee distributes notice of request for waiver, any Members wishing
to claim rights to the Player shall do so by giving notice by telephone
and in a Writing of such claim to the Commissioner or the Commissioner's
designee within forty-eight (48) hours after the time of such notice. A
Team may not withdraw a claim to the rights to a Player on waivers.
Notwithstanding Article 40 of the NBA Constitution, Saturdays, Sundays
and legal holidays shall be included when computing the above-referenced
waiver period.

5.05. \emph{Waiver Preferences.} (a) In the event that more than one (1)
Member shall have claimed the rights to a Player placed on waivers, the
claiming Member with the lowest team standing at the time the waiver was
requested shall be entitled to acquire the rights to such Player. If the
request for waiver shall occur after the last day of the Season and
before 11:59 p.m. eastern time on the following November 30, the
standings at the close of the previous Season shall govern. (b) If the
winning percentage of two (2) claiming Teams are the same, then the tie
shall be determined, if possible, on the basis of the Regular Season
Games between the two (2) Teams during the Season or during the
preceding Season, as the case may be. If still tied, a toss of a coin
shall determine priority. For the purpose of determining standings, both
Conferences of the Association shall be deemed merged and a consolidated
standing shall control.

5.06. \emph{Players Acquired Through Waivers.} A Member who has acquired
the rights and title to the contract of a Player through the waiver
procedure may not sell or trade such rights for a period of thirty (30)
days after the acquisition thereof; provided, however, that if the
rights to such Player were acquired between Seasons, the 30-day period
described herein shall begin on the first day of the next succeeding
Season.

5.07. \emph{Additional Waiver Rules.} The Commissioner or the Board of
Governors may from time to time adopt additional rules (supplementary to
those set forth in this Section 5) with respect to the operation of the
waiver procedure. Such rules shall not be inconsistent with the
provisions of this Section 5 and shall apply to but shall not be limited
to the mechanics of notice, inadvertent omission of notification to a
Member, and rules of construction as to time.

\newpage

\subsection{AGENT CERTIFICATION}\label{agent-certification}

(To be completed only if Player was represented by an agent who
negotiated the terms of this Contract.)

I, the undersigned, having negotiated this Contract on behalf of
\_\_\_\_\_\_\_\_\_\_\_\_\_\_\_\_\_, do hereby swear and certify, under
penalties of perjury, that the terms of Paragraph 20 of this Contract
(``Entire Agreement'') are true and correct to the best of my knowledge
and belief.

\begin{longtable}[]{@{}r@{}}
\toprule
\_\_\_\_\_\_\_\_\_\_\_\_\_\_\_\_\_\_\_\_\_\_\_\_\_\_\_\_\_\_\_\_\_\_\_\_\_\_\_\_\_\_\_\_\_\tabularnewline
Player Reresentative\tabularnewline
\tabularnewline
\_\_\_\_\_\_\_\_\_\_\_\_\_\_\_\_\_\_\_\_\_\_\_\_\_\_\_\_\_\_\_\_\_\_\_\_\_\_\_\_\_\_\_\_\_\tabularnewline
(Print or Type Name of Player Representative)\tabularnewline
\bottomrule
\end{longtable}

State of\\
County of

On \_\_\_\_\_\_\_\_\_\_\_\_\_, before me personally came
\_\_\_\_\_\_\_\_\_\_\_\_\_\_ and acknowledged to me that he/she had
executed the foregoing Agent Certification.

\begin{longtable}[]{@{}r@{}}
\toprule
\_\_\_\_\_\_\_\_\_\_\_\_\_\_\_\_\_\_\_\_\_\_\_\_\_\_\_\_\_\_\_\_\_\_\_\_\_\_\_\_\_\_\_\_\_\tabularnewline
Notary Public\tabularnewline
\bottomrule
\end{longtable}

\newpage

\section{UNIFORM PLAYER CONTRACT}\label{uniform-player-contract-1}

\subsection{Exhibit 1 --- Compensation}\label{exhibit-1-compensation}

Player: \_\_\_\_\_\_\_\_\_\_\_\_\_\_\_\_\_\_\_\_\_\_\_\_\_\_\\
Team: \_\_\_\_\_\_\_\_\_\_\_\_\_\_\_\_\_\_\_\_\_\_\_\_\_\_\_\_\\
Date: \_\_\_\_\_\_\_\_\_\_\_\_\_\_\_\_\_\_\_\_\_\_\_\_\_\_\_\_

\begin{longtable}[]{@{}ccc@{}}
\toprule
Season & Current Base Compensation & Deferred Base
Compensation\tabularnewline
\midrule
\endhead
\_\_\_\_\_\_\_\_ & \_\_\_\_\_\_\_\_\_\_\_\_\_\_\_\_\_\_\_\_\_\_\_ &
\_\_\_\_\_\_\_\_\_\_\_\_\_\_\_\_\_\_\_\_\_\_\_\_\_\tabularnewline
\_\_\_\_\_\_\_\_ & \_\_\_\_\_\_\_\_\_\_\_\_\_\_\_\_\_\_\_\_\_\_\_ &
\_\_\_\_\_\_\_\_\_\_\_\_\_\_\_\_\_\_\_\_\_\_\_\_\_\tabularnewline
\_\_\_\_\_\_\_\_ & \_\_\_\_\_\_\_\_\_\_\_\_\_\_\_\_\_\_\_\_\_\_\_ &
\_\_\_\_\_\_\_\_\_\_\_\_\_\_\_\_\_\_\_\_\_\_\_\_\_\tabularnewline
\_\_\_\_\_\_\_\_ & \_\_\_\_\_\_\_\_\_\_\_\_\_\_\_\_\_\_\_\_\_\_\_ &
\_\_\_\_\_\_\_\_\_\_\_\_\_\_\_\_\_\_\_\_\_\_\_\_\_\tabularnewline
\_\_\_\_\_\_\_\_ & \_\_\_\_\_\_\_\_\_\_\_\_\_\_\_\_\_\_\_\_\_\_\_ &
\_\_\_\_\_\_\_\_\_\_\_\_\_\_\_\_\_\_\_\_\_\_\_\_\_\tabularnewline
\bottomrule
\end{longtable}

\textbf{Payment Schedule} (if different from paragraph 3):

Current Base:

Deferred Base:

\textbf{Signing Bonus} (include dates of payment):

\textbf{Incentive Compensation} (include dates of payment):

\textbf{Other Arrangements:}

\begin{longtable}[]{@{}ll@{}}
\toprule
Initialed: &\tabularnewline
\midrule
\endhead
\_\_\_\_\_\_\_\_\_\_\_\_\_\_ &
\_\_\_\_\_\_\_\_\_\_\_\_\_\_\tabularnewline
Player & Team\tabularnewline
\bottomrule
\end{longtable}

\newpage

\subsubsection{Exhibit 1A --- Compensation: Minimum Player
Salary}\label{exhibit-1a-compensation-minimum-player-salary}

Player: \_\_\_\_\_\_\_\_\_\_\_\_\_\_\_\_\_\_\_\_\_\_\_\_\_\_\_\_\\
Team: \_\_\_\_\_\_\_\_\_\_\_\_\_\_\_\_\_\_\_\_\_\_\_\_\_\_\_\_\\
Date: \_\_\_\_\_\_\_\_\_\_\_\_\_\_\_\_\_\_\_\_\_\_\_\_\_\_\_\_

\begin{longtable}[]{@{}ccc@{}}
\toprule
Season & Current Base Compensation & Deferred Base
Compensation\tabularnewline
\midrule
\endhead
\_\_\_\_\_\_\_\_ & \_\_\_\_\_\_\_\_\_\_\_\_\_\_\_\_\_\_\_\_\_\_\_ &
\_\_\_\_\_\_\_\_\_\_\_\_\_\_\_\_\_\_\_\_\_\_\_\_\_\tabularnewline
\_\_\_\_\_\_\_\_ & \_\_\_\_\_\_\_\_\_\_\_\_\_\_\_\_\_\_\_\_\_\_\_ &
\_\_\_\_\_\_\_\_\_\_\_\_\_\_\_\_\_\_\_\_\_\_\_\_\_\tabularnewline
\_\_\_\_\_\_\_\_ & \_\_\_\_\_\_\_\_\_\_\_\_\_\_\_\_\_\_\_\_\_\_\_ &
\_\_\_\_\_\_\_\_\_\_\_\_\_\_\_\_\_\_\_\_\_\_\_\_\_\tabularnewline
\_\_\_\_\_\_\_\_ & \_\_\_\_\_\_\_\_\_\_\_\_\_\_\_\_\_\_\_\_\_\_\_ &
\_\_\_\_\_\_\_\_\_\_\_\_\_\_\_\_\_\_\_\_\_\_\_\_\_\tabularnewline
\_\_\_\_\_\_\_\_ & \_\_\_\_\_\_\_\_\_\_\_\_\_\_\_\_\_\_\_\_\_\_\_ &
\_\_\_\_\_\_\_\_\_\_\_\_\_\_\_\_\_\_\_\_\_\_\_\_\_\tabularnewline
\bottomrule
\end{longtable}

\textbf{This Contract is intended to provide for a Base Compensation for
the \_\_\_\_\_\_\_\_\_\_\_\_\_\_ Season(s) equal to the Minimum Player
Salary for such Season(s) (with no bonuses of any kind) and shall be
deemed amended to the extent necessary to so provide.}

\textbf{Payment Schedule} (if different from paragraph 3):

\textbf{Other Arrangements:}

\begin{longtable}[]{@{}ll@{}}
\toprule
Initialed: &\tabularnewline
\midrule
\endhead
\_\_\_\_\_\_\_\_\_\_\_\_\_\_ &
\_\_\_\_\_\_\_\_\_\_\_\_\_\_\tabularnewline
Player & Team\tabularnewline
\bottomrule
\end{longtable}

\newpage

\subsubsection{Exhibit 1B --- Compensation: Two-Way Player
Salary}\label{exhibit-1b-compensation-two-way-player-salary}

Player: \_\_\_\_\_\_\_\_\_\_\_\_\_\_\_\_\_\_\_\_\_\_\_\_\_\_\\
Team: \_\_\_\_\_\_\_\_\_\_\_\_\_\_\_\_\_\_\_\_\_\_\_\_\_\_\_\_\\
Date: \_\_\_\_\_\_\_\_\_\_\_\_\_\_\_\_\_\_\_\_\_\_\_\_\_\_\_\_

\begin{longtable}[]{@{}ccc@{}}
\toprule
Season & Two-Way NBA Salary (daily rate) & Two-Way NBADL Salary (daily
rate)\tabularnewline
\midrule
\endhead
\_\_\_\_\_\_ & \_\_\_\_\_\_\_\_\_\_\_\_\_\_\_\_\_\_\_\_\_\_\_\_\_ &
\_\_\_\_\_\_\_\_\_\_\_\_\_\_\_\_\_\_\_\_\_\_\_\_\_\tabularnewline
\_\_\_\_\_\_ & \_\_\_\_\_\_\_\_\_\_\_\_\_\_\_\_\_\_\_\_\_\_\_\_\_ &
\_\_\_\_\_\_\_\_\_\_\_\_\_\_\_\_\_\_\_\_\_\_\_\_\_\tabularnewline
\_\_\_\_\_\_ & \_\_\_\_\_\_\_\_\_\_\_\_\_\_\_\_\_\_\_\_\_\_\_\_\_ &
\_\_\_\_\_\_\_\_\_\_\_\_\_\_\_\_\_\_\_\_\_\_\_\_\_\tabularnewline
\_\_\_\_\_\_ & \_\_\_\_\_\_\_\_\_\_\_\_\_\_\_\_\_\_\_\_\_\_\_\_\_ &
\_\_\_\_\_\_\_\_\_\_\_\_\_\_\_\_\_\_\_\_\_\_\_\_\_\tabularnewline
\_\_\_\_\_\_ & \_\_\_\_\_\_\_\_\_\_\_\_\_\_\_\_\_\_\_\_\_\_\_\_\_ &
\_\_\_\_\_\_\_\_\_\_\_\_\_\_\_\_\_\_\_\_\_\_\_\_\_\tabularnewline
\bottomrule
\end{longtable}

This Contract is intended to provide for a Base Compensation for the
\_\_\_\_\_\_\_\_\_\_\_\_\_\_ Season(s) equal to the Two-Way Player
Salary for such Season(s) (with no bonuses of any kind) and shall be
deemed amended to the extent necessary to so provide.

Standard NBA Contract Conversion Option: Team shall have the option to
convert this Contract to a Standard NBA Contract (``Standard NBA
Contract Conversion Option''). Team's Standard NBA Contract Conversion
Option may be exercised by providing written notice to Player that is
either personally delivered to Player or his representative or sent by
email or pre-paid certified, registered, or overnight mail to the last
known address of Player or his representative with a copy to the Players
Association and the NBA. If Team exercises the Standard NBA Contract
Conversion Option, the Base Compensation amount set forth above in this
Exhibit 1B will immediately become null and void and of no further force
or effect, Player's Compensation shall be equal to the Player's
applicable Minimum Player Salary for a term equal to the remainder of
the original term of this Contract beginning on the date such option is
exercised, and all other terms and conditions of this Contract,
including the Base Compensation protection set forth in Exhibit 2 (if
any), shall remain applicable.

\begin{longtable}[]{@{}ll@{}}
\toprule
Initialed: &\tabularnewline
\midrule
\endhead
\_\_\_\_\_\_\_\_\_\_\_\_\_\_ &
\_\_\_\_\_\_\_\_\_\_\_\_\_\_\tabularnewline
Player & Team\tabularnewline
\bottomrule
\end{longtable}

\newpage

\subsection{Exhibit 2 --- Compensation
Protection}\label{exhibit-2-compensation-protection}

Player: \_\_\_\_\_\_\_\_\_\_\_\_\_\_\_\_\_\_\_\_\_\_\_\_\_\_\\
Team: \_\_\_\_\_\_\_\_\_\_\_\_\_\_\_\_\_\_\_\_\_\_\_\_\_\_\\
Date: \_\_\_\_\_\_\_\_\_\_\_\_\_\_\_\_\_\_\_\_\_\_\_\_\_\_\_

\begin{longtable}[]{@{}cccc@{}}
\toprule
\begin{minipage}[b]{0.07\columnwidth}\centering\strut
Season\strut
\end{minipage} & \begin{minipage}[b]{0.20\columnwidth}\centering\strut
Type of Protection\strut
\end{minipage} & \begin{minipage}[b]{0.22\columnwidth}\centering\strut
Amount of Protection\strut
\end{minipage} & \begin{minipage}[b]{0.32\columnwidth}\centering\strut
Additional Conditions or Limitations\strut
\end{minipage}\tabularnewline
\midrule
\endhead
\begin{minipage}[t]{0.07\columnwidth}\centering\strut
\_\_\_\_\_\strut
\end{minipage} & \begin{minipage}[t]{0.20\columnwidth}\centering\strut
\_\_\_\_\_\_\_\_\_\_\_\strut
\end{minipage} & \begin{minipage}[t]{0.22\columnwidth}\centering\strut
\_\_\_\_\_\_\_\_\_\_\_\_\_\_\_\strut
\end{minipage} & \begin{minipage}[t]{0.32\columnwidth}\centering\strut
\_\_\_\_\_\_\_\_\_\_\_\_\_\_\_\_\_\_\strut
\end{minipage}\tabularnewline
\begin{minipage}[t]{0.07\columnwidth}\centering\strut
\_\_\_\_\_\strut
\end{minipage} & \begin{minipage}[t]{0.20\columnwidth}\centering\strut
\_\_\_\_\_\_\_\_\_\_\_\strut
\end{minipage} & \begin{minipage}[t]{0.22\columnwidth}\centering\strut
\_\_\_\_\_\_\_\_\_\_\_\_\_\_\_\strut
\end{minipage} & \begin{minipage}[t]{0.32\columnwidth}\centering\strut
\_\_\_\_\_\_\_\_\_\_\_\_\_\_\_\_\_\_\strut
\end{minipage}\tabularnewline
\begin{minipage}[t]{0.07\columnwidth}\centering\strut
\_\_\_\_\_\strut
\end{minipage} & \begin{minipage}[t]{0.20\columnwidth}\centering\strut
\_\_\_\_\_\_\_\_\_\_\_\strut
\end{minipage} & \begin{minipage}[t]{0.22\columnwidth}\centering\strut
\_\_\_\_\_\_\_\_\_\_\_\_\_\_\_\strut
\end{minipage} & \begin{minipage}[t]{0.32\columnwidth}\centering\strut
\_\_\_\_\_\_\_\_\_\_\_\_\_\_\_\_\_\_\strut
\end{minipage}\tabularnewline
\begin{minipage}[t]{0.07\columnwidth}\centering\strut
\_\_\_\_\_\strut
\end{minipage} & \begin{minipage}[t]{0.20\columnwidth}\centering\strut
\_\_\_\_\_\_\_\_\_\_\_\strut
\end{minipage} & \begin{minipage}[t]{0.22\columnwidth}\centering\strut
\_\_\_\_\_\_\_\_\_\_\_\_\_\_\_\strut
\end{minipage} & \begin{minipage}[t]{0.32\columnwidth}\centering\strut
\_\_\_\_\_\_\_\_\_\_\_\_\_\_\_\_\_\_\strut
\end{minipage}\tabularnewline
\begin{minipage}[t]{0.07\columnwidth}\centering\strut
\_\_\_\_\_\strut
\end{minipage} & \begin{minipage}[t]{0.20\columnwidth}\centering\strut
\_\_\_\_\_\_\_\_\_\_\_\strut
\end{minipage} & \begin{minipage}[t]{0.22\columnwidth}\centering\strut
\_\_\_\_\_\_\_\_\_\_\_\_\_\_\_\strut
\end{minipage} & \begin{minipage}[t]{0.32\columnwidth}\centering\strut
\_\_\_\_\_\_\_\_\_\_\_\_\_\_\_\_\_\_\strut
\end{minipage}\tabularnewline
\bottomrule
\end{longtable}

\textbf{Automatic Stretch Provision:} In the event that the Team
terminates this Contract (resulting in the Player's separation of
service from the Team), and the Team is obligated thereafter to make
payments to the Player pursuant to this Exhibit 2, such payments shall
be made in accordance with the following schedule:

\begin{enumerate}
\def\labelenumi{(\arabic{enumi})}
\tightlist
\item
  If, as of the date of the Player's separation from service, the
  aggregate amount owed to the Player pursuant to this Exhibit 2 is two
  hundred fifty thousand dollars (\$250,000) or less, such amount shall
  be paid in accordance with the semi-monthly installments prescribed by
  the payment schedule set forth in this Contract. Each installment
  shall equal the amount of Base Compensation that was due per pay
  period for the applicable Season immediately before the Player's
  separation until the aggregate amount of the remaining Base
  Compensation owed to the Player pursuant to this Exhibit 2 is paid in
  full.
\item
  If, as of the date of the Player's separation from service, the
  aggregate amount owed to the Player pursuant to this Exhibit 2 exceeds
  two hundred fifty thousand dollars (\$250,000), such amount shall be
  paid as follows:

  \begin{enumerate}
  \def\labelenumii{(\roman{enumii})}
  \tightlist
  \item
    The Base Compensation, if any, owed to the Player pursuant to this
    Exhibit 2 with respect to the ``current season'' (as defined below)
    at the time when the request for waivers on the Player is made shall
    be paid in accordance with the payment schedule set forth in this
    Contract. Each installment shall equal the amount of Base
    Compensation that was due per pay period immediately before the
    Player's separation until the aggregate amount of the remaining Base
    Compensation owed to the Player pursuant to this Exhibit 2 with
    respect to the current season is paid in full. For purposes of this
    Paragraph 2 only, the ``current season'' means the period from
    September 1 through June 30.
  \item
    The remaining Base Compensation, if any, owed to the Player pursuant
    to this Exhibit 2 shall be aggregated and paid in equal amounts per
    year over a period equal to twice the number of NBA Seasons
    (including any Season covered by a Player Option Year) remaining on
    this Contract following the date upon which the request for waivers
    occurred, plus one NBA Season. For this purpose, if the request for
    waivers is made during the period from September 1 through June 30,
    the number of NBA Seasons remaining on this Contract shall not
    include the current season (as defined in subparagraph (i) above).
    The rescheduled payments described above shall be paid over the
    applicable number of NBA Seasons in equal semi-monthly installments
    on the pay dates prescribed by Paragraph 3(a) of this Contract.
  \end{enumerate}
\end{enumerate}

For purposes of Section 409A of the Internal Revenue Code, each
installment of the amount payable pursuant to this Exhibit 2 shall be
treated as a separate payment.

\textbf{Standard Conditions or Limitations:} The Player's Base
Compensation protection for each Season hereunder shall not be
applicable if the Player's lack of skill, death, injury or illness
and/or mental disability (as applicable) results from the Player's:

\begin{enumerate}
\def\labelenumi{(\arabic{enumi})}
\tightlist
\item
  participation in activities prohibited by paragraph 12 of the Contract
  (as such paragraph may be modified by Exhibit 5), which includes,
  among other things, engaging in any activity that a reasonable person
  would recognize as involving or exposing the participant to a
  substantial risk of bodily injury including, but not limited to (i)
  sky-diving, hang gliding, snow skiing, rock or mountain climbing (as
  distinguished from hiking), water or jet skiing, whitewater rafting,
  rappelling, bungee jumping, trampoline jumping and mountain biking;
  (ii) any fighting, boxing, or wrestling; (iii) using fireworks or
  participating in any activity involving firearms or other weapons;
  (iv) riding on electric scooters or hoverboards; (v) driving or riding
  on a motorcycle or moped or four-wheeling/off-roading of any kind;
  (vi) riding in or on any motorized vehicle in any kind of race or
  racing contest; (vii) operating an aircraft of any kind; (viii)
  engaging in any other activity excluded or prohibited by or under any
  insurance policy which the Team procures against the injury, illness
  or disability to or of the Player, or death of the Player, for which
  the Player has received written notice from the Team prior to the
  execution of this Contract; or (ix) participating in any game or
  exhibition of basketball, football, baseball, hockey, lacrosse, or
  other team sport or competition;
\item
  intentional self-inflicted injury, attempted suicide and/or suicide;
\item
  abuse of alcohol;
\item
  use of any Prohibited Substance or controlled substance;
\item
  abuse of or addiction to prescription drugs;
\item
  conduct occurring during a commission of any felony for which the
  player is convicted (including a plea of guilty, no contest or nolo
  contendere);
\item
  participation in any riot, insurrection or war or other military
  activities; or
\item
  failure to comply with the requirements of Paragraphs 7(d) -- (i) of
  this Contract.
\end{enumerate}

\textbf{Additional Conditions or Limitations:}

\begin{longtable}[]{@{}ll@{}}
\toprule
Initialed: &\tabularnewline
\midrule
\endhead
\_\_\_\_\_\_\_\_\_\_\_\_\_\_ &
\_\_\_\_\_\_\_\_\_\_\_\_\_\_\tabularnewline
Player & Team\tabularnewline
\bottomrule
\end{longtable}

\newpage

\subsection{Exhibit 3 --- Prior Injury
Exclusion}\label{exhibit-3-prior-injury-exclusion}

Player: \_\_\_\_\_\_\_\_\_\_\_\_\_\_\_\_\_\_\_\_\_\_\_\_\_\_\_\_\\
Team: \_\_\_\_\_\_\_\_\_\_\_\_\_\_\_\_\_\_\_\_\_\_\_\_\_\_\_\_\\
Date: \_\_\_\_\_\_\_\_\_\_\_\_\_\_\_\_\_\_\_\_\_\_\_\_\_\_\_\_

The Player's right to receive his Compensation as set forth in
paragraphs 7(c), 16(a)(iii), 16(b) of this Contract, or otherwise is
limited or eliminated with respect to the following reinjury of the
injury or aggravation of the condition set forth below:

\begin{longtable}[]{@{}l@{}}
\toprule
Describe injury or condition:\tabularnewline
\midrule
\endhead
\_\_\_\_\_\_\_\_\_\_\_\_\_\_\_\_\_\_\_\_\_\_\_\_\_\_\_\_\_\_\_\_\_\_\_\_\_\_\_\_\_\_\_\_\_\_\_\_\_\_\_\_\_\_\_\_\_\_\_\_\_\tabularnewline
\_\_\_\_\_\_\_\_\_\_\_\_\_\_\_\_\_\_\_\_\_\_\_\_\_\_\_\_\_\_\_\_\_\_\_\_\_\_\_\_\_\_\_\_\_\_\_\_\_\_\_\_\_\_\_\_\_\_\_\_\_\tabularnewline
\_\_\_\_\_\_\_\_\_\_\_\_\_\_\_\_\_\_\_\_\_\_\_\_\_\_\_\_\_\_\_\_\_\_\_\_\_\_\_\_\_\_\_\_\_\_\_\_\_\_\_\_\_\_\_\_\_\_\_\_\_\tabularnewline
\_\_\_\_\_\_\_\_\_\_\_\_\_\_\_\_\_\_\_\_\_\_\_\_\_\_\_\_\_\_\_\_\_\_\_\_\_\_\_\_\_\_\_\_\_\_\_\_\_\_\_\_\_\_\_\_\_\_\_\_\_\tabularnewline
\_\_\_\_\_\_\_\_\_\_\_\_\_\_\_\_\_\_\_\_\_\_\_\_\_\_\_\_\_\_\_\_\_\_\_\_\_\_\_\_\_\_\_\_\_\_\_\_\_\_\_\_\_\_\_\_\_\_\_\_\_\tabularnewline
\_\_\_\_\_\_\_\_\_\_\_\_\_\_\_\_\_\_\_\_\_\_\_\_\_\_\_\_\_\_\_\_\_\_\_\_\_\_\_\_\_\_\_\_\_\_\_\_\_\_\_\_\_\_\_\_\_\_\_\_\_\tabularnewline
\_\_\_\_\_\_\_\_\_\_\_\_\_\_\_\_\_\_\_\_\_\_\_\_\_\_\_\_\_\_\_\_\_\_\_\_\_\_\_\_\_\_\_\_\_\_\_\_\_\_\_\_\_\_\_\_\_\_\_\_\_\tabularnewline
\_\_\_\_\_\_\_\_\_\_\_\_\_\_\_\_\_\_\_\_\_\_\_\_\_\_\_\_\_\_\_\_\_\_\_\_\_\_\_\_\_\_\_\_\_\_\_\_\_\_\_\_\_\_\_\_\_\_\_\_\_\tabularnewline
\_\_\_\_\_\_\_\_\_\_\_\_\_\_\_\_\_\_\_\_\_\_\_\_\_\_\_\_\_\_\_\_\_\_\_\_\_\_\_\_\_\_\_\_\_\_\_\_\_\_\_\_\_\_\_\_\_\_\_\_\_\tabularnewline
\_\_\_\_\_\_\_\_\_\_\_\_\_\_\_\_\_\_\_\_\_\_\_\_\_\_\_\_\_\_\_\_\_\_\_\_\_\_\_\_\_\_\_\_\_\_\_\_\_\_\_\_\_\_\_\_\_\_\_\_\_\tabularnewline
\_\_\_\_\_\_\_\_\_\_\_\_\_\_\_\_\_\_\_\_\_\_\_\_\_\_\_\_\_\_\_\_\_\_\_\_\_\_\_\_\_\_\_\_\_\_\_\_\_\_\_\_\_\_\_\_\_\_\_\_\_\tabularnewline
\bottomrule
\end{longtable}

\begin{longtable}[]{@{}l@{}}
\toprule
\begin{minipage}[b]{0.97\columnwidth}\raggedright\strut
Describe the extent to which liability for Compensation is limited or
eliminated:\strut
\end{minipage}\tabularnewline
\midrule
\endhead
\begin{minipage}[t]{0.97\columnwidth}\raggedright\strut
\_\_\_\_\_\_\_\_\_\_\_\_\_\_\_\_\_\_\_\_\_\_\_\_\_\_\_\_\_\_\_\_\_\_\_\_\_\_\_\_\_\_\_\_\_\_\_\_\_\_\_\_\_\_\_\_\_\_\_\_\_\strut
\end{minipage}\tabularnewline
\begin{minipage}[t]{0.97\columnwidth}\raggedright\strut
\_\_\_\_\_\_\_\_\_\_\_\_\_\_\_\_\_\_\_\_\_\_\_\_\_\_\_\_\_\_\_\_\_\_\_\_\_\_\_\_\_\_\_\_\_\_\_\_\_\_\_\_\_\_\_\_\_\_\_\_\_\strut
\end{minipage}\tabularnewline
\begin{minipage}[t]{0.97\columnwidth}\raggedright\strut
\_\_\_\_\_\_\_\_\_\_\_\_\_\_\_\_\_\_\_\_\_\_\_\_\_\_\_\_\_\_\_\_\_\_\_\_\_\_\_\_\_\_\_\_\_\_\_\_\_\_\_\_\_\_\_\_\_\_\_\_\_\strut
\end{minipage}\tabularnewline
\begin{minipage}[t]{0.97\columnwidth}\raggedright\strut
\_\_\_\_\_\_\_\_\_\_\_\_\_\_\_\_\_\_\_\_\_\_\_\_\_\_\_\_\_\_\_\_\_\_\_\_\_\_\_\_\_\_\_\_\_\_\_\_\_\_\_\_\_\_\_\_\_\_\_\_\_\strut
\end{minipage}\tabularnewline
\begin{minipage}[t]{0.97\columnwidth}\raggedright\strut
\_\_\_\_\_\_\_\_\_\_\_\_\_\_\_\_\_\_\_\_\_\_\_\_\_\_\_\_\_\_\_\_\_\_\_\_\_\_\_\_\_\_\_\_\_\_\_\_\_\_\_\_\_\_\_\_\_\_\_\_\_\strut
\end{minipage}\tabularnewline
\begin{minipage}[t]{0.97\columnwidth}\raggedright\strut
\_\_\_\_\_\_\_\_\_\_\_\_\_\_\_\_\_\_\_\_\_\_\_\_\_\_\_\_\_\_\_\_\_\_\_\_\_\_\_\_\_\_\_\_\_\_\_\_\_\_\_\_\_\_\_\_\_\_\_\_\_\strut
\end{minipage}\tabularnewline
\begin{minipage}[t]{0.97\columnwidth}\raggedright\strut
\_\_\_\_\_\_\_\_\_\_\_\_\_\_\_\_\_\_\_\_\_\_\_\_\_\_\_\_\_\_\_\_\_\_\_\_\_\_\_\_\_\_\_\_\_\_\_\_\_\_\_\_\_\_\_\_\_\_\_\_\_\strut
\end{minipage}\tabularnewline
\begin{minipage}[t]{0.97\columnwidth}\raggedright\strut
\_\_\_\_\_\_\_\_\_\_\_\_\_\_\_\_\_\_\_\_\_\_\_\_\_\_\_\_\_\_\_\_\_\_\_\_\_\_\_\_\_\_\_\_\_\_\_\_\_\_\_\_\_\_\_\_\_\_\_\_\_\strut
\end{minipage}\tabularnewline
\begin{minipage}[t]{0.97\columnwidth}\raggedright\strut
\_\_\_\_\_\_\_\_\_\_\_\_\_\_\_\_\_\_\_\_\_\_\_\_\_\_\_\_\_\_\_\_\_\_\_\_\_\_\_\_\_\_\_\_\_\_\_\_\_\_\_\_\_\_\_\_\_\_\_\_\_\strut
\end{minipage}\tabularnewline
\begin{minipage}[t]{0.97\columnwidth}\raggedright\strut
\_\_\_\_\_\_\_\_\_\_\_\_\_\_\_\_\_\_\_\_\_\_\_\_\_\_\_\_\_\_\_\_\_\_\_\_\_\_\_\_\_\_\_\_\_\_\_\_\_\_\_\_\_\_\_\_\_\_\_\_\_\strut
\end{minipage}\tabularnewline
\begin{minipage}[t]{0.97\columnwidth}\raggedright\strut
\_\_\_\_\_\_\_\_\_\_\_\_\_\_\_\_\_\_\_\_\_\_\_\_\_\_\_\_\_\_\_\_\_\_\_\_\_\_\_\_\_\_\_\_\_\_\_\_\_\_\_\_\_\_\_\_\_\_\_\_\_\strut
\end{minipage}\tabularnewline
\bottomrule
\end{longtable}

\begin{longtable}[]{@{}ll@{}}
\toprule
Initialed: &\tabularnewline
\midrule
\endhead
\_\_\_\_\_\_\_\_\_\_\_\_\_\_ &
\_\_\_\_\_\_\_\_\_\_\_\_\_\_\tabularnewline
Player & Team\tabularnewline
\bottomrule
\end{longtable}

\newpage

\subsection{Exhibit 4 --- Trade
Payments}\label{exhibit-4-trade-payments}

Player: \_\_\_\_\_\_\_\_\_\_\_\_\_\_\_\_\_\_\_\_\_\_\_\_\_\_\_\_\\
Team: \_\_\_\_\_\_\_\_\_\_\_\_\_\_\_\_\_\_\_\_\_\_\_\_\_\_\_\_\\
Date: \_\_\_\_\_\_\_\_\_\_\_\_\_\_\_\_\_\_\_\_\_\_\_\_\_\_\_\_

In the event this Contract is traded by the Team executing the Contract
to another NBA Team, the Player shall be entitled to receive from the
assignor Team, within thirty (30) days of the date of such trade, the
following payment:

\begin{longtable}[]{@{}l@{}}
\toprule
\_\_\_\_\_\_\_\_\_\_\_\_\_\_\_\_\_\_\_\_\_\_\_\_\_\_\_\_\_\_\_\_\_\_\_\_\_\_\_\_\_\_\_\_\_\_\_\_\_\_\_\_\_\_\_\_\_\_\_\_\_\tabularnewline
\_\_\_\_\_\_\_\_\_\_\_\_\_\_\_\_\_\_\_\_\_\_\_\_\_\_\_\_\_\_\_\_\_\_\_\_\_\_\_\_\_\_\_\_\_\_\_\_\_\_\_\_\_\_\_\_\_\_\_\_\_\tabularnewline
\_\_\_\_\_\_\_\_\_\_\_\_\_\_\_\_\_\_\_\_\_\_\_\_\_\_\_\_\_\_\_\_\_\_\_\_\_\_\_\_\_\_\_\_\_\_\_\_\_\_\_\_\_\_\_\_\_\_\_\_\_\tabularnewline
\_\_\_\_\_\_\_\_\_\_\_\_\_\_\_\_\_\_\_\_\_\_\_\_\_\_\_\_\_\_\_\_\_\_\_\_\_\_\_\_\_\_\_\_\_\_\_\_\_\_\_\_\_\_\_\_\_\_\_\_\_\tabularnewline
\_\_\_\_\_\_\_\_\_\_\_\_\_\_\_\_\_\_\_\_\_\_\_\_\_\_\_\_\_\_\_\_\_\_\_\_\_\_\_\_\_\_\_\_\_\_\_\_\_\_\_\_\_\_\_\_\_\_\_\_\_\tabularnewline
\_\_\_\_\_\_\_\_\_\_\_\_\_\_\_\_\_\_\_\_\_\_\_\_\_\_\_\_\_\_\_\_\_\_\_\_\_\_\_\_\_\_\_\_\_\_\_\_\_\_\_\_\_\_\_\_\_\_\_\_\_\tabularnewline
\_\_\_\_\_\_\_\_\_\_\_\_\_\_\_\_\_\_\_\_\_\_\_\_\_\_\_\_\_\_\_\_\_\_\_\_\_\_\_\_\_\_\_\_\_\_\_\_\_\_\_\_\_\_\_\_\_\_\_\_\_\tabularnewline
\_\_\_\_\_\_\_\_\_\_\_\_\_\_\_\_\_\_\_\_\_\_\_\_\_\_\_\_\_\_\_\_\_\_\_\_\_\_\_\_\_\_\_\_\_\_\_\_\_\_\_\_\_\_\_\_\_\_\_\_\_\tabularnewline
\_\_\_\_\_\_\_\_\_\_\_\_\_\_\_\_\_\_\_\_\_\_\_\_\_\_\_\_\_\_\_\_\_\_\_\_\_\_\_\_\_\_\_\_\_\_\_\_\_\_\_\_\_\_\_\_\_\_\_\_\_\tabularnewline
\_\_\_\_\_\_\_\_\_\_\_\_\_\_\_\_\_\_\_\_\_\_\_\_\_\_\_\_\_\_\_\_\_\_\_\_\_\_\_\_\_\_\_\_\_\_\_\_\_\_\_\_\_\_\_\_\_\_\_\_\_\tabularnewline
\_\_\_\_\_\_\_\_\_\_\_\_\_\_\_\_\_\_\_\_\_\_\_\_\_\_\_\_\_\_\_\_\_\_\_\_\_\_\_\_\_\_\_\_\_\_\_\_\_\_\_\_\_\_\_\_\_\_\_\_\_\tabularnewline
\_\_\_\_\_\_\_\_\_\_\_\_\_\_\_\_\_\_\_\_\_\_\_\_\_\_\_\_\_\_\_\_\_\_\_\_\_\_\_\_\_\_\_\_\_\_\_\_\_\_\_\_\_\_\_\_\_\_\_\_\_\tabularnewline
\_\_\_\_\_\_\_\_\_\_\_\_\_\_\_\_\_\_\_\_\_\_\_\_\_\_\_\_\_\_\_\_\_\_\_\_\_\_\_\_\_\_\_\_\_\_\_\_\_\_\_\_\_\_\_\_\_\_\_\_\_\tabularnewline
\_\_\_\_\_\_\_\_\_\_\_\_\_\_\_\_\_\_\_\_\_\_\_\_\_\_\_\_\_\_\_\_\_\_\_\_\_\_\_\_\_\_\_\_\_\_\_\_\_\_\_\_\_\_\_\_\_\_\_\_\_\tabularnewline
\_\_\_\_\_\_\_\_\_\_\_\_\_\_\_\_\_\_\_\_\_\_\_\_\_\_\_\_\_\_\_\_\_\_\_\_\_\_\_\_\_\_\_\_\_\_\_\_\_\_\_\_\_\_\_\_\_\_\_\_\_\tabularnewline
\bottomrule
\end{longtable}

\begin{longtable}[]{@{}ll@{}}
\toprule
Initialed: &\tabularnewline
\midrule
\endhead
\_\_\_\_\_\_\_\_\_\_\_\_\_\_ &
\_\_\_\_\_\_\_\_\_\_\_\_\_\_\tabularnewline
Player & Team\tabularnewline
\bottomrule
\end{longtable}

\newpage

\subsection{Exhibit 5 --- Other
Activities}\label{exhibit-5-other-activities}

Player: \_\_\_\_\_\_\_\_\_\_\_\_\_\_\_\_\_\_\_\_\_\_\_\_\_\_\_\_\\
Team: \_\_\_\_\_\_\_\_\_\_\_\_\_\_\_\_\_\_\_\_\_\_\_\_\_\_\_\_\\
Date: \_\_\_\_\_\_\_\_\_\_\_\_\_\_\_\_\_\_\_\_\_\_\_\_\_\_\_\_

Notwithstanding the provisions of paragraph 12 of this Contract, the
Player and the Team agree that the Player need not obtain the consent of
the Team in order to engage in the activities set forth below:

\begin{longtable}[]{@{}l@{}}
\toprule
\_\_\_\_\_\_\_\_\_\_\_\_\_\_\_\_\_\_\_\_\_\_\_\_\_\_\_\_\_\_\_\_\_\_\_\_\_\_\_\_\_\_\_\_\_\_\_\_\_\_\_\_\_\_\_\_\_\_\_\_\_\tabularnewline
\_\_\_\_\_\_\_\_\_\_\_\_\_\_\_\_\_\_\_\_\_\_\_\_\_\_\_\_\_\_\_\_\_\_\_\_\_\_\_\_\_\_\_\_\_\_\_\_\_\_\_\_\_\_\_\_\_\_\_\_\_\tabularnewline
\_\_\_\_\_\_\_\_\_\_\_\_\_\_\_\_\_\_\_\_\_\_\_\_\_\_\_\_\_\_\_\_\_\_\_\_\_\_\_\_\_\_\_\_\_\_\_\_\_\_\_\_\_\_\_\_\_\_\_\_\_\tabularnewline
\_\_\_\_\_\_\_\_\_\_\_\_\_\_\_\_\_\_\_\_\_\_\_\_\_\_\_\_\_\_\_\_\_\_\_\_\_\_\_\_\_\_\_\_\_\_\_\_\_\_\_\_\_\_\_\_\_\_\_\_\_\tabularnewline
\_\_\_\_\_\_\_\_\_\_\_\_\_\_\_\_\_\_\_\_\_\_\_\_\_\_\_\_\_\_\_\_\_\_\_\_\_\_\_\_\_\_\_\_\_\_\_\_\_\_\_\_\_\_\_\_\_\_\_\_\_\tabularnewline
\_\_\_\_\_\_\_\_\_\_\_\_\_\_\_\_\_\_\_\_\_\_\_\_\_\_\_\_\_\_\_\_\_\_\_\_\_\_\_\_\_\_\_\_\_\_\_\_\_\_\_\_\_\_\_\_\_\_\_\_\_\tabularnewline
\_\_\_\_\_\_\_\_\_\_\_\_\_\_\_\_\_\_\_\_\_\_\_\_\_\_\_\_\_\_\_\_\_\_\_\_\_\_\_\_\_\_\_\_\_\_\_\_\_\_\_\_\_\_\_\_\_\_\_\_\_\tabularnewline
\_\_\_\_\_\_\_\_\_\_\_\_\_\_\_\_\_\_\_\_\_\_\_\_\_\_\_\_\_\_\_\_\_\_\_\_\_\_\_\_\_\_\_\_\_\_\_\_\_\_\_\_\_\_\_\_\_\_\_\_\_\tabularnewline
\_\_\_\_\_\_\_\_\_\_\_\_\_\_\_\_\_\_\_\_\_\_\_\_\_\_\_\_\_\_\_\_\_\_\_\_\_\_\_\_\_\_\_\_\_\_\_\_\_\_\_\_\_\_\_\_\_\_\_\_\_\tabularnewline
\_\_\_\_\_\_\_\_\_\_\_\_\_\_\_\_\_\_\_\_\_\_\_\_\_\_\_\_\_\_\_\_\_\_\_\_\_\_\_\_\_\_\_\_\_\_\_\_\_\_\_\_\_\_\_\_\_\_\_\_\_\tabularnewline
\_\_\_\_\_\_\_\_\_\_\_\_\_\_\_\_\_\_\_\_\_\_\_\_\_\_\_\_\_\_\_\_\_\_\_\_\_\_\_\_\_\_\_\_\_\_\_\_\_\_\_\_\_\_\_\_\_\_\_\_\_\tabularnewline
\_\_\_\_\_\_\_\_\_\_\_\_\_\_\_\_\_\_\_\_\_\_\_\_\_\_\_\_\_\_\_\_\_\_\_\_\_\_\_\_\_\_\_\_\_\_\_\_\_\_\_\_\_\_\_\_\_\_\_\_\_\tabularnewline
\_\_\_\_\_\_\_\_\_\_\_\_\_\_\_\_\_\_\_\_\_\_\_\_\_\_\_\_\_\_\_\_\_\_\_\_\_\_\_\_\_\_\_\_\_\_\_\_\_\_\_\_\_\_\_\_\_\_\_\_\_\tabularnewline
\_\_\_\_\_\_\_\_\_\_\_\_\_\_\_\_\_\_\_\_\_\_\_\_\_\_\_\_\_\_\_\_\_\_\_\_\_\_\_\_\_\_\_\_\_\_\_\_\_\_\_\_\_\_\_\_\_\_\_\_\_\tabularnewline
\_\_\_\_\_\_\_\_\_\_\_\_\_\_\_\_\_\_\_\_\_\_\_\_\_\_\_\_\_\_\_\_\_\_\_\_\_\_\_\_\_\_\_\_\_\_\_\_\_\_\_\_\_\_\_\_\_\_\_\_\_\tabularnewline
\bottomrule
\end{longtable}

\begin{longtable}[]{@{}ll@{}}
\toprule
Initialed: &\tabularnewline
\midrule
\endhead
\_\_\_\_\_\_\_\_\_\_\_\_\_\_ &
\_\_\_\_\_\_\_\_\_\_\_\_\_\_\tabularnewline
Player & Team\tabularnewline
\bottomrule
\end{longtable}

\newpage

\subsection{Exhibit 6 --- Physical Exam}\label{exhibit-6-physical-exam}

Player: \_\_\_\_\_\_\_\_\_\_\_\_\_\_\_\_\_\_\_\_\_\_\_\_\_\_\_\_\\
Team: \_\_\_\_\_\_\_\_\_\_\_\_\_\_\_\_\_\_\_\_\_\_\_\_\_\_\_\_\\
Date: \_\_\_\_\_\_\_\_\_\_\_\_\_\_\_\_\_\_\_\_\_\_\_\_\_\_\_\_

The Player and the Team agree that this Contract will be invalid and of
no force and effect unless the Player passes, in the sole discretion of
the Team, exercised in good faith, in consultation with one or more of
the Team's physicians, a physical examination in accordance with Article
II, Section 13(h) of the CBA that is (i) conducted within three (3)
business days of the execution of this Contract, and (ii) the results of
which are reported by the Team to the Player within six (6) business
days of the execution of this Contract. The Player agrees to supply
complete and truthful information in connection with any such
examinations.

\begin{longtable}[]{@{}ll@{}}
\toprule
Initialed: &\tabularnewline
\midrule
\endhead
\_\_\_\_\_\_\_\_\_\_\_\_\_\_ &
\_\_\_\_\_\_\_\_\_\_\_\_\_\_\tabularnewline
Player & Team\tabularnewline
\bottomrule
\end{longtable}

\newpage

\subsection{Exhibit 7 --- Substitution for UPC Paragraph
7(b)}\label{exhibit-7-substitution-for-upc-paragraph-7b}

Player: \_\_\_\_\_\_\_\_\_\_\_\_\_\_\_\_\_\_\_\_\_\_\_\_\_\_\_\_\\
Team: \_\_\_\_\_\_\_\_\_\_\_\_\_\_\_\_\_\_\_\_\_\_\_\_\_\_\_\_\\
Date: \_\_\_\_\_\_\_\_\_\_\_\_\_\_\_\_\_\_\_\_\_\_\_\_\_\_\_\_

Paragraph 7(b) is hereby deleted and the following shall be substituted
in place and instead thereof:

\begin{quote}
``7. (b) The Player agrees, notwithstanding any other provision of this
Contract, that he will to the best of his ability maintain himself in
physical condition sufficient to play skilled basketball at all times.
If the Player, in the reasonable judgment of the physician designated
for that purpose by the Team, is not in good physical condition at the
date of his first scheduled game for the Team, or if, at the beginning
of or during any Season, he fails to remain in good physical condition,
in either event so as to render the Player unfit in the reasonable
judgment of said physician to play skilled basketball, the Team shall
have the right to suspend the Player for successive one-week periods
until the Player, in the reasonable judgment of the Team's physician, is
in good physical condition; provided, however, that at the end of each
such one-week period of suspension, if the Team notifies the Player,
orally or in writing, that in its reasonable judgment it believes the
Player is still not in good physical condition, and if the Player so
requests, then the Player shall be examined by a physician or physicians
designated for such purpose by the President, or any Vice President if
the President is not available, of the American Society of Orthopedic
Physicians, or equivalent organization (the''Reviewing Physician``),
whose sole judgment concerning the physical condition of the Player to
play skilled basketball shall be binding upon the Team and the Player
for purposes of this paragraph. The suspension of the Player shall be
terminated promptly upon the failure of the Team to give the Player the
notice required at the end of the one-week period or upon the finding of
said Reviewing Physician that the Player is in physical condition
sufficient to play skilled basketball. In the event of a suspension
permitted hereunder, the Compensation (excluding any signing bonus or
Incentive Compensation) payable to the Player for any Season during such
suspension shall be reduced in the same proportion as the length of the
period of disability so determined bears to the length of the Season.
Nothing in this paragraph 7(b) shall authorize the Team to suspend the
Player solely because the Player is injured or ill.''
\end{quote}

\begin{longtable}[]{@{}ll@{}}
\toprule
Initialed: &\tabularnewline
\midrule
\endhead
\_\_\_\_\_\_\_\_\_\_\_\_\_\_ &
\_\_\_\_\_\_\_\_\_\_\_\_\_\_\tabularnewline
Player & Team\tabularnewline
\bottomrule
\end{longtable}

\newpage

\subsection{Exhibit 8 --- Sign and
Trade}\label{exhibit-8-sign-and-trade}

Player: \_\_\_\_\_\_\_\_\_\_\_\_\_\_\_\_\_\_\_\_\_\_\_\_\_\_\_\_\\
Team: \_\_\_\_\_\_\_\_\_\_\_\_\_\_\_\_\_\_\_\_\_\_\_\_\_\_\_\_\\
Date: \_\_\_\_\_\_\_\_\_\_\_\_\_\_\_\_\_\_\_\_\_\_\_\_\_\_\_\_

The Player and the Team agree that this {[}Contract{]} {[}amendment{]}
will be invalid and of no force and effect unless the {[}Contract{]}
{[}amendment{]} is traded to the {[}assignee Team{]} within forty-eight
(48) hours of its execution, and all conditions to such trade are
ultimately satisfied.

\begin{longtable}[]{@{}ll@{}}
\toprule
Initialed: &\tabularnewline
\midrule
\endhead
\_\_\_\_\_\_\_\_\_\_\_\_\_\_ &
\_\_\_\_\_\_\_\_\_\_\_\_\_\_\tabularnewline
Player & Team\tabularnewline
\bottomrule
\end{longtable}

\newpage

\subsection{Exhibit 9 --- One-Season, Non-Guaranteed Training Camp
Contracts}\label{exhibit-9-one-season-non-guaranteed-training-camp-contracts}

Player: \_\_\_\_\_\_\_\_\_\_\_\_\_\_\_\_\_\_\_\_\_\_\_\_\_\_\_\_\\
Team: \_\_\_\_\_\_\_\_\_\_\_\_\_\_\_\_\_\_\_\_\_\_\_\_\_\_\_\_\\
Date: \_\_\_\_\_\_\_\_\_\_\_\_\_\_\_\_\_\_\_\_\_\_\_\_\_\_\_\_

The Player's right to receive any Compensation under this Contract
(other than Compensation in accordance with paragraph 3(b)) is
eliminated in the event the Contract is terminated prior to the first
day of the Regular Season covered by the Contract; provided, however,
that if the Player is injured as a direct result of playing for the Team
and, accordingly, would have been entitled (but for this Exhibit 9) to
Compensation pursuant to paragraphs 7(c), 16(a)(iii), 16(b), or
otherwise, the Team's sole liability shall be to pay the Player \$6,000
upon termination of the Player's Contract.

\begin{longtable}[]{@{}ll@{}}
\toprule
Initialed: &\tabularnewline
\midrule
\endhead
\_\_\_\_\_\_\_\_\_\_\_\_\_\_ &
\_\_\_\_\_\_\_\_\_\_\_\_\_\_\tabularnewline
Player & Team\tabularnewline
\bottomrule
\end{longtable}

\newpage

\subsection{Exhibit 10 --- NBADL Bonus and Two-Way Player
Conversion}\label{exhibit-10-nbadl-bonus-and-two-way-player-conversion}

Player: \_\_\_\_\_\_\_\_\_\_\_\_\_\_\_\_\_\_\_\_\_\_\_\_\_\_\_\\
Team: \_\_\_\_\_\_\_\_\_\_\_\_\_\_\_\_\_\_\_\_\_\_\_\_\_\_\_\_\\
Bonus Amount*: \_\_\_\_\_\_\_\_\_\_\_\_\_\_\_\_\_\_\_\_\_\_\\
NBADL Affiliate: \_\_\_\_\_\_\_\_\_\_\_\_\_\_\_\_\_\_\_\_\_\\
Conversion Protection Amount: \_\_\_\_\_\_\_\_\_\_\_\_\\
Date: \_\_\_\_\_\_\_\_\_\_\_\_\_\_\_\_\_\_\_\_\_\_\_\_\_\_\_\_

\textbf{Contract Termination/NBADL:} In the event this Contract is
terminated by the Team in accordance with the NBA waiver procedure, the
Player shall be entitled to receive from the Team the Bonus Amount (if
applicable) provided above, provided that the Player (a) signs with the
NBADL prior to the deadline set by the NBADL for NBADL teams to
designate affiliate players, (b) is initially assigned by the NBADL to
the NBADL affiliate listed above (or the NBADL affiliate of any Team
that acquires the Contract, if applicable) and timely reports to such
affiliate, (c) does not leave the NBADL (e.g., by buying out his
contract with the NBADL and signing a contract with an international
team) for a period of sixty (60) days after signing with the NBADL
(``60-Day Bonus Window''), with such bonus payable (if applicable)
within thirty (30) days after the 60-Day Bonus Window.

\textbf{Two-Way Player Conversion Option:} Team shall have the option to
convert this Contract to a Two-Way Contract (``Two-Way Player Conversion
Option''); provided, however, that (a) such option must be exercised
prior to the first day of the NBA Regular Season, and (b) may not be
exercised if it would result in a violation of Article X, Section 4(d)
of the CBA. Team's Two-Way Player Conversion Option may be exercised by
providing written notice to Player that is either personally delivered
to Player or his representative or sent by email or pre-paid certified,
registered, or overnight mail to the last known address of Player or his
representative with a copy to the Players Association and the NBA. If
Team exercises the Two-Way Player Conversion Option, this Contract's
Exhibit 1A will immediately become null and void and of no further force
or effect and the Player's Compensation shall be equal to the Two-Way
Player Salary applicable for such Season. Further, upon conversion, the
Player's right to the Bonus Amount (if applicable) set forth above
pursuant to this Exhibit 10 will be rescinded and the Player's Contract,
notwithstanding the absence of an Exhibit 2, shall be protected for lack
of skill and injury or illness at an amount equal to the Conversion
Protection Amount in this Exhibit 10. All other terms and conditions of
this Contract shall remain applicable.

\textbf{Standard NBA Contract Conversion Option:} In the event the
Two-Way Player Conversion Option is exercised by the Team, Team shall
thereafter have the option to convert the Contract to a Standard NBA
Contract (``Standard NBA Contract Conversion Option''). Team's Standard
NBA Contract Conversion Option may be exercised by providing written
notice to Player that is either personally delivered to Player or his
representative or sent by email or pre-paid certified, registered, or
overnight mail to the last known address of Player or his representative
with a copy to the Players Association and the NBA. If Team exercises
the Standard NBA Contract Conversion Option, the Base Compensation
amount applicable to the Two-Way Contract as set forth in this Exhibit
10 will immediately become null and void and of no further force or
effect, Player's Compensation shall be equal to the Player's applicable
Minimum Player Salary for such Season beginning on the date such option
is exercised, and all other terms and conditions of this Contract,
including the Base Compensation protection set forth in this Exhibit 10,
shall remain applicable.

*Bonus Amount must be equal to the Conversion Protection Amount and may
only be included if Team has an NBADL Affiliate.

\begin{longtable}[]{@{}ll@{}}
\toprule
Initialed: &\tabularnewline
\midrule
\endhead
\_\_\_\_\_\_\_\_\_\_\_\_\_\_ &
\_\_\_\_\_\_\_\_\_\_\_\_\_\_\tabularnewline
Player & Team\tabularnewline
\bottomrule
\end{longtable}

\chapter{}\label{section}

\section{2017-18 NBA ROOKIE SCALE}\label{nba-rookie-scale}

\begin{longtable}[]{@{}clllcc@{}}
\toprule
\begin{minipage}[b]{0.07\columnwidth}\centering\strut
Pick\strut
\end{minipage} & \begin{minipage}[b]{0.12\columnwidth}\raggedright\strut
1st Year Salary\strut
\end{minipage} & \begin{minipage}[b]{0.13\columnwidth}\raggedright\strut
2nd Year Salary\strut
\end{minipage} & \begin{minipage}[b]{0.13\columnwidth}\raggedright\strut
3rd Year Option Salary\strut
\end{minipage} & \begin{minipage}[b]{0.17\columnwidth}\centering\strut
4th Year Option: Percentage Increase Over 3rd Year Salary\strut
\end{minipage} & \begin{minipage}[b]{0.17\columnwidth}\centering\strut
Qualifying Offer: Percentage Increase Over 4th Year Salary\strut
\end{minipage}\tabularnewline
\midrule
\endhead
\begin{minipage}[t]{0.07\columnwidth}\centering\strut
1\strut
\end{minipage} & \begin{minipage}[t]{0.12\columnwidth}\raggedright\strut
\$5,855,200\strut
\end{minipage} & \begin{minipage}[t]{0.13\columnwidth}\raggedright\strut
\$6,949,900\strut
\end{minipage} & \begin{minipage}[t]{0.13\columnwidth}\raggedright\strut
\$8,121,000\strut
\end{minipage} & \begin{minipage}[t]{0.17\columnwidth}\centering\strut
26.1\%\strut
\end{minipage} & \begin{minipage}[t]{0.17\columnwidth}\centering\strut
30.0\%\strut
\end{minipage}\tabularnewline
\begin{minipage}[t]{0.07\columnwidth}\centering\strut
2\strut
\end{minipage} & \begin{minipage}[t]{0.12\columnwidth}\raggedright\strut
\$5,238,800\strut
\end{minipage} & \begin{minipage}[t]{0.13\columnwidth}\raggedright\strut
\$6,218,300\strut
\end{minipage} & \begin{minipage}[t]{0.13\columnwidth}\raggedright\strut
\$7,266,100\strut
\end{minipage} & \begin{minipage}[t]{0.17\columnwidth}\centering\strut
26.2\%\strut
\end{minipage} & \begin{minipage}[t]{0.17\columnwidth}\centering\strut
30.5\%\strut
\end{minipage}\tabularnewline
\begin{minipage}[t]{0.07\columnwidth}\centering\strut
3\strut
\end{minipage} & \begin{minipage}[t]{0.12\columnwidth}\raggedright\strut
\$4,704,500\strut
\end{minipage} & \begin{minipage}[t]{0.13\columnwidth}\raggedright\strut
\$5,584,000\strut
\end{minipage} & \begin{minipage}[t]{0.13\columnwidth}\raggedright\strut
\$6,525,000\strut
\end{minipage} & \begin{minipage}[t]{0.17\columnwidth}\centering\strut
26.4\%\strut
\end{minipage} & \begin{minipage}[t]{0.17\columnwidth}\centering\strut
31.2\%\strut
\end{minipage}\tabularnewline
\begin{minipage}[t]{0.07\columnwidth}\centering\strut
4\strut
\end{minipage} & \begin{minipage}[t]{0.12\columnwidth}\raggedright\strut
\$4,241,700\strut
\end{minipage} & \begin{minipage}[t]{0.13\columnwidth}\raggedright\strut
\$5,034,600\strut
\end{minipage} & \begin{minipage}[t]{0.13\columnwidth}\raggedright\strut
\$5,882,900\strut
\end{minipage} & \begin{minipage}[t]{0.17\columnwidth}\centering\strut
26.5\%\strut
\end{minipage} & \begin{minipage}[t]{0.17\columnwidth}\centering\strut
31.9\%\strut
\end{minipage}\tabularnewline
\begin{minipage}[t]{0.07\columnwidth}\centering\strut
5\strut
\end{minipage} & \begin{minipage}[t]{0.12\columnwidth}\raggedright\strut
\$3,841,000\strut
\end{minipage} & \begin{minipage}[t]{0.13\columnwidth}\raggedright\strut
\$4,559,100\strut
\end{minipage} & \begin{minipage}[t]{0.13\columnwidth}\raggedright\strut
\$5,327,300\strut
\end{minipage} & \begin{minipage}[t]{0.17\columnwidth}\centering\strut
26.7\%\strut
\end{minipage} & \begin{minipage}[t]{0.17\columnwidth}\centering\strut
32.6\%\strut
\end{minipage}\tabularnewline
\begin{minipage}[t]{0.07\columnwidth}\centering\strut
6\strut
\end{minipage} & \begin{minipage}[t]{0.12\columnwidth}\raggedright\strut
\$3,488,600\strut
\end{minipage} & \begin{minipage}[t]{0.13\columnwidth}\raggedright\strut
\$4,140,900\strut
\end{minipage} & \begin{minipage}[t]{0.13\columnwidth}\raggedright\strut
\$4,838,700\strut
\end{minipage} & \begin{minipage}[t]{0.17\columnwidth}\centering\strut
26.8\%\strut
\end{minipage} & \begin{minipage}[t]{0.17\columnwidth}\centering\strut
33.4\%\strut
\end{minipage}\tabularnewline
\begin{minipage}[t]{0.07\columnwidth}\centering\strut
7\strut
\end{minipage} & \begin{minipage}[t]{0.12\columnwidth}\raggedright\strut
\$3,184,700\strut
\end{minipage} & \begin{minipage}[t]{0.13\columnwidth}\raggedright\strut
\$3,780,100\strut
\end{minipage} & \begin{minipage}[t]{0.13\columnwidth}\raggedright\strut
\$4,417,000\strut
\end{minipage} & \begin{minipage}[t]{0.17\columnwidth}\centering\strut
27.0\%\strut
\end{minipage} & \begin{minipage}[t]{0.17\columnwidth}\centering\strut
34.1\%\strut
\end{minipage}\tabularnewline
\begin{minipage}[t]{0.07\columnwidth}\centering\strut
8\strut
\end{minipage} & \begin{minipage}[t]{0.12\columnwidth}\raggedright\strut
\$2,917,600\strut
\end{minipage} & \begin{minipage}[t]{0.13\columnwidth}\raggedright\strut
\$3,463,100\strut
\end{minipage} & \begin{minipage}[t]{0.13\columnwidth}\raggedright\strut
\$4,046,500\strut
\end{minipage} & \begin{minipage}[t]{0.17\columnwidth}\centering\strut
27.2\%\strut
\end{minipage} & \begin{minipage}[t]{0.17\columnwidth}\centering\strut
34.8\%\strut
\end{minipage}\tabularnewline
\begin{minipage}[t]{0.07\columnwidth}\centering\strut
9\strut
\end{minipage} & \begin{minipage}[t]{0.12\columnwidth}\raggedright\strut
\$2,681,900\strut
\end{minipage} & \begin{minipage}[t]{0.13\columnwidth}\raggedright\strut
\$3,183,300\strut
\end{minipage} & \begin{minipage}[t]{0.13\columnwidth}\raggedright\strut
\$3,719,700\strut
\end{minipage} & \begin{minipage}[t]{0.17\columnwidth}\centering\strut
27.4\%\strut
\end{minipage} & \begin{minipage}[t]{0.17\columnwidth}\centering\strut
35.5\%\strut
\end{minipage}\tabularnewline
\begin{minipage}[t]{0.07\columnwidth}\centering\strut
10\strut
\end{minipage} & \begin{minipage}[t]{0.12\columnwidth}\raggedright\strut
\$2,547,700\strut
\end{minipage} & \begin{minipage}[t]{0.13\columnwidth}\raggedright\strut
\$3,024,100\strut
\end{minipage} & \begin{minipage}[t]{0.13\columnwidth}\raggedright\strut
\$3,533,500\strut
\end{minipage} & \begin{minipage}[t]{0.17\columnwidth}\centering\strut
27.5\%\strut
\end{minipage} & \begin{minipage}[t]{0.17\columnwidth}\centering\strut
36.2\%\strut
\end{minipage}\tabularnewline
\begin{minipage}[t]{0.07\columnwidth}\centering\strut
11\strut
\end{minipage} & \begin{minipage}[t]{0.12\columnwidth}\raggedright\strut
\$2,420,400\strut
\end{minipage} & \begin{minipage}[t]{0.13\columnwidth}\raggedright\strut
\$2,872,900\strut
\end{minipage} & \begin{minipage}[t]{0.13\columnwidth}\raggedright\strut
\$3,357,000\strut
\end{minipage} & \begin{minipage}[t]{0.17\columnwidth}\centering\strut
32.7\%\strut
\end{minipage} & \begin{minipage}[t]{0.17\columnwidth}\centering\strut
36.9\%\strut
\end{minipage}\tabularnewline
\begin{minipage}[t]{0.07\columnwidth}\centering\strut
12\strut
\end{minipage} & \begin{minipage}[t]{0.12\columnwidth}\raggedright\strut
\$2,299,400\strut
\end{minipage} & \begin{minipage}[t]{0.13\columnwidth}\raggedright\strut
\$2,729,400\strut
\end{minipage} & \begin{minipage}[t]{0.13\columnwidth}\raggedright\strut
\$3,189,300\strut
\end{minipage} & \begin{minipage}[t]{0.17\columnwidth}\centering\strut
37.8\%\strut
\end{minipage} & \begin{minipage}[t]{0.17\columnwidth}\centering\strut
37.6\%\strut
\end{minipage}\tabularnewline
\begin{minipage}[t]{0.07\columnwidth}\centering\strut
13\strut
\end{minipage} & \begin{minipage}[t]{0.12\columnwidth}\raggedright\strut
\$2,184,400\strut
\end{minipage} & \begin{minipage}[t]{0.13\columnwidth}\raggedright\strut
\$2,592,900\strut
\end{minipage} & \begin{minipage}[t]{0.13\columnwidth}\raggedright\strut
\$3,029,800\strut
\end{minipage} & \begin{minipage}[t]{0.17\columnwidth}\centering\strut
42.9\%\strut
\end{minipage} & \begin{minipage}[t]{0.17\columnwidth}\centering\strut
38.3\%\strut
\end{minipage}\tabularnewline
\begin{minipage}[t]{0.07\columnwidth}\centering\strut
14\strut
\end{minipage} & \begin{minipage}[t]{0.12\columnwidth}\raggedright\strut
\$2,075,300\strut
\end{minipage} & \begin{minipage}[t]{0.13\columnwidth}\raggedright\strut
\$2,463,200\strut
\end{minipage} & \begin{minipage}[t]{0.13\columnwidth}\raggedright\strut
\$2,878,400\strut
\end{minipage} & \begin{minipage}[t]{0.17\columnwidth}\centering\strut
48.1\%\strut
\end{minipage} & \begin{minipage}[t]{0.17\columnwidth}\centering\strut
39.1\%\strut
\end{minipage}\tabularnewline
\begin{minipage}[t]{0.07\columnwidth}\centering\strut
15\strut
\end{minipage} & \begin{minipage}[t]{0.12\columnwidth}\raggedright\strut
\$1,971,300\strut
\end{minipage} & \begin{minipage}[t]{0.13\columnwidth}\raggedright\strut
\$2,339,900\strut
\end{minipage} & \begin{minipage}[t]{0.13\columnwidth}\raggedright\strut
\$2,734,100\strut
\end{minipage} & \begin{minipage}[t]{0.17\columnwidth}\centering\strut
53.3\%\strut
\end{minipage} & \begin{minipage}[t]{0.17\columnwidth}\centering\strut
39.8\%\strut
\end{minipage}\tabularnewline
\begin{minipage}[t]{0.07\columnwidth}\centering\strut
16\strut
\end{minipage} & \begin{minipage}[t]{0.12\columnwidth}\raggedright\strut
\$1,872,900\strut
\end{minipage} & \begin{minipage}[t]{0.13\columnwidth}\raggedright\strut
\$2,223,000\strut
\end{minipage} & \begin{minipage}[t]{0.13\columnwidth}\raggedright\strut
\$2,597,700\strut
\end{minipage} & \begin{minipage}[t]{0.17\columnwidth}\centering\strut
53.4\%\strut
\end{minipage} & \begin{minipage}[t]{0.17\columnwidth}\centering\strut
40.5\%\strut
\end{minipage}\tabularnewline
\begin{minipage}[t]{0.07\columnwidth}\centering\strut
17\strut
\end{minipage} & \begin{minipage}[t]{0.12\columnwidth}\raggedright\strut
\$1,779,200\strut
\end{minipage} & \begin{minipage}[t]{0.13\columnwidth}\raggedright\strut
\$2,111,900\strut
\end{minipage} & \begin{minipage}[t]{0.13\columnwidth}\raggedright\strut
\$2,467,600\strut
\end{minipage} & \begin{minipage}[t]{0.17\columnwidth}\centering\strut
53.6\%\strut
\end{minipage} & \begin{minipage}[t]{0.17\columnwidth}\centering\strut
41.2\%\strut
\end{minipage}\tabularnewline
\begin{minipage}[t]{0.07\columnwidth}\centering\strut
18\strut
\end{minipage} & \begin{minipage}[t]{0.12\columnwidth}\raggedright\strut
\$1,690,300\strut
\end{minipage} & \begin{minipage}[t]{0.13\columnwidth}\raggedright\strut
\$2,006,300\strut
\end{minipage} & \begin{minipage}[t]{0.13\columnwidth}\raggedright\strut
\$2,344,400\strut
\end{minipage} & \begin{minipage}[t]{0.17\columnwidth}\centering\strut
53.8\%\strut
\end{minipage} & \begin{minipage}[t]{0.17\columnwidth}\centering\strut
41.9\%\strut
\end{minipage}\tabularnewline
\begin{minipage}[t]{0.07\columnwidth}\centering\strut
19\strut
\end{minipage} & \begin{minipage}[t]{0.12\columnwidth}\raggedright\strut
\$1,614,100\strut
\end{minipage} & \begin{minipage}[t]{0.13\columnwidth}\raggedright\strut
\$1,915,900\strut
\end{minipage} & \begin{minipage}[t]{0.13\columnwidth}\raggedright\strut
\$2,238,800\strut
\end{minipage} & \begin{minipage}[t]{0.17\columnwidth}\centering\strut
54.0\%\strut
\end{minipage} & \begin{minipage}[t]{0.17\columnwidth}\centering\strut
42.6\%\strut
\end{minipage}\tabularnewline
\begin{minipage}[t]{0.07\columnwidth}\centering\strut
20\strut
\end{minipage} & \begin{minipage}[t]{0.12\columnwidth}\raggedright\strut
\$1,549,500\strut
\end{minipage} & \begin{minipage}[t]{0.13\columnwidth}\raggedright\strut
\$1,839,200\strut
\end{minipage} & \begin{minipage}[t]{0.13\columnwidth}\raggedright\strut
\$2,149,000\strut
\end{minipage} & \begin{minipage}[t]{0.17\columnwidth}\centering\strut
54.2\%\strut
\end{minipage} & \begin{minipage}[t]{0.17\columnwidth}\centering\strut
43.3\%\strut
\end{minipage}\tabularnewline
\begin{minipage}[t]{0.07\columnwidth}\centering\strut
21\strut
\end{minipage} & \begin{minipage}[t]{0.12\columnwidth}\raggedright\strut
\$1,487,500\strut
\end{minipage} & \begin{minipage}[t]{0.13\columnwidth}\raggedright\strut
\$1,765,700\strut
\end{minipage} & \begin{minipage}[t]{0.13\columnwidth}\raggedright\strut
\$2,063,200\strut
\end{minipage} & \begin{minipage}[t]{0.17\columnwidth}\centering\strut
59.3\%\strut
\end{minipage} & \begin{minipage}[t]{0.17\columnwidth}\centering\strut
44.1\%\strut
\end{minipage}\tabularnewline
\begin{minipage}[t]{0.07\columnwidth}\centering\strut
22\strut
\end{minipage} & \begin{minipage}[t]{0.12\columnwidth}\raggedright\strut
\$1,428,100\strut
\end{minipage} & \begin{minipage}[t]{0.13\columnwidth}\raggedright\strut
\$1,695,100\strut
\end{minipage} & \begin{minipage}[t]{0.13\columnwidth}\raggedright\strut
\$1,980,700\strut
\end{minipage} & \begin{minipage}[t]{0.17\columnwidth}\centering\strut
64.5\%\strut
\end{minipage} & \begin{minipage}[t]{0.17\columnwidth}\centering\strut
44.8\%\strut
\end{minipage}\tabularnewline
\begin{minipage}[t]{0.07\columnwidth}\centering\strut
23\strut
\end{minipage} & \begin{minipage}[t]{0.12\columnwidth}\raggedright\strut
\$1,371,000\strut
\end{minipage} & \begin{minipage}[t]{0.13\columnwidth}\raggedright\strut
\$1,627,300\strut
\end{minipage} & \begin{minipage}[t]{0.13\columnwidth}\raggedright\strut
\$1,901,500\strut
\end{minipage} & \begin{minipage}[t]{0.17\columnwidth}\centering\strut
69.7\%\strut
\end{minipage} & \begin{minipage}[t]{0.17\columnwidth}\centering\strut
45.5\%\strut
\end{minipage}\tabularnewline
\begin{minipage}[t]{0.07\columnwidth}\centering\strut
24\strut
\end{minipage} & \begin{minipage}[t]{0.12\columnwidth}\raggedright\strut
\$1,316,200\strut
\end{minipage} & \begin{minipage}[t]{0.13\columnwidth}\raggedright\strut
\$1,562,200\strut
\end{minipage} & \begin{minipage}[t]{0.13\columnwidth}\raggedright\strut
\$1,825,600\strut
\end{minipage} & \begin{minipage}[t]{0.17\columnwidth}\centering\strut
74.9\%\strut
\end{minipage} & \begin{minipage}[t]{0.17\columnwidth}\centering\strut
46.2\%\strut
\end{minipage}\tabularnewline
\begin{minipage}[t]{0.07\columnwidth}\centering\strut
25\strut
\end{minipage} & \begin{minipage}[t]{0.12\columnwidth}\raggedright\strut
\$1,263,500\strut
\end{minipage} & \begin{minipage}[t]{0.13\columnwidth}\raggedright\strut
\$1,499,700\strut
\end{minipage} & \begin{minipage}[t]{0.13\columnwidth}\raggedright\strut
\$1,752,500\strut
\end{minipage} & \begin{minipage}[t]{0.17\columnwidth}\centering\strut
80.1\%\strut
\end{minipage} & \begin{minipage}[t]{0.17\columnwidth}\centering\strut
46.9\%\strut
\end{minipage}\tabularnewline
\begin{minipage}[t]{0.07\columnwidth}\centering\strut
26\strut
\end{minipage} & \begin{minipage}[t]{0.12\columnwidth}\raggedright\strut
\$1,221,600\strut
\end{minipage} & \begin{minipage}[t]{0.13\columnwidth}\raggedright\strut
\$1,450,000\strut
\end{minipage} & \begin{minipage}[t]{0.13\columnwidth}\raggedright\strut
\$1,694,300\strut
\end{minipage} & \begin{minipage}[t]{0.17\columnwidth}\centering\strut
80.3\%\strut
\end{minipage} & \begin{minipage}[t]{0.17\columnwidth}\centering\strut
47.6\%\strut
\end{minipage}\tabularnewline
\begin{minipage}[t]{0.07\columnwidth}\centering\strut
27\strut
\end{minipage} & \begin{minipage}[t]{0.12\columnwidth}\raggedright\strut
\$1,186,300\strut
\end{minipage} & \begin{minipage}[t]{0.13\columnwidth}\raggedright\strut
\$1,408,200\strut
\end{minipage} & \begin{minipage}[t]{0.13\columnwidth}\raggedright\strut
\$1,645,500\strut
\end{minipage} & \begin{minipage}[t]{0.17\columnwidth}\centering\strut
80.4\%\strut
\end{minipage} & \begin{minipage}[t]{0.17\columnwidth}\centering\strut
48.3\%\strut
\end{minipage}\tabularnewline
\begin{minipage}[t]{0.07\columnwidth}\centering\strut
28\strut
\end{minipage} & \begin{minipage}[t]{0.12\columnwidth}\raggedright\strut
\$1,179,100\strut
\end{minipage} & \begin{minipage}[t]{0.13\columnwidth}\raggedright\strut
\$1,399,600\strut
\end{minipage} & \begin{minipage}[t]{0.13\columnwidth}\raggedright\strut
\$1,635,300\strut
\end{minipage} & \begin{minipage}[t]{0.17\columnwidth}\centering\strut
80.5\%\strut
\end{minipage} & \begin{minipage}[t]{0.17\columnwidth}\centering\strut
49.0\%\strut
\end{minipage}\tabularnewline
\begin{minipage}[t]{0.07\columnwidth}\centering\strut
29\strut
\end{minipage} & \begin{minipage}[t]{0.12\columnwidth}\raggedright\strut
\$1,170,500\strut
\end{minipage} & \begin{minipage}[t]{0.13\columnwidth}\raggedright\strut
\$1,389,300\strut
\end{minipage} & \begin{minipage}[t]{0.13\columnwidth}\raggedright\strut
\$1,623,400\strut
\end{minipage} & \begin{minipage}[t]{0.17\columnwidth}\centering\strut
80.5\%\strut
\end{minipage} & \begin{minipage}[t]{0.17\columnwidth}\centering\strut
50.0\%\strut
\end{minipage}\tabularnewline
\begin{minipage}[t]{0.07\columnwidth}\centering\strut
30\strut
\end{minipage} & \begin{minipage}[t]{0.12\columnwidth}\raggedright\strut
\$1,162,100\strut
\end{minipage} & \begin{minipage}[t]{0.13\columnwidth}\raggedright\strut
\$1,379,300\strut
\end{minipage} & \begin{minipage}[t]{0.13\columnwidth}\raggedright\strut
\$1,611,800\strut
\end{minipage} & \begin{minipage}[t]{0.17\columnwidth}\centering\strut
80.5\%\strut
\end{minipage} & \begin{minipage}[t]{0.17\columnwidth}\centering\strut
50.0\%\strut
\end{minipage}\tabularnewline
\bottomrule
\end{longtable}

\newpage

\section{2017-18 BASELINE ROOKIE SCALE}\label{baseline-rookie-scale}

\begin{longtable}[]{@{}clllcc@{}}
\toprule
\begin{minipage}[b]{0.07\columnwidth}\centering\strut
Pick\strut
\end{minipage} & \begin{minipage}[b]{0.12\columnwidth}\raggedright\strut
1st Year Salary\strut
\end{minipage} & \begin{minipage}[b]{0.13\columnwidth}\raggedright\strut
2nd Year Salary\strut
\end{minipage} & \begin{minipage}[b]{0.13\columnwidth}\raggedright\strut
3rd Year Option Salary\strut
\end{minipage} & \begin{minipage}[b]{0.17\columnwidth}\centering\strut
4th Year Option: Percentage Increase Over 3rd Year Salary\strut
\end{minipage} & \begin{minipage}[b]{0.17\columnwidth}\centering\strut
Qualifying Offer: Percentage Increase Over 4th Year Salary\strut
\end{minipage}\tabularnewline
\midrule
\endhead
\begin{minipage}[t]{0.07\columnwidth}\centering\strut
1\strut
\end{minipage} & \begin{minipage}[t]{0.12\columnwidth}\raggedright\strut
\$5,091,500\strut
\end{minipage} & \begin{minipage}[t]{0.13\columnwidth}\raggedright\strut
\$5,346,100\strut
\end{minipage} & \begin{minipage}[t]{0.13\columnwidth}\raggedright\strut
\$5,600,700\strut
\end{minipage} & \begin{minipage}[t]{0.17\columnwidth}\centering\strut
26.1\%\strut
\end{minipage} & \begin{minipage}[t]{0.17\columnwidth}\centering\strut
30.0\%\strut
\end{minipage}\tabularnewline
\begin{minipage}[t]{0.07\columnwidth}\centering\strut
2\strut
\end{minipage} & \begin{minipage}[t]{0.12\columnwidth}\raggedright\strut
\$4,555,500\strut
\end{minipage} & \begin{minipage}[t]{0.13\columnwidth}\raggedright\strut
\$4,783,300\strut
\end{minipage} & \begin{minipage}[t]{0.13\columnwidth}\raggedright\strut
\$5,011,100\strut
\end{minipage} & \begin{minipage}[t]{0.17\columnwidth}\centering\strut
26.2\%\strut
\end{minipage} & \begin{minipage}[t]{0.17\columnwidth}\centering\strut
30.5\%\strut
\end{minipage}\tabularnewline
\begin{minipage}[t]{0.07\columnwidth}\centering\strut
3\strut
\end{minipage} & \begin{minipage}[t]{0.12\columnwidth}\raggedright\strut
\$4,090,900\strut
\end{minipage} & \begin{minipage}[t]{0.13\columnwidth}\raggedright\strut
\$4,295,400\strut
\end{minipage} & \begin{minipage}[t]{0.13\columnwidth}\raggedright\strut
\$4,500,000\strut
\end{minipage} & \begin{minipage}[t]{0.17\columnwidth}\centering\strut
26.4\%\strut
\end{minipage} & \begin{minipage}[t]{0.17\columnwidth}\centering\strut
31.2\%\strut
\end{minipage}\tabularnewline
\begin{minipage}[t]{0.07\columnwidth}\centering\strut
4\strut
\end{minipage} & \begin{minipage}[t]{0.12\columnwidth}\raggedright\strut
\$3,688,400\strut
\end{minipage} & \begin{minipage}[t]{0.13\columnwidth}\raggedright\strut
\$3,872,800\strut
\end{minipage} & \begin{minipage}[t]{0.13\columnwidth}\raggedright\strut
\$4,057,200\strut
\end{minipage} & \begin{minipage}[t]{0.17\columnwidth}\centering\strut
26.5\%\strut
\end{minipage} & \begin{minipage}[t]{0.17\columnwidth}\centering\strut
31.9\%\strut
\end{minipage}\tabularnewline
\begin{minipage}[t]{0.07\columnwidth}\centering\strut
5\strut
\end{minipage} & \begin{minipage}[t]{0.12\columnwidth}\raggedright\strut
\$3,340,000\strut
\end{minipage} & \begin{minipage}[t]{0.13\columnwidth}\raggedright\strut
\$3,507,000\strut
\end{minipage} & \begin{minipage}[t]{0.13\columnwidth}\raggedright\strut
\$3,674,000\strut
\end{minipage} & \begin{minipage}[t]{0.17\columnwidth}\centering\strut
26.7\%\strut
\end{minipage} & \begin{minipage}[t]{0.17\columnwidth}\centering\strut
32.6\%\strut
\end{minipage}\tabularnewline
\begin{minipage}[t]{0.07\columnwidth}\centering\strut
6\strut
\end{minipage} & \begin{minipage}[t]{0.12\columnwidth}\raggedright\strut
\$3,033,600\strut
\end{minipage} & \begin{minipage}[t]{0.13\columnwidth}\raggedright\strut
\$3,185,300\strut
\end{minipage} & \begin{minipage}[t]{0.13\columnwidth}\raggedright\strut
\$3,337,000\strut
\end{minipage} & \begin{minipage}[t]{0.17\columnwidth}\centering\strut
26.8\%\strut
\end{minipage} & \begin{minipage}[t]{0.17\columnwidth}\centering\strut
33.4\%\strut
\end{minipage}\tabularnewline
\begin{minipage}[t]{0.07\columnwidth}\centering\strut
7\strut
\end{minipage} & \begin{minipage}[t]{0.12\columnwidth}\raggedright\strut
\$2,769,300\strut
\end{minipage} & \begin{minipage}[t]{0.13\columnwidth}\raggedright\strut
\$2,907,800\strut
\end{minipage} & \begin{minipage}[t]{0.13\columnwidth}\raggedright\strut
\$3,046,200\strut
\end{minipage} & \begin{minipage}[t]{0.17\columnwidth}\centering\strut
27.0\%\strut
\end{minipage} & \begin{minipage}[t]{0.17\columnwidth}\centering\strut
34.1\%\strut
\end{minipage}\tabularnewline
\begin{minipage}[t]{0.07\columnwidth}\centering\strut
8\strut
\end{minipage} & \begin{minipage}[t]{0.12\columnwidth}\raggedright\strut
\$2,537,000\strut
\end{minipage} & \begin{minipage}[t]{0.13\columnwidth}\raggedright\strut
\$2,663,900\strut
\end{minipage} & \begin{minipage}[t]{0.13\columnwidth}\raggedright\strut
\$2,790,700\strut
\end{minipage} & \begin{minipage}[t]{0.17\columnwidth}\centering\strut
27.2\%\strut
\end{minipage} & \begin{minipage}[t]{0.17\columnwidth}\centering\strut
34.8\%\strut
\end{minipage}\tabularnewline
\begin{minipage}[t]{0.07\columnwidth}\centering\strut
9\strut
\end{minipage} & \begin{minipage}[t]{0.12\columnwidth}\raggedright\strut
\$2,332,100\strut
\end{minipage} & \begin{minipage}[t]{0.13\columnwidth}\raggedright\strut
\$2,448,700\strut
\end{minipage} & \begin{minipage}[t]{0.13\columnwidth}\raggedright\strut
\$2,565,300\strut
\end{minipage} & \begin{minipage}[t]{0.17\columnwidth}\centering\strut
27.4\%\strut
\end{minipage} & \begin{minipage}[t]{0.17\columnwidth}\centering\strut
35.5\%\strut
\end{minipage}\tabularnewline
\begin{minipage}[t]{0.07\columnwidth}\centering\strut
10\strut
\end{minipage} & \begin{minipage}[t]{0.12\columnwidth}\raggedright\strut
\$2,215,400\strut
\end{minipage} & \begin{minipage}[t]{0.13\columnwidth}\raggedright\strut
\$2,326,200\strut
\end{minipage} & \begin{minipage}[t]{0.13\columnwidth}\raggedright\strut
\$2,436,900\strut
\end{minipage} & \begin{minipage}[t]{0.17\columnwidth}\centering\strut
27.5\%\strut
\end{minipage} & \begin{minipage}[t]{0.17\columnwidth}\centering\strut
36.2\%\strut
\end{minipage}\tabularnewline
\begin{minipage}[t]{0.07\columnwidth}\centering\strut
11\strut
\end{minipage} & \begin{minipage}[t]{0.12\columnwidth}\raggedright\strut
\$2,104,700\strut
\end{minipage} & \begin{minipage}[t]{0.13\columnwidth}\raggedright\strut
\$2,209,900\strut
\end{minipage} & \begin{minipage}[t]{0.13\columnwidth}\raggedright\strut
\$2,315,200\strut
\end{minipage} & \begin{minipage}[t]{0.17\columnwidth}\centering\strut
32.7\%\strut
\end{minipage} & \begin{minipage}[t]{0.17\columnwidth}\centering\strut
36.9\%\strut
\end{minipage}\tabularnewline
\begin{minipage}[t]{0.07\columnwidth}\centering\strut
12\strut
\end{minipage} & \begin{minipage}[t]{0.12\columnwidth}\raggedright\strut
\$1,999,500\strut
\end{minipage} & \begin{minipage}[t]{0.13\columnwidth}\raggedright\strut
\$2,099,500\strut
\end{minipage} & \begin{minipage}[t]{0.13\columnwidth}\raggedright\strut
\$2,199,500\strut
\end{minipage} & \begin{minipage}[t]{0.17\columnwidth}\centering\strut
37.8\%\strut
\end{minipage} & \begin{minipage}[t]{0.17\columnwidth}\centering\strut
37.6\%\strut
\end{minipage}\tabularnewline
\begin{minipage}[t]{0.07\columnwidth}\centering\strut
13\strut
\end{minipage} & \begin{minipage}[t]{0.12\columnwidth}\raggedright\strut
\$1,899,500\strut
\end{minipage} & \begin{minipage}[t]{0.13\columnwidth}\raggedright\strut
\$1,994,500\strut
\end{minipage} & \begin{minipage}[t]{0.13\columnwidth}\raggedright\strut
\$2,089,500\strut
\end{minipage} & \begin{minipage}[t]{0.17\columnwidth}\centering\strut
42.9\%\strut
\end{minipage} & \begin{minipage}[t]{0.17\columnwidth}\centering\strut
38.3\%\strut
\end{minipage}\tabularnewline
\begin{minipage}[t]{0.07\columnwidth}\centering\strut
14\strut
\end{minipage} & \begin{minipage}[t]{0.12\columnwidth}\raggedright\strut
\$1,804,600\strut
\end{minipage} & \begin{minipage}[t]{0.13\columnwidth}\raggedright\strut
\$1,894,800\strut
\end{minipage} & \begin{minipage}[t]{0.13\columnwidth}\raggedright\strut
\$1,985,100\strut
\end{minipage} & \begin{minipage}[t]{0.17\columnwidth}\centering\strut
48.1\%\strut
\end{minipage} & \begin{minipage}[t]{0.17\columnwidth}\centering\strut
39.1\%\strut
\end{minipage}\tabularnewline
\begin{minipage}[t]{0.07\columnwidth}\centering\strut
15\strut
\end{minipage} & \begin{minipage}[t]{0.12\columnwidth}\raggedright\strut
\$1,714,200\strut
\end{minipage} & \begin{minipage}[t]{0.13\columnwidth}\raggedright\strut
\$1,799,900\strut
\end{minipage} & \begin{minipage}[t]{0.13\columnwidth}\raggedright\strut
\$1,885,600\strut
\end{minipage} & \begin{minipage}[t]{0.17\columnwidth}\centering\strut
53.3\%\strut
\end{minipage} & \begin{minipage}[t]{0.17\columnwidth}\centering\strut
39.8\%\strut
\end{minipage}\tabularnewline
\begin{minipage}[t]{0.07\columnwidth}\centering\strut
16\strut
\end{minipage} & \begin{minipage}[t]{0.12\columnwidth}\raggedright\strut
\$1,628,600\strut
\end{minipage} & \begin{minipage}[t]{0.13\columnwidth}\raggedright\strut
\$1,710,000\strut
\end{minipage} & \begin{minipage}[t]{0.13\columnwidth}\raggedright\strut
\$1,791,500\strut
\end{minipage} & \begin{minipage}[t]{0.17\columnwidth}\centering\strut
53.4\%\strut
\end{minipage} & \begin{minipage}[t]{0.17\columnwidth}\centering\strut
40.5\%\strut
\end{minipage}\tabularnewline
\begin{minipage}[t]{0.07\columnwidth}\centering\strut
17\strut
\end{minipage} & \begin{minipage}[t]{0.12\columnwidth}\raggedright\strut
\$1,547,100\strut
\end{minipage} & \begin{minipage}[t]{0.13\columnwidth}\raggedright\strut
\$1,624,500\strut
\end{minipage} & \begin{minipage}[t]{0.13\columnwidth}\raggedright\strut
\$1,701,800\strut
\end{minipage} & \begin{minipage}[t]{0.17\columnwidth}\centering\strut
53.6\%\strut
\end{minipage} & \begin{minipage}[t]{0.17\columnwidth}\centering\strut
41.2\%\strut
\end{minipage}\tabularnewline
\begin{minipage}[t]{0.07\columnwidth}\centering\strut
18\strut
\end{minipage} & \begin{minipage}[t]{0.12\columnwidth}\raggedright\strut
\$1,469,800\strut
\end{minipage} & \begin{minipage}[t]{0.13\columnwidth}\raggedright\strut
\$1,543,300\strut
\end{minipage} & \begin{minipage}[t]{0.13\columnwidth}\raggedright\strut
\$1,616,800\strut
\end{minipage} & \begin{minipage}[t]{0.17\columnwidth}\centering\strut
53.8\%\strut
\end{minipage} & \begin{minipage}[t]{0.17\columnwidth}\centering\strut
41.9\%\strut
\end{minipage}\tabularnewline
\begin{minipage}[t]{0.07\columnwidth}\centering\strut
19\strut
\end{minipage} & \begin{minipage}[t]{0.12\columnwidth}\raggedright\strut
\$1,403,600\strut
\end{minipage} & \begin{minipage}[t]{0.13\columnwidth}\raggedright\strut
\$1,473,800\strut
\end{minipage} & \begin{minipage}[t]{0.13\columnwidth}\raggedright\strut
\$1,544,000\strut
\end{minipage} & \begin{minipage}[t]{0.17\columnwidth}\centering\strut
54.0\%\strut
\end{minipage} & \begin{minipage}[t]{0.17\columnwidth}\centering\strut
42.6\%\strut
\end{minipage}\tabularnewline
\begin{minipage}[t]{0.07\columnwidth}\centering\strut
20\strut
\end{minipage} & \begin{minipage}[t]{0.12\columnwidth}\raggedright\strut
\$1,347,400\strut
\end{minipage} & \begin{minipage}[t]{0.13\columnwidth}\raggedright\strut
\$1,414,800\strut
\end{minipage} & \begin{minipage}[t]{0.13\columnwidth}\raggedright\strut
\$1,482,100\strut
\end{minipage} & \begin{minipage}[t]{0.17\columnwidth}\centering\strut
54.2\%\strut
\end{minipage} & \begin{minipage}[t]{0.17\columnwidth}\centering\strut
43.3\%\strut
\end{minipage}\tabularnewline
\begin{minipage}[t]{0.07\columnwidth}\centering\strut
21\strut
\end{minipage} & \begin{minipage}[t]{0.12\columnwidth}\raggedright\strut
\$1,293,500\strut
\end{minipage} & \begin{minipage}[t]{0.13\columnwidth}\raggedright\strut
\$1,358,200\strut
\end{minipage} & \begin{minipage}[t]{0.13\columnwidth}\raggedright\strut
\$1,422,900\strut
\end{minipage} & \begin{minipage}[t]{0.17\columnwidth}\centering\strut
59.3\%\strut
\end{minipage} & \begin{minipage}[t]{0.17\columnwidth}\centering\strut
44.1\%\strut
\end{minipage}\tabularnewline
\begin{minipage}[t]{0.07\columnwidth}\centering\strut
22\strut
\end{minipage} & \begin{minipage}[t]{0.12\columnwidth}\raggedright\strut
\$1,241,800\strut
\end{minipage} & \begin{minipage}[t]{0.13\columnwidth}\raggedright\strut
\$1,303,900\strut
\end{minipage} & \begin{minipage}[t]{0.13\columnwidth}\raggedright\strut
\$1,366,000\strut
\end{minipage} & \begin{minipage}[t]{0.17\columnwidth}\centering\strut
64.5\%\strut
\end{minipage} & \begin{minipage}[t]{0.17\columnwidth}\centering\strut
44.8\%\strut
\end{minipage}\tabularnewline
\begin{minipage}[t]{0.07\columnwidth}\centering\strut
23\strut
\end{minipage} & \begin{minipage}[t]{0.12\columnwidth}\raggedright\strut
\$1,192,200\strut
\end{minipage} & \begin{minipage}[t]{0.13\columnwidth}\raggedright\strut
\$1,251,800\strut
\end{minipage} & \begin{minipage}[t]{0.13\columnwidth}\raggedright\strut
\$1,311,400\strut
\end{minipage} & \begin{minipage}[t]{0.17\columnwidth}\centering\strut
69.7\%\strut
\end{minipage} & \begin{minipage}[t]{0.17\columnwidth}\centering\strut
45.5\%\strut
\end{minipage}\tabularnewline
\begin{minipage}[t]{0.07\columnwidth}\centering\strut
24\strut
\end{minipage} & \begin{minipage}[t]{0.12\columnwidth}\raggedright\strut
\$1,144,500\strut
\end{minipage} & \begin{minipage}[t]{0.13\columnwidth}\raggedright\strut
\$1,201,700\strut
\end{minipage} & \begin{minipage}[t]{0.13\columnwidth}\raggedright\strut
\$1,259,000\strut
\end{minipage} & \begin{minipage}[t]{0.17\columnwidth}\centering\strut
74.9\%\strut
\end{minipage} & \begin{minipage}[t]{0.17\columnwidth}\centering\strut
46.2\%\strut
\end{minipage}\tabularnewline
\begin{minipage}[t]{0.07\columnwidth}\centering\strut
25\strut
\end{minipage} & \begin{minipage}[t]{0.12\columnwidth}\raggedright\strut
\$1,098,700\strut
\end{minipage} & \begin{minipage}[t]{0.13\columnwidth}\raggedright\strut
\$1,153,600\strut
\end{minipage} & \begin{minipage}[t]{0.13\columnwidth}\raggedright\strut
\$1,208,600\strut
\end{minipage} & \begin{minipage}[t]{0.17\columnwidth}\centering\strut
80.1\%\strut
\end{minipage} & \begin{minipage}[t]{0.17\columnwidth}\centering\strut
46.9\%\strut
\end{minipage}\tabularnewline
\begin{minipage}[t]{0.07\columnwidth}\centering\strut
26\strut
\end{minipage} & \begin{minipage}[t]{0.12\columnwidth}\raggedright\strut
\$1,062,300\strut
\end{minipage} & \begin{minipage}[t]{0.13\columnwidth}\raggedright\strut
\$1,115,400\strut
\end{minipage} & \begin{minipage}[t]{0.13\columnwidth}\raggedright\strut
\$1,168,500\strut
\end{minipage} & \begin{minipage}[t]{0.17\columnwidth}\centering\strut
80.3\%\strut
\end{minipage} & \begin{minipage}[t]{0.17\columnwidth}\centering\strut
47.6\%\strut
\end{minipage}\tabularnewline
\begin{minipage}[t]{0.07\columnwidth}\centering\strut
27\strut
\end{minipage} & \begin{minipage}[t]{0.12\columnwidth}\raggedright\strut
\$1,031,600\strut
\end{minipage} & \begin{minipage}[t]{0.13\columnwidth}\raggedright\strut
\$1,083,200\strut
\end{minipage} & \begin{minipage}[t]{0.13\columnwidth}\raggedright\strut
\$1,134,800\strut
\end{minipage} & \begin{minipage}[t]{0.17\columnwidth}\centering\strut
80.4\%\strut
\end{minipage} & \begin{minipage}[t]{0.17\columnwidth}\centering\strut
48.3\%\strut
\end{minipage}\tabularnewline
\begin{minipage}[t]{0.07\columnwidth}\centering\strut
28\strut
\end{minipage} & \begin{minipage}[t]{0.12\columnwidth}\raggedright\strut
\$1,025,300\strut
\end{minipage} & \begin{minipage}[t]{0.13\columnwidth}\raggedright\strut
\$1,076,600\strut
\end{minipage} & \begin{minipage}[t]{0.13\columnwidth}\raggedright\strut
\$1,127,800\strut
\end{minipage} & \begin{minipage}[t]{0.17\columnwidth}\centering\strut
80.5\%\strut
\end{minipage} & \begin{minipage}[t]{0.17\columnwidth}\centering\strut
49.0\%\strut
\end{minipage}\tabularnewline
\begin{minipage}[t]{0.07\columnwidth}\centering\strut
29\strut
\end{minipage} & \begin{minipage}[t]{0.12\columnwidth}\raggedright\strut
\$1,017,800\strut
\end{minipage} & \begin{minipage}[t]{0.13\columnwidth}\raggedright\strut
\$1,068,700\strut
\end{minipage} & \begin{minipage}[t]{0.13\columnwidth}\raggedright\strut
\$1,119,600\strut
\end{minipage} & \begin{minipage}[t]{0.17\columnwidth}\centering\strut
80.5\%\strut
\end{minipage} & \begin{minipage}[t]{0.17\columnwidth}\centering\strut
50.0\%\strut
\end{minipage}\tabularnewline
\begin{minipage}[t]{0.07\columnwidth}\centering\strut
30\strut
\end{minipage} & \begin{minipage}[t]{0.12\columnwidth}\raggedright\strut
\$1,010,500\strut
\end{minipage} & \begin{minipage}[t]{0.13\columnwidth}\raggedright\strut
\$1,061,000\strut
\end{minipage} & \begin{minipage}[t]{0.13\columnwidth}\raggedright\strut
\$1,111,600\strut
\end{minipage} & \begin{minipage}[t]{0.17\columnwidth}\centering\strut
80.5\%\strut
\end{minipage} & \begin{minipage}[t]{0.17\columnwidth}\centering\strut
50.0\%\strut
\end{minipage}\tabularnewline
\bottomrule
\end{longtable}

\newpage

\section{2018-19 AND 2019-20 NBA ROOKIE
SCALES}\label{and-2019-20-nba-rookie-scales}

Note: Scales are illustrative and assume 5\% annual growth in the Salary
Cap.

\subsection{2018-19 NBA Rookie Scale}\label{nba-rookie-scale-1}

\begin{longtable}[]{@{}clllcc@{}}
\toprule
\begin{minipage}[b]{0.07\columnwidth}\centering\strut
Pick\strut
\end{minipage} & \begin{minipage}[b]{0.12\columnwidth}\raggedright\strut
1st Year Salary\strut
\end{minipage} & \begin{minipage}[b]{0.13\columnwidth}\raggedright\strut
2nd Year Salary\strut
\end{minipage} & \begin{minipage}[b]{0.13\columnwidth}\raggedright\strut
3rd Year Option Salary\strut
\end{minipage} & \begin{minipage}[b]{0.17\columnwidth}\centering\strut
4th Year Option: Percentage Increase Over 3rd Year Salary\strut
\end{minipage} & \begin{minipage}[b]{0.17\columnwidth}\centering\strut
Qualifying Offer: Percentage Increase Over 4th Year Salary\strut
\end{minipage}\tabularnewline
\midrule
\endhead
\begin{minipage}[t]{0.07\columnwidth}\centering\strut
1\strut
\end{minipage} & \begin{minipage}[t]{0.12\columnwidth}\raggedright\strut
\$6,949,900\strut
\end{minipage} & \begin{minipage}[t]{0.13\columnwidth}\raggedright\strut
\$8,139,400\strut
\end{minipage} & \begin{minipage}[t]{0.13\columnwidth}\raggedright\strut
\$8,527,100\strut
\end{minipage} & \begin{minipage}[t]{0.17\columnwidth}\centering\strut
26.1\%\strut
\end{minipage} & \begin{minipage}[t]{0.17\columnwidth}\centering\strut
30.0\%\strut
\end{minipage}\tabularnewline
\begin{minipage}[t]{0.07\columnwidth}\centering\strut
2\strut
\end{minipage} & \begin{minipage}[t]{0.12\columnwidth}\raggedright\strut
\$6,218,300\strut
\end{minipage} & \begin{minipage}[t]{0.13\columnwidth}\raggedright\strut
\$7,282,600\strut
\end{minipage} & \begin{minipage}[t]{0.13\columnwidth}\raggedright\strut
\$7,629,400\strut
\end{minipage} & \begin{minipage}[t]{0.17\columnwidth}\centering\strut
26.2\%\strut
\end{minipage} & \begin{minipage}[t]{0.17\columnwidth}\centering\strut
30.5\%\strut
\end{minipage}\tabularnewline
\begin{minipage}[t]{0.07\columnwidth}\centering\strut
3\strut
\end{minipage} & \begin{minipage}[t]{0.12\columnwidth}\raggedright\strut
\$5,584,100\strut
\end{minipage} & \begin{minipage}[t]{0.13\columnwidth}\raggedright\strut
\$6,539,700\strut
\end{minipage} & \begin{minipage}[t]{0.13\columnwidth}\raggedright\strut
\$6,851,300\strut
\end{minipage} & \begin{minipage}[t]{0.17\columnwidth}\centering\strut
26.4\%\strut
\end{minipage} & \begin{minipage}[t]{0.17\columnwidth}\centering\strut
31.2\%\strut
\end{minipage}\tabularnewline
\begin{minipage}[t]{0.07\columnwidth}\centering\strut
4\strut
\end{minipage} & \begin{minipage}[t]{0.12\columnwidth}\raggedright\strut
\$5,034,700\strut
\end{minipage} & \begin{minipage}[t]{0.13\columnwidth}\raggedright\strut
\$5,896,300\strut
\end{minipage} & \begin{minipage}[t]{0.13\columnwidth}\raggedright\strut
\$6,177,100\strut
\end{minipage} & \begin{minipage}[t]{0.17\columnwidth}\centering\strut
26.5\%\strut
\end{minipage} & \begin{minipage}[t]{0.17\columnwidth}\centering\strut
31.9\%\strut
\end{minipage}\tabularnewline
\begin{minipage}[t]{0.07\columnwidth}\centering\strut
5\strut
\end{minipage} & \begin{minipage}[t]{0.12\columnwidth}\raggedright\strut
\$4,559,100\strut
\end{minipage} & \begin{minipage}[t]{0.13\columnwidth}\raggedright\strut
\$5,339,400\strut
\end{minipage} & \begin{minipage}[t]{0.13\columnwidth}\raggedright\strut
\$5,593,700\strut
\end{minipage} & \begin{minipage}[t]{0.17\columnwidth}\centering\strut
26.7\%\strut
\end{minipage} & \begin{minipage}[t]{0.17\columnwidth}\centering\strut
32.6\%\strut
\end{minipage}\tabularnewline
\begin{minipage}[t]{0.07\columnwidth}\centering\strut
6\strut
\end{minipage} & \begin{minipage}[t]{0.12\columnwidth}\raggedright\strut
\$4,140,900\strut
\end{minipage} & \begin{minipage}[t]{0.13\columnwidth}\raggedright\strut
\$4,849,600\strut
\end{minipage} & \begin{minipage}[t]{0.13\columnwidth}\raggedright\strut
\$5,080,600\strut
\end{minipage} & \begin{minipage}[t]{0.17\columnwidth}\centering\strut
26.8\%\strut
\end{minipage} & \begin{minipage}[t]{0.17\columnwidth}\centering\strut
33.4\%\strut
\end{minipage}\tabularnewline
\begin{minipage}[t]{0.07\columnwidth}\centering\strut
7\strut
\end{minipage} & \begin{minipage}[t]{0.12\columnwidth}\raggedright\strut
\$3,780,100\strut
\end{minipage} & \begin{minipage}[t]{0.13\columnwidth}\raggedright\strut
\$4,427,100\strut
\end{minipage} & \begin{minipage}[t]{0.13\columnwidth}\raggedright\strut
\$4,637,800\strut
\end{minipage} & \begin{minipage}[t]{0.17\columnwidth}\centering\strut
27.0\%\strut
\end{minipage} & \begin{minipage}[t]{0.17\columnwidth}\centering\strut
34.1\%\strut
\end{minipage}\tabularnewline
\begin{minipage}[t]{0.07\columnwidth}\centering\strut
8\strut
\end{minipage} & \begin{minipage}[t]{0.12\columnwidth}\raggedright\strut
\$3,463,000\strut
\end{minipage} & \begin{minipage}[t]{0.13\columnwidth}\raggedright\strut
\$4,055,800\strut
\end{minipage} & \begin{minipage}[t]{0.13\columnwidth}\raggedright\strut
\$4,248,800\strut
\end{minipage} & \begin{minipage}[t]{0.17\columnwidth}\centering\strut
27.2\%\strut
\end{minipage} & \begin{minipage}[t]{0.17\columnwidth}\centering\strut
34.8\%\strut
\end{minipage}\tabularnewline
\begin{minipage}[t]{0.07\columnwidth}\centering\strut
9\strut
\end{minipage} & \begin{minipage}[t]{0.12\columnwidth}\raggedright\strut
\$3,183,300\strut
\end{minipage} & \begin{minipage}[t]{0.13\columnwidth}\raggedright\strut
\$3,728,100\strut
\end{minipage} & \begin{minipage}[t]{0.13\columnwidth}\raggedright\strut
\$3,905,700\strut
\end{minipage} & \begin{minipage}[t]{0.17\columnwidth}\centering\strut
27.4\%\strut
\end{minipage} & \begin{minipage}[t]{0.17\columnwidth}\centering\strut
35.5\%\strut
\end{minipage}\tabularnewline
\begin{minipage}[t]{0.07\columnwidth}\centering\strut
10\strut
\end{minipage} & \begin{minipage}[t]{0.12\columnwidth}\raggedright\strut
\$3,024,000\strut
\end{minipage} & \begin{minipage}[t]{0.13\columnwidth}\raggedright\strut
\$3,541,600\strut
\end{minipage} & \begin{minipage}[t]{0.13\columnwidth}\raggedright\strut
\$3,710,200\strut
\end{minipage} & \begin{minipage}[t]{0.17\columnwidth}\centering\strut
27.5\%\strut
\end{minipage} & \begin{minipage}[t]{0.17\columnwidth}\centering\strut
36.2\%\strut
\end{minipage}\tabularnewline
\begin{minipage}[t]{0.07\columnwidth}\centering\strut
11\strut
\end{minipage} & \begin{minipage}[t]{0.12\columnwidth}\raggedright\strut
\$2,872,900\strut
\end{minipage} & \begin{minipage}[t]{0.13\columnwidth}\raggedright\strut
\$3,364,600\strut
\end{minipage} & \begin{minipage}[t]{0.13\columnwidth}\raggedright\strut
\$3,524,900\strut
\end{minipage} & \begin{minipage}[t]{0.17\columnwidth}\centering\strut
32.7\%\strut
\end{minipage} & \begin{minipage}[t]{0.17\columnwidth}\centering\strut
36.9\%\strut
\end{minipage}\tabularnewline
\begin{minipage}[t]{0.07\columnwidth}\centering\strut
12\strut
\end{minipage} & \begin{minipage}[t]{0.12\columnwidth}\raggedright\strut
\$2,729,300\strut
\end{minipage} & \begin{minipage}[t]{0.13\columnwidth}\raggedright\strut
\$3,196,500\strut
\end{minipage} & \begin{minipage}[t]{0.13\columnwidth}\raggedright\strut
\$3,348,700\strut
\end{minipage} & \begin{minipage}[t]{0.17\columnwidth}\centering\strut
37.8\%\strut
\end{minipage} & \begin{minipage}[t]{0.17\columnwidth}\centering\strut
37.6\%\strut
\end{minipage}\tabularnewline
\begin{minipage}[t]{0.07\columnwidth}\centering\strut
13\strut
\end{minipage} & \begin{minipage}[t]{0.12\columnwidth}\raggedright\strut
\$2,592,800\strut
\end{minipage} & \begin{minipage}[t]{0.13\columnwidth}\raggedright\strut
\$3,036,600\strut
\end{minipage} & \begin{minipage}[t]{0.13\columnwidth}\raggedright\strut
\$3,181,300\strut
\end{minipage} & \begin{minipage}[t]{0.17\columnwidth}\centering\strut
42.9\%\strut
\end{minipage} & \begin{minipage}[t]{0.17\columnwidth}\centering\strut
38.3\%\strut
\end{minipage}\tabularnewline
\begin{minipage}[t]{0.07\columnwidth}\centering\strut
14\strut
\end{minipage} & \begin{minipage}[t]{0.12\columnwidth}\raggedright\strut
\$2,463,300\strut
\end{minipage} & \begin{minipage}[t]{0.13\columnwidth}\raggedright\strut
\$2,884,800\strut
\end{minipage} & \begin{minipage}[t]{0.13\columnwidth}\raggedright\strut
\$3,022,300\strut
\end{minipage} & \begin{minipage}[t]{0.17\columnwidth}\centering\strut
48.1\%\strut
\end{minipage} & \begin{minipage}[t]{0.17\columnwidth}\centering\strut
39.1\%\strut
\end{minipage}\tabularnewline
\begin{minipage}[t]{0.07\columnwidth}\centering\strut
15\strut
\end{minipage} & \begin{minipage}[t]{0.12\columnwidth}\raggedright\strut
\$2,339,900\strut
\end{minipage} & \begin{minipage}[t]{0.13\columnwidth}\raggedright\strut
\$2,740,300\strut
\end{minipage} & \begin{minipage}[t]{0.13\columnwidth}\raggedright\strut
\$2,870,800\strut
\end{minipage} & \begin{minipage}[t]{0.17\columnwidth}\centering\strut
53.3\%\strut
\end{minipage} & \begin{minipage}[t]{0.17\columnwidth}\centering\strut
39.8\%\strut
\end{minipage}\tabularnewline
\begin{minipage}[t]{0.07\columnwidth}\centering\strut
16\strut
\end{minipage} & \begin{minipage}[t]{0.12\columnwidth}\raggedright\strut
\$2,223,000\strut
\end{minipage} & \begin{minipage}[t]{0.13\columnwidth}\raggedright\strut
\$2,603,500\strut
\end{minipage} & \begin{minipage}[t]{0.13\columnwidth}\raggedright\strut
\$2,727,600\strut
\end{minipage} & \begin{minipage}[t]{0.17\columnwidth}\centering\strut
53.4\%\strut
\end{minipage} & \begin{minipage}[t]{0.17\columnwidth}\centering\strut
40.5\%\strut
\end{minipage}\tabularnewline
\begin{minipage}[t]{0.07\columnwidth}\centering\strut
17\strut
\end{minipage} & \begin{minipage}[t]{0.12\columnwidth}\raggedright\strut
\$2,111,800\strut
\end{minipage} & \begin{minipage}[t]{0.13\columnwidth}\raggedright\strut
\$2,473,300\strut
\end{minipage} & \begin{minipage}[t]{0.13\columnwidth}\raggedright\strut
\$2,591,000\strut
\end{minipage} & \begin{minipage}[t]{0.17\columnwidth}\centering\strut
53.6\%\strut
\end{minipage} & \begin{minipage}[t]{0.17\columnwidth}\centering\strut
41.2\%\strut
\end{minipage}\tabularnewline
\begin{minipage}[t]{0.07\columnwidth}\centering\strut
18\strut
\end{minipage} & \begin{minipage}[t]{0.12\columnwidth}\raggedright\strut
\$2,006,300\strut
\end{minipage} & \begin{minipage}[t]{0.13\columnwidth}\raggedright\strut
\$2,349,700\strut
\end{minipage} & \begin{minipage}[t]{0.13\columnwidth}\raggedright\strut
\$2,461,600\strut
\end{minipage} & \begin{minipage}[t]{0.17\columnwidth}\centering\strut
53.8\%\strut
\end{minipage} & \begin{minipage}[t]{0.17\columnwidth}\centering\strut
41.9\%\strut
\end{minipage}\tabularnewline
\begin{minipage}[t]{0.07\columnwidth}\centering\strut
19\strut
\end{minipage} & \begin{minipage}[t]{0.12\columnwidth}\raggedright\strut
\$1,915,900\strut
\end{minipage} & \begin{minipage}[t]{0.13\columnwidth}\raggedright\strut
\$2,243,900\strut
\end{minipage} & \begin{minipage}[t]{0.13\columnwidth}\raggedright\strut
\$2,350,700\strut
\end{minipage} & \begin{minipage}[t]{0.17\columnwidth}\centering\strut
54.0\%\strut
\end{minipage} & \begin{minipage}[t]{0.17\columnwidth}\centering\strut
42.6\%\strut
\end{minipage}\tabularnewline
\begin{minipage}[t]{0.07\columnwidth}\centering\strut
20\strut
\end{minipage} & \begin{minipage}[t]{0.12\columnwidth}\raggedright\strut
\$1,839,200\strut
\end{minipage} & \begin{minipage}[t]{0.13\columnwidth}\raggedright\strut
\$2,154,000\strut
\end{minipage} & \begin{minipage}[t]{0.13\columnwidth}\raggedright\strut
\$2,256,500\strut
\end{minipage} & \begin{minipage}[t]{0.17\columnwidth}\centering\strut
54.2\%\strut
\end{minipage} & \begin{minipage}[t]{0.17\columnwidth}\centering\strut
43.3\%\strut
\end{minipage}\tabularnewline
\begin{minipage}[t]{0.07\columnwidth}\centering\strut
21\strut
\end{minipage} & \begin{minipage}[t]{0.12\columnwidth}\raggedright\strut
\$1,765,600\strut
\end{minipage} & \begin{minipage}[t]{0.13\columnwidth}\raggedright\strut
\$2,067,900\strut
\end{minipage} & \begin{minipage}[t]{0.13\columnwidth}\raggedright\strut
\$2,166,400\strut
\end{minipage} & \begin{minipage}[t]{0.17\columnwidth}\centering\strut
59.3\%\strut
\end{minipage} & \begin{minipage}[t]{0.17\columnwidth}\centering\strut
44.1\%\strut
\end{minipage}\tabularnewline
\begin{minipage}[t]{0.07\columnwidth}\centering\strut
22\strut
\end{minipage} & \begin{minipage}[t]{0.12\columnwidth}\raggedright\strut
\$1,695,100\strut
\end{minipage} & \begin{minipage}[t]{0.13\columnwidth}\raggedright\strut
\$1,985,200\strut
\end{minipage} & \begin{minipage}[t]{0.13\columnwidth}\raggedright\strut
\$2,079,700\strut
\end{minipage} & \begin{minipage}[t]{0.17\columnwidth}\centering\strut
64.5\%\strut
\end{minipage} & \begin{minipage}[t]{0.17\columnwidth}\centering\strut
44.8\%\strut
\end{minipage}\tabularnewline
\begin{minipage}[t]{0.07\columnwidth}\centering\strut
23\strut
\end{minipage} & \begin{minipage}[t]{0.12\columnwidth}\raggedright\strut
\$1,627,400\strut
\end{minipage} & \begin{minipage}[t]{0.13\columnwidth}\raggedright\strut
\$1,905,900\strut
\end{minipage} & \begin{minipage}[t]{0.13\columnwidth}\raggedright\strut
\$1,996,600\strut
\end{minipage} & \begin{minipage}[t]{0.17\columnwidth}\centering\strut
69.7\%\strut
\end{minipage} & \begin{minipage}[t]{0.17\columnwidth}\centering\strut
45.5\%\strut
\end{minipage}\tabularnewline
\begin{minipage}[t]{0.07\columnwidth}\centering\strut
24\strut
\end{minipage} & \begin{minipage}[t]{0.12\columnwidth}\raggedright\strut
\$1,562,200\strut
\end{minipage} & \begin{minipage}[t]{0.13\columnwidth}\raggedright\strut
\$1,829,600\strut
\end{minipage} & \begin{minipage}[t]{0.13\columnwidth}\raggedright\strut
\$1,916,800\strut
\end{minipage} & \begin{minipage}[t]{0.17\columnwidth}\centering\strut
74.9\%\strut
\end{minipage} & \begin{minipage}[t]{0.17\columnwidth}\centering\strut
46.2\%\strut
\end{minipage}\tabularnewline
\begin{minipage}[t]{0.07\columnwidth}\centering\strut
25\strut
\end{minipage} & \begin{minipage}[t]{0.12\columnwidth}\raggedright\strut
\$1,499,700\strut
\end{minipage} & \begin{minipage}[t]{0.13\columnwidth}\raggedright\strut
\$1,756,400\strut
\end{minipage} & \begin{minipage}[t]{0.13\columnwidth}\raggedright\strut
\$1,840,100\strut
\end{minipage} & \begin{minipage}[t]{0.17\columnwidth}\centering\strut
80.1\%\strut
\end{minipage} & \begin{minipage}[t]{0.17\columnwidth}\centering\strut
46.9\%\strut
\end{minipage}\tabularnewline
\begin{minipage}[t]{0.07\columnwidth}\centering\strut
26\strut
\end{minipage} & \begin{minipage}[t]{0.12\columnwidth}\raggedright\strut
\$1,450,000\strut
\end{minipage} & \begin{minipage}[t]{0.13\columnwidth}\raggedright\strut
\$1,698,200\strut
\end{minipage} & \begin{minipage}[t]{0.13\columnwidth}\raggedright\strut
\$1,779,000\strut
\end{minipage} & \begin{minipage}[t]{0.17\columnwidth}\centering\strut
80.3\%\strut
\end{minipage} & \begin{minipage}[t]{0.17\columnwidth}\centering\strut
47.6\%\strut
\end{minipage}\tabularnewline
\begin{minipage}[t]{0.07\columnwidth}\centering\strut
27\strut
\end{minipage} & \begin{minipage}[t]{0.12\columnwidth}\raggedright\strut
\$1,408,100\strut
\end{minipage} & \begin{minipage}[t]{0.13\columnwidth}\raggedright\strut
\$1,649,200\strut
\end{minipage} & \begin{minipage}[t]{0.13\columnwidth}\raggedright\strut
\$1,727,700\strut
\end{minipage} & \begin{minipage}[t]{0.17\columnwidth}\centering\strut
80.4\%\strut
\end{minipage} & \begin{minipage}[t]{0.17\columnwidth}\centering\strut
48.3\%\strut
\end{minipage}\tabularnewline
\begin{minipage}[t]{0.07\columnwidth}\centering\strut
28\strut
\end{minipage} & \begin{minipage}[t]{0.12\columnwidth}\raggedright\strut
\$1,399,500\strut
\end{minipage} & \begin{minipage}[t]{0.13\columnwidth}\raggedright\strut
\$1,639,100\strut
\end{minipage} & \begin{minipage}[t]{0.13\columnwidth}\raggedright\strut
\$1,717,100\strut
\end{minipage} & \begin{minipage}[t]{0.17\columnwidth}\centering\strut
80.5\%\strut
\end{minipage} & \begin{minipage}[t]{0.17\columnwidth}\centering\strut
49.0\%\strut
\end{minipage}\tabularnewline
\begin{minipage}[t]{0.07\columnwidth}\centering\strut
29\strut
\end{minipage} & \begin{minipage}[t]{0.12\columnwidth}\raggedright\strut
\$1,389,300\strut
\end{minipage} & \begin{minipage}[t]{0.13\columnwidth}\raggedright\strut
\$1,627,100\strut
\end{minipage} & \begin{minipage}[t]{0.13\columnwidth}\raggedright\strut
\$1,704,600\strut
\end{minipage} & \begin{minipage}[t]{0.17\columnwidth}\centering\strut
80.5\%\strut
\end{minipage} & \begin{minipage}[t]{0.17\columnwidth}\centering\strut
50.0\%\strut
\end{minipage}\tabularnewline
\begin{minipage}[t]{0.07\columnwidth}\centering\strut
30\strut
\end{minipage} & \begin{minipage}[t]{0.12\columnwidth}\raggedright\strut
\$1,379,300\strut
\end{minipage} & \begin{minipage}[t]{0.13\columnwidth}\raggedright\strut
\$1,615,400\strut
\end{minipage} & \begin{minipage}[t]{0.13\columnwidth}\raggedright\strut
\$1,692,400\strut
\end{minipage} & \begin{minipage}[t]{0.17\columnwidth}\centering\strut
80.5\%\strut
\end{minipage} & \begin{minipage}[t]{0.17\columnwidth}\centering\strut
50.0\%\strut
\end{minipage}\tabularnewline
\bottomrule
\end{longtable}

\newpage

\subsection{2019-20 NBA Rookie Scale}\label{nba-rookie-scale-2}

\begin{longtable}[]{@{}clllcc@{}}
\toprule
\begin{minipage}[b]{0.07\columnwidth}\centering\strut
Pick\strut
\end{minipage} & \begin{minipage}[b]{0.12\columnwidth}\raggedright\strut
1st Year Salary\strut
\end{minipage} & \begin{minipage}[b]{0.13\columnwidth}\raggedright\strut
2nd Year Salary\strut
\end{minipage} & \begin{minipage}[b]{0.13\columnwidth}\raggedright\strut
3rd Year Option Salary\strut
\end{minipage} & \begin{minipage}[b]{0.17\columnwidth}\centering\strut
4th Year Option: Percentage Increase Over 3rd Year Salary\strut
\end{minipage} & \begin{minipage}[b]{0.17\columnwidth}\centering\strut
Qualifying Offer: Percentage Increase Over 4th Year Salary\strut
\end{minipage}\tabularnewline
\midrule
\endhead
\begin{minipage}[t]{0.07\columnwidth}\centering\strut
1\strut
\end{minipage} & \begin{minipage}[t]{0.12\columnwidth}\raggedright\strut
\$8,139,400\strut
\end{minipage} & \begin{minipage}[t]{0.13\columnwidth}\raggedright\strut
\$8,546,400\strut
\end{minipage} & \begin{minipage}[t]{0.13\columnwidth}\raggedright\strut
\$8,953,400\strut
\end{minipage} & \begin{minipage}[t]{0.17\columnwidth}\centering\strut
26.1\%\strut
\end{minipage} & \begin{minipage}[t]{0.17\columnwidth}\centering\strut
30.0\%\strut
\end{minipage}\tabularnewline
\begin{minipage}[t]{0.07\columnwidth}\centering\strut
2\strut
\end{minipage} & \begin{minipage}[t]{0.12\columnwidth}\raggedright\strut
\$7,282,500\strut
\end{minipage} & \begin{minipage}[t]{0.13\columnwidth}\raggedright\strut
\$7,646,700\strut
\end{minipage} & \begin{minipage}[t]{0.13\columnwidth}\raggedright\strut
\$8,010,900\strut
\end{minipage} & \begin{minipage}[t]{0.17\columnwidth}\centering\strut
26.2\%\strut
\end{minipage} & \begin{minipage}[t]{0.17\columnwidth}\centering\strut
30.5\%\strut
\end{minipage}\tabularnewline
\begin{minipage}[t]{0.07\columnwidth}\centering\strut
3\strut
\end{minipage} & \begin{minipage}[t]{0.12\columnwidth}\raggedright\strut
\$6,539,800\strut
\end{minipage} & \begin{minipage}[t]{0.13\columnwidth}\raggedright\strut
\$6,866,700\strut
\end{minipage} & \begin{minipage}[t]{0.13\columnwidth}\raggedright\strut
\$7,193,800\strut
\end{minipage} & \begin{minipage}[t]{0.17\columnwidth}\centering\strut
26.4\%\strut
\end{minipage} & \begin{minipage}[t]{0.17\columnwidth}\centering\strut
31.2\%\strut
\end{minipage}\tabularnewline
\begin{minipage}[t]{0.07\columnwidth}\centering\strut
4\strut
\end{minipage} & \begin{minipage}[t]{0.12\columnwidth}\raggedright\strut
\$5,896,400\strut
\end{minipage} & \begin{minipage}[t]{0.13\columnwidth}\raggedright\strut
\$6,191,200\strut
\end{minipage} & \begin{minipage}[t]{0.13\columnwidth}\raggedright\strut
\$6,485,900\strut
\end{minipage} & \begin{minipage}[t]{0.17\columnwidth}\centering\strut
26.5\%\strut
\end{minipage} & \begin{minipage}[t]{0.17\columnwidth}\centering\strut
31.9\%\strut
\end{minipage}\tabularnewline
\begin{minipage}[t]{0.07\columnwidth}\centering\strut
5\strut
\end{minipage} & \begin{minipage}[t]{0.12\columnwidth}\raggedright\strut
\$5,339,400\strut
\end{minipage} & \begin{minipage}[t]{0.13\columnwidth}\raggedright\strut
\$5,606,400\strut
\end{minipage} & \begin{minipage}[t]{0.13\columnwidth}\raggedright\strut
\$5,873,300\strut
\end{minipage} & \begin{minipage}[t]{0.17\columnwidth}\centering\strut
26.7\%\strut
\end{minipage} & \begin{minipage}[t]{0.17\columnwidth}\centering\strut
32.6\%\strut
\end{minipage}\tabularnewline
\begin{minipage}[t]{0.07\columnwidth}\centering\strut
6\strut
\end{minipage} & \begin{minipage}[t]{0.12\columnwidth}\raggedright\strut
\$4,849,600\strut
\end{minipage} & \begin{minipage}[t]{0.13\columnwidth}\raggedright\strut
\$5,092,100\strut
\end{minipage} & \begin{minipage}[t]{0.13\columnwidth}\raggedright\strut
\$5,334,600\strut
\end{minipage} & \begin{minipage}[t]{0.17\columnwidth}\centering\strut
26.8\%\strut
\end{minipage} & \begin{minipage}[t]{0.17\columnwidth}\centering\strut
33.4\%\strut
\end{minipage}\tabularnewline
\begin{minipage}[t]{0.07\columnwidth}\centering\strut
7\strut
\end{minipage} & \begin{minipage}[t]{0.12\columnwidth}\raggedright\strut
\$4,427,100\strut
\end{minipage} & \begin{minipage}[t]{0.13\columnwidth}\raggedright\strut
\$4,648,500\strut
\end{minipage} & \begin{minipage}[t]{0.13\columnwidth}\raggedright\strut
\$4,869,700\strut
\end{minipage} & \begin{minipage}[t]{0.17\columnwidth}\centering\strut
27.0\%\strut
\end{minipage} & \begin{minipage}[t]{0.17\columnwidth}\centering\strut
34.1\%\strut
\end{minipage}\tabularnewline
\begin{minipage}[t]{0.07\columnwidth}\centering\strut
8\strut
\end{minipage} & \begin{minipage}[t]{0.12\columnwidth}\raggedright\strut
\$4,055,700\strut
\end{minipage} & \begin{minipage}[t]{0.13\columnwidth}\raggedright\strut
\$4,258,600\strut
\end{minipage} & \begin{minipage}[t]{0.13\columnwidth}\raggedright\strut
\$4,461,300\strut
\end{minipage} & \begin{minipage}[t]{0.17\columnwidth}\centering\strut
27.2\%\strut
\end{minipage} & \begin{minipage}[t]{0.17\columnwidth}\centering\strut
34.8\%\strut
\end{minipage}\tabularnewline
\begin{minipage}[t]{0.07\columnwidth}\centering\strut
9\strut
\end{minipage} & \begin{minipage}[t]{0.12\columnwidth}\raggedright\strut
\$3,728,200\strut
\end{minipage} & \begin{minipage}[t]{0.13\columnwidth}\raggedright\strut
\$3,914,600\strut
\end{minipage} & \begin{minipage}[t]{0.13\columnwidth}\raggedright\strut
\$4,101,000\strut
\end{minipage} & \begin{minipage}[t]{0.17\columnwidth}\centering\strut
27.4\%\strut
\end{minipage} & \begin{minipage}[t]{0.17\columnwidth}\centering\strut
35.5\%\strut
\end{minipage}\tabularnewline
\begin{minipage}[t]{0.07\columnwidth}\centering\strut
10\strut
\end{minipage} & \begin{minipage}[t]{0.12\columnwidth}\raggedright\strut
\$3,541,600\strut
\end{minipage} & \begin{minipage}[t]{0.13\columnwidth}\raggedright\strut
\$3,718,700\strut
\end{minipage} & \begin{minipage}[t]{0.13\columnwidth}\raggedright\strut
\$3,895,700\strut
\end{minipage} & \begin{minipage}[t]{0.17\columnwidth}\centering\strut
27.5\%\strut
\end{minipage} & \begin{minipage}[t]{0.17\columnwidth}\centering\strut
36.2\%\strut
\end{minipage}\tabularnewline
\begin{minipage}[t]{0.07\columnwidth}\centering\strut
11\strut
\end{minipage} & \begin{minipage}[t]{0.12\columnwidth}\raggedright\strut
\$3,364,600\strut
\end{minipage} & \begin{minipage}[t]{0.13\columnwidth}\raggedright\strut
\$3,532,800\strut
\end{minipage} & \begin{minipage}[t]{0.13\columnwidth}\raggedright\strut
\$3,701,100\strut
\end{minipage} & \begin{minipage}[t]{0.17\columnwidth}\centering\strut
32.7\%\strut
\end{minipage} & \begin{minipage}[t]{0.17\columnwidth}\centering\strut
36.9\%\strut
\end{minipage}\tabularnewline
\begin{minipage}[t]{0.07\columnwidth}\centering\strut
12\strut
\end{minipage} & \begin{minipage}[t]{0.12\columnwidth}\raggedright\strut
\$3,196,500\strut
\end{minipage} & \begin{minipage}[t]{0.13\columnwidth}\raggedright\strut
\$3,356,300\strut
\end{minipage} & \begin{minipage}[t]{0.13\columnwidth}\raggedright\strut
\$3,516,200\strut
\end{minipage} & \begin{minipage}[t]{0.17\columnwidth}\centering\strut
37.8\%\strut
\end{minipage} & \begin{minipage}[t]{0.17\columnwidth}\centering\strut
37.6\%\strut
\end{minipage}\tabularnewline
\begin{minipage}[t]{0.07\columnwidth}\centering\strut
13\strut
\end{minipage} & \begin{minipage}[t]{0.12\columnwidth}\raggedright\strut
\$3,036,600\strut
\end{minipage} & \begin{minipage}[t]{0.13\columnwidth}\raggedright\strut
\$3,188,500\strut
\end{minipage} & \begin{minipage}[t]{0.13\columnwidth}\raggedright\strut
\$3,340,300\strut
\end{minipage} & \begin{minipage}[t]{0.17\columnwidth}\centering\strut
42.9\%\strut
\end{minipage} & \begin{minipage}[t]{0.17\columnwidth}\centering\strut
38.3\%\strut
\end{minipage}\tabularnewline
\begin{minipage}[t]{0.07\columnwidth}\centering\strut
14\strut
\end{minipage} & \begin{minipage}[t]{0.12\columnwidth}\raggedright\strut
\$2,884,900\strut
\end{minipage} & \begin{minipage}[t]{0.13\columnwidth}\raggedright\strut
\$3,029,100\strut
\end{minipage} & \begin{minipage}[t]{0.13\columnwidth}\raggedright\strut
\$3,173,400\strut
\end{minipage} & \begin{minipage}[t]{0.17\columnwidth}\centering\strut
48.1\%\strut
\end{minipage} & \begin{minipage}[t]{0.17\columnwidth}\centering\strut
39.1\%\strut
\end{minipage}\tabularnewline
\begin{minipage}[t]{0.07\columnwidth}\centering\strut
15\strut
\end{minipage} & \begin{minipage}[t]{0.12\columnwidth}\raggedright\strut
\$2,740,400\strut
\end{minipage} & \begin{minipage}[t]{0.13\columnwidth}\raggedright\strut
\$2,877,400\strut
\end{minipage} & \begin{minipage}[t]{0.13\columnwidth}\raggedright\strut
\$3,014,400\strut
\end{minipage} & \begin{minipage}[t]{0.17\columnwidth}\centering\strut
53.3\%\strut
\end{minipage} & \begin{minipage}[t]{0.17\columnwidth}\centering\strut
39.8\%\strut
\end{minipage}\tabularnewline
\begin{minipage}[t]{0.07\columnwidth}\centering\strut
16\strut
\end{minipage} & \begin{minipage}[t]{0.12\columnwidth}\raggedright\strut
\$2,603,500\strut
\end{minipage} & \begin{minipage}[t]{0.13\columnwidth}\raggedright\strut
\$2,733,600\strut
\end{minipage} & \begin{minipage}[t]{0.13\columnwidth}\raggedright\strut
\$2,863,900\strut
\end{minipage} & \begin{minipage}[t]{0.17\columnwidth}\centering\strut
53.4\%\strut
\end{minipage} & \begin{minipage}[t]{0.17\columnwidth}\centering\strut
40.5\%\strut
\end{minipage}\tabularnewline
\begin{minipage}[t]{0.07\columnwidth}\centering\strut
17\strut
\end{minipage} & \begin{minipage}[t]{0.12\columnwidth}\raggedright\strut
\$2,473,200\strut
\end{minipage} & \begin{minipage}[t]{0.13\columnwidth}\raggedright\strut
\$2,597,000\strut
\end{minipage} & \begin{minipage}[t]{0.13\columnwidth}\raggedright\strut
\$2,720,500\strut
\end{minipage} & \begin{minipage}[t]{0.17\columnwidth}\centering\strut
53.6\%\strut
\end{minipage} & \begin{minipage}[t]{0.17\columnwidth}\centering\strut
41.2\%\strut
\end{minipage}\tabularnewline
\begin{minipage}[t]{0.07\columnwidth}\centering\strut
18\strut
\end{minipage} & \begin{minipage}[t]{0.12\columnwidth}\raggedright\strut
\$2,349,700\strut
\end{minipage} & \begin{minipage}[t]{0.13\columnwidth}\raggedright\strut
\$2,467,200\strut
\end{minipage} & \begin{minipage}[t]{0.13\columnwidth}\raggedright\strut
\$2,584,700\strut
\end{minipage} & \begin{minipage}[t]{0.17\columnwidth}\centering\strut
53.8\%\strut
\end{minipage} & \begin{minipage}[t]{0.17\columnwidth}\centering\strut
41.9\%\strut
\end{minipage}\tabularnewline
\begin{minipage}[t]{0.07\columnwidth}\centering\strut
19\strut
\end{minipage} & \begin{minipage}[t]{0.12\columnwidth}\raggedright\strut
\$2,243,800\strut
\end{minipage} & \begin{minipage}[t]{0.13\columnwidth}\raggedright\strut
\$2,356,100\strut
\end{minipage} & \begin{minipage}[t]{0.13\columnwidth}\raggedright\strut
\$2,468,300\strut
\end{minipage} & \begin{minipage}[t]{0.17\columnwidth}\centering\strut
54.0\%\strut
\end{minipage} & \begin{minipage}[t]{0.17\columnwidth}\centering\strut
42.6\%\strut
\end{minipage}\tabularnewline
\begin{minipage}[t]{0.07\columnwidth}\centering\strut
20\strut
\end{minipage} & \begin{minipage}[t]{0.12\columnwidth}\raggedright\strut
\$2,154,000\strut
\end{minipage} & \begin{minipage}[t]{0.13\columnwidth}\raggedright\strut
\$2,261,700\strut
\end{minipage} & \begin{minipage}[t]{0.13\columnwidth}\raggedright\strut
\$2,369,300\strut
\end{minipage} & \begin{minipage}[t]{0.17\columnwidth}\centering\strut
54.2\%\strut
\end{minipage} & \begin{minipage}[t]{0.17\columnwidth}\centering\strut
43.3\%\strut
\end{minipage}\tabularnewline
\begin{minipage}[t]{0.07\columnwidth}\centering\strut
21\strut
\end{minipage} & \begin{minipage}[t]{0.12\columnwidth}\raggedright\strut
\$2,067,800\strut
\end{minipage} & \begin{minipage}[t]{0.13\columnwidth}\raggedright\strut
\$2,171,300\strut
\end{minipage} & \begin{minipage}[t]{0.13\columnwidth}\raggedright\strut
\$2,274,700\strut
\end{minipage} & \begin{minipage}[t]{0.17\columnwidth}\centering\strut
59.3\%\strut
\end{minipage} & \begin{minipage}[t]{0.17\columnwidth}\centering\strut
44.1\%\strut
\end{minipage}\tabularnewline
\begin{minipage}[t]{0.07\columnwidth}\centering\strut
22\strut
\end{minipage} & \begin{minipage}[t]{0.12\columnwidth}\raggedright\strut
\$1,985,200\strut
\end{minipage} & \begin{minipage}[t]{0.13\columnwidth}\raggedright\strut
\$2,084,400\strut
\end{minipage} & \begin{minipage}[t]{0.13\columnwidth}\raggedright\strut
\$2,183,700\strut
\end{minipage} & \begin{minipage}[t]{0.17\columnwidth}\centering\strut
64.5\%\strut
\end{minipage} & \begin{minipage}[t]{0.17\columnwidth}\centering\strut
44.8\%\strut
\end{minipage}\tabularnewline
\begin{minipage}[t]{0.07\columnwidth}\centering\strut
23\strut
\end{minipage} & \begin{minipage}[t]{0.12\columnwidth}\raggedright\strut
\$1,905,900\strut
\end{minipage} & \begin{minipage}[t]{0.13\columnwidth}\raggedright\strut
\$2,001,200\strut
\end{minipage} & \begin{minipage}[t]{0.13\columnwidth}\raggedright\strut
\$2,096,400\strut
\end{minipage} & \begin{minipage}[t]{0.17\columnwidth}\centering\strut
69.7\%\strut
\end{minipage} & \begin{minipage}[t]{0.17\columnwidth}\centering\strut
45.5\%\strut
\end{minipage}\tabularnewline
\begin{minipage}[t]{0.07\columnwidth}\centering\strut
24\strut
\end{minipage} & \begin{minipage}[t]{0.12\columnwidth}\raggedright\strut
\$1,829,600\strut
\end{minipage} & \begin{minipage}[t]{0.13\columnwidth}\raggedright\strut
\$1,921,100\strut
\end{minipage} & \begin{minipage}[t]{0.13\columnwidth}\raggedright\strut
\$2,012,700\strut
\end{minipage} & \begin{minipage}[t]{0.17\columnwidth}\centering\strut
74.9\%\strut
\end{minipage} & \begin{minipage}[t]{0.17\columnwidth}\centering\strut
46.2\%\strut
\end{minipage}\tabularnewline
\begin{minipage}[t]{0.07\columnwidth}\centering\strut
25\strut
\end{minipage} & \begin{minipage}[t]{0.12\columnwidth}\raggedright\strut
\$1,756,400\strut
\end{minipage} & \begin{minipage}[t]{0.13\columnwidth}\raggedright\strut
\$1,844,200\strut
\end{minipage} & \begin{minipage}[t]{0.13\columnwidth}\raggedright\strut
\$1,932,100\strut
\end{minipage} & \begin{minipage}[t]{0.17\columnwidth}\centering\strut
80.1\%\strut
\end{minipage} & \begin{minipage}[t]{0.17\columnwidth}\centering\strut
46.9\%\strut
\end{minipage}\tabularnewline
\begin{minipage}[t]{0.07\columnwidth}\centering\strut
26\strut
\end{minipage} & \begin{minipage}[t]{0.12\columnwidth}\raggedright\strut
\$1,698,200\strut
\end{minipage} & \begin{minipage}[t]{0.13\columnwidth}\raggedright\strut
\$1,783,100\strut
\end{minipage} & \begin{minipage}[t]{0.13\columnwidth}\raggedright\strut
\$1,868,000\strut
\end{minipage} & \begin{minipage}[t]{0.17\columnwidth}\centering\strut
80.3\%\strut
\end{minipage} & \begin{minipage}[t]{0.17\columnwidth}\centering\strut
47.6\%\strut
\end{minipage}\tabularnewline
\begin{minipage}[t]{0.07\columnwidth}\centering\strut
27\strut
\end{minipage} & \begin{minipage}[t]{0.12\columnwidth}\raggedright\strut
\$1,649,100\strut
\end{minipage} & \begin{minipage}[t]{0.13\columnwidth}\raggedright\strut
\$1,731,600\strut
\end{minipage} & \begin{minipage}[t]{0.13\columnwidth}\raggedright\strut
\$1,814,100\strut
\end{minipage} & \begin{minipage}[t]{0.17\columnwidth}\centering\strut
80.4\%\strut
\end{minipage} & \begin{minipage}[t]{0.17\columnwidth}\centering\strut
48.3\%\strut
\end{minipage}\tabularnewline
\begin{minipage}[t]{0.07\columnwidth}\centering\strut
28\strut
\end{minipage} & \begin{minipage}[t]{0.12\columnwidth}\raggedright\strut
\$1,639,100\strut
\end{minipage} & \begin{minipage}[t]{0.13\columnwidth}\raggedright\strut
\$1,721,100\strut
\end{minipage} & \begin{minipage}[t]{0.13\columnwidth}\raggedright\strut
\$1,802,900\strut
\end{minipage} & \begin{minipage}[t]{0.17\columnwidth}\centering\strut
80.5\%\strut
\end{minipage} & \begin{minipage}[t]{0.17\columnwidth}\centering\strut
49.0\%\strut
\end{minipage}\tabularnewline
\begin{minipage}[t]{0.07\columnwidth}\centering\strut
29\strut
\end{minipage} & \begin{minipage}[t]{0.12\columnwidth}\raggedright\strut
\$1,627,100\strut
\end{minipage} & \begin{minipage}[t]{0.13\columnwidth}\raggedright\strut
\$1,708,500\strut
\end{minipage} & \begin{minipage}[t]{0.13\columnwidth}\raggedright\strut
\$1,789,800\strut
\end{minipage} & \begin{minipage}[t]{0.17\columnwidth}\centering\strut
80.5\%\strut
\end{minipage} & \begin{minipage}[t]{0.17\columnwidth}\centering\strut
50.0\%\strut
\end{minipage}\tabularnewline
\begin{minipage}[t]{0.07\columnwidth}\centering\strut
30\strut
\end{minipage} & \begin{minipage}[t]{0.12\columnwidth}\raggedright\strut
\$1,615,400\strut
\end{minipage} & \begin{minipage}[t]{0.13\columnwidth}\raggedright\strut
\$1,696,100\strut
\end{minipage} & \begin{minipage}[t]{0.13\columnwidth}\raggedright\strut
\$1,777,000\strut
\end{minipage} & \begin{minipage}[t]{0.17\columnwidth}\centering\strut
80.5\%\strut
\end{minipage} & \begin{minipage}[t]{0.17\columnwidth}\centering\strut
50.0\%\strut
\end{minipage}\tabularnewline
\bottomrule
\end{longtable}

\chapter{2017-18 MINIMUM ANNUAL SALARY
SCALE}\label{minimum-annual-salary-scale}

\begin{longtable}[]{@{}clllll@{}}
\toprule
Years of Service & Year 1 & Year 2 & Year 3 & Year 4 & Year
5\tabularnewline
\midrule
\endhead
0 & \$815,615 & & & &\tabularnewline
1 & \$1,312,611 & \$1,378,242 & & &\tabularnewline
2 & \$1,471,382 & \$1,544,951 & \$1,618,520 & &\tabularnewline
3 & \$1,524,305 & \$1,600,520 & \$1,676,735 & \$1,752,950
&\tabularnewline
4 & \$1,577,230 & \$1,656,092 & \$1,734,954 & \$1,813,816 &
\$1,892,678\tabularnewline
5 & \$1,709,538 & \$1,795,015 & \$1,880,492 & \$1,965,969 &
\$2,051,446\tabularnewline
6 & \$1,841,849 & \$1,933,941 & \$2,026,033 & \$2,118,125 &
\$2,210,217\tabularnewline
7 & \$1,974,159 & \$2,072,867 & \$2,171,575 & \$2,270,283 &
\$2,368,991\tabularnewline
8 & \$2,106,470 & \$2,211,794 & \$2,317,118 & \$2,422,442 &
\$2,527,766\tabularnewline
9 & \$2,116,955 & \$2,222,803 & \$2,328,651 & \$2,434,499 &
\$2,540,347\tabularnewline
10+ & \$2,328,652 & \$2,445,085 & \$2,561,518 & \$2,677,951 &
\$2,794,384\tabularnewline
\bottomrule
\end{longtable}

\chapter{BRI EXPENSE RATIOS}\label{bri-expense-ratios}

\section{Team and Related Party Expenses, Article VII, Section
1(a)(6)(v)}\label{team-and-related-party-expenses-article-vii-section-1a6v}

\begin{longtable}[]{@{}ll@{}}
\toprule
\textbf{Category} & \textbf{Ratio of Expenses to
Revenues}\tabularnewline
\midrule
\endhead
&\tabularnewline
Uniform Expense Cap & 9.5\%\tabularnewline
\bottomrule
\end{longtable}

\section{League Expenses, Article VII, Section
1(a)(1)(ix)}\label{league-expenses-article-vii-section-1a1ix}

\begin{longtable}[]{@{}ll@{}}
\toprule
\textbf{Category} & \textbf{Ratio of Expenses to
Revenues}\tabularnewline
\midrule
\endhead
&\tabularnewline
Sponsorships & 19\%\tabularnewline
NBA Entertainment & 35\%\tabularnewline
International Television & 22\%\tabularnewline
Special Events & 100\%\tabularnewline
\bottomrule
\end{longtable}

\chapter{NOTICE TO VETERAN PLAYERS CONCERNING SUMMER
LEAGUES}\label{notice-to-veteran-players-concerning-summer-leagues}

\begin{enumerate}
\def\labelenumi{\arabic{enumi}.}
\item
  Under the Uniform Player Contract and the Collective Bargaining
  Agreement between the NBA and the Players Association, the Team cannot
  require players to participate in any summer league.
\item
  The failure of a player to participate in a summer league will not, by
  itself, prejudice or disadvantage such player in his Team standing or
  relationship.
\item
  The Team reserves the right to determine how many and which players it
  may enroll in any summer league.
\end{enumerate}

We would appreciate your signing in the space provided below to
acknowledge that you have freely chosen to participate in summer league
play on a voluntary basis during the summer of \_\_\_\_.

\begin{longtable}[]{@{}lc@{}}
\toprule
Agreed to and Accepted: &\tabularnewline
\midrule
\endhead
\_\_\_\_\_\_\_\_\_\_\_\_\_\_\_\_\_\_\_\_\_ & \_\_\_\_\_\_\tabularnewline
(Name of Player) &\tabularnewline
\_\_\_\_\_\_\_\_\_\_\_\_\_\_\_\_\_\_\_\_\_ & \_\_\_\_\_\_\tabularnewline
(Date) &\tabularnewline
\bottomrule
\end{longtable}

\chapter{JOINT NBA/NBPA POLICY ON DOMESTIC VIOLENCE, SEXUAL ASSAULT, AND
CHILD
ABUSE}\label{joint-nbanbpa-policy-on-domestic-violence-sexual-assault-and-child-abuse}

\chaptermark{JOINT NBA/NBPA POLICY ON DOMESTIC VIOLENCE, SEXUAL ASSAUL \ldots}

Through this Policy, the National Basketball Association (``NBA'') and
the National Basketball Players Association (``NBPA'') (collectively,
``the Parties'') have agreed to work together to address domestic
violence, sexual assault, and child abuse in the NBA.

\textbf{Covered Behavior}

Acts that constitute domestic violence, sexual assault, and child abuse
are prohibited at all times and regardless of where they occur.

For purposes of this Policy, ``domestic violence'' includes, but is not
limited to, any actual or attempted violent act that is committed by one
party in an intimate or family relationship against another party in
that relationship. Such an act may include physical assault or battery,
sexual assault, stalking, harassment, or other forms of physical or
psychological abuse. It may also include behavior that intimidates,
manipulates, humiliates, isolates, frightens, terrorizes, coerces,
threatens, injures, or places another person in fear of bodily harm.
Domestic violence can be perpetrated by current or former spouses,
current or former domestic or same sex partners, persons who are living
together or have cohabitated, persons with children in common, persons
who have or had an intimate or dating relationship, and family members.
Domestic violence can be a single act or a pattern of behavior in a
relationship.

For purposes of this Policy, ``sexual assault'' includes, but is not
limited to, any actual or attempted sexual contact or act to which one
party has not consented. Lack of consent is deemed to exist when a
person uses or threatens the use of force, harassment, or any other form
of coercion against another. Lack of consent is also deemed to exist
when a person is mentally incapable of giving consent, as a result of
disability, incapacitation, intoxication, or otherwise.

For purposes of this Policy, ``child abuse'' includes, but is not
limited to, any act or failure to act by a parent, caregiver, or adult
that results in death, serious physical or emotional harm, or sexual or
other exploitation of a child. Child abuse also includes behavior that
poses an imminent risk of such harm to a child.

\textbf{Policy Committee}

The Parties shall establish a joint committee to provide education,
support, treatment, referrals, counseling, and other resources for
players, their family members, and others at risk (the ``Policy
Committee''). The Policy Committee will be comprised of two
representatives from the NBA and two representatives from the NBPA (the
``Party Representatives''), as well as three independent experts with
experience in domestic violence, sexual assault, and/or child abuse (the
``Expert Representatives''). All decisions of the Policy Committee shall
be made by a majority vote, unless otherwise stated in this Policy, and
shall be final, binding, and unappealable.

The Party Representatives shall jointly select the three Expert
Representatives to serve on the Policy Committee within 60 days of the
issuance of this Policy. There shall be at least one Expert
Representative on the Policy Committee at all times with specific
expertise in each of the three subject areas (i.e., domestic violence,
sexual assault, and child abuse). The Expert Representatives will each
serve for the duration of this Policy; provided, however, that either
the NBA or the NBPA may discharge any of them on an annual basis by
serving written notice upon the Expert Representative(s) and upon the
other Party within 60 days of the anniversary of the appointment of such
person. If an Expert Representative is discharged, the Party
Representatives shall jointly select a successor Expert Representative
within 30 days of the notice of discharge.

In the event that the Party Representatives are unable to agree upon and
jointly select any or all of the Expert Representatives within 60 days
of the issuance of this Policy or within 30 days of the notice of any
discharge of an Expert Representative, the following process will be
implemented. Within five days following the deadline to select the
Expert Representative(s), the Party Representatives shall exchange lists
containing the names and qualifications of three proposed Expert
Representatives per open position. Within five days following the
exchange of such lists, the Party Representatives shall jointly select
from that group of individuals the Expert Representative(s) needed to
serve on the Policy Committee. If they are unable to do so, then, within
an additional three-day period, the Party Representatives shall engage
in a process of alternatively striking names from the lists until one
name remains for each open position, and such person(s) shall be
appointed as the Expert Representative(s).

\textbf{Training and Education}

The Parties seek to prevent incidents of domestic violence, sexual
assault, and child abuse from occurring through educational programs and
awareness training.

The Policy Committee will implement and oversee all training and
educational programs for NBA players that address issues of domestic
violence, sexual assault, and child abuse, and shall make all
determinations related thereto including, but not limited to, the
staffing, content, format, and frequency of such programs. The Policy
Committee will annually review such programs to ensure that they are
effective and that the content is appropriate, thorough, and properly
communicated to the players.

\textbf{Hotline}

Within 60 days of the issuance of this Policy, the Parties shall jointly
select a service provider to support a 24-hour, confidential hotline
that can be used by players, their families, and other victims of
domestic violence, sexual assault, and child abuse as defined by this
Policy to seek assistance and referrals (the ``Service Provider'').

If the Parties are unable to do so, then, within five days following the
deadline to select the Service Provider, they shall exchange lists
containing the names, qualifications, and cost of three proposed Service
Providers. Within five days following the exchange of such lists, the
Parties shall jointly select the Service Provider. If the Parties are
unable to do so, then, within an additional three-day period, they shall
engage in a process of alternatively striking names from the lists until
one name remains, and such organization shall be appointed as the
Service Provider.

\textbf{Treatment and Intervention}

\begin{enumerate}
\def\labelenumi{\arabic{enumi}.}
\item
  \textbf{General}

  The NBA or the NBPA may refer a player to the Policy Committee in any
  of the following circumstances:

  \begin{enumerate}
  \def\labelenumii{\alph{enumii}.}
  \tightlist
  \item
    As part of a disciplinary determination of the Commissioner for
    conduct in violation of this Policy; or
  \item
    After a Player is criminally convicted of an offense that involves
    conduct in violation of this Policy.
  \end{enumerate}

  The Policy Committee will also be available as a resource to any
  player who voluntarily seeks assistance.

  Once a player has been referred to the Policy Committee, an expert
  selected by the Policy Committee will conduct an initial evaluation of
  the player as soon as is practicable. Following such evaluation, the
  Policy Committee will develop a Treatment and Accountability Plan
  (``TAP'') for the player, as may be appropriate. As part of the TAP,
  the Policy Committee may require that the player submit to
  psychological or other evaluations and/or attend counseling sessions
  with a licensed professional, and take other steps that it deems
  necessary. In developing the TAP, the Policy Committee will take into
  account any treatment or counseling that the player may have initiated
  on his own or pursuant to a criminal resolution of any charges against
  him.

  The Policy Committee will oversee the player's compliance with any
  TAP, and shall provide additional support to the player as needed. Any
  treating professionals shall provide regular, written status reports
  to the Policy Committee that detail the player's progress and
  compliance with the TAP. The Policy Committee may periodically revise,
  modify, extend, or close the TAP on its own initiative, on the
  recommendation of the player's treating professional(s), or upon
  petition of the player. All information related to a player's
  involvement with the Policy Committee shall be kept confidential.

  The Policy Committee shall determine whether the player has
  successfully completed his TAP, and may also issue a revised TAP at
  any time. A player must receive a certification of completion from the
  Policy Committee in order to conclude his treatment and the oversight
  of the Policy Committee.
\item
  \textbf{Non-Compliance}

  Players are required to comply with the directives of the Policy
  Committee, including with his TAP. If the Policy Committee determines
  that a player has failed to comply without a reasonable explanation,
  it shall notify the NBA. For the first such instance of
  non-compliance, the NBA shall issue a warning to the player. If such
  non-compliance continues for three additional days after the warning
  is issued, or for the second or any additional instances of
  non-compliance as determined by the Policy Committee, the NBA shall
  fine the player in the amount of \$10,000 for each day that he fails
  to comply. Such fines shall continue until the player has, in the
  judgment of the Policy Committee, resumed full compliance.

  If the Policy Committee determines that a player has demonstrated
  substantial non-compliance, without a reasonable explanation, through
  a pattern of behavior that demonstrates a mindful disregard for his
  treatment responsibilities, it shall notify the NBA, which shall
  thereupon impose:

  \begin{enumerate}
  \def\labelenumii{\alph{enumii}.}
  \tightlist
  \item
    A one-game suspension for the first instance of substantial
    non-compliance; and
  \item
    A suspension that is at least one game longer than his
    immediately-preceding suspension for each additional instance of
    substantial non-compliance and that shall continue until, in the
    judgment of the Policy Committee, the player resumes full compliance
    with its directives, including with his TAP.
  \end{enumerate}
\end{enumerate}

\textbf{Costs}

Any and all costs of the training, education, treatment, intervention,
and other resources described above including, but not limited to, the
Policy Committee, Expert Representatives, education and training
programs, hotline, experts, and counselors, will be shared equally by
the Parties (unless otherwise covered by the NBA Players Group Health
Plan or other insurance plan provided to NBA players). The NBPA's share
shall be paid by the NBA and included in Player Benefits under Article
IV, Section 6 of the CBA. The NBA's share will be excluded from the
calculation of Benefits under the CBA.

\textbf{Investigation of Incidents}

The NBA will give the NBPA and the player prompt notice of the
commencement of any investigation into an alleged violation of this
Policy.

The NBA's investigation may include the use of third party resources
including, but not limited to, outside legal counsel, outside
investigators, or other individuals with relevant experience or
expertise.

The NBA will notify the NBPA when it has concluded its investigation and
report whether it believes a violation of the Policy has occurred.

\textbf{Cooperation}

Except in circumstances where the player has a reasonable apprehension
of criminal prosecution, players shall cooperate fully with any NBA
investigation under this Policy. Any player interviewed by the NBA as
part of its investigation is entitled to have a representative from the
NBPA present during the interview, and the NBA will provide the NBPA
with at least 48 hours' notice before any in-person interview.

Failing to cooperate in full, or interfering in any manner, with an NBA
investigation will subject the non-cooperative individual to discipline
consistent with the terms of Article VI, Section 11(a) of the CBA. It
may constitute a violation of this cooperation requirement for a player
to attempt to or enter into any agreement with a witness, victim, or
other party that would discourage or prevent that individual from
cooperating with an NBA investigation. However, the player is under no
obligation to demand, request or otherwise encourage anyone to cooperate
with an NBA investigation.

\textbf{Administrative Leave}

While an investigation is pending, the Commissioner may at any time
place the player on administrative leave with pay for a reasonable
period of time. The parties agree that administrative leave is not
intended to be routinely applied during the pendency of every player
investigation under this Policy. Instead, administrative leave should be
applied in only those cases in which a balancing of all relevant factors
clearly establishes that it is reasonable to do so under the totality of
the circumstances.

In deciding whether to place a player on paid administrative leave, the
Commissioner shall consider among other relevant factors the following
non-exhaustive list of factors:

\begin{itemize}
\tightlist
\item
  The nature and severity of the allegation(s), including whether a
  weapon was involved and whether any injury was suffered by anyone
  (including the player);
\item
  Whether the allegations are supported by credible information;
\item
  The relationship between the player and accuser;
\item
  Information regarding the player's history of prior similar conduct,
  or lack thereof;
\item
  The prior criminal or disciplinary history of the player, or lack
  thereof;
\item
  The status of any criminal investigation and/or prosecution regarding
  the alleged incident, including whether any arrests have been made;
\item
  The character of the player;
\item
  The player's reputation within the NBA community;
\item
  The NBA's past practice regarding discipline imposed on a player for
  similar allegations; and
\item
  The risk of reputational damage to the NBA and/or the player's team.
\end{itemize}

The NBA will give prompt notice to the NBPA, the player's team, and the
player of any decision to place a player on paid administrative leave
pursuant to this Policy. The decision to place the player on paid
administrative leave pending an investigation shall not preclude further
disciplinary action by the Commissioner against the player in accordance
with the provisions of this Policy.

While on administrative leave, the player shall be ineligible to play in
any of his team's games. However, the player will continue to receive
his salary and other welfare benefits to which he would be entitled as
an active player. The player and the player's team may also request that
the player be allowed to participate in non-public practices, workouts,
or other team activities with the consent of the NBA, which shall not be
unreasonably withheld.

A player may challenge the decision to be placed on paid administrative
leave under the Grievance and Arbitration Procedure of the CBA. In
evaluating such a challenge, the Grievance Arbitrator will determine
whether it was reasonable for the Commissioner to place the player on
administrative leave. A player may also request the Grievance Arbitrator
review the length of a period of administrative leave that exceeds seven
days. In such a proceeding, the Grievance Arbitrator will determine
whether administrative leave in excess of seven days is reasonable based
on the totality of the circumstances. Once a player challenges the
decision to be placed on paid administrative leave, or the duration of
such leave, the hearing before the Grievance Arbitrator must take place
within 72 hours.

\textbf{Discipline}

Based on a finding of just cause, the Commissioner may fine, suspend, or
dismiss and disqualify from any further association with the NBA and its
teams a player who engages in prohibited conduct in violation of this
Policy. Repeat offenders will be subject to enhanced discipline.

Notwithstanding the foregoing, an admission to, or conviction for, any
offense that involves conduct that violates this Policy, whether after
trial or upon a plea of guilty, as well as any plea of no contest or
nolo contendere, will conclusively establish a violation of this Policy.
A violation based on this ground, however, shall in no way limit or
prevent the NBA from continuing to investigate the incident.
Additionally, such admission, conviction, or plea is not required in
order for a Policy violation to have occurred. However, a player who is
acquitted after trial in a criminal proceeding may not be subject to
disciplinary penalties under this Policy.

In conjunction with any discipline imposed by the Commissioner for a
violation of this Policy, the NBA may also require the player to undergo
an evaluation under the supervision of the Policy Committee, to
participate in relevant training, education, or counseling programs as
determined by the Policy Committee, and/or to perform community service.
Any discipline determined by the Commissioner may be referred to the
player's team for imposition.

Prior to the determination of any discipline, the Parties shall meet to
discuss the matter. This conference shall be considered confidential,
and no statements made during the discussion shall be admissible in any
subsequent challenge to any discipline imposed on the player.

The Commissioner will determine all discipline under this Policy on a
case-by-case basis, upon consideration of all facts and circumstances,
including aggravating and mitigating factors.

Potential aggravating factors include, but are not limited to:

\begin{itemize}
\tightlist
\item
  Prior allegations of, or convictions for, prohibited conduct;
\item
  The use of a weapon or other means of coercion;
\item
  The use of, or threat to use, force or violence;
\item
  The vulnerability of the victim;
\item
  The presence of a minor;
\item
  The nature and extent of any injury to the victim; and
\item
  A civil verdict against the player for the underlying conduct.
\end{itemize}

Potential mitigating factors include, but are not limited to:

\begin{itemize}
\tightlist
\item
  Acceptance of responsibility;
\item
  Evidence of self-defense;
\item
  Complete and truthful cooperation with the investigation;
\item
  Voluntary participation in any treatment or counseling programs;
\item
  The player's overall good character;
\item
  The player's reputation in the NBA community; and
\item
  A civil verdict in favor of the player for the underlying conduct.
\end{itemize}

In cases where the Commissioner imposes a suspension, any period of time
the player spent on paid administrative leave will be credited toward
the suspension provided that the player remits to the League the
applicable portion of salary that the player received while on paid
administrative leave.

Challenges to any disciplinary action shall be made through the
Grievance Arbitration process of the CBA.

\textbf{Confidentiality}

The Parties recognize the importance of confidentiality and privacy to
the success of this Policy. Accordingly, the Parties will maintain
confidentiality throughout the investigatory, disciplinary, and
treatment process, and will take reasonable measures to protect the
information gathered pursuant to this Policy, including by any outside
advisors or experts. Any medical information obtained during the
investigatory, disciplinary, and treatment process will be kept
confidential as required by applicable law.

At the same time, the Parties recognize that disclosure of certain
information may be necessary to further the NBA's investigation or may
be required by law, including by court order or subpoena. Accordingly,
the Parties cannot and do not guarantee that complete confidentiality
will be maintained. The Parties also reserve the right to make
notifications to law enforcement or other appropriate authorities if
either the NBA or the NBPA becomes aware that there is a threat of
imminent harm to any individual or in cases where the victim is a child
or is either mentally or physically incapacitated. Additionally, in
matters where a violation is found and discipline is imposed, such
findings and discipline may be the subject of public statements by the
NBA and/or the NBPA.

\textbf{Retaliation}

Under this Policy, it is prohibited to retaliate, or threaten to
retaliate, against any individual who, in good faith, reports a
potential violation of this Policy or who honestly participates in an
investigation of such a report. It does not matter whether the
investigation establishes that a violation of the Policy occurred, as
long as the report of the violation or participation in the
investigation is in good faith. Such retaliation includes, but is not
limited to, threats, intimidation, harassment, and any adverse
employment or other action, whether express or implied. Anyone who
retaliates, or threatens to retaliate, against an individual who
reports, or participates in an investigation into, an alleged violation
of this Policy, or against any victim or other witness, will be subject
to independent disciplinary action.

As with any complaint brought in bad faith, any individual, including
coaches, general managers, or other team officials, who reports a
violation of this Policy knowing such claim is malicious, false, or
fundamentally frivolous shall be subject to disciplinary action.

\textbf{Reporting}

Anyone who is the victim of or acting on behalf of a victim of domestic
violence, sexual assault, or child abuse, as defined by this Policy, is
strongly encouraged to call the hotline established under this Policy as
soon as possible after the incident to discuss the availability of
counseling, treatment, security, and other appropriate resources.

If you are in immediate danger or involved in a situation in which
another person is in immediate danger, the Parties recommend that you
contact 911 or your local police department. Support and crisis
intervention is also available from the National Domestic Violence
Hotline at 1-800-799-SAFE (7233).

\chapter{OFFER SHEET}\label{offer-sheet}

\begin{longtable}[]{@{}ll@{}}
\toprule
\begin{minipage}[t]{0.39\columnwidth}\raggedright\strut
Name of Player:\strut
\end{minipage} & \begin{minipage}[t]{0.39\columnwidth}\raggedright\strut
Date:\strut
\end{minipage}\tabularnewline
\begin{minipage}[t]{0.39\columnwidth}\raggedright\strut
\_\_\_\_\_\_\_\_\_\_\_\_\_\_\_\_\_\_\_\_\_\_\_\_\_\strut
\end{minipage} & \begin{minipage}[t]{0.39\columnwidth}\raggedright\strut
\_\_\_\_\_\_\_\_\_\_\_\_\_\_\_\_\_\_\_\_\_\_\_\_\_\strut
\end{minipage}\tabularnewline
\begin{minipage}[t]{0.39\columnwidth}\raggedright\strut
Address of Player and Email Address of Player:\strut
\end{minipage} & \begin{minipage}[t]{0.39\columnwidth}\raggedright\strut
Name of New Team:\strut
\end{minipage}\tabularnewline
\begin{minipage}[t]{0.39\columnwidth}\raggedright\strut
\_\_\_\_\_\_\_\_\_\_\_\_\_\_\_\_\_\_\_\_\_\_\_\_\_\strut
\end{minipage} & \begin{minipage}[t]{0.39\columnwidth}\raggedright\strut
\strut
\end{minipage}\tabularnewline
\begin{minipage}[t]{0.39\columnwidth}\raggedright\strut
\_\_\_\_\_\_\_\_\_\_\_\_\_\_\_\_\_\_\_\_\_\_\_\_\_\strut
\end{minipage} & \begin{minipage}[t]{0.39\columnwidth}\raggedright\strut
\strut
\end{minipage}\tabularnewline
\begin{minipage}[t]{0.39\columnwidth}\raggedright\strut
\_\_\_\_\_\_\_\_\_\_\_\_\_\_\_\_\_\_\_\_\_\_\_\_\_\strut
\end{minipage} & \begin{minipage}[t]{0.39\columnwidth}\raggedright\strut
\_\_\_\_\_\_\_\_\_\_\_\_\_\_\_\_\_\_\_\_\_\_\_\_\_\strut
\end{minipage}\tabularnewline
\begin{minipage}[t]{0.39\columnwidth}\raggedright\strut
Name, Address and Email Address of Player's Representative Authorized to
Act for Player:\strut
\end{minipage} & \begin{minipage}[t]{0.39\columnwidth}\raggedright\strut
Name of ROFR Team:\strut
\end{minipage}\tabularnewline
\begin{minipage}[t]{0.39\columnwidth}\raggedright\strut
\_\_\_\_\_\_\_\_\_\_\_\_\_\_\_\_\_\_\_\_\_\_\_\_\_\strut
\end{minipage} & \begin{minipage}[t]{0.39\columnwidth}\raggedright\strut
\_\_\_\_\_\_\_\_\_\_\_\_\_\_\_\_\_\_\_\_\_\_\_\_\_\strut
\end{minipage}\tabularnewline
\begin{minipage}[t]{0.39\columnwidth}\raggedright\strut
\_\_\_\_\_\_\_\_\_\_\_\_\_\_\_\_\_\_\_\_\_\_\_\_\_\strut
\end{minipage} & \begin{minipage}[t]{0.39\columnwidth}\raggedright\strut
Address of ROFR Team:\strut
\end{minipage}\tabularnewline
\begin{minipage}[t]{0.39\columnwidth}\raggedright\strut
\_\_\_\_\_\_\_\_\_\_\_\_\_\_\_\_\_\_\_\_\_\_\_\_\_\strut
\end{minipage} & \begin{minipage}[t]{0.39\columnwidth}\raggedright\strut
\_\_\_\_\_\_\_\_\_\_\_\_\_\_\_\_\_\_\_\_\_\_\_\_\_\strut
\end{minipage}\tabularnewline
\begin{minipage}[t]{0.39\columnwidth}\raggedright\strut
\_\_\_\_\_\_\_\_\_\_\_\_\_\_\_\_\_\_\_\_\_\_\_\_\_\strut
\end{minipage} & \begin{minipage}[t]{0.39\columnwidth}\raggedright\strut
\_\_\_\_\_\_\_\_\_\_\_\_\_\_\_\_\_\_\_\_\_\_\_\_\_\strut
\end{minipage}\tabularnewline
\bottomrule
\end{longtable}

Attached hereto is an unsigned Player Contract that the New Team has
offered to the Player and that the Player desires to accept. The
attached Player Contract separately specifies in its exhibits those
Principal Terms that will be included in the Player Contract with the
ROFR Team if that Team gives the Player a timely First Refusal Exercise
Notice.

\begin{longtable}[]{@{}ll@{}}
\toprule
Player: & New Team:\tabularnewline
&\tabularnewline
By \_\_\_\_\_\_\_\_\_\_\_\_\_\_\_\_\_\_\_\_\_\_\_\_\_ & By
\_\_\_\_\_\_\_\_\_\_\_\_\_\_\_\_\_\_\_\_\_\_\_\_\_\tabularnewline
\bottomrule
\end{longtable}

\chapter{FIRST REFUSAL EXERCISE
NOTICE}\label{first-refusal-exercise-notice}

\begin{longtable}[]{@{}ll@{}}
\toprule
\begin{minipage}[t]{0.39\columnwidth}\raggedright\strut
Name of Player:\strut
\end{minipage} & \begin{minipage}[t]{0.39\columnwidth}\raggedright\strut
Date:\strut
\end{minipage}\tabularnewline
\begin{minipage}[t]{0.39\columnwidth}\raggedright\strut
\_\_\_\_\_\_\_\_\_\_\_\_\_\_\_\_\_\_\_\_\_\_\_\_\_\strut
\end{minipage} & \begin{minipage}[t]{0.39\columnwidth}\raggedright\strut
\_\_\_\_\_\_\_\_\_\_\_\_\_\_\_\_\_\_\_\_\_\_\_\_\_\strut
\end{minipage}\tabularnewline
\begin{minipage}[t]{0.39\columnwidth}\raggedright\strut
Address of Player:\strut
\end{minipage} & \begin{minipage}[t]{0.39\columnwidth}\raggedright\strut
Name of New Team:\strut
\end{minipage}\tabularnewline
\begin{minipage}[t]{0.39\columnwidth}\raggedright\strut
\_\_\_\_\_\_\_\_\_\_\_\_\_\_\_\_\_\_\_\_\_\_\_\_\_\strut
\end{minipage} & \begin{minipage}[t]{0.39\columnwidth}\raggedright\strut
\strut
\end{minipage}\tabularnewline
\begin{minipage}[t]{0.39\columnwidth}\raggedright\strut
\_\_\_\_\_\_\_\_\_\_\_\_\_\_\_\_\_\_\_\_\_\_\_\_\_\strut
\end{minipage} & \begin{minipage}[t]{0.39\columnwidth}\raggedright\strut
\strut
\end{minipage}\tabularnewline
\begin{minipage}[t]{0.39\columnwidth}\raggedright\strut
\_\_\_\_\_\_\_\_\_\_\_\_\_\_\_\_\_\_\_\_\_\_\_\_\_\strut
\end{minipage} & \begin{minipage}[t]{0.39\columnwidth}\raggedright\strut
\_\_\_\_\_\_\_\_\_\_\_\_\_\_\_\_\_\_\_\_\_\_\_\_\_\strut
\end{minipage}\tabularnewline
\begin{minipage}[t]{0.39\columnwidth}\raggedright\strut
Name and Address of Player's Representative Authorized to Act for
Player\strut
\end{minipage} & \begin{minipage}[t]{0.39\columnwidth}\raggedright\strut
Name of ROFR Team:\strut
\end{minipage}\tabularnewline
\begin{minipage}[t]{0.39\columnwidth}\raggedright\strut
\_\_\_\_\_\_\_\_\_\_\_\_\_\_\_\_\_\_\_\_\_\_\_\_\_\strut
\end{minipage} & \begin{minipage}[t]{0.39\columnwidth}\raggedright\strut
\_\_\_\_\_\_\_\_\_\_\_\_\_\_\_\_\_\_\_\_\_\_\_\_\_\strut
\end{minipage}\tabularnewline
\begin{minipage}[t]{0.39\columnwidth}\raggedright\strut
\_\_\_\_\_\_\_\_\_\_\_\_\_\_\_\_\_\_\_\_\_\_\_\_\_\strut
\end{minipage} & \begin{minipage}[t]{0.39\columnwidth}\raggedright\strut
Address of ROFR Team:\strut
\end{minipage}\tabularnewline
\begin{minipage}[t]{0.39\columnwidth}\raggedright\strut
\_\_\_\_\_\_\_\_\_\_\_\_\_\_\_\_\_\_\_\_\_\_\_\_\_\strut
\end{minipage} & \begin{minipage}[t]{0.39\columnwidth}\raggedright\strut
\_\_\_\_\_\_\_\_\_\_\_\_\_\_\_\_\_\_\_\_\_\_\_\_\_\strut
\end{minipage}\tabularnewline
\begin{minipage}[t]{0.39\columnwidth}\raggedright\strut
\_\_\_\_\_\_\_\_\_\_\_\_\_\_\_\_\_\_\_\_\_\_\_\_\_\strut
\end{minipage} & \begin{minipage}[t]{0.39\columnwidth}\raggedright\strut
\_\_\_\_\_\_\_\_\_\_\_\_\_\_\_\_\_\_\_\_\_\_\_\_\_\strut
\end{minipage}\tabularnewline
\bottomrule
\end{longtable}

The undersigned member of the NBA hereby exercises its Right of First
Refusal so as to create a binding agreement with the Player containing
the Principal Terms set forth in the Player Contract annexed to the
Player's Offer Sheet (a copy of which is attached hereto).

\begin{longtable}[]{@{}l@{}}
\toprule
ROFR Team:\tabularnewline
By \_\_\_\_\_\_\_\_\_\_\_\_\_\_\_\_\_\_\_\_\_\tabularnewline
\bottomrule
\end{longtable}

\chapter{}\label{section-1}

\section{AUTHORIZATION FOR TESTING}\label{authorization-for-testing}

\begin{longtable}[]{@{}lc@{}}
\toprule
To: & \_\_\_\_\_\_\_\_\_\_\_\_\_\_\_\_\_\_\_\tabularnewline
&\tabularnewline
Player & \_\_\_\_\_\_\_\_\_\_\_\_\_\_\_\_\_\_\_\tabularnewline
\bottomrule
\end{longtable}

Please be advised that on
\_\_\_\_\_\_\_\_\_\_\_\_\_\_\_\_\_\_\_\_\_\_\_\_\_\_, you were the
subject of a meeting or conference call held pursuant to the Anti-Drug
Program set forth in Article XXXIII of the Collective Bargaining
Agreement between the NBA and the National Basketball Players
Association, dated January 19, 2017, (the ``Agreement''). Following the
meeting or conference call, I authorized the NBA to conduct the testing
procedures set forth in the Agreement, and you are hereby directed to
submit to those testing procedures, on demand, no more than four (4)
times during the next six (6) weeks.

Please be advised that your failure to submit to these procedures may
result in substantial penalties, including but not limited to your
dismissal and disqualification from the NBA.

\begin{longtable}[]{@{}l@{}}
\toprule
\_\_\_\_\_\_\_\_\_\_\_\_\_\_\_\_\_\_\_\_\_\_\_\_\tabularnewline
Independent Expert\tabularnewline
Dated:\tabularnewline
\_\_\_\_\_\_\_\_\_\_\_\_\_\_\_\_\_\_\_\_\_\_\_\_\tabularnewline
\bottomrule
\end{longtable}

\section{PROHIBITED SUBSTANCES}\label{prohibited-substances}

A. Drugs of Abuse

\begin{itemize}
\tightlist
\item
  Benzodiazepines:

  \begin{itemize}
  \tightlist
  \item
    Alprazolam (also called Xanax or Niravam)
  \item
    Chlordiazepoxide (also called Librium, Mitran, Poxi or H-Tran)
  \item
    Clonazepam (also called Klonopin, Ceberclon or Valpaz)
  \item
    Diazepam (also called Valium)
  \item
    Lorazepam (also called Ativan)
  \end{itemize}
\item
  Synthetic Cathinones

  \begin{itemize}
  \tightlist
  \item
    4-methyl-N-ethylcathinone (also called 4-MEC)
  \item
    4-methyl-alpha-pyrrolidinopropiophenone (also called 4-MePPP)
  \item
    Alpha-pyrrolidinopentiophenone (also called alpha-PVP)
  \item
    1-(1,3-benzodioxol-5-yl)-2-(methylamino)butan-1-one (also called
    butylone)
  \item
    2-(methylamino)-1-phenylpentan-1-one (also called pentedrone)
  \item
    1-(1,3-benzodioxol-5-yl)-2-(methylamino)pentan-1-one (also called
    pentylone)
  \item
    4-fluoro-N-methylcathinone (also called 4-FMC)
  \item
    3-fluoro-N-methylcathinone (also called 3-FMC)
  \item
    1-(naphthalen-2-yl)-2-(pyrrolidin-1-yl)pentan-1-one (also called
    naphyrone)
  \item
    alpha-pyrrolidinobutiophenone (also called alpha-PBP)
  \end{itemize}
\item
  Cocaine
\item
  Gamma Hydroxybutyrate (GHB)
\item
  Ketamine
\item
  LSD
\item
  Methamphetamine, MDMA, MDA and MDEA
\item
  Opiates:

  \begin{itemize}
  \tightlist
  \item
    Heroin
  \item
    Codeine
  \item
    Morphine
  \item
    Oxycodone (also called Oxycontin, Percocet, Percodan, Roxicet,
    Tylox, Dazidox, Endocet or Endodan)
  \item
    Hydrocodone (also called Vicodin, Lorcet, Lortab, Hydocan or Norco)
  \item
    Methadone (also called Methadose or Dolophine)
  \item
    Hydromorphone (also called Dilaudid)
  \item
    Fentanyl (also called Actiq or Duragesic)
  \item
    Propoxyphene (also called Darvon or Darvocet)
  \item
    Phencyclidine (PCP)
  \end{itemize}
\end{itemize}

B. Marijuana Marijuana and its By-Products

\begin{itemize}
\tightlist
\item
  Synthetic Cannabinoids:

  \begin{itemize}
  \tightlist
  \item
    5-(1,1-dimethylheptyl)-2-{[}(1R,3S)-3-hydroxycyclohexyl{]}-phenol
    (also called CP-47,497)
  \item
    5-(1,1-dimethyloctyl)-2-{[}(1R,3S)-3-hydroxycyclohexyl{]}-phenol
    (also called cannabicyclohexanol or CP-47,497 C8-homolog)
  \item
    1-pentyl-3-(1-naphthoyl)indole (also called JWH-018 or AM678)
  \item
    1-butyl-3-(1-naphthoyl)indole (also called JWH-073)
  \item
    1-hexyl-3-(1-naphthoyl)indole (also called JWH-019)
  \item
    1-{[}2-(4-morpholinyl)ethyl{]}-3-(1-naphthoyl)indole (JWH-200)
  \item
    1-pentyl-3-(2-methoxyphenylacetyl)indole (JWH-250)
  \item
    1-pentyl-3-{[}1-(4-methoxynaphthoyl){]}indole (JWH-081)
  \item
    1-pentyl-3-(4-methyl-1-naphthoyl)indole (JWH-122)
  \item
    1-pentyl-3-(4-chloro-1-naphthoyl)indole (JWH-398)
  \item
    1-(5-fluoropentyl)-3-(1-naphthoyl)indole (also called AM2201)
  \item
    1-(5-fluoropentyl)-3-(2-iodobenzoyl)indole (also called AM694)
  \item
    1-pentyl-3-{[}(4-methoxy)-benzoyl{]}indole (also called SR-19 or
    RCS-4)
  \item
    1-cyclohexylethyl-3-(2-methoxyphenylacetyl)indole (also called SR-18
    or RCS-8)
  \item
    1-pentyl-3-(2-chlorophenylacetyl)indole (also called JWH-203)
  \end{itemize}
\end{itemize}

C. Steroids and Performance Enhancing Drugs (SPEDs)

\begin{itemize}
\tightlist
\item
  Adrafinil
\item
  Alexamorelin
\item
  Aminoglutethimide
\item
  Amiphenazole
\item
  Amphetamine and its analogs (with the exceptions of Methamphetamine,
  MDMA, MDA and MDEA)
\item
  Anamorelin
\item
  Anastrozole
\item
  Androsta-1,4,6-triene-3,17-dione (also called Androstatrienedione or
  ATD)
\item
  Androsta-3, 5-diene-7, 17-dione (also called Arimistane)
\item
  Androst-2-en-17-one (also called 2-Androstenone and Delta-2)
\item
  Androstanediol
\item
  Androstanedione
\item
  Androstenediol
\item
  Androstenedione
\item
  Androstene-3,6,17-trione (also called 6-OXO or 4-AT)
\item
  Bolasterone
\item
  Boldenone
\item
  Boldione
\item
  Bromantan

  \begin{itemize}
  \tightlist
  \item
    6-bromo-androstan-3,17-dione (also called 6-Bromo)
  \item
    6-bromo-androsta-1,4-diene,3,17-dione (also called Aromadrol)
  \end{itemize}
\item
  Calusterone

  \begin{itemize}
  \tightlist
  \item
    4-chloro-17a-methyl-androsta-1,4-diene-3,17b-diol (also called
    Halodrol, Halovar and Helladrol)
  \item
    4-chloro-17a-methyl-androst-4-ene-3b,17b-diol (also called P-Mag and
    Promagnon)
  \item
    4-chloro-17a-methyl-17b-hydroxyandrost-4-ene-3-one (also called
    Mechabol)
  \item
    4-chloro-17a-methyl-17b-hydroxyandrost-4-ene-3,11-dione (also called
    Oxyguno)
  \end{itemize}
\item
  Clenbuterol
\item
  Clobenzorex
\item
  Clomiphene
\item
  Clostebol
\item
  Cyclofenil
\item
  Danazol
\item
  Dehydrochloromethyltestosterone (also called DHCMT and Turinabol)
\item
  Dehydroepiandrosterone (DHEA)
\item
  Desoxymethyltestosterone (DMT)
\item
  Dihydrotestosterone

  \begin{itemize}
  \tightlist
  \item
    4-dihydrotestosterone
  \item
    1, 3-dimethylamylamine (also called DMAA, Methylhexaneamine and
    Dimethylpentylamine)
  \item
    2a,17a-dimethyl-17b-hydroxy-5bandrostan-3-one (also called
    Superdrol)
  \end{itemize}
\item
  Dromostanolone
\item
  Drostanolone
\item
  Ephedra (also called Ma Huang Bishop's Tea and Chi Powder)
\item
  Ephedrine
\item
  Epitestosterone

  \begin{itemize}
  \tightlist
  \item
    2a,3a-epithio-17a-methyl-5aandrostan-17b-ol (also called Epistane
    and Havoc)
  \end{itemize}
\item
  Erythropoietin (EPO)
\item
  Estra-4,9,11-triene, 17-dione (also called Tren, Trenavar, Trendione
  and Trenazone)

  \begin{itemize}
  \tightlist
  \item
    13a-ethyl-17a-hydroxygon-4-en-3-one
  \end{itemize}
\item
  Ethylestrenol
\item
  Etilefrine
\item
  Exemestane
\item
  Fencamfamin
\item
  Fenethylline
\item
  Fenfluramine
\item
  Fenproporex
\item
  Fluoxymesterone
\item
  Formebolone
\item
  Formestane (also called 4-hydroxyandrostenedione)
\item
  Fulvestrant
\item
  Furazabol

  \begin{itemize}
  \tightlist
  \item
    {[}3,2-c{]}-furazan-5a-androstan-17b-ol (also called Furazan or
    Furuza)
  \end{itemize}
\item
  Gestrinone
\item
  Ghrelin
\item
  Growth Hormone Releasing Peptide (GHRP)-1

  \begin{itemize}
  \tightlist
  \item
    GHRP-2 (also called Pralmorelin)
  \item
    GHRP-4
  \item
    GHRP-5
  \item
    GHRP-6
  \end{itemize}
\item
  Hexarelin

  \begin{itemize}
  \tightlist
  \item
    18a-homo-17b-hydroxyestr-4-en-3-one
  \item
    18a-homo-3-hydroxy-estra-2,5(10)-dien-17-one (also called M-LMG)
  \end{itemize}
\item
  Human Chorionic Gonadotropin
\item
  Human Growth Hormone (HGH)

  \begin{itemize}
  \tightlist
  \item
    3b-hydroxy-5a-androst-1-en-17-one (also called 1-Androsterone,
    1-Andro and 1-DHEA)
  \item
    17b-hydroxy-androstano{[}2,3-d{]}isoxazole (also called
    Androisoxazole)
  \item
    17b-hydroxy-androstano{[}3,2-c{]}isoxazole
  \item
    3b-hydroxy-estra-4,9,11-trien-17-one
  \item
    4-hydroxytestosterone
  \end{itemize}
\item
  Ibutamoren
\item
  Insulin-like Growth Factor (IGF-1)
\item
  Ipamorelin
\item
  Letrozole
\item
  Luteinizing Hormone (LH)
\item
  Mefenorex
\item
  Meldonium
\item
  Mestanolone
\item
  Mesterolone
\item
  Methandienone (also called Methandrostenolone)
\item
  Methandriol
\item
  Methasterone
\item
  Methenolone (also called Metenolone)
\item
  Methyldienolone

  \begin{itemize}
  \tightlist
  \item
    17a-methyl-androsta-1,4-diene-3,17bdiol (also called M1 and 4ADD)
  \item
    17a-methyl-androst-2-ene-3,17b-diol
  \item
    6a-methyl-androst-4-ene-3,17-dione
  \item
    17a-methyl-androstan-3-hydroxyimine-17b-ol (also called D-Plex)
  \item
    17a-methyl-5a-androstan-17b-ol (also called Methylandrostanol and
    Protobol)
  \item
    17a-methyl-3a,17b-dihydroxy-5aandrostane
  \item
    17a-methyl-3b,17bdihydroxy-5a-androstane
  \item
    17a-methyl-3b,17b-dihydroxyandrost-4-ene
  \item
    17a-methyl-1-dihydrotestosterone
  \item
    17a-methyl-4-hydroxynandrolone
  \end{itemize}
\item
  Methylephedrine
\item
  Methylphenidate
\item
  Methyltestosterone
\item
  Methyltrienolone (also called Metribolone)
\item
  Mibolerone
\item
  Modafinil
\item
  Nandrolone (also called 19-nortestosterone)
\item
  Nikethamide

  \begin{itemize}
  \tightlist
  \item
    19-norandrostenediol (also called Boldandiol)
  \item
    19-norandrostenedione
  \end{itemize}
\item
  Norbolethone (also called Norboletone)
\item
  Norclostebol
\item
  Norethandrolone
\item
  Norfenfluramine
\item
  Normethandrolone (also called Methylnortestosterone or MENT)
\item
  Norpseudoephedrine (also called Cathine)
\item
  Oxabolone (also called 4-hydroxy-19-nortestosterone)
\item
  Oxandrolone
\item
  Oxymesterone
\item
  Oxymetholone
\item
  Pemoline
\item
  Pentetrazol
\item
  Phendimetrazine
\item
  Phenmetrazine
\item
  Phentermine
\item
  Phenylpropanolamine (PPA)
\item
  Probenecid
\item
  Prostanozol
\item
  Pseudoephedrine

  \begin{itemize}
  \tightlist
  \item
    {[}3,2,c{]}pyrazole-androst-4-en-17b-ol
  \end{itemize}
\item
  Raloxifene
\item
  Quinbolone
\item
  Selective Androgen Receptor Modulator (SARM) S-1

  \begin{itemize}
  \tightlist
  \item
    SARM S-4 (also called Andarine)
  \item
    SARM S-9
  \item
    SARM S-22 (also called Ostarine)
  \item
    SARM S-23
  \item
    SARM S-24
  \item
    SARM BMS-564,929
  \item
    SARM LGD-2226
  \item
    SARM LGD-4033 (also called Ligandrol)
  \end{itemize}
\item
  Sermorelin
\item
  Stanozolol
\item
  Stenbolone
\item
  Strychnine
\item
  Tamoxifen
\item
  Tesamorelin
\item
  Testolactone
\item
  Testosterone
\item
  Tetrahydrogestrinone (THG)
\item
  Tibolone
\item
  Toremifene
\item
  Trenbolone
\item
  Zeranol
\item
  Zilpaterol
\end{itemize}

D. Diuretics

\begin{itemize}
\tightlist
\item
  Acetazolamide
\item
  Amiloride
\item
  Bendroflumethiazide
\item
  Benzthiazide
\item
  Bumetanide
\item
  Canrenone
\item
  Chlorothiazide
\item
  Chlorthalidone
\item
  Clopamide
\item
  Cyclothiazide
\item
  Dichlorphenamide
\item
  Ethacrynic Acid
\item
  Flumethiazide
\item
  Furosemide
\item
  Hydrochlorothiazide
\item
  Hydroflumethiazide
\item
  Indapamide
\item
  Methyclothiazide
\item
  Metolazone
\item
  Polythiazide
\item
  Quinethazone
\item
  Spironolactone
\item
  Triamterene
\item
  Trichlormethiazide
\end{itemize}

\section{URINE COLLECTION PROCEDURES}\label{urine-collection-procedures}

During the Season, collections for random testing will be scheduled to
occur before practices on non-game days, and before shoot-arounds and
games on game days. For random drug testing of a visiting team scheduled
at game-day shoot-arounds, tests will be scheduled to occur before the
shoot-around for that team commences, and for any tests that are not
completed by the time the visiting team bus is scheduled to leave the
arena or practice facility after the shoot-around is completed, the team
will provide alternate transportation to the team hotel for any player
that must remain at the arena or practice facility to complete the
testing process and will ensure that a Team staff member remains with
the affected player(s) and accompanies him or them back to the Team's
hotel. Random drug tests can be scheduled to occur at any time during
the Off-Season.

When the player arrives at the collection site, the collector will
ensure that the player is positively identified through presentation of
photo ID or identification by a team representative. If the player's
identity cannot be established, the collector shall not proceed with the
collection.

The player will be asked to select a sealed urine specimen cup. The
player will then provide his urine specimen under the direct observation
of the collector.

The collector shall ensure that the player has provided a urine specimen
of sufficient volume for accurate testing. If such a sample cannot
immediately be provided by the player, he shall be instructed to remain
at the testing site for a reasonable period of time until he can provide
such a specimen. Once the specimen has been obtained, the player will
select a sealed specimen kit, which contains two bottles. The collector,
in the presence of the player, will pour the specimen into two bottles.
One bottle will be used as the primary or ``A'' specimen and the other
will be used as the split or ``B'' specimen. The specimen bottles will
be sealed with tamper-proof seals in the presence of the player. The
seals will contain a unique identification number that corresponds to
the number on the chain of custody form.

The player and collector will complete the chain of custody form (which
may be in hard copy or electronic form) that documents the handling of
the specimen. The collector will note any irregularities concerning the
specimen on the chain of custody form. Both the player and collector
will sign the chain of custody form. The kit will be sealed and sent via
an overnight delivery service to the laboratory for testing. If a hard
copy chain-of-custody form is used, it will be included in the kit
containing the two specimens that is sent by overnight delivery service
to the laboratory. If an electronic chain-of-custody form is used, it
will be sent to the laboratory electronically.

Once the specimens arrive at the laboratory, the primary specimen will
be analyzed. If the primary specimen tests positive, the split sample
will be placed in frozen storage and will be available for testing by a
different laboratory, if requested by the player.

\section{BLOOD COLLECTION PROCEDURES}\label{blood-collection-procedures}

During the Season, collections for random testing will be scheduled to
occur after practices on non-game days, and after games on game day.
Random tests can be scheduled to occur at any time during the
Off-Season.

When the player arrives at the collection site, the collector will
ensure that the player is positively identified through presentation of
photo ID or identification by a team representative. If the player's
identity cannot be established, the collector shall not proceed with the
collection. After check-in, the player will be advised to take a
15-minute hydration period if immediately coming off the court.

The player will be asked to select two blood collection kits (one as a
back-up) and another kit which will be used to transport the specimen.
Collections will be performed by blood collection specialists (who are
phlebotomists, registered and licensed nurses, medical assistants or
paramedics).

The blood collection specialist shall collect a total of 10 ml of blood
(approximately 1 tablespoon). The player's non-dominant arm will be used
to make the initial blood draw attempt. If the blood draw is not
possible or successful from the non-dominant arm, the dominant arm may
be used. No more than three (3) attempts will be made to draw a blood
specimen. After that, the collection will be discontinued. Upon
completing the blood draw, the blood collection specialist will ensure
that the draw site is not bleeding and bandage the site.

The player and collector will complete the chain of custody form (which
may be in hard copy or electronic form) that documents the handling of
the specimens. Both the player and collector will sign the chain of
custody form. The two specimens will be sealed in a blood specimen bag,
and sent via an overnight delivery service to the laboratory for
testing. If a hard copy chain-of-custody form is used, it will be
included in the kit containing the two specimens that is sent by
overnight delivery service to the laboratory. If an electronic
chain-of-custody form is used, it will be sent to the laboratory
electronically.

Once the specimens arrive at the laboratory, the primary specimen will
be analyzed. If the primary specimen tests positive, the split sample
will be placed in frozen storage and will be available for testing by a
different laboratory, if requested by the player.

\section{DRUGS OF ABUSE AND MARIJUANA CONFIRMATORY LABORATORY ANALYSIS
LEVELS}\label{drugs-of-abuse-and-marijuana-confirmatory-laboratory-analysis-levels}

\textbf{Drugs of Abuse}

\begin{itemize}
\item
  Benzodiazepines 100 ng/ml
\item
  Synthetic Cathinones Any detectable level
\item
  Cocaine Metabolites 150 ng/ml
\item
  Gamma Hydroxybutyrate (GHB) 10 mcg/ml
\item
  Ketamine 100 ng/ml
\item
  LSD 200 pg/ml
\item
  Methamphetamine 500 ng/ml (must also contain amphetamine at a
  concentration equal to or greater than 200 ng/ml)
\item
  MDMA, MDA and MDEA 500 ng/ml
\item
  Opiates:

  \begin{itemize}
  \tightlist
  \item
    Heroin Metabolite 6-acetylmorphine---10 ng/ml (only if the opiate
    metabolites are in excess of 2,000 ng/ml)
  \item
    Codeine Metabolites 2,000 ng/ml
  \item
    Morphine Metabolites 2,000 ng/ml
  \item
    Oxycodone 100 ng/ml
  \item
    Hydrocodone 300 ng/ml
  \item
    Methadone 300 ng/ml
  \item
    Hydromorphone 300 ng/ml
  \item
    Fentanyl 300 pg/ml
  \item
    Propoxyphene 200 ng/ml
  \item
    Phencyclidine (PCP) 25 ng/ml
  \end{itemize}
\end{itemize}

\textbf{Marijuana}

\begin{itemize}
\tightlist
\item
  Marijuana Metabolites 35 ng/ml
\item
  Synthetic Cannabinoids Any detectable level
\end{itemize}

\section{STEROIDS AND PERFORMANCE-ENHANCING DRUGS CONFIRMATORY
LABORATORY ANALYSIS
LEVELS}\label{steroids-and-performance-enhancing-drugs-confirmatory-laboratory-analysis-levels}

All SPEDs (including Human Growth Hormone in its synthetic form and
Testosterone in its synthetic form detected through IRMS analysis),
except those listed below, at any detectable level.

\begin{longtable}[]{@{}ll@{}}
\toprule
SPED & Amount\tabularnewline
\midrule
\endhead
Amphetamines and their analogs & 500 ng/ml\tabularnewline
Ephedra/Ephedrine & 10 mcg/ml\tabularnewline
Methylephedrine & 10 mcg/ml\tabularnewline
Nandrolone & 2 ng/ml\tabularnewline
Norpseudoephedrine & 5 mcg/ml\tabularnewline
Phenylpropanolamine (PPA) & 25 mcg/ml\tabularnewline
Pseudoephedrine & 150 mcg/ml\tabularnewline
\bottomrule
\end{longtable}

\section{CREATION OF PLAYER LONGITUDINAL
PROFILES}\label{creation-of-player-longitudinal-profiles}

The following protocol will be used to create the Longitudinal Profiles
described in Article XXXIII, Section 18 above:

\begin{enumerate}
\def\labelenumi{\arabic{enumi}.}
\item
  \textbf{Step 1:} Beginning with the 2017-18 Season, the Program's drug
  collection company will assign each player a unique personal
  identification number. A player's personal identification number will
  remain the same for all periods of time he is covered by the Program,
  and will only be used for the purposes of the Longitudinal Profile.
  Other than to the designated representatives or employees within the
  drug collection company and the Laboratory, the drug collection
  company will not disclose the personal identification number that
  corresponds to the player's name to any individual other than one
  representative each of the NBA and the Players Association.
\item
  \textbf{Step 2:} The Laboratory (as defined in Article XXXIII, Section
  18(a)) will maintain a secure, separate database for each player's
  personal identification number that contains his corresponding
  Testosterone concentration, epitestosterone concentration and
  Testosterone/Epitestosterone (``T/E'') ratio (referred to collectively
  as the ``Baseline Values''). This database will not contain any
  identifying information for the players.
\item
  \textbf{Step 3:} The Baseline Values will be calculated, pursuant to
  the Laboratory's operating standards, by averaging a player's T/E
  ratio, Testosterone concentration and Epitestosterone concentration,
  respectively, from three negative tests conducted under the Program.
  After a player's Baseline Values are established, those values will be
  considered a player's Longitudinal Profile for the duration of his
  coverage under the Program. New Baseline Values will be calculated for
  a player upon the recommendation of the director of the Laboratory.
\item
  \textbf{Step 4:} The Laboratory will compare the Baseline Values to
  the corresponding Specimen Values (as defined in Article XXXIII,
  Section 18(c)) in subsequent tests identified with a player's personal
  identification number in determining whether it will conduct IRMS
  analysis (as defined in Section 18(a) above) on a urine specimen.
\end{enumerate}

\chapter{}\label{section-2}

\section{FORM OF CONFIDENTIALITY
AGREEMENT}\label{form-of-confidentiality-agreement}

{[}Date{]}

National Basketball Players Association\\
1133 Avenue of the Americas\\
New York, NY 10036

Re: Confidentiality Agreement

Sir/Madam:

This will confirm the agreement of the National Basketball Players
Association (on behalf of itself and its employees, officers, NBA team
player representatives (``Player Representatives'') and outside advisors
(collectively, the ``Players Association'')) to maintain the
confidentiality of all Confidential Information (as defined in Paragraph
6 below) provided to the Players Association in connection with the
audit, with respect to the \texttt{20\_\_-20\_\_} Salary Cap Year, of
(i) the National Basketball Association (``NBA''), and any
League-related entities associated with generating BRI, (ii) any NBA
team that is included in such audit with respect to such Salary Cap Year
(the ``Team(s)''), under the Collective Bargaining Agreement entered
into January 19, 2017 (``CBA''), between the Players Association and the
NBA (collectively, the ``Audit''). Capitalized terms not defined herein
shall have the meaning ascribed to such terms in the CBA.

\begin{enumerate}
\def\labelenumi{\arabic{enumi}.}
\item
  The NBA and the Team(s) shall make available Confidential Information
  for purposes of the Audit based on the Players Association's
  representation that it (and its employees, officers,
  PlayerRepresentatives and outside advisors) shall comply with the
  terms of this Confidentiality Agreement at all times during and after
  the Audit. To that end, before any employee, officer, Player
  Representative or outside advisor of the Players Association may be
  permitted to review any Confidential Information, the Players
  Association shall require such employee, officer, Player
  Representative or outside advisor to agree, in writing (in the form of
  acknowledgment annexed hereto), to comply with the terms of this
  Confidentiality Agreement, and the Players Association shall promptly
  provide copies of such writings to the NBA.
\item
  The Players Association shall maintain the absolute confidentiality of
  all Confidential Information at all times and shall not disclose,
  disseminate or provide Confidential Information to any person or
  entity (including, but not limited to, any NBA players who are not
  officers of the Players Association and any representative of any
  player) at any time or for any purpose, except as permitted herein.
  The Players Association agrees that it may use or refer to
  Confidential Information only during the course of the Audit and
  solely for the purpose of conducting the Audit in accordance with the
  terms and conditions of the CBA and this Confidentiality Agreement,
  and that Confidential Information may not be used or referred to by
  the Players Association, at any time, for any other purpose.
  Notwithstanding the foregoing, or anything else in this letter
  agreement, the Players Association may only disclose or provide a
  summary of Confidential Information to Player Representatives in
  aggregate form without identifying any specific information (e.g., by
  sponsor). Notwithstanding anything to the contrary in this
  Confidentiality Agreement, the Players Association shall not be deemed
  to have violated any provision herein if the Players Association
  discloses to such third party that the Audit is being undertaken and
  that the Players Association is subject to a confidentiality agreement
  and, therefore, not permitted to discuss the Audit. The foregoing
  shall not foreclose the Players Association from disclosing
  Confidential Information during the course of a proceeding before the
  System Arbitrator, an appeal to the Appeals Panel of an award of the
  System Arbitrator and a judicial action to enforce any such proceeding
  or award.
\item
  The Players Association shall adopt and implement such procedures to
  insure the confidentiality of Confidential Information as would be
  employed by a reasonable and prudent person to safeguard the
  confidentiality of his or her own most confidential information, or,
  if more stringent, such procedures as are employed for such purposes
  by thePlayers Association for such information. Such procedures shall
  include, but not be limited to, steps to insure that: (a) such
  Confidential Information is disclosed only to those Players
  Association employees, officers, outside advisors, and, subject to the
  restrictions set forth in Paragraph 2 above, Player Representatives
  who have a need to have access to such Confidential Information and
  only for the purpose of conducting the Audit in accordance with the
  terms of the CBA and this Confidentiality Agreement; and (b) before
  any such person is permitted to review any Confidential Information,
  he or she agrees in writing to comply with the terms of this
  Confidentiality Agreement by signing the form of acknowledgment
  annexed hereto as provided for in Paragraph 1 above. The foregoing
  shall not foreclose the Players Association from disclosing
  Confidential Information during the course of a proceeding before the
  System Arbitrator, an appeal to the Appeals Panel of an award of the
  System Arbitrator and a judicial action to enforce any such proceeding
  or award.
\item
  The Players Association agrees that no copies of Confidential
  Information made available by the NBA and the Teams at their
  respective offices in connection with the Audit may be removed from
  such offices without the express written consent of the NBA or the
  Teams (as applicable) (for example, in connection with the use of
  online data rooms to permit access to information provided
  electronically during the on-site audit or to respond to information
  requests). Should the NBA or the Teams permit copies of Confidential
  Information to be removed from their offices in connection with the
  Audit, then at the request of the NBA, all such copies shall be
  returned to the NBA within thirty (30) days following completion of
  the Audit. Notwithstanding the foregoing, the Players Association
  shall be under no obligation to return copies of the final Audit
  Report or any debriefing memoranda (except to the extent such
  memoranda append contract documents) prepared by the Accountants and
  provided to the Players Association in connection with any audit
  pursuant to Article VII, Section 10.
\item
  If the Players Association is required by governmental or judicial
  authorities (by oral questions, interrogatories, requests for
  information or documents, subpoena, civil investigative demand or any
  other similar process) to disclose any Confidential Information, it
  shall provide the NBA and/or the Teams with prompt notice so that the
  NBA and/or the Teams may seek an appropriate protective order. If, in
  the absence of a protective order, the Players Association is, after
  giving notice in accordance with the preceding sentence, compelled to
  disclose Confidential Information or else stand liable for contempt or
  suffer other censure or penalty, the Players Association may disclose
  only such Confidential Information as is necessary to avoid such
  liability without incurring liability hereunder.
\item
  For purposes of this Confidentiality Agreement, ``Confidential
  Information'' shall mean all documents, materials and other
  information reviewed or made available (whether in written or oral
  form) in connection with the Audit (including, without limitation, all
  documents, debriefing memoranda, materials and other information made
  available by PricewaterhouseCoopers, LLP (``PwC'')), and shall include
  all excerpts, extracts, summaries and contents thereof and notes taken
  by the Players Association during the Audit; provided, however that
  Confidential Information shall not include information that (a) is or
  becomes generally available to the public other than as a result of
  disclosure by the Players Association (including Players Association
  affiliates or representatives), (b) was available to the Players
  Association prior to its disclosure by the NBA, the Team(s) or PwC (as
  applicable), or (c) becomes available to the Players Association from
  a source other than the NBA, the Team(s) or PwC, provided that such
  source is not bound by a confidentiality agreement with the NBA, the
  Teams, the Players Association or PwC.
\item
  The Players Association acknowledges that the terms and conditions
  contained in this Confidentiality Agreement are reasonable and
  necessary to protect the legitimate interests of the NBA and the
  Teams, do not cause the Players Association undue hardship, and that
  any violation of the provisions of this Confidentiality Agreement or
  disclosure of any Confidential Information without the NBA's or the
  Teams' (as applicable) prior written consent will result in
  irreparable injury to the NBA and/or the Teams for which there is no
  adequate remedy at law. Accordingly, in the event of any such
  violation or disclosure, the NBA and/or the Teams shall be entitled to
  preliminary and permanent injunctive relief from any federal or state
  court of competent jurisdiction located in New York, New York, and the
  Players Association hereby consents to, and waives any objection to,
  venue and jurisdiction in such courts. In addition, the Players
  Association shall indemnify and hold harmless the NBA and its member
  teams and their respective affiliates, owners, directors, governors,
  officers and employees, and the successors, assigns and personal
  representatives of the foregoing parties (``NBA indemnified
  parties''), from and against all liability, damages and costs
  (including attorneys fees) arising out of any claim asserted against
  any NBA indemnified party relating to any violation of this
  Confidentiality Agreement by the Players Association, provided that:
  (a) such violation resulted from the Players Association's negligent
  or intentional use or disclosure of Confidential Information; (b) the
  Players Association is given prompt notice of any such claim; (c) the
  Players Association has the right to approve counsel and/or has the
  opportunity to undertake the defense of such claim; and (d) the
  indemnified party does not admit liability with respect to and does
  not settle such claim without the prior written consent of the Players
  Association. The Players Association also agrees that the relief
  provided for in this Paragraph 7 shall be cumulative and in addition
  to any other rights or remedies to which the NBA and the Teams may be
  entitled.
\item
  This Confidentiality Agreement is the final and complete agreement
  between the parties with respect to its subject matter. Any waiver of
  or modification to this Confidentiality Agreement must be in a writing
  and signed by each party. Any waiver in any particular instance of the
  rights and limitations contained herein shall not be deemed and is not
  intended to be a general waiver of any rights or limitations contained
  herein and shall not operate as a waiver beyond the particular
  instance.
\item
  This Confidentiality Agreement shall be governed by and construed and
  enforced in accordance with the laws of the State of New York, without
  giving effect to the principles of conflicts of law thereof.
\end{enumerate}

If the foregoing coincides with your understanding of our agreement,
please sign the enclosed copy of this letter and return it to me.

Sincerely,

\begin{longtable}[]{@{}l@{}}
\toprule
NATIONAL BASKETBALL ASSOCIATION\tabularnewline
By:
\_\_\_\_\_\_\_\_\_\_\_\_\_\_\_\_\_\_\_\_\_\_\_\_\_\_\_\_\_\_\_\_\_\_\_\tabularnewline
\tabularnewline
AGREED TO AND ACCEPTED:\tabularnewline
NATIONAL BASKETBALL PLAYERS ASSOCIATION\tabularnewline
By:
\_\_\_\_\_\_\_\_\_\_\_\_\_\_\_\_\_\_\_\_\_\_\_\_\_\_\_\_\_\_\_\_\_\_\_\tabularnewline
\bottomrule
\end{longtable}

\newpage

\section{J-2}\label{j-2}

January 19, 2017

Michele Roberts, Esq.\\
Executive Director\\
National Basketball Players Association\\
1133 Avenue of the Americas\\
5th Floor\\
New York, New York 10036

Dear Michele:

This will confirm our agreement that the attached accounting procedures
are the procedures that will be in effect for purposes of Article VII,
Section 10 of the Collective Bargaining Agreement entered into on
January 19, 2017, unless such procedures shall be modified by agreement
of the parties.

If the foregoing coincides with your understanding of our agreement,
please sign this letter in the space provided below.

Sincerely,\\
/s/ RICHARD W. BUCHANAN\\
Richard W. Buchanan

AGREED TO AND\\
ACCEPTED:

NATIONAL BASKETBALL PLAYERS ASSOCIATION

By:\\
/s/ MICHELE ROBERTS\\
Michele Roberts\\
Executive Director

\newpage

\subsection{Minimum Procedures To Be Provided By The
Accountants}\label{minimum-procedures-to-be-provided-by-the-accountants}

\textbf{General}

\begin{itemize}
\tightlist
\item
  The Audit Report (and any Interim Audit Report or Interim Escrow Audit
  Report) must be prepared in accordance with the relevant terms of the
  Collective Bargaining Agreement (``CBA''), which should be reviewed
  and understood by all auditors.
\item
  The Basketball Related Income Reporting Package and instructions
  should be reviewed and understood by all auditors.
\item
  All audit workpapers should be made available for review by
  representatives of the NBA and Players Association prior to issuance
  of the report.
\item
  A summary of all audit findings (including any unusual or
  non-recurring transactions) and proposed adjustments must be jointly
  reviewed with representatives of the NBA and Players Association prior
  to issuance of the report.
\item
  Any problems or questions raised during the audit should be resolved
  jointly with representatives of the NBA and Players Association (or by
  the Accountants, to the extent called for under the CBA).
\item
  All estimates should be reviewed in accordance with the CBA. Estimates
  are to be reviewed based upon the previous year's actual results and
  current year activity. All estimates should be confirmed with third
  parties when possible.
\item
  Revenue and expense amounts that have been estimated should be
  reconfirmed with the controller or other team representatives prior to
  the issuance of the Audit Report on or before the last day of the
  Moratorium Period.
\item
  Where appropriate, team and NBA revenues and expenses should be
  reconciled to audited financial statements.
\item
  All reporting packages and supporting schedules are to be completed in
  U.S. dollars.
\item
  The Auditors may consider, but are not bound by, the value attributed
  to or treatment of revenue or expense items in prior years.
\item
  Auditors should be aware of revenues excluded from BRI. The Teams
  should be instructed to make available to the Auditors all information
  necessary to determine categories of revenues they have excluded from
  BRI. Questions regarding whether revenues or expenses are includable
  or excludable from BRI should be reviewed with both parties to
  determine proper treatment. Auditors should perform a review for
  revenues improperly excluded from, or included in, BRI.
\end{itemize}

\textbf{Team Salaries}

\begin{itemize}
\tightlist
\item
  Trace amounts to the team's general ledger or other supporting
  documentation for agreement.
\item
  Foot all schedules and perform other clerical tests.
\item
  Examine an appropriate sample of player contracts, noting agreement of
  all salary amounts, in accordance with the definition of Salary in the
  CBA.
\item
  Compare player names with all player lists for the season in question.
\item
  Inquire of controller or other representative of each team if any
  additional compensation was paid to players and not included on the
  schedule, and, if so, whether or not such amounts were paid for
  basketball services. Also inquire if any business arrangements were
  entered into by the team or team affiliate with players or their
  affiliates, including with retired players who played for the team
  within the past five (5) years.
\item
  Review performance bonuses to determine whether such bonuses were
  actually earned for such season.
\item
  Review signing bonuses to determine if they have been properly
  allocated in accordance with the terms of the CBA.
\item
  Confirm that, where provided in the CBA, certain contracts have been
  averaged.
\end{itemize}

\textbf{Benefits}

\begin{itemize}
\tightlist
\item
  Trace amounts to the team's general ledger or other supporting
  documentation for agreement.
\item
  Foot all schedules and perform other clerical tests.
\item
  Investigate variations in amounts from the prior year through
  discussion with the controller or other representative of the team.
\item
  Review each team's insurance expenses for premium credits (refunds)
  received from Planet Insurance Ltd. (owned by Teams) and the players'
  medical and dental insurance carriers (amounts can be obtained from
  League Office).
\item
  Review League Office supporting documentation with respect to
  Benefits.
\end{itemize}

\textbf{Basketball Related Income}

\begin{itemize}
\tightlist
\item
  Trace amounts to team's general ledger or other supporting
  documentation for agreement.
\item
  Foot all schedules and perform other clerical tests.
\item
  Trace gate receipts to general ledger and test supporting
  documentation where appropriate.
\item
  Gate receipts should be reviewed and reconciled to League Office gate
  receipts summary.
\item
  Verify amounts reported as luxury suite revenues with supporting
  documentation from the entity that sold, leased or licensed such
  luxury suites.
\item
  Verify amounts reported as complimentary tickets and tickets traded
  for goods or services with supporting documentation from the team.
\item
  Trace amounts reported for novelties and concessions, game parking,
  game programs, Team sponsorships and promotions, arena signage and
  arena club sales to general ledgers and test supporting documentation
  where appropriate.
\item
  Where reported amounts include proceeds received by a Related Party,
  verify the amounts reported with supporting documentation from the
  Related Party.
\item
  Examine the National Television and Cable contracts at the League
  Office, and agree to amounts reported.
\item
  Review, at League Office, expenses deducted from the National
  contracts in accordance with the terms of the CBA. Review supporting
  documentation and test where applicable.
\item
  Examine local television, local cable and local radio contracts.
  Verify to amounts reported by teams.
\item
  When local broadcast revenues are not verifiable by reviewing a
  contract, detailed supporting documentation should be reviewed and
  tested.
\item
  All loans, advances, bonuses, etc. received by the League Office or
  its teams should be noted in the report and included in BRI where
  appropriate.
\item
  Schedules of NBA Radio, NBA TV, international broadcast, NBA Media
  Ventures, copyright royalty revenues and expenses should be obtained
  from the NBA. Schedules should be verified by agreeing to general
  ledgers and examining supporting documentation where applicable.
\item
  Schedules of revenues and expenses reported by Properties for
  sponsorship, NBA related revenues from NBA Entertainment, and NBA
  Special Events should be obtained from the NBA. Schedules should be
  verified by agreeing to general ledgers and examining supporting
  documentation where applicable.
\item
  Net exhibition revenues and expenses should be verified to supporting
  documentation where appropriate.
\item
  All amounts of other revenues should be reviewed for proper
  inclusion/exclusion in BRI. Test appropriateness of balances where
  appropriate.
\item
  Determine the ratio of expenses to revenues for those categories of
  proceeds that come within the provisions of Article VII, Section
  1(a)(6) of the CBA and determine the extent to which expenses should
  be disallowed, if at all, pursuant to the provisions of that Section.
\end{itemize}

\textbf{Playoff Revenues}

\begin{itemize}
\tightlist
\item
  All sources of playoff revenues and expenses should be verified per
  the procedure outlined for Basketball Related Income.
\item
  Because of the late timing of the Playoffs, special attention should
  be given to revenue and expense estimates.
\item
  Playoff gate receipts should be recorded net of Taxes. Payments made
  to the Playoff Pool should not be deducted. Odd game payments should
  not be either deducted by the paying team or recorded by the receiving
  team.
\item
  Other playoff expenses should be reviewed in accordance with the terms
  of the CBA.
\item
  Team expenses paid by the League Playoff Pool, including travel
  expenses, should not be deducted by teams.
\item
  Review League Office supporting documentation as to expenses deducted
  from the Playoff Pool.
\end{itemize}

\textbf{Related Party Transactions}

\begin{itemize}
\tightlist
\item
  Inquire of the controller or other representative of the team what, if
  any, Related Parties exist, and discuss with the parties what, if any,
  amounts should be included in BRI.
\item
  Review information provided as to the team's Related Parties and
  revenues that arise from Related Party transactions, and request
  supporting details where appropriate.
\item
  Any revenue from a Related Party should be reviewed with both parties
  to determine proper treatment under the CBA.
\item
  Request that details be provided, where appropriate.
\item
  Prepare a summary of any changes, corrections or additions to Related
  Party information previously reported.
\end{itemize}


\end{document}
