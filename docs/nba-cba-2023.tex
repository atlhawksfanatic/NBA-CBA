% Options for packages loaded elsewhere
\PassOptionsToPackage{unicode}{hyperref}
\PassOptionsToPackage{hyphens}{url}
%
\documentclass[
]{book}
\usepackage{amsmath,amssymb}
\usepackage{lmodern}
\usepackage{iftex}
\ifPDFTeX
  \usepackage[T1]{fontenc}
  \usepackage[utf8]{inputenc}
  \usepackage{textcomp} % provide euro and other symbols
\else % if luatex or xetex
  \usepackage{unicode-math}
  \defaultfontfeatures{Scale=MatchLowercase}
  \defaultfontfeatures[\rmfamily]{Ligatures=TeX,Scale=1}
\fi
% Use upquote if available, for straight quotes in verbatim environments
\IfFileExists{upquote.sty}{\usepackage{upquote}}{}
\IfFileExists{microtype.sty}{% use microtype if available
  \usepackage[]{microtype}
  \UseMicrotypeSet[protrusion]{basicmath} % disable protrusion for tt fonts
}{}
\makeatletter
\@ifundefined{KOMAClassName}{% if non-KOMA class
  \IfFileExists{parskip.sty}{%
    \usepackage{parskip}
  }{% else
    \setlength{\parindent}{0pt}
    \setlength{\parskip}{6pt plus 2pt minus 1pt}}
}{% if KOMA class
  \KOMAoptions{parskip=half}}
\makeatother
\usepackage{xcolor}
\usepackage{longtable,booktabs,array}
\usepackage{calc} % for calculating minipage widths
% Correct order of tables after \paragraph or \subparagraph
\usepackage{etoolbox}
\makeatletter
\patchcmd\longtable{\par}{\if@noskipsec\mbox{}\fi\par}{}{}
\makeatother
% Allow footnotes in longtable head/foot
\IfFileExists{footnotehyper.sty}{\usepackage{footnotehyper}}{\usepackage{footnote}}
\makesavenoteenv{longtable}
\usepackage{graphicx}
\makeatletter
\def\maxwidth{\ifdim\Gin@nat@width>\linewidth\linewidth\else\Gin@nat@width\fi}
\def\maxheight{\ifdim\Gin@nat@height>\textheight\textheight\else\Gin@nat@height\fi}
\makeatother
% Scale images if necessary, so that they will not overflow the page
% margins by default, and it is still possible to overwrite the defaults
% using explicit options in \includegraphics[width, height, ...]{}
\setkeys{Gin}{width=\maxwidth,height=\maxheight,keepaspectratio}
% Set default figure placement to htbp
\makeatletter
\def\fps@figure{htbp}
\makeatother
\setlength{\emergencystretch}{3em} % prevent overfull lines
\providecommand{\tightlist}{%
  \setlength{\itemsep}{0pt}\setlength{\parskip}{0pt}}
\setcounter{secnumdepth}{5}
\usepackage[margin=1in]{geometry}
\usepackage{booktabs}
\usepackage{enumitem}

% Without these you get ! LaTeX Error: Too deeply nested.
\setlistdepth{10}
\renewlist{enumerate}{enumerate}{10}
\setlist[itemize]{labelsep=.5em}

% Correct for the way articles/sections are defined and pretty the TOC
\usepackage{fancyhdr}
\usepackage{tocbasic}

\DeclareTOCStyleEntry[%
  entryformat=\bfseries,
  pagenumberformat=\bfseries,
]{tocline}{chapter}
\DeclareTOCStyleEntries[
  pagenumberbox=\hbox,
  dynnumwidth
]{tocline}{%
  chapter,section,subsection,subsubsection,paragraph,subparagraph,%
  figure,table
}
\DeclareTOCStyleEntries[
  dynindent
]{tocline}{subsection,subsubsection,subparagraph}

\renewcommand{\chaptername}{Article}
\renewcommand{\thechapter}{\Roman{chapter}}
\renewcommand{\appendixname}{Exhibit}
\ifLuaTeX
  \usepackage{selnolig}  % disable illegal ligatures
\fi
\usepackage[]{natbib}
\bibliographystyle{plainnat}
\IfFileExists{bookmark.sty}{\usepackage{bookmark}}{\usepackage{hyperref}}
\IfFileExists{xurl.sty}{\usepackage{xurl}}{} % add URL line breaks if available
\urlstyle{same} % disable monospaced font for URLs
\hypersetup{
  pdftitle={NBA Collective Bargaining Agreement - 2023},
  pdfauthor={Robert},
  hidelinks,
  pdfcreator={LaTeX via pandoc}}

\title{NBA Collective Bargaining Agreement - 2023}
\author{Robert}
\date{2023-07-11}

\begin{document}
\maketitle

{
\setcounter{tocdepth}{1}
\tableofcontents
}
\hypertarget{preface}{%
\chapter*{Preface}\label{preface}}
\addcontentsline{toc}{chapter}{Preface}

\emph{If you are only interested in a pdf or epub of the CBA, then please click on the download icon at the top left (it is next to the ``A'') and select the format you wish to download.}

This is the 2023 NBA's Collective Bargaining Agreement (CBA) converted to markdown and disseminated in the format you are currently viewing this as.

The original version of the 2023 CBA can be found on my Github website \href{https://atlhawksfanatic.github.io/}{atlhawksfanatic.github.io} (\href{https://github.com/atlhawksfanatic/atlhawksfanatic.github.io/raw/master/research/CBA/2023-NBA-NBPA-Collective-Bargaining-Agreement.pdf}{pdf found here}) as a way to cross-reference any potential discrepancies.

The original CBA was converted to a text file with \href{https://github.com/ropensci/pdftools}{pdftools}, then broken up by Article and Exhibit through regular expressions into .Rmd files. Those .Rmd files serves as the basis for this bookdown site.

The purpose of this project is three-fold:

\begin{enumerate}
\def\labelenumi{\arabic{enumi}.}
\tightlist
\item
  to historically document collective bargaining agreements of the NBA;
\item
  to provide easier navigation of the CBA through a structured format of each Article and Section; and
\item
  for me to better understand the CBA through this exercise.
\end{enumerate}

I do not own any rights to the CBA and am simply redistributing it in a different format. This is not the official version of the CBA and I am not responsible for any errors that might be present in this document. If you believe you have found an error, please let me know and I will correct it.

Contact: \href{atlhawksfanatic@gmail.com}{via email}, \href{https://github.com/atlhawksfanatic}{through Github}, or \href{https://twitter.com/atlhawksfanatic}{on Twitter}

\hypertarget{definitions}{%
\chapter{DEFINITIONS}\label{definitions}}

\hypertarget{definitions.}{%
\section{Definitions.}\label{definitions.}}

As used in this Agreement, the following terms shall have the following meanings:

\begin{enumerate}
\def\labelenumi{\arabic{enumi}.}
\tightlist
\item
  ``2011 CBA'' means the collective bargaining agreement between the NBA and the Players Association, effective December 8, 2011 through June 30, 2017.
\item
  ``2017 CBA'' means the collective bargaining agreement between the NBA and the Players Association, effective July 1, 2017 through June 30, 2023.
\item
  ``Active List'' means the list of players, maintained by the NBA, who have signed Player Contracts with a Team and are otherwise eligible to participate in a Regular Season game.
\item
  ``Agreement'' means this Collective Bargaining Agreement entered into as of June 28, 2023.
\item
  ``Audit Report'' or ``final Audit Report'' means the audit report prepared in accordance with Article VII, Section 10.
\item
  ``Average Player Salary'' means, with respect to any Salary Cap Year, Total Salaries (plus any amounts paid by a Team in respect of such Salary Cap Year pursuant to Article VII, Section 2(c)(2)(i) or 2(c)(5)) divided by an amount equal to the product of the number of Teams in the NBA in such Salary Cap Year (other than Expansion Teams during their first two (2) Salary Cap Years) multiplied by thirteen and two-tenths (13.2).
\item
  ``Base Compensation'' means the component of Compensation other than bonuses of any kind.
\item
  ``Basketball Related Income'' or ``BRI'' means basketball related income as defined in Article VII, Sections 1(a) and (b).
\item
  ``Benefits'' or ``Total Benefits'' means the sum of all amounts paid or to be paid on an accrual basis during any Salary Cap Year by the NBA or NBA Teams, other than Expansion Teams during their first two Salary Cap Years, for the specific benefits set forth in Article IV.
\item
  ``Commissioner'' means the Commissioner of the NBA.
\item
  ``Compensation'' means the compensation that is or could be earned by, or is paid or payable to, an NBA player (including players whose Player Contracts have been terminated) in accordance with a Player Contract (whether such payment is sent to the player directly or to a person or entity designated by a player).
\item
  ``Contract'' (see ``Uniform Player Contract'').
\item
  ``Current Base Compensation'' means the component of Base Compensation other than Deferred Base Compensation.
\item
  ``Deferred Base Compensation'' means the component of Deferred Compensation other than bonuses of any kind.
\item
  ``Deferred Compensation'' means the component of Compensation for a Season that is payable to a player during the period commencing after the May 1 following such Season, in accordance with the rules set forth in Articles VII and XXV. The determination of whether Compensation is Deferred Compensation will be based upon the time set by the Player Contract for the player to receive the Compensation, without regard to whether the obligation is funded currently or secured in any fashion.
\item
  ``Designated Veteran Player'' means a player with whom a Team has, pursuant to Article II, Sections 7(a)(ii) or 7(e) and Article VII, Section 7(a), entered into either a Designated Veteran Player Extension or Designated Veteran Player Contract.
\item
  ``Designated Veteran Player Contract'' means a Contract entered into between a Team and its Designated Veteran Player who is a Qualifying Veteran Free Agent that covers five (5) Seasons and provides for Salary for the first Salary Cap Year equal to such percentage above thirty percent (30\%) but not greater than thirty-five percent (35\%) (as agreed upon by the Team and the player) of the Salary Cap in effect at the time the Contract is executed. Annual increases and decreases in Salary and Unlikely Bonuses in a Designated Veteran Player Contract shall be governed by Article VII, Section 5(a)(2).
\item
  ``Designated Veteran Player Extension'' means an Extension of a Contract entered into between a Team and its Designated Veteran Player that covers six (6) Seasons from the date the Extension is signed and provides for Salary for the first Salary Cap Year covered by the extended term equal to thirty percent (30\%) or thirty-five percent (35\%) (or such other percentage between 30\% and 35\% as agreed upon by the Team and the player) of the Salary Cap in effect during the first Season of the extended term. Annual increases and decreases in Salary in a Designated Veteran Player Extension shall be governed by Article VII, Section 5(a)(3).
\item
  ``Draft'' or ``NBA Draft'' means the NBA's annual draft of Rookie basketball players.
\item
  ``Early Qualifying Veteran Free Agent'' means a Veteran Free Agent who, prior to becoming a Veteran Free Agent, played under one (1) or more Player Contracts covering some or all of each of the two (2) preceding Seasons, and who either played exclusively with his Prior Team during such two (2) Seasons, or, if he played for more than one (1) Team during such period, changed Teams only (i) by means of trade, (ii) by means of assignment via the NBA's waiver procedures, or (iii) by signing with his Prior Team during the first of the two (2) Seasons.
\item
  ``Early Termination Option'' (or ``ETO'') means an option in favor of a player to shorten the stated number of years covered by a Player Contract in accordance with Article XII.
\item
  ``Effective Season'' means, with respect to an Early Termination Option, the first Season covered by the Early Termination Option. (For example, if a Contract were to contain an Early Termination Option exercisable following the 2025-26 Season, the Effective Season would be the 2026-27 Season.)
\item
  ``Estimated Average Player Salary'' means, for a particular Salary Cap Year, one hundred four and one-half percent (104.5\%) of the prior Salary Cap Year's Average Player Salary.
\item
  ``Exception'' means an exception to the rule that a Team's Team Salary may not exceed the Salary Cap.
\item
  ``Expansion Team'' means any Team that becomes a member of the NBA through expansion following the effective date of this Agreement and commences play during the term of this Agreement.
\item
  ``Extension'' means an amendment to a Player Contract lengthening the term of the Contract for a specified period of years.
\item
  ``First Round Pick'' means a player selected by a Team in the first round of the Draft.
\item
  ``Free Agent'' means: (i) a Veteran Free Agent; (ii) a Rookie Free Agent; (iii) a Veteran whose Player Contract has been terminated in accordance with the NBA waiver procedure; or (iv) a player whose last Player Contract was a 10-Day Contract and who either completed the Contract by rendering the playing services called for thereunder or was released early from such Contract.
\item
  ``Generally Recognized League Honors'' means the following NBA league honors awarded to players: NBA Most Valuable Player; NBA Finals Most Valuable Player; NBA Defensive Player of the Year; NBA Sixth Man Award; NBA Most Improved Player; All-NBA Team (First, Second, or Third); NBA All-Defensive Team (First or Second); and All-Star Team Selection.
\item
  ``Inactive List'' means the list of players, maintained by the NBA, who have signed Player Contracts with a Team and are otherwise ineligible to participate in a Regular Season game.
\item
  ``Incentive Compensation'' means the component of Compensation consisting of one (1) or more bonuses described in Article II, Sections 3(b)(ii) and (iii) and 3(c).
\item
  ``Likely Bonus'' means Incentive Compensation included in a player's Salary in accordance with Article VII, Section 3(d).
\item
  ``Maximum Annual Salary'' means the maximum amount of Salaries and Unlikely Bonuses a player is eligible to receive in the first Salary Cap Year covered by a Contract or Extension as calculated in accordance with Article II, Section 7.
\item
  ``Member'' or ``Team'' means any team that is a member of the NBA.
\item
  ``Minimum Annual Salary'' means the minimum Salary that must be included in a Player Contract (other than a Two-Way Contract) that covers the entire Regular Season in accordance with Article II, Section 6(a).
\item
  ``Minimum Annual Salary Scale'' means: (i) for the 2023-24 Salary Cap Year, the table of Salary amounts equal to the Salary amounts set forth in the Baseline Minimum Annual Salary Scale table (annexed hereto as Exhibit C) adjusted by applying the percentage increase in the Salary Cap from the 2022-23 Salary Cap Year to the 2023-24 Salary Cap Year; and (ii) for each Salary Cap Year commencing with the 2024-25 Salary Cap Year, the table of Salary amounts equal to the Salary amounts set forth in the preceding Salary Cap Year's Minimum Annual Salary Scale adjusted by applying the percentage increase in the Salary Cap from the preceding Salary Cap Year to the then-current Salary Cap Year.
\item
  ``Minimum Player Salary'' means: (i) with respect to a Contract (other than a Two-Way Contract) that covers the entire Regular Season, the Minimum Annual Salary called for under Article II, Section 6(a); (ii) with respect to a Contract that covers less than the entire Regular Season (other than a Two-Way Contract or 10-Day Contract), the Minimum Annual Salary called for under Article II, Section 6(a) multiplied by a fraction, the numerator of which is the number of days remaining in the NBA Regular Season as of the date such Contract is entered into, and the denominator of which is the total number of days of that NBA Regular Season; and (iii) with respect to a 10-Day Contract, the Minimum Annual Salary called for under Article II, Section 6(a) multiplied by a fraction, the numerator of which is the number of days covered by the Contract and the denominator of which is the total number of days of that NBA Regular Season.
\item
  ``Minimum Team Salary'' means the minimum Team Salary a Team must have for a Salary Cap Year as determined in accordance with Article VII, Section 2(b).(mm) ``Moratorium Period'' means, with respect to a Salary Cap Year, the period from 12:01 a.m. eastern time on July 1 of such Salary Cap Year through 12:00 p.m. eastern time on the following July 6 (for clarity, regardless of whether July 6 is a business day).
\item
  The term ``negotiate'' means, with respect to a player or his representatives on the one hand, and a Team or its representatives on the other hand, to engage in any written or oral communication relating to the possible employment, or terms of employment, of such player by such Team as a basketball player, regardless of who initiates such communication.
\item
  ``NBAGL'' means the NBA G League or any successor entity.
\item
  ``NBAGL Regular Season'' means, with respect to any NBAGL Season, the period beginning on the first day and ending on the last day of regularly scheduled (as opposed to exhibition or playoff) competition between NBAGL teams.
\item
  ``NBAGL Season'' means the period beginning on the first day of NBAGL training camp and ending immediately after the last game of the NBAGL playoffs.
\item
  ``Non-Qualifying Veteran Free Agent'' means a Veteran Free Agent who is not a Qualifying Veteran Free Agent or an Early Qualifying Veteran Free Agent.
\item
  ``Option'' means an option in a Player Contract in favor of a Team or player to extend such Contract beyond its stated term.
\item
  ``Option Year'' means the year that would be added to a Player Contract if an Option were exercised.
\item
  ``Performance Bonus'' means any Incentive Compensation described in Article II, Section 3(b)(ii).
\item
  ``Player Contract'' (see ``Uniform Player Contract'').
\item
  ``Prior Team'' means the Team for which a player was last under Contract prior to becoming a Qualifying Veteran Free Agent, Early Qualifying Veteran Free Agent, or a Non-Qualifying Veteran Free Agent.
\item
  ``Qualifying Offer'' means a qualifying offer as defined in Article XI, Section 1(e).
\item
  ``Qualifying Veteran Free Agent'' means a Veteran Free Agent who, prior to becoming a Veteran Free Agent, played under one (1) or more Player Contracts covering some or all of each of the three (3) preceding Seasons, and who either played exclusively with his Prior Team during such three (3) Seasons, or, if he played for more than one (1) Team during such period, changed Teams only (i) by means of trade, (ii) by means of assignment via the NBA's waiver procedures during the first of the three (3) Seasons, or (iii) by signing with his Prior Team during the first of the three (3) Seasons.
\item
  ``Regular Salary'' means a player's Salary, less any component thereof that is a signing bonus (or deemed a signing bonus in accordance with Article VII) and any component thereof that is Incentive Compensation.
\item
  ``Regular Season'' means, with respect to any Season, the period beginning on the first day and ending on the last day of regularly scheduled (as opposed to Exhibition, Play-In, or playoff) competition between NBA Teams.
\item
  ``Renegotiation,'' ``renegotiate,'' or ``renegotiated'' means a Contract amendment that provides for an increase in Salary and/or Unlikely Bonuses.
\item
  ``Replacement Player'' means, where appropriate, either a player who is acquired by a Team pursuant to the Traded Player Exception, or a player who is signed or acquired by a Team pursuant to the Disabled Player Exception.
\item
  ``Required Tender'' means an offer of a Uniform Player Contract to a Draft Rookie, signed by the Team, that: (i) on or before the date specified in Article X is either personally delivered to the player or his representative or sent by email or pre-paid certified, registered, or overnight mail to the last known address of the player or his representative; (ii) with respect to a First Round Pick, (A) affords the player until at least the first day of the following Regular Season to accept, and (B) satisfies the requirements of a Rookie Scale Contract set forth in Article VIII, Section 1 or 2; and (iii) with respect to a Second Round Pick, (A) affords the player until at least the earlier of (x) four (4) days before the date of the first day of the immediately following Regular Season, or (y) the immediately following October 15, to accept, (B) has a stated term of one (1) Season, and (C) calls for at least the Minimum Player Salary then applicable to the player. In addition, a Team shall be permitted to include in any Required Tender an Exhibit 6 to the Uniform Player Contract requiring that the player, if he signs the Required Tender, pass a physical examination to be performed by a physician designated by the Team as a condition precedent to the validity of the Contract.
\item
  ``Restricted Free Agent'' means a Veteran Free Agent who is subject to a Team's Right of First Refusal in accordance with Article XI.
\item
  ``Rookie'' means a person who has never signed a Player Contract with an NBA Team.

  \begin{enumerate}
  \def\labelenumii{\roman{enumii}.}
  \tightlist
  \item
    ``Draft Rookie'' means a Rookie who is selected in the NBA Draft.
  \item
    ``Non-Draft Rookie'' means a Rookie who is not selected in the NBA Draft for which he is first eligible.
  \end{enumerate}
\item
  ``Rookie Free Agent'' means: (i) a Draft Rookie who, pursuant to the provisions of Article VIII, Section 3 or Article X, is no longer subject to the exclusive negotiating rights of any Team, and who may be signed by any Team; or (ii) a Non-Draft Rookie.
\item
  ``Rookie Salary Scale'' means the Rookie Salary Scale table for a Salary Cap Year prepared immediately upon the determination of the Salary Cap for such Salary Cap Year and including the adjusted Rookie Scale Amounts for such Salary Cap Year as calculated in accordance with Section 1(iii) below.
\item
  ``Rookie Scale Amounts'' means: (i) for the 2023-24 Salary Cap Year, the Salary amounts set forth in the Baseline Rookie Salary Scale (annexed hereto as Exhibit B) adjusted by applying the percentage increase in the Salary Cap from the 2022-23 Salary Cap Year to the 2023-24 Salary Cap Year; and (ii) for each Salary Cap Year commencing with the 2024-25 Salary Cap Year, the Salary amounts set forth in the preceding Salary Cap Year's Rookie Salary Scale adjusted by applying the percentage increase in the Salary Cap from the preceding Salary Cap Year to the current Salary Cap Year. For clarity, the applicable percentages in the ``4th Year Option: Percentage Increase Over 3rd Year Salary'' and ``Qualifying Offer: Percentage Increase Over 4th Year Salary'' columns specified in the Baseline Rookie Salary Scale shall remain the same for each Salary Cap Year during the term of this Agreement and shall be included in the Rookie Salary Scale prepared for each Salary Cap Year in accordance with Section 1(hhh) above.
\item
  ``Rookie Scale Contract'' means the initial Uniform Player Contract entered into, in accordance with Article VIII, Section 1 or 2, between a First Round Pick and the Team that holds his draft rights.
\item
  ``Room'' means the extent to which: (i) a Team's then-current Team Salary is less than the Salary Cap; or (ii) a Team is entitled to use one of the Salary Cap Exceptions set forth in Article VII, Sections 6(c), (d), (e), (f), (g), and (j) (Disabled Player Exception, Bi-annual Exception, Non-Taxpayer Mid-Level Salary Exception, Taxpayer Mid-Level Salary Exception, Mid-Level Salary Exception for Room Teams, and Traded Player Exception).
\item
  ``Salary'' means, with respect to a Salary Cap Year, a player's Compensation with respect to the Season covered by such Salary Cap Year, plus any other amount that is deemed to constitute Salary in accordance with the terms of this Agreement, not including Unlikely Bonuses, any benefits the player received in accordance with the terms of this Agreement (including, e.g., the benefits provided for by Article IV, per diem, and moving expenses), and any portion of the player's Compensation that is attributable to another Salary Cap Year in accordance with this Agreement. Salary also includes any consideration received by a retired player that is deemed to constitute Salary in accordance with the terms of Article XIII.
\item
  ``Salary Cap'' means the maximum allowable Team Salary for each Team for a Salary Cap Year, subject to the rules and exceptions set forth in this Agreement.
\item
  ``Salary Cap Year'' means the period from July 1 through the following June 30.(ooo) ``Season'' or ``NBA Season'' means the period beginning on the first day of NBA training camp and ending immediately after the last game of the NBA Finals.
\item
  ``Second Round Pick'' means a player selected by a Team in the second round of the Draft.
\item
  ``Standard NBA Contract'' means a Contract other than a Two-Way Contract.
\item
  ``Standard NBA Contract Conversion Option'' means an option in a Two-Way Contract in favor of a Team to convert the Contract to a Standard NBA Contract that provides for a Salary for each Salary Cap Year equal to the player's applicable Minimum Player Salary and a term equal to the remainder of the original term of the Two-Way Contract, in accordance with Article II, Section 11(f).
\item
  ``Team'' or ``NBA Team'' (see ``Member'').
\item
  ``Team Affiliate'' means:

  \begin{enumerate}
  \def\labelenumii{\roman{enumii}.}
  \tightlist
  \item
    any individual or entity who or which (directly or indirectly) holds an ownership interest in a Team (other than ownership of publicly-traded securities constituting less than five percent (5\%) of the ownership interests in a Team) (a ``Team Owner'');
  \item
    any individual or entity who or which (directly or indirectly) controls, is controlled by or is under common control with, or who or which is an entity affiliated with or an individual related to, a Team;
  \item
    any individual or entity who or which (directly or indirectly) controls, is controlled by or is under common control with, or who or which is an entity affiliated with or an individual related to, an individual or entity described in Section 1(ttt)(i) or (ii) above; or
  \item
    any entity as to which (x) a Team Owner, or (y) an individual or entity that holds (directly or indirectly) an ownership interest in an entity described in Section 1(ttt)(ii) above,either (a) holds (directly or indirectly) more than five percent (5\%) of its ownership interests, or (b) participates in or influences its management or operations.
    For the purposes of this Section 1(ttt): an individual shall only be deemed to be ``related to'' a Team or another individual or entity if such individual is an officer, director, trustee, or executive employee of such Team or entity, or is a member of such individual's immediate family; and ``controls'' or ``is controlled by'' shall include (without limitation) the circumstance in which an individual or a Team or entity has or can exercise effective control.
  \end{enumerate}
\item
  ``Team Salary'' means, with respect to a Salary Cap Year, the sum of all Salaries attributable to a Team's active and former players plus other amounts as computed in accordance with Article VII, less applicable credit amounts as computed in accordance with Article VII.
\item
  ``Total Salaries'' means the total Salaries included in the Team Salary of all NBA Teams for or with respect to a Salary Cap Year in accordance with this Agreement, other than the Salaries included in the Team Salary of Expansion Teams during their first two (2) Salary Cap Years, as determined in accordance with Article VII. For purposes of this definition, Total Salaries:

  \begin{enumerate}
  \def\labelenumii{\roman{enumii}.}
  \tightlist
  \item
    shall include: (a) all Incentive Compensation excluded from Salaries in accordance with Article VII, Section 3(d) but actually earned by NBA players during such Salary Cap Year, and shall exclude all Incentive Compensation included in Salaries in accordance with Article VII, Section 3(d) but not actually earned by NBA players during such Salary Cap Year; (b) the aggregate Salaries, if any, that are excluded from Team Salaries pursuant to Article VII, Section 4(h); (c) any consideration received by a retired player that is included in Team Salary in accordance with the terms of Article XIII; (d) all Two-Way Player Salaries earned in respect of such Salary Cap Year; and (e) any Exhibit 10 Bonus a Team pays its players pursuant to Article II, Section 3(s)(i);
  \item
    shall be reduced by an amount equal to fifty percent (50\%) of the amount of any reductions made to NBA players' Compensation in respect of such Salary Cap Year for asuspension imposed by the NBA or a Team in accordance with Article VI, Section 1 or Article XLI, Section 4(e); and
  \item
    shall be adjusted consistent with Article VII, Section 7(d)(6)(iv).
  \end{enumerate}
\item
  ``Total Salaries and Benefits'' means the sum of Total Salaries plus Total Benefits.
\item
  ``Traded Player'' means a player whose Player Contract is assigned by one Team to another Team other than by means of the NBA waiver procedure.
\item
  ``Two-Way Contract'' means a Contract between a Two-Way Player and a Team made in accordance with Article II, Section 11. In the event that a Two-Way Contract is converted to a Standard NBA Contract pursuant to the Team's exercise of its Standard NBA Contract Conversion Option, the Contract shall no longer be a Two-Way Contract for the purposes of this Agreement.
\item
  ``Two-Way List'' means the list of players, maintained by the NBA, who have signed Two-Way Contracts and are eligible to provide services to an NBAGL team in accordance with the provisions of this Agreement.
\item
  ``Two-Way Player'' means a player under a Two-Way Contract in accordance with Article II, Section 11.
\item
  ``Two-Way Player Salary'' means, with respect to any Two-Way Contract, the Salary called for under Article II, Section 11(a).
\item
  ``Two-Way Player Conversion Option'' means an option in a Player Contract with an Exhibit 10 in favor of a Team to convert the Contract to a Two-Way Contract in accordance with Article II, Section 3(s)(ii) and Section 11(h).
\item
  ``Uniform Player Contract'' or ``Player Contract'' or ``Contract'' means the standard form of written agreement between a person and a Team required for use in the NBA by Article II, pursuant to which such person is employed by such Team as a professional basketball player.
\item
  ``Unlikely Bonus'' means Incentive Compensation excluded from a player's Salary in accordance with Article VII, Section 3(d).
\item
  ``Unrestricted Free Agent'' means a Free Agent who is not subject to a Team's Right of First Refusal.
\item
  ``Veteran'' or ``Veteran Player'' means a person who has signed at least one Player Contract with an NBA Team.
\item
  ``Veteran Free Agent'' means a Veteran who completed his Player Contract (other than a 10-Day Contract) by rendering the playing services called for thereunder.
\item
  ``Years of Service'' means the number of years of NBA service credited to a player in accordance with the following: a player will be credited with one (1) year of NBA service for each year that he is on an NBA Active List or Inactive List for one (1) or more days during the Regular Season. Notwithstanding the above, a player will not receive credit for a Year of Service for any year in which he: (i) withholds playing services called for by a Player Contract or this Agreement for more than thirty (30) days after the Season begins, or (ii) is a Restricted Free Agent, has been tendered a Qualifying Offer by his Prior Team and the Prior Team has extended the date by which the player may accept the Qualifying Offer until March 1 in accordance with the Article XI, Section 4(c)(i), and has not signed a Player Contract with any Team by March 1. In addition, notwithstanding the above, a player will not receive credit for a Year of Service for being on an NBA Active List or Inactive List as a result of signing a Player Contract that is disapproved by the Commissioner. In no event can a player be credited with more than one (1) Year of Service with respect to any one NBA Season. A Year of Service will be credited to a player at the conclusion of the Salary Cap Year encompassing the Season with respect to which it is being credited. Under no circumstances shall the definition of Years of Service herein be used for purposes of determining a player's years of credited eligibility, benefit, and/or vesting service under any benefit plan or program provided for under Article IV of this Agreement, including, without limitation, the Pension Plan, 401(k) Plan, Health and Welfare Benefit Plan (including the Retiree Medical Plan, HRA Benefit, and tuition reimbursement program), or Post-Career Income Plan. Players shall be credited with Years of Service pursuant to this Section 1(iiii) only in respect of Seasons covered by this Agreement. Years of Service credit for Seasons prior to the 2005 NBA/NBPA Collective Bargaining Agreement shall be determined in accordance with the provisions of the 1999 NBA/NBPA Collective Bargaining Agreement.
\end{enumerate}

\hypertarget{uniform-player-contract}{%
\chapter{UNIFORM PLAYER CONTRACT}\label{uniform-player-contract}}

\hypertarget{required-form.}{%
\section{Required Form.}\label{required-form.}}

The Player Contract to be entered into by each player and the Team by which he is employed shall be a Uniform Player Contract in the form annexed hereto as Exhibit A.

\hypertarget{limitation-on-amendments.}{%
\section{Limitation on Amendments.}\label{limitation-on-amendments.}}

\begin{enumerate}
\def\labelenumi{(\alph{enumi})}
\tightlist
\item
  Except as provided in Sections 3, 6, 7(d), 9, 10, and 12 of this Article, and in Article VII, Section 7 (Extensions, Renegotiations, and Other Amendments) or Article XII (Option Clauses), no amendments to the form of Uniform Player Contract provided for by Section 1 of this Article shall be permitted.
\item
  Notwithstanding Section 2(a) above, except as provided: (i) in Sections 3(f), (i), (j), (l), (m), (n), (o), (p), and (r), and Section 11 of this Article, no amendments to Two-Way Contracts shall be permitted; and (ii) in Sections 3(e), (h), (j), (l), (m), (n), (o), (p), (r), and (s), and Section 11 of this Article, no amendments to Contracts containing an Exhibit 10 shall be permitted. For the avoidance of doubt, in no event may a Team and a player extend, renegotiate, or include an Option Year or Early Termination Option in a Two-Way Contract or a Contract containing an Exhibit 10.
\item
  If a Team and a player enter into (i) a Uniform Player Contract containing an amendment not specifically permitted by this Agreement or (ii) a subsequent amendment to an existing Player Contract where such amendment is not specifically permitted by this Agreement, then such Contract or subsequent amendment, as the case may be, shall be disapproved by the Commissioner and, consequently, rendered null and void.
\end{enumerate}

\hypertarget{allowable-amendments.}{%
\section{Allowable Amendments.}\label{allowable-amendments.}}

In their individual contract negotiations, a player and a Team may amend the provisions of a Uniform Player Contract, but only in the following respects:

\begin{enumerate}
\def\labelenumi{(\alph{enumi})}
\item
  By agreeing upon provisions (to be set forth in Exhibit 1 to a Uniform Player Contract) setting forth the Compensation to be paid or amounts to be loaned to the player for each Season of the Contract for rendering the services and performing the obligations described in such Contract.
\item
  By agreeing upon provisions (to be set forth in Exhibit 1 to a Uniform Player Contract) setting forth lump sum bonuses, and the payment date for each such bonus, to be paid as a result of: (i) the player's execution of a Uniform Player Contract or Extension (a ``signing bonus''); (ii) the player's achievement of agreed-upon benchmarks relating to his performance as a player or the Team's performance during a particular NBA Season, subject to the limitations imposed by Paragraph 3(c) of the Uniform Player Contract and Section 12(d) below; or (iii) the player's achievement of agreed-upon benchmarks relating to his physical condition or academic achievement (e.g., earning a college degree or completion of a certified leadership training program), including the player's attendance at and participation in an off-season summer league and/or an off-season skill and/or conditioning program upon terms and conditions agreed upon by the Team and player (subject to the provisions of Section 12(c) below). Any amendment agreed upon pursuant to subsections (ii) or (iii) of this Section 3(b) must be structured so as to provide an incentive for positive achievement by the player and/or the Team; and any amendment agreed upon pursuant to subsection (ii) must be based upon specific numerical benchmarks or Generally Recognized League Honors. By way of example and not limitation, an amendment agreed upon pursuant to Section 3(b)(ii) may provide for the player to receive a bonus if his free-throw percentage exceeds eighty percent (80\%), but may not provide for the player to receive a bonus if his free-throw percentage improves over his previous Season's percentage. For purposes of any bonus agreed upon pursuant to subsection (ii), the performance benchmarks must be based solely upon official NBA statistics, and the determination of whether a player has earned any such performance bonus shall be made solely by reference to official NBA statistics as published on NBA.com.
\item
  By agreeing upon provisions (to be set forth in Exhibit 1 to a Uniform Player Contract) with respect to extra promotional appearances to be performed by the player (in addition to those required by Paragraph 13 of such Contract) and the Compensation therefor.
\item
  By agreeing upon a Compensation payment schedule (to be set forth in Exhibit 1 to a Uniform Player Contract) different from that provided for by Paragraph 3(a) of the Uniform Player Contract; provided, however, that such amendment shall comply with the provisions of Section 3(b) above (relating to lump sum bonus payments) and Section 13(e) below and, provided, further that: (i) the only such amendment that shall be permitted with respect to any Season in which the player's Compensation is not greater than the Minimum Player Salary shall be as described in Section 6(g) or Section 11(a)(ii) below; and (ii) the only such amendments that shall be permitted with respect to any Season in which the player's Compensation is greater than the Minimum Player Salary shall be as follows: (y) a Uniform Player Contract may provide for the player's Compensation (other than advances pursuant to clause (z) below and amounts paid on a deferred basis in accordance with Article XXV of this Agreement) to be paid in either twelve (12) equal semi-monthly payments or thirty-six (36) equal semi-monthly payments beginning with the first of said payments on November 1 of each year covered by the Contract and continuing with such payments on the first and fifteenth of each month until said Compensation is paid in full; and (z) a Uniform Player Contract that, at the time the Contract is signed, is fully or partially protected for lack of skill and injury or illness for a Season may provide for the player to be paid a portion of his Compensation for such Season, up to the Maximum Advance Amount as defined below, prior to November 1 of such Season. The Maximum Advance Amount for a Season shall equal the lesser of eighty percent (80\%) of the amount of the player's Compensation for such Season that is protected for lack of skill and injury or illness, or fifty percent (50\%) of the player's Base Compensation for such Season; provided that no more than twenty-five percent (25\%) of the player's Base Compensation for such Season may be paid to the player prior to the October 1 immediately preceding the first day of the Regular Season.
\item
  By agreeing upon provisions (to be set forth in Exhibit 1A to a Uniform Player Contract) stating that the Contract is intended to provide for Base Compensation equal to the Minimum Player Salary (with no bonuses of any kind) for each Season of the Contract for rendering the services and performing the obligations described in such Contract, in accordance with Section 6 below.
\item
  By agreeing upon provisions (to be set forth in Exhibit 1B to a Uniform Player Contract) (i) setting forth the Compensation to be paid to the player (with no bonuses of any kind) for each Season of the Contract for rendering the services and performing the obligations described in such Contract as a Two-Way Player, in accordance with Section 11 below (a ``Two-Way Contract''), and (ii) containing a Standard NBA Contract Conversion Option in accordance with Section 11(f) below.
\item
  By agreeing upon provisions (to be set forth in Exhibit 1 or Exhibit 1A to a Uniform Player Contract, as applicable), subject to the provisions of Article XXIV, prohibiting or limiting the Team's right to trade such Contract to another Team.
\item
  By agreeing upon provisions (to be set forth in Exhibit 1 or Exhibit 1A to a Uniform Player Contract, as applicable) stating that a player who, pursuant to Article VII, Section 8(b), cannot be traded without his consent, agrees to eliminate his right to consent to a trade.
\item
  By agreeing upon provisions (to be set forth in Exhibit 2 to a Uniform Player Contract) stating that the Base Compensation provided for by a Uniform Player Contract (as described in Exhibit 1, 1A, or 1B to such Contract) shall be, in whole or in part, and subject to the standard conditions or limitations set forth in Section 4 below (and in the form of Exhibit 2) and any additional conditions or limitations that are negotiated by the player and Team to the extent permitted in accordance with Section 4(l) below, protected (as provided for by, and in accordance with the definitions set forth in, Section 4 below) in the event that such Contract is terminated by the Team by reason of the player's:

  \begin{enumerate}
  \def\labelenumii{(\roman{enumii})}
  \tightlist
  \item
    lack of skill;
  \item
    death not covered by an insurance policy procured by a Team for the player's benefit (``death'');
  \item
    disability or unfitness to play skilled basketball resulting from a basketball-related injury not covered by an insurance policy procured by a Team for the player's benefit (``basketball-related injury''), or disability or unfitness to play skilled basketball resulting from any injury or illness not covered by an insurance policy procured by a Team for the player's benefit (``injury or illness''), provided that a Contract can contain protection in only one of the two categories set forth in this Section 3(i)(iii), and further provided that, for clarity and without limitation, protection for injury or illness shall not include protection for mental disability; and/or
  \item
    mental disability not covered by an insurance policy procured by a Team for the player's benefit (``mental disability'').
  \end{enumerate}
\item
  By agreeing upon provisions (to be set forth in Exhibit 3 to a Uniform Player Contract) limiting or eliminating the player's right to receive his Base Compensation (in accordance with Paragraphs 7(c), 16(a)(iii), and 16(b) of the Uniform Player Contract) when the player's disability or unfitness to play skilled basketball is caused by the re-injury of one or more injuries sustained prior to, or by the aggravation of one or more conditions that existed prior to, the execution of the Uniform Player Contract providing for such Base Compensation. Notwithstanding the foregoing, the provisions set forth in Exhibit 3 to a Uniform Player Contract shall not apply for a Season in the event such Contract is terminated during the period from the February 1 of such Season through the end of that Season.
\item
  By agreeing upon provisions (to be set forth in Exhibit 4 to a Uniform Player Contract), subject to the provisions of Article XXIV, entitling a player to earn Compensation if such player's Uniform Player Contract is traded to another NBA team.
\item
  By agreeing upon provisions (to be set forth in Exhibit 5 to a Uniform Player Contract) permitting the player to participate or engage in some or all of the activities otherwise prohibited by Paragraph 12 of the Uniform Player Contract; provided, however, that no amendment to Paragraph 12 of the Uniform Player Contract shall permit a player to participate in any public game or public exhibition of basketball not approved in accordance with Article XXIII of this Agreement.
\item
  By agreeing upon provisions (to be set forth in Exhibit 6 to a Uniform Player Contract) establishing that the player must report for and submit to a physical examination to be performed by a physician designated by the Team, subject to the provisions of Section 13(h) below.
\item
  By agreeing to delete Paragraph 7(b) of the Uniform Player Contract in its entirety and substituting therefor the provision set forth in Exhibit 7 to a Uniform Player Contract.
\item
  By agreeing either (i) to delete Paragraph 13(b) of the Uniform Player Contract in its entirety, or (ii) to delete the last sixteen (16) words of the first sentence of Paragraph 13(b) of such Contract.
\item
  By agreeing upon provisions for the purpose of terminating an already-existing Uniform Player Contract prior to the expiration of its stated term, stating as follows: (i) the Team will request waivers on the player in accordance with Paragraph 16 of the Contract immediately following the Commissioner's approval of such amendment; and (ii) should the player clear waivers and his Contract thereupon be terminated (x) the amount of any Compensation protection contained in the Contract will immediately be reduced or eliminated, and/or (y) the Team's right of set-off under Article XXVII of this Agreement will be modified or eliminated.
\item
  By agreeing upon provisions (to be set forth in Exhibit 8 to a Uniform Player Contract) stating that the Contract will be traded to another team within forty-eight (48) hours of its execution or amendment, with such trade and the consummation of such trade to be conditions precedent to the validity of the Contract or an amendment thereto; provided, however, that any such sign-and-trade transaction must comply with Article VII, Section 8(e).
\item
  By agreeing upon provisions (to be set forth in Exhibit 9 to a Uniform Player Contract) eliminating the player's right to receive his Base Compensation (in accordance with Paragraphs 7(c), 16(a)(iii), and 16(b) of the Uniform Player Contract) in the event the Contract is terminated prior to the first day of the Regular Season covered by such Contract; provided, however, that such amendment shall be permitted only if: (i) the Contract is for one (1) Season in length, provides for the Minimum Player Salary (with no bonuses of any kind) or Two-Way Salary and does not provide for Compensation protection of any kind pursuant to Section 3(i) above (a ``Non-Guaranteed, Training Camp Contract''); (ii) at the time of signing the Non-Guaranteed, Training Camp Contract, the Team has no fewer than fourteen (14) players signed to Player Contracts (not including any player signed to a Two-Way Contract or a Non-Guaranteed, Training Camp Contract) on the Team's roster in respect of the upcoming (or, after the first day of training camp, the then-current) Season; and (iii) no Team may be a party at any one time to more than six (6) Non-Guaranteed, Training Camp Contracts.
\item
  By agreeing upon provisions (to be set forth in Exhibit 10 to a Uniform Player Contract), subject to Section 11(h) below:

  \begin{enumerate}
  \def\labelenumii{(\roman{enumii})}
  \tightlist
  \item
    entitling a player to receive a bonus (the ``Exhibit 10 Bonus'') in an amount between \$5,000 and the ``Maximum Exhibit 10 Bonus Amount'' (defined below) for the Salary Cap Year in which the Contract is signed if (1) the Contract is terminated by the Team in accordance with the NBA waiver procedure prior to the first day of the Regular Season, and (2) the player (a) signs with the NBAGL prior to the deadline set by the NBAGL for NBAGL teams to designate affiliate players, (b) is initially assigned by the NBAGL to such Team's NBAGL affiliate as listed in Exhibit 10 and timely reports to such affiliate, and (c) does not leave the NBAGL (e.g., by buying out his contract with the NBAGL and signing a contract with an international team) prior to providing sixty (60) consecutive days of service during the NBAGL Season (the ``60-Day Service Period''), provided that, in the event the player is signed to one or more Contract(s) by the Team prior to completing the 60-Day Service Period, the player shall still satisfy this clause (c) if he timely returns to such Team's NBAGL affiliate upon the completion or termination of such Contract(s) and completes the outstanding portion of the 60-Day Service Period. For clarity, a player will not satisfy this clause (c) if at any time prior to completing the 60-Day Service Period he signs a contract with a professional basketball team other than the Team. In the event a player fails to satisfy clause (c) above as a result of an injury resulting directly from his playing for the Team's NBAGL affiliate, such player shall nonetheless be entitled to receive the Exhibit 10 Bonus set forth in his Contract. Notwithstanding anything to the contrary in the foregoing, an Exhibit 10 may only contain an Exhibit 10 Bonus if the Team has an NBAGL affiliate at the time of the execution of the Contract; provided, however, that if a Team with an NBAGL affiliate acquires by assignment a Contract with a Conversion Protection Amount but without an Exhibit 10 Bonus (the ``Acquired Exhibit 10''), the Acquired Exhibit 10 shall be deemed to include an Exhibit 10 Bonus equal to the Conversion Protection Amount; and
  \item
    stating that, if prior to the first day of the Regular Season (A) the Team exercises the Two-Way Player Conversion Option in accordance with Section 11(h) below, and/or (B) the Contract is not terminated by the Team, the Compensation provided for by the Contract will be protected for lack of skill and injury or illness in an amount (the ``Conversion Protection Amount'') between \$5,000 and the Maximum Exhibit 10 Bonus Amount; provided, however, that if the Exhibit 10 contains an Exhibit 10 Bonus, the Exhibit 10 must also contain a Conversion Protection Amount and the Conversion Protection Amount must be equal to the Exhibit 10 Bonus.
  \end{enumerate}

  The ``Maximum Exhibit 10 Bonus Amount'' shall be: (1) \$75,000 for the 2023-24 Salary Cap Year, and (2) for each subsequent Salary Cap Year, \$75,000 multiplied by a fraction, the numerator of which is the Salary Cap for the applicable Salary Cap Year and the denominator of which is the Salary Cap for the 2023-24 Salary Cap Year.

  In the event that NBAGL rules permit a Team, other than the Team that last requested waivers on the player, to designate the player as an affiliate player (the ``Designating Team''), the Designating Team shall be responsible for paying the Exhibit 10 Bonus to the player provided that (a) the Designating Team designates the player as an affiliate player, (b) prior to the waiver, the Designating Team was a party to the Contract containing the Exhibit 10 Bonus, and (c) the player satisfies the conditions set forth in Section 3(s)(i) above with respect to the Designating Team's NBAGL affiliate.

  With respect to a player, if the NBAGL affiliate of an NBA Team is permitted, pursuant to NBAGL rules, to designate the player as a returning player (such NBA Team, the ``Returning Rights Team''), then any Team other than the Returning Rights Team shall, prior to entering into a Contract containing an Exhibit 10 with the player, be required to provide written notice to the player (with a copy to the Players Association) that, pursuant to NBAGL rules, the NBAGL affiliate of the Returning Rights Team holds the right to so designate the player. The NBA shall impose a fine of no less than \$25,000 on any Team that fails to provide the notice required by this paragraph.

  No Team may (a) be a party at any one time to more than six (6) Contracts containing an Exhibit 10, or (b) enter into a Player Contract with an Exhibit 10 unless such Contract is for one (1) Season in length, provides for the Minimum Player Salary (with no bonuses of any kind other than the Exhibit 10 Bonus), and does not provide for Compensation protection of any kind pursuant to Section 3(i) above (other than in connection with Section 3(s)(ii) above).

  A Team may enter into a Contract with both an Exhibit 9 and an Exhibit 10 in accordance with the preceding terms; provided, however, that if a Team exercises its Two-Way Player Conversion Option, the Contract's Exhibit 9 shall be rendered null and void and of no further force or effect upon the exercise of such Two-Way Player Conversion Option.
\end{enumerate}

\hypertarget{compensation-protection.}{%
\section{Compensation Protection.}\label{compensation-protection.}}

\begin{enumerate}
\def\labelenumi{(\alph{enumi})}
\tightlist
\item
  \textbf{Lack of Skill.} When a Team agrees to protect, in whole or in part, the Base Compensation provided for by a Uniform Player Contract in the event such Contract is terminated by the Team, pursuant to Paragraph 16(a)(iii) thereof, by reason of the player's lack of skill, such agreement shall mean that, subject to any conditions or limitations set forth in this Section 4(a) or Exhibit 2 to the Uniform Player Contract, or expressly set forth elsewhere in this Agreement, notwithstanding the provisions of Paragraphs 16(a)(iii), 16(d), 16(e), and 16(g) of such Contract, the termination of such Contract by the Team on account of the player's failure to exhibit sufficient skill or competitive ability shall in no way affect the player's right to receive, in whole or in part, the Base Compensation payable pursuant to Exhibit 1 (or Exhibit 1A or Exhibit 1B, as applicable) to such Contract in the amounts and at the times called for by such Exhibit; provided, however, that: (i) such lack of skill does not result from the player's participation in activities prohibited by Paragraph 12 of the Uniform Player Contract (as such Paragraph may be modified by Exhibit 5 to the Player Contract), attempted suicide, intentional self-inflicted injury, abuse of alcohol, use of any Prohibited Substance or controlled substance, abuse of or addiction to prescription drugs, conduct occurring during the commission of any felony for which the player is convicted (including by a plea of guilty, no contest, or nolo contendere), participation in any riot, insurrection, or war or other military activities, or failure to comply with the requirements of Paragraphs 7(d)-(i) of the Uniform Player Contract; (ii) at the time of the player's failure to render playing services, the player is not in material breach of such Contract; (iii) if the Team, for its own benefit, seeks to procure an insurance policy covering the player's lack of skill, the player cooperates with the Team in procuring such an insurance policy, including by, among other things, supplying all information requested of him, completing application forms, or otherwise, and submitting to all examinations and tests requested of him by or on behalf of the insurance company in connection with the Team's efforts to procure such policy; and (iv) if the Team, for its own benefit, has procured such an insurance policy, the player cooperates (in the manner described above) with the Team and insurance company in the processing of the Team's claim under such policy.
\item
  \textbf{Death.} When a Team agrees to protect, in whole or in part, the Base Compensation provided for by a Uniform Player Contract in the event such Contract is terminated by the Team, pursuant to Paragraph 16(a)(iv) thereof, by reason of the player's failure to render his services thereunder, if such failure has been caused by the player's death, such agreement shall mean that, subject to any conditions or limitations set forth in this Section 4(b) or Exhibit 2 to the Uniform Player Contract, or expressly set forth elsewhere in this Agreement, notwithstanding the provisions of Paragraphs 16(a)(iii), 16(b), 16(c), 16(d), 16(e), and 16(g) of such Contract, the termination of such Contract by the Team shall in no way affect the player's (or his estate's or duly appointed beneficiary's) right to receive, in whole or in part, the Base Compensation payable pursuant to Exhibit 1 (or Exhibit 1A or Exhibit 1B, as applicable) to such Contract in the amounts and at the times called for by such Exhibit; provided, however, that: (i) such death does not result from the player's participation in activities prohibited by Paragraph 12 of the Uniform Player Contract (as such Paragraph may be modified by Exhibit 5 to the Player Contract), suicide, intentional self-inflicted injury, abuse of alcohol, use of any Prohibited Substance or controlled substance, abuse of or addiction to prescription drugs, conduct occurring during the commission of any felony for which the player is convicted (including by a plea of guilty, no contest, or nolo contendere), participation in any riot, insurrection, or war or other military activities, or failure to comply with the requirements of Paragraphs 7(d)-(i) of the Uniform Player Contract; (ii) at the time of the player's failure to renderplaying services, the player is not in material breach of such Contract; (iii) if the Team, for its own benefit, seeks to procure an insurance policy covering the player's death, the player cooperates with the Team in procuring such an insurance policy, including by, among other things, supplying all information requested of him, completing application forms, or otherwise, and submitting to all examinations and tests requested of him by or on behalf of the insurance company in connection with the Team's efforts to procure such policy; and (iv) if the Team, for its own benefit, has procured such an insurance policy, the player's estate and/or duly appointed beneficiary cooperates (in the manner described above) with the Team and insurance company in the processing of the Team's claim under such policy.
\item
  \textbf{Basketball-Related Injury.} When a Team agrees to protect, in whole or in part, the Base Compensation provided for by a Uniform Player Contract in the event such Contract is terminated by the Team, pursuant to Paragraph 16(a)(iv) thereof, by reason of the player's failure to render his services thereunder, if such failure has been caused by the player's disability and/or unfitness to play skilled basketball as a direct result of an injury sustained while participating in any basketball practice or game played for the Team, such agreement shall mean that, subject to any conditions or limitations set forth in this Section 4(c) or Exhibit 2 to the Uniform Player Contract, or expressly set forth elsewhere in this Agreement, notwithstanding the provisions of Paragraphs 7(b), 7(c), 16(a)(iii), 16(b), 16(c), 16(d), and 16(g) of such Contract, the termination of such Contract by the Team shall in no way affect the player's right to receive, in whole or in part, the Base Compensation payable pursuant to Exhibit 1 (or Exhibit 1A or Exhibit 1B, as applicable) to such Contract in the amounts and at the times called for by such Exhibit; provided, however, that: (i) such injury does not result from the player's participation in activities prohibited by Paragraph 12 of the Uniform Player Contract (as such Paragraph may be modified in Exhibit 5 to the Player Contract), attempted suicide, intentional self-inflicted injury, abuse of alcohol, use of any Prohibited Substance or controlled substance, abuse of or addiction to prescription drugs, conduct occurring during the commission of any felony for which the player is convicted (including by a plea of guilty, no contest, or nolo contendere), participation in any riot, insurrection, or war or other military activities, or failure to comply with the requirements of Paragraphs 7(d)-(i) of the Uniform Player Contract; (ii) at the time of the player's termination, the player is not in material breach of such Contract; (iii) if the Team, for its own benefit, seeks to procure an insurance policy covering the player's injury, the player cooperates with the Team in procuring such an insurance policy, including by, among other things, supplying all information requested of him, completing application forms, or otherwise, and submitting to all examinations and tests requested of him by or on behalf of the insurance company in connection with the Team's efforts to procure such policy; and (iv) if the Team, for its own benefit, has procured such an insurance policy, the player cooperates (in the manner described above) with the Team and the insurance company in the processing of the Team's claim under such policy.
\item
  \textbf{Injury or Illness.} When a Team agrees to protect, in whole or in part, the Base Compensation provided for by a Uniform Player Contract in the event such contract is terminated by the Team, pursuant to Paragraph 16(a)(iv) thereof, by reason of the player's failure to render his services thereunder, if such failure has been caused by an injury, illness, or disability suffered or sustained by the player, such agreement shall mean that, subject to any conditions or limitations set forth in this Section 4(d) or Exhibit 2 to the Uniform Player Contract, or expressly set forth elsewhere in this Agreement, notwithstanding the provisions of Paragraphs 7(b), 7(c), 16(a)(iii), 16(b), 16(c), 16(d), and 16(g) of such Contract, the termination of such Contract by the Team shall in no way affect the player's right to receive, in whole or in part, the Base Compensation payable pursuant to Exhibit 1 (or Exhibit 1A or Exhibit 1B, as applicable) to such Contract in the amounts and at the times called for by such Exhibit; provided, however, that: (i) such injury, illness, or disability does not result from the player's participation in activities prohibited by Paragraph 12 of the Uniform Player Contract (as such Paragraph may be modified in Exhibit 5 to the Player Contract), attempted suicide, intentional self-inflicted injury, abuse of alcohol, use of any Prohibited Substance or controlled substance, abuse of or addiction to prescription drugs, conduct occurring during the commission of any felony for which the player is convicted (including by a plea of guilty, no contest, or nolo contendere), participation in any riot, insurrection, or war or other military activities, or failure to comply with the requirements of Paragraphs 7(d)-(i) of the Uniform Player Contract; (ii) at the time of such injury, illness, or disability the player is not in material breach of such Contract; (iii) if the Team, for its own benefit, seeks to procure an insurance policy covering the player's injury and/or illness, the player cooperates with the Team in procuring such an insurance policy, including by, among other things, supplying all information requested of him, completing application forms, or otherwise, and submitting to all examinations and tests requested of him by or on behalf of the insurance company in connection with the Team's efforts to procure such policy; and (iv) if the Team, for its own benefit, has procured such an insurance policy, the player cooperates (in the manner described above) with the Team and insurance company in the processing of the Team's claim under such policy.
\item
  \textbf{Mental Disability.} When a Team agrees to protect, in whole or in part, the Base Compensation provided for by a Uniform Player Contract in the event such Contract is terminated by the Team, pursuant to Paragraph 16(a)(iv) thereof, by reason of the player's failure to render his services thereunder, if such failure has been caused by the player's mental disability, such agreement shall mean that, subject to any conditions or limitations set forth in this Section 4(e) or Exhibit 2 to the Uniform Player Contract, or expressly set forth elsewhere in this Agreement, notwithstanding the provisions of Paragraphs 16(a)(iii), 16(b), 16(c), 16(d), 16(e), and 16(g) of such Contract, the termination of such Contract by the Team shall in no way affect the player's (or his duly appointed legal representative's) right to receive, in whole or in part, the Base Compensation payable pursuant to Exhibit 1 (or Exhibit 1A or Exhibit 1B, as applicable) to such Contract in the amounts and at the times called for by such Exhibit; provided, however, that: (i) such mental disability does not result from the player's participation in activities prohibited by Paragraph 12 of the Uniform Player Contract (as such Paragraph may be modified in Exhibit 5 to the Player Contract), attempted suicide, intentional self-inflicted injury, the use of any Prohibited Substance or controlled substance, abuse of or addiction to prescription drugs, conduct occurring during the commission of any felony for which the player is convicted (including by a plea of guilty, no contest, or nolo contendere), participation in any riot, insurrection, or war or other military activities, or failure to comply with the requirements of Paragraphs 7(d)-(i) of the Uniform Player Contract; (ii) at the time of the player's failure to render playing services, the player is not in material breach of such Contract; (iii) if the Team, for its own benefit, seeks to procure an insurance policy covering the player's mental disability, the player (and/or his duly appointed legal representative) cooperates with the Team in procuring such an insurance policy, including by, among other things, supplying all information requested of him, completing application forms, or otherwise, and submitting to all examinations and tests requested of him by the insurance company in connection with the Team's efforts to procure such policy; and (iv) if the Team, for its own benefit, has procured such an insurance policy, the player (and/or his duly appointed legal representative) cooperates (in the manner described above) with the Team and insurance company in the processing of the Team's claim under such policy.
\item
  No agreement by a Team to protect, in whole or in part, the Base Compensation provided for by a Uniform Player Contract shall require (or be construed as requiring) such Team to continue to employ the player (whether on the Active List, Inactive List, Two-Way List, or otherwise); nor shall any such agreement afford the player any right to be employed, or to be deemed as having been employed, by such Team for any purpose.
\item
  Notwithstanding any other provision of this Agreement, when a Team agrees to protect, in whole or in part, the Base Compensation provided for by a Uniform Player Contract, and such protection is contingent on the satisfaction of a condition expressly set forth in Exhibit 2 to that Contract, such protection shall be applicable and effective only if the Player Contract has not previously been terminated at the time such condition is satisfied.
\item
  Notwithstanding any other provision of this Agreement, when a Team agrees to protect, in whole or in part, the Base Compensation provided for in any Option Year in favor of the Team included in a Uniform Player Contract, such protection shall be applicable and effective only if the option to extend the term provided for in the Contract was exercised by the Team prior to the termination of the Contract. When a Team agrees to protect, in whole or in part, the Base Compensation provided for in any Option Year in favor of the player, the applicability of such protection in the circumstance where the Option has not been exercised by the player shall be governed by the provisions of Article XII, Section 2(a).
\item
  During the term of a Player Contract, the percentage of protected Base Compensation for any future Season shall not exceed the percentage of unearned protected Base Compensation for any prior Season. Thus, for example, a Team could not provide for fifty percent (50\%) Base Compensation protection in the first Season of a Player Contract and one hundred percent (100\%) Base Compensation protection in the second Season of the Contract. However, the foregoing rule does not prevent a Team from providing a percentage of Base Compensation protection in a future Season that is higher than in a prior Season if the higher level of Base Compensation for the future Season is conditional and the condition cannot be satisfied until the completion of the prior Season. For example, it is permissible for a Contract to provide that Base Compensation protection for the first Season of a Player Contract equals fifty percent (50\%) and Base Compensation protection for the second Season will be increased from fifty percent (50\%) to one hundred percent (100\%) if the player is on the Team's roster as of the August 1 prior to the second Season of the Player Contract.
\item
  With respect to Player Contracts entered into or extended on or after the effective date of this Agreement:

  \begin{enumerate}
  \def\labelenumii{(\roman{enumii})}
  \tightlist
  \item
    The maximum amount of aggregate Base Compensation that can be protected for death is thirty million dollars (\$30,000,000); and
  \item
    If a player (other than a player signed to a Contract that provides in any Season for the player to earn Compensation equal to his applicable Minimum Player Salary that (x) is signed after the first day of the Regular Season, or (y) does not provide for full Base Compensation protection for lack of skill and injury or illness for the first Season of such Contract) elects to purchase term life insurance for his benefit, his Team shall be permitted to reimburse him each Season for the premiums paid for such insurance with respect to such Season and any other future Season(s); provided, however, that:

    \begin{enumerate}
    \def\labelenumiii{(\Alph{enumiii})}
    \tightlist
    \item
      The amount of coverage for which premiums are reimbursed by the Team in any Season shall not exceed the lesser of (x) the aggregate amount of the player's unearned Base Compensation for such Season and each remaining Season (excluding an Option Year if not yet exercised) that is not protected for death, and (y) the difference between (i) eighty-five million dollars (\$85,000,000) and (ii) the aggregate amount of the player's unearned Base Compensation for such Season and each remaining Season (excluding an Option Year if not yet exercised) that is protected for death.
    \item
      Any such premium reimbursement shall not exceed the cost for ten-year guaranteed term coverage at preferred rates.
    \end{enumerate}
  \item
    If a Contract contains death protection covering ten million dollars (\$10,000,000) or more of Base Compensation, the player shall be precluded from purchasing life insurance for a period of ninety (90) days following the execution or extension (as applicable) of the Contract or until such earlier time as the Team notifies the player in writing that it is no longer attempting to purchase life insurance coverage on the player (up to the amount of the player's Base Compensation protection for death) for the Team's benefit. During such ninety (90) day period or until such time as the Team issues the foregoing written notification to the player, the Team's efforts to purchase life insurance on the player for the Team's benefit shall be conducted diligently and in good faith.
  \end{enumerate}
\item
  With respect to Player Contracts entered into or extended on or after the effective date of this Agreement, in the event that a Team terminates a Player Contract (resulting in the player's separation of service from the Team), and the Team is obligated thereafter to make payments to the player pursuant to Exhibit 2 of the Contract, such payments shall be made in accordance with the following schedule:

  \begin{enumerate}
  \def\labelenumii{(\roman{enumii})}
  \tightlist
  \item
    If, as of the date of the player's separation from service, the aggregate Base Compensation owed to the player pursuant to Exhibit 2 of the Contract is five hundred thousand dollars (\$500,000) or less, such amount shall be paid in accordance with the semi-monthly installments prescribed by the payment schedule set forth in the Contract. Each installment shall equal the amount of Base Compensation that was due per pay period for the applicable Season immediately before the Player's separation until the aggregate amount of the remaining Base Compensation owed to the player pursuant to Exhibit 2 of the Contract is paid in full.
  \item
    If, as of the date of the player's separation from service, the aggregate Base Compensation owed to the player pursuant to Exhibit 2 of the Contract exceeds five hundred thousand dollars (\$500,000), such amount shall be paid as follows:

    \begin{enumerate}
    \def\labelenumiii{(\alph{enumiii})}
    \setcounter{enumiii}{23}
    \tightlist
    \item
      The Base Compensation, if any, owed to the player pursuant to Exhibit 2 of the Contract with respect to the ``current season'' (as defined below) at the time when the request for waivers on the player is made shall be paid in accordance with the payment schedule set forth in the Contract. Each installment shall equal the amount of Base Compensation that was due per pay period immediately before the player's separation until the aggregate amount of the remaining Base Compensation owed to the player pursuant to Exhibit 2 of the Contract with respect to the current season is paid in full. For purposes of this subparagraph (x) and subparagraph (y) below only, the ``current season'' means the period from September 1 through June 30.
    \item
      The remaining Base Compensation, if any, owed to the player pursuant to Exhibit 2 of the Contract shall be aggregated and paid in equal amounts per year over a period equal to twice the number of NBA Seasons (including any Season covered by a Player Option Year) remaining on this Contract following the date upon which the request for waivers occurred, plus one NBA Season. For this purpose, if the request for waivers is made during the period from September 1 through June 30, the number of NBA Seasons remaining on this Contract shall not include the current season (as defined in subparagraph (x) above). The rescheduled payments described above shall be paid over the applicable number of NBA Seasons in equal semi-monthly installments on the pay dates prescribed by Paragraph 3(a) of the Uniform Player Contract.
    \end{enumerate}

    The following example is for clarity. A player has four (4) Seasons remaining on his Contract with protected Base Compensation of the following amounts: \$4 million in Season 1, \$4.3 million in Season 2, \$4.7 million in Season 3, and \$5 million in Season 4. The player is waived on December 1 of Season 1. Under Section 4(k)(ii)(x) above, the player would receive the remainder of his \$4 million in Base Compensation for Season 1 in accordance with the payment schedule set forth in his Contract. Under Section 4(k)(ii)(y) above, the \$14 million of protected Base Compensation remaining to be paid for Seasons 2-4 of the Contract would be paid at a rate of \$2 million per Season for the next seven (7) Seasons in accordance with the payment schedule set forth in Paragraph 3 of the Contract. If the same player is instead waived on July 30 prior to Season 1, the \$18 million of protected Base Compensation remaining to be paid for Seasons 1-4 of the Contract would be paid -- under Section 4(k)(ii) above -- at a rate of \$2 million per Season for the next nine (9) Seasons in accordance with the payment schedule set forth in Paragraph 3 of the Contract.
  \end{enumerate}
\item
  In addition to the standard conditions or limitations set forth above in this Section 4 (as set forth in the form of Exhibit 2 to the Uniform Player Contract), a Team and a player are authorized under Article II, Sections 4(a)-(e) to negotiate additional conditions or limitations applicable to the player's Compensation protection for such categories as the Team and player agree to protect that relate to only the following: (i) whether the Team waives a player by a certain time (e.g., providing that a player's Base Compensation protection increases if the Team does not request waivers on the player by a certain date); (ii) achievement of certain benchmarks relating to Team and/or player performance or a player's physical condition (e.g., providing that a player's Base Compensation protection increases if the player achieves certain performance criteria or meets specified weigh-in criteria), provided that any such performance benchmarks must be based solely upon official NBA statistics, the determination of whether a player has met any such performance benchmark shall be made solely by reference to official NBA statistics as published on NBA.com, and any amendment agreed upon pursuant to this subsection is structured so as to provide an incentive for positive achievement by the Team and/or the player; (iii) a player experiencing a particular injury, illness, or other medical condition (e.g., providing that a player's Base Compensation protection does not apply if the Team terminates a Contract due to a particular injury to a player's left knee); and (iv) the Team's ability to obtain insurance, using best efforts, of a certain type and dollar amount within a specified period of time following execution or extension (as applicable) of the Contract. Other than the standard conditions or limitations set forth above in this Section 4 (as set forth in the form of Exhibit 2 to the Uniform Player Contract) and any individually-negotiated conditions or limitations in accordance with this Section (l), no Player Contract entered into or extended on or after the effective date of this Agreement (but in the case of Extensions only with respect to the extended term) may contain any additional condition or limitation of any kind on a player's Compensation protection.
\end{enumerate}

\hypertarget{conformity.}{%
\section{Conformity.}\label{conformity.}}

\begin{enumerate}
\def\labelenumi{(\alph{enumi})}
\tightlist
\item
  All currently effective Player Contracts, and all Player Contracts entered into on or after the effective date of this Agreement that do not otherwise so provide, shall be deemed amended in such manner to require the parties to comply with all terms of this Agreement, including the terms of the Uniform Player Contract annexed hereto as Exhibit A. All Player Contracts shall be subject to the terms of this Agreement, which shall supersede the terms of any Player Contract inconsistent herewith. No Player Contract shall provide for the waiver by a player or a Team of any benefits or the sacrifice of any rights to which the player or the Team is entitled by virtue of a Uniform Player Contract or this Agreement.
\item
  Notwithstanding Section 5(a) above, no Player Contract entered into prior to the effective date of this Agreement shall be affected by any provisions of this Agreement expressly indicating that they apply only to Player Contracts entered into on or after the effective date of this Agreement.
\end{enumerate}

\hypertarget{minimum-player-salary.}{%
\section{Minimum Player Salary.}\label{minimum-player-salary.}}

\begin{enumerate}
\def\labelenumi{(\alph{enumi})}
\item
  Except with respect to 10-Day Contracts provided for in Section 9 below, Rest-of-Season Contracts provided for in Section 10 below, and Two-Way Contracts provided for in Section 11 below, no Player Contract shall provide for a Salary of less than the applicable scale amount contained in the Minimum Annual Salary Scale applicable for such Salary Cap Year. The Minimum Annual Salary Scale applicable to a player's Contract is determined by the Salary Cap Year encompassing the first Season covered by the Contract. Accordingly, for example, if the first Season covered by a player's Contract is the 2023-24 Season, then the Minimum Annual Salary Scale for the 2023-24 Salary Cap Year shall apply for each Season of the Contract.
\item
  No 10-Day Contract or Rest-of-Season Contract (as those terms are defined in Sections 9 and 10 below) shall provide for a Salary of less than the Minimum Player Salary applicable to that player.
\item
  In determining whether a Player Contract provides for a Salary of no less than the Minimum Player Salary applicable to that player, the allocation of a deemed signing bonus in respect of an ``international player payment'' in excess of the Excluded International Player Payment Amount for such Salary Cap Year as set forth in Article VII, Section 3(e) (but no other bonuses) shall be considered as part of the Salary provided for by a Player Contract, provided that such Player Contract makes clear that the Salary for each Season (including the allocation of any such deemed signing bonus) equals or exceeds the Minimum Player Salary for such Season.
\item
  On July 1 of each Salary Cap Year, any Player Contract (other than a Two-Way Contract), whether entered into before or after the effective date of this Agreement, that provides for a Salary for the upcoming Season that is less than the applicable Minimum Player Salary based on the Minimum Annual Salary Scale applicable to the player's Contract shall be deemed amended to provide for the applicable Minimum Player Salary based on such Minimum Annual Salary Scale.
\item
  Nothing in this Section 6 shall alter the respective rights and liabilities of a player and a Team, as provided for in the Uniform Player Contract or in this Agreement, with respect to the termination of a Player Contract.
\item
  Every Contract entered into between a player and Team that is intended to provide for Compensation equal to the Minimum Player Salary (with no bonuses of any kind) for each Season must contain the following sentence in Exhibit 1A of such Contract and shall be deemed amended in the manner described in such sentence: ``This Contract is intended to provide for a Base Compensation for the \_\_\_\_\_\_\_\_\_\_\_\_ Season(s) equal to the Minimum Player Salary for such Season(s) (with no bonuses of any kind) and shall be deemed amended to the extent necessary to so provide.'' The reference in the preceding sentence to ``no bonuses of any kind'' shall not be construed to limit the ability of a Team and player (i) to agree upon provisions entitling a player to earn Compensation if such player's Uniform Player Contract is traded to another NBA team in accordance with Section 3(k) above, or (ii) to enter into a Contract with an Exhibit 10 Bonus, subject to the limitations in Section 3(s) above and Section 11(h) below.
\item
  A Uniform Player Contract (other than a Two-Way Contract) that provides in any Season for the player to earn Compensation not greater than his applicable Minimum Player Salary (with no bonuses of any kind) that, at the time the Contract is signed, is fully or partially protected for lack of skill and injury or illness may be amended to provide for the player to be paid a portion of his Compensation for such Season (the ``Advance''), up to the Minimum Player Salary Advance Limit as defined below, prior to November 1 of such Season. The Minimum Player Salary Advance Limit for a Season shall equal the lesser of (i) eighty percent (80\%) of the amount of the player's Compensation for such Season that is protected for lack of skill and injury or illness, or (ii) seven and one half percent (7.5\%) of the player's Base Compensation for such Season. Any Advance paid to a player for a Season pursuant to the foregoing must be deducted in full from the first installment of Current Base Compensation (i.e., on November 1) and, if necessary after reducing in full the first installment, the second installment of Current Base Compensation (i.e., on November 15) for such Season that the player would have received pursuant to Paragraph 3(a) of the Contract had there been no such Advance. To effectuate the requirement set forth in the preceding sentence, every such Contract that provides for an Advance must contain the following language (and only such language) under the ``Payment Schedule'' heading in Exhibit 1 or Exhibit 1A (as applicable) with respect to each applicable Season:

  \begin{quote}
  ``Player's Current Base Compensation with respect to the \_\_\_\_\_\_\_\_\_ Season(s) shall be paid in accordance with Paragraph 3(a), except that the November 1 installment of such Current Base Compensation and, if necessary after reducing in full the November 1 installment, the November 15 installment of such Current Base Compensation shall be reduced by \${[}amount of Advance{]}, which amount shall be paid to Player in advance on {[}date{]}.''
  \end{quote}
\end{enumerate}

\hypertarget{maximum-annual-salary.}{%
\section{Maximum Annual Salary.}\label{maximum-annual-salary.}}

\begin{enumerate}
\def\labelenumi{(\alph{enumi})}
\tightlist
\item
  Notwithstanding any other provision of this Agreement, no Player Contract entered into on or after the effective date of this Agreement may provide for a Salary plus Unlikely Bonuses in the first Season covered by the Contract that exceeds the following amounts:

  \begin{enumerate}
  \def\labelenumii{(\roman{enumii})}
  \tightlist
  \item
    for any player who has completed fewer than seven (7) Years of Service, the greater of (x) twenty-five percent (25\%) of the Salary Cap in effect at the time the Contract is executed, or (y) one hundred five percent (105\%) of the Salary for the final Season of the player's prior Contract; provided, however, that a player who has four (4) Years of Service as of the June 30 following the end of the last Season covered by his Player Contract (``5th Year Eligible Players'') shall be eligible to receive from his Prior Team up to thirty percent (30\%) of the Salary Cap in effect at the time the Contract is executed if the player has met at least one of the following criteria (the ``Higher Max Criteria'') as of the July 1 following the player's fourth Season:

    \begin{enumerate}
    \def\labelenumiii{(\Alph{enumiii})}
    \tightlist
    \item
      the player was named to the All-NBA first, second, or third team, or was named Defensive Player of the Year, in the immediately preceding Season or in two (2) Seasons during the immediately preceding three (3) Seasons; or
    \item
      the player was named NBA MVP during one of the immediately preceding three (3) Seasons;
    \end{enumerate}
  \item
    for any player who has completed at least seven (7) but fewer than ten (10) Years of Service, the greater of (x) thirty percent (30\%) of the Salary Cap in effect at the time the Contract is executed, or (y) one hundred five percent (105\%) of the Salary for the final Season of the player's prior Contract; provided, however, that a player who has eight (8) or nine (9) Years of Service at the time the Contract is executed and rendered such Years of Service for the Team with which he first executed a Player Contract (or, if he was under a Player Contract for more than one Team during such period, changed Teams only by trade during the first four (4) Salary Cap Years in which he was under a Player Contract) shall be eligible to enter into a Designated Veteran Player Contract pursuant to which he receives from his Prior Team up to thirty-five percent (35\%) of the Salary Cap in effect at the time the Contract is executed if the player has met at least one of the Higher Max Criteria at the time his Contract is executed; or
  \item
    for any player who has completed ten (10) or more Years of Service, the greater of (x) thirty-five percent (35\%) of the Salary Cap in effect at the time the Contract is executed, or (y) one hundred five percent (105\%) of the Salary for the final Season of the player's prior Contract.
  \end{enumerate}
\item
  Notwithstanding any other provision of this Agreement, no Renegotiation may provide for a Salary plus Unlikely Bonuses in the Renegotiation Season (as defined in Article VII, Section 7(c)) that exceeds the following amounts:

  \begin{enumerate}
  \def\labelenumii{(\roman{enumii})}
  \tightlist
  \item
    for any player who has completed fewer than seven (7) Years of Service, the greater of (x) twenty-five percent (25\%) of the Salary Cap in effect at the time the Renegotiation is executed, or (y) one hundred five percent (105\%) of the Salary for the Season prior to the Renegotiation Season;
  \item
    for any player who has completed at least seven (7) but fewer than ten (10) Years of Service, the greater of (x) thirty percent (30\%) of the Salary Cap in effect at the time the Renegotiation is executed, or (y) one hundred five (105\%) of the Salary for the Season prior to the Renegotiation Season; or
  \item
    for any player who has completed ten (10) or more Years of Service, the greater of (x) thirty-five percent (35\%) of the Salary Cap in effect at the time the Renegotiation is executed, or (y) one hundred five percent (105\%) of the Salary for the Season prior to the Renegotiation Season.
  \end{enumerate}
\item
  The parties recognize that it may not be possible to ascertain at the time an Extension is executed whether the Salary plus Unlikely Bonuses called for in the first Season of the extended term will exceed the Maximum Annual Salary set forth in this Section 7. Accordingly, and notwithstanding any other provision of this Agreement, the following rule shall apply to any Extension in which the extended term begins on or after the effective date of this Agreement: if, on July 1 of the Salary Cap Year encompassing the first Season of the extended term of such Extension, the Salary plus Unlikely Bonuses provided for in such Season exceeds the following amounts:

  \begin{enumerate}
  \def\labelenumii{(\roman{enumii})}
  \tightlist
  \item
    for any player who has completed fewer than seven (7) Years of Service, the greater of (x) twenty-five percent (25\%) of the Salary Cap in effect on July 1 of the Salary Cap Year encompassing the first Season of the extended term of such Extension, or (y) one hundred five percent (105\%) of the Salary provided for in the final Season of the original term of the Contract; provided, however, that a 5th Year Eligible Player who signed a Rookie Scale Extension in accordance with Section 7(d) below shall be eligible to receive the percentage that is agreed upon by the Team and player, which shall be no less than twenty-five percent (25\%) or greater than thirty percent (30\%) of the Salary Cap in effect on July 1 of the Salary Cap Year encompassing the first Season of the extended term of such Extension if the player has met at least one of the Higher Max Criteria;
  \item
    for any player who has completed at least seven (7) but fewer than ten (10) Years of Service, the greater of (x) thirty percent (30\%) of the Salary Cap in effect on July 1 of the Salary Cap Year encompassing the first Season of the extended term of such Extension, or (y) one hundred five percent (105\%) of the Salary provided for in the final Season of the original term of the Contract; provided, however, that a player who (A) has one Season, or two Seasons (including any Option Year), remaining on his Contract, and (B) has seven (7) or eight (8) Years of Service at the time the Extension is executed (i.e., a player entering their 8th or 9th year in the NBA), and (C) rendered such Years of Service for the Team with which he first executed a Player Contract (or, if he was under a Player Contract for more than one Team during such period, changed Teams only by trade during the first four (4) Salary Cap Years in which he was under a Player Contract) shall be eligible to enter into a Designated Veteran Player Extension pursuant to which the player receives the percentage that is agreed upon by the Team and player, which shall be no less than thirty percent (30\%) and no greater than thirty-five percent (35\%) of the Salary Cap in effect on July 1 of the Salary Cap Year encompassing the first Season of the extended term of such Extension if the player has met at least one of the Higher Max Criteria; or
  \item
    for any player who has completed ten (10) or more Years of Service, the greater of (x) thirty-five percent (35\%) of the Salary Cap in effect on July 1 of the Salary Cap Year encompassing the first Season of the extended term of such Extension, or (y) one hundred five percent (105\%) of the Salary provided for in the final Season of the original term of the Contract;
  \end{enumerate}

  then such Salary plus Unlikely Bonuses shall immediately be deemed amended to provide for the maximum amount allowed by the applicable subsection (c)(i), (c)(ii), or (c)(iii) set forth above as follows: (1) if there is a signing bonus allocated to the first Salary Cap Year covered by the extended term, the amount of such allocation shall be reduced first; (2) if the reduction in clause (1) is insufficient to reduce the Salary plus Unlikely Bonuses to the maximum amount allowed pursuant to the applicable subsection (c)(i), (c)(ii), or (c)(iii) above (including because there is no signing bonus allocated to the first Salary Cap Year covered by the extended term) and the Extension provides for Incentive Compensation, the amount of Likely Bonuses and Unlikely Bonuses in the first Salary Cap Year covered by the extended term shall be reduced next (on a pro-rata basis); and (3) if the reductions in clauses (1) and (2) are insufficient to reduce the Salary plus Unlikely Bonuses to the maximum amount allowed pursuant to the applicable subsection (c)(i), (c)(ii), or (c)(iii) above (including because there is no signing bonus allocated to the first Salary Cap Year covered by the extended term and/or the Extension does not provide for Incentive Compensation), the amount of Base Compensation provided for in the first Salary Cap Year covered by the extended term shall be reduced last. In the event that the amount of a signing bonus allocation is deemed amended pursuant to the foregoing, then the amount of any signing bonus allocation in respect of each subsequent Salary Cap Year covered by the extended term shall also immediately be deemed amended proportionately (e.g., in the event that the amount of a signing bonus allocation is reduced by 50\% in respect of the first Salary Cap Year covered by the extended term, then the amount of any signing bonus allocation in respect of each subsequent Salary Cap Year covered by the extended term shall also be reduced by 50\%; and in the event that the amount of a signing bonus allocation is reduced by 100\% in respect of the first Salary Cap Year covered by the extended term, then the amount of any signing bonus allocation in respect of each subsequent Salary Cap Year covered by the extended term shall also be reduced by 100\%). In the event that the amount of any Likely Bonuses, Unlikely Bonuses, and/or Base Compensation is deemed amended pursuant to the foregoing, then the amount of any Likely Bonuses, Unlikely Bonuses, and/or Base Compensation in respect of each subsequent Salary Cap Year covered by the extended term shall also immediately be deemed amended to the extent necessary to comply with the maximum allowable increases or decreases over the amended Likely Bonuses, Unlikely Bonuses, and/or Base Compensation in the first Salary Cap Year covered by the extended term in accordance with Article VII, Section 5(a).
\item
  A player and a Team may provide in a Rookie Scale Extension that the player's Salary (in the first Season of the extended term) will equal ``the Maximum Annual Salary applicable to such player in the first Season of the extended term'' or:

  \begin{enumerate}
  \def\labelenumii{(\roman{enumii})}
  \tightlist
  \item
    in the case of a Rookie Scale Extension for a First Round Pick who at the time the Extension is executed has already met at least one of the Higher Max Criteria, the player and Team may instead provide in the Extension that the player's Salary (in the first Season of the extended term) will equal ``{[}\_\_\_\_\_{]}\% of the Salary Cap in effect during the first Season of the extended term.'' The percentage to be included where brackets are indicated in the foregoing language shall equal the percentage that is agreed upon by the Team and player, which shall in no event be less than twenty-five percent (25\%) or greater than thirty percent (30\%); or
  \item
    in the case of a Rookie Scale Extension for any other First Round Pick (i.e., a First Round Pick who at the time the Extension is executed had not yet met at least one of the Higher Max Criteria), the player and Team may instead provide in the Extension that the player's Salary (in the first Season of the extended term) will equal ``25\% of the Salary Cap in effect during the first Season of the extended term, or, if the player meets at least one of the applicable Higher Max Criteria during the fourth Season of his Rookie Scale Contract, {[} {]}\% of the Salary Cap in effect during the first Season of the extended term.'' The percentage to be included where brackets are indicated in the foregoing language shall equal the percentage of the Salary Cap that is agreed upon by the Team and player, which shall in no event be less than twenty-five percent (25\%) or greater than thirty percent (30\%).
  \item
    As an alternative to (i) or (ii) above, the Team may instead provide in the Extension that the player's Salary (in the first Season of the extended term) will equal alternative percentages of the Salary Cap (which shall in no event be less than twenty-five percent (25\%) or greater than thirty percent (30\%)) based upon how and whether the player satisfies the applicable Higher Max Criteria. Accordingly, for example, with respect to a Rookie Scale Extension in which the first Season of the extended term commences with the 2024-25 Season, the Team and player could agree that the player's Salary (in the first Season of the extended term) would be 25\% of the Salary Cap in effect during the first Season of the extended term, or the applicable percentage of the Salary Cap set forth below if, during the fourth Season of his Rookie Scale Contract, the player meets the Higher Max Criteria set forth opposite such percentage:
  \end{enumerate}

  \begin{longtable}[]{@{}lc@{}}
  \toprule()
  Higher Max Criteria Percentage & Percentage \\
  \midrule()
  \endhead
  All-NBA Second Team & 27\% \\
  All-NBA First Team & 28\% \\
  NBA MVP & 30\% \\
  \bottomrule()
  \end{longtable}

  The player and Team may provide in a Rookie Scale Extension that the Salaries in any Seasons after the first Season of the extended term will be increased or decreased based on percentages specified by the parties that comply with Article VII, Section 5(a). In the case of a Rookie Scale Extension entered into pursuant to (ii) or (iii) above, the player and Team may instead provide that Salaries in any Seasons after the first Season of the extended term will be increased or decreased by a different percentage based on the percentage of the Salary Cap that the player receives in Salary in the first Season of the extended term. Any such Rookie Scale Extension shall be deemed amended on July 1 of the Salary Cap Year covering the first Season of the extended term to provide for specific Salaries for each Season of the extended term, based on the Maximum Annual Salary applicable to such player on such July 1. A Rookie Scale Extension entered into pursuant to this subsection may not include any Incentive Compensation.
\item
  A player and a Team may provide in a Designated Veteran Player Extension that the player's Salary (in the first Season of the extended term) will equal ``{[}\_\_\_\_\_{]}\% of the Salary Cap in effect during the first Season of the extended term.'' The percentage to be included where brackets are indicated in the foregoing language shall equal the percentage that is agreed upon by the Team and player, which percentage shall in no event be less than thirty percent (30\%) or greater than thirty-five percent (35\%). The player and Team may provide in a Designated Veteran Player Extension that the Salaries in any Seasons after the first Season of the extended term will be increased or decreased based on percentages specified by the parties that comply with Article VII, Section 5(a). Any such Designated Veteran Player Extension shall be deemed amended on July 1 of the Salary Cap Year covering the first Season of the extended term to provide for specific Salaries for each Season of the extended term, based on the Maximum Annual Salary applicable to such player on such July 1. A Designated Veteran Player Extension entered into pursuant to this subsection may not include any Incentive Compensation.
\item
  Notwithstanding any other provision of this Agreement, if a trade of a Uniform Player Contract would, by reason of a trade bonus contained in such Contract, cause the player's Salary plus Unlikely Bonuses for the Salary Cap Year in which such trade occurs to exceed the following amounts:

  \begin{enumerate}
  \def\labelenumii{(\roman{enumii})}
  \tightlist
  \item
    for any player who has completed fewer than seven (7) Years of Service, the greater of (x) twenty-five percent (25\%) of the Salary Cap in effect at the time the trade bonus is earned, or (y) one hundred five percent (105\%) of the player's Salary for the Season prior to the Season in which the trade bonus is earned, or in the case of a 5th Year Eligible Player who met at least one of the Higher Max Criteria and signed a Contract or Rookie Scale Extension (as applicable) that provided for up to thirty percent (30\%) of the Salary Cap, {[}\_\_{]}\% of the Salary Cap in effect at the time the trade bonus is earned with the applicable percentage where brackets are indicated equal to the percentage of the Salary Cap paid to the player in the first year of his Contract or the first year of the extended term in the case of a Rookie Scale Extension;
  \item
    for any player who has completed at least seven (7) but fewer than ten (10) Years of Service, the greater of (x) thirty percent (30\%) of the Salary Cap in effect at the time the trade bonus is earned, or (y) one hundred five percent (105\%) of the player's Salary for the Season prior to the Season in which the trade bonus is earned, or in the case of a Designated Veteran Player who signed a Designated Veteran Player Contract or a Designated Veteran Player Extension (as applicable) that provided for up to thirty-five percent (35\%) of the Salary Cap, {[}\_\_{]}\% of the Salary Cap in effect at the time the trade bonus is earned with the applicable percentage where brackets are indicated equal to the percentage of the Salary Cap paid to the player in the first year of his Contract (or the first year of the extended term in the case of a Designated Veteran Player Extension); or
  \item
    for any player who has completed ten (10) or more Years of Service, the greater of (x) thirty-five percent (35\%) of the Salary Cap in effect at the time the trade bonus is earned, or (y) one hundred five percent (105\%) of the player's Salary for the Season prior to the Season in which the trade bonus is earned;
  \end{enumerate}

  then such player's trade bonus shall be deemed amended to the extent necessary to reduce the player's Salary plus Unlikely Bonuses to the maximum amount allowed by the applicable subsection (f)(i), (f)(ii), or (f)(iii) set forth above.
\item
  Notwithstanding any other provision of this Agreement, any Contract or Rookie Scale Extension entered into between a 5th Year Eligible Player and a Team that provides for Salary plus Unlikely Bonuses in the first Season covered by the Contract or Rookie Scale Extension (as applicable) greater than twenty-five percent (25\%) of the Salary Cap in effect during the first Season of the Contract or extended term (as applicable) in accordance with the rules set forth in this Section 7 must be for at least four (4) Seasons (excluding any Option Year) and, in the case of a Rookie Scale Extension, excluding the last Season covered by the player's Rookie Scale Contract.
\end{enumerate}

\hypertarget{promotional-activities.}{%
\section{Promotional Activities.}\label{promotional-activities.}}

\begin{enumerate}
\def\labelenumi{(\alph{enumi})}
\tightlist
\item
  A player's obligation (pursuant to Paragraph 13(d) of a Uniform Player Contract) to participate, upon request, in all other reasonable promotional activities of the Team and the NBA shall be deemed satisfied if:

  \begin{enumerate}
  \def\labelenumii{(\roman{enumii})}
  \tightlist
  \item
    during each Salary Cap Year of the period covered by such Contract, the player makes seven (7) individual personal appearances (at least two (2) of which shall be in connection with season ticketholder events) and five (5) group appearances for or on behalf of or at the request of the Team (or Team Affiliate) by which he is employed and/or the NBA. Up to two (2) of these twelve (12) appearances may be assigned by the Team and/or the NBA in any Salary Cap Year to NBA Properties. The player shall be reimbursed for the actual expenses incurred in connection with any such appearance, provided that such expenses result directly from the appearance and are ordinary and reasonable. The player shall also receive compensation from the Team by which he is employed of \$3,500, in accordance with Paragraph 13(d) of the Uniform Player Contract, for each promotional appearance he makes for a commercial sponsor of such Team. Notwithstanding the preceding sentence, with respect to any Salary Cap Year during which a player makes at least eight (8) appearances pursuant to this Section 8(a)(i), for each subsequent appearance made by the player for a commercial sponsor of the Team during such Salary Cap Year, the player shall receive compensation from the Team by which he is employed of \$4,500.
  \item
    Any personal or group appearance required under this subsection (a) must:

    \begin{enumerate}
    \def\labelenumiii{(\Alph{enumiii})}
    \tightlist
    \item
      take place during (1) the period from the first day of a Season through the day of the NBA Draft following such Season, or (2) the off-season, provided that no player may be required to make more than one off-season appearance in any year covered by his Contract and no player may be required to make such an off-season appearance unless he resides in or is otherwise located in the area where the appearance is to take place;
    \item
      occur in the home city (or geographic vicinity thereof) of the player's Team (subject to Section 8(a)(ii)(A)(2) above) or in a city (or geographic vicinity thereof) to which the player has traveled to play in a scheduled NBA game;
    \item
      not occur at a time that would interfere with a player's reasonable preparation to play on the day of a Team game;
    \item
      not occur at a time that would interfere with a player's ability to attend and participate fully in any practice session conducted by the Team, taking into account the commuting time from the practice to the appearance;
    \item
      be scheduled with the player at least fourteen (14) days in advance (by providing written notice to the player of the time, nature, location, and expected duration of the appearance) and called to his attention again seven (7) days prior to the appearance;
    \item
      not exceed a reasonable period of time; and
    \item
      not require the player to sign autographs as the primary purpose of the appearance.
    \end{enumerate}
  \item
    During each Salary Cap Year, a player's participation in any of the following activities shall count as one appearance required by this Section 8 and Paragraph 13(d) of the Uniform Player Contract:

    \begin{enumerate}
    \def\labelenumiii{(\Alph{enumiii})}
    \tightlist
    \item
      If requested by the NBA, an NBA Player Day as described in Article XXXVII, Section 1(b); or
    \item
      If requested by his Team, (1) a live social media Q\&A session with fans conducted by the Team, or (2) a player-focused content session conducted by the Team at a location (other than a Team facility or Team-controlled space) that has been secured by the Team or, if mutually agreeable, the player's home or other player-controlled space; provided, however, that no more than four (4) of a player's required appearances may be satisfied by participating in the activities set forth in this Section 8(a)(iii)(B).
    \end{enumerate}
  \item
    The player participates in reasonable fan appreciation activities before and after home games, including, but not limited to, signing autographs for fans, greeting fans, and participating in merchandise giveaways to fans; provided, however, that no player shall be required to participate in more than four (4) such activities per Season.
  \item
    Teams shall be required to track promotional appearances made by players in accordance with this Section 8 and Paragraph 13(d) of the Uniform Player Contract and report such information to the NBA. Upon request in respect of a Team, the NBA shall provide such information to the Players Association.
  \end{enumerate}
\item
  Upon request by the Team, the NBA, or a League-related entity, and subject to the conditions and limitations set forth below, the player shall wear a wireless microphone during any game or practice, including warm-up periods and going to and from the locker room to the playing floor. The rights in any audio captured by such microphone shall belong to the NBA or a League-related entity and may be used in any manner for publicity or promotional purposes.

  \begin{enumerate}
  \def\labelenumii{(\roman{enumii})}
  \tightlist
  \item
    The NBA or a League-related entity will be responsible for providing the audio equipment and for the placement of the microphone on the player in a location and manner that minimizes interference with the player's performance.
  \item
    The audio captured by the wireless microphone worn by the player (``Player Audio'') will be screened and approved prior to airing by the telecast producer and an NBA representative, and no such Player Audio will be aired live without the prior consent of the player.
  \item
    The NBA will use best efforts to ensure that a game telecast will not include any Player Audio that contains profanity or that could reasonably be considered prejudicial or detrimental to the player or other players.
  \item
    All audio tapes containing approved Player Audio will be returned by the telecaster to the NBA and archived.
  \item
    At the request of the player or the Players Association, the NBA shall make available a copy of the Player Audio.
  \item
    In the event a player believes that any Player Audio excerpt would be prejudicial or detrimental to him if replayed in any non-game programming (e.g., home videos) or other publicity or promotional content, and notifies the NBA to that effect in writing within one hundred twenty (120) hours of the recording of such audio, then neither the NBA nor any League-related entity, following receipt of such notice from the player, shall incorporate, or license others to incorporate, such excerpt into any such content.
  \item
    No player, without his consent, may be required to wear a wireless microphone (A) for nationally-televised games, more than one (1) game per month in any Regular Season covered by his Contract, (B) for locally-televised games, more than one (1) game per month in any Season covered by his Contract, or (C) for playoff games, more than two (2) games per playoff round in any Season covered by his Contract.
  \item
    At the beginning of each Season, players will receive written notice of the conditions and limitations set forth in Sections 8(b)(i)-(vii) above.
  \item
    Notwithstanding anything to the contrary in this Agreement, Player Audio shall not be used as the basis for the imposition of discipline upon any player.
  \end{enumerate}
\item
  Upon request by the NBA or the Team, a player that is in attendance but not dressed for or able to play in a game shall participate in an in-game interview from the Team bench. No player, without his consent, may be required to participate in more than one such in-game interview per week.
\item
  Each player shall be required to participate each Season, upon request, in promotional activities for the benefit of the NBA's television partners, provided that such participation does not exceed one (1) hour per player per Season and that the player is reimbursed for any reasonable expenses he incurs in connection with such participation.
\end{enumerate}

\hypertarget{day-contracts.}{%
\section{10-Day Contracts.}\label{day-contracts.}}

\begin{enumerate}
\def\labelenumi{(\alph{enumi})}
\item
  Beginning on January 5 of any NBA Season, a Team may enter into a Player Contract (other than a Two-Way Contract) with a player for the longer of (i) ten (10) days, or (ii) a period encompassing three (3) games played by such Team (a ``10-Day Contract'').
\item
  The Salary provided for by a 10-Day Contract shall be the Minimum Player Salary.
\item
  No Team may enter into a 10-Day Contract with the same player more than twice during the course of any one Season. No Team may be a party at any one time to more 10-Day Contracts than the following:

  \begin{longtable}[]{@{}
    >{\centering\arraybackslash}p{(\columnwidth - 2\tabcolsep) * \real{0.6452}}
    >{\centering\arraybackslash}p{(\columnwidth - 2\tabcolsep) * \real{0.3548}}@{}}
  \toprule()
  \begin{minipage}[b]{\linewidth}\centering
  Aggregate Number of Players on Team's Active List and Inactive List (including players signed to 10-Day Contracts, but not including Two-Way Players)
  \end{minipage} & \begin{minipage}[b]{\linewidth}\centering
  Maximum Number of the Team's players who can be signed to 10-Day Contracts
  \end{minipage} \\
  \midrule()
  \endhead
  12 & 0 \\
  13 & 1 \\
  14 & 2 \\
  15 & 3 \\
  \bottomrule()
  \end{longtable}

  For example, if a Team has thirteen (13) players on its Active List (not including any Two-Way Players) and no players on its Inactive List, then the Team may have one player under a 10-Day Contract. If a Team has thirteen (13) players on its Active List (including one (1) Two-Way Player) and two (2) players on its Inactive List (not including any Two-Way Players), then the Team may have two (2) players under a 10-Day Contract. If a Team has twelve (12) players on its Active List (not including any Two-Way Players) and three (3) players on its Inactive List (not including any Two-Way Players), then the Team may have three (3) players under a 10-Day Contract.
\item
  No Team may enter into a 10-Day Contract if the length of such Contract, in accordance with Section 9(a), would extend to or past the date of the Team's last Regular Season game for such Season.
\item
  Notwithstanding anything to the contrary in Section 9(a) or 9(d) above, in the event the NBA authorizes a Team to sign a Player Contract pursuant to the NBA's hardship rules, then (i) such Contract shall be a 10-Day Contract regardless of when during the Season such Contract is signed; and (ii) if the length of such 10-Day Contract (as determined in accordance with Section 9(a) above) would extend to or past the date of the Team's last Regular Season game in such Season, then the term of such 10-Day Contract shall be the number of days remaining in such Regular Season (including the day on which the 10-Day Contract is signed).
\item
  Notwithstanding anything to the contrary contained in a Uniform Player Contract, a 10-Day Contract shall be terminated simply by providing written notice to the player (and not by following the waiver procedure set forth in Paragraph 16 of the Uniform Player Contract) and paying only such sums as are set forth in Exhibit 1A of such Contract.
\item
  If a player's 10-Day Contract with a Team is terminated by the Team prior to the expiration of its stated term, then the Team and player shall not be permitted to enter into a new Contract prior to the expiration of the stated term of such terminated 10-Day Contract.
\item
  A Team and player who are parties to a 10-Day Contract may, prior to the expiration (or termination, if applicable) of the 10-Day Contract, negotiate and enter into a Standard NBA Contract that is a Rest-of-Season Contract (defined below) that will take effect on the day following the date on which the stated term of such 10-Day Contract expires.
\end{enumerate}

\hypertarget{rest-of-season-contracts.}{%
\section{Rest-of-Season Contracts.}\label{rest-of-season-contracts.}}

\begin{enumerate}
\def\labelenumi{(\alph{enumi})}
\tightlist
\item
  At any time after the first day of an NBA Regular Season, a Team may enter into a Player Contract that may provide Compensation to a player for the remainder of that Season (a ``Rest-of-Season Contract'').
\item
  The Salary provided for in a Rest-of-Season Contract shall not be less than the Minimum Player Salary.
\item
  Notwithstanding the foregoing, Two-Way Contracts shall not be subject to the requirements set forth in this Section 10.
\end{enumerate}

\hypertarget{two-way-contracts.}{%
\section{Two-Way Contracts.}\label{two-way-contracts.}}

\begin{enumerate}
\def\labelenumi{(\alph{enumi})}
\item
  \textbf{Two-Way Player Salary.}

  \begin{enumerate}
  \def\labelenumii{(\roman{enumii})}
  \item
    Subject to the limitations set forth in this Section 11, an NBA Team may enter into a Player Contract that provides a player (``Two-Way Player'') with a Salary as set forth in Section 11(a)(ii) below for providing services to both an NBAGL team and the NBA Team (``Two-Way Contract'').
  \item
    The Salary provided for in a Two-Way Contract (the ``Two-Way Player Salary'') for a Season shall equal fifty percent (50\%) of the Minimum Annual Salary called for under Article II, Section 6(a) for a player with zero (0) Years of Service (irrespective of how many Years of Service the player has accrued prior to the Contract or accrues during the term of the Contract), multiplied by a fraction, the numerator of which is the number of days remaining in the Regular Season as of the date such Contract is entered into (including the day on which the Contract is entered into), and the denominator of which is the total number of days of that Regular Season.
  \item
    Notwithstanding anything to the contrary in this Agreement, no Two-Way Contract may include or provide for any (A) bonuses or Incentive Compensation of any kind, (B) deferred compensation, or (C) loans.
  \item
    Every Two-Way Contract must contain an Exhibit 1B and include the following sentence in such Exhibit (which shall be deemed amended in the manner described in such sentence): ``This Contract is intended to provide for a Base Compensation for the \_\_\_\_\_\_\_\_\_\_\_\_ Season(s) equal to the Two-Way Player Salary for such Season(s) (with no bonuses of any kind) and shall be deemed amended to the extent necessary to so provide.''
  \item
    A Two-Way Contract that, at the time the Two-Way Contract is signed, is partially protected for lack of skill and injury or illness for a Season may be amended to provide for the Two-Way Player to be paid a portion of his Base Compensation for such Season (the ``Advance''), up to the Two-Way Contract Advance Limit as defined below, prior to November 1 of such Season. The Two-Way Contract Advance Limit for a Season shall equal fifty percent (50\%) of the amount of the Two-Way Player's Base Compensation for such Season that is protected for lack of skill and injury or illness at the time of signing. Any Advance paid to a player for a Season pursuant to the foregoing must be deducted from the first installment of Base Compensation (i.e., on November 1) and, if and as necessary after reducing in full the first installment, each subsequent installment of Base Compensation for such Season that such player would have received pursuant to Paragraph 3(a) of the Uniform Player Contract had there been no such Advance. To effectuate the requirement set forth in the preceding sentence, every such Two-Way Contract that provides for an Advance must contain the following language (and, with respect to an Advance, only such language) in Exhibit 1B with respect to each applicable Season:

    \begin{quote}
    ``\textbf{Payment Schedule} (if different from Paragraph 3): Player's Base Compensation with respect to the \_\_\_\_\_\_\_\_\_ Season(s) shall be paid in accordance with Paragraph 3(a), except that the November 1 installment of such Base Compensation and, if and as necessary after reducing in full the November 1 installment, each subsequent installment of such Base Compensation for such Season shall be reduced by \${[}amount of Advance{]}, which amount shall be paid to Player in advance on {[}date{]}.''
    \end{quote}
  \end{enumerate}
\item
  \textbf{Roster Limitations.}

  \begin{enumerate}
  \def\labelenumii{(\roman{enumii})}
  \tightlist
  \item
    No Team may have on its roster at any time more than three (3) Two-Way Players.
  \item
    No player under a Two-Way Contract may be on the Active List for more than fifty (50) games during a Regular Season. If a player is signed to a Two-Way Contract after the start of a Regular Season, the maximum number of games for which such player may be on the Active List during that Regular Season shall be fifty (50) multiplied by a fraction, the numerator of which is the number of days remaining in such Regular Season as of the date such Two-Way Contract is entered into (including the day on which the Two-Way Contract is entered into), and the denominator of which is the total number of days of such Regular Season, rounded to the nearest whole number; provided, however, that in no event shall the maximum number of games for which a player may be on the Active List be fewer than one (1).
  \item
    Any Regular Season game for which a Team has fewer than fifteen (15) players signed to Standard NBA Contracts shall be an ``Under-Fifteen Game.'' No Team shall be permitted to have a Two-Way Player on its Active List for more than ninety (90) Under-Fifteen Games during a Regular Season. For purposes of the foregoing rule: (A) an Under-Fifteen Game for which a Team has one (1) Two-Way Player on its Active List shall count as one (1) Under-Fifteen Game; (B) an Under-Fifteen Game for which a Team has two (2) Two-Way Players on its Active List shall count as two (2) Under-Fifteen Games; and (C) an Under-Fifteen Game for which a Team has three (3) Two-Way Players on its Active List shall count as three (3) Under-Fifteen Games.
  \end{enumerate}
\item
  \textbf{Compensation Protection.}

  \begin{enumerate}
  \def\labelenumii{(\roman{enumii})}
  \tightlist
  \item
    The maximum amount of aggregate Base Compensation protection for a Season in a Two-Way Contract is the ``Maximum Two-Way Protection Amount'' (defined below) for such Season, provided that if such Contract is signed after the first day of the Regular Season, the maximum amount of aggregate Base Compensation protection for such Season is fifty percent (50\%) of the Base Compensation provided for by such Contract for such Season. In addition, a Two-Way Contract may also provide for Base Compensation protection for a Season to increase to up to fifty percent (50\%) of the Base Compensation provided for by such Contract for such Season if the Team does not request waivers on the player by a certain date that is on or after the first day of the Regular Season encompassed by such Season.
  \item
    The ``Maximum Two-Way Protection Amount'' shall be \$75,000 for the 2023-24 Season, and for each subsequent Season shall be \$75,000 multiplied by a fraction, the numerator of which is the Salary Cap for the Salary Cap Year encompassing the applicable Season and the denominator of which is the Salary Cap for the 2023-24 Salary Cap Year.
  \item
    If a Team assigns or terminates a Player Contract that contains aggregate Base Compensation protection in respect of the then-current and any future Salary Cap Year that exceeds the Maximum Two-Way Protection Amount for the Season encompassed by the Salary Cap Year in which such assignment or termination occurs, then, during such Salary Cap Year, the player shall be precluded from: (x) playing under an NBAGL contract for such Team's NBAGL affiliate, and (y) entering into a Two-Way Contract with such Team.
  \end{enumerate}
\item
  \textbf{Contract Term.} The term of a Two-Way Contract may not exceed two (2) Seasons in length and may not include any Option Year or Early Termination Option.
\item
  \textbf{Eligibility.} The following eligibility rules shall apply to all Two-Way Contracts:

  \begin{enumerate}
  \def\labelenumii{(\roman{enumii})}
  \tightlist
  \item
    No Team may sign a player to a Two-Way Contract after March 4 of any Season.
  \item
    No Team may sign or convert a player to a Two-Way Contract if the player has or may have four (4) or more Years of Service at any point during the Contract. For example, a player with three (3) Years of Service would not be eligible to sign a Two-Way Contract with a term of two (2) years. Notwithstanding the foregoing, a Team may sign or convert a player who has four (4) Years of Service to a Two-Way Contract covering no more than one (1) Season if the player was credited with one (1) or more Years of Service in respect of a Season in which he (A) did not play in a Regular Season, Play-In, or playoff game and (B) was on a Team's roster at all times from the first day of the Regular Season through the end of the last day of the Regular Season.
  \item
    No Team may sign or convert a player to a Two-Way Contract, or acquire a Two-Way Contract by means of assignment, if, as a result, the player would or could be under a Two-Way Contract for any part of more than three (3) Salary Cap Years with the same NBA Team. For example, a player who completes a two-year Two-Way Contract with a Team could not subsequently sign a two-year Two-Way Contract with that Team.
  \end{enumerate}
\item
  \textbf{Standard NBA Contract Conversion Option.} Every Two-Way Contract shall provide the Team with an option to convert the Two-Way Contract during its term to a Contract that is not a Two-Way Contract (``Standard NBA Contract'') that provides for a Salary for each Salary Cap Year equal to the player's applicable Minimum Player Salary and a term equal to the remainder of the original term of the Two-Way Contract beginning on the date such option is exercised (``Standard NBA Contract Conversion Option''). Such player's applicable Minimum Player Salary shall be determined in accordance with Section 6 above. For the day the Standard NBA Contract Conversion Option is exercised, the player shall be compensated only under the new Standard NBA Contract, and not under his Two-Way Contract. The Standard NBA Contract Conversion Option may be exercised at any point during the period beginning on July 1 and ending just prior to the start of the Team's last Regular Season game in each Salary Cap Year covered by the Two-Way Contract. Upon conversion, such Contract shall become a Standard NBA Contract and shall no longer be governed by the provisions of this Agreement governing Two-Way Contracts. To effectuate the requirements set forth in the preceding sentences, every Two-Way Contract with an Exhibit 1B must contain the following language (and only such language) under the ``Standard NBA Contract Conversion Option'' heading:

  \begin{quote}
  ``Team shall have the option to convert this Contract to a Standard NBA Contract (''Standard NBA Contract Conversion Option''). Team's Standard NBA Contract Conversion Option may be exercised by providing written notice to Player that is either personally delivered to Player or his representative or sent by email or pre-paid certified, registered, or overnight mail to the last known address of Player or his representative with a copy to the Players Association and the NBA. If Team exercises the Standard NBA Contract Conversion Option, the Base Compensation amount set forth above in this Exhibit 1B will immediately become null and void and of no further force or effect, Player's Compensation shall be equal to the Player's applicable Minimum Player Salary for a term equal to the remainder of the original term of this Contract beginning on the date such option is exercised, and all other terms and conditions of this Contract, including the Base Compensation protection set forth in Exhibit 2 (if any), shall remain applicable.''
  \end{quote}
\item
  \textbf{Exclusive Rights.}

  \begin{enumerate}
  \def\labelenumii{(\roman{enumii})}
  \tightlist
  \item
    During the term of a Two-Way Contract, the Team that is the party to the Two-Way Contract shall be the only Team with which the Two-Way Player may negotiate or sign a Standard NBA Contract.
  \item
    The Team and the Two-Way Player who are parties to such Two-Way Contract shall have the right to negotiate and agree to a Standard NBA Contract in accordance with the terms of this Agreement. Notwithstanding anything to the contrary in this Agreement or the Uniform Player Contract, (1) such Standard NBA Contract may not include an Exhibit 10, and (2) upon execution of the Standard NBA Contract, the prior Two-Way Contract between the Team and player will immediately be rendered null and void and of no further force or effect. For the day the Standard NBA Contract is executed, the player shall be compensated only under the new Standard NBA Contract, and not under the prior Two-Way Contract.
  \end{enumerate}
\item
  \textbf{Exhibit 10.}

  \begin{enumerate}
  \def\labelenumii{(\roman{enumii})}
  \tightlist
  \item
    Every Contract with an Exhibit 10 shall provide the Team with an option (to be set forth in Exhibit 10) to convert the Contract to a Two-Way Contract that provides for the Two-Way Player Salary (``Two-Way Player Conversion Option''); provided, however, that the Two-Way Player Conversion Option (a) must be exercised prior to the first day of the NBA Regular Season, and (b) may not be exercised if it would result in a violation of Article X, Section 4(d). If a Team exercises the Two-Way Player Conversion Option, (w) the Contract's Exhibit 1A will immediately become null and void and of no further force or effect and the Player's Compensation shall be equal to the Two-Way Player Salary applicable for such Season, (x) the Player's right to an Exhibit 10 Bonus (if applicable) will be rescinded, (y) the Player's Contract, notwithstanding the absence of an Exhibit 2, shall have Base Compensation protection for lack of skill and injury or illness at an amount equal to the Conversion Protection Amount, and (z) all other terms and conditions of the Contract shall remain applicable.
  \item
    If a Team exercises a Two-Way Player Conversion Option pursuant to a Contract with an Exhibit 10, such Contract shall be considered a Two-Way Contract for the purposes of this Agreement and subject to all applicable Two-Way Contract rules herein (including, but not limited to, the Standard NBA Contract Conversion Option) except that such Contract need not contain an Exhibit 1B.
  \item
    To effectuate the requirements set forth above, every Contract with an Exhibit 10 must contain the following language (and only such language) under the ``Two-Way Player Conversion Option'' and ``Standard NBA Contract Conversion Option'' headings, respectively:
  \end{enumerate}

  \begin{quote}
  \textbf{Two-Way Player Conversion Option:} Team shall have the option to convert this Contract to a Two-Way Contract (``Two-Way Player Conversion Option''); provided, however, that (a) such option must be exercised prior to the first day of the NBA Regular Season, and (b) may not be exercised if it would result in a violation of Article X, Section 4(d) of the CBA. Team's Two-Way Player Conversion Option may be exercised by providing written notice to Player that is either personally delivered to Player or his representative or sent by email or pre-paid certified, registered, or overnight mail to the last known address of Player or his representative with a copy to the Players Association and the NBA. If Team exercises the Two-Way Player Conversion Option, this Contract's Exhibit 1A will immediately become null and void and of no further force or effect and the Player's Compensation shall be equal to the Two-Way Player Salary applicable for such Season. Further, upon conversion, the Player's right to the Bonus Amount (if applicable) set forth above pursuant to this Exhibit 10 will be rescinded and the Player's Contract, notwithstanding the absence of an Exhibit 2, shall be protected for lack of skill and injury or illness at an amount equal to the Conversion Protection Amount in this Exhibit 10. All other terms and conditions of this Contract shall remain applicable.
  \end{quote}

  \begin{quote}
  \textbf{Standard NBA Contract Conversion Option:} In the event the Two-Way Player Conversion Option is exercised by the Team, Team shall thereafter have the option to convert the Contract to a Standard NBA Contract (``Standard NBA Contract Conversion Option''). Team's Standard NBA Contract Conversion Option may be exercised by providing written notice to Player that is either personally delivered to Player or his representative or sent by email or pre-paid certified, registered, or overnight mail to the last known address of Player or his representative with a copy to the Players Association and the NBA. If Team exercises the Standard NBA Contract Conversion Option, the Base Compensation amount applicable to the Two-Way Contract as set forth in this Exhibit 10 will immediately become null and void and of no further force or effect, Player's Compensation shall be equal to the Player's applicable Minimum Player Salary for such Season beginning on the date such option is exercised, and all other terms and conditions of this Contract, including the Base Compensation protection set forth in this Exhibit 10, shall remain applicable.
  \end{quote}
\end{enumerate}

\hypertarget{bonuses.}{%
\section{Bonuses.}\label{bonuses.}}

\begin{enumerate}
\def\labelenumi{(\alph{enumi})}
\item
  Notwithstanding any other provision of this Agreement:

  \begin{enumerate}
  \def\labelenumii{(\roman{enumii})}
  \tightlist
  \item
    No Uniform Player Contract may provide for Incentive Compensation for a Season that exceeds twenty percent (20\%) of the Regular Salary called for by the Contract for such Season;
  \item
    No Uniform Player Contract may provide for a signing bonus that exceeds fifteen percent (15\%) of the Compensation (excluding Incentive Compensation) called for by the Contract (or, in the case of an Extension, in the extended term of the Extension); and
  \item
    No Offer Sheet may provide for a signing bonus that exceeds ten percent (10\%) of the Compensation (excluding Incentive Compensation) called for by the Offer Sheet.
  \end{enumerate}
\item
  If a player's Contract provides for a signing bonus and the player is suspended for the intentional failure or refusal to render the services required under his Contract, the Team shall be entitled to a return from the player of an amount equal to the product of the signing bonus multiplied by a fraction, the numerator of which is the number of Regular Season games that the player is suspended as a result of his failure or refusal to render such services and the denominator of which is the total number of Regular Season games to be played by the Team during the term of the Contract (excluding any Option Year). The foregoing shall not limit any other rights or remedies a Team may have under the Contract or by law.
\item
  \begin{enumerate}
  \def\labelenumii{(\roman{enumii})}
  \tightlist
  \item
    No Uniform Player Contract may provide for the player's attendance at and participation in an off-season skill and/or conditioning program that exceeds two (2) weeks in length.
  \item
    A Uniform Player Contract that contains a bonus to be paid as a result of the player's attendance at and participation in an off-season summer league and/or an off-season skill and/or conditioning program in accordance with subsection b(i) above may also contain a provision providing that such bonus will be paid if: (A) the Team elects in writing to waive the requirement that the player perform the specified services; (B) the player, in lieu of providing the specified services, participates in training and/or plays games with his national team during the off-season; and/or (C) the player has an injury, illness, or other medical condition that renders the player unable to participate in such summer league and/or skill and conditioning program. If a Contract contains a provision of the type described in (A) above and the Team exercises its right to waive the requirement that the player perform the specified services with respect to one or more off-seasons, the amounts paid to the player shall continue to be treated as a bonus for the player's participation in an off-season summer league or off-season skill and/or conditioning program and shall continue to be subject to the rules in this Agreement relating to such bonuses.
  \item
    If a Uniform Player Contract contains a bonus to be paid as a result of the player's attendance at and participation in an off-season summer league and/or an off-season skill and/or conditioning program, the Team shall be required to provide the player with a reasonable opportunity to earn the bonus by, for example, providing the player with the dates, times, and location(s) at which the specified services are to be performed. A Team's failure to comply with this requirement with respect to any off-season shall be deemed to constitute a waiver of the requirement that the player perform the specified services for such off-season.
  \end{enumerate}
\item
  No Uniform Player Contract may contain a bonus for the player being on a Team's roster as of a specified date or for a specified duration, or for the player dressing in uniform for or being eligible to play in a specified number of games.
\item
  If a Player Contract contains Incentive Compensation, a Team and player shall not be permitted at any time to amend the Contract to modify the conditions that the player must satisfy in order to earn all or any portion of such Incentive Compensation.
\end{enumerate}

\hypertarget{general.}{%
\section{General.}\label{general.}}

\begin{enumerate}
\def\labelenumi{(\alph{enumi})}
\item
  \begin{enumerate}
  \def\labelenumii{(\roman{enumii})}
  \tightlist
  \item
    Subject to Section 15 below, any oral or written agreement between a player and a Team concerning terms and conditions of employment shall be reduced to writing in the form of a Uniform Player Contract or an amendment thereto as soon as practicable. Immediately upon the consummation of any such oral or written agreement, the Team shall notify the NBA by email and provide the NBA with all economic terms of such agreement. Upon its receipt of an executed Uniform Player Contract, the NBA shall provide a copy of the same to the Players Association by email within two (2) business days.
  \item
    Notwithstanding subsection (a)(i) above, neither the NBA, any Team, nor the Players Association, or any player, shall contend that any agreement concerning terms and conditions of employment is binding upon the player or the Team until a Player Contract embodying such terms and conditions has been duly executed by the parties. Nothing herein is intended to affect (A) any authority of the Commissioner to approve or disapprove Player Contracts, or (B) the effect of the Commissioner's approval or disapproval on the validity of such Player Contracts.
  \item
    A violation of the first sentence of subsection (a)(i) above may be considered evidence of a violation of Article XIII.
  \end{enumerate}
\item
  No player shall attend the regular training camp of any Team, or participate in games or organized practices with the Team at any time, unless he is a party to a Player Contract then in effect. For purposes of this Section 13(b), a player shall be considered to be a party to a Player Contract then in effect if such Contract has been extended in accordance with an Option permitted by this Agreement.
\item
  The only form of Compensation that a Team may pay a player under his Uniform Player Contract is cash via a check made payable to the player or via a direct deposit made to the player's bank account. Compensation of any other kind is prohibited.
\item
  No Team shall make any direct or indirect payment of any money, property, investments, loans, or anything else of value for fees or otherwise to an agent, attorney, or representative of a player (for or in connection with such person's representation of such player); nor shall any Player Contract provide for such payment. No player shall assign or otherwise transfer to any third party his right to receive Compensation from the Team under his Uniform Player Contract. Nothing in this subsection (d), however, shall prevent a Team from sending a player's regular paycheck to a player's agent, attorney, or representative if so instructed in writing by the player.
\item
  Every Uniform Player Contract must provide that for each Season of such Contract, the player will be paid at least ten percent (10\%) of his Salary for such Season, excluding Likely Bonuses and any portion of the player's Salary attributable to a trade bonus, in Current Base Compensation in accordance with the payment schedule provided in Paragraph 3 of the Contract or in twelve (12) equal semi-monthly payments beginning with the first of said payments on November 1 of each year covered by the Contract and continuing with such payments on the first and fifteenth of each month until said Compensation is paid in full.
\item
  No Uniform Player Contract may provide for the payment of any Compensation earned for a Season prior to the first semi-monthly payment date that is at least seven (7) days following the completion of the Audit Report for the Salary Cap Year covering the immediately prior Season.
\item
  A Team's termination of a Uniform Player Contract by reason of the player's ``lack of skill'' (under Paragraph 16(a)(iii) of the Uniform Player Contract) shall be interpreted to include a termination based on the Team's determination that, in view of the player's level of skill (in the sole opinion of the Team), the Compensation paid (or to be paid) to the player is no longer commensurate with the Team's financial plans or needs. The foregoing sentence shall not affect any post-termination obligation to pay Compensation that may result from Compensation protection provisions included in a Uniform Player Contract.
\item
  The following provisions shall govern an agreement (to be set forth in Exhibit 6 to a Uniform Player Contract) establishing that the player must report for and submit to a physical examination to be performed by one or more physician(s) designated by the Team:

  \begin{enumerate}
  \def\labelenumii{(\roman{enumii})}
  \tightlist
  \item
    The player must report for such physical examination at the time designated by the Team (which shall be no later than the third business day following the execution of the Contract), and must, upon reporting, supply all information reasonably requested of him, provide complete and truthful answers to all questions posed to him, and submit to all examinations and tests requested of him. The determination of whether the player has passed the physical examination shall be made by the Team in its sole discretion, exercised in good faith, in consultation with one or more of the Team's physicians; and a Team shall have the right to determine in good faith that a player has failed to pass the physical examination due to the risk of a future injury, illness, or other medical condition notwithstanding that the player is currently able to play. If the player does not pass the physical examination, the Team shall so notify the player no later than the sixth business day following the execution of the Contract.
  \item
    The Team's determination that the player has passed the physical examination shall be a condition precedent to the validity of the Contract. Accordingly, and without limiting the generality of the preceding sentence, until such time as a player has passed the physical examination, the prohibitions set forth in Section 13(b) above shall continue to apply to the Team and player.
  \item
    A Required Tender or a Qualifying Offer may contain an Exhibit 6. If a player accepts such a Required Tender or Qualifying Offer but does not pass the required physical examination, the Required Tender or Qualifying Offer shall be deemed to have been withdrawn, which shall have the consequences described in Article X, Section 4 or Article XI, Section 4, as the case may be.
  \end{enumerate}
\item
  A player who knows he has an injury, illness, or other medical condition that renders, or he knows will likely render, him unable to perform the playing services required under a Player Contract may not validly enter into such Contract without prior written disclosure of such injury, illness, or other medical condition to the Team.
\item
  Neither the IST Finals Game, Play-In Games, nor a Team's or a player's performance during any such games, shall be considered for purposes of determining whether, as a result of his achievement of agreed-upon benchmarks related to a player's performance as a player or the Team's performance during a particular Season:

  \begin{enumerate}
  \def\labelenumii{(\roman{enumii})}
  \tightlist
  \item
    The player has earned a Performance Bonus included in his Player Contract in accordance with Section 3(b)(ii) above; or
  \item
    Any additional conditions or limitations applicable to a player's Compensation protection in accordance with Section 4(l)(ii) above have been satisfied.
  \end{enumerate}
\item
  No Player Contract may provide for (x) one (1) or more Performance Bonuses in accordance with Section 3(b)(ii) above, or (y) any additional conditions or limitations applicable to Compensation protection in accordance with Section 4(l)(ii) above, that are in either or both cases based in whole or in part on:

  \begin{enumerate}
  \def\labelenumii{(\roman{enumii})}
  \tightlist
  \item
    The Team's or the player's performance during any In-Season Tournament games or Play-In Games;
  \item
    The Team qualifying to participate in any In-Season Tournament knockout stage game or winning the IST Finals Game; or
  \item
    The Team qualifying to participate in one (1) or more Play-In Games;
  \end{enumerate}

  provided, however, that the foregoing shall not prevent a Player Contract from providing for (A) one (1) or more Performance Bonuses in accordance with Section 3(b)(ii) above, or (B) any additional conditions or limitations applicable to Compensation protection in accordance with Section 4(l)(ii) above, that are in either or both cases based in whole or in part on a player's or Team's performance in all Regular Season games.
\end{enumerate}

\hypertarget{void-contracts.}{%
\section{Void Contracts.}\label{void-contracts.}}

If a Player Contract fails to take effect or becomes void as a result of a Commissioner disapproval, the player's failure to pass a physical examination conducted pursuant to Exhibit 6 to such Contract, or the rescission of a trade conducted pursuant to Article VII, Section 8(e), then, in each such case:

\begin{enumerate}
\def\labelenumi{(\alph{enumi})}
\tightlist
\item
  the Team shall continue to possess such rights with respect to the player as the Team possessed at the time of the execution of the Contract, including, without limitation, any such rights that the Team possessed pursuant to Article VII, Section 6(b), Article X, and Article XI;
\item
  any Required Tender or Qualifying Offer that was outstanding at the time the Contract was executed shall continue in effect as if the Contract had not been executed (including if the original deadline for accepting the Required Tender or Qualifying Offer expired following the execution of the Contract), but for no fewer than six (6) business days following the Commissioner's disapproval, the Team's issuance of notice to the player that he did not pass the physical examination, or the rescission of such trade, as the case may be; and
\item
  in the case of a player who does not pass a physical examination pursuant to Exhibit 6: (i) the player shall not be permitted to accept such Required Tender or Qualifying Offer for a period of two (2) business days following his receipt of notice from the Team that he did not pass his physical examination, during which period the Team may elect to withdraw the Required Tender or Qualifying Offer, which shall have the consequences described in Article X, Section 4 or Article XI, Section 4, as the case may be; and (ii) if the Required Tender or Qualifying Offer is not withdrawn by the Team during this period, the Required Tender or Qualifying Offer shall thereafter be deemed amended so as to eliminate any Exhibit 6 that may be contained therein.
\end{enumerate}

\hypertarget{moratorium-period.}{%
\section{Moratorium Period.}\label{moratorium-period.}}

Except as permitted in the remainder of this Section 15, notwithstanding any other provision of this Agreement, no player and Team may enter into any oral or written agreement concerning terms and conditions of the player's employment, or reduce any such agreement to writing in the form of a Uniform Player Contract or amendment, during the Moratorium Period. The following shall be permitted:

\begin{enumerate}
\def\labelenumi{(\alph{enumi})}
\tightlist
\item
  During the Moratorium Period,

  \begin{enumerate}
  \def\labelenumii{(\roman{enumii})}
  \tightlist
  \item
    a player and a Team may negotiate over the terms and conditions of a Player Contract or an Extension that may be entered into following the conclusion of the Moratorium Period;
  \item
    a player and a Team may negotiate an Offer Sheet (as defined in Article XI, Section 5(b)) that may be entered into beginning at 12:01 p.m. eastern time on the first day of the Moratorium Period;
  \item
    a player may accept any Required Tender, Qualifying Offer, or ``Maximum Qualifying Offer'' (as defined in Article XI, Section 4(a)(ii)) that is outstanding; and
  \item
    a Team may exercise a Two-Way Contract's Standard NBA Contract Conversion Option in accordance with Article II, Section 11(f) above.
  \end{enumerate}
\item
  Beginning at 12:01 p.m. eastern time on the first day of the Moratorium Period,

  \begin{enumerate}
  \def\labelenumii{(\roman{enumii})}
  \tightlist
  \item
    a player and a Team may enter into an Offer Sheet;
  \item
    a First Round Pick and the Team that holds his draft rights may enter into a Rookie Scale Contract;
  \item
    a Second Round Pick and the Team that holds his draft rights may enter into a Player Contract signed pursuant to the Second Round Pick Exception;
  \item
    a player and a Team may enter into a Player Contract, not to exceed two (2) Seasons in length, that provides for a Salary for each Salary Cap Year equal to the Two-Way Player Salary or the Minimum Player Salary applicable to the player (with no bonuses of any kind); and
  \item
    a Team may exercise the Two-Way Player Conversion Option in a Contract with an Exhibit 10 in accordance with Article II, Section 11(h) above.
  \end{enumerate}
\end{enumerate}

\hypertarget{player-expenses}{%
\chapter{PLAYER EXPENSES}\label{player-expenses}}

\hypertarget{moving-expenses.}{%
\section{Moving Expenses.}\label{moving-expenses.}}

\begin{enumerate}
\def\labelenumi{(\alph{enumi})}
\tightlist
\item
  \textbf{Moving Expenses.} A Team's obligation to reimburse a player for ``reasonable'' expenses related to the assignment of a Player Contract from one Team to another (in accordance with Paragraph 10 of a Uniform Player Contract) shall extend to the reimbursement of the actual expenses incurred by such player in moving to the home territory of his new Team, provided that such expenses result directly from the assignment and are ordinary and reasonable, and provided further that, prior to his actually incurring such expenses, the player (i) consults with the Team to which his Contract has been assigned in advance concerning his move, and (ii) furnishes the Team with a written estimate of such proposed expenses from an established moving company so as to afford such assignee Team an opportunity to make reasonably comparable alternative arrangements for the move of the player. The player shall furnish such written estimate to the Team within a reasonable time following the notice of the assignment of the Player Contract. Upon receipt of such estimate from the player, the Team shall, within ten (10) days, either agree to reimburse the player for the expenses set forth in such estimate or make alternative arrangements (at the Team's expense) for the move of the player. ``Reasonable'' moving expenses shall include the cost of moving not more than one (1) automobile for the player (and not more than two (2) automobiles if the player is married).
\item
  \textbf{Hotel Accommodations.} A player whose Contract is assigned from one Team to another shall be reimbursed by the assignee Team for the cost of a hotel room in a hotel (comparable to that in which such Team's players are lodged while ``on the road'') in the assignee Team's home city for up to forty-six (46) days following the assignment.
\item
  \textbf{Housing Costs Reimbursement.} A player whose Contract is assigned from one Team to another shall be reimbursed by the assignee Team for the cost of his living quarters (either rent or mortgage expense) in the city from which he is assigned, for a period of three months after the date of the assignment; provided, however, that such payment shall: (i) be made only if and to the extent that the player is legally obligated for such costs; and (ii) not exceed \$6,000 per month. Any such payments shall be made on a pro rata basis if a full month is not involved.
\item
  \textbf{Proof of Expenses.} Prior to reimbursing an assigned player as provided in this Section, an assignee Team may require satisfactory proof that the player has paid the amounts for which he seeks reimbursement, and, in the case of housing costs reimbursements, satisfactory proof that the player is legally obligated to pay such housing costs and the amount thereof. Upon notice to the player, the assignee Team may, as an alternative to reimbursement, pay the expenses incurred upon assignment (in accordance with the foregoing provisions of this Section) directly to the persons, firms, or corporations involved.
\item
  \textbf{Player Obligation to Minimize Potential Liability.} So as to minimize the potential liability of NBA Teams under this Section, a player who does not establish permanent or year-round residence in the home city (or geographic vicinity thereof) of the Team by which he is employed shall use his best efforts (i) to obtain a short-term lease on the living quarters he selects, and (ii) to procure lease provisions authorizing him to sublet such premises and/or granting such Team the option to take over such lease in the event the Contract of such player is assigned to another NBA Team.
\end{enumerate}

\hypertarget{meal-expense-allowance.}{%
\section{Meal Expense Allowance.}\label{meal-expense-allowance.}}

\begin{enumerate}
\def\labelenumi{(\alph{enumi})}
\tightlist
\item
  The meal expense allowance, provided for in Paragraph 4 of a Uniform Player Contract, shall be as follows:

  \begin{enumerate}
  \def\labelenumii{(\roman{enumii})}
  \tightlist
  \item
    For the 2023-24 Season: \$156 per day.
  \item
    For each subsequent Season of this Agreement: the preceding Season's meal expense allowance amount adjusted for cost of living by applying to the preceding Season's meal expense allowance amount the percentage increase (or decrease) in the national Consumer Price Index for All Urban Consumers (CPI-U) from the June 1 through the May 31 immediately preceding such Season, and which shall be rounded off to the nearest whole dollar per day.
  \end{enumerate}
\item
  When a Team is ``on the road'' for less than a full day, a partial meal expense shall be paid based upon the time of departure from or time of arrival in the Team's home city, in accordance with the following:

  \begin{enumerate}
  \def\labelenumii{(\roman{enumii})}
  \tightlist
  \item
    Departure after 9:00 a.m. or arrival before 7:00 a.m., no meal expense allowance for breakfast.
  \item
    Departure after 1:00 p.m. or arrival before 11:30 a.m., no meal expense allowance for lunch.
  \item
    Departure after 7:00 p.m. or arrival before 5:30 p.m., no meal expense allowance for dinner.
  \end{enumerate}
\end{enumerate}

For purposes of this Section 2(b), the meal expense allowance for breakfast shall be deemed to be eighteen percent (18\%) of the applicable daily meal expense allowance (rounded off to the nearest whole dollar); the meal expense allowance for lunch shall be deemed to be twenty-eight percent (28\%) of the applicable daily meal expense allowance (rounded off to the nearest whole dollar); and the meal expense allowance for dinner shall be deemed to be fifty-four percent (54\%) of the applicable daily meal expense allowance (rounded off to the nearest whole dollar).

For purposes of this Agreement and Paragraph 4 of the Uniform Player Contract, the ``home city'' of an NBA Team shall be deemed to include only the city in which the facility regularly used by the Team for home games is located and any other location at which such home games are played, provided that such other location(s) is not more than seventy-five (75) miles from such city.

\hypertarget{benefits}{%
\chapter{BENEFITS}\label{benefits}}

\hypertarget{player-pension-benefits.}{%
\section{Player Pension Benefits.}\label{player-pension-benefits.}}

Subject to approval by the Internal Revenue Service (the ``IRS'') (to the extent such approval may be obtained pursuant to IRS procedures) and to the extent permitted by applicable law, the NBA shall provide the following pension benefits to NBA players and former NBA players in accordance with and subject to the terms and conditions of the National Basketball Association Players' Pension Plan, as restated effective July 1, 2017, and as amended from time to time and as to be modified as set forth herein (the ``Pension Plan''), and the Amended and Restated Agreement of Trust for the Pension Plan, effective as of July 1, 2017, and as amended from time to time (the ``Pension Trust Agreement''). (All capitalized terms used in this Section 1 not otherwise defined in this Agreement shall have the meanings set forth in the Pension Plan.)

\begin{enumerate}
\def\labelenumi{(\alph{enumi})}
\tightlist
\item
  \textbf{Benefits.}

  \begin{enumerate}
  \def\labelenumii{(\arabic{enumii})}
  \tightlist
  \item
    \textbf{Current Benefit.} As of the effective date of this Agreement, the monthly amount per Year of Credited Service payable as a Normal Retirement Pension (the ``Monthly Benefit'') is \$1,001.47.
  \item
    \textbf{Benefit Increase.}

    \begin{enumerate}
    \def\labelenumiii{(\roman{enumiii})}
    \tightlist
    \item
      The Pension Plan shall be amended to provide that, effective as of February 2, 2024, (A) the Normal Retirement Date shall be the first of the month following a player's sixty-second (62nd) birthday, and (B) the Early Retirement Date shall be any date on or after the first day of the month following the player's forty-fifth (45th) birthday and prior to the player's Normal Retirement Date. The Early Retirement Pension shall be the actuarial equivalent of the Normal Retirement Pension, as determined using modified actuarial equivalence factors to be specified in the Pension Plan amendment effective as of February 2, 2024.
    \item
      Effective for the Plan Year commencing February 2, 2024, and for each subsequent Plan Year during the term of this Agreement, the Normal Retirement Pension shall be adjusted (the Monthly Benefit following any such adjustment, the ``New Monthly Benefit'') such that, subject to Section 1(d) below, the New Monthly Benefit shall equal (A) the maximum annual dollar amount permitted under the Internal Revenue Code of 1986, as amended (the ``Code'') (and the regulations issued thereunder), as the Code and regulations are in effect as of the effective date of this Agreement, as such maximum benefit amount may be adjusted for future increases in the cost-of-living in the manner prescribed by Section 415(d)(2) of the Code, divided by (B) one hundred twenty (120). The maximum annual dollar amount permitted under the Code (and the regulations issued thereunder) for a player's Early Retirement Pension shall be determined using modified actuarial equivalence factors to be specified in the Pension Plan amendment effective as of February 2, 2024.
    \item
      Any increase in the Normal Retirement Pension or Early Retirement Pension payable on or after the date of this Agreement: (A) shall apply only to those players and beneficiaries (x) who have not yet received or begun to receive a benefit under the Pension Plan as of the first day of the month following the beginning of the Plan Year to which the increase relates (the ``New Benefit Increase Commencement Date'') or (y) who are receiving monthly benefits under the Pension Plan as of the New Benefit Increase Commencement Date; (B) shall be effective as of the New Benefit Increase Commencement Date; (C) shall apply only to any benefit payment(s) to be made on or after the applicable New Benefit Increase Commencement Date; and (D) shall not require the recalculation of any benefit payment(s) made prior to the applicable New Benefit Increase Commencement Date.
    \end{enumerate}
  \end{enumerate}
\item
  \textbf{Two-Way Players.} The Pension Plan shall be amended to provide that, for each Regular Season during the term of this Agreement, a Two-Way Player shall be considered to be on a Roster if he is (1) on an Active List, Inactive List, or Two-Way List of any Team on February 2nd of such Regular Season (or such other date that the parties may agree to), or (2) on the Active List of any Team for fifty percent (50\%) or more of the total Regular Season games played by the Team during such Regular Season.
\item
  \textbf{Pre-1965 Players and Pre-1965 Retirees.} Effective for the Plan Year commencing February 2, 2018, and for each subsequent Plan Year during the term of this Agreement:

  \begin{enumerate}
  \def\labelenumii{(\arabic{enumii})}
  \tightlist
  \item
    Pre-1965 Players shall continue to be entitled to receive the Normal Retirement Benefit in the amount and on the terms and conditions set forth in Article XIV of the Pension Plan.
  \item
    Pre-1965 Retirees shall continue to be entitled to receive the Retirement Benefit in the amount and on the terms and conditions set forth in Article XV of the Pension Plan.
  \item
    Any benefits that are unable to be paid to Pre-1965 Players or Pre-1965 Retirees under the Pension Plan because of the benefit limitations imposed by Section 415 of the Code shall be paid to such Pre-1965 Players and Pre-1965 Retirees pursuant to the National Basketball Association Excess Benefit Plan for Pre-1965 Players (the ``Pre-1965 Players Excess Benefit Plan'').
  \end{enumerate}
\item
  \textbf{Limitations on Benefits.} Notwithstanding anything contained herein to the contrary:

  \begin{enumerate}
  \def\labelenumii{(\arabic{enumii})}
  \tightlist
  \item
    Neither: (i) the pension benefits accrued or payable to any player or beneficiary for a Plan Year nor (ii) the New Monthly Benefit for a Plan Year shall exceed the maximum benefit amount permitted under the Code (and the regulations issued thereunder) as in effect for that Plan Year (as adjusted in accordance with the actuarial factors specified in the Pension Plan and as in effect on the date that the benefit accrues or commences (or is paid) or for the Plan Year for which the New Monthly Benefit is determined), as such maximum benefit amount may be adjusted for future increases in the cost-of-living in the manner provided under Section 415(d)(2) of the Code.
  \item
    Neither the pension benefits accrued nor payable to any player or beneficiary for a Plan Year shall exceed the maximum benefit amount permitted under the Code (and the regulations issued thereunder), as in effect as of the effective date of this Agreement, as adjusted in accordance with the actuarial factors specified in the Pension Plan, and as may be adjusted for future increases in the cost-of-living in the manner prescribed by Section 415(d)(2) of the Code.
  \item
    If all or any portion of the actuarially-determined annual contributions to be made to the Pension Plan would not be fully deductible under the Code when paid to the Pension Plan, the New Monthly Benefit shall not exceed the amount which would result in all of such contributions being fully-deductible when paid. In the event that any such contribution or portion thereof is not fully deductible when paid, the NBA and the Players Association agree to bargain in good faith with respect to an alternative arrangement to be provided by the NBA Teams to the players. The costs of any such alternative arrangement shall be at an annual cost (as determined on an after-tax basis) to the NBA Teams substantially equal to but no greater than the annual accrual cost that such Teams would have incurred under the Pension Plan to fund the amount by which the New Monthly Benefit is reduced pursuant to this Section 1(d)(3). If despite good faith negotiations, the NBA and the Players Association fail to agree with respect to an alternative arrangement as described above, such failure to agree shall not create any right: (A) to unilaterally implement during the term of this Agreement any terms concerning the provision of pension benefits to the players; (B) to lockout; or (C) to strike.
  \end{enumerate}
\item
  \textbf{Administration.}

  \begin{enumerate}
  \def\labelenumii{(\arabic{enumii})}
  \tightlist
  \item
    Subject to the provisions of Section 3.3(f) of the Pension Trust Agreement, which are hereby incorporated by reference and expressly designed to survive the expiration or termination of this Agreement, the Pension Plan shall continue to be jointly operated and administered by the NBA and the Players Association in accordance with Section 302(c)(5) of the Labor Management Relations Act of 1947, as amended, and the provisions of the Pension Trust Agreement and the Pension Plan.
  \item
    It is intended by the NBA and the Players Association that: (i) the Pension Plan shall continue to constitute a collectively-bargained multiemployer defined benefit pension plan that is tax-qualified under Section 401(a) of the Code; and (ii) the Pension Plan's corresponding Trust is exempt from taxation under the provisions of Section 501(a) of the Code.
  \item
    The daily operations of the Pension Plan shall continue to be delegated to one or more independent third-party administrators, as selected by the Board of Trustees of the Pension Plan in its sole discretion.
  \end{enumerate}
\item
  \textbf{Contributions/Funding.} The NBA and Players Association acknowledge and agree that the Teams shall continue at all times to contribute to the Plan at least the amount necessary to meet the Pension Plan's statutory minimum funding requirements under Section 412, Section 431, and, if applicable, Section 432 of the Code, or any other applicable law (the ``Minimum Funding Standards'') for such Plan Year, as determined by the actuaries of the Pension Plan. For any period during the term of this Agreement during which a new ``funding improvement plan'' (a ``FIP'') is required to be adopted by the Pension Plan under the Minimum Funding Standards, the funding benchmark for such FIP shall equal the funding benchmark required by the Minimum Funding Standards. The Teams may, in the sole discretion of the NBA, contribute to the Pension Plan more than the amount necessary to meet the Minimum Funding Standards; provided, however, that any such additional contribution amount shall not be greater than the contribution amount determined by the actuaries of the Pension Plan in accordance with the Pension Plan's historical scheduled contribution methodology. All contributions shall be conditioned on their being fully deductible by the Teams when paid.
\item
  \textbf{Players Employed by Toronto.}

  \begin{enumerate}
  \def\labelenumii{(\arabic{enumii})}
  \tightlist
  \item
    Players employed by Maple Leaf Sports \& Entertainment Partnership (or any successor thereto) (``Toronto'') or by an NBA Team located in any country other than the United States shall continue to receive pension benefits of comparable value. Except as otherwise provided in Section 1(g)(2), players employed by Toronto (``Toronto Players'') shall continue to receive such benefits by means of the Pension Plan and the Toronto Raptors Players' Pension Plan, as restated effective February 2, 2019, and as amended from time to time (the ``Toronto Plan''); provided, however, that a player shall not be eligible to participate (or continue to participate) in the Pension Plan for any period of time during which the player is both a resident of Canada for income tax purposes and a Toronto Player (a ``Canadian Resident'') but shall instead be eligible to receive a cash payment as described in Section 7 below.
  \item
    If the participation of Toronto Players in the Pension Plan would, at any time, result in the Pension Plan becoming subject to Canadian provincial pension legislation and/or Canadian federal income tax laws (to the extent that the application of such laws would result in adverse tax consequences to the Pension Plan, the NBA Teams or the Toronto Players) or result in the Toronto Plan's failure, at any future time, to either be qualified under the Code or registered under Canadian provincial pension legislation or Canadian federal tax laws, then any obligation to establish, maintain, or make contributions to the Pension Plan in respect of Toronto Players and the Toronto Plan pursuant to this Agreement or pursuant to any prior collective bargaining agreement shall terminate; provided, however, that any such termination shall not impair the legally binding effect of any other provision of this Agreement or the legally binding effect (if any) of any other provision of any prior collective bargaining agreement, nor shall it create any right: (i) to unilaterally implement during the term of this Agreement any terms concerning the provision of pension benefits to the players; (ii) to lockout; or (iii) to strike. In the event of such termination, the NBA and Players Association agree to bargain in good faith with respect to an alternative arrangement to be provided by Toronto to the Toronto Players. Any such alternative arrangement shall be at an annual cost (as determined on an after-tax basis) to Toronto substantially equal to but no greater than the annual accrual cost that Toronto would have incurred under the Pension Plan and the Toronto Plan. If despite good faith negotiations, the NBA and the Players Association fail to agree with respect to an alternative arrangement as described above, such failure to agree shall not create any right: (A) to unilaterally implement during the term of this Agreement any terms concerning the provision of pension benefits to the players; (B) to lockout; or (C) to strike.
  \item
    Subject to the provisions of Section 9.1(a) of the Toronto Plan, which are hereby incorporated by reference and expressly designed to survive the expiration or termination of this Agreement, the Toronto Plan shall continue to be jointly operated and administered by the NBA and Players Association in accordance with Section 302(c)(5) of the Labor Management Relations Act of 1947, as amended, Section 8(1)(b) of the Pension Benefits Act (as defined in the Toronto Plan), and the provisions of the Toronto Plan.
  \item
    It is intended by the NBA and the Players Association that: (i) the Toronto Plan shall continue to constitute a collectively-bargained single employer defined benefit pension plan that is sponsored by Toronto and is tax-qualified under Section 401(a) of the Code and registered under Section 147.1 of the Income Tax Act (as defined in the Toronto Plan) and Sections 9 and 12 of the Pension Benefits Act; and (ii) the Toronto Plan's corresponding trust fund shall continue to be exempt from taxation under the provisions of Section 149(1)(o) of the Income Tax Act and Section 501(a) of the Code.
  \item
    The daily operations of the Toronto Plan shall continue to be delegated to one or more independent third-party administrators, as selected by the Committee of the Toronto Plan in its sole discretion.
  \end{enumerate}
\item
  \textbf{Pension Plan Tax-Qualification Status.} Notwithstanding anything else in this Agreement: (1) if any change or amendment made to the Code, ERISA, or other applicable law, or to any regulations (whether final, temporary, or proposed) or rulings issued thereunder; (2) if any interpretation, application or enforcement (or any proposed interpretation, application, or enforcement), by a court of competent jurisdiction in the United States or by the IRS, of the Code, ERISA, or other applicable law, or any regulations or rulings issued thereunder; (3) if any regulations (whether final, temporary, or proposed) or rulings issued by the IRS under the Code or ERISA; or (4) if any provisions of this Agreement, including, without limitation, any of the amendments or benefit increases to be provided under the Pension Plan pursuant to this Section 1, would result in the Pension Plan no longer being a tax-qualified plan under Section 401(a) of the Code, or would require NBA Teams to incur costs over and above any costs required to be incurred to implement the provisions of this Agreement or any prior collective bargaining agreement in order for the Pension Plan to maintain its tax-qualified status under Section 401(a) of the Code (but only to the extent that such additional costs are incurred in connection with the provision of pension benefits to their non-player employees or to non-player employees of affiliates (within the meaning of Sections 414(b), (c) or (m) of the Code) of such Teams), then any obligation to continue to provide for the accrual of additional benefits under the Pension Plan pursuant to this Agreement or pursuant to any prior collective bargaining agreement shall terminate; provided, however, that any such termination shall not impair the legally binding effect of any other provision of this Agreement or the legally binding effect (if any) of any other provision of any prior collective bargaining agreement, nor shall it create any right: (i) to unilaterally implement during the term of this Agreement any terms concerning the provision of pension benefits to the players; (ii) to lockout; or (iii) to strike. In the event of such termination, the NBA and Players Association agree to bargain in good faith with respect to an alternative arrangement to be provided by the NBA Teams to the players. The costs of any such alternative arrangement shall be at an annual cost (as determined on an after-tax basis) to the NBA Teams substantially equal to but no greater than the annual accrual cost that such Teams would have incurred under the Pension Plan to fund the benefit described in this Section 1, commencing on the date of termination. If despite good faith negotiations, the NBA and the Players Association fail to agree with respect to an alternative arrangement as described above, such failure to agree shall not create any right: (A) to unilaterally implement during the term of this Agreement any terms concerning the provision of pension benefits to the players; (B) to lockout; or (C) to strike.
\item
  \textbf{Additional Pension Benefits Costs.} The NBA Teams shall pay all costs, including, without limitation, the cost of professional fees (e.g., attorneys, accountants, actuaries, and consultants) (``Professional Fees''), incurred in connection with: (1) the operation and administration of the Toronto Plan (but excluding the cost of contributions made by Toronto to the Toronto Plan), and (2) the determination and implementation of any alternative benefits pursuant to Sections 1(g)(2) and/or 1(h).
\item
  \textbf{Actuarial Determinations.} All actuarial determinations that need to be made in connection with, or under, the Pension Plan, including, without limitation, those necessary to implement this Section 1 and Section 9 below, shall be made by the actuaries of the Pension Plan. Any such actuarial determinations shall be binding and conclusive.
\end{enumerate}

\hypertarget{player-401k-benefits.}{%
\section{Player 401(k) Benefits.}\label{player-401k-benefits.}}

To the extent permitted by the Code and applicable law, the NBA shall provide the following 401(k) benefits to NBA players and former NBA players in accordance with and subject to the terms and conditions of the National Basketball Association Players' 401(k) Savings Plan as restated effective November 1, 2014, as amended from time to time and as to be modified as set forth herein (the ``401(k) Plan''). (All capitalized terms used in this Section 2 not otherwise defined in this Agreement shall have the meanings set forth in the 401(k) Plan and, for purposes of this Section 2, the term ``Compensation'' shall have the meaning set forth in the 401(k) Plan and not Article I or Exhibit A of this Agreement.)

\begin{enumerate}
\def\labelenumi{(\alph{enumi})}
\tightlist
\item
  \textbf{Current Benefits.} For each Plan Year commencing during the term of this Agreement, the 401(k) Plan shall continue to provide for: (1) Salary Deferral Contributions by players, (2) except as may be limited below, Matching Contributions by Teams in respect of Salary Deferral Contributions for a Salary Cap Year, as requested in writing by the Players Association, and (3) After Tax Contributions by players. The request for the Matching Contributions by the Players Association for a Season shall be made in writing prior to the commencement of that Season.
\item
  \textbf{Two-Way Matching Contributions.} Effective for all Regular Seasons during the term of this Agreement, the 401(k) Plan shall be amended to eliminate the matching contribution formula currently applicable to Two-Way Players and provide Two-Way Players who are Eligible Players a Matching Contribution pursuant to the formula set forth in Section 3.7 of the 401(k) Plan.
\item
  \textbf{Timing of Matching Contributions.} Any Matching Contributions to be made to the 401(k) Plan in respect of each Season shall be made no later than thirty (30) days following the completion of the Audit Report for the Salary Cap Year covering such Season.
\item
  \textbf{Limitations on Benefits.} Notwithstanding anything contained herein to the contrary:

  \begin{enumerate}
  \def\labelenumii{(\arabic{enumii})}
  \tightlist
  \item
    Matching Contributions, Salary Deferral Contributions, and After Tax Contributions shall at all times be subject to all applicable limitations under the Code, including, without limitation, the maximum limitation on contributions under Code Section 415, the maximum limitation on compensation under Code Section 401(a)(17), and the maximum limitation on 401(k) deferrals under Code Section 402(g).
  \item
    The total amount of the Salary Deferral Contributions and Matching Contributions to be made to the 401(k) Plan shall be limited to an amount that, taking into account only Compensation paid to current players by the Teams, would result in all of such Salary Deferral Contributions and Matching Contributions being fully deductible under the Code (and, where applicable, Canadian income tax laws) when paid to the 401(k) Plan. If, for any reason, all or a portion of the Salary Deferral Contributions and Matching Contributions to be made to the 401(k) Plan will not, when paid to the 401(k) Plan, be fully deductible under the Code, the NBA and the Players Association agree that the contributions shall be reduced to result in all such contributions being fully deductible when paid.
  \end{enumerate}
\item
  \textbf{Players Employed by Toronto.} The terms of the 401(k) Plan shall continue to permit participation by Toronto Players on a tax-effective basis under Canadian income tax laws; provided, however, that a player shall not be eligible to participate in the 401(k) Plan for the period of time during which the player is a Canadian Resident but shall instead be eligible to receive a cash payment as described in Section 7 below. If the NBA and the Players Association should determine that the 401(k) Plan cannot continue to be provided to Toronto Players on a tax-effective basis under Canadian federal income tax laws, the NBA and Players Association agree to bargain in good faith with respect to an alternative arrangement to be provided by Toronto to the Toronto Players. The costs of any such alternative arrangement shall be at an annual cost (as determined on an after-tax basis) to Toronto substantially equal to but no greater than the annual cost that Toronto would have incurred under the 401(k) Plan with respect to the Matching Contributions for the Toronto Players. If despite good faith negotiations, the NBA and the Players Association fail to agree with respect to an alternative arrangement as described above, such failure to agree shall not create any right: (i) to unilaterally implement during the term of this Agreement any terms concerning the provision of 401(k) benefits to the players; (ii) to lockout; or (iii) to strike.
\item
  \textbf{401(k) Plan Tax-Qualification Status.} Notwithstanding anything else in this Agreement: (1) if any change or amendment made to the Code, ERISA, or other applicable law, or to any regulations (whether final, temporary, or proposed) or rulings issued thereunder; (2) if any interpretation, application, or enforcement (or any proposed interpretation, application, or enforcement), by a court of competent jurisdiction in the United States or by the IRS, of the Code, ERISA, or other applicable law, or any regulations or rulings issued thereunder; (3) if any regulations (whether final, temporary, or proposed) or rulings issued by the IRS under the Code or ERISA; or (4) if any provisions of this Agreement would result in the 401(k) Plan no longer being a tax-qualified plan under Section 401(a) of the Code, or would require NBA Teams to incur costs over and above any costs required to be incurred to implement the provisions of this Agreement or any prior collective bargaining agreement in order for the 401(k) Plan to maintain its tax-qualified status under Section 401(a) of the Code (but only to the extent that such additional costs are incurred in connection with the provision of benefits to their non-player employees or to non-player employees of affiliates (within the meaning of Sections 414(b), (c), or (m) of the Code) of such Teams), then any obligation to maintain or make contributions to the 401(k) Plan pursuant to this Agreement or pursuant to any prior collective bargaining agreement shall terminate; provided, however, that any such termination shall not impair the legally binding effect of any other provision of this Agreement or the legally binding effect (if any) of any other provision of any prior collective bargaining agreement, nor shall it create any right: (i) to unilaterally implement during the term of this Agreement any terms concerning the provision of 401(k) benefits to the players; (ii) to lockout; or (iii) to strike. In the event of such termination, the NBA and Players Association agree to bargain in good faith with respect to an alternative arrangement to be provided by the NBA Teams to the players. Any such alternative arrangement shall be at an annual cost (as determined on an after-tax basis) to the NBA Teams substantially equal to but no greater than the annual cost that such Teams would have incurred under the 401(k) Plan with respect to Matching Contributions commencing on the date of termination. If despite good faith negotiations, the NBA and the Players Association fail to agree with respect to an alternative arrangement as described above, such failure to agree shall not create any right: (x) to unilaterally implement during the term of this Agreement any terms concerning the provision of 401(k) benefits to the players; (y) to lockout; or (z) to strike.
\item
  \textbf{Additional 401(k) Benefits Costs.} The NBA Teams shall pay all costs incurred in connection with the operation and administration of the 401(k) Plan (and in connection with the determination and implementation of any alternative arrangement pursuant to Section 2(e) and/or Section 2(f)), including, without limitation, the cost of Professional Fees and the 401(k) Plan's recordkeeper's fixed fee for recordkeeping and other administrative services provided to the 401(k) Plan. Notwithstanding the previous sentence, this Section 2(g) shall not apply to: (1) any costs or fees attributable to a participant-initiated transaction under the 401(k) Plan; or (2) any investment fees or expenses charged directly against the return on any investment options under the 401(k) Plan, which, in each case, shall be paid by the applicable participant.
\end{enumerate}

\hypertarget{player-health-and-welfare-benefits.}{%
\section{Player Health and Welfare Benefits.}\label{player-health-and-welfare-benefits.}}

Except as set forth below in this Section 3, as of the effective date of this Agreement, and continuing until the expiration or termination of this Agreement, to the extent permitted by applicable law, the NBA shall provide the following health and welfare benefits to NBA players and former NBA players in accordance with and subject to the terms and conditions of the National Basketball Association Players' Health and Welfare Benefit Plan, as in effect on the date of this Agreement, as amended from time to time and as to be modified as set forth herein (the ``Health and Welfare Benefit Plan'') and the Agreement of Trust for the NBA Players' Health and Welfare Benefit Plan, as restated effective July 1, 2017, and as amended from time to time (the ``Health and Welfare Benefit Trust Agreement'' and the trust, the ``Health and Welfare Benefit Trust''). (All capitalized terms used in this Section 3 not otherwise defined in this Agreement shall have the meanings set forth in the Health and Welfare Benefit Plan.)

\begin{enumerate}
\def\labelenumi{(\alph{enumi})}
\tightlist
\item
  \textbf{Benefits.} The Health and Welfare Benefit Plan shall continue to provide the following benefits, which shall be operated and administered through the Health and Welfare Benefit Trust:

  \begin{enumerate}
  \def\labelenumii{(\arabic{enumii})}
  \tightlist
  \item
    A health reimbursement arrangement (the ``HRA Benefit'') for players who played in the NBA during and/or after the 2000-01 Season will continue to be operated in accordance with the Health and Welfare Benefit Plan, which arrangement shall be administered and operated in compliance with IRS and U.S. Department of Labor rules applicable to such arrangements. Except as may otherwise be agreed to by the NBA and the Players Association, any contributions to fund the HRA Benefit in respect of each Salary Cap Year shall be made no later than ninety (90) days following the completion of the Audit Report for such Salary Cap Year.
  \item
    The following insurance benefits provided to players:

    \begin{enumerate}
    \def\labelenumiii{(\roman{enumiii})}
    \tightlist
    \item
      Life insurance and accidental death and dismemberment benefits, which, as of the date of this Agreement, are being provided through the Metropolitan Life Insurance Company Policy No.~0122986; provided, however, the Health and Welfare Benefit Plan shall be amended to provide that Two-Way Players shall receive the same level of coverage under such policy as players who signed a Standard NBA Contract.
    \item
      Disability insurance benefits, which, as of the date of this Agreement, are being provided through the Houston Casualty Company Policy No.~20/7005744.
    \item
      Except as otherwise provided in Section 3(a)(2)(iv), medical, dental, vision, and prescription drug insurance benefits which, as of the date of this Agreement, are being provided through the CIGNA HealthCare Policy No.~3211244 and the EyeMed Vision Care Policy No.~9886987; provided, however, that the Health and Welfare Benefit Plan shall be amended to provide that Two-Way Players shall not be required to contribute toward their medical, dental, vision, and prescription drug insurance premiums.
    \item
      For players other than Two-Way Players who are ``qualified expatriates'' under the Expatriate Health Coverage Clarification Act of 2014, expatriate medical, vision, and prescription drug insurance benefits through the Cigna Policy No.~07578A and the EyeMed Vision Care Policy No.~9886987.
    \end{enumerate}
  \item
    All of the benefits provided for in Section 3(a)(2) are subject to their permissibility and availability under applicable law.
  \item
    The Board of Trustees of the Health and Welfare Benefit Trust (the ``Health and Welfare Trustees'') may make changes to any of the insurance programs provided under Section 3(a)(2), provided that any such change that would result in an increase in the costs or a change in the types or levels of any of the benefits, or that would change any such program from an insured program to a self-insured program or vice versa, must be mutually agreed to in writing by the NBA and the Players Association.
  \item
    Subject to Sections 3(a)(5)(i)-(ii) below, the NBA and the Players Association shall continue to provide retiree health insurance benefits which, as of the effective date of this Agreement, are being provided through UnitedHealthcare Policy Numbers 908971, 16160, 16161 and 16162, Cigna Policy Number 3342982, and EyeMed Policy Number 10250821001 (collectively, the ``Retiree Medical Plan'').

    \begin{enumerate}
    \def\labelenumiii{(\roman{enumiii})}
    \tightlist
    \item
      The Retiree Medical Plan will be continued only for the term of this Agreement; provided, however, that the NBA and the Players Association (or, if so delegated by the NBA and the Players Association in writing, the Health and Welfare Trustees) reserve the right, by mutual written agreement, to modify, amend, or terminate, in whole or in part, the Retiree Medical Plan with respect to any or all eligible retirees and their eligible dependents at any time or for any reason, and no eligible retirees or eligible dependents (or other NBA players, retired NBA players or their dependents) shall under any circumstances have any vested rights of any nature with respect to the Retiree Medical Plan or any retiree health benefit (whether or not the player or retired player, or their dependents, has participated in the Retiree Medical Plan).
    \item
      The NBA and the Players Association (or, if so delegated by the NBA and the Players Association in writing, the Health and Welfare Trustees) reserve the right, by mutual written agreement, to increase or otherwise change the amount of monthly premiums under the Retiree Medical Plan charged to players at any time and for any reason.
    \end{enumerate}
  \item
    The Health and Welfare Benefit Plan shall continue to provide for a tuition and career transition benefit (the ``Tuition Reimbursement Benefit''), to be modified as set forth below, which reimburses eligible players for qualifying educational expenses (``Educational Expenses''). For purposes of this Section 3(a)(6), an ``eligible player'' is a player who is eligible under plan rules to use the Tuition Reimbursement Benefit.

    \begin{enumerate}
    \def\labelenumiii{(\roman{enumiii})}
    \tightlist
    \item
      Each eligible player in respect of each Salary Cap Year during or after the 2023-24 Salary Cap Year shall be entitled to a Tuition Reimbursement Benefit equal to the lesser of (A) \$41,667, and (B) the difference between \$125,000 and the sum of the amount of all Tuition Reimbursement Benefits previously earned by such player in respect of prior Salary Cap Years (including, for clarity, Salary Cap Years prior to the 2023-24 Salary Cap Year). Each eligible player with three (3) or more Years of NBA Service as of the date of this Agreement shall receive a one-time increase in his Tuition Reimbursement Benefit equal to \$24,000.
    \item
      All eligible players, including players who have a Tuition Reimbursement Benefit as of the date of this Agreement, may be reimbursed for each calendar year up to a maximum of \$62,500. Notwithstanding anything to the contrary, the maximum aggregate amount of Educational Expenses for which all eligible players may be reimbursed for each Salary Cap Year is \$4,276,185.
    \end{enumerate}
  \item
    The Health and Welfare Benefit Plan shall be amended to provide that, for purposes of the HRA, retiree medical, and tuition reimbursement benefits described in Sections 3(a)(1), 3(a)(5), and 3(a)(6) above, for each Regular Season during the term of this Agreement, a Two-Way Player shall earn a Year of NBA Service if he is (i) on an Active List, Inactive List, or Two-Way List of any Team on February 2nd of such Regular Season (or such other date that the parties may agree to), or (ii) on the Active List of any Team for fifty percent (50\%) or more of the total Regular Season games played by the Team during such Regular Season.
  \end{enumerate}
\item
  \textbf{Administration.}

  \begin{enumerate}
  \def\labelenumii{(\arabic{enumii})}
  \tightlist
  \item
    The Health and Welfare Benefit Trust shall continue to be jointly operated and administered by the NBA and Players Association in accordance with Section 302(c)(5) of the Labor Management Relations Act of 1947, as amended, and the provisions of the Health and Welfare Benefit Trust Agreement and the Health and Welfare Benefit Plan, as to be amended pursuant to this Agreement. It is intended by the NBA and Players Association that the Health and Welfare Benefit Plan and Health and Welfare Benefit Trust shall continue to constitute a collectively-bargained voluntary employees' beneficiary association that qualifies as a tax exempt organization under the provisions of Section 501(c)(9) of the Code.
  \item
    The Health and Welfare Benefit Trust Agreement shall continue to provide that the Health and Welfare Benefit Trust and Health and Welfare Benefit Plan will be administered by the Health and Welfare Trustees. The daily operations of the Health and Welfare Benefit Plan and each of the benefits provided thereunder shall continue to be delegated to one or more independent third-party administrators and/or insurers, as applicable.
  \item
    For the avoidance of doubt, nothing in this Section 3(b) shall prevent the Education Trust (defined below) from engaging or hiring an academic advisor or career counselor to assist with player outreach and similar functions with respect to the tuition reimbursement and career transition program set forth in Section 3(a)(6).
  \end{enumerate}
\item
  \textbf{Players Employed by Toronto.} The terms of the Health and Welfare Benefit Plan shall continue to permit participation by Toronto Players on the same basis as players who are not Toronto Players; provided, however, that a player shall not be eligible to participate in the HRA Benefit for the period of time during which the player is a Canadian Resident but shall instead be eligible to receive a cash payment as described in Section 7 below. If the NBA and the Players Association determine that the Health and Welfare Benefit Plan cannot provide one or more of the benefits described in Section 3(a) to Toronto Players (1) that are substantially equivalent to the benefits provided to players employed by Teams located in the United States or (2) on a tax-effective basis under Canadian federal income tax laws, the NBA and Players Association agree to bargain in good faith with respect to an alternative arrangement to be provided by Toronto to the Toronto Players. The annual cost incurred by the Teams in connection with any such alternative arrangement (as determined on an after-tax basis) shall not exceed the annual cost that such Teams would have incurred to fund the applicable benefit(s) described in Section 3(a) for such Toronto Player. The cost to Toronto of funding any alternative arrangement(s) to any of the benefit(s) described in Section 3(a) shall be subject to the limitations set forth in this Agreement. If despite good faith negotiations, the NBA and the Players Association fail to agree with respect to any alternative arrangement(s) as described above, such failure to agree shall not create any right (A) to unilaterally implement during the term of this Agreement any terms concerning the provision of benefits provided or to be provided by the Health and Welfare Benefit Plan; (B) to lockout; or (C) to strike.
\item
  \textbf{Deductibility of Contributions/Regulatory Changes.}

  \begin{enumerate}
  \def\labelenumii{(\arabic{enumii})}
  \tightlist
  \item
    The Health and Welfare Benefit Trust and the Health and Welfare Benefit Plan shall be operated and administered in a manner that will result in all contributions by the Teams being fully deductible under the Code (and, where applicable, Canadian income tax laws) when paid to the Health and Welfare Benefit Trust (or directly to an insurance carrier for a benefit provided under the Health and Welfare Benefit Plan). If any Team is disallowed a deduction (in whole or in part) for such contributions, and unless the NBA determines otherwise, the obligation to provide the benefit (or portion of the benefit under the Health and Welfare Benefit Plan to which the contribution relates) and to make further contributions to provide the benefit (or portion of the benefit under the Health and Welfare Benefit Plan to which the contribution relates) shall immediately terminate and the provisions of Section 3(d)(3) shall apply.
  \item
    In the event that any benefit under the Health and Welfare Benefit Plan is no longer permissible or available due to applicable laws (a ``Regulatory Change''), the obligation to provide the benefit shall immediately terminate and the provisions of Section 3(d)(3) shall apply.
  \item
    Any termination of the Health and Welfare Benefit Plan or a benefit under such plan pursuant to Sections 3(d)(1)-(2) shall not impair the legally binding effect of any other provision of this Agreement, or the legally binding effect (if any) of any other provision of any prior collective bargaining agreement, nor shall it create any right (i) to unilaterally implement, during the term of this Agreement, any terms concerning the provision of the Health and Welfare Benefit Plan (or the applicable benefit provided or to be provided); (ii) to lockout; or (iii) to strike. In the event of any termination pursuant to Sections 3(d)(1)-(2) of the Health and Welfare Benefit Plan or a benefit under such plan, the NBA and Players Association agree to bargain in good faith with respect to alternative arrangement(s) to be provided by the NBA Teams to the players; provided, however, that any such alternative arrangement(s) shall be subject to the terms and conditions set forth in this Agreement, including, without limitation, with respect to an alternative arrangement to the Retiree Medical Plan, the terms and conditions set forth in Section 3(a)(5). The annual cost incurred by the NBA Teams in connection with any such alternative arrangement(s) (as determined on an after-tax basis) shall not exceed the annual cost that such Teams would have incurred in providing the relevant benefit(s) under the Health and Welfare Benefit Plan commencing on the date of termination. Any such alternative arrangement(s) shall, to the extent permitted by applicable law and the Health and Welfare Benefit Plan, be funded by such monies as may then remain in the Health and Welfare Benefit Trust and, if the monies remaining in the Health and Welfare Benefit Trust may not lawfully be used for, or are insufficient for, such purpose, such alternative arrangement(s) shall be funded by the NBA Teams. Any such alternative arrangement(s) shall be operated and administered in a manner that will result in all contributions by the Teams being fully deductible under the Code (and, where applicable, Canadian income tax laws) when paid. The costs of funding any alternative arrangement(s) shall be subject to the limitations set forth in this Agreement. If despite good faith negotiations, the NBA and the Players Association fail to agree with respect to any alternative arrangement(s) as described above, such failure to agree shall not create any right: (A) to unilaterally implement, during the term of this Agreement, any terms concerning the provision of benefits provided or to be provided by the Health and Welfare Benefit Plan; (B) to lockout; or (C) to strike.
  \end{enumerate}
\item
  \textbf{Additional Health and Welfare Benefits Costs.} Except as otherwise set forth in Section 9(b)(12), the NBA Teams shall pay all costs, including, without limitation, the cost of Professional Fees, incurred in connection with (1) the operation and administration of the Health and Welfare Benefit Plan Trust and the Health and Welfare Benefit Plan (including, without limitation, the operation and administration of the benefits set forth in Section 3(a) above and any other benefits to be provided under the Health and Welfare Benefit Plan), and (2) the determination and implementation of any alternative arrangement pursuant to Section 3(d)(3). Notwithstanding the preceding sentence, this Section 3(e) shall not apply to any costs or fees attributable to investment management fees in connection with the investment of Health and Welfare Benefit Trust assets. Such costs and fees shall: (i) be paid out of the assets of the Health and Welfare Benefit Trust; and (ii) be excluded for purposes of all calculations called for under this Agreement of, or relating to, Benefits (including, without limitation, for purposes of: (A) preparing the Audit Report, Interim Audit Report, or Interim Designated Share Audit Report; and (B) calculating Total Benefits, Total Salaries and Benefits, and Projected Benefits).
\end{enumerate}

\hypertarget{the-post-career-income-plan.}{%
\section{The Post-Career Income Plan.}\label{the-post-career-income-plan.}}

To the extent permitted by the Code and applicable law, the NBA shall provide the following post-career income benefits to NBA players and former NBA players in accordance with and subject to the terms and conditions of the National Basketball Association Players' Qualified Post-Career Income Plan, as restated effective July 1, 2017, and as amended from time to time (the ``Qualified Plan'') and the National Basketball Association Players' Non-Qualified Post-Career Income Plan, as restated effective July 1, 2017, and as amended from time to time (the ``Non-Qualified Plan,'' and, when referenced collectively with the Qualified Plan, the ``Post-Career Income Plan''). (All capitalized terms used in this Section 4 not otherwise defined in this Agreement shall have the meanings set forth in the Post-Career Income Plan.)

\begin{enumerate}
\def\labelenumi{(\alph{enumi})}
\tightlist
\item
  \textbf{Current Benefits.}

  \begin{enumerate}
  \def\labelenumii{(\arabic{enumii})}
  \tightlist
  \item
    Effective for the Contribution Year (defined below) commencing November 1, 2023, and for each subsequent Contribution Year during the term of this Agreement, the Post-Career Income Plan shall provide for: (i) a Team contribution to the Post-Career Income Plan for Eligible Players to be used to purchase Post-Career Annuities (the ``Team Contribution''); and (ii) elective Player Contributions made by Qualifying Players to the Non-Qualified Plan to be used to purchase Post-Career Annuities on such players' behalf. The Team Contribution for each Eligible Player for each Contribution Year shall equal (A) the Additional Benefit Amount (defined below) divided by the total number of Eligible Players for such Contribution Year (including, for this purpose only, any Canadian Resident who but for the fact that he is a Canadian Resident would otherwise be an Eligible Player) (such quotient, an Eligible Player's ``Allocated Share''), less (B) tax withholding (solely with respect to contributions made to the Non-Qualified Plan) in the manner described in Section 3.3 of the Non-Qualified Plan (``Tax Withholding''). For each Contribution Year, a portion of a player's Allocated Share shall be contributed to the Qualified Plan on behalf of such player pursuant to the terms and conditions described in the Qualified Plan, and a portion to the Non-Qualified Plan pursuant to the terms and conditions described in the Non-Qualified Plan. For purposes of this Section 4, a ``Contribution Year'' means each November 1 through October 31 in respect of which a Team Funding Pool (defined below) is provided under this Section 4.
  \item
    Notwithstanding anything in this Section 4(a) to the contrary, and subject to the requirements of the Code and IRS rules and regulations, if the Board of Trustees of the Post-Career Income Plan (the ``PCIP Trustees'') determines, after Post-Career Annuities have been purchased for Eligible Players for a Contribution Year, that a present or former player should have received an Allocated Share for such Contribution Year but did not receive an Allocated Share, such present or former player shall be entitled to an Allocated Share equal to the amount of the Allocated Share made to the other Eligible Players for such Contribution Year, which shall be used to purchase one or more Post-Career Annuities in the same manner and on the same terms as the other Eligible Players for such Contribution Year. Unless practicable and otherwise agreed to by the PCIP Trustees, the cost of such Allocated Share shall not require a retroactive reduction in the Allocated Share and Post-Career Annuities of the other Eligible Players for such Contribution Year but rather shall be paid from the Additional Benefit Amount for the next Season (or, to the extent the Additional Benefit Amount for the next Season is insufficient, future Seasons). In addition, the cost of any additional fees or expenses charged by the Insurer for the purchase of such Post-Career Annuity (or for the purchase of any other Post-Career Annuity(ies) under the Post-Career Income Plan on a retroactive basis) shall also be paid from the Additional Benefit Amount for the next Season (or, to the extent the Additional Benefit Amount for the next Season is insufficient, future Seasons).
  \end{enumerate}
\item
  \textbf{Deductibility of Team Contributions/Regulatory Changes.}

  \begin{enumerate}
  \def\labelenumii{(\arabic{enumii})}
  \tightlist
  \item
    The Post-Career Income Plan shall be structured and maintained in a manner that will result in the Team Funding Pool being fully deductible under the Code (and, where applicable, Canadian laws) when used toward Team Contributions contributed to the Post-Career Income Plan. In the event that a Team is disallowed a deduction (in whole or in part) for its portion of the Team Funding Pool, then the Team shall be returned such disallowed deduction from the Post-Career Income Plan; provided, however, that, if such portion may not be returned to the Team under the terms of the Plan or the applicable Group Annuity Contract or applicable law, then such Team shall instead be reimbursed for the lost tax benefit resulting from the disallowance of the deduction from the Additional Benefit Amount for the next Season (or, to the extent the Additional Benefit Amount for the next Season is insufficient, future Seasons) following the date such Team submits satisfactory documentation of the disallowance to the PCIP Trustees.
  \item
    Notwithstanding anything else in this Agreement, if any event or occurrence, including, without limitation, (i) any change or amendment made to the Code, ERISA, or other applicable law, or to any regulations (whether final, temporary, or proposed regulations), or rulings or formal guidance issued thereunder, (ii) any interpretation, application, or enforcement (or any proposed interpretation, application, or enforcement), by a court of competent jurisdiction in the United States or by the IRS, of the Code, ERISA, or other applicable law, or any regulations or rulings issued thereunder, (iii) any regulations (whether final, temporary, or proposed regulations), or rulings or formal guidance issued by the IRS under the Code or ERISA or (iv) any provisions of this Agreement, including, without limitation, the provisions of this Section 4(b), would result in the Teams being disallowed a deduction (in whole or in part) for contributions made to the Post-Career Income Plan, then any obligation to maintain the Post-Career Income Plan pursuant to this Agreement shall, at the option of the NBA, terminate; provided, however, that any such termination shall not impair the legally binding effect of any other provision of this Agreement or the legally binding effect (if any) of any other provision of any prior collective bargaining agreement, nor shall it create any right (A) to unilaterally implement during the term of this Agreement any terms concerning the provision of post-employment benefits to the players; (B) to lockout; or (C) to strike.
  \item
    Notwithstanding anything else in this Agreement, if any event or occurrence, including, without limitation, (i) any change or amendment made to the Code, ERISA, or other applicable law, to any regulations (whether final, temporary, or proposed regulations), or rulings or formal guidance issued thereunder, (ii) any interpretation, application, or enforcement (or any proposed interpretation, application, or enforcement), by a court of competent jurisdiction in the United States or by the IRS, of the Code, ERISA, or other applicable law, or any regulations or rulings issued thereunder, (iii) any regulations (whether final, temporary, or proposed regulations), or rulings or formal guidance issued by the IRS under the Code or ERISA or (iv) any provisions of this Agreement, including, without limitation, the provisions of this Section 4, would result in the Qualified Plan no longer being a tax-qualified plan under Section 401(a) of the Code or would require NBA Teams to incur costs over and above any costs required to be incurred to implement the Qualified Plan in order to maintain its tax-qualified status under Section 401(a) of the Code (but only to the extent that such additional costs are incurred in connection with the provision of benefits to their non-player employees or to non-player employees of affiliates (within the meaning of Sections 414(b), (c), or (m) of the Code) of such Teams), then any obligation to maintain and/or make contributions in respect of the Qualified Plan pursuant to this Agreement shall terminate; provided, however, that any such termination shall not impair the legally binding effect of any other provision of this Agreement or the legally binding effect (if any) of any other provision of any prior collective bargaining agreement, nor shall it create any right (A) to unilaterally implement during the term of this Agreement any terms concerning the provision of post-employment benefits to the players; (B) to lockout; or (C) to strike.
  \item
    If the Taxable Allocated Share attributable to Eligible Players would be subject to a federal income tax rate higher than the rate that would apply if the Taxable Allocated Share were paid as Base Compensation, then any obligation to maintain the Post-Career Income Plan pursuant to this Agreement shall, at the option of the Players Association, terminate; provided, however, that any such termination shall not impair the legally binding effect of any other provision of this Agreement or the legally binding effect (if any) of any other provision of any prior collective bargaining agreement, nor shall it create any right (i) to unilaterally implement during the term of this Agreement any terms concerning the provision of post-employment benefits to the players; (ii) to lockout; or (iii) to strike.
  \item
    In the event of a termination described in Sections 4(b)(2)-(4), the NBA and Players Association agree to bargain in good faith with respect to an alternative arrangement to be provided by the NBA Teams to the players. The annual cost to the Teams of any such alternative arrangement (as determined on an after-tax basis) shall be substantially equal to but no greater than the annual cost that such Teams would have incurred under the Post-Career Income Plan on the date of termination. The cost of funding of any such alternative arrangement shall be as set forth in Section 4(d)(1). If despite good faith negotiations, the NBA and the Players Association fail to agree with respect to an alternative arrangement as described above, such failure to agree shall not create any right (i) to unilaterally implement during the term of this Agreement any terms concerning the provision of post-employment benefits to the players; (ii) to lockout; or (iii) to strike.
  \end{enumerate}
\item
  \textbf{Players Employed by Toronto.} The terms of the Post-Career Income Plan shall continue to permit participation by Toronto Players on a tax-effective basis under Canadian income tax laws; provided, however, that a player shall not be eligible to participate in the Post-Career Income Plan for the period of time during which the player is a Canadian Resident but shall instead be eligible to receive a cash payment as described in Section 7 below. If the NBA and the Players Association should determine that the Post-Career Income Plan cannot continue to be provided to Toronto Players on a tax-effective basis under Canadian federal income tax laws or that either the Qualified Plan or the Non-Qualified Plan would become subject to Ontario's Pension Benefits Act, the NBA and Players Association agree to bargain in good faith with respect to an alternative arrangement to be provided by Toronto to the Toronto Players. The cost of any such alternative arrangement to be provided in any Contribution Year shall come from such year's Team Funding Pool and shall equal an Eligible Player's Allocated Share for such Contribution Year as reduced by all federal, state, local, payroll, or other tax obligations of any kind (including, where applicable, Canadian tax) applicable to such player as Toronto, in the exercise of its reasonable discretion, deems necessary. If despite good faith negotiations, the NBA and the Players Association fail to agree with respect to an alternative arrangement as described above, such failure to agree shall not create any right (1) to unilaterally implement during the term of this Agreement any terms concerning the provision of post-employment benefits to the players; (2) to lockout; or (3) to strike.
\item
  \textbf{Funding.}

  \begin{enumerate}
  \def\labelenumii{(\arabic{enumii})}
  \tightlist
  \item
    For each Season, except as provided below, one percent (1\%) of BRI for such Season (the ``Additional Benefit Amount'') shall be used to fund the Team Funding Pool (or the alternative arrangement referenced in Sections 4(b)(5) and 4(c)); provided, however, that the Additional Benefit Amount for a Season shall be subject to reduction or elimination pursuant to Article VII, Section 12(b)(1). In no event shall the Additional Benefit Amount be used for any purpose other than as set forth in the immediately foregoing sentence. For purposes of all calculations called for under this Agreement of, or relating to, Benefits (including, without limitation, for purposes of (i) preparing the Audit Report, Interim Audit Report, or Interim Designated Share Audit Report, and (ii) calculating Total Benefits, Total Salaries and Benefits, and Projected Benefits), the amount to be included with respect to the Additional Benefit Amount shall be the full Additional Benefit Amount specified in this Section 4(d) and not the reduced Additional Benefit Amount provided for under Article VII, Section 12(b)(1).
  \item
    For each Contribution Year, all or a portion of the Additional Benefit Amount as determined under Section 4(d)(1) (the ``Team Funding Pool'') shall be used to fund the Post-Career Income Plan.
  \item
    The Teams shall contribute the Team Funding Pool, less Tax Withholding, into the Post-Career Income Plan each November following the Contribution Year to which it relates or, if later, within one hundred and twenty (120) days following the completion of the Audit Report covering the November 1 of such Contribution Year.
  \item
    For each Salary Cap Year in which the amortized portion of the amount to be included in BRI from applicable equity securities, pursuant to Article VII, Section 1(a)(13), exceeds \$5 million, the NBA and the Players Association shall meet and confer to discuss in good faith the possibility of creating a funding pool equal to 50\% of such amortized portion (the ``Equity Proceeds Funding Pool'') to fund after-tax contributions to the Non-Qualified Plan for players who were Eligible Players in any of the years during the period beginning with the Salary Cap Year in which the equity securities were received and continuing through the first Salary Cap Year in respect of which contributions to the Non-Qualified Plan are made in respect of such equity. For clarity, in the event that the NBA and the Players Association agree to provide the above-described after-tax benefits, then, for purposes of all calculations called for under this Agreement of, or relating to, Benefits (including, without limitation, for purposes of: (i) preparing the Audit Report, Interim Audit Report, or Interim Designated Share Audit Report; and (ii) calculating Total Benefits, Total Salaries and Benefits, and Projected Benefits), the amount to be included with respect to the benefits shall be the amount of the Equity Proceeds Funding Pool and not the amount of the aggregate after-tax amounts contributed to the Non-Qualified Plan.
  \end{enumerate}
\item
  \textbf{Additional Post-Career Income Benefits Costs.} The NBA Teams shall pay all costs, including, without limitation, the cost of Professional Fees and other administrative services provided to the Post-Career Income Plan (but excluding the cost of contributions made to the Post-Career Income Plan) in connection with the operation and administration of the Post-Career Income Plan (or any alternative arrangement pursuant to Sections 4(b)(5) and (c)).
\end{enumerate}

\hypertarget{labor-management-cooperation-and-education-trust.}{%
\section{Labor-Management Cooperation and Education Trust.}\label{labor-management-cooperation-and-education-trust.}}

\begin{enumerate}
\def\labelenumi{(\alph{enumi})}
\tightlist
\item
  Except as set forth below in this Section 5, as of the effective date of this Agreement, and continuing until the expiration or termination of this Agreement, the National Basketball Players Association/National Basketball Association Labor-Management Cooperation and Education Trust (the ``Education Trust'') shall continue to be jointly operated and administered by the NBA and the Players Association in accordance with the provisions of the Agreement and Declaration of Trust Establishing the National Basketball Players Association/National Basketball Association Labor-Management Cooperation and Education Trust, as restated effective December 1, 2014, and as amended from time to time (the ``Education Trust Agreement''). It is intended by the NBA and the Players Association that, at all times, the Education Trust shall comply with the provisions of Section 302(c)(9) of the Labor Management Relations Act of 1947, as amended, and shall qualify as an exempt organization under the provisions of Sections 501(c)(5) or 501(c)(3) of the Code.
\item
  The Education Trust shall continue to be operated and administered for the purpose of establishing and providing (1) health education programs, and (2) education, career transition, and career counseling programs designed to assist the NBA, NBA Teams, and NBA players in solving problems of mutual concern not susceptible to resolution within the collective bargaining process and to enhance the involvement of NBA players in making decisions that affect their working lives. The NBA and the Players Association agree to provide jointly-run financial education programming, which shall be operated and administered by the Education Trust, subject to the programming being structured to qualify as a permitted activity of an exempt organization under the provisions of Section 501(c)(5) of the Code. In the event that a jointly-run financial education program cannot be structured to qualify as a permitted activity of an exempt organization under the provisions of Section 501(c)(5) of the Code, the NBA and Players Association agree to meet and confer regarding the establishment of such program through a different vehicle than the Education Trust.
\item
  Alongside jointly-run financial education programs, the NBA and the Players Association shall each develop and implement such independent financial education programs as it deems appropriate to be offered on a voluntary basis in such forms and at such times as deemed appropriate by the NBA and the Players Association, respectively. The NBA and the Players Association agree to confer periodically to share details and thoughts on best practices with respect to such programs. The NBA and the Players Association shall be responsible for the costs of their respective independent programs and such costs shall be excluded for purposes of all calculations called for under this Agreement of, or relating to, Benefits (including, without limitation, for purposes of (1) preparing the Audit Report, Interim Audit Report, or Interim Designated Share Audit Report, and (2) calculating Total Benefits, Total Salaries and Benefits, and Projected Benefits).
\item
  The NBA and Players Association agree that, subject to the limitations set forth in this Section 5:

  \begin{enumerate}
  \def\labelenumii{(\arabic{enumii})}
  \tightlist
  \item
    The amount to be paid by the Teams to fund the education and career counseling programs to be operated and administered by the Education Trust for the 2023-24 Salary Cap Year shall be no greater than \$2,020,339;
  \item
    The amount to be paid by the Teams to fund the career transition program to be operated and administered by the Education Trust for the 2023-24 Salary Cap Year shall be no greater than \$893,397;
  \item
    The amount to be paid by the Teams to fund the health education programs (or any programs that, pursuant to Section 5(g) below, are substituted for the health education programs) to be operated and administered by the Education Trust for the 2023-24 Salary Cap Year shall be no greater than \$628,549;
  \item
    The maximum funding amount for each of the programs described in Sections 5(d)(1)-(3) above shall be increased by five percent (5\%) for each subsequent Salary Cap Year during the term of this Agreement after the 2023-24 Salary Cap Year;
  \item
    For each Salary Cap Year, the amount to be paid by the Teams to fund the jointly-run financial education programs to be operated and administered by the Education Trust, if any, shall be mutually agreed upon by the parties; and
  \item
    Payment of the amount necessary to fund the Education Trust in respect of each Salary Cap Year shall be made within thirty (30) days following the completion of the Audit Report for such Salary Cap Year.
  \end{enumerate}
\item
  The Education Trust shall be operated and administered in a manner that will result in all contributions by the Teams being fully deductible under the Code (and, where applicable, Canadian income tax laws) when paid. If any Team is disallowed a deduction (in whole or in part) for such contributions, and unless the NBA determines otherwise, the obligation to maintain the Education Trust and to make further contributions to the Education Trust shall immediately terminate; provided, however, that any such termination shall not impair the legally binding effect of any other provision of this Agreement, and shall not create any right (1) to unilaterally implement, during the term of this Agreement, any terms concerning the provision of education programs provided or to be provided by the Education Trust; (2) to lockout; or (3) to strike.
\item
  In the event of any termination pursuant to Section 5(e) above, the NBA and Players Association agree to bargain in good faith with respect to an alternative arrangement designed to provide the programs described in the Education Trust Agreement. Such alternative arrangement shall, to the extent permitted by applicable law, be funded by such monies as may then remain in the Education Trust and, if the monies remaining in the Education Trust may not lawfully be used for, or are insufficient for, such purpose, such alternative arrangement shall be funded, by the NBA Teams; provided, however, that the annual cost incurred by the Teams in connection with such alternative arrangement (as determined on an after-tax basis) shall not exceed the annual cost that such Teams would have incurred to fund the Education Trust commencing on the date of termination. Any such alternative arrangement shall be operated and administered in a manner that will result in all contributions by the Teams being fully deductible under the Code (and, where applicable, Canadian income tax laws) when paid; and, if funded by the Teams (and not out of existing monies remaining in the Education Trust), the costs of funding any alternative to the Education Trust shall be subject to the limitations set forth in this Agreement. If despite good faith negotiations, the NBA and the Players Association fail to agree with respect to an alternative arrangement as described above, such failure to agree shall not create any right (1) to unilaterally implement, during the term of this Agreement, any terms concerning the provision of programs provided or to be provided by the Education Trust; (2) to lockout; or (3) to strike.
\item
  Upon written notice delivered to the NBA at least six (6) months prior to the commencement of any Salary Cap Year, the Players Association may elect to terminate the programs currently provided by the Education Trust and substitute alternative programs; provided, however, that the NBA consents to such substitution, which such consent shall not be unreasonably withheld; and provided, further, that any new programs shall comply with the provisions of Section 302(c)(9) of the Labor Management Relations Act of 1947, as amended, and shall qualify as a permitted activity of an exempt organization under Section 501(c)(5) of the Code.
\end{enumerate}

\hypertarget{additional-player-benefits.}{%
\section{Additional Player Benefits.}\label{additional-player-benefits.}}

Except as set forth below, the NBA shall provide the following additional benefits:

\begin{enumerate}
\def\labelenumi{(\alph{enumi})}
\tightlist
\item
  Workers' compensation benefits in accordance with applicable statutes. Such benefits will be provided for players and Two-Way Players.
\item
  Funding for the annual Players Association High School Basketball Camp (or any substitute program mutually agreed upon by the parties) in the amount of \$1,595,792 for the 2023-24 Season, increasing by seven and one-half percent (7.5\%) per Season thereafter for the term of this Agreement.
\item
  A Player Playoff Pool for each Salary Cap Year in an amount equal to the greater of: (i) \$31,014,350 multiplied by a fraction, the numerator of which is BRI for the Salary Cap Year immediately preceding the then-current Salary Cap Year and the denominator of which is BRI for the 2021-22 Salary Cap Year, and (ii) the amount of the Player Playoff Pool for the immediately preceding Salary Cap Year.

  \begin{enumerate}
  \def\labelenumii{(\arabic{enumii})}
  \tightlist
  \item
    If, for a Salary Cap Year, the NBA increases the number of Teams participating in the playoffs above sixteen (16), then the Player Playoff Pool shall be calculated pursuant to Section 6(c) above and then increased by \$615,000 for each Team added above sixteen (16) Teams.
  \item
    Each year, the NBA will consult with the Players Association with respect to the method of allocation of the Player Playoff Pool.
  \item
    The players on a Team that receive amounts from the Player Playoff Pool in respect of a Salary Cap Year shall not be permitted to share with Team personnel amounts that, in the aggregate, exceed five percent (5\%) of the total amount received by the players on that Team, collectively, from the Player Playoff Pool in respect of such Salary Cap Year.
  \end{enumerate}
\item
  An In-Season Tournament Prize Pool for each Salary Cap Year in an amount equal to the total prize amounts paid to players in accordance with the following.

  \begin{enumerate}
  \def\labelenumii{(\arabic{enumii})}
  \tightlist
  \item
    For each Salary Cap Year, the prize amounts paid to players shall be as follows:

    \begin{enumerate}
    \def\labelenumiii{(\roman{enumiii})}
    \tightlist
    \item
      For the 2023-24 Salary Cap Year: (A) \$500,000 to each ``IST Player'' (defined below) on the Team that wins the IST Finals Game; (B) \$200,000 to each IST Player on the Team that loses the IST Finals Game; (C) \$100,000 to each IST Player on a Team that loses an IST Semifinals game; and (D) \$50,000 to each IST Player on a Team that loses an IST Quarterfinals game; and
    \item
      For each subsequent Salary Cap Year: (A) for each IST Player on the Team that wins the IST Finals Game, an amount equal to \$500,000 multiplied by the ``BRI Growth Factor'' (defined below) for such Salary Cap Year; (B) for each IST Player on the Team that loses the IST Finals Game, an amount equal to \$200,000 multiplied by the BRI Growth Factor for such Salary Cap Year; (C) for each IST Player on a Team that loses an IST Semifinals game, \$100,000 multiplied by the BRI Growth Factor for such Salary Cap Year; and (D) for each IST Player on a Team that loses an IST Quarterfinals game, \$50,000 multiplied by the BRI Growth Factor for such Salary Cap Year;
    \end{enumerate}

    provided, however, that for each IST Player, the applicable amount set forth in Section 6(d)(1)(i) or (ii) above shall be multiplied by a fraction, the numerator of which is the number of knockout stage games (i.e., IST Quarterfinals games, IST Semifinals games, and the IST Finals Game) for which the player was on the Team's Active or Inactive List and the denominator of which is the total number of knockout stage games played by the Team. For the purposes of the calculation described in this Section 6(d)(1), a knockout stage game for which an IST Player was on a Team's Active or Inactive List while under a Two-Way Contract or a 10-Day Contract shall count as one-half of a knockout stage game for which such player was on the Team's Active or Inactive List. For example, if a Two-Way Player is on the Active or Inactive List of the Team that wins the In-Season Tournament for all four (4) of the Team's knockout stage games during the 2023-24 Salary Cap Year, then such player will receive a prize amount equal to \$250,000 (i.e., \$500,000 (i.e., the amount set forth in Section 6(d)(1)(i)(A) above) multiplied by a fraction, the numerator of which is two (2) (i.e., one-half of a knockout stage game for each knockout stage game for which the Two-Way Player was on the Team's Active or Inactive List), and the denominator of which is four (4) (i.e., the total number of knockout stage games played by the Team)).
  \item
    For purposes of this Section 6(d), for each Salary Cap Year:

    \begin{enumerate}
    \def\labelenumiii{(\roman{enumiii})}
    \tightlist
    \item
      An ``IST Player'' is a player who is on a Team's Active or Inactive List for at least one (1) In-Season Tournament knockout stage game during such Salary Cap Year.
    \item
      The BRI Growth Factor for a Salary Cap Year is a fraction, the numerator of which is BRI for the immediately preceding Salary Cap Year and the denominator of which is BRI for the 2022-23 Salary Cap Year; provided, however, that the NBA and Players Association may agree to reduce the BRI Growth Factor for one (1) or more Salary Cap Years to a smaller fraction with value of no less than one (1).
    \end{enumerate}
  \end{enumerate}
\item
  The employer's portion of payroll taxes.
\item
  The Players Association's one-half share of the payment of fees and expenses to the Accountants (as defined in Article VII, Section 10(a) below) in connection with any audit conducted under this Agreement, and the Players Association's one-half share of the payment of fees and expenses payable with respect to the TV Expert (as defined in Article VII, Section 1(a)(7)(ii) below) and any expert selected in accordance with Article VII, Section 1(a)(7)(i).
\item
  The Players Association's share of the costs of the Anti-Drug Program as provided for by Article XXXIII.
\item
  The sum of the Compensation paid to each player with three (3) or more Years of Service who signs a one-year, 10-Day, or Rest-of-Season Contract for the Minimum Player Salary during a Season, less, for each such player, the Minimum Player Salary for a player with two (2) Years of Service. The Compensation paid to any such player shall be paid by the player's Team pursuant to the terms of such player's Uniform Player Contract, and then reimbursed to the Team out of a League-wide fund created and maintained by the NBA. Such reimbursement shall be made at the conclusion of the Season covered by the Contract.
\item
  One-half of the annual funding of \$1.5 million for the NBA Players Legacy Fund that is provided jointly by the NBA and the Players Association.
\item
  Any additional contributions that may be required to be made to the Pension Plan because of any new law, change, or amendment made to ERISA, the Code, and/or any other applicable law or to any regulations (whether final, temporary, or proposed), rulings or formal guidance issued thereunder that is effective for a Plan Year that first begins after the effective date of this Agreement.
\item
  Costs of player attendance at the partner forums as set forth in the following sentence. For the purposes of enhancing career exposure and professional development, the NBA agrees to permit current and former players to attend partner forums held from time-to-time with NBA business partners, subject to advance notice by the players and there being a reasonable number of player attendees such that the primary purpose of the forums (i.e., to facilitate interaction between the NBA and business partners) will be maintained. To the extent reasonably practicable, the NBA agrees to provide the Players Association with advance notice of partner forums that it is aware of.
\item
  The Players Association's one-half share of the costs of: (1) the Fitness-to-Play Panels as provided for by Article XXII, Section 11; (2) the player care survey as provided for by Article XXII, Section 12; and (3) the Wearables Committee, including, without limitation, the costs of retaining experts, as provided for by Article XXII, Section 13.
\item
  Costs described in Sections 1(i), 2(g), 3(e), and 4(e) above.
\item
  Costs attributable to the operation and administration of the Education Trust, including, without limitation, the cost of Professional Fees.
\item
  The cost of the Professional Fees and vendor fees in connection with the design, implementation, operation, and maintenance of an online benefits portal, the content, vendor, and other details of which shall be mutually agreed upon by the parties, to provide players access to their benefits information and benefit plan accounts and make transactions related to their benefits, as applicable.
\item
  The cost of premiums to purchase fiduciary liability insurance coverage applicable to the 401(k) Plan, the Pension Plan, the Post-Career Income Plan, and the Health and Welfare Benefit Plan.
\end{enumerate}

\hypertarget{canadian-residents.}{%
\section{Canadian Residents.}\label{canadian-residents.}}

As of the effective date of this Agreement, and continuing until the expiration or termination of this Agreement, Toronto shall continue to provide the following benefits to Canadian Residents:

\begin{enumerate}
\def\labelenumi{(\alph{enumi})}
\tightlist
\item
  \textbf{Definitions.} All capitalized terms used in this Section 7 not otherwise defined in this Agreement shall have the meanings set forth below:

  \begin{enumerate}
  \def\labelenumii{(\arabic{enumii})}
  \tightlist
  \item
    ``Eligible Canadian Resident'' shall mean a Canadian Resident who would be eligible to participate in the Pension Plan, the Post-Career Income Plan, the HRA Benefit, and/or the 401(k) Plan, in each case, but for the fact that he is a Canadian Resident.
  \item
    ``EHT'' shall mean the Ontario Employer Health Tax.
  \item
    ``Gross Amount'' for a Season shall mean, as applicable, the sum of:

    \begin{enumerate}
    \def\labelenumiii{(\roman{enumiii})}
    \tightlist
    \item
      if the player is an Eligible Canadian Resident in respect of the Pension Plan, the annual accrual cost that Toronto would have incurred under the Pension Plan for such Eligible Canadian Resident for such Season but for the fact that he was Canadian Resident; and
    \item
      if the player is an Eligible Canadian Resident in respect of the Post-Career Income Plan, the amount of the per-player Allocated Share for such Season; and
    \item
      if the player is an Eligible Canadian Resident in respect of the HRA Benefit, the amount of the contribution to fund the HRA Benefit for such Season that such player would be entitled to under Section 3(a)(1) but for the fact that he is a Canadian Tax Resident; provided that, for the avoidance of doubt, the Gross Amount(s) previously allocated to such player in lieu of the HRA Benefit for years in which he was an Eligible Canadian Resident in respect of the HRA Benefit shall be applied against the \$150,000 limit applicable to aggregate HRA Benefit contributions per player under the Health and Welfare Benefit Plan; and
    \item
      if the player is an Eligible Canadian Resident in respect of the 401(k) Plan, the amount of the Matching Contribution (as defined in the 401(k) Plan) for such season assuming that the Eligible Canadian Resident had made the maximum player deferral permitted under the 401(k) Plan for such Season.
    \end{enumerate}
  \item
    ``Adjusted Gross Amount'' shall mean the adjusted gross amount that is equal to the Eligible Canadian Resident's Gross Amount less the amount of EHT on such adjusted gross amount.
  \end{enumerate}
\item
  \textbf{Cash Payment.} For each Season during the term of this Agreement, each Eligible Canadian Resident shall be entitled to a single sum payment subject to the following terms and conditions:

  \begin{enumerate}
  \def\labelenumii{(\arabic{enumii})}
  \tightlist
  \item
    The amount of the payment shall equal the Eligible Canadian Resident's Adjusted Gross Amount in respect of such Season, less all amounts required to be withheld by any governmental authority, and less the employer's share of payroll taxes for the Eligible Canadian Resident (the ``Cash Payment'').
  \item
    The Cash Payment shall be paid in Canadian dollars to the Eligible Canadian Resident by no later than the December 31 immediately following the end of the Season to which the payment relates. For purposes of calculating the Cash Payment, the Adjusted Gross Amount shall be calculated in U.S. dollars and then converted to Canadian dollars using the daily exchange rate quoted by the Bank of Canada for converting U.S. dollars into Canadian dollars on the first day of the month in which the Cash Payment is made, or if there is no such U.S. dollar to Canadian dollar exchange rate quoted for that date, the closest preceding date on which such exchange rate is quoted by the Bank of Canada.
  \end{enumerate}
\item
  \textbf{Funding of Gross Amount.}

  \begin{enumerate}
  \def\labelenumii{(\arabic{enumii})}
  \tightlist
  \item
    The cost of the portion of the Gross Amount attributable to the Pension Plan, the 401(k) Plan, and the HRA Benefit shall be paid by Toronto and the cost of the portion of the Gross Amount attributable to the Post-Career Income Plan shall be funded from the Team Funding Pool.
  \item
    The NBA Teams shall pay all costs incurred in connection with the determination and implementation of this Section 7, including, without limitation, the cost of Professional Fees (but excluding the cost of the Gross Amount).
  \end{enumerate}
\end{enumerate}

\hypertarget{projected-benefits.}{%
\section{Projected Benefits.}\label{projected-benefits.}}

\begin{enumerate}
\def\labelenumi{(\alph{enumi})}
\tightlist
\item
  For purposes of computing the Salary Cap in accordance with Article VII, ``Projected Benefits'' shall mean the projected amounts, as estimated by the NBA in good faith, to be paid or accrued by the NBA or the Teams, other than Expansion Teams during their first two Salary Cap Years, for the upcoming Salary Cap Year with respect to the benefits to be provided for such Salary Cap Year. In the event that the amount of any benefit for the upcoming Salary Cap Year is not reasonably calculable, then, for purposes of computing Projected Benefits, such amount shall be projected to be one hundred four and one-half percent (104.5\%) of the amount attributable to the same benefit for the prior Salary Cap Year.
\item
  For purposes of computing Projected Benefits, the amount to be included with respect to players with three (3) or more Years of Service who receive the Minimum Player Salary shall be the same amount included in Benefits with respect to such players for the immediately preceding Season.
\item
  For purposes of computing Projected Benefits with respect to a Salary Cap Year, the amount to be included with respect to the Additional Benefit Amount shall be one percent (1\%) of Projected BRI for such Salary Cap Year.
\end{enumerate}

\hypertarget{benefit-exclusion-amount.}{%
\section{Benefit Exclusion Amount.}\label{benefit-exclusion-amount.}}

\begin{enumerate}
\def\labelenumi{(\alph{enumi})}
\tightlist
\item
  An amount equal to the Benefit Exclusion Amount (defined below) shall be (1) paid by the Teams and (2) excluded for purposes of all calculations called for under this Agreement of, or relating to, Benefits (including, without limitation, for purposes of: (i) preparing the Audit Report, Interim Audit Report, or Interim Designated Share Audit Report; and (ii) calculating Total Benefits, Total Salaries and Benefits, and Projected Benefits).
\item
  The ``Benefit Exclusion Amount,'' for each Salary Cap Year, shall mean the sum of:

  \begin{enumerate}
  \def\labelenumii{(\arabic{enumii})}
  \tightlist
  \item
    The ``Pension Exclusion Amount,'' which shall equal fifty percent (50\%) of the portion of the increase in the amount of the actuarially-determined annual contributions to be made to the Pension Plan to fund the portion of the liabilities for the 2017-18 Benefit Increase (defined below) that is attributable to the Current Retiree Group (defined below), as determined by the actuaries of the Pension Plan. The ``2017-18 Benefit Increase'' means the increase in the Monthly Benefit from \$572.13 to \$812.50; and
  \item
    Fifty percent (50\%) of the portion of the increase in the amount of the actuarially-determined annual contributions to be made to the Pension Plan, and fifty percent (50\%) of the portion of the increase in the cost under the Pre-1965 Players' Excess Benefit Plan, to fund the 2017-18 Pre-1965 Benefit Increase (defined below). The ``2017-18 Pre-1965 Benefit Increase'' shall mean the increase in the Normal Retirement Benefit payable to a Pre-1965 Player and the ``A portion'' of the Retirement Benefit payable to a Pre-1965 Retiree from \$300 to \$400 per month for each Year of Pre-1965 Credited Service or Year of Eligible Pre-1965 Retiree Service, respectively; and
  \item
    Fifty percent (50\%) of the portion of the costs (including, without limitation, the cost of Professional Fees) that were approved by both an NBA designee and a Players Association designee as having been properly incurred in connection with the operation and administration of the Retiree Medical Plan (``Administrative Costs''), but only to the extent that such costs are attributable to the Current Retiree Group. The portion of the Administrative Costs for a Salary Cap Year that is attributable to the Current Retiree Group shall be determined by multiplying the total Administrative Costs for the Salary Cap Year by the ``Allocation Percentage'' (defined below) for such Salary Cap Year. The ``Allocation Percentage,'' for a Salary Cap Year, means the fraction, when expressed as a percentage, the numerator of which is the number of players in the Current Retiree Group who are enrolled in the Retiree Medical Plan on the day that is sixty (60) days prior to the last day of such Salary Cap Year, and the denominator of which is the total number of players who are enrolled in the Retiree Medical Plan on such date; and
  \item
    Fifty percent (50\%) of the portion of the contributions made by the Teams to the Health and Welfare Benefit Trust during the Salary Cap Year to fund the premium costs of the Retiree Medical Plan attributable to the Current Retiree Group. Such portion shall be determined by multiplying (i) the total amount contributed by the Teams to the Health and Welfare Benefit Trust during the Salary Cap Year to fund the premium costs of the Retiree Medical Plan by (ii) a fraction, when expressed as a percentage, (A) the numerator of which is the total premium costs paid by the Health and Welfare Benefit Trust to the insurer of the Retiree Medical Plan (excluding the participant share of premium contributions) that are attributable to the Current Retiree Group, and (B) the denominator of which is the total premium costs paid by the Health and Welfare Benefit Trust to the insurer of the Retiree Medical Plan (excluding the participant share of premium contributions) that are attributable to all players who are enrolled in the Retiree Medical Plan. Such premium costs shall be calculated based on the schedules provided by the insurer of the Retiree Medical Plan that set forth the monthly premium payments for each eligible retiree or any eligible dependent based on the applicable coverage level elected; and
  \item
    Fifty percent (50\%) of the portion of reimbursable tuition reimbursement and career transition benefits for players under the Health and Welfare Benefit Plan (as described in Section 3(a)(6) above) that is attributable to the Current Retiree Group; and
  \item
    In respect of the Two-Way Player 401(k) Plan benefits, an amount equal to \$40,000 multiplied by a fraction, the numerator of which is the Salary Cap for such Salary Cap Year and the denominator of which is the Salary Cap for the 2023-24 Salary Cap Year; provided, however, that the foregoing amount shall be decreased by thirty-three and one-third percent (33-1/3\%) for any Salary Cap Year following the Players Association's exercise of the PA Third Two-Way Option set forth in Article XXIX, Section 5(b); and
  \item
    The amount that is the difference between (i) the portion of the premium costs paid to the applicable insurer(s) to provide the medical, prescription drug, dental, and vision insurance benefits to Two-Way Players as described in Section 3(a)(2)(iii), and (ii) the sum of the ``Applicable Portion'' (as defined below) for all Two-Way Players. The Applicable Portion for each such player shall be calculated by multiplying: (A) the total monthly premium payment for a Two-Way Player who elected that coverage level under the relevant Two-Way Players' insurance policy that Season; by (B) a fraction, expressed as a percentage of a premium payment, the numerator of which is the portion of the total monthly premium payment contributed by an NBAGL player for the same coverage level under the corresponding insurance policy covering NBAGL players during the NBAGL regular season occurring within the Salary Cap Year immediately preceding that Season, and the denominator of which is the total monthly premium payment for that NBAGL player for the same coverage level under the corresponding insurance policy covering NBAGL players during the NBAGL regular season occurring within the Salary Cap Year immediately preceding that Season; and
  \item
    The portion of the premium costs paid to the applicable insurer(s) to provide the life and accidental death and dismemberment insurance benefits to Two-Way Players as described in Section 3(a)(2)(i) above (excluding the cost of increasing such life insurance and accidental death and dismemberment benefits as described in Section 3(a)(2)(i) above); and
  \item
    The amount equal to: (i) the premium costs under the workers' compensation policy covering NBAGL players in the Salary Cap Year, divided by the average number per month of participants covered under such policy during the NBAGL regular season occurring within such Salary Cap Year; multiplied by (ii) the average number per month of Two-Way Players (excluding, for any month's calculation, Two-Way Players who were signed or converted to Standard NBA Contracts in that or a prior month) during the Regular Season; and
  \item
    In respect of the employer's share of payroll taxes for Two-Way Players, an amount equal to \$201,000 multiplied by a fraction, the numerator of which is the Salary Cap for such Salary Cap Year and the denominator of which is the Salary Cap for the 2023-24 Salary Cap Year; provided, however, that the foregoing amount shall be decreased by thirty-three and one-third percent (33-1/3\%) for any Salary Cap Year following the Players Association's exercise of the PA Third Two-Way Option set forth in Article XXIX, Section 5(b); and
  \item
    Fifty percent (50\%) of the cost of Professional Fees that were approved by both an NBA designee and a Players Association designee as having been properly incurred on or after the effective date of this Agreement by the Pension Plan's and the Toronto Plan's third-party administrator in connection with the administration of the Pension Plan and the Toronto Plan; and
  \item
    Fifty percent (50\%) of the portion of the cost of Professional Fees that were approved by both an NBA designee and a Players Association designee as having been properly incurred on or after the effective date of this Agreement by the Health and Welfare Benefit Plan's third-party administrator that is attributable to the provision of insured benefits to current players as described in Section 3(a)(2) above; and
  \item
    The portion of the contributions made by the Teams to the Education Trust during the Salary Cap Year to fund the jointly-run financial education programs or, if such programs are not operated and administered by the Education Trust, fifty percent (50\%) of the portion of the costs that were approved by both an NBA designee and a Players Association designee as having been properly incurred in connection with providing such programs.
  \end{enumerate}
\item
  \textbf{Benefit Reduction in Respect of Two-Way Salaries.} Each Salary Cap Year, a portion of the Compensation paid to Two-Way Players equal to the ``Two-Way Salary Exclusion Amount'' shall be deducted from all calculations called for under this Agreement of, or relating to, Benefits (including, without limitation, for purposes of: (x) preparing the Audit Report, Interim Audit Report, or Interim Designated Share Audit Report, and (y) calculating Total Benefits, Total Salaries and Benefits, and Projected Benefits). The Two-Way Salary Exclusion Amount shall be an amount equal to \$5,250,000 multiplied by a fraction, the numerator of which is the Salary Cap for such Salary Cap Year, and the denominator of which is the Salary Cap for the 2023-24 Salary Cap Year; provided, however, that the foregoing amount shall be decreased by thirty-three and one-third percent (33-1/3\%) for any Salary Cap Year following the Players Association's exercise of the PA Third Two-Way Option set forth in Article XXIX, Section 5(b).
\item
  \textbf{Current Retiree Group and Current Player Group.}

  \begin{enumerate}
  \def\labelenumii{(\arabic{enumii})}
  \tightlist
  \item
    The ``Current Retiree Group'' shall mean those former players whose last day on an NBA Active List or Inactive List during a Regular Season occurred before the 2016-17 Season.
  \item
    The ``Current Player Group'' shall mean those players whose last day on an NBA Active List or Inactive List during a Regular Season will occur during or after the 2016-17 Season.
  \item
    If a player who is included in the Current Retiree Group for one or more Salary Cap Years returns to an NBA Active List or Inactive List and thereby moves to the Current Player Group in a later Salary Cap Year, the Benefit Exclusion Amount for the Salary Cap Year during which he returns to an NBA Active List or Inactive List shall be reduced by the amount of the portion of the Benefit Exclusion Amount for the prior Salary Cap Year(s) that is attributable to such player.
  \end{enumerate}
\item
  If the NBA and Players Association provide an alternative arrangement to any benefit referenced in Sections 9(b)(1) through 9(b)(8) above, the amount to be included in the calculation of the Benefit Exclusion Amount with regard to that alternative arrangement shall not exceed the amount referenced in the applicable part of Section 9(b) with regard to the benefit being replaced by that alternative arrangement for the most recent Salary Cap Year before such benefit was replaced.
\item
  For the avoidance of doubt, other than the Benefit Exclusion Amount and the Two-Way Salary Exclusion Amount, all amounts paid or to be paid during any Salary Cap Year by the NBA or the NBA Teams for or relating to the benefits described in this Article IV shall be included for purposes of all calculations called for under this Agreement of, or relating to, Benefits (including, without limitation, for purposes of (1) preparing the Audit Report, Interim Audit Report, or Interim Designated Share Audit Report, and (2) calculating Total Benefits, Total Salaries and Benefits, and Projected Benefits).
\end{enumerate}

\hypertarget{compensation-and-expenses-in-connection-with-military-duty}{%
\chapter{COMPENSATION AND EXPENSES IN CONNECTION WITH MILITARY DUTY}\label{compensation-and-expenses-in-connection-with-military-duty}}

\hypertarget{salary.}{%
\section{Salary.}\label{salary.}}

A player drafted into military service during the Season, or a player serving on active duty with a reserve unit during the Season, shall be compensated for so long as the player remains on the Active or Inactive List of the Team in such amount as may be negotiated between the player and the Team by which he is employed, subject to the provisions of this Agreement.

\hypertarget{travel-expenses.}{%
\section{Travel Expenses.}\label{travel-expenses.}}

\begin{enumerate}
\def\labelenumi{(\alph{enumi})}
\tightlist
\item
  A player serving on military weekend duty with a reserve unit during the Season shall be entitled to reimbursement for any net out-of-pocket expenses incurred by such player in traveling to and from his place of duty to enable him to join his Team for purposes of participating in a Regular Season game.
\item
  In the event that the Player Contract of a player who is required to serve on military weekend duty with a reserve unit is assigned to another Team, the player shall be entitled to reimbursement for any out-of-pocket expenses incurred by such player in traveling during the off-season to and from his home and his place of military weekend duty with a reserve unit; provided that (i) the player makes reasonable efforts to change his reserve unit location to one located reasonably close to his home, and (ii) such obligation to reimburse the player shall cease six (6) months from the date that such player's Contract is assigned.
\end{enumerate}

\hypertarget{player-conduct}{%
\chapter{PLAYER CONDUCT}\label{player-conduct}}

\hypertarget{games.}{%
\section{Games.}\label{games.}}

\begin{enumerate}
\def\labelenumi{(\alph{enumi})}
\tightlist
\item
  In addition to any other rights a Team or the NBA may have by contract (including but not limited to the rights set forth in Paragraphs 9 and 16 of the Uniform Player Contract) or by law:

  \begin{enumerate}
  \def\labelenumii{(\roman{enumii})}
  \tightlist
  \item
    When a player (A) fails or refuses, without proper and reasonable cause or excuse, to render the services required by a Player Contract or this Agreement, or (B) is suspended by his Team or the NBA for failing or refusing, without proper and reasonable cause or excuse, to render the services required by a Player Contract or this Agreement, the Current Base Compensation payable to the player for the year of the Contract during which such failure or refusal and/or suspension occurs shall be reduced by 1/91.6th of the player's Base Compensation for each missed Exhibition, Regular Season, Play-In, or playoff game; and
  \item
    When a player is, for proper cause other than the player's failure or refusal to render the services required by a Player Contract or this Agreement, suspended by his Team or the NBA in accordance with the terms of such Contract or this Agreement, the Current Base Compensation payable to the player for the year of the Contract during which such suspension occurs shall be reduced by (i) 1/145th of the player's Base Compensation for each missed Exhibition, Regular Season, Play-In, or playoff game for any suspension of fewer than twenty (20) games and (ii) 1/110th of the player's Base Compensation for each missed Exhibition, Regular Season, Play-In, or playoff game for any suspension of twenty (20) games or more (including any indefinite suspension that persists for twenty (20) games or more or consecutive suspensions for continuing acts or conduct that persist for twenty (20) games or more).
  \end{enumerate}
\item
  Notwithstanding Section 1(a)(ii) above, for the first game in a Season for which a player is suspended by the NBA, if such suspension is for conduct on the playing court (as that term is defined in Article XXXI, Section 9(c)) and is a one-game suspension, the Current Base Compensation payable to the player for the year of the Contract during which such suspension occurs shall be reduced by an amount equal to the player's Current Base Compensation for such Season multiplied by a fraction, the numerator of which is one (1) and the denominator of which is the number of days in the Regular Season. For clarity, for any other game for which such player is suspended by his Team or the NBA during such Season, the Current Base Compensation payable to the player for the year of the Contract during which such suspension occurs shall be reduced in accordance with Section 1(a) above.
\item
  In the event that, at the start of a Regular Season, (x) a player is a Free Agent who has games remaining to be served on a suspension that was previously imposed on him by the NBA either when he was under a Contract with a Team or when he was a Free Agent, and (y) such player, but for the remainder of the suspension to be served, is otherwise eligible and able to play, then:

  \begin{enumerate}
  \def\labelenumii{(\roman{enumii})}
  \tightlist
  \item
    The player's suspension shall be deemed to have been served as of the day following the day on which the Team to which he was under contract when the suspension was imposed (or, if he was not under contract when the suspension was imposed, the last Team to which he was under contract prior to the suspension being imposed) has played a number of games in such Regular Season equal to one and one-half (1.5) times the number of games that remained to be served on the term of the suspension as of the first day of such Regular Season (rounded up to the nearest whole number); and
  \item
    If the player subsequently signs one or more Player Contracts, the Current Base Compensation payable to the player for such Season (or, if he does not sign a Player Contract during such Regular Season, the first subsequent Season thereafter for which he signs a Player Contract) under one or more Contracts shall be reduced in accordance with Section 1(a) above for the number of games that remained to be served on the term of the suspension as of the first day of such Regular Season.
  \end{enumerate}

  For clarity, if a player is a Free Agent on the first day of such Regular Season and subsequently signs a Player Contract before his suspension has been deemed to have been served pursuant to Section 1(c)(i) above, the number of games of the suspension that will be deemed to have been served as of the date he signs such Player Contract shall equal two-thirds (2/3) of the number of games played by his prior Team in such Regular Season as of the date he signs such Player Contract (rounded to the nearest whole number).
\end{enumerate}

\hypertarget{practices.}{%
\section{Practices.}\label{practices.}}

\begin{enumerate}
\def\labelenumi{(\alph{enumi})}
\tightlist
\item
  When a player, without proper and reasonable excuse, fails to attend a practice session scheduled by his Team, he shall be subject to the following discipline: (i) for the first missed practice during a Season -- \$2,500; (ii) for the second missed practice during such Season -- \$5,000; (iii) for the third missed practice during such Season -- \$7,500; and (iv) for the fourth (or any additional) missed practice during such Season -- such discipline as is reasonable under the circumstances.
\item
  Notwithstanding Section 2(a) above, when a player, without proper and reasonable excuse, refuses or intentionally fails to attend any practice session scheduled by his Team, he shall be subject to such discipline as is reasonable under the circumstances.
\end{enumerate}

\hypertarget{promotional-appearances.}{%
\section{Promotional Appearances.}\label{promotional-appearances.}}

When a player, without proper and reasonable excuse, fails or refuses to attend a promotional appearance required by and in accordance with Article II, Section 8 and Paragraph 13(d) of the Uniform Player Contract, he shall be fined \$20,000.

\hypertarget{mandatory-programs.}{%
\section{Mandatory Programs.}\label{mandatory-programs.}}

\begin{enumerate}
\def\labelenumi{(\alph{enumi})}
\tightlist
\item
  NBA players shall be required to attend and participate in educational and life skills programs designated as ``mandatory programs'' by the NBA and the Players Association. Such ``mandatory programs,'' which shall be jointly administered by the NBA and the Players Association, shall include a Rookie Transition Program (for rookies only), Team Awareness Meetings (which shall cover, among other things, substance abuse awareness, HIV awareness, gambling awareness, healthy relationships, mental health and wellness programming, and recommendations and educational materials regarding the health benefits of vaccinations recommended by the Centers for Disease Control and Prevention (CDC)), and such other programs as the NBA and the Players Association shall jointly designate as mandatory.
\item
  When a player, without proper and reasonable excuse, fails or refuses to attend a ``mandatory program,'' he shall be fined \$20,000 by the NBA; provided, however, that if the player misses the Rookie Transition Program, he shall be suspended for five (5) games.
\item
  Each year, the NBA and Players Association shall work together to (i) identify players who did not attend a Rookie Transition Program (for rookies only) or a Team Awareness Meeting covering information relating to the Anti-Drug Program (e.g., because the player was signed to a Contract after such Team Awareness Meeting took place), and (ii) as soon as practicable, provide any such player with educational materials regarding the Anti-Drug Program.
\end{enumerate}

\hypertarget{media-training-business-of-basketball-anti-gambling-training-and-system-rules-training.}{%
\section{Media Training, Business of Basketball, Anti-Gambling Training, and System Rules Training.}\label{media-training-business-of-basketball-anti-gambling-training-and-system-rules-training.}}

\begin{enumerate}
\def\labelenumi{(\alph{enumi})}
\tightlist
\item
  All players shall be required each Season to attend and participate in one (1) media training session conducted by their Team and/or the NBA. If a player, without proper and reasonable excuse, fails or refuses to attend a media training session, he shall be fined \$20,000.
\item
  All players shall be required each Season to attend and participate in one (1) ``business of basketball'' program conducted by the Team and/or the NBA. If a player, without proper and reasonable excuse, fails or refuses to attend such program, he shall be fined \$5,000. Each Team's Governor shall attend his or her Team's annual ``business of basketball'' program.
\item
  All players shall be required each Season to attend and participate in one (1) anti-gambling training session conducted by their Team and/or the NBA. If a player, without proper and reasonable excuse, fails or refuses to attend an anti-gambling training session, he shall be fined \$100,000. Each year, the NBA and Players Association shall work together to (i) identify players who did not attend a Rookie Transition Program (for rookies only) or the anti-gambling training session, and (ii) as soon as practicable, provide any such player with educational materials with information from such training session. In addition, each year the NBA shall work with the NBAGL to ensure that players in the NBAGL are provided with the same anti-gambling training that is provided to NBA players.
\item
  Each year, the NBA and Players Association will jointly make available to players training on System rules (e.g., online or by videoconference).
\end{enumerate}

\hypertarget{charitable-contributions.}{%
\section{Charitable Contributions.}\label{charitable-contributions.}}

\begin{enumerate}
\def\labelenumi{(\alph{enumi})}
\tightlist
\item
  In the event that (i) a fine or suspension is imposed on a player, (ii) such fine or suspension-related Compensation amount is collected by the League, and (iii) the fine or suspension is not grieved pursuant to Article XXXI, then the NBA shall remit fifty percent (50\%) of the amount collected to the National Basketball Players Association Foundation (the ``NBPA Foundation'') or such other charitable organization selected by the Players Association that qualifies for treatment under Section 501(c)(3) of the Internal Revenue Code of 1986, as now in effect or as it may hereafter be amended (a ``Section 501(c)(3) Organization''), and that is approved by the NBA (which approval shall not be unreasonably withheld) (both hereinafter, the ``NBPA-Selected Charitable Organization''); provided, however, that any contributions made by the NBPA-Selected Charitable Organization to a player charitable foundation cannot be intended to reimburse the player for the financial impact of a fine or suspension. The NBA shall remit the remaining fifty percent (50\%) of the amount collected to a Section 501(c)(3) Organization selected by the NBA and approved by the Players Association, which approval shall not be unreasonably withheld. For purposes of this Section 6(a), and with respect to any suspension imposed on a player by the NBA of five (5) games or more, the NBA shall be required to collect a suspension-related Compensation amount equal to at least five (5) games of such suspension.
\item
  The remittances made by the NBA pursuant to this Section 6 shall be made annually, ninety (90) days following the Accountants' (as defined in Article VII, Section 10(a)) submission to the NBA and the Players Association of a final Audit Report or an Interim Designated Share Audit Report (as defined in Article VII, Section 10(a)(1)) for the Salary Cap Year covering the Season during which the fines and suspension-related Compensation amounts are collected by the NBA.
\item
  If a timely Grievance is filed under Article XXXI challenging a fine or suspension of the kind designated in Section 6(a) above, and, following the disposition of the Grievance, the Grievance Arbitrator determines that all or part of the fine or suspension-related amount (plus any accrued interest thereon) is payable by the player to the League, then the League shall remit the amount collected by the League (plus any interest) in accordance with the provisions of Sections 6(a) and (b) above.
\end{enumerate}

\hypertarget{unlawful-violence.}{%
\section{Unlawful Violence.}\label{unlawful-violence.}}

When a player is convicted of (including by a plea of guilty, no contest, or nolo contendere to) a violent felony, he shall immediately be suspended by the NBA for a minimum of ten (10) games.

\hypertarget{counseling-for-violent-misconduct.}{%
\section{Counseling for Violent Misconduct.}\label{counseling-for-violent-misconduct.}}

\begin{enumerate}
\def\labelenumi{(\alph{enumi})}
\tightlist
\item
  In addition to any other rights a Team or the NBA may have by contract or law, when the NBA and the Players Association agree that there is reasonable cause to believe that a player has engaged in any type of off-court violent conduct, the player will (if the NBA and the Players Association so agree) be required to undergo a clinical evaluation by a neutral expert and, if deemed necessary by such expert, appropriate counseling, with such evaluation and counseling program to be developed and supervised by the NBA and the Players Association, unless the player has engaged in acts covered by the Joint NBA/NBPA Policy on Domestic Violence, Sexual Assault, and Child Abuse, in which case the terms of that Policy shall apply. For purposes of this paragraph, ``violent conduct'' shall include, but not be limited to, any conduct involving the use or threat of physical violence or the use of, or threat to use, a deadly weapon, any conduct which could be categorized as a ``hate crime,'' and any conduct involving dog fighting or animal cruelty.
\item
  Any player who is convicted of (including by a plea of guilty, no contest, or nolo contendere to) a crime involving violent conduct shall be required to attend at least five (5) counseling sessions with a therapist or counselor jointly selected by the NBA and the Players Association, unless the player has engaged in acts covered by the Joint NBA/NBPA Policy on Domestic Violence, Sexual Assault, and Child Abuse, in which case the terms of that Policy shall apply. These sessions shall be in addition to any discipline imposed on the player by the NBA for the conduct underlying his conviction. The therapist or counselor who is jointly selected by the NBA and the Players Association shall determine the total number of counseling sessions to be attended by the player; however, in no event shall a player be required to attend more than ten (10) sessions.
\item
  Any player who, after being notified in writing by the NBA that he is required to undergo the clinical evaluation and/or counseling program authorized by Section 8(a) or 8(b) above, refuses or fails, without a reasonable explanation, to attend or participate in such evaluation and counseling program within seventy-two (72) hours following such notice, shall be fined by the NBA in the amount of \$10,000 for each day following such seventy-two (72) hours that the player refuses or fails to participate in such program.
\end{enumerate}

\hypertarget{firearms-and-other-weapons.}{%
\section{Firearms and Other Weapons.}\label{firearms-and-other-weapons.}}

\begin{enumerate}
\def\labelenumi{(\alph{enumi})}
\tightlist
\item
  Whenever a player is physically present at a facility or venue owned, operated, or being used by a Team, the NBA, or any League-related entity, and whenever a player is traveling on any NBA-related business, whether on behalf of the player's Team, the NBA, or any League-related entity, such player shall not possess a firearm of any kind or any other deadly weapon. For purposes of the foregoing, ``a facility or venue'' includes, but is not limited to: an arena; a practice facility; a Team or League office or facility; a facility or venue used for an NBA event (such as an In-Season Tournament, All-Star, or NBA playoff venue); and the site of a promotional or charitable appearance.
\item
  At the commencement of each Season, and if the player owns or possesses any firearm, the player will provide the Team with proof that the player possesses a license or registration as required by law for any such firearm. Each player is also required to provide the Team with proof of any modifications or additions made to this information during the Season.
\item
  Any violation of Section 9(a) or Section 9(b) above shall be considered conduct prejudicial to the NBA under Article 35(d) of the NBA Constitution and By-Laws, and shall therefore subject the player to discipline by the NBA in accordance with such Article.
\end{enumerate}

\hypertarget{one-penalty.}{%
\section{One Penalty.}\label{one-penalty.}}

\begin{enumerate}
\def\labelenumi{(\alph{enumi})}
\tightlist
\item
  The NBA and a Team shall not discipline a player for the same act or conduct. The NBA's disciplinary action will preclude or supersede disciplinary action by any Team for the same act or conduct.
\item
  When the NBA becomes aware of any potential or actual disciplinary action which may be or has been imposed by a Team for a player's act or conduct, the NBA may, within forty-eight (48) hours, prohibit the discipline from being imposed or rescind the discipline that has been imposed, as applicable. If the NBA prohibits or rescinds the discipline, only the NBA shall thereafter be permitted to impose discipline on the player for that act or conduct. If the NBA does not prohibit or rescind the Team's discipline, the Team may impose its proposed discipline or the Team's discipline will remain in effect, as applicable, and, if the Team's discipline becomes effective or remains in effect, the NBA may not thereafter impose discipline on the player for that act or conduct.
\item
  Notwithstanding anything to the contrary contained in Section 10(a) or 10(b), (i) the same act or conduct by a player may result in both a termination of the player's Uniform Player Contract by his Team and the suspension of the player by the NBA if the egregious nature of the act or conduct is so lacking in justification as to warrant such double penalty, and (ii) both the NBA and the Team to which a player is traded may impose discipline for a player's failure to report for a trade in accordance with Paragraph 10(d) of the Uniform Player Contract.
\end{enumerate}

\hypertarget{league-investigations.}{%
\section{League Investigations.}\label{league-investigations.}}

\begin{enumerate}
\def\labelenumi{(\alph{enumi})}
\tightlist
\item
  Players are required to cooperate with investigations of alleged player misconduct conducted by the NBA. Failure to so cooperate, in the absence of a reasonable apprehension of criminal prosecution, will subject the player to reasonable fines and/or suspensions imposed by the NBA. Any investigations of alleged misconduct that is covered by the Joint NBA/NBPA Policy on Domestic Violence, Sexual Assault, and Child Abuse shall be governed by the terms of that Policy.
\item
  Except as set forth in Section 11(c) below, the NBA shall provide the Players Association with such advance notice as is reasonable in the circumstances of any interview or meeting to be held (in person or by telephone) between an NBA representative and a player under investigation by the NBA for alleged misconduct, and shall invite a representative of the Players Association to participate or attend. The failure or inability of a Players Association representative to participate in or attend the interview or meeting, however, shall not prevent the interview or meeting from proceeding as scheduled. A willful disregard by the NBA of its obligation to notify the Players Association as provided for by this Section 11(b) shall bar the NBA from using as evidence against the player in a proceeding involving such alleged misconduct any statements made by the player in the interview or meeting conducted by the NBA representative.
\item
  The provisions of Section 11(b) above shall not apply to interviews or meetings: (i) held by the NBA as part of an investigation with respect to alleged player misconduct that occurred at the site of a game; and (ii) which take place during the course of, or immediately preceding or following, such game. With respect to any such interview or meeting, the NBA's only obligation shall be to provide notice to the Players Association that the NBA will be conducting an investigation and holding an interview or meeting in connection therewith. Such notice may be given by telephone at a telephone number or by email at an email address to be designated in writing by the Players Association.
\end{enumerate}

\hypertarget{on-court-conduct.}{%
\section{On-Court Conduct.}\label{on-court-conduct.}}

\begin{enumerate}
\def\labelenumi{(\alph{enumi})}
\tightlist
\item
  The parties have agreed to all of the rules governing the conduct of players on the playing court (as that term is defined in Article XXXI, Section 9(c) below) that are contained in the Player Conduct, NBA Uniform Requirements, Dress Code, and Other Player-Related Matters Memo distributed by the NBA and dated June 1, 2023. Beginning with the 2023-24 Season, the NBA and the Players Association will bargain over any new rules governing the conduct of players on the playing court (including disciplinary penalties associated therewith) or any change to the agreed-upon rules governing the conduct of players on the playing court (including disciplinary penalties associated therewith); provided, however, that this obligation to bargain does not apply to the official playing rules of the NBA (or any change or modification thereof) or any rule affecting the integrity of the game or game play (or any change or modification thereof), except with respect to any change or modification to the disciplinary penalties associated with a player's violation of such rules.
\item
  Nothing in Section 12(a) above shall be construed to modify or alter (i) the NBA's existing disciplinary authority in this Agreement or Article 35 of the NBA Constitution governing the conduct of players on the playing court (as that term is defined in Article XXXI, Section 9(c) below), including, but not limited to, the NBA's ability to provide notice to players that it regards a type of on-court conduct to be violative of its disciplinary standards, (ii) the NBA's existing disciplinary authority in this Agreement and/or Article 35 of the NBA Constitution governing off-court conduct, or (iii) Article XXXVII, Section 2 of this Agreement governing player uniforms.
\item
  Prior to the imposition of a suspension on a player for conduct on the playing court (as defined in Article XXXI, Section 9(c)), the player will have the opportunity to request a telephonic meeting with the President, League Operations, the Executive Vice President, Basketball Operations, or their designee to discuss the incident and be heard as to why a suspension is unwarranted; provided, however, that the player must promptly notify the NBA of his desire for such a meeting, which will be scheduled to take place within a reasonable time period that will not interfere with the NBA's investigatory process and will not preclude the NBA from issuing a suspension prior to the player's next game. Notice to the player of a possible suspension may be given by the NBA to the Players Association by telephone at a telephone number or by email at an email address to be designated in writing by the Players Association. Notice by the player of his request for a meeting pursuant to this Section 12(c) may be provided through the Players Association on the player's behalf, and a representative of the Players Association may participate in any such telephone call. The NBA will consider any information provided during the meeting before finalizing its decision; provided, however, that nothing contained herein will require the NBA to alter its disciplinary decision or affect any rights the player has under Article XXXI to appeal that decision.
\end{enumerate}

\hypertarget{off-court-conduct.}{%
\section{Off-Court Conduct.}\label{off-court-conduct.}}

Following the imposition of discipline on a player by the NBA for off-court conduct, and upon request by the Players Association, the NBA shall identify for the Players Association the key evidence or other materials upon which the disciplinary decision was based. The foregoing obligation, including, but not limited to, the NBA's provision of such information and the extent or nature of the information provided, shall be without prejudice to the NBA, including by not limiting the evidence or other materials upon which the NBA may rely in any proceeding relating to the discipline imposed.

\hypertarget{motor-vehicles.}{%
\section{Motor Vehicles.}\label{motor-vehicles.}}

At the commencement of each Season, and if the player owns or operates any motor vehicle, the player will provide the Team with proof that the player possesses a valid driver's license, registration documents, and insurance for any such vehicle. For players who sign Player Contracts during the Season, the player will provide the Team with such information within fourteen (14) days following the execution of his Contract. Each player is also required to provide the Team with proof of any modifications or additions made to this information during the Season.

\hypertarget{player-convictions-and-other-discipline-involving-alcohol-or-controlled-substances.}{%
\section{Player Convictions and Other Discipline Involving Alcohol or Controlled Substances.}\label{player-convictions-and-other-discipline-involving-alcohol-or-controlled-substances.}}

\begin{enumerate}
\def\labelenumi{(\alph{enumi})}
\tightlist
\item
  In addition to any other discipline imposed by the NBA for such conduct, any player who is convicted of (including by a plea of guilty, no contest, or nolo contendere to) driving while intoxicated, driving under the influence, driving under the influence of a controlled substance (if that controlled substance is not a Prohibited Substance) or any similar crime shall be required to submit to a mandatory evaluation by the Medical Director of the Anti-Drug Program. After that mandatory evaluation, the Medical Director may require the player to attend up to ten (10) substance abuse counseling sessions.
\item
  No player shall use any Marijuana Product (defined below) while he is physically present at a facility or venue owned, operated, or being used by a Team, the NBA, or any Team- or League-related entity. Any violation of this Section 15(b) shall subject the player to discipline as is reasonable under the circumstances. With respect to discipline imposed by the NBA and/or the Team, the One Penalty rule set forth in Article VI, Section 10 of this Agreement shall apply.
\end{enumerate}

\hypertarget{player-arrests.}{%
\section{Player Arrests.}\label{player-arrests.}}

A Team shall not impose discipline on a player solely on the basis of the fact that the player has been arrested. Notwithstanding the foregoing, (a) a Team may impose discipline on a player for the conduct underlying the player's arrest if it has an independent basis for doing so, (b) nothing herein shall permit a Team to discipline a player for his failure to cooperate with a Team's investigation of his alleged misconduct if he has a reasonable apprehension of criminal prosecution, and (c) nothing herein shall prevent a Team from precluding a player from participating in Team activities without loss of pay to the extent it otherwise has the right to do so.

\hypertarget{joint-nbanbpa-policy-on-domestic-violence-sexual-assault-and-child-abuse.}{%
\section{Joint NBA/NBPA Policy on Domestic Violence, Sexual Assault, and Child Abuse.}\label{joint-nbanbpa-policy-on-domestic-violence-sexual-assault-and-child-abuse.}}

The parties have agreed to the Joint NBA/NBPA Policy on Domestic Violence, Sexual Assault, and Child Abuse (and any amendments thereto), which is attached as Exhibit F hereto. Any evaluation, counseling, treatment, and/or discipline of a player for engaging in acts covered by this Policy shall be governed by the terms of the Policy.

\hypertarget{trades.}{%
\section{Trades.}\label{trades.}}

Any player (or, for clarity, any player representative or person acting with authority on behalf of a player) who publicly expresses a desire to be traded to another Team shall be subject to a fine and/or a suspension. The maximum fine that may be imposed by the NBA on a player pursuant to the foregoing shall be \$150,000.

\hypertarget{player-involvement-with-gaming-companies.}{%
\section{Player Involvement with Gaming Companies.}\label{player-involvement-with-gaming-companies.}}

\begin{enumerate}
\def\labelenumi{(\alph{enumi})}
\tightlist
\item
  As used in this Section 19, the following terms shall have the following meanings:

  \begin{enumerate}
  \def\labelenumii{(\roman{enumii})}
  \tightlist
  \item
    ``Gaming Company'' means a Sports Betting Company, a Fantasy Sports Company, or any other entity that offers contests, wagers, or other transactions on which consumers can put money or other things of value at risk and the outcome of which is determined, in whole or in part, based upon the performance of NBA League players or NBA League teams in NBA League games or events.
  \item
    ``Sports Betting Company'' means an entity (A) that directly or indirectly offers, accepts, or facilitates wagering related to sporting events, or (B) whose operations are substantially dedicated to content related to wagering on NBA and other sporting events.
  \item
    ``Fantasy Sports Company'' means an entity that offers or facilitates contests in which participants submit entries in a contest (season-long, daily, or single-game), comprised of one or more selected teams or players, with the winning entries determined by the performance or statistics of the selected teams or players.
  \item
    ``NBA League'' means the NBA, WNBA, NBAGL, NBA 2K League, Basketball Africa League, and any other league associated with the NBA.
  \end{enumerate}
\item
  Investment in Gaming Companies.

  \begin{enumerate}
  \def\labelenumii{(\roman{enumii})}
  \tightlist
  \item
    Subject to Article XIII, a player may hold a direct or indirect ownership interest in a Gaming Company only if:

    \begin{enumerate}
    \def\labelenumiii{(\Alph{enumiii})}
    \tightlist
    \item
      Such interest is passive (i.e., includes no management, governance, voting, or executive role or other operational rights or roles);
    \item
      The player's ownership interest: (1) for any entity that offers, accepts, or facilitates NBA League-related bets, contests, or other transactions, is equal to less than a one percent (1\%) beneficial interest in any class of securities (or other class of ownership interests) in the entity (including via a partnership interest in a fund that owns an interest in such entity); or (2) for any entity that does not offer, accept, or facilitate NBA League-related bets, contests, or other transactions, is less than a fifty percent (50\%) beneficial interest in any class of securities (or other class of ownership interests) in the entity (including via a partnership interest in a fund that owns an interest in such entity); and
    \item
      Such interest is held, and such entity operates, in compliance with all applicable laws and regulations relating to sports wagering, fantasy sports contests, or similar transactions.
    \end{enumerate}
  \item
    Any player who holds an ownership interest in a Gaming Company shall disclose to the League Office (attn: General Counsel) and the Players Association, within 30 days of acquiring such interest, (A) the identity of the Gaming Company in which the player holds such interest, and (B) the percentage of the Gaming Company's overall ownership such interest represents.
  \end{enumerate}
\item
  Promotion and Endorsement of Gaming Companies.

  \begin{enumerate}
  \def\labelenumii{(\roman{enumii})}
  \tightlist
  \item
    Subject to Article XIII, a player may participate in the promotion or endorsement of a Gaming Company only if:

    \begin{enumerate}
    \def\labelenumiii{(\Alph{enumiii})}
    \tightlist
    \item
      Such participation is limited to (1) general brand promotion or endorsement, or (2) promotion or endorsement of betting on non-NBA League sports;
    \item
      Compensation for such participation is not determined in any respect by NBA League wagering or outcomes of NBA League games (e.g., compensation to the player may not be based on the amount wagered on NBA League games); and
    \item
      Such participation and such Gaming Company's operation comply with all applicable laws and regulations relating to sports wagering, fantasy sports contests, or similar transactions. The operation of a Gaming Company that is party to an agreement with the NBA or a Team shall, during the term of such agreement, be deemed in compliance with this subsection (c)(i)(C).
    \end{enumerate}
  \item
    For clarity, no player may participate in endorsement or promotional activity of a Gaming Company where such endorsement or promotion involves NBA League-related bets or contests.
  \end{enumerate}
\item
  For clarity, any investments in or promotions or endorsements of Gaming Companies not expressly permitted by this Section 19 are prohibited. In the event a player engages in a prohibited investment, promotion, or endorsement, then, without limiting other NBA rights or remedies, the player shall be required to promptly dispose of his ownership interest in the prohibited investment and/or immediately terminate his participation in the prohibited promotion or endorsement, as applicable.
\end{enumerate}

\hypertarget{player-involvement-with-cannabis-companies.}{%
\section{Player Involvement with Cannabis Companies.}\label{player-involvement-with-cannabis-companies.}}

\begin{enumerate}
\def\labelenumi{(\alph{enumi})}
\tightlist
\item
  As used in this Section 20, the following terms shall have the following meanings:

  \begin{enumerate}
  \def\labelenumii{(\roman{enumii})}
  \tightlist
  \item
    ``CBD'' means hemp-derived compounds that have a concentration of tetrahydrocannabinol (``THC'') at or below 0.3\% and contain no other form or amount of cannabis.
  \item
    ``CBD Products'' means supplements and other products containing CBD as an ingredient (e.g., oils, creams, drinks, pills, powders, and roll-ons), but does not mean products that meet the definition of ``Marijuana Products'' below or products containing any substance on the list of Prohibited Substances set forth in Exhibit I-2 to this Agreement or on Schedule I or II of the Controlled Substances Act.
  \item
    ``Marijuana Company'' means an entity that (A) produces or sells one or more Marijuana Products, including an entity that produces or sells both CBD Products and one or more Marijuana Products, and/or (B) produces or sells CBD Products and has an affiliate that produces or sells one or more Marijuana Products under the same or a substantially similar brand as such entity or CBD Products.
  \item
    ``Marijuana Products'' means supplements and other products (e.g., flower, oils, creams, drinks, pills, powders, and roll-ons) containing (A) a non-CBD form of cannabis as an ingredient, and/or (B) a concentration of THC above 0.3\%. For purposes of this Section 20, any products containing both CBD and a non-CBD form of cannabis, and any products containing kratom, shall be Marijuana Products.
  \end{enumerate}
\item
  Investment in Cannabis Companies.

  \begin{enumerate}
  \def\labelenumii{(\roman{enumii})}
  \tightlist
  \item
    Subject to Section 20(b)(ii) below and Article XIII, a player may hold a direct or indirect ownership interest (whether controlling or non-controlling) in an entity that produces or sells CBD Products, provided that (A) such entity does not also produce or sell one or more products containing any Prohibited Substance or any other Schedule I or II substance under the Controlled Substances Act, and (B) such interest is held, and such entity operates, in compliance with all applicable laws and regulations.
  \item
    Subject to Article XIII, a player may hold a direct or indirect ownership interest in a Marijuana Company, provided that:

    \begin{enumerate}
    \def\labelenumiii{(\Alph{enumiii})}
    \tightlist
    \item
      Such interest is passive (i.e., includes no management, governance, voting, or executive role or other operational rights or roles); and
    \item
      The player's ownership interest is equal to less than a fifty percent (50\%) beneficial interest in any class of securities (or other class of ownership interests) in such Marijuana Company (including via a partnership interest in a fund that owns an interest in such Marijuana Company); and
    \item
      Such interest is held, and such entity operates, in compliance with all applicable laws and regulations.
    \end{enumerate}
  \item
    For clarity, except as set forth in Section 20(b)(ii) above, no player may hold any ownership interest (whether direct or indirect, including via a partnership interest in a fund) in an entity that produces or sells any products containing any Prohibited Substance or any other Schedule I or II substance under the Controlled Substances Act.
  \end{enumerate}
\item
  Promotion and Endorsement of Cannabis Companies.

  \begin{enumerate}
  \def\labelenumii{(\roman{enumii})}
  \tightlist
  \item
    Subject to Article XIII, a player may participate in the promotion or endorsement of any brand, product, or service of an entity that produces or sells CBD Products, provided that such entity (A) is not a Marijuana Company, (B) does not also produce or sell one or more products containing any Prohibited Substance or any other Schedule I or II substance under the Controlled Substances Act, and (C) such participation and such entity's operation comply with all applicable laws and regulations.
  \item
    Notwithstanding Section 20(c)(i) above, a player may request permission from the NBA and the Players Association to promote or endorse any CBD Products that are produced or sold by a Marijuana Company. Such request must be in writing and include (A) a complete list of the products that the Marijuana Company produces or sells, (B) a complete list of all ingredients of such products, (C) a description of the player's proposed promotion or endorsement activity for the Marijuana Company's CBD Products, and (D) a detailed summary of the non-financial terms of any proposed promotion or endorsement agreement between the player and the Marijuana Company. Unless a player's request has been approved in writing by the NBA and the Players Association, the player may not promote or endorse any CBD Products that are produced or sold by a Marijuana Company.
  \item
    Upon receiving a player's written request pursuant to Section 20(c)(ii) above, the NBA and the Players Association shall each consider and determine whether to approve such request. Without limiting such approval right of the NBA and the Players Association, the promotion or endorsement by a player of a CBD Product that is produced or sold by a Marijuana Company (A) will not be permitted if such CBD Product is associated by the Marijuana Company with any Marijuana Product (e.g., the CBD Product is marketed or sold under a brand that also includes or refers to Marijuana Products) or if any proposed promotion creates a reasonable risk of public confusion with any Marijuana Product, and (B) if approved, shall be subject to any terms and conditions imposed by the NBA and/or the Players Association. In the event that any information provided in a player's request is inaccurate at the time it is submitted to the NBA or the Players Association, or in the event that such information later becomes inaccurate, the NBA or the Players Association may in their discretion withdraw their approval of the player's request.
  \end{enumerate}
\item
  For clarity, any investments in or promotions or endorsements of entities that produce or sell products containing a form of cannabis (including, for clarity, a CBD form of cannabis) not expressly permitted by this Section 20 are prohibited. In the event a player engages in a prohibited investment, promotion, or endorsement, then, without limiting other NBA rights or remedies, the player shall be required to promptly dispose of his ownership interest in the prohibited investment and/or immediately terminate his participation in the prohibited promotion or endorsement, as applicable.
\end{enumerate}

\hypertarget{gambling-by-former-nbagl-players.}{%
\section{Gambling by Former NBAGL Players.}\label{gambling-by-former-nbagl-players.}}

\begin{enumerate}
\def\labelenumi{(\alph{enumi})}
\tightlist
\item
  A player shall be subject to discipline imposed by the NBA for violations of NBAGL rules pertaining to gambling involving the NBA, NBAGL, and/or NBA-affiliated leagues that were committed during any prior period of time during which the player was subject to NBAGL rules relating to gambling. The NBA may impose such discipline only in circumstances where, and only to the extent that, discipline would be authorized by the NBA under this Agreement for the same conduct under NBA rules pertaining to gambling by players involving the NBA, NBAGL, and/or NBA-affiliated leagues.
\item
  Any player suspended by the NBAGL for violations of the NBAGL rules pertaining to gambling involving the NBA, NBAGL, and/or NBA-affiliated leagues who signs a Uniform Player Contract before the full term of the suspension is served shall serve the remainder of the suspension in the NBA. In addition, any player suspended under NBAGL rules pertaining to gambling involving the NBA, NBAGL, and/or NBA-affiliated leagues whose NBAGL contract ends before the full term of the suspension is served shall be subject to Article VI, Section 1(c) of this Agreement with respect to the NBAGL suspension if, at the start of the following NBA Regular Season, he is a Free Agent who has games remaining to be served on the NBAGL suspension. For purposes of Article VI, Section 1(c), the ``Team to which he was under contract when the suspension was imposed'' shall be deemed to be the NBA team, if any, with which the player first signs a Player Contract following imposition of the NBAGL suspension.
\end{enumerate}

\hypertarget{basketball-related-income-salary-cap-minimum-team-salary-tax-level-apron-levels-and-designated-share-arrangement}{%
\chapter{BASKETBALL RELATED INCOME, SALARY CAP, MINIMUM TEAM SALARY, TAX LEVEL, APRON LEVELS, AND DESIGNATED SHARE ARRANGEMENT}\label{basketball-related-income-salary-cap-minimum-team-salary-tax-level-apron-levels-and-designated-share-arrangement}}

\chaptermark{BASKETBALL RELATED INCOME, SALARY CAP, MINIMUM TEAM SALARY \ldots}

\hypertarget{definitions.-1}{%
\section{Definitions.}\label{definitions.-1}}

For purposes of this Agreement, the following terms shall have the meanings set forth below:

\begin{enumerate}
\def\labelenumi{(\alph{enumi})}
\tightlist
\item
  \textbf{Basketball Related Income.}

  \begin{enumerate}
  \def\labelenumii{(\arabic{enumii})}
  \item
    ``Basketball Related Income'' (``BRI'') for a Salary Cap Year means the aggregate operating revenues (including the value of any property or services received in any barter transactions), accounted for in accordance with Section 1(b)(1) below, received or to be received for or with respect to such Salary Cap Year by the NBA, NBA Properties, Inc., including any of its subsidiaries whether now in existence or created in the future (hereinafter, ``Properties''), NBA Media Ventures LLC (``Media Ventures''), any other entity which is controlled, or in which at least fifty percent (50\%) of the issued and outstanding ownership interests are owned, by the NBA, Properties, Media Ventures, and/or a group of NBA Teams (hereinafter, ``League-related entity'') (but excluding the amount of such League-related entity's revenues equal to the portion of its total revenues that is proportionate to the share of the entity's profits to which ownership interests not owned by the NBA, Properties, Media Ventures and/or a group of NBA Teams are entitled), all NBA Teams other than Expansion Teams during their first two (2) Salary Cap Years (but including the Expansion Teams' shares of national television, radio, cable and other broadcast revenues, and any other League-wide revenues shared by the Expansion Teams, provided such revenues are otherwise included in BRI) and Related Parties (in accordance with Section 1(a)(7)(i) below), from all sources, whether known or unknown, whether now in existence or created in the future, to the extent derived from, relating to, or arising directly or indirectly out of, the performance of Players in NBA basketball games or in NBA-related activities. For purposes of this definition of BRI: (x) ``operating revenues'' shall include, but not be limited to, any type of revenue included in BRI for the 1995-96 and 1996-97 Salary Cap Years (without regard to whether such type of revenue is received on a lump-sum, non-recurring or extraordinary basis, but subject to any specific rules set forth in this Article VII relating to the recognition or amortization of such amounts); and (y) ``Player'' means a person: who is under a Player Contract to an NBA Team; who completed the playing services called for under a Player Contract with an NBA Team at the conclusion of the prior Season; or who was under a Player Contract with an NBA Team during (but not at the conclusion of) the prior Season, but only with respect to the period for which he was under such Contract. Subject to the foregoing, BRI shall include, but not be limited to, the following revenues:

    \begin{enumerate}
    \def\labelenumiii{(\roman{enumiii})}
    \tightlist
    \item
      Regular Season gate receipts (or practice facility NBA-event receipts), net of applicable taxes, surcharges, imposts, facility fees, and other charges (including, without limitation, charges related to arena financings) imposed by governmental or quasi-governmental agencies other than income taxes (collectively, ``Taxes''), and net of all reasonable and customary Team and Related Party ticket-related expenses and premium seating ticket expenses related thereto, subject to the provisions of Section 1(a)(6) below, including, without limitation, gate receipts received or to be received by a Related Party in accordance with Section 1(a)(7)(i) below, including: (A) the value (determined on the basis of the price of the ticket) of all tickets traded by a Team for goods or services; and (B) the value (determined on the basis of the League-wide average ticket price for ``Non-Season Tickets'') of all tickets for Regular Season games provided by a Team on a complimentary basis, without monetary or other compensation to a Team including complimentary admission to luxury suites (including standing room only tickets and tickets provided to Team employees other than Players); provided, however, that (x) the value of the ``Excluded Complimentary Tickets'' with respect to all Regular Season games in a Season shall be excluded from BRI, and (y) in addition, tickets provided as part of sponsorships and other transactions, where the proceeds from such transactions have been included in BRI, shall not be included in determining the number of complimentary tickets in any Season. For purposes of the foregoing, (1) ``Non-Season Tickets'' shall mean only single-game tickets and tickets sold in packages covering fewer than fifty percent (50\%) of a Team's Regular Season home games and (2) ``Excluded Complimentary Tickets'' shall mean (a) 2.1 million tickets for each Season during the term of the Agreement, subject to increase pursuant to the following sentence, and (b) any tickets provided on a complimentary basis to or on behalf of Players. If, in any Salary Cap Year after the 2023-24 Season, the ratio of tickets sold to Regular Season home games (including contractually delivered sponsorship and trade tickets) is less than eighty percent (80\%) of the seating capacity for those Regular Season home games, then the number of Excluded Complimentary Tickets for that Salary Cap Year shall be increased by a number equal to (x) the difference between eighty percent (80\%) and the actual ratio of tickets sold to seating capacity, multiplied by (y) 2.1 million tickets. By way of example, if the actual ratio of tickets sold to seating capacity in the 2024-25 Season were seventy-nine percent (79\%), then Excluded Complimentary Tickets would increase by 21,000 tickets (i.e., (80\% - 79\%) * 2.1 million) for the 2024-25 Salary Cap Year;
    \item
      All proceeds of any kind, net of reasonable and customary expenses related thereto, subject to the provisions of Section 1(a)(6) below, from the broadcast or exhibition of, or the sale, license or other conveyance or exploitation of the right to broadcast or exhibit, NBA preseason, Regular Season and Playoff games and summer league and other NBA-related off-season games involving Players, highlights or portions of such games, and non-game NBA programming, on any and all forms of radio, television, telephone, internet, and any other communications media, forms of reproduction and other technologies, whether presently existing or not, anywhere in the world, whether live or on any form of delay, including, without limitation, network, local, cable, direct broadcast satellite and any form of pay television, and all other means of distribution and exploitation, whether presently existing or not and whether now known or hereafter developed, including, without limitation, such proceeds received or to be received by a Related Party (in accordance with Section 1(a)(7)(i) below), but not including the value of any broadcast, cablecast or telecast time provided as part of any such transaction that is used solely: (A) to promote or advertise the NBA, its Teams, League-related entities that generate BRI, Players, the NBA G League (the ``NBAGL'') (except to the extent the value of such time for the NBAGL exceeds \$5 million), the Women's National Basketball Association (the ``WNBA'') (it being agreed that the value of such time used to promote or advertise the WNBA shall not be less than \$2.5 million each Salary Cap Year), or the sport of basketball; (B) to promote or advertise products, programming, merchandise, services or events that (1) produce revenues that are includable in BRI or (2) are jointly licensed or otherwise agreed upon by the NBA and the Players Association; (C) to promote or advertise charitable, not-for-profit or governmental organizations or agencies; or (D) for public service announcements;
    \item
      All proceeds of any kind from Exhibition games including at least one NBA Team, net of Taxes and all reasonable and customary game, pre-season and training camp expenses (including summer league expenses), subject to the provisions of Section 1(a)(6) below, including, without limitation, such proceeds received or to be received by a Related Party (in accordance with Section 1(a)(7)(i) below);
    \item
      All playoff gate receipts of any kind, net of Taxes, arena rentals to the extent reasonable and customary, and all other reasonable and customary expenses, except the Player Playoff Pool, including, without limitation, such proceeds received or to be received by a Related Party (in accordance with Section 1(a)(7)(i) below);
    \item
      All proceeds of any kind, net of reasonable and customary expenses (including Taxes) related thereto, subject to the provisions of Section 1(a)(6) below, from: (A) in-arena (or in practice facility) sales of novelties and concessions (including revenues derived from the sale of novelties and concessions: (1) during (and immediately preceding or after) the Team's games or other public Team events at the arena (or practice facility), from carts and kiosks or other similar sales locations that are only operated on an intermittent basis (i.e., principally when an NBA, NHL, or other public event is being held at the arena (or, respectively, the practice facility)) or from restaurants that are only operated on an intermittent basis (i.e., principally when an NBA, NHL, or other public event is being held at the arena (or, respectively, the practice facility)), in (i) the arena plaza or elsewhere on the immediate perimeter of the arena (or, respectively, the practice facility), or (ii) directly across the street from the arena (or, respectively, the practice facility); and (2) from Team-organized viewing parties of NBA games held in any location), (B) sales of novelties and concessions in Team-identified stores located within such radius of the Team's home arena as is permitted by the NBA, (C) NBA game (or practice facility NBA-event) parking and programs, (D) Team sponsorships (whether or not the proceeds are directly or indirectly donated to charity), (E) Team promotions, (F) temporary arena signage (as defined in Section 1(a)(1)(vi) below), (G) arena club revenues, (H) summer camps, (I) non-NBA basketball tournaments, (J) mascot and dance team appearances, (K) the sale of the right to pour beverages or (except as provided in Section 1(a)(2)(xx) below) to provide concessions, (L) sales of jersey patch rights, and (M) other practice facility events to the extent such proceeds would be included in BRI if the event occurred in the Team's home arena, in each case, to the extent that such proceeds are related to the performance of Players in NBA basketball games or NBA-related activities, including, without limitation, such proceeds received or to be received by a Related Party (in accordance with Sections 1(a)(1)(vi) and 1(a)(7)(i) below). For the purposes of clarity, ``Team-identified stores'' includes stores owned by Teams or Related Parties that sell predominately Team-branded merchandise, whether or not the store is Team-identified;
    \item
      Fifty percent (50\%) of the gross proceeds, net of fifty percent (50\%) of Taxes, and net of fifty percent (50\%) of all reasonable and customary Team and Related Party expenses related thereto, subject to the provisions of Section 1(a)(6) below, from the sale of fixed arena signage within or outside of the arena in which an NBA Team plays more than one-half of its Regular Season home games, including, without limitation, such proceeds received or to be received by a Related Party (in accordance with this Section and Section 1(a)(7)(i) below). ``Fixed'' arena signage means signs (including, without limitation, electronic signs) that are displayed during all Regular Season NBA games and at least (A) seventy-five percent (75\%) of non-NBA paid ticketed events at the arena during the Regular Season and (B) ten (10) non-NBA paid ticketed events at the arena during the Regular Season (in each case prorated to reflect contracts in effect beginning in-season), with all other signs being treated as ``temporary'' signage (for clarity, subject to applicable allocations). Fixed arena signage also includes ``sponsorship entitlement areas'' that are accessible or visible during all Regular Season NBA games and at least (1) seventy-five percent (75\%) of non-NBA paid ticketed events at the arena during the Regular Season and (2) ten (10) non-NBA paid ticketed events at the arena during the Regular Season (in each case prorated to reflect contracts in effect beginning in-season). Revenues from sponsorship entitlement areas that do not qualify under the preceding sentence shall be treated as temporary signage. Revenues from signage outside the arena shall be included in BRI as fixed arena or temporary signage, as applicable, if: (a) the signage is attached to the arena or a physically connected parking facility; (b) the right to the signage revenues is conveyed in the Team's arena lease or other agreement, if applicable, governing a Team's use of an arena entered into by or on behalf of the Team (for clarity, in circumstances where the Team has a lease or similar agreement with a Related Party arena company, the foregoing is not intended to apply to any lease or similar agreement provisions, if any, between the Related Party arena company and the property owner governing the arena company's use of any property other than the arena itself); (c) only in the case of revenues received by the Team (and not by any Related Party), the signage is Team-identified (i.e., contains Team name, marks, logo, intellectual property); or (d) the signage is (x) in the area immediately proximate to the arena in an arena plaza in front of a main arena entrance or (y) attached to a standalone parking facility that is directly across the street from the arena (except that for fixed signage that falls within BRI solely under this subsection (d), twenty-five percent (25\%) of the gross proceeds (net of twenty-five percent (25\%) of Taxes, and net of twenty-five percent (25\%) of all reasonable and customary expenses related thereto subject to the provisions of this Section 1(a)(1)(vi) and Section 1(a)(6) below) shall be included as BRI revenues). Revenues from signage outside a Team's practice facility shall be included in BRI as fixed arena or temporary signage, as applicable, to the same extent, and subject to the same inclusion percentages, as signage outside the arena described in the preceding sentence. Other revenues received by a Team or Related Party from signage outside the arena or practice facility shall be excluded from BRI;
    \item
      Fifty percent (50\%) of the gross proceeds of any kind, net of fifty percent (50\%) of Taxes, and net of fifty percent (50\%) of all reasonable and customary Team and Related Party expenses related thereto, subject to the provisions of Section 1(a)(6) below, from the sale, lease or licensing of luxury suites calculated on the basis of the actual proceeds received by the entity, including, without limitation, proceeds received or to be received by a Related Party (in accordance with Section 1(a)(7)(i) below), that sold, leased, or licensed such luxury suites; provided, however, that, other than the additional amounts paid by luxury suite holders to the Team for tickets pursuant to arrangements in which admission to games is not part of the agreement to buy, lease or license the luxury suite, thereby requiring the luxury suiteholder to make a separate payment for such admission, if any, this amount shall be the only amount included in BRI for the sale, lease or licensing of luxury suites and that, to the extent that the sale, lease or licensing of the luxury suite grants rights to the luxury suite for a period of more than one (1) year, for purposes of calculating the amount includable in BRI for any Salary Cap Year, the proceeds shall be determined on the basis of the annual fee or charge provided for in any such transaction and, if payments are made in addition to or in the absence of such an annual fee or charge, the value of such payments shall be amortized over the period of the sale, lease or license, unless such period exceeds twenty (20) years, in which event an amortization period of twenty (20) years shall be used;
    \item
      Fifty percent (50\%) of the gross proceeds, net of fifty percent (50\%) of Taxes, and net of fifty percent (50\%) of all reasonable and customary Team and Related Party expenses related thereto, subject to the provisions of Section 1(a)(6) below, from arena naming rights agreements with respect to arenas in which an NBA Team plays more than one-half of its Regular Season home games, including, without limitation, such proceeds received or to be received by a Related Party (in accordance with Section 1(a)(7)(i) below);
    \item
      Except as provided in Section 1(a)(2) below, proceeds received by Properties or any other League-related entity, net of reasonable and customary expenses (including Taxes) related thereto, subject to the provisions of Section 1(a)(6) below, from the following: (A) international television (``ITV''); (B) sponsorships; (C) NBA-related revenues from NBA Entertainment (``NBAE''); (D) the All-Star Game; (E) other NBA special events; and (F) all other sources of revenue received by Properties or any other League-related entity, in each case under (A)-(F), to the extent that such proceeds are related to the performance of Players in NBA basketball games or NBA-related activities. Without limiting what constitutes reasonable and customary expenses in the applicable BRI categories, the revenues and expenses to be included in NBAE and ITV will be recorded consistent with the revenues and expenses recorded in NBAE and ITV as reflected in the Audit Report for the 2021-22 Salary Cap Year. For the avoidance of doubt, ITV shall be limited to revenues and related expenses from international linear telecast licensing fees, advertising revenues from such telecasts, satellite reimbursements, and international NBA TV affiliate fees;
    \item
      Proceeds from premium seat licenses (other than licenses of luxury suites, which are governed by Section 1(a)(1)(vii) above), net of Taxes, and all reasonable and customary Team and Related Party expenses related thereto, subject to the provisions of Section 1(a)(6) below, attributable to NBA-related events amortized over the period of the license (including, without limitation, such proceeds received or to be received by a Related Party (in accordance with Section 1(a)(7)(i) below), unless such period exceeds twenty (20) years, in which event an amortization period of twenty (20) years shall be used;
    \item
      Fifty percent (50\%) of the gross proceeds, net of fifty percent (50\%) of Taxes, and fifty percent (50\%) of reasonable and customary Team and Related Party expenses related thereto, subject to the provisions of Section 1(a)(6) below, from the sale of naming rights with respect to practice facilities used by NBA Teams, including, without limitation, such proceeds received or to be received by a Related Party (in accordance with Section 1(a)(7)(i) below);
    \item
      If the right to receive revenues included in BRI is sold or transferred to an entity other than an entity referred to in Section 1(a)(1) above (such that those revenues would not be included in BRI pursuant to that subsection), then BRI shall be deemed to include the amount of revenues that would have been received by the seller or transferor and would have been included in BRI in such Salary Cap Year (subject to any applicable allocations provided for above), absent such sale or transfer, provided that a pledge, hypothecation, collateral assignment or other similar transaction involving such revenues, shall not be considered a sale or transfer within the meaning of this Section 1(a)(1)(xii). The NBA will work in good faith to secure access to appropriate third-party books and records in the event the parties agree, or it is determined by an arbitrator, that a sale/transfer of BRI has occurred or been agreed to in accordance with this Section 1(a)(1)(xii). In any dispute over the value of BRI sold/transferred, subject to an arbitrator's determinations of admissibility and relevance, neither party shall be barred from seeking to rely on the terms of the underlying transaction;
    \item
      All proceeds, net of Taxes, less reasonable and customary expenses (which expenses shall be subject to New Venture treatment, if applicable, under Section 1(a)(6)(iii) below), subject to the provisions of Section 1(a)(6) below, from gambling on NBA games or any aspect of NBA games, subject to appropriate treatment of categories of excluded revenues or other amounts, if applicable, under Section 1(a)(2) below and allocations for multi-element deals. BRI shall exclude revenues from gambling on NBA games or any aspect of NBA games generated by casinos or other gambling businesses, owned or operated by a Team, Related Party, or a League-related entity, whose total revenues are not predominantly from gambling on NBA games or any aspect of NBA games;
    \item
      All proceeds, net of Taxes and reasonable and customary expenses related thereto, subject to Section 1(a)(6) below, from a Team's championship parade, provided, however, that in no event shall such expenses cause the amount included in BRI relating to the championship parade to be less than zero (0) for any Salary Cap Year;
    \item
      Fifty percent (50\%) of the gross proceeds, net of fifty percent (50\%) of reasonable and customary expenses (including Taxes) related thereto, subject to Section 1(a)(6) below, from (A) tours of the Team's home arena, and (B) fees from ATMs in the Team's home arena;
    \item
      Player income or ``privilege'' tax payments to Teams or Related Parties, provided that such payments will continue to be excluded from BRI for any Team or Related Party that received such payments in the 2015-16 Salary Cap Year (e.g., Memphis, New Orleans);
    \item
      Consistent with the parties' practice under the 2017 CBA, payments from the NBA to Teams for participation in international Regular Season games will be included in miscellaneous BRI at the Team level, with the NBA recording its expenses (including such payments to Teams) at the League level in Special Events;
    \item
      Licensing revenues from League and Team licensed products that are not co-licensed by current or retired players, net of applicable sales or similar taxes (e.g., VAT, HST), and all reasonable and customary expenses related thereto (``Net Licensing Revenues''), for an amount equal to, for each Salary Cap Year, the lesser of: (A) Net Licensing Revenues in such Salary Cap Year, or (B) the Incremental Content Expenses for such Salary Cap Year (as defined below). For purposes of this Article VII, Section 1, Incremental Content Expenses means, for each Salary Cap Year, an amount determined by the following calculation: (1) total deductible Team Content Expenses for that Salary Cap Year, less (2) an amount equal to the total deductible Team Content Expenses for the 2021-22 Salary Cap Year (i.e., \$78,862,052), growing at a rate of three percent (3\%) per Salary Cap Year, compounded.
    \end{enumerate}

    For example, if Net Licensing Revenues for the 2025-26 Salary Cap Year were \$180 million, and total deductible Team Content Expenses for such Salary Cap Year were \$250 million, then the amount included in BRI in respect of Net Licensing Revenues would be \$161.24 million, which is the lesser of: (a) \$180 million, and (b) \$161.24 million (i.e., \$250 million minus \$88.76 million (i.e., \$78,862,052 growing at three percent (3\%) per Salary Cap Year for four (4) Salary Cap Years)).

    Net Licensing Revenues shall not include value, if any, from contractual provisions (including, but not limited to, those previously identified by the Players Association in connection with prior BRI audits) that (x) require NBA or Team licensing partners to utilize the licensing rights purchased from the NBA and/or Teams in licensing deals, or (y) impose on NBA and/or Team licensing partners marketing or promotional obligations;

    \begin{enumerate}
    \def\labelenumiii{(\roman{enumiii})}
    \setcounter{enumiii}{18}
    \tightlist
    \item
      Twenty-five percent (25\%) of the gross proceeds, net of twenty-five percent (25\%) of Taxes, and twenty-five (25\%) of reasonable and customary Team and Related Party expenses related thereto, subject to the provisions of Section 1(a)(6) below, from arena plaza naming rights agreements with respect to arenas in which an NBA Team plays more than one-half of its Regular Season home games, including, without limitation, such proceeds received or to be received by a Related Party (in accordance with Section 1(a)(7)(i) below); and
    \item
      The specified value (included as barter) of data received pursuant to a contract to the extent that a data clause is specifically valued within the contract terms, net of reasonable and customary expenses (including Taxes) related thereto, subject to Section 1(a)(6) below, including, without limitation, such specified value received or to be received by a Related Party (in accordance with Section 1(a)(7)(i)).
    \end{enumerate}
  \item
    Notwithstanding anything to the contrary in Section 1(a)(1) above, it is understood that the following is a non-exclusive list of examples of revenues that are or may be received by the NBA, Properties, Media Ventures, other League-related entities, NBA Teams and Related Parties (the foregoing persons or entities, beginning with ``NBA,'' collectively referred to in this Section 1(a)(2) only as ``NBA-related entities'') that are not derived from, and do not relate to or arise out of, the performance of Players in NBA basketball games or in NBA-related activities or are otherwise expressly excluded from the definition of BRI:

    \begin{enumerate}
    \def\labelenumiii{(\roman{enumiii})}
    \tightlist
    \item
      Proceeds from the assignment of Player Contracts;
    \item
      Proceeds (A) from the sale, transfer or other disposition of any of the assets or property (excluding ordinary course sales of inventory and the revenues (if any) deemed to be included in BRI pursuant to Section 1(a)(1)(xii) above) of, or ownership interests in, any NBA-related entity, or (B) from loans or other financing transactions;
    \item
      Proceeds from the grant of Expansion Teams and relocation fees paid by existing Teams to NBA-related entities;
    \item
      Dues;
    \item
      Capital contributions received by an NBA-related entity from one of its owners, shareholders, members or partners;
    \item
      Fines and compensation withheld in connection with suspensions;
    \item
      Revenue sharing (by means of revenue transfers or otherwise) among Teams;
    \item
      Interest income;
    \item
      Insurance recoveries, except where, and only to the extent that, such recoveries are in respect of lost revenues that would have otherwise been included in BRI, in which event such recoveries shall be included in BRI in the Salary Cap Year in which they are received;
    \item
      Proceeds from the sale or rental of real estate;
    \item
      Any thing of value received in connection with the design or construction of a new or renovated arena or other team facility including, but not limited to, receipt of title to or a leasehold interest in real property or improvements, reimbursement of project-related expenses, benefits from project-related infrastructure improvements, or tax abatements, unless (and only to the extent that) such value is being provided to the Team or a Related Party in lieu of payments that the Team or Related Party would have otherwise received pursuant to an arena lease or other instrument concerning a Team's use of an arena (``lease'') and would have constituted BRI if paid to the Team or a Related Party; provided, however, that the determination of the amount, if any, to be included in BRI with respect to the value of any of the foregoing shall be made either (A) in accordance with the provisions of Section 1(a)(4) below or (B) based upon direct evidence that the Team or Related Party, after proposing that it would receive certain revenues constituting arena-generated BRI, subsequently agreed specifically to forego such revenues in direct exchange for a thing of value (as described above in this Section 1(a)(2)(xi)) with the consequence that the arena-generated BRI revenues received or to be received by the Team or Related Party were or would be (in the opinion of the Accountants) less than the fair market value of arena-generated BRI revenues received or to be received by other NBA Teams in similar transactions, or (C) based upon direct evidence that the parties to the transaction had agreed that certain revenues constituting arena-generated BRI would be paid to the Team or Related Party and that such revenues were subsequently foregone by the Team or the Related Party in direct exchange for a thing of value (as described above in this Section 1(a)(2)(xi)); and provided further that, when a determination is made pursuant to clause (B) or clause (C) of this Section 1(a)(2)(xi), the amount(s), if any, to be included in BRI shall be allocated (with an appropriate interest adjustment to reflect the time value of money where the thing of value received by the Team or Related Party is in the form of cash or a cash equivalent, such as a check or wire transfer) over the Salary Cap Years in which the arena-generated BRI revenues foregone would have been received by the Team or Related Party (up to a maximum of twenty (20) Salary Cap Years) and not on a lump-sum basis;
    \item
      Any thing of value that induces or is intended to induce a Team either to relocate to or remain in a particular geographic location (whether or not provided in connection with a new or renegotiated arena lease), unless (and only to the extent that) such value is being provided to the Team or a Related Party in lieu of payments that the Team or Related Party would have otherwise received pursuant to an arena lease and that would have constituted BRI had they been paid to the Team or a Related Party; provided, however, that the determination of the amount, if any, to be included in BRI shall be made either (A) in accordance with the provisions of Section 1(a)(4) below or (B) based upon direct evidence that the parties to the transaction had agreed that certain revenues constituting arena-generated BRI would be foregone by the Team or Related Party, in direct exchange for a thing of value as described above in this Section 1(a)(2)(xii), and provided, further that, when a determination is made pursuant to clause (B) of this Section 1(a)(2)(xii), the amount(s), if any, to be included in BRI shall be allocated (with an appropriate interest adjustment to reflect the time value of money where the thing of value received by the Team or Related Party is in the form of cash or a cash equivalent, such as a check or wire transfer) over the Salary Cap Years in which the arena-generated BRI revenues foregone would have been received by the Team or Related Party (up to a maximum of fifteen (15) Salary Cap Years) and not on a lump-sum basis. With respect to transactions involving payments asserted to fall within the exclusion in this Section 1(a)(2)(xii), the NBA will provide the Players Association with the executed memoranda of understanding, term sheet, or other such executed summary of terms, if any, for such underlying transactions;
    \item
      Payments made to Teams or to the NBA pursuant to the provisions of Article VII, Section 12 (Designated Share Arrangement) below;
    \item
      Distributions, dividends or royalties paid by any NBA-related entity to owners, shareholders, members or partners;
    \item
      Any category or source of revenue or proceeds that was expressly identified in any BRI Report (as defined in Section 10(b) below) or in any document or written communication (including debriefing memos) authored by the Accountants and provided to the Players Association and the NBA (but excluding any underlying work papers) in connection with the Audit Reports for any of the 1995-96 through 2021-22 Salary Cap Years that was not included in BRI for such Salary Cap Years, unless such category or source was included on the ``open issues'' list prepared by the Accountants in connection with any of the Audit Reports for the 2005-06 through 2021-22 Salary Cap Years, in which case such category or source shall be included in or excluded from BRI, as the case may be, in accordance with the other terms of this Article;
    \item
      Proceeds received by (A) Properties (and its related entities) that were treated or, consistent with past practice, would have been treated as within the scope of the Agreement between NBA Properties, Inc., and the National Basketball Players Association, dated as of September 18, 1995, as amended January 20, 1999, July 29, 2005 and December 8, 2011 (the ``2011 Group License Agreement'') (including, but not limited to, proceeds received pursuant to the license of ``fantasy games,'' which proceeds would have been included in the computation of Player Merchandise Revenues in accordance with the 2011 Group License Agreement), or (B) a League-related entity relating to the following categories defined in the same manner as was used in the audited League Entities' Combined Financial Statements for the year ended September 30, 2021: (1) licensing, other than Net Licensing Revenues to the extent included in BRI pursuant to Section 1(a)(1)(xviii), above; and/or (2) a League-related entity's representation of, and services performed for, third parties. For purposes of the foregoing sentence, ``third parties'' refers to persons or entities that are not owned or controlled by persons or entities that own a majority interest in or otherwise control an NBA Team or, if such third party is a Related Party, proceeds received by the League-related entity shall not be included in BRI if representation of such Related Party does not relate either to such entity's NBA ownership or NBA Players;
    \item
      Monies collected from Team-related fundraising for charitable purposes or other charitable activities (including Team-organized ``50/50 raffles''), other than monies paid pursuant to Team sponsorship agreements that are included in BRI pursuant to Section 1(a)(1)(v) above;
    \item
      Proceeds solely related to the NBA 2K League, NBAGL and other leagues, teams and basketball organizations (e.g., an international league) that do not involve the playing of basketball by any then-current NBA players;
    \item
      Proceeds from the leasing or use of any Team physical assets (e.g., a Team plane);
    \item
      Any thing of value received from a concessionaire, food service vendor or other third party equipment or service provider that, if received in kind, is installed in an NBA arena or, if received in cash, is directed to defraying the costs of the construction or substantial renovation of an NBA arena;
    \item
      Proceeds from businesses outside the arena (e.g., restaurants, casinos, hotels, retail businesses, etc.), except for revenues otherwise included in BRI for Team-identified stores. For clarity, the foregoing exclusion will not apply to revenues from the business operations of the NBA basketball team that are otherwise includable as BRI under other provisions of this Agreement, including, without limitation, revenues received from sales of Team game tickets, media rights, sponsorships, signage outside the arena (subject to the limitations set forth in Section 1(a)(1)(vi) above), and arena plaza game-day sales of novelties and concessions (subject to the limitations set forth in Section 1(a)(1)(v) above);
    \item
      Without limitation to any other basis for non-inclusion in BRI, BRI shall not include the value, if any, of (A) goods or services that are operationally necessary to the performance of a contract, including, without limitation, certain ticketing platforms and tools (e.g., Pricemaster, Presence, Open Distribution, Archtics, Ticketmaster Marketplace,Ticketmaster Account Manager), (B) product discounts, or (C) waived fees;
    \item
      Value, if any, attributable to data received by an NBA-related entity (or the right to receive such data), whether or not the provision of such data is pursuant to a contractual obligation, other than amounts that are included in BRI pursuant to Section 1(a)(1)(xx) above; and
    \item
      Value, if any, from contractual provisions (including but not limited to those previously identified by the Players Association in connection with prior BRI audits) that require NBA or Team marketing partners to utilize the marketing rights purchased from an NBA-related entity in sponsorship deals.
    \end{enumerate}
  \item
    The parties agree that (i) in determining whether a category or source of revenue or proceeds constitutes BRI: (A) consideration shall be given to whether such category or source is more similar in kind or nature to the included categories and sources listed in Section 1(a)(1)(i) through (xx) above, on the one hand, or to the excluded categories and sources listed in Section 1(a)(2)(i) through (xxiv) above, on the other; and (B) no inference may be drawn from the fact that such category or source was not included in the categories and sources listed in Section 1(a)(1)(i) through (xx) above, or the fact that such category or source was not included in the categories and sources listed in Section 1(a)(2)(i) through (xxiv) above; and (ii) in any proceeding involving a dispute over (A) the includability or categorization of any revenue or expense item for BRI purposes; (B) the amount to be included in or deducted from BRI with respect to any revenue or expense item; or (C) the accounting methodology used by the Accountants in connection with any audit of BRI, the parties may refer to the past practice of the parties or the Accountants in connection with the Audit Reports for any of the 1999-2000 through 2021-22 Salary Cap Years; provided, however, that no reference may be made to the past practice of the parties or the Accountants with respect to any source or category of revenue or expense that was included on the ``open issues'' list prepared by the Accountants in connection with any of such Audit Reports; provided, further, that any such past practice shall be superseded to the extent changed or clarified by the terms of this Agreement. In addition, no reference may be made, with respect to expenses related to the NBA's non-international business, to the fact that such category of expenses falls within Section 1(a)(14) below, to argue for the inclusion or exclusion of expenses related to the League's non-international business.
  \item
    The parties agree that, with respect to any lease entered into after the date of this Agreement between a Team (or a Related Party) and an arena that is not a Related Party, the Accountants may attribute to the Team (or a Related Party) for purposes of computing BRI for a Salary Cap Year portions of arena revenues received by the arena or its related entities that would be included in BRI if received by the Team (or a Related Party) to the following extent: in the event of a renewal, extension or renegotiation of a lease between the same parties, or a new lease entered into by a Team (or a Related Party) with an arena that is not a Related Party, the Team will be deemed to receive in the first Salary Cap Year covered by the new lease or by the renewal, extension or renegotiation of the existing lease (as the case may be) the greater of (i) the amount of such revenues that the Team or the Related Party in fact receives under the lease or, (ii) if in the opinion of the Accountants, the Team (and/or the Related Party) is receiving substantially less than fair market value as determined by the Accountants (taking into account factors such as the rent paid by the Team or the Related Party, the number and identity of other major tenants in the arena, market conditions, the extent to which arena revenues are used to fund construction or renovations of the arena, and comparable lease arrangements in the NBA), an amount determined by the Accountants to constitute the fair market value of the revenues that a tenant, in the same circumstances as the Team or Related Party, would receive for such Salary Cap Year. In either of the preceding cases, the Accountants will also determine the amount to be included in BRI for Salary Cap Years beyond the first Salary Cap Year.
  \item
    \begin{enumerate}
    \def\labelenumiii{(\roman{enumiii})}
    \tightlist
    \item
      In no event shall the same revenues be included in BRI, directly or indirectly, more than once (including as a result of changes in accounting methods or practices), the purpose of this provision being to preclude the double-counting of revenues, whether in the same or in multiple Salary Cap Years.
    \item
      In no event shall the same expenses be deducted from BRI, directly or indirectly, more than once (including as a result of changes in accounting methods or practices), the purpose of this provision being to preclude the double-counting of expenses, whether in the same or in multiple Salary Cap Years.
    \end{enumerate}
  \item
    Subject to Section 11 below (Players Association Audit Rights):

    \begin{enumerate}
    \def\labelenumiii{(\roman{enumiii})}
    \tightlist
    \item
      With respect to expenses incurred in connection with all proceeds coming within Section 1(a)(1)(v) above, all reported expenses shall be conclusively presumed to be reasonable and customary, and such expenses shall not be the subject of the accounting procedures set forth in Section 10 below.
    \item
      With respect to expenses incurred in connection with all proceeds coming within Section 1(a)(1)(ix) above that are consistent with the types and categories of expenses incurred by Properties as reflected in the audited financial reports of Properties for the year ended July 31, 1994, (1) all such reported expenses shall be conclusively presumed to be reasonable and customary, and such expenses shall not be the subject of the accounting procedures set forth in Section 10 below, but (2) such expenses shall be disallowed to the extent they exceed the ratio of expenses to revenues for the category of revenues set forth in Exhibit D hereto.
    \item
      With respect to the NBA Store (the ``Store'') and any other new venture or business (whether or not involving the creation of a new entity) undertaken by the NBA, Properties, Media Ventures, or any other League-related entity requiring significant capital investment or start-up costs (``New Venture''), the League-related entities shall be able to deduct from BRI reasonable and customary expenses related thereto, including, but not limited to, cost of goods sold, sales tax, all reasonable operating expenses of the Store or New Venture (including, but not limited to, salaries and benefits directly related to the operations of the Store or New Venture, promotional and advertising costs, rent, direct overhead, general and administrative expenses of the Store or New Venture), reasonable financing costs and amortization of capital improvements and start-up costs; provided, however, that in no event shall the expenses attributable to the Store or New Venture cause the amount included in BRI for the Store or New Venture to be less than zero (0) for any Salary Cap Year.
    \item
      With respect to miscellaneous BRI or new categories of BRI (other than revenues attributable to the Store or a New Venture), the NBA, Properties, Media Ventures, other League-related entities, Teams and Related Parties shall be able to deduct all reasonable and customary expenses (including reasonable and customary Taxes), including, for example, in connection with All-Star Weekend, subject to the terms of this Section 1(a)(6).
    \item
      In each Salary Cap Year, except for (A) Team Content Revenues and Team Content Expenses (as defined below) and (B) Playoff-Related Revenues and Expenses (as defined below), all Team and Related Party revenues included in, and all Team and Related Party expenses deducted from, BRI are subject to an aggregate uniform percentage-of-revenues expense cap of eleven and one-tenth percent (11.1\%) (see also Exhibit D hereto), with any such expenses disallowed to the extent they exceed that cap. For the avoidance of doubt, for the 2021-22 Salary Cap Year, the total Team and Related Party revenues and expenses that would have been subject to the new eleven and one-tenth percent (11.1\%) expense cap were the amounts identified in the parties' letter agreement, dated June 28, 2023. For the purposes of this Article VII, Section 1, (1) ``Team Content Revenues'' and ``Team Content Expenses'' mean, respectively, revenues and related expenses from local TV, cable, and Team direct-to-consumer media and (2) ``Playoff-Related Revenues and Expenses'' means the revenues and expenses reported in the ``playoff gate receipts, net'' amount shown in the Audit Report for the 2015-16 Salary Cap Year. Team Content Revenues, Team Content Expenses, and Playoff-Related Revenues and Expenses are not subject to the above uniform expense cap. Team Content Expenses will be deductible and uncapped, and will include, without limitation those expenses identified in the parties' letter agreement, dated June 28, 2023. Playoff-related expenses will continue to be deductible in accordance with the terms of the 2011 CBA as reflected in the Audit Report for the 2015-16 Salary Cap Year. For the avoidance of doubt, Taxes will be deducted from revenues included in BRI under Sections 1(a)(1)(i), (iii), (vi), (vii), (viii), (x), (xi), (xiii), (xiv), (xviii), and (xix) (to the extent set forth in those subsections) before the application of the eleven and one-tenth percent (11.1\%) ratio in calculating the uniform expense cap, and before the deduction of expenses.
    \item
      To the extent that, for a Salary Cap Year, total Team Content Expenses and League Content Expenses (inclusive of NBAE and ITV expenses after application of caps) exceeds the product of Team Content Revenues and League Content Revenues (inclusive of NBAE and ITV revenues) multiplied by the Rollover Ratio (as defined below), that excess would be amortized over such Salary Cap Year and the next two Salary Cap Years. The ``Rollover Ratio'' shall be eighteen and one-half percent (18.5\%), except that, beginning in the 2026-27 Salary Cap Year, the Rollover Ratio shall be the higher of (A) eighteen and one-half percent (18.5\%) or (B) the highest actual ratio of Team Content Expenses and League Content Expenses to Team Content Revenues and League Content Revenues in any of the preceding Salary Cap Years under the CBA. Any amortized amount would be excluded from each Salary Cap Year's rollover threshold calculation. An imputed interest rate equal to the 1-month Secured Overnight Financing Rate plus 1.225\% (calculated as of the July 1 of the Salary Cap Year during which such interest accrues) will be applied on amortized amounts and recouped in each Salary Cap Year in which amounts are amortized. For the purposes of this Article VII, Section 1, ``League Content Revenues'' and ``League Content Expenses'' mean, respectively, revenues and expenses related to all NBA and League-related content categories of BRI, including Digital, International League Pass (DBS), NBAE, ITV, Media Ventures, Radio, and National TV. For the avoidance of doubt, League Content Expenses will remain deductible and, with the exception of NBAE and ITV, uncapped.
    \end{enumerate}

    For example, if, in the 2025-26 Salary Cap Year, the sum of Team Content Revenues and League Content Revenues were \$5 billion, and the sum of Team Content Expenses and League Content Expenses (after application of caps on NBAE and ITV expenses) were \$1 billion, and the imputed interest rate calculated pursuant to this Section 1(a)(6)(vi) were four percent (4\%), then the amortized amount would be \$75 million (i.e., \$1 billion -- \$925 million (i.e., 18.5\% of \$5 billion)). \$25 million of the amortized amount would be deducted from BRI in the 2025-26 Salary Cap Year. In the 2026-27 Salary Cap Year, the amortized amount to be deducted from BRI in respect of the 2025-26 Salary Cap Year would be \$26 million (i.e., \$25 million, grown at four percent (4\%) interest per Salary Cap Year for one Salary Cap Year). In the 2027-28 Salary Cap Year, the amortized amount to be deducted from BRI in respect of the 2025-26 Salary Cap Year would be \$27.04 million (i.e., \$25 million, grown at four percent (4\%) interest per Salary Cap Year for two Salary Cap Years).
  \item
    It is acknowledged by the parties hereto that for purposes of determining BRI:

    \begin{enumerate}
    \def\labelenumiii{(\roman{enumiii})}
    \tightlist
    \item
      Some NBA Teams have engaged or may engage in transactions with third parties that control, or own at least fifty percent (50\%) of, the NBA Team or that are controlled or owned at least fifty percent (50\%) by the persons or entities controlling or owning at least fifty percent (50\%) of the NBA Team (such third parties are referred to in this Agreement as a ``Related Party''), and Related Parties themselves engage in transactions with third parties that may result in a Related Party's receipt of revenues that constitute BRI. (Any entity that was an ``entity related to an NBA team'' as defined by Article VII, Section 1(a)(4)(i) of the September 18, 1995 Collective Bargaining Agreement between the NBA and the Players Association (the ``1995 CBA'') shall be deemed a Related Party under this Agreement for so long as such entity continues to be an entity related to an NBA Team within the meaning of the 1995 CBA.) As provided in Section 1(a)(1) above, the relevant proceeds received by any Related Party that come within such subsection and that relate to such Related Party's Team shall be included in BRI. However, except in connection with telecast agreements (which are subject to Section 1(a)(7)(ii) below), with respect to any such revenues or proceeds retained or received by a Related Party (other than arena revenues that relate to such Related Party's Team including, but not limited to, in-arena sales of novelties and concessions, NBA game parking, arena club revenues, suite and seat revenues and fixed and temporary in-arena signage, which shall be included in BRI as if received by the Team), or by a Team pursuant to a transaction with a Related Party, such revenues or proceeds shall be included in BRI only to the extent that the NBA and the Players Association agree or, if they fail to agree, the Accountants shall reasonably determine the amount, if any, of such revenues or proceeds to attribute to the Team (taking into account factors such as the nature of the transaction, arrangement and/or relationship between the Team and the Related Party or between the Related Party and a third party, any amounts included in BRI with respect to other Teams (or Related Parties) that have entered into comparable transactions, arrangements and/or relationships with third parties, market conditions, the nature of any services or activities performed by the Related Party for, or in connection with, the generation of revenues or proceeds and the amount of revenues or proceeds that the Related Party would be expected to retain or receive with respect to comparable transactions, arrangements and/or relationships with third parties), and the amount so attributed shall be the only amount included in BRI. To the extent that the amount of such proceeds to be included in BRI cannot reasonably be determined with respect to any particular transaction, the Accountants shall determine a reasonable amount with respect to such transaction, which shall be included in BRI. (In the event the Accountants refuse to make any such determination, such determination shall be made by a jointly selected expert with respect to any such transaction.) Without limiting the foregoing, in no event shall BRI include consideration paid to a Related Party in connection with rights acquired by such Related Party from a Team for fair market value, even if such consideration relates to NBA games or NBA-related activities (including, by way of example and not limitation, advertising revenue or subscriber fees earned by a Related Party television network that relate, directly or indirectly, to the telecast of NBA games licensed to the television network by a Team).
    \item
      In the event that, following the execution of this Agreement, a Team (other than the New York Knicks (``Knicks'')) enters into a local or regional telecast agreement with a Related Party, a copy of such agreement shall be provided to the Players Association within ten (10) days of approval of such agreement by the NBA. The Players Association and the NBA shall each have the right, not later than ten (10) days following the date on which the Players Association receives a copy of such agreement, to submit such agreement to a jointly-selected television valuation expert or (in the absence of such agreement) determined in accordance with the procedure set forth in this subsection (``TV Expert'') for the limited purpose set forth in this Section 1(a)(7)(ii). In the event that a party has so elected to submit such agreement to a TV Expert and the parties have not jointly selected a TV Expert within twenty (20) days following the date on which the Players Association receives a copy of such agreement, each party shall appoint its own television valuation designee and the two designees so appointed shall within ten (10) days of their appointment, jointly select a third party to serve as the TV Expert. Such TV Expert shall review such agreement to determine if the aggregate amount to be paid to the Team by the Related Party for the rights to telecast the Team's games pursuant to such agreement is more than fifteen percent (15\%) above or more than fifteen percent (15\%) below the fair market value of such rights over the term of such agreement. In making such determination, the TV Expert may take into account factors such as the nature of the transaction, arrangement and/or relationship between the Team and the Related Party, any amounts included in BRI with respect to other Teams (or Related Parties) that have entered into comparable transactions, arrangements and/or relationships with other programming licensors, market conditions, the nature of any services or activities performed by the Related Party for, or in connection with, the generation of revenues or proceeds and the amount of revenues or proceeds that the Related Party would be expected to retain or receive with respect to comparable transactions, arrangements and/or relationships with third parties; provided that in no event shall BRI include consideration paid to a Related Party in connection with rights acquired by such Related Party from a Team for fair market value, even if such consideration relates to NBA games or NBA-related activities (including, by way of example and not limitation, advertising revenue or subscriber fees earned by a Related Party television network that relate, directly or indirectly, to the telecast of NBA games licensed to the television network by a Team). In the event that the TV Expert determines that such aggregate amount is more than fifteen percent (15\%) above or below fair market value, the TV Expert shall be instructed to submit to the parties the amount for each Season of such agreement that he determines reflects the fair market value of such rights and such amounts, and no other amounts, shall be included in BRI with respect to such agreement for each Salary Cap Year covered by such agreement. Any determination made by the TV Expert pursuant to either of the preceding two sentences shall be submitted to the parties no later than twenty (20) days from the date on which such agreement was submitted to the TV Expert for his review. Any fees or costs associated with the retention or determination of the TV Expert shall be borne equally by the Players Association and NBA. The Players Association and the TV Expert shall maintain the confidentiality of any such agreement (and any determination made by the TV expert in accordance with this Section 1(a)(7)(ii)) pursuant to the terms of Section 11(c) below relating to the confidentiality of BRI Audits.
    \item
      With respect to the transactions listed below in this Section 1(a)(7)(iii), the parties agree that, because the proceeds attributable to these transactions cannot be accurately ascertained, the following procedures shall be used for each NBA Season in which MSG Network is a Related Party of the Knicks (in the case of Section 1(a)(7)(iii)(A) below) and the Madison Square Garden arena is a Related Party of the Knicks (in the case of Section 1(a)(7)(iii)(B) below):

      \begin{enumerate}
      \def\labelenumiv{(\Alph{enumiv})}
      \tightlist
      \item
        New York Knicks transaction with MSG Network regarding the sale of local media rights: BRI for the Knicks for each NBA Season covered by this Agreement shall include an amount equal to the net proceeds included in BRI attributable to the Los Angeles Lakers' sale, license or other conveyance of all local media rights (including, but not limited to, broadcast and cable television and radio) for such NBA season.
      \item
        New York Knicks transactions with Related Parties involving signage: BRI for the Knicks for the 2021-22 NBA Season shall include \$16,560,026 for signage. In each subsequent Season covered by this Agreement, this amount shall be increased (or decreased, as the case may be) by the League-wide percentage increase (or decrease) in signage as determined in accordance with Sections 1(a)(1)(v) and (a)(1)(vi) above.
      \end{enumerate}

      At such time as the MSG Network and/or the Madison Square Garden Arena are no longer Related Parties, BRI for the New York Knicks in the categories described in Section 1(a)(7)(iii)(A) and/or (B) above, as the case may be, shall not be determined in accordance with the foregoing and will instead be determined by the applicable provisions of Sections 1(a)(1) and (a)(7)(ii) above.
    \end{enumerate}
  \item
    In the event that, pursuant to the NBA's national broadcast, national telecast and network cable television agreements, NBA Teams receive revenue sharing proceeds that are attributable to NBA game telecasts in more than one Salary Cap Year, such proceeds shall be allocated over the same number of Salary Cap Years (beginning with first Salary Cap Year after the Salary Cap Year in which such proceeds are actually received) as the number of Salary Cap Years in which such games were televised. Any other contingent payments received by the NBA pursuant to such agreements shall be included in BRI to the extent and in a manner agreed upon by the parties, or, if the parties cannot agree, in a reasonable manner determined by the Accountants.
  \item
    The NBA and each NBA Team shall in good faith act and use their commercially reasonable efforts to increase BRI for each Salary Cap Year during the term of this Agreement. In the exercise of such commercially reasonable efforts, the NBA and each NBA Team shall be entitled to act in a manner consistent with their reasonable business judgment and shall not (i) take any action intended to benefit, at the expense of BRI, other commercial activities (such as the WNBA and the NBAGL) unrelated to the performance of Players in NBA basketball games or in NBA-related activities, or (ii) shift or forgo revenues attributable to Salary Cap Years during the term of this Agreement in exchange for revenues or benefits during Salary Cap Years following the expiration of this Agreement (unless there is a reasonable business justification unrelated to collective bargaining for such shift or forgoing). There shall be no obligation on the part of the NBA or any NBA Team to accelerate into Salary Cap Years within the term of this Agreement revenues attributable to Salary Cap Years following the expiration of this Agreement. In evaluating compliance with this subsection, the parties and the System Arbitrator shall consider and give substantial weight to the reasonable business judgment of the NBA or the NBA Team but no deference will be applied where the NBA is alleged to have shifted or forgone revenues of \$350 million or more for the purpose of securing leverage in collective bargaining, in which case any finding of non-compliance shall require proof by a clear preponderance of the evidence. The following is a list of decisions in respect of which the business judgment of the NBA or an NBA Team shall conclusively be deemed reasonable: membership location; arena capacity or configuration; number and location of games played; whether to outsource or operate a line of business; and whether to accept or decline a sponsorship, advertising or naming rights opportunity. The foregoing list shall not limit in any manner the circumstances in which the business judgment of the NBA or an NBA Team may be deemed reasonable.
  \item
    The parties agree that upon a finding by the System Arbitrator (which, if appealed, is affirmed by the Appeals Panel) that the NBA or an NBA Team (or a Related Party) has willfully failed to provide to the Accountants information concerning revenues or expenses material to the Accountants' preparation of an Audit Report, and that such failure to provide information resulted in an understatement of BRI of more than \$5,511,614 with respect to the 2023-24 Salary Cap Year (increasing by four and one-half percent (4.5\%) for each subsequent Salary Cap Year of this Agreement, beginning with the 2024-25 Salary Cap Year), then the amount by which BRI was understated shall be included in BRI in the Salary Cap Year in which such finding is made, with interest accruing from the date of the Audit Report for the Salary Cap Year in which such amount would have been included but for such understatement, with interest (at a rate equal to the one (1) year Treasury Bill rate as published in The Wall Street Journal on the date of the issuance of such Audit Report). In addition, if any Team, or if the NBA, violates the foregoing, it shall be fined \$3 million for its first violation during the term of this Agreement and an additional \$1.5 million for each additional violation. (For example, if a Team violates the foregoing for the first time, it shall be fined \$3 million; if such Team violates the foregoing a second time, it shall be fined \$4.5 million; and if such Team violates the foregoing a third time, it shall be fined \$6 million.) Fifty percent (50\%) of any such fine amounts shall be remitted by the NBA to an NBPA-Selected Charitable Organization (as defined in Article VI, Section 6 above) and fifty percent (50\%) shall be remitted by the NBA to a Section 501(c)(3) Organization selected by the NBA.
  \item
    Neither the NBA or a League-related entity nor a Team or a Related Party will enter into any lease or other agreement providing for the receipt of revenues includable in BRI that contains provisions that purport to limit access of the Accountants to the books and records of the NBA, such League-related entity, such Team, or such Related Party in a manner inconsistent with the terms of this Agreement or that would preclude the calculation of revenues (if any) to be included in BRI pursuant to the provisions of Section 1(a)(1)(xii) above.
  \item
    Premium payments made by a Team for any insurance that, if paid, would be includable in BRI pursuant to Section 1(a)(2)(ix) above, shall be deducted from such Team's BRI for the Salary Cap Year in which any such insurance recovery is received.
  \item
    \emph{Equity Transactions.}

    \begin{enumerate}
    \def\labelenumiii{(\roman{enumiii})}
    \tightlist
    \item
      The value of equity securities received by NBA-related entities (as defined in Section 1(a)(2) above) in entities that were not NBA-related entities prior to such receipt, to the extent otherwise constituting BRI under this Agreement, shall be included in BRI as follows:

      \begin{enumerate}
      \def\labelenumiv{(\Alph{enumiv})}
      \tightlist
      \item
        if the equity securities (including contingent securities, as defined below) are Publicly Tradable when received, the Publicly Traded Value of those securities will be included in BRI commencing in the Salary Cap Year in which they are received;
      \item
        if the equity securities consist of options, warrants, convertible securities or similar securities (``contingent securities''), and (x) those contingent securities are sold, the Net Proceeds will be included in BRI commencing in the Salary Cap Year in which the sale occurs, or (y) those contingent securities are exercised or converted into other securities that are or become Publicly Tradable, the Publicly Traded Value of the resulting securities (net of any exercise or conversion price and taxes, as determined below) will be included in BRI commencing in the Salary Cap Year in which the exercise or conversion occurs (if the resulting securities were Publicly Tradable at that time) or in the Salary Cap Year in which the resulting securities later become Publicly Tradable, whichever is first;
      \item
        if the equity securities (including contingent securities and any securities resulting from the exercise or conversion of contingent securities) are not Publicly Tradable at the time of receipt (or, in the case of contingent securities, at the time of exercise or conversion), no BRI value shall be attributable to such securities until (x) they become Publicly Tradable or are sold or otherwise transferred for consideration other than securities that are not Publicly Tradable, whichever is first, at which time the Publicly Traded Value or Net Proceeds, as applicable, will be included in BRI commencing in the Salary Cap Year in which such event occurs or (y) they are specifically pledged and valued as part of a transaction that provides liquidity without selling the equity position, even if other assets are also pledged as a part of such transaction, in which case BRI will include an amount equal to the specific value assigned to the equity securities (``Specifically Assigned Value'') commencing in the Salary Cap Year in which they are specifically pledged and valued; or
      \item
        notwithstanding the foregoing, if any contingent securities are exercisable or convertible into securities that are or become Publicly Tradable, but those contingent securities are not exercised or converted within one (1) year of any such right, the Players Association shall have the right, by written notice to the NBA, to have the Publicly Traded Value of such securities included in BRI as if those contingent securities had been exercised or converted on the date of such notice (net of any exercise or conversion price and taxes, as determined below).
      \end{enumerate}
    \item
      For purposes of this Section 1(a)(13), (A) ``Publicly Tradable'' means (x) the applicable equity securities have been registered for sale under applicable state, federal and foreign laws, are listed and tradable on a generally recognized stock exchange or in the over-the-counter market, or (y) the applicable equity securities can be readily purchased and sold on a nationally recognized secondary market (e.g., without limitation on any example, shares in ``Facebook'' as of the date of the 2011 CBA), and in each case under (x) and (y), any contractual or other prohibition or limitation on sale would not preclude a sale; (B) ``Publicly Traded Value'' means the weighted average daily trading price of the applicable equity securities for the thirty (30) trading days (x) preceding the date of receipt if the securities are Publicly Tradable prior to that date or (y) following the date they become Publicly Tradable; provided that if such equity securities are sold during the Salary Cap Year in which their Publicly Traded Value is first included in BRI, the ``Publicly Traded Value'' of such equity securities shall be the Net Proceeds from such sale; (C) ``Net Proceeds'' means the proceeds received by the selling entity from the applicable sale, net of commissions and reasonable expenses relating to such sale, any exercise or conversion price with respect to securities resulting from the exercise or conversion of contingent securities, and any applicable taxes of the selling entity (or if the selling entity is a pass-through entity for income tax purposes, such entity's owners), which shall be determined using a tax rate equivalent to the highest marginal combined federal, state and local tax rate that would be applicable in the locale where the principal place of business of the selling entity is located, which in the case of a Team or a Related Party of a Team shall be deemed to be the locale of the arena in which the Team plays more than one-half of its Regular Season home games; and (D) a sale of equity securities shall not be subject to inclusion in BRI if the sale is part of a larger transaction in which (x) BRI has been fully accounted for or (y) all or substantially all of the assets of an NBA-related entity or business unit thereof are sold and such equity securities do not represent a majority of the value in such transaction. In all cases, the Publicly Traded Value of, Net Proceeds from, or Specifically Assigned Value of the applicable equity securities will be included in BRI over a five (5) year amortization period (inclusive of the Salary Cap Year in which such Publicly Traded or Specifically Assigned Value is first included in BRI), even if such equity securities are sold during such five (5) year period.
    \item
      For the avoidance of doubt, (A) in no event shall the value of, or proceeds or distributions from, equity securities in NBA-related entities be included in BRI, and (B) the value of, or proceeds or distributions from, equity securities in non-NBA-related entities shall be included in BRI exclusively pursuant to this Section 1(a)(13), and only once under the applicable provision of Section 1(a)(13)(i) above.
    \end{enumerate}
  \item
    \emph{International Development and Operations Expenses.} The NBA and League-related entities may deduct from BRI expenses related to the development and operation of the League's international business (``Newly-Deductible International Expenses''), subject to a limit of ten percent (10\%) of the League's gross BRI international revenues (the allowed amount of such expense following application of the ten percent (10\%) cap being the ``Allowed Newly-Deductible International Expenses''). Newly-Deductible International Expenses for any Salary Cap Year shall include all such international expenses incurred at the League level that are not otherwise deductible under this Agreement (excluding expenses in currently-deductible categories that are in excess of applicable percentage-of-revenue expense caps and the write-down of equity investments). For the purposes of this Agreement, the League's gross BRI international revenue and Newly-Deductible International Expenses for the 2021-22 Salary Cap Year were the amounts set forth in the parties' letter agreement dated June 28, 2023.
  \item
    \emph{Miscellaneous BRI Accounting Rules.}

    \begin{enumerate}
    \def\labelenumiii{(\roman{enumiii})}
    \tightlist
    \item
      Team charter travel expenses for Regular Season games, associated with broadcast and other personnel for whom such travel expenses are otherwise deductible (for example, without limitation on any other example, Team personnel traveling in connection with the sale of Team sponsorships), shall be deductible at fifty percent (50\%).
    \item
      BRI for premium seating, in respect of: (A) bunker, super, and party suites, (B) theatre boxes, loge boxes, and other such non-traditional premium seating inventory, and (C) traditional club seats, shall be calculated by using the parties' previously-agreed upon methods, as reflected in the Audit Report for the 2015-16 Salary Cap Year.
    \item
      To the extent salary paid to a person who also owns an interest in the Team would otherwise be deductible from BRI, such salary shall only be deductible for BRI purposes only if all of the following criteria are met and if the expense otherwise qualifies for such deduction (for example, without limitation, the salary is related to the BRI against which it is deducted): (A) the owner owns less than seven and one-half percent (7.5\%) of the Team; (B) the job being performed by the owner would otherwise be performed by a non-owner staff member; (C) the job being performed by the owner is his/her full time job and he/she has no other roles with outside companies (with the exception of limited duty Board roles); (D) the salary being earned is reasonable and customary, relative to what a non-owner staff member would earn, for the services being provided; and (E) there are no other individuals performing substantially the same role employed by the Team where the role is such that ordinarily there is only one person performing it (for example, without limitation, Team president).
    \item
      With respect to expenses associated with League-related entity advertising and public relations campaigns: (A) the expenses will be allocated to BRI and non-BRI revenue categories according to the methodology agreed to by the parties in connection with the final Audit Report for the 2015-16 Salary Cap Year, except that such expenses shall also be allocated to the NBA's Regular Season gate assessment during the Salary Cap Year in addition to the other categories previously included in the allocation; and (B) the expenses shall be deducted from BRI, subject to Section 1(a)(6)(ii) above.
    \item
      Revenues and expenses related to the new In-Season Tournament (``IST'') shall be included in BRI as follows: (A) revenues and expenses from home-market Group Stage games and IST Quarterfinals games not played in a neutral market will be treated as if earned or incurred during a Regular Season game; (B) notwithstanding anything to the contrary in this Agreement, including, without limitation, Article XX, Section 4(c), for purposes of Article VII, Section 1(a), revenues and expenses from neutral-market games will be treated as league ``Special Events,'' subject to the 100\% expense cap for Special Events, except that, during the 2023-24 and 2024-25 Salary Cap Years only, 50\% of the expenses related to any neutral-market IST game (other than the IST Finals Game, and excluding any payments from the NBA to Teams described in the succeeding subpart) will be treated as Team expenses subject to Section (1)(a)(6)(v) above; and (C) payments from the NBA to Teams for participation in neutral-market IST games will be included in miscellaneous BRI at the Team level, with the NBA recording an expense relating to such payment at the League level in Special Events.
    \item
      Barter otherwise includable in BRI will not be excluded solely on the basis that the bartered-for goods or services were not used. For the avoidance of doubt, expenses associated with the receipt and use of bartered goods or services will be deductible from BRI subject to rules applicable to other expenses (e.g., that a barter expense is deductible only where it relates to a category that allows for deduction of expenses).
    \end{enumerate}
  \end{enumerate}
\item
  \textbf{Accounting Methods/Lump Sum Payments.}

  \begin{enumerate}
  \def\labelenumii{(\arabic{enumii})}
  \tightlist
  \item
    Subject to Sections 1(b)(2) and (b)(3) below, and any provision hereof that expressly provides for an alternative accounting treatment, BRI for each Salary Cap Year shall be calculated exclusively pursuant to the accrual method of financial accounting (and not, for any purpose, the cash method of financial accounting) and in accordance with United States Generally Accepted Accounting Principles. By way of example, and not limitation, in the event a team receives a signing bonus in consideration for its agreement to enter into a five (5) year contract for the local telecast of its games, such signing bonus shall be amortized in equal annual amounts over the five (5) Salary Cap Years covered by such television contract.
  \item
    Except as otherwise provided in the case of luxury suites and premium seat licenses, in no event shall the amortization period for any lump sum payment exceed seven (7) years.
  \item
    Any payments that constitute BRI and that are subject to being repaid to the payor under certain circumstances (the ``Contingencies'') shall constitute BRI in the Salary Cap Year in which such payments would have been earned but for the Contingencies unless, at the time of such payments, the Contingencies under which the payments would be repaid are likely to occur, in which case the payments will not be included in BRI unless and until such time as the Contingencies under which such repayments would be made do not occur or are not likely to occur. In the event that a payment that has been included in BRI is subsequently repaid, BRI shall be reduced by the amount of such repayment in the Salary Cap Year in which such repayment is made. In any proceeding commenced before the System Arbitrator relating to the terms of this Section 1(b)(3), the NBA will bear the burden of demonstrating that the applicable Contingencies are likely to occur.
  \item
    With respect to lump sum payments (e.g., signing bonuses) that constitute BRI and are received by a Team or Related Party under agreements entered into by a Team or Related Party, for the period, if any, between (x) the date when the lump sum payment is received and (y) the beginning of the Salary Cap Year when the amortization period for the lump sum payment begins pursuant to Section 1(b)(1) above, BRI shall include in each applicable Salary Cap Year during this period imputed interest on the amount of such lump sum payment at a rate equal to the one (1) year Treasury Bill rate as published in The Wall Street Journal on the date the payment was received; provided, however, that such imputed interest shall only be calculated and included in BRI if each of the following is satisfied: (i) the lump sum payment received by the Team or Related Party is in the amount of one million dollars (\$1,000,000) or more; (ii) the lump sum payment is received by the Team or Related Party at least twelve (12) months before the start of the Salary Cap Year in which it will first be included in BRI under Section 1(b)(1) above; and (iii) the lump sum payment is not related to a ticket, luxury suite or seat licensing transaction including, without limitation, revenues included in BRI under Sections 1(a)(1)(i), 1(a)(1)(vii), and 1(a)(1)(x) above.
  \item
    Loan proceeds from contractual counterparties that would be included in BRI if received as payment (rather than a loan) will be included in BRI if there is no realistic expectation of repayment, subject to application of the lump sum amortization rules in this Section 1(b) and allocations for multi-element deals. If amounts in respect of such a loan are included in BRI, but the loan is subsequently repaid, BRI shall be reduced by the amount of such repayment in the Salary Cap Year in which such repayment is made.
  \end{enumerate}
\item
  ``Projected BRI'' for a Salary Cap Year means the amount determined as follows: Prior to the start of each Salary Cap Year, the NBA and the Players Association shall meet for the purpose of agreeing upon Projected BRI for that Salary Cap Year. In the absence of an agreement of the parties otherwise on or prior to the last day of the Moratorium Period of the applicable Salary Cap Year, Projected BRI for such Salary Cap Year shall be the sum of amounts determined in accordance with the following:

  \begin{enumerate}
  \def\labelenumii{(\arabic{enumii})}
  \tightlist
  \item
    With respect to BRI sources other than national broadcast, national telecast or network cable television contracts, Projected BRI shall include BRI for the preceding Salary Cap Year, increased by four and one-half percent (4.5\%). For purposes of this Section 1(c)(1), a contract between or among any League-related entities and/or Teams shall not be considered national broadcast, national telecast or network cable television contracts.
  \item
    With respect to national broadcast, national telecast or network cable television contracts including the NBA/ABC agreement dated October 3, 2014 (``NBA/ABC Agreement'') (a copy of which has been provided to the Players Association) and the NBA/TBS agreement, dated October 3, 2014 (``NBA/TBS Agreement'') (a copy of which has been provided to the Players Association), and national broadcast, national telecast or network cable television contracts covering Seasons that succeed the Seasons covered by the NBA/ABC and NBA/TBS Agreements (``Successor Agreements'') (copies of which shall be provided to the Players Association within ten (10) days of execution), Projected BRI for a Salary Cap Year shall include (i) the rights fees or other non-contingent payments stated in such contracts with respect to the Season covered by such Salary Cap Year (as such rights fees or non-contingent payments may be adjusted by agreement of the parties to such contracts); (ii) the amounts of revenue sharing proceeds, if any, that are includable in BRI for such Salary Cap Year pursuant to Section 1(a)(8) above; (iii) the amounts with respect to contingent payments (other than revenue sharing proceeds), if any, attributable to Salary Cap Years covered by this Agreement in Successor Agreements as such amounts are agreed upon by the parties, or if the parties do not reach agreement, by the Accountants; and (iv) the amount included in BRI for the preceding Salary Cap Year with respect to the value of advertising or promotional time provided to the NBA as part of the NBA/ABC and NBA/TBS Agreements (or any Successor Agreements) that is used for any purpose other than those listed in Sections 1(a)(1)(ii)(A)-(D).
  \item
    In no event shall the same amounts be included in Projected BRI or Interim Projected BRI, directly or indirectly, more than once (including in the event that the terms of any Successor Agreements would cause the same amounts to be indirectly included in Projected BRI or Interim Projected BRI for a Salary Cap Year pursuant to both subsections (1) and (2) above), the purpose of this provision being to preclude the double-counting of amounts in the calculation of Projected BRI or Interim Projected BRI, whether in the same or in multiple Salary Cap Years.
  \end{enumerate}
\item
  ``Local Expansion Team BRI'' means the BRI of the Expansion Teams during their first two (2) Seasons, but not including the Expansion Teams' share of League-wide revenues that are otherwise included in BRI (including, but not limited to, their share of national television, cable, radio and other broadcast revenues).
\item
  ``Projected Local Expansion Team BRI'' means Local Expansion Team BRI for the immediately preceding Season, increased by four and one-half percent (4.5\%).
\item
  ``Interim Projected BRI'' means a projection of BRI for a Salary Cap Year using Estimated BRI in place of BRI for the previous Salary Cap Year.
\item
  ``Barter'' means to trade by exchanging one commodity, service or other non-cash item for another.
\item
  ``Estimated Total Benefits'' means the estimate of Total Benefits for a Salary Cap Year as set forth in the Interim Audit Report (as defined in Section 10(a) below) for such Salary Cap Year.
\item
  ``Estimated Total Salaries'' means the estimate of Total Salaries for a Salary Cap Year as set forth in the Interim Audit Report for such Salary Cap Year.
\item
  ``Estimated Total Salaries and Benefits'' means the sum of Estimated Total Benefits and Estimated Total Salaries for a Salary Cap Year as set forth in the Interim Audit Report for such Salary Cap Year.
\item
  ``Estimated BRI'' means the estimate of BRI for a Salary Cap Year as set forth in the Interim Audit Report for such Salary Cap Year.
\end{enumerate}

\hypertarget{salary-cap-minimum-team-salary-tax-level-apron-levels-and-draft-pick-penalty.}{%
\section{Salary Cap, Minimum Team Salary, Tax Level, Apron Levels, and Draft Pick Penalty.}\label{salary-cap-minimum-team-salary-tax-level-apron-levels-and-draft-pick-penalty.}}

\begin{enumerate}
\def\labelenumi{(\alph{enumi})}
\item
  \textbf{Calculation of Salary Cap, Minimum Team Salary, Tax Level, and Apron Levels.}

  \begin{enumerate}
  \def\labelenumii{(\arabic{enumii})}
  \item
    For each Salary Cap Year during the term of this Agreement, there shall be a Salary Cap. The Salary Cap for each Salary Cap Year covered by the Term of this Agreement will equal forty-four and seventy-four one hundredths percent (44.74\%) of Projected BRI for such Salary Cap Year, less Projected Benefits for such Salary Cap Year, divided by the number of Teams scheduled to play in the NBA during such Salary Cap Year, other than Expansion Teams during their first two (2) Salary Cap Years in the NBA.
  \item
    Notwithstanding Section 2(a)(1) above, in the event that Projected BRI for any Salary Cap Year in which one or more Expansion Teams is scheduled to play its second Season, plus Projected Local Expansion Team BRI for such Salary Cap Year, multiplied by the applicable percentage of Projected BRI set forth in Section 2(a)(1) above, less Projected Benefits for such Salary Cap Year (including for the Expansion Team(s)), divided by the number of Teams scheduled to play in the NBA during such Salary Cap Year (including the Expansion Team(s)), exceeds the Salary Cap calculated in accordance with Section 2(a)(1) above, the Salary Cap shall equal the amount calculated pursuant to this Section 2(a)(2).
  \item
    In the event that the Salary Cap for a Salary Cap Year is calculated based upon an Interim Audit Report for the prior Salary Cap Year in accordance with Section 2(a)(7) below and BRI and Total Salaries and Benefits as set forth in the Audit Report for the prior Salary Cap Year are different from those in the Interim Audit Report such that the Salary Cap would have been different from that based upon the Interim Audit Report, any such difference in the Salary Cap shall be debited or credited, as the case may be, to the Salary Cap for the subsequent Salary Cap Year, except that, with respect to the 2029-30 Salary Cap Year (or, in the alternative, if either the NBA or Players Association exercises its option to terminate this Agreement pursuant to Article XXXIX, the 2028-29 Salary Cap Year) any such differences shall be debited or credited, as the case may be, to the Salary Cap for the then current Salary Cap Year, in all such cases with interest (at a rate equal to the one (1) year Treasury Bill rate as published in The Wall Street Journal on the date of the issuance of the Interim Audit Report).
  \item
    For each Salary Cap Year covered by the term of this Agreement, there shall be:

    \begin{enumerate}
    \def\labelenumiii{(\roman{enumiii})}
    \item
      a Minimum Team Salary equal to ninety percent (90\%) of the Salary Cap for such Salary Cap Year;
    \item
      a ``Tax Level'' equal to one hundred twenty-one and one-half percent (121.5\%) of the Salary Cap for such Salary Cap Year; (iii) a ``First Apron Level'' and a ``Second Apron Level'' as follows:

      \begin{enumerate}
      \def\labelenumiv{(\Alph{enumiv})}
      \tightlist
      \item
        For the 2023-24 Salary Cap Year, the First Apron Level shall equal the sum of: (1) the Tax Level for the 2023-24 Salary Cap Year, and (2) \$6.716 million multiplied by a fraction, the numerator of which is the average of the Salary Cap for the 2022-23 Salary Cap Year and the Salary Cap for the 2023-24 Salary Cap Year, and the denominator of which is the Salary Cap for the 2022-23 Salary Cap Year. For each subsequent Salary Cap Year, the First Apron Level shall equal the First Apron Level for the 2023-24 Salary Cap Year multiplied by a fraction, the numerator of which is the Salary Cap for the applicable Salary Cap Year and the denominator of which is the Salary Cap for the 2023-24 Salary Cap Year.
      \item
        For the 2023-24 Salary Cap Year, the Second Apron Level shall equal the sum of: (1) the Tax Level for the 2023-24 Salary Cap Year, and (2) \$17.5 million. For each subsequent Salary Cap Year, the Second Apron Level shall equal the Second Apron Level for the 2023-24 Salary Cap Year multiplied by a fraction, the numerator of which is the Salary Cap for the applicable Salary Cap Year and the denominator of which is the Salary Cap for the 2023-24 Salary Cap Year.
      \end{enumerate}

      For clarity, for purposes of the foregoing calculations, the Salary Cap shall be the Salary Cap as calculated in accordance with Sections 2(a)(1)-(3) above without regard to Sections 2(a)(5) and (6) below.

      \emph{For example, assume the Salary Cap for the 2023-24 Salary Cap Year is \$134 million, for the 2024-25 Salary Cap Year is \$140.7 million, and for the 2025-26 Salary Cap Year is \$147.735 million (and that no adjustments pursuant to Section 2(a)(5) below are required in any of those years).}

      \emph{For the 2023-24 Salary Cap Year:}

      \begin{itemize}
      \tightlist
      \item
        \emph{The Tax Level would be \$162.81 million (i.e., \$134 million (the 2023-24 Salary Cap) multiplied by 121.5\%);}
      \item
        \emph{The First Apron Level would be \$169.807 million (i.e., \$162.81 million (the 2023-24 Tax Level) plus an amount equal to \$6.716 million multiplied by a fraction, the numerator of which is \$128.828 million (the average of the 2022-23 Salary Cap of \$123.655 million and the 2023-24 Salary Cap of \$134 million), and the denominator of which is \$123.655 million (the 2022-23 Salary Cap)); and}
      \item
        \emph{The Second Apron Level would be \$180.31 million (i.e., \$162.81 million (the 2023-24 Tax Level) plus \$17.5 million).}
      \end{itemize}

      \emph{For the 2024-25 Salary Cap Year:}

      \begin{itemize}
      \tightlist
      \item
        \emph{The Tax Level would be \$170.951 million (i.e., \$140.7 million multiplied by 121.5\%);}
      \item
        \emph{The First Apron Level would be \$178.297 million (i.e., \$169.807 million (the 2023-24 First Apron Level) multiplied by a fraction, the numerator of which is \$140.7 million (the 2024-25 Salary Cap) and the denominator of which is \$134 million (the 2023-24 Salary Cap)); and}
      \item
        \emph{The Second Apron Level would be \$189.326 million (i.e., \$180.31 million (the 2023-24 Second Apron Level) multiplied by a fraction, the numerator of which is \$140.7 million (the 2024-25 Salary Cap) and the denominator of which is \$134 million (the 2023-24 Salary Cap)).}
      \end{itemize}

      \emph{For the 2025-26 Salary Cap Year:}

      \begin{itemize}
      \tightlist
      \item
        \emph{The Tax Level would be \$179.498 million (i.e., \$147.735 million multiplied by 121.5\%);}
      \item
        \emph{The First Apron Level would be \$187.212 million (i.e., \$169.807 million (the 2023-24 First Apron Level) multiplied by a fraction, the numerator of which is \$147.735 million (the 2025-26 Salary Cap) and the denominator of which is \$134 million (the 2023-24 Salary Cap)); and}
      \item
        \emph{The Second Apron Level would be \$198.792 million (i.e., \$180.31 million (the 2023-24 Second Apron Level) multiplied by a fraction, the numerator of which is \$147.735 (the 2025-26 Salary Cap) and the denominator of which is \$134 million (the 2023-24 Salary Cap)).}
      \end{itemize}
    \end{enumerate}
  \item
    \begin{enumerate}
    \def\labelenumiii{(\roman{enumiii})}
    \item
      For each Salary Cap Year beginning with the 2023-24 Salary Cap Year, in the event that (A) there is a Shortfall Amount (as defined in Section 12(a)(21) below) for such Salary Cap Year, and (B) the Carryover Amount (as defined in Section 12(a)(12) below) in respect of the subsequent Salary Cap Year is equal to zero (0), then the Salary Cap, Minimum Team Salary, Tax Level, First Apron Level, and Second Apron Level for such subsequent Salary Cap Year (as calculated in accordance with Sections 2(a)(1)-(4) above) shall each be increased by an amount equal to the Shortfall Amount divided by the number of Teams in the NBA during such subsequent Salary Cap Year (other than Expansion Teams in their first two (2) Salary Cap Years in the NBA).
    \item
      For each Salary Cap Year beginning with the 2023-24 Salary Cap Year, in the event that there is an Overage Amount (as defined in Section 12(a)(20) below) for such Salary Cap Year that exceeds six percent (6\%) of Total Salaries and Benefits, then the Salary Cap, Minimum Team Salary, Tax Level, First Apron Level, and Second Apron Level for the subsequent Salary Cap Year (as calculated in accordance with Sections 2(a)(1)-(4) above) shall each be reduced by an amount calculated as follows:

      \begin{itemize}
      \item
        STEP 1: Subtract six percent (6\%) of Total Salaries and Benefits from the Overage Amount.
      \item
        STEP 2: If Projected BRI for the subsequent Salary Cap Year does not exceed BRI for the Salary Cap Year by more than eight percent (8\%) of BRI for the Salary Cap Year or the Overage Amount described above exceeds nine percent (9\%) of Total Salaries and Benefits, then divide the result of Step 1 by the number of Teams in the NBA during the subsequent Salary Cap Year (other than Expansion Teams in their first two (2) Salary Cap Years in the NBA). The result of this calculation is the amount of the reduction in each of the Salary Cap, Minimum Team Salary, Tax Level, First Apron Level, and Second Apron Level for such subsequent Salary Cap Year, and no further steps are required.

        If Projected BRI for the subsequent Salary Cap Year exceeds one hundred eight percent (108\%) of BRI for the Salary Cap Year and the Overage Amount described above does not exceed nine percent (9\%) of Total Salaries and Benefits, then proceed to Step 3.
      \item
        STEP 3: Subtract one hundred eight percent (108\%) of BRI for the Salary Cap Year from Projected BRI for the subsequent Salary Cap Year.
      \item
        STEP 4: Multiply the result of Step 3 by fifty percent (50\%).
      \item
        STEP 5: Subtract the result of Step 4 from the result of Step 1. If the result of this step is less than zero (0), then no adjustments shall be made to the Salary Cap, Minimum Team Salary, Tax Level, First Apron Level, or Second Apron Level for the subsequent Salary Cap Year, and no further steps are required.
      \item
        STEP 6: Divide the result of Step 5 by the number of Teams in the NBA during such subsequent Salary Cap Year (other than Expansion Teams in their first two (2) Salary Cap Years in the NBA). The result of this calculation is the amount of the reduction in each of the Salary Cap, Minimum Team Salary, Tax Level, First Apron Level, and Second Apron Level for such subsequent Salary Cap Year.
      \end{itemize}
    \end{enumerate}

    \emph{Example: Assume: (i) 2024-25 Total Salaries and Benefits is \$5.5 billion, the 2024-25 Designated Share is \$5.1 billion, and the resulting 2024-25 Overage Amount is \$400 million (which equals approximately 7.3\% of Total Salaries and Benefits); (ii) 2025-26 Projected BRI is \$10.5 billion, 2024-25 BRI is \$10 billion, and thus 2025-26 Projected BRI exceeds 2024-25 BRI by 5\%; and (iii) there are 30 Teams in the NBA in the 2025-26 Season. The 2025-26 Salary Cap, Minimum Team Salary, Tax Level, First Apron Level, and Second Apron Level would each be reduced by \$2,333,333 (i.e., \$70 million (i.e., 2024-25 Overage Amount of \$400 million less \$330 million (i.e., 6\% of 2024-25 Total Salaries and Benefits) divided by 30 (i.e., the number of Teams in the NBA during the 2025-26 Season)).}

    \emph{Example: Same assumptions as in the prior example, except assume 2025-26 Projected BRI is \$10.9 billion (instead of \$10.5 billion), and thus 2025-26 Projected BRI exceeds 2024-25 BRI by 9\%. The 2025-26 Salary Cap, Minimum Team Salary, Tax Level, First Apron Level, and Second Apron Level would each be reduced by \$666,667 (i.e., the difference between the \$70 million from the prior example and \$50 million (i.e., 50\% of \$100 million (i.e., 2025-26 Projected BRI of \$10.9 billion less \$10.8 billion (i.e., 108\% of 2024-25 BRI of \$10 billion)) divided by 30 (i.e., the number of Teams in the NBA during the 2025-26 Season))).}
  \item
    Notwithstanding anything to the contrary in Sections 2(a)(1)-(5) above, in no event shall any of the Salary Cap, Minimum Team Salary, Tax Level, First Apron Level, or Second Apron Level for a Salary Cap Year (a) decrease to an amount that is less than its amount for the immediately preceding Salary Cap Year, or (b) increase to an amount that exceeds one hundred ten percent (110\%) of its amount for the immediately preceding Salary Cap Year.
  \item
    In the event that the Audit Report for a Salary Cap Year has not been completed as of the last day of such Salary Cap Year, and the NBA and the Players Association have not reached an agreement on Projected BRI and Projected Benefits pursuant to Article VII, Section 1 and Article IV, Section 8 for the immediately following Salary Cap Year, then the Salary Cap, Minimum Team Salary, Tax Level, First Apron Level, and Second Apron Level for the immediately following Salary Cap Year will be calculated pursuant to Sections 2(a)(1)-(6) above, except that Interim Projected BRI shall be utilized instead of Projected BRI, Estimated BRI shall be utilized instead of BRI, and Estimated Total Salaries and Benefits shall be utilized instead of Total Salaries and Benefits. In the event that the Interim Audit Report for a Salary Cap Year has not been completed as of the last day of such Salary Cap Year, and the NBA and Players Association have not reached agreement on Projected BRI and Projected Benefits pursuant to Article VII, Section 1 and Article IV, Section 8, then the Salary Cap for the immediately following Salary Cap Year shall, until such Interim Audit Report is completed, be an amount that would have been the Salary Cap for the preceding Salary Cap Year had Projected BRI or Interim Projected BRI, as the case may be, for such preceding Salary Cap Year included, with respect to the NBA's national broadcast, national telecast, or network cable television contracts, the rights fees or other non-contingent payments stated in such contracts for the Season following the Season covered by such preceding Salary Cap Year instead of for the Season covered by such preceding Salary Cap Year.
  \item
    The Salary Cap, Minimum Team Salary, Tax Level, First Apron Level, and Second Apron Level for a Salary Cap Year will be in effect commencing on the first day of the Salary Cap Year and shall continue through and including the last day of the Salary Cap Year.
  \end{enumerate}
\item
  \textbf{Operation of Salary Cap.}

  \begin{enumerate}
  \def\labelenumii{(\arabic{enumii})}
  \tightlist
  \item
    \textbf{Basic Rule.} A Team's Team Salary may not exceed the Salary Cap at any time unless the Team is using one of the Exceptions set forth in Section 6 below.
  \item
    \textbf{Room.} Subject to the other provisions of this Agreement, including, without limitation, Article II, Section 7, any Team with Room may enter into a Player Contract that calls for a Salary in the first Salary Cap Year covered by such Contract that would not exceed the Team's then-current Room.
  \end{enumerate}
\item
  \textbf{Operation of Minimum Team Salary.}

  \begin{enumerate}
  \def\labelenumii{(\arabic{enumii})}
  \tightlist
  \item
    As used in this Agreement, the following terms shall have the following meanings:

    \begin{enumerate}
    \def\labelenumiii{(\roman{enumiii})}
    \tightlist
    \item
      ``MTS Cap Hold Team Salary'' means, for a Team for a Salary Cap Year, the Team's Team Salary calculated in the same manner as Team Salary is calculated by the Accountants for purposes of computing Total Salaries and Benefits in the Audit Report (as defined in Section 10(a)(1) below).
    \item
      ``MTS Payment Team Salary'' means, for a Team for a Salary Cap Year, the Team's MTS Cap Hold Team Salary as of the start of the first day of the Regular Season occurring within such Salary Cap Year:

      \begin{enumerate}
      \def\labelenumiv{(\Alph{enumiv})}
      \tightlist
      \item
        plus any Salary in respect of such Salary Cap Year that is excluded from the Team's Team Salary pursuant to Section 4(h) below;
      \item
        minus any Salary in respect of such Salary Cap Year that is included in the Team's Team Salary pursuant to Section 3(e) below; and
      \item
        plus any Salary in respect of such Salary Cap Year that is excluded from the Team's Team Salary pursuant to Section 4(b) below.
      \end{enumerate}
    \item
      ``MTS Threshold'' means, for a Team for a Salary Cap Year, the lesser of (A) the Minimum Team Salary for such Salary Cap Year, and (B) such Team's MTS Cap Hold Team Salary as of the start of the first day of the Regular Season occurring within such Salary Cap Year.
    \end{enumerate}
  \item
    In the event that a Team's MTS Payment Team Salary for a Salary Cap Year is less than the applicable Minimum Team Salary for that Salary Cap Year, then:

    \begin{enumerate}
    \def\labelenumiii{(\roman{enumiii})}
    \tightlist
    \item
      The NBA shall cause such Team to make a payment to the NBA equal to the difference between the Team's MTS Payment Team Salary and the Minimum Team Salary; and
    \item
      Subject to Section 2(c)(7) below, and notwithstanding anything to the contrary in Section 2(d) below, such Team shall be prohibited from receiving a share of any tax amount that the NBA elects to distribute to non-taxpaying Teams in respect of such Salary Cap Year pursuant to Section 2(d)(4)(i) below.
    \end{enumerate}
  \item
    Beginning at the start of the first day of the Regular Season and continuing through the end of the Salary Cap Year encompassing such Regular Season, a Team's Team Salary shall include an amount equal to the amount (if any) by which the Minimum Team Salary exceeds the lesser of such Team's (i) then-current MTS Cap Hold Team Salary, and (ii) MTS Cap Hold Team Salary as of the start of the first day of the Regular Season.
  \item
    If, on a day during the Regular Season, a Team's MTS Cap Hold Team Salary decreases to an amount that is less than its MTS Threshold, then the Team will be required to increase its MTS Cap Hold Team Salary to an amount equal to or greater than its MTS Threshold by the end of the immediately following day.
  \item
    If, as of the end of the last day of a Salary Cap Year, the Minimum Team Salary for such Salary Cap Year exceeds the portion of total MTS Cap Hold Team Salaries for which a Team is financially responsible plus any payment that the Team is required to make pursuant to Section 2(c)(2)(i) above, then, in addition to any payment required pursuant to Section 2(c)(2)(i) above, such Team shall make a payment to the NBA equal to the amount of such excess. For the purposes of this Section 2(c)(5), MTS Cap Hold Team Salaries shall:

    \begin{enumerate}
    \def\labelenumiii{(\roman{enumiii})}
    \tightlist
    \item
      include any Incentive Compensation excluded from Salaries in accordance with Article VII, Section 3(d) but actually earned by players during such Salary Cap Year; and
    \item
      exclude all Incentive Compensation included in Salaries in accordance with Article VII, Section 3(d) but not actually earned by players during such Salary Cap Year.
    \end{enumerate}
  \item
    Any payment due by a Team in respect of a Salary Cap Year pursuant to Section 2(c)(2)(i) or 2(c)(5) above shall be made by the Team to the NBA no later than ten (10) business days following the completion of the Governing Audit Report for such Salary Cap Year. The NBA shall then distribute any such payments equally to each Team within ten (10) business days following its receipt of such payments.
  \item
    Notwithstanding Section 2(c)(2)(ii) above, for the 2023-24 Salary Cap Year only, if (i) a Team's MTS Payment Team Salary for the 2023-24 Salary Cap Year is less than the Minimum Team Salary for such Salary Cap Year, and (ii) such Team does not owe a Tax for such Salary Cap Year, then such Team shall be entitled to receive a fifty percent (50\%) share of any tax amount that the NBA elects to distribute to non-taxpaying Teams in respect of such Salary Cap Year pursuant to Section 2(d)(4)(i) below. For example, if there were twenty-four (24) non-taxpaying Teams for the 2023-24 Salary Cap Year and one (1) of such Teams had a MTS Payment Team Salary for such Salary Cap Year less than the Minimum Team Salary, then such Team, rather than receiving one-twenty-fourth (1/24th) of the total amount that the NBA elects to distribute to non-taxpaying Teams pursuant to Section 2(d)(4)(i) below, would instead receive a Tax distribution amount equal to the total amount to be distributed to non-taxpaying Teams multiplied by a fraction, the numerator of which is one-half (0.5) and the denominator of which is twenty-three and one-half (0.5/23.5). Each of the other twenty-three (23) non-taxpaying Teams for the 2023-24 Salary Cap Year would receive a Tax distribution amount equal to the total amount to be distributed to non-taxpaying Teams multiplied by a fraction, the numerator of which is one (1) and the denominator of which is twenty-three and one-half (1/23.5).
  \item
    Nothing contained herein shall preclude a Team from having a Team Salary in excess of the Minimum Team Salary, provided that the Team's Team Salary does not exceed the Salary Cap plus any additional amounts authorized pursuant to the Exceptions set forth in this Article VII.
  \end{enumerate}
\item
  \textbf{Operation of Tax Level.}

  \begin{enumerate}
  \def\labelenumii{(\arabic{enumii})}
  \item
    As used in this Agreement, the following terms shall have the following meanings:

    \begin{enumerate}
    \def\labelenumiii{(\roman{enumiii})}
    \tightlist
    \item
      ``Tax Team Salary'' means, for a Team for a Salary Cap Year, the Team's Team Salary as of the start of its last Regular Season game occurring within such Salary Cap Year, calculated by the Accountants in the same manner as Team Salary is calculated by the Accountants for purposes of computing Total Salaries and Benefits in the Audit Report:

      \begin{enumerate}
      \def\labelenumiv{(\Alph{enumiv})}
      \tightlist
      \item
        plus all Incentive Compensation excluded from Salary under Section 3(d) below but actually earned by the player during such Salary Cap Year;
      \item
        minus all Incentive Compensation included in Salary under Section 3(d) below but not actually earned by the player during such Salary Cap Year;
      \item
        plus, with respect to any trade that occurs following the conclusion of the Team's last Regular Season game, the portion of any trade bonus earned by a player that is included in the Team's Team Salary for such Salary Cap Year;
      \item
        plus any amount that is added to the Team's Team Salary for such Salary Cap Year following the start of the Team's last Regular Season game pursuant to Section 4(a)(1)(iii) below;
      \item
        minus fifty percent (50\%) of any reduction made to a player's Compensation as a result of a suspension by the NBA (but not by a Team); and
      \item
        plus, with respect to a Standard NBA Contract between a Team and a Free Agent with zero (0) Years of Service or one (1) Year of Service, the amount (if any) by which (x) the Minimum Player Salary that would be applicable to a player with two (2) Years of Service as set forth in the Minimum Annual Salary Scale for the Salary Cap Year in which such Free Agent was signed (or in the event such Free Agent's Contract is terminated during the Regular Season, the Minimum Player Salary that would be applicable to a player with two (2) Years of Service as set forth in the Minimum Annual Salary Scale for the Salary Cap Year in which such Free Agent was signed, reduced pro rata to reflect the player's post-termination Salary), exceeds (y) the Salary attributable to such Standard NBA Contract. For the purposes of this Section 2(d)(1)(i)(F):

        \begin{enumerate}
        \def\labelenumv{(\arabic{enumv})}
        \tightlist
        \item
          a Standard NBA Contract between a Team and a Two-Way Player (either signed pursuant to Article II, Section 11(h), or the result of the exercise of a Standard NBA Contract Conversion Option) will be deemed to be a ``Standard NBA Contract between a Team and a Free Agent'' provided that such Two-Way Player's Two-Way Contract was: (i) signed by the player as a Free Agent; or (ii) the result of the exercise of the Two-Way Player Conversion Option provided for in the Exhibit 10 of a Contract that he signed as a Free Agent; and
        \item
          a Standard NBA Contract between a Team and a player under a 10-Day Contract (signed pursuant to Article II, Section 9(g)) will be deemed to be a Standard NBA Contract between a Team and a Free Agent provided that such 10-Day Contract was signed by the player as a Free Agent.
        \end{enumerate}
      \end{enumerate}
    \item
      ``Tax Bracket Amount'' means, for a Salary Cap Year, an amount equal to \$5 million multiplied by a fraction, the numerator of which is the Salary Cap for such Salary Cap Year and the denominator of which is the Salary Cap for the 2023-24 Salary Cap Year.
    \end{enumerate}

    \emph{For example, assume the Salary Cap for the 2023-24 Salary Cap Year is \$134 million and the Salary Cap for the 2024-25 Salary Cap Year is \$140.7 million. The Tax Bracket Amount for the 2023-24 Salary Cap Year would be \$5 million (i.e., \$5 million multiplied by a fraction, the numerator of which is \$134 million and the denominator of which is \$134 million), and the Tax Bracket Amount for the 2024-25 Salary Cap Year would be \$5.25 million (i.e., \$5 million multiplied by a fraction, the numerator of which is \$140.7 million (the 2024-25 Salary Cap) and the denominator of which is \$134 million (the 2023-24 Salary Cap)).}
  \item
    Each Team whose Tax Team Salary exceeds the Tax Level for any Salary Cap Year shall be required to pay a tax to the NBA. For each Salary Cap Year, the tax shall be calculated: (A) using the applicable rates in Section 2(d)(2)(i) (``Standard Tax Rates'') for any Team whose Tax Team Salary did not exceed the Tax Level in three (3) or more of the four (4) Salary Cap Years immediately preceding such Salary Cap Year; and (B) using the applicable rates shown in Section 2(d)(2)(ii) (``Repeater Tax Rates'') for any Team whose Tax Team Salary exceeded the Tax Level in three (3) or more of the four (4) Salary Cap Years immediately preceding such Salary Cap Year.

    \begin{enumerate}
    \def\labelenumiii{(\roman{enumiii})}
    \tightlist
    \item
      Standard Tax Rates:
    \end{enumerate}

    \begin{longtable}[]{@{}
      >{\raggedright\arraybackslash}p{(\columnwidth - 4\tabcolsep) * \real{0.2045}}
      >{\raggedright\arraybackslash}p{(\columnwidth - 4\tabcolsep) * \real{0.3977}}
      >{\raggedright\arraybackslash}p{(\columnwidth - 4\tabcolsep) * \real{0.3977}}@{}}
    \toprule()
    \begin{minipage}[b]{\linewidth}\raggedright
    Incremental Team Salary Above Tax Level
    \end{minipage} & \begin{minipage}[b]{\linewidth}\raggedright
    Tax Rate for Increment 2023-24 and 2024-25 Salary Cap Years
    \end{minipage} & \begin{minipage}[b]{\linewidth}\raggedright
    Tax Rate for Increment Beginning with 2025-26 Salary Cap Year
    \end{minipage} \\
    \midrule()
    \endhead
    \$0 -- 100\% of Tax Bracket Amount & \$1.50-for-\$1 & \$1.00-for-\$1 \\
    100\% of Tax Bracket Amount -- 200\% of Tax Bracket Amount & \$1.75-for-\$1 & \$1.25-for-\$1 \\
    200\% of Tax Bracket Amount -- 300\% of Tax Bracket Amount & \$2.50-for-\$1 & \$3.50-for-\$1 \\
    300\% of Tax Bracket Amount -- 400\% of Tax Bracket Amount & \$3.25-for-\$1 & \$4.75-for-\$1 \\
    400\% of Tax Bracket Amount and Over & Tax rates increase by \$0.50 for each additional 100\% of Tax Bracket Amount above the Tax Level (e.g., for Tax Team Salary 400\% of Tax Bracket Amount to 500\% of Tax Bracket Amount above the Tax Level, the Tax rate is \$3.75-for-\$1 for that increment). & Tax rates increase by \$0.50 for each additional 100\% of Tax Bracket Amount above the Tax Level (e.g., for Tax Team Salary 400\% of Tax Bracket Amount to 500\% of Tax Bracket Amount above the Tax Level, the Tax rate is \$5.25-for-\$1 for that increment). \\
    \bottomrule()
    \end{longtable}

    \begin{enumerate}
    \def\labelenumiii{(\roman{enumiii})}
    \setcounter{enumiii}{1}
    \tightlist
    \item
      Repeater Tax Rates:
    \end{enumerate}

    \begin{longtable}[]{@{}
      >{\raggedright\arraybackslash}p{(\columnwidth - 4\tabcolsep) * \real{0.2045}}
      >{\raggedright\arraybackslash}p{(\columnwidth - 4\tabcolsep) * \real{0.3977}}
      >{\raggedright\arraybackslash}p{(\columnwidth - 4\tabcolsep) * \real{0.3977}}@{}}
    \toprule()
    \begin{minipage}[b]{\linewidth}\raggedright
    Incremental Team Salary Above Tax Level
    \end{minipage} & \begin{minipage}[b]{\linewidth}\raggedright
    Tax Rate for Increment 2023-24 and 2024-25 Salary Cap Years
    \end{minipage} & \begin{minipage}[b]{\linewidth}\raggedright
    Tax Rate for Increment Beginning with 2025-26 Salary Cap Year
    \end{minipage} \\
    \midrule()
    \endhead
    \$0 -- 100\% of Tax Bracket Amount & \$2.50-for-\$1 & \$3.00-for-\$1 \\
    100\% of Tax Bracket Amount -- 200\% of Tax Bracket Amount & \$2.75-for-\$1 & \$3.25-for-\$1 \\
    200\% of Tax Bracket Amount -- 300\% of Tax Bracket Amount & \$3.50-for-\$1 & \$5.50-for-\$1 \\
    300\% of Tax Bracket Amount -- 400\% of Tax Bracket Amount & \$4.25-for-\$1 & \$6.75-for-\$1 \\
    400\% of Tax Bracket Amount and Over & Tax rates increase by \$0.50 for each additional 100\% of Tax Bracket Amount above the Tax Level (e.g., for Tax Team Salary 400\% of Tax Bracket Amount to 500\% of Tax Bracket Amount above the Tax Level, the Tax rate is \$4.75-for-\$1 for that increment). & Tax rates increase by \$0.50 for each additional 100\% of Tax Bracket Amount above the Tax Level (e.g., for Tax Team Salary 400\% of Tax Bracket Amount to 500\% of Tax Bracket Amount above the Tax Level, the Tax rate is \$7.25-for-\$1 for that increment). \\
    \bottomrule()
    \end{longtable}

    \emph{Example: In respect of the 2023-24 Salary Cap Year, the Tax Bracket Amount is \$5 million. Assume that Team A is subject to the Standard Tax Rates, and Team A has a Tax Team Salary that exceeds the Tax Level by \$15 million. Team A would pay a Tax of \$28.75 million (i.e., \$5 million times \$1.50, plus \$5 million times \$1.75, plus \$5 million times \$2.50).}

    \emph{Example: Assume that, in respect of the 2025-26 Salary Cap Year, the Tax Bracket Amount is \$6 million, Team B is subject to the Standard Tax Rates, and Team B has a Tax Team Salary that exceeds the Tax Level by \$15 million. Team B would pay a Tax of \$24 million (i.e., \$6 million times \$1.00, plus \$6 million times \$1.25, plus \$3 million times \$3.50).}

    \emph{Example: Assume that, in respect of the 2026-27 Salary Cap Year, the Tax Bracket Amount is \$6.5 million, Team C is subject to the Repeater Tax Rates, and Team C has a Tax Team Salary that exceeds the Tax Level by \$15 million. Team C would pay a Tax of \$51.625 million (i.e., \$6.5 million times \$3.00, plus \$6.5 million times \$3.25, plus \$2 million times \$5.50).}
  \item
    Each Team that owes a tax in respect of a Salary Cap Year shall make the required tax payment to the NBA no later than ten (10) business days following the completion of the Governing Audit Report for such Salary Cap Year.
  \item
    All amounts remitted to the NBA by NBA Teams pursuant to this Section 2(d) shall be the exclusive property of the NBA, and such amounts shall be used and/or distributed as follows:

    \begin{enumerate}
    \def\labelenumiii{(\roman{enumiii})}
    \tightlist
    \item
      Subject to Sections 2(c)(2)(ii) and 2(c)(7) above, the NBA may elect to distribute up to fifty percent (50\%) of such amounts to one (1) or more Teams based in whole or in part on the fact that such Team(s) did not owe a tax for such Salary Cap Year (e.g., subject to Sections 2(c)(2)(ii) and 2(c)(7) above, the NBA could elect to distribute fifty percent (50\%) of such amounts in equal shares to all non-taxpayers in such Salary Cap Year); and
    \item
      amounts not distributed in accordance with Section 2(d)(4)(i) above shall be used for one (1) or more ``League purposes'' (as defined below) selected by the NBA. For purposes of this Section 2(d)(4), the use of tax amounts for a ``League purpose'' shall mean the use of such amounts for any purpose, including, but not limited to, the distribution of such amounts to one (1) or more Teams; provided, however, that such amounts may not be distributed to a Team or expended for the benefit or detriment of a Team in a manner that is based, directly or indirectly, in whole or in part, on the amount of the Team's Team Salary or on whether the Team is a taxpayer. Without limiting the foregoing, a team assistance plan adopted by the NBA and funded, in whole or in part, with tax amounts shall be considered a ``League purpose'' if, pursuant to the plan, a Team's entitlement to an assistance receipt and/or the amount of such receipt is based, in whole or in part, on a profit, loss, and/or expense computation determined by the NBA under which the Team is credited with a Team Salary no less than the league average; provided, however, that in order to qualify as a ``League purpose,'' such a plan may not otherwise base a Team's entitlement to assistance and/or the amount of such assistance on the amount of a Team's Team Salary or on whether the Team is a taxpayer.
    \end{enumerate}
  \end{enumerate}
\item
  \textbf{Operation of Apron Levels.}

  \begin{enumerate}
  \def\labelenumii{(\arabic{enumii})}
  \item
    ``Apron Team Salary'' means, for a Team for a Salary Cap Year, the Team's Team Salary:

    \begin{enumerate}
    \def\labelenumiii{(\roman{enumiii})}
    \tightlist
    \item
      plus all Performance Bonuses excluded from a player's Salary under Section 3(d) below;
    \item
      plus the Salary attributable to a Contract signed by a Free Agent with zero (0) Years of Service or one (1) Year of Service provided for in Section 2(d)(1)(i)(F) above;
    \item
      plus any amount that could be added to the Team's Team Salary for such Salary Cap Year pursuant to Section 4(a)(1)(iii) below;
    \item
      minus any Free Agent Amounts as described in Section 4(a)(2) below;
    \item
      plus, with respect to any Restricted Free Agent, the greater of (A) the Salary plus Unlikely Bonuses called for in any outstanding Qualifying Offer (or any outstanding Maximum Qualifying Offer, if applicable) tendered to the player, or (B) the Salary plus Unlikely Bonuses called for in any First Refusal Exercise Notice (as defined in Article XI, Section 5(g)) issued with respect to such player;
    \item
      minus any amounts with respect to unsigned First Round Picks described in Section 4(a)(4) below;
    \item
      plus the amount of any outstanding Required Tender to a First Round Pick;
    \item
      minus the amount of any Salary Cap Exception that is deemed included in Team Salary pursuant to Sections 4(a)(7) and 6(n)(2) below;
    \item
      plus any amount excluded from its Team Salary pursuant to Section 4(l) below; and
    \item
      minus the amount of any incomplete roster cap hold amount added to the Team's Salary pursuant to Section 4(f) below.
    \end{enumerate}
  \item
    \begin{enumerate}
    \def\labelenumiii{(\roman{enumiii})}
    \tightlist
    \item
      At any point during a Salary Cap Year, the following rules shall apply with respect to the transactions listed in the table in Section 2(e)(4) below (the ``Transaction Restrictions Table''):

      \begin{enumerate}
      \def\labelenumiv{(\Alph{enumiv})}
      \tightlist
      \item
        A Team may not engage in a transaction set forth in the Transaction Restrictions Table if, immediately following such transaction, the Team's Apron Team Salary for such Salary Cap Year would exceed the ``Applicable Apron Level'' that corresponds with such transaction in the table; and
      \item
        A Team that engages in a transaction set forth in the Transaction Restrictions Table may not, for the remainder of such Salary Cap Year, have an Apron Team Salary that exceeds the Applicable Apron Level that corresponds with such transaction in the table.
      \end{enumerate}
    \item
      During the period beginning on the day after the last day of a Regular Season through the last day of the Salary Cap Year encompassing such Regular Season, the following rules, in addition to the rules set forth in Section 2(e)(2)(i) above, shall apply with respect to the transactions listed in the Transaction Restrictions Table:

      \begin{enumerate}
      \def\labelenumiv{(\Alph{enumiv})}
      \tightlist
      \item
        A Team may not engage in any transaction set forth in rows E through J of the Transaction Restrictions Table if, immediately following such transaction, the Team's Apron Team Salary for the immediately following Salary Cap Year (for purposes of this Section 2(e), the ``Subsequent Salary Cap Year'') would exceed the Applicable Apron Level (for such Subsequent Salary Cap Year) that corresponds with such transaction in the table; and
      \item
        A Team that engages in any transaction set forth in rows E through J of the Transaction Restrictions Table may not, at any time from immediately following such transaction through the end of the Subsequent Salary Cap Year, have an Apron Team Salary for such Subsequent Salary Cap Year that exceeds the Applicable Apron Level (for such Subsequent Salary Cap Year) that corresponds with such transaction in the table.
      \end{enumerate}
    \item
      The following additional restrictions will apply to Teams that use the Taxpayer Mid-Level Salary Exception:

      \begin{enumerate}
      \def\labelenumiv{(\Alph{enumiv})}
      \tightlist
      \item
        During the 2023-24 Salary Cap Year, a Team may not engage in any transaction set forth in rows A through E of the Transaction Restrictions Table if it has previously signed a Player Contract pursuant to the Taxpayer Mid-Level Salary Exception during such Salary Cap Year.
      \item
        For each Salary Cap Year beginning with the 2024-25 Salary Cap Year, a Team may not engage in any transaction set forth in rows A through F of the Transaction Restrictions Table if it has previously signed a Player Contract pursuant to the Taxpayer Mid-Level Salary Exception during such Salary Cap Year.
      \end{enumerate}
    \end{enumerate}
  \item
    To effectuate the rules set forth in Section 2(e)(2)(ii) above, during the period beginning on the day after the last day of a Regular Season through the last day of the Salary Cap Year encompassing such Regular Season, a Team shall not be permitted to engage in any transaction if such transaction would result in the Team failing to comply with the rules set forth in Section 2(e)(2)(ii) as of the first day of the Subsequent Salary Cap Year assuming that:

    \begin{enumerate}
    \def\labelenumiii{(\roman{enumiii})}
    \tightlist
    \item
      For purposes of determining the Team's Apron Team Salary for the Subsequent Salary Cap Year:

      \begin{enumerate}
      \def\labelenumiv{(\Alph{enumiv})}
      \tightlist
      \item
        all Team or Player Options in respect of such Subsequent Salary Cap Year are exercised;
      \item
        no outstanding ETOs in respect of such Subsequent Salary Cap Year are exercised;
      \item
        the Team engages in no additional transactions for the remainder of the then-current Salary Cap Year; and
      \item
        any player on the Team whose Salary for the Subsequent Salary Cap Year may increase by virtue of meeting the Higher Max Criteria during the fourth Season of his Rookie Scale Contract achieves the highest Salary that he is eligible to earn based on any Generally Recognized League Honors for the just-completed Regular Season for which winners have not yet been announced; and
      \end{enumerate}
    \item
      The amount of the Salary Cap, First Apron Level, and Second Apron Level for the Subsequent Salary Cap Year is equal to the amount of the Salary Cap, First Apron Level, and Second Apron Level, respectively, for the then-current Salary Cap Year.
    \end{enumerate}
  \item
    Transaction Restrictions Table:

    \begin{longtable}[]{@{}
      >{\raggedright\arraybackslash}p{(\columnwidth - 2\tabcolsep) * \real{0.6757}}
      >{\raggedright\arraybackslash}p{(\columnwidth - 2\tabcolsep) * \real{0.3243}}@{}}
    \toprule()
    \begin{minipage}[b]{\linewidth}\raggedright
    Transaction
    \end{minipage} & \begin{minipage}[b]{\linewidth}\raggedright
    Applicable Apron Level
    \end{minipage} \\
    \midrule()
    \endhead
    A. Team signs or acquires a player using the Bi-annual Exception (as described in Section 6(d) below) & First Apron Level \\
    B. Team signs or acquires a player using the Non-Taxpayer Mid-Level Salary Exception (as described in Section 6(e) below) & First Apron Level \\
    C. Team acquires a player pursuant to a Contract entered into in accordance with Section 8(e)(1) below & First Apron Level \\
    D. Team signs a Contract during the Regular Season with a player who was previously under a Contract that: (i) was terminated during such Regular Season; and (ii) prior to such termination, provided for a Salary for the Salary Cap Year encompassing such Regular Season of greater than the amount of the Non-Taxpayer Mid-Level Salary Exception for such Salary Cap Year & First Apron Level \\
    E. Team acquires a player using an Expanded Traded Player Exception (as described in Section 6(j)(1)(iv) below) & First Apron Level \\
    F. Team acquires a player using a Standard Traded Player Exception (as described in Section 6(j)(1)(i) below) (i) after the end of the Regular Season in which such Traded Player Exception arose, or (ii) if such Traded Player Exception arose during the period from the day following the last day of a Regular Season through the day before the first day of the immediately following Regular Season, after the last day of such following Regular Season & First Apron Level \\
    G. Team acquires a player using a Transition Traded Player Exception (as described in Section 6(j)(1)(iii) below) & First Apron Level \\
    H. Team acquires a player using an Aggregated Standard Traded Player Exception (as described in Section 6(j)(1)(ii) below) & Second Apron Level \\
    I. Team pays cash to another Team in connection with a trade in accordance with Section 8(a) below & Second Apron Level \\
    J. Team acquires a player using a Traded Player Exception (as described in Section 6(j)(1)(i), (ii), (iii), or (iv) below), which Traded Player Exception is in respect of a Player Contract signed and traded pursuant to Section 8(e)(1) below & Second Apron Level \\
    K. Team signs a player using the Taxpayer Mid-Level Salary Exception (as described in Section 6(f) below) & Second Apron Level \\
    \bottomrule()
    \end{longtable}
  \item
    Notwithstanding anything to the contrary in Section 2(e)(2) above, a Team that engages in one or more of the transactions set forth in rows F through J in the Transaction Restrictions Table during the 2023-24 Salary Cap Year will not by virtue of engaging in any such transaction(s) be prohibited from having an Apron Team Salary in the 2023-24 Salary Cap Year that exceeds the Applicable Apron Level for such Salary Cap Year.
  \end{enumerate}

  \emph{Examples: Assume the following First Apron Levels and Second Apron Levels:}

  \begin{longtable}[]{@{}ccc@{}}
  \toprule()
  \emph{Salary Cap Year} & \emph{First Apron Level} & \emph{Second Apron Level} \\
  \midrule()
  \endhead
  \emph{2023-24} & \emph{\$170 million} & \emph{\$180.5 million} \\
  \emph{2024-25} & \emph{\$178.5 million} & \emph{\$189.5 million} \\
  \emph{2025-26} & \emph{\$187.5 million} & \emph{\$199 million} \\
  \bottomrule()
  \end{longtable}

  \begin{enumerate}
  \def\labelenumii{\arabic{enumii}.}
  \tightlist
  \item
    \emph{On July 8, 2023, having not previously signed a Player Contract pursuant to the Taxpayer Mid-Level Salary Exception at any point during the 2023-24 Salary Cap Year, Team A signs a Player Contract pursuant to the Non-Taxpayer Mid-Level Salary Exception. Immediately following such signing, Team A's 2023-24 Apron Team Salary is \$165 million and the remaining portion of the Non-Taxpayer Mid-Level Salary Exception is \$2 million. Such signing (a transaction set forth in row B of the Transaction Restrictions Table with an Applicable Apron Level of the First Apron Level): (a) is not prohibited by Section 2(e)(2)(iii)(A) above because Team A had not previously signed a Player Contract pursuant to the Taxpayer Mid-Level Salary Exception at any point during the 2023-24 Salary Cap Year; and (b) is not prohibited by Section 2(e)(2)(i)(A) above because, immediately following such signing, Team A's 2023-24 Apron Team Salary is less than or equal to \$170 million (i.e., the 2023-24 First Apron Level). As a result of such signing, pursuant to Section 2(e)(2)(i)(B) above, Team A may not, for the remainder of the 2023-24 Salary Cap Year, have a 2023-24 Apron Team Salary that exceeds \$170 million (i.e., the 2023-24 First Apron Level).}
  \item
    \emph{On June 20, 2025 (after the conclusion of the 2024-25 Regular Season in April 2025), having not previously signed a Player Contract pursuant to the Taxpayer Mid-Level Salary Exception at any point during the 2024-25 Salary Cap Year, Team B acquires by assignment a Player Contract using a Standard Traded Player Exception that arose during the 2024-25 Regular Season. Immediately following such trade, Team B's 2024-25 Apron Team Salary is \$177 million, and Team B's 2025-26 Apron Team Salary (calculated in accordance with Section 2(e)(3)(i) above) is \$175 million. Such trade (a transaction set forth in row F of the Transaction Restrictions Table with an Applicable Apron Level of the First Apron Level): (a) is not prohibited by Section 2(e)(2)(iii)(B) above because Team A had not previously signed a Player Contract pursuant to the Taxpayer Mid-Level Salary Exception at any point during the 2024-25 Salary Cap Year; (b) is not prohibited by Section 2(e)(2)(i)(A) above because, immediately following such trade, Team B's 2024-25 Apron Team Salary is less than or equal to \$178.5 million (i.e., the 2024-25 First Apron Level); and (c) is not prohibited by Section 2(e)(2)(ii)(A) above because immediately following such trade, Team B's 2025-26 Apron Team Salary (calculated in accordance with Section 2(e)(3)(i) above) is less than or equal to \$178.5 million (i.e., the assumed 2025-26 First Apron Level as calculated in accordance with Section 2(e)(3)(ii) above). As a result of the trade: (i) pursuant to Section 2(e)(2)(i)(B) above, Team B may not for the remainder of the 2024-25 Salary Cap Year, have a 2024-25 Apron Team Salary that exceeds \$178.5 million (i.e., the 2024-25 First Apron Level); (ii) pursuant to Section 2(e)(2)(ii)(B) above, Team B may not, for the remainder of the 2024-25 Salary Cap Year, have a 2025-26 Apron Team Salary (calculated in accordance with Section 2(e)(3)(i) above) that exceeds \$178.5 million (i.e., the assumed 2025-26 First Apron Level); and (iii) pursuant to Section 2(e)(2)(ii)(B) above, Team B may not, for the entirety of the 2025-26 Salary Cap Year, have a 2025-26 Apron Team Salary that exceeds \$187.5 million (i.e., the 2025-26 First Apron Level).}
  \item
    \emph{On July 10, 2025, having not previously engaged (x) in a transaction set forth in row E or F of the Transaction Restrictions Table at any point from the day following the last day of the 2024-25 Regular Season through the end of 2024-25 Salary Cap Year or (y) in any transactions set forth in rows A through F of the Transaction Restrictions Table at any point during the 2025-26 Salary Cap Year, Team C signs a Player Contract pursuant to the Taxpayer Mid-Level Salary Exception. (Note, for clarity, that it is necessarily the case that Team C also did not engage in the transaction set forth in row G of the Transactions Restriction Table at any point from the day following the last day of the 2024-25 Regular Season through the date of the Player Contract signing in this example, because such transaction can occur only during the 2023-24 Salary Cap Year.) Immediately following such signing, Team C's 2025-26 Apron Team Salary is \$195 million. Such signing (a transaction set forth in row K of the Transaction Restrictions Table with an Applicable Apron of the Second Apron Level): (a) is not prohibited by Section 2(e)(2)(ii)(B) above because Team C had not previously engaged in a transaction set forth in rows E through G of the Transaction Restrictions Table (i.e., transactions that would have prohibited Team C from having an Apron Team Salary that exceeds the First Apron Level in the 2025-26 Salary Cap Year) at any point from the day following the last day of the 2024-25 Regular Season through the end of 2024-25 Salary Cap Year; (b) is not prohibited by Section 2(e)(2)(i)(B) above because Team C had not previously engaged in any transactions set forth in rows A through G of the Transaction Restrictions Table (i.e., transactions that would have prohibited Team C from having an Apron Team Salary that exceeds the First Apron Level in the 2025-26 Salary Cap Year) at any point during the 2025-26 Salary Cap Year; and (c) is not prohibited by Section 2(e)(2)(i)(A) above because, immediately following such signing, Team C's 2025-26 Apron Team Salary is less than or equal to \$199 million (i.e., the 2025-26 Second Apron Level). As a result of such signing, for the remainder of the 2025-26 Salary Cap Year: (i) pursuant to Section 2(e)(2)(i)(B) above, Team C may not have a 2025-26 Apron Team Salary that exceeds \$199 million (i.e., the 2025-26 Second Apron Level); and (ii) pursuant to Section 2(e)(2)(iii)(B) above, Team C may not engage in any of the transactions set forth in rows A through F of the Transaction Restrictions Table.}
  \item
    \emph{On July 9, 2024, having not previously engaged (x) in any of the transactions set forth in rows E through J of the Transaction Restrictions Table at any point from the day following the last day of the 2023-24 Regular Season through the end of 2023-24 Salary Cap Year or (y) in any of the transactions set forth in the Transactions Restrictions Table at any point during the 2024-25 Salary Cap Year, Team D seeks to acquire by trade a player using the Aggregated Standard Traded Player Exception. Immediately following such trade, Team D's 2024-25 Apron Team Salary would be \$195 million. The proposed trade (a transaction set forth in row H of the Transaction Restrictions Table with an Applicable Apron Level of the Second Apron Level) is prohibited by Section 2(e)(2)(i)(A) above because, immediately following the trade, Team D's 2024-25 Apron Team Salary would exceed \$189.5 million (i.e., the 2024-25 Second Apron Level).}
  \item
    \emph{On June 22, 2024 (after the conclusion of the 2023-24 Regular Season in April 2024), having not previously engaged in any of the transactions set forth in rows A through E of the Transaction Restrictions Table at any point during the 2023-24 Salary Cap Year, Team E acquires by trade a Player Contract using the Transition Traded Player Exception. Immediately following such trade, Team E's 2023-24 Apron Team Salary is \$175 million, and Team E's 2024-25 Apron Team Salary (calculated in accordance with Section 2(e)(3)(i) above) is \$168 million. Such trade (a transaction set forth in row G of the Transaction Restrictions Table with an Applicable Apron Level of the First Apron Level): (a) is not prohibited by Section 2(e)(2)(i)(B) above because Team E had not previously engaged in any of the transactions set forth in rows A through E of the Transaction Restrictions Table at any point during the 2023-24 Salary Cap Year (i.e., transactions that would have prohibited Team E from having an Apron Team Salary that exceeds the First Apron Level in the 2023-24 Salary Cap Year); (b) is not prohibited by Section 2(e)(2)(i)(A) above because, notwithstanding that Team E's 2023-24 Apron Team Salary exceeds \$170 million (i.e., the 2023-24 First Apron Level), pursuant to Section 2(e)(5) above, a Team that uses the Transition Traded Player Exception to acquire a player during the 2023-24 Salary Cap Year will not by virtue of using such Traded Player Exception be prohibited from having a 2023-24 Apron Team Salary that exceeds the 2023-24 First Apron Level; and (c) is not prohibited by Section 2(e)(2)(ii)(A) above because, immediately following such trade, Team E's 2024-25 Apron Team Salary (calculated in accordance with Section 2(e)(3)(i) above) is less than or equal to \$170 million (i.e., the assumed 2024-25 First Apron Level as calculated in accordance with Section 2(e)(3)(ii) above). As a result of the trade, pursuant to Section 2(e)(2)(ii)(B) above: (i) for the remainder of the 2023-24 Salary Cap Year, Team E may not have a 2024-25 Apron Team Salary (calculated in accordance with Section 2(e)(3)(i) above) that exceeds \$170 million (i.e., the assumed 2024-25 First Apron Level); and (ii) for the entirety of the 2024-25 Salary Cap Year, Team E may not have a 2024-25 Apron Team Salary that exceeds \$178.5 million (i.e., the 2024-25 First Apron Level).}
  \end{enumerate}
\item
  \textbf{Draft Pick Penalty.}

  \begin{enumerate}
  \def\labelenumii{(\arabic{enumii})}
  \tightlist
  \item
    As used in this Agreement, the following terms shall have the following meanings:

    \begin{enumerate}
    \def\labelenumiii{(\roman{enumiii})}
    \tightlist
    \item
      ``Second Apron Team'' means, for a Salary Cap Year, a Team that, as of the start of the Team's last Regular Season game occurring within such Salary Cap Year, has an Apron Team Salary for such Salary Cap Year that exceeds the Second Apron Level for such Salary Cap Year.
    \item
      ``Draft Pick Penalty'' means, for a Team's first round draft pick, that such draft pick shall be the final draft pick in the first round of the applicable NBA Draft (regardless of the position in the first round of the Draft at which the Team otherwise would have selected pursuant to NBA rules governing the order of selection by Teams in the Draft); provided, however, that, if multiple Teams' first round draft picks are each subject to a Draft Pick Penalty in respect of the same NBA Draft, then the Teams with such first round draft picks shall select in the inverse order of their winning percentage for the Regular Season immediately preceding such NBA Draft (with priority in selection among any such Teams tied on a winning percentage basis established pursuant to NBA rules governing the order of selection by Teams in the Draft). For example, if Team A's and Team B's first round draft picks in the 2032 NBA Draft are each subject to a Draft Pick Penalty, and Team A finished with a better winning percentage than Team B for the 2031-32 Regular Season, then Team A would make the final selection in the first round of the 2032 NBA Draft and Team B would make the immediately preceding selection.
    \end{enumerate}
  \item
    Beginning with the 2024-25 Salary Cap Year, if a Team is a Second Apron Team for a Salary Cap Year, then:

    \begin{enumerate}
    \def\labelenumiii{(\roman{enumiii})}
    \item
      the Team shall be prohibited from trading (either conditionally or unconditionally) its first round draft pick in the first NBA Draft that occurs following the seventh Season that follows the Season occurring within such Salary Cap Year; and
    \item
      with respect to the four (4) Salary Cap Years immediately following such Salary Cap Year:

      \begin{enumerate}
      \def\labelenumiv{(\Alph{enumiv})}
      \tightlist
      \item
        If the Team is a Second Apron Team for two (2) or more of such four (4) Salary Cap Years, then such first round draft pick shall be subject to a Draft Pick Penalty; and
      \item
        If the Team is a Second Apron Team for fewer than two (2) of such four (4) Salary Cap Years, then, as of the day following the last day of the Regular Season encompassed by the third of such four (4) Salary Cap Years in which the Team is not a Second Apron Team, such Team shall be permitted to trade (conditionally or unconditionally) such first round draft pick. For clarity, such first round draft pick shall not be subject to a Draft Pick Penalty.
      \end{enumerate}

      \emph{Example: If Team A is a Second Apron Team for the 2024-25 Salary Cap Year, then it would be prohibited from trading its 2032 first round draft pick (i.e., its first round draft pick in the first NBA Draft that occurs following the seventh Season that follows the Season occurring within such Salary Cap Year). If Team A is also a Second Apron Team for the 2025-26 and 2028-29 Salary Cap Years, then Team A's 2032 first round draft pick would be subject to a Draft Pick Penalty.}

      \emph{Example: If Team B is a Second Apron Team for the 2024-25 Salary Cap Year, then it would be prohibited from trading its 2032 first round draft pick. If Team B is not a Second Apron Team in the 2025-26, 2026-27, and 2027-28 Salary Cap Years, then Team B would be permitted to trade its 2032 first round draft pick as of the day following the last day of the 2027-28 Regular Season (and such first round draft pick would not be subject to a Draft Pick Penalty).}
    \end{enumerate}
  \end{enumerate}
\item
  \textbf{Expansion Team Salary Caps, Minimum Team Salaries, Tax Levels, and Apron Levels.} Each Expansion Team shall have the same Salary Cap, Minimum Team Salary, Tax Level, First Apron Level, and Second Apron Level as all other Teams, except as follows:

  \begin{enumerate}
  \def\labelenumii{(\arabic{enumii})}
  \tightlist
  \item
    During the first Salary Cap Year in which it begins play, an Expansion Team shall have a Salary Cap equal to sixty-six and two-thirds percent (66-2/3\%) of the Salary Cap calculated pursuant to Section 2(a) above (the ``First Year Expansion Team Salary Cap''); and shall have a Minimum Team Salary equal to ninety percent (90\%) of the First Year Expansion Team Salary Cap.
  \item
    During the Salary Cap Year immediately following the Salary Cap Year in which it begins play, an Expansion Team shall have a Salary Cap equal to eighty percent (80\%) of the Salary Cap calculated pursuant to Section 2(a) above (the ``Second Year Expansion Team Salary Cap''); and shall have a Minimum Team Salary equal to ninety percent (90\%) of the Second Year Expansion Team Salary Cap.
  \end{enumerate}
\end{enumerate}

\hypertarget{determination-of-salary.}{%
\section{Determination of Salary.}\label{determination-of-salary.}}

For the purposes of determining a player's Salary with respect to a Salary Cap Year, the following rules shall apply:

\begin{enumerate}
\def\labelenumi{(\alph{enumi})}
\tightlist
\item
  \textbf{Deferred Compensation.}

  \begin{enumerate}
  \def\labelenumii{(\arabic{enumii})}
  \tightlist
  \item
    \emph{General Rule:} All Player Contracts entered into, extended, or renegotiated after the date of this Agreement shall specify the Season(s) in which any Deferred Compensation is earned. Deferred Compensation shall be included in a player's Salary for the Salary Cap Year encompassing the Season in which such Deferred Compensation is earned.
  \item
    \emph{Over 38 Rule:} The following provisions shall apply to any Player Contract entered into, extended, or renegotiated that, beginning with the date such Contract, Extension, or Renegotiation is signed, covers four (4) or more Seasons, including one (1) or more Seasons commencing after such player will reach or has reached age thirty-eight (38) (an ``Over 38 Contract''):

    \begin{enumerate}
    \def\labelenumiii{(\roman{enumiii})}
    \tightlist
    \item
      Except as provided in Sections 3(a)(2)(ii)-(iii) below, the aggregate Salaries in an Over 38 Contract for Salary Cap Years commencing with the fourth Salary Cap Year of such Over 38 Contract or the first Salary Cap Year that covers a Season that follows the player's 38th birthday, whichever is later, shall be attributed to the prior Salary Cap Years pro rata on the basis of the Salaries for such prior Salary Cap Years.
    \item
      If a Qualifying Veteran Free Agent who is age 35 or 36 enters into an Over 38 Contract with his Prior Team covering five (5) Seasons, the Salary in such Over 38 Contract for the fifth Salary Cap Year shall be attributed to the prior Salary Cap Years pro rata on the basis of the Salaries for such prior Salary Cap Years. For purposes of this Section 3(a)(2)(ii), a Qualifying Veteran Free Agent who (x) enters into an Over 38 Contract with his Prior Team prior to October 1 of a Salary Cap Year, (y) is age 34 at the time he enters into the Contract, and (z) will turn age 35 on or before such October 1 shall be deemed to be 35 at the time he enters into such Over 38 Contract.
    \item
      For each Salary Cap Year of an Over 38 Contract beginning with the second Salary Cap Year prior to the First Zero Year (as defined in Section 3(a)(2)(vi) below), if the player's Contract has not been terminated as of the July 1 of such Salary Cap Year, then the Salaries of the player for such Salary Cap Year and the subsequent two (2) or fewer Salary Cap Years covered by the Contract (including any Zero Year (as defined in Section 3(a)(2)(vi) below)) shall, on such July 1, be aggregated and attributed in equal shares to each of such three (3) or fewer Salary Cap Years.
    \item
      Notwithstanding Section 3(a)(2)(i) above, there shall be no re-allocation of Salaries pursuant to this Section 3(a)(2) for any Contract between a Qualifying Veteran Free Agent and his Prior Team covering four (4) or fewer Seasons entered into by a player at age 35 or 36. For purposes of this Section 3(a)(2)(iv), a Qualifying Veteran Free Agent who (x) enters into an Over 38 Contract with his Prior Team prior to October 1 of a Salary Cap Year, (y) is age 34 at the time he enters into the Contract, and (z) will turn age 35 on or before such October 1 shall be deemed to be 35 at the time he enters into such Over 38 Contract.
    \item
      For purposes of determining whether a Contract is an Over 38 Contract pursuant to this Section 3(a)(2) only, Seasons shall be deemed to commence on October 1 and conclude on the last day of the Salary Cap Year.
    \item
      ``Zero Year'' means, with respect to an Over 38 Contract, any Salary Cap Year in which the Salary called for under the Contract has been attributed, in accordance with Section 3(a)(2)(i), (ii), or (iii) above, to prior Salary Cap Years of the Contract. ``First Zero Year'' means, with respect to an Over 38 Contract, the earliest Salary Cap Year in which the Salary called for under the Contract has been attributed, in accordance with Section 3(a)(2)(i), (ii), or (iii) above, to prior Salary Cap Years of the Contract.
    \item
      For purposes of this subsection (a)(2): (i) a player (A) whose birthday is on a date during the Moratorium Period and (B) who signs a Contract, Extension, or Renegotiation on or before the fifth day following the date on which the Moratorium Period concludes shall be treated as if his age, at the time of such signing, was his age on the immediately preceding June 30; and (ii) any player whose Over 38 Contract is signed pursuant to Section 8(e)(1) below shall not be considered a Qualifying Veteran Free Agent.
    \end{enumerate}
  \end{enumerate}
\item
  \textbf{Signing Bonuses.}

  \begin{enumerate}
  \def\labelenumii{(\arabic{enumii})}
  \tightlist
  \item
    \emph{Amounts Treated as Signing Bonuses:} For purposes of determining a player's Salary, the term ``signing bonus'' shall include:

    \begin{enumerate}
    \def\labelenumiii{(\roman{enumiii})}
    \tightlist
    \item
      any amount provided for in a Player Contract that is earned upon the signing of such Contract;
    \item
      at the time of a trade of a Player Contract, any amount that, under the terms of the Contract, is earned in the form of a bonus upon the trade of the Contract; and
    \item
      payments in excess of the Excluded International Player Payment Amount, in accordance with Section 3(e) below.
    \end{enumerate}
  \item
    \emph{Proration:} Any signing bonus contained in a Player Contract shall be allocated over the number of Salary Cap Years (or over the then-current and any remaining Salary Cap Years in the case of a signing bonus described in Section 3(b)(1)(ii) above) covered by such Contract in proportion to the percentage of Base Compensation in each such Salary Cap Year that, at the time of allocation, is protected for lack of skill; provided, however, that if the Player Contract provides for an ETO, the foregoing allocation shall be performed only over Salary Cap Years that precede the Effective Season of such ETO. In the event that, at the time of allocation, none of the Base Compensation provided for by a Player Contract (or none of the then-current or remaining Base Compensation in the case of a signing bonus described in Section 3(b)(1)(ii) above) is protected for lack of skill, then the entire amount of the signing bonus shall be allocated to the first Salary Cap Year of the Contract (or, in the case of a signing bonus described in Section 3(b)(1)(ii) above, the Salary Cap Year during which the player's Contract is traded).
  \item
    \emph{Extensions:}

    \begin{enumerate}
    \def\labelenumiii{(\roman{enumiii})}
    \tightlist
    \item
      In the event that a Team with a Team Salary at or over the Salary Cap enters into an Extension that calls for or contains a signing bonus, such signing bonus shall be paid no sooner than the first day of the first Salary Cap Year covered by the extended term and shall be allocated, in equal parts, over the number of Salary Cap Years covered by the extended term in proportion to the percentage of Base Compensation in each such Salary Cap Year that, at the time of allocation, is protected for lack of skill. In the event that, at the time of the allocation, none of the Base Compensation provided for during the extended term is protected for lack of skill, then the entire amount of the signing bonus shall be allocated to the first Salary Cap Year of the extended term.
    \item
      A Team with a Team Salary below the Salary Cap may enter into an Extension that calls for or contains a signing bonus to be paid at any time during the Contract's original or extended term. In the event that a Team with a Team Salary below the Salary Cap enters into an Extension that calls for or contains a signing bonus to be paid no sooner than the first day of the Salary Cap Year covered by such extended term, the bonus shall be allocated in accordance with the proration rules set forth in Section 3(b)(3)(i) above. In the event a Team with a Team Salary below the Salary Cap enters into an Extension that calls for or contains a signing bonus to be paid prior to the first day of the first Salary Cap Year covered by the extended term, the following rules shall apply:

      \begin{enumerate}
      \def\labelenumiv{(\Alph{enumiv})}
      \tightlist
      \item
        The signing bonus shall be allocated over the remaining Salary Cap Years (including the then-current Salary Cap Year) under the original term of the Contract and the extended term in proportion to the percentage of Base Compensation in each such Salary Cap Year that, at the time of allocation, is protected for lack of skill. In the event that, at the time of allocation, none of the Base Compensation provided for during the then-current and any remaining Salary Cap Years under the original term of the Contract or during the extended term is protected for lack of skill, then the entire amount of the signing bonus shall be allocated to the Salary Cap Year during which the Extension is signed; and
      \item
        The Extension shall be deemed a Renegotiation and shall be subject to the rules governing Renegotiations set forth in Section 7 below; and
      \item
        Notwithstanding Article II, Section 3(b), the Exhibit 1 of such Extension must provide that the signing bonus shall be paid in two (2) installments as follows:

        \begin{enumerate}
        \def\labelenumv{(\arabic{enumv})}
        \tightlist
        \item
          the first installment shall be paid on a specified date prior to the first day of the first Salary Cap Year covered by the extended term, and shall be for an amount equal to the portion of the signing bonus that is allocated to the Salary Cap Year(s) covered by the original term of the Contract; and
        \item
          the second installment shall be paid on a specified date on or after the first day of the first Salary Cap Year covered by the extended term, and shall be for an amount equal to the portion of the signing bonus that is allocated to the Salary Cap Year(s) covered by the extended term of the Contract.
        \end{enumerate}
      \end{enumerate}
    \item
      If a Team and player enter into an Extension and provide that the trade bonus provision contained in the original Contract would not be applicable to the extended term in accordance with Article XXIV, Section 2(a)(v), then, in the case of an earned signing bonus described in Section 3(b)(1)(ii) above, the signing bonus shall be allocated over the then-current and any remaining Salary Cap Year(s) covered by the original term of the extended Contract (and not any of the Salary Cap Years covered by the extended term) in proportion to the percentage of Base Compensation in each such Salary Cap Year that, at the time of allocation, is protected for lack of skill. In the event that, at the time of allocation, none of the then-current or applicable remaining Base Compensation is protected for lack of skill, then the entire amount of the signing bonus shall be allocated to the Salary Cap Year during which the player's extended Contract is traded.
    \item
      If a Team and player enter into an Extension that contains a trade bonus provision that is applicable to the Contract's original and extended term, then, in the case of a signing bonus described in Section 3(b)(1)(ii) above that is earned prior to the first day of the Salary Cap Year covered by the extended term:

      \begin{enumerate}
      \def\labelenumiv{(\Alph{enumiv})}
      \tightlist
      \item
        For purposes of calculating the signing bonus and allocating the signing bonus to the applicable Salary Cap Year(s), the Base Compensation in the extended term of the Contract shall be the Base Compensation as set forth in the Contract; provided, however, that:

        \begin{enumerate}
        \def\labelenumv{(\arabic{enumv})}
        \tightlist
        \item
          If the Contract provides for Base Compensation in the first Salary Cap Year of the extended term that is expressed as a percentage of the Salary Cap in accordance with Article II, Section 7(d) or Section 7(e), then the Base Compensation in the extended term of the Contract shall be determined assuming that the Salary Cap will increase by four and one-half percent (4.5\%) each Salary Cap Year beginning with the Salary Cap Year following the then-current Salary Cap Year and ending with the first Salary Cap Year covered by the extended term; or
        \item
          If the Contract provides for Salary plus Unlikely Bonuses in the first Salary Cap Year of the extended term that exceeds the applicable Maximum Annual Salary that would apply to such player assuming that (a) the player will be credited with a Year of Service for each remaining year of the original term of the Contract, and (b) the Salary Cap will increase by four and one-half percent (4.5\%) each Salary Cap Year beginning with the Salary Cap Year following the then-current Salary Cap Year and ending with the first Salary Cap Year covered by the extended term, then the Base Compensation in the extended term of the Contract shall be determined to be the Base Compensation that would result from the deemed amendment(s) pursuant to Article II, Section 7(c) using the assumptions described in clauses (a) and (b) of this subsection.
        \end{enumerate}
      \item
        Notwithstanding the Exhibit 4 to such Contract, the signing bonus shall be paid in two (2) installments as follows:

        \begin{enumerate}
        \def\labelenumv{(\arabic{enumv})}
        \tightlist
        \item
          The first installment shall be paid within thirty (30) days of the date of the trade to which the bonus applies, and shall be for an amount equal to the portion of the signing bonus that is allocated to the Salary Cap Year(s) covered by the original term of the Contract; and
        \item
          The second installment shall be paid within thirty (30) days of the first day of the first Salary Cap Year covered by the extended term, and shall be for an amount equal to the portion of the signing bonus that is allocated to the Salary Cap Year(s) covered by the extended term of the Contract.
        \end{enumerate}
      \end{enumerate}
    \item
      In the event that a team is required to make signing bonus payment(s) pursuant to Section 3(b)(3)(i), 3(b)(3)(ii)(C), or 3(b)(3)(iv) above, and the amount(s) of the signing bonus allocation in respect of such signing bonus are deemed amended pursuant to Article II, Section 7(c), then the amount of the payment required pursuant to Section 3(b)(3)(i), 3(b)(3)(ii)(C)(2), or 3(b)(3)(iv)(B)(2) above shall be reduced to equal the sum of the signing bonus allocation amount(s) that result from the deemed amendment(s) pursuant to Article II, Section 7(c).
    \end{enumerate}
  \end{enumerate}
\item
  \textbf{Loans to Players.} The following rules shall apply to any loan made by any Team to a player:

  \begin{enumerate}
  \def\labelenumii{(\arabic{enumii})}
  \tightlist
  \item
    If any such loan bears no interest (or annual interest at an effective rate lower than the ``Target Rate'' (as defined below)), then the interest shall be imputed on the outstanding balance at a rate equal to the difference between the Target Rate and the actual rate of interest to be paid by the player and such imputed interest shall be included in the player's Salary. The ``Target Rate'' means the ``Prime Rate'' (as defined below) plus one percent (1\%) as of the date the loan is agreed upon, except that the ``Target Rate'' shall be no lower than seven percent (7\%) or greater than nine percent (9\%). For purposes of this Section 3(c)(1), ``Prime Rate'' means the prime rate reported in the ``Money Rates'' column or any successor column of \emph{The Wall Street Journal.}
  \item
    No loan made to a player may (along with other outstanding loans to the player) exceed the amount of the player's Salary for the then-current Salary Cap Year that is protected for lack of skill. All loans must be repaid through deductions from the player's remaining Current Base Compensation over the years of the Contract that, at the time the loan is agreed upon, provide for Base Compensation that is fully protected for lack of skill (prior to the Effective Season of any ETO) in equal annual amounts (the ``annual allocable repayment amounts''). If a loan is made at a time when the remaining Current Base Compensation due for the relevant Season that is fully protected for lack of skill is less than the annual allocable repayment amount that would be owed on a loan for the full amount of the player's Current Base Compensation that is fully protected for lack of skill for the relevant Season (the ``maximum annual allocable repayment amount''), the maximum loan amount for that Season shall be reduced by the amount by which the maximum annual allocable repayment amount exceeds the amount of remaining Current Base Compensation that is fully protected for lack of skill. (For example, if a Player has \$2 million in Current Base Compensation (fully protected for lack of skill) in the first Season of a five-year Contract, and a loan is made during that Season at a time when the Player has already received his Current Base Compensation for that Season, the loan may not exceed \$1.6 million.)
  \item
    In addition to the restrictions set forth in Section 3(c)(2) above: (i) no loan may be made that would result in a violation of Article II, Section 13(e); and (ii) no loan may be made to a player whose Contract provides for Base Compensation equal to the Minimum Player Salary.
  \item
    Any forgiveness by a Team of a loan to a player shall be deemed a Renegotiation in the Salary Cap Year of such forgiveness and shall be subject to the rules governing Renegotiations set forth in Section 7 below.
  \end{enumerate}
\item
  \textbf{Incentive Compensation.}

  \begin{enumerate}
  \def\labelenumii{(\arabic{enumii})}
  \tightlist
  \item
    For purposes of determining a player's Salary each Salary Cap Year, except as provided in Sections 3(d)(2)-(4) below, any Performance Bonus (provided such Performance Bonus may be included in a Player Contract in accordance with Section 5(b) below), shall be included in Salary only if such Performance Bonus would be earned if the Team's or player's performance were identical to the performance in the immediately preceding Salary Cap Year.
  \item
    Notwithstanding Section 3(d)(1) above, in the event that, at the time of the signing of a Contract, Renegotiation or Extension, the NBA or the Players Association believes that the performance of a player and/or his Team during the immediately preceding Salary Cap Year does not fairly predict the likelihood of the player earning a Performance Bonus during any Salary Cap Year covered by the Contract, Renegotiation, or extended term of the Extension (as the case may be), the NBA or the Players Association may request that a jointly selected basketball expert (``Expert'') determine whether (i) in the case of an NBA challenge, it is more likely than not that the bonus will be earned, or (ii) in the case of a Players Association challenge, it is very likely that the bonus will not be earned. The party initiating a proceeding before the Expert shall carry the burden of proof. The Expert shall conduct a hearing within five (5) business days after the initiation of the proceeding, and shall render a determination within five (5) business days after the hearing. Notwithstanding anything to the contrary in this Section 3(d)(2), no party may, in connection with any proceeding before the Expert, refer to the facts that, absent a challenge pursuant to this Section 3(d)(2), a Performance Bonus would or would not be included in a player's Salary pursuant to Section 3(d)(1) above, or would be termed ``Likely'' or ``Unlikely'' pursuant to Article I, Section 1(ff) or (eeee). If, following an NBA challenge, the Expert determines that a Performance Bonus is more likely than not to be earned, the bonus shall be included in the player's Salary. If, following a Players Association challenge, the Expert determines that a Performance Bonus is very likely not to be earned, the bonus shall be excluded from the player's Salary. The Expert's determination that a Performance Bonus is more likely than not to be earned or very likely not to be earned shall be final, binding and unappealable. The fees and costs of the Expert in connection with any proceeding brought pursuant to this Section 3(d)(2) shall be borne equally by the parties.
  \item
    In the case of a Rookie or a Veteran who did not play during the immediately preceding Salary Cap Year who signs a Contract containing a Performance Bonus, or in the case of a player signed or acquired by an Expansion Team whose Contract contains a Performance Bonus to be paid as a result of, in whole or in part, the player's achievement of agreed-upon benchmarks relating to the Team's performance during its first Salary Cap Year, such Performance Bonus will be included in Salary if it is likely to be earned. In the event that the NBA and the Players Association cannot agree as to whether a Performance Bonus is likely to be earned, such dispute will be referred to the Expert, who will determine whether the bonus is likely to be earned or not likely to be earned. The Expert shall conduct a hearing within five (5) business days after the initiation of the proceeding, and shall render a determination within five (5) business days after the hearing. The Expert's determination that a Performance Bonus is likely to be earned or not likely to be earned shall be final, binding, and unappealable. The fees and costs of the Expert in connection with any proceeding brought pursuant to this Section 3(d)(3) shall be borne equally by the parties.
  \item
    In the event that either party initiates a proceeding pursuant to Section 3(d)(2) or (3) above, the player's Salary plus the full amount of any disputed bonuses shall be included in Team Salary during the pendency of the proceeding.
  \item
    In the event the NBA and the Players Association cannot agree on an Expert, any challenge pursuant to Sections 3(d)(2) and (3) above may be filed with the Grievance Arbitrator in accordance with Article XXXI, Sections 2-7 and 15.
  \item
    All Incentive Compensation described in Article II, Sections 3(b)(iii) and 3(c) shall be included in Salary.
  \end{enumerate}
\item
  \textbf{International Player Payments.}

  \begin{enumerate}
  \def\labelenumii{(\arabic{enumii})}
  \tightlist
  \item
    Any amount in excess of the amounts set forth below (``Excluded International Player Payment Amounts'') paid or to be paid by or at the direction of any NBA Team to (i) any basketball team other than an NBA Team, or (ii) any other entity, organization, representative or person, for the purpose of inducing a player who is participating in the game of basketball as a professional outside of the United States to enter into a Player Contract or in connection with securing the right to enter into a Player Contract with such a player shall be deemed Salary (in the form of a signing bonus) to the player:
  \end{enumerate}

  \begin{longtable}[]{@{}cc@{}}
  \toprule()
  Salary Cap Year & Excluded International Player Payment Amount \\
  \midrule()
  \endhead
  2023-24 & \$825,000 \\
  2024-25 & \$850,000 \\
  2025-26 & \$875,000 \\
  2026-27 & \$900,000 \\
  2027-28 & \$925,000 \\
  2028-29 & \$950,000 \\
  2029-30 & \$975,000 \\
  \bottomrule()
  \end{longtable}

  \begin{enumerate}
  \def\labelenumii{(\arabic{enumii})}
  \setcounter{enumii}{1}
  \tightlist
  \item
    Subject to Article XIII, any payment up to the Excluded International Player Payment Amount for a Salary Cap Year paid by or at the direction of any NBA Team pursuant to Section 3(e)(1) above to a professional basketball team outside the United States to secure the contractual release of a player shall not be deemed Salary to the player.
  \item
    Any payment for a Salary Cap Year paid by or at the direction of any NBA Team pursuant to Section 3(e)(1) above may be paid in a single installment or in multiple installments. The Excluded International Player Payment Amount, whether used in whole or in part, may be used by an NBA Team whenever it signs a player to a new Player Contract, except that the Excluded International Player Payment Amount may not be used, in whole or in part, more than once in any three-Salary Cap Year period with respect to the same player.
  \item
    The Excluded International Player Payment Amount, or any part of it, shall be deemed to have been used as of the date of the Player Contract to which it applies, regardless of when it is actually paid. A schedule of payments relating to the Excluded International Player Payment Amount, or any part of it, agreed upon at the time of the signing of the Player Contract to which it applies, shall not be deemed a multiple use of the Excluded International Player Payment Amount.
  \item
    Notwithstanding Section 3(e)(1) above, no amount paid or to be paid pursuant to this Section 3(e) shall be counted toward the Minimum Team Salary obligation of a Team in accordance with Section 2(b) or (c) above.
  \item
    Within two (2) business days following the NBA's receipt of notice of any payments made by any NBA Team that are governed by this Section 3(e), the NBA shall provide the Players Association with written notice of such payments.
  \item
    Notwithstanding anything to the contrary in this Section 3(e), Teams shall be prohibited from making any payment governed by this Section 3(e) for the purpose of inducing a player to enter into a Two-Way Contract or a Contract with an Exhibit 10, or in connection with securing the right to enter into a Two-Way Contract or a Contract with an Exhibit 10 with a player; and any Team that agrees to make a payment governed by this Section 3(e) with respect to a player shall be prohibited from entering into a Two-Way Contract or a Contract with an Exhibit 10 with such player for a period of one (1) year following the date of such agreement.
  \end{enumerate}
\item
  \textbf{One-Year Minimum Contracts.} Except where otherwise stated in this Agreement, the Salary of every player who signs a one-year, 10-Day, or Rest-of-Season Contract for the Minimum Player Salary applicable to such player shall be the lesser of (1) such Minimum Player Salary, or (2) the portion of such Minimum Player Salary that is not reimbursed out of the League-wide benefits fund described in Article IV, Section 6(h).
\item
  \textbf{Insurance Premium Reimbursement.} If a Team reimburses a player for life insurance premiums pursuant to Article II, Section 4(j)(ii), such premium reimbursement shall not be included in the computation of the player's Salary.
\item
  \textbf{Averaging.} In accordance with Article XI, Section 5(d)(iii), a player's Salary for each Salary Cap Year covered by his Contract shall be deemed in certain circumstances to be the average of the aggregate Salaries for each such Salary Cap Year.
\item
  \textbf{Player Conduct-Related Compensation Reductions.} The computation of a player's Salary shall be made without regard to any reduction made (or to be made) to his Compensation in accordance with Article VI, Section 1 or Article XLI, Section 4(e). For clarity, this Section 3(i) shall not apply to the computation of a player's Adjustment Salary in accordance with Article VII, Section 12.
\item
  \textbf{Existing Contracts.} A player's Salary with respect to any Salary Cap Year covered by a Contract entered into prior to the effective date of this Agreement shall continue to be calculated in accordance with the Salary Cap rules that were in existence at the time the Contract was entered into except as provided in Section 7(d)(6) below. In no event shall the preceding sentence apply to the calculation of Salary with respect to any Contract, Extension (with respect to the extended term), Renegotiation, transaction, or event entered into or occurring on or after the effective date of this Agreement.
\end{enumerate}

\hypertarget{determination-of-team-salary.}{%
\section{Determination of Team Salary.}\label{determination-of-team-salary.}}

\begin{enumerate}
\def\labelenumi{(\alph{enumi})}
\tightlist
\item
  \textbf{Computation.} For purposes of computing Team Salary under this Agreement, all of the following amounts shall be included:

  \begin{enumerate}
  \def\labelenumii{(\arabic{enumii})}
  \item
    Subject to the rules set forth in this Article VII, the aggregate Salaries of all active players (and former players to the extent provided by the terms of this Agreement) attributable to a particular Salary Cap Year, including, without limitation:

    \begin{enumerate}
    \def\labelenumiii{(\roman{enumiii})}
    \tightlist
    \item
      Salaries paid or to be paid to players whose Player Contracts have been terminated pursuant to the NBA's waiver procedure (without regard to any revised payment schedule that might be provided for in the terminated Player Contracts), except that, with respect to any Player Contract that has been terminated pursuant to the NBA's waiver procedure, if the waiving Team elects in writing to have the player's Salary stretched for Team Salary purposes in accordance with applicable CBA stretch rules, then the amount to be included in Team Salary for a Salary Cap Year in respect of the terminated Player Contract shall equal the amount allocated to such Salary Cap Year in accordance with such rules.
    \item
      Any amount called for in a retired player's Player Contract paid or to be paid to the player. When a player retires and the Team continues to pay such amounts, then, for purposes of computing the player's Salary for the then-current and any remaining Salary Cap Year covered by the Contract, the aggregate of such amounts, notwithstanding the payment schedule, shall be allocated pro rata over the then-current and each remaining Salary Cap Year on the basis of the remaining unearned protected Compensation in each such Salary Cap Year at the time of retirement.
    \item
      Amounts paid or to be paid pursuant to awards for, or settlements of, Grievances between a player and a Team concerning Compensation obligations under a Player Contract in accordance with the following rules (which, except for purposes of Section 4(a)(1)(iii)(C) below, shall be applied with respect to each Season for which there is any Compensation in dispute, as if the grievance relates only to such Season):

      \begin{enumerate}
      \def\labelenumiv{(\Alph{enumiv})}
      \item
        \begin{enumerate}
        \def\labelenumv{(\arabic{enumv})}
        \tightlist
        \item
          When a player initiates a Grievance (as defined in Article XXXI) against a Team seeking the payment of Compensation for a Season covered by the current or any future Salary Cap Year that the Team asserts is not owed, fifty percent (50\%) of the disputed amount shall be included in Team Salary for the Salary Cap Year to which the Grievance relates. If the Grievance is resolved during or prior to the Salary Cap Year to which it relates, following resolution of the Grievance, whether by award or settlement, the disputed amount payable by the Team in excess of the fifty percent (50\%) allocation shall be included in Team Salary for the Salary Cap Year to which the Grievance relates, or, alternatively, the amount by which the fifty percent (50\%) allocation exceeds the disputed amount payable by the Team shall be subtracted from Team Salary for the Salary Cap Year to which the Grievance relates.
        \item
          If a Grievance described in the first sentence of Section 4(a)(1)(iii)(A)(1) above is resolved after the conclusion of the Salary Cap Year to which it relates, the disputed amount payable by the Team related to such Salary Cap Year in excess of the fifty percent (50\%) allocation shall be included in Team Salary for the Salary Cap Year in which the Grievance is resolved, or, alternatively, the amount by which the fifty percent (50\%) allocation exceeds the disputed amount payable by the Team related to such Salary Cap Year shall be subtracted from Team Salary for the Salary Cap Year in which the grievance is resolved. Notwithstanding the preceding sentence: (i) a Team shall be required to pay additional tax to the NBA if and to the extent that, due to the operation of this Section 4(a)(1)(iii)(A)(2), the aggregate tax it pays to the NBA pursuant to Section 2(d) above for the two (2) Salary Cap Years in question (the Salary Cap Year for which the fifty percent (50\%) allocation was made and the subsequent Salary Cap Year in which the Grievance was resolved) is less than it would have been had the Grievance been resolved during the Salary Cap Year to which it related; and (ii) a Team shall be entitled to a tax refund from the NBA if and to the extent that, due to the operation of this Section 4(a)(1)(iii)(A)(2), the aggregate tax it pays to the NBA pursuant to Section 2(d) above for the two (2) Salary Cap Years in question is greater than it would have been had the grievance been resolved during the Salary Cap Year to which it related. In order to facilitate any such required tax refund from the NBA to the Team, the NBA shall set aside, pending resolution of the Grievance, the amount of tax paid by that Team in the Salary Cap Year to which the Grievance relates that is attributable to the fifty percent (50\%) allocation. Following resolution of the Grievance, the NBA shall pay to the Team the tax refund to which it is entitled (if any) based upon the resolution of the Grievance, and the remainder of the set aside tax funds shall be distributed by the NBA to one (1) or more Teams or otherwise used by the League in such manner as the NBA may reasonably determine, consistent with the provisions of Section 2(d)(4) above.
        \end{enumerate}
      \item
        When a player initiates a Grievance against a Team seeking the payment of Compensation for a Season covered by a prior Salary Cap Year that the Team asserts is not owed, following resolution of the Grievance, whether by award or settlement, the disputed amount payable by the Team, if any, shall be included in Team Salary for the Salary Cap Year in which the Grievance is resolved (but only to the extent that it had been previously excluded from Team Salary). Notwithstanding the preceding sentence: (i) a Team shall be required to pay additional tax to the NBA if and to the extent that, due to the operation of this Section 4(a)(1)(iii)(B), the aggregate tax it pays to the NBA pursuant to Section 2(d) above for the two (2) Salary Cap Years in question (the Salary Cap Year to which the Grievance related and the subsequent Salary Cap Year in which the Grievance was resolved) is less than it would have been had the disputed amount payable by the Team been included in Team Salary during the Salary Cap Year to which it related; and (ii) a Team shall be entitled to a tax refund from the NBA if and to the extent that, due to the operation of this Section 4(a)(1)(iii)(B), the aggregate tax it pays to the NBA pursuant to Section 2(d) above for the two (2) Salary Cap Years in question is greater than it would have been had the disputed amount payable by the Team been included in Team Salary during the Salary Cap Year to which it related.
      \item
        If a Grievance relates to a player's Compensation for more than one (1) Season, for purposes of determining the disputed amount payable by the Team with respect to each such Season following the resolution of the Grievance, the aggregate amounts payable to the player for all Seasons pursuant to the resolution of the Grievance, whether by award or settlement, shall be allocated to each such Season in proportion to the amount of Compensation that was in dispute for such Season, unless, in the case of an award, the Grievance Arbitrator allocates the amounts payable to the player to specific Seasons.
      \item
        Immediately upon reaching any agreement (oral or written) to resolve a Grievance relating to a player's Compensation, a Team shall notify the NBA by email and provide the NBA with the terms of such agreement. A Team's failure to comply with the preceding sentence may be considered evidence of a violation of Article XIII. If a Team delays or attempts to delay in any manner the processing or resolution of a Grievance relating to a player's Compensation for the purpose of creating or increasing its Room in any Salary Cap Year or for the purpose of reducing or deferring a tax payment to the NBA, such conduct shall constitute a violation of Article XIII.
      \end{enumerate}
    \item
      Salaries anticipated to be included in Team Salary based upon any agreement disclosed to the NBA pursuant to Article II, Section 13(a)(i) (including, without limitation, any executed Player Contract whose validity is conditional on the passage of a physical examination by the player or on the assignment of the Contract), except to the extent that any such Salary is less than a player's Free Agent Amount (as defined in Section 4(d) below).
    \end{enumerate}
  \item
    \begin{enumerate}
    \def\labelenumiii{(\roman{enumiii})}
    \tightlist
    \item
      With respect to each Veteran Free Agent who last played for a Team who is an Unrestricted Free Agent, the Free Agent Amount (as defined in Section 4(d) below) attributable to such Veteran Free Agent.
    \item
      With respect to each Veteran Free Agent who last played for a Team who is a Restricted Free Agent, the greater of (A) the Free Agent Amount (as defined in Section 4(d) below) attributable to such Veteran Free Agent, (B) the Salary called for in any outstanding Qualifying Offer (other than a Two-Way Qualifying Offer, as defined in Article XI, Section 1(e)(iii)(B) below) tendered to such Veteran Free Agent (or, if the Restricted Free Agent was also tendered a Maximum Qualifying Offer pursuant to Article XI, Section 4(a)(ii), the Salary called for in such outstanding Maximum Qualifying Offer), or (C) the Salary called for in any First Refusal Exercise Notice (as defined in Article XI, Section 5(g)) issued with respect to such Veteran Free Agent.
    \end{enumerate}
  \item
    The aggregate Salaries called for under all outstanding Offer Sheets (as defined in Article XI, Section 5(b)).
  \item
    An amount with respect to a Team's unsigned First Round Pick, if any, as determined in accordance with Section 4(e) below.
  \item
    An amount with respect to the number of players fewer than twelve (12) included in a Team's Team Salary, as determined in accordance with Section 4(f) below.
  \item
    Value or consideration received by retired players that is determined to be includable in Team Salary in accordance with Article XIII, Section 5.
  \item
    The amount of any Salary Cap Exception that is deemed included in Team Salary in accordance with Section 6(n)(2) below.
  \item
    An amount, if any, included in Team Salary in accordance with the Minimum Team Salary rules set forth in Section 2(c)(3) above.
  \end{enumerate}
\item
  \textbf{Expansion.} The Salary of any player selected by an Expansion Team in an expansion draft and terminated in accordance with the NBA waiver procedure before the first day of the Expansion Team's first Season shall not be included in the Expansion Team's Team Salary, except, to the extent such Salary is paid, for purposes of determining whether the Expansion Team has satisfied its Minimum Team Salary obligation for such Season.
\item
  \textbf{Assigned Contracts.} For purposes of calculating Team Salary, with respect to any Player Contract that is assigned, the assignee Team shall, upon assignment, have included in its Team Salary the entire Salary for the then-current Salary Cap Year and for all future Salary Cap Years.
\item
  \textbf{Free Agents.} Subject to Section 4(a)(2)(ii) above, until a Team's Veteran Free Agent re-signs with his Team, signs with another NBA Team, or is renounced, he will be included in his Prior Team's Team Salary at one of the following amounts (``Free Agent Amounts''):

  \begin{enumerate}
  \def\labelenumii{(\arabic{enumii})}
  \item
    \begin{enumerate}
    \def\labelenumiii{(\roman{enumiii})}
    \tightlist
    \item
      A Qualifying Veteran Free Agent, other than a Qualifying Veteran Free Agent described in Section 4(d)(1)(ii) below, will be included at one hundred fifty percent (150\%) of his prior Salary if it was equal to or greater than the Estimated Average Player Salary for the prior Salary Cap Year, and one hundred ninety percent (190\%) of his prior Salary if it was less than the Estimated Average Player Salary for the prior Salary Cap Year.
    \item
      A Qualifying Veteran Free Agent following the second Option Year of his Rookie Scale Contract will be included at two hundred fifty percent (250\%) of the player's prior Salary if it was equal to or greater than the Estimated Average Player Salary for the prior Salary Cap Year, and three hundred percent (300\%) of his prior Salary if it was less than the Estimated Average Player Salary for the prior Salary Cap Year.
    \end{enumerate}
  \item
    An Early Qualifying Veteran Free Agent will be included at one hundred thirty percent (130\%) of his prior Salary; provided, however, that the player's Prior Team may, by written notice to the NBA, renounce its rights to sign the player pursuant to the Early Qualifying Veteran Free Agent Exception, in which case the player will be deemed a Non-Qualifying Veteran Free Agent for purposes of this Section 4(d) and Sections 6(b) and 6(j)(5) below.
  \item
    A Non-Qualifying Veteran Free Agent will be included at one hundred twenty percent (120\%) of his prior Salary.
  \item
    Notwithstanding Sections 4(d)(1)-(3) above, if the player's prior Salary was equal to or less than the Minimum Player Salary applicable to such player, he will be included at the portion of the then-current Minimum Annual Salary applicable to such player that would not be reimbursed out of the League-wide benefits fund described in Article IV, Section 6(h).
  \item
    Notwithstanding Sections 4(d)(1)-(3) above, at no time shall a player's Free Agent Amount exceed the Maximum Player Salary applicable to such player or be less than the portion of the Minimum Annual Salary applicable to such player that would not be reimbursed out of the League-wide benefits fund described in Article IV, Section 6(h).
  \item
    Notwithstanding Sections 4(d)(1)-(3) above, at no time shall a Free Agent Amount for a Veteran Free Agent following the second or third Season of his Rookie Scale Contract exceed the maximum amount the Team may pay the player pursuant to Section 6(n)(3) below.
  \item
    Notwithstanding Sections 4(d)(1)-(5) above, if a Two-Way Player completes a Two-Way Contract, the player's Free Agent Amount will be the Minimum Annual Salary applicable to a player completing a Standard NBA Contract for the zero (0) Years of Service Minimum Annual Salary.
  \item
    For purposes of this Section 4(d) only, a player's ``prior Salary'' means his Regular Salary for the prior Season plus any signing bonus allocation and the amount of any Incentive Compensation actually earned for such Season under the Player Contract in effect when the player finished the prior Season.
  \end{enumerate}
\item
  \textbf{First Round Picks.}

  \begin{enumerate}
  \def\labelenumii{(\arabic{enumii})}
  \tightlist
  \item
    A First Round Pick, immediately upon selection in the Draft, shall be included in the Team Salary of the Team that holds his draft rights at one hundred twenty percent (120\%) of his applicable Rookie Scale Amount (``Rookie Scale Cap Hold Amount''), and, subject to Sections 4(e)(2) and (3) below, shall continue to be included in the Team Salary of any Team that holds his draft rights (including any Team to which the player's draft rights are assigned) until such time as the player signs with such Team or until the Team loses or assigns its exclusive draft rights to the player.
  \item
    In the event that a First Round Pick signs with a non-NBA team, the player's applicable Rookie Scale Cap Hold Amount shall be excluded from the Team Salary of the Team that holds his draft rights beginning on the date he signs such non-NBA contract or the first day of the Regular Season, whichever is later, and shall be included again in his Team's Team Salary at the applicable Rookie Scale Cap Hold Amount on the following July 1 or the date the player's contract ends (or the player is released from his non-NBA contractual obligations), whichever is earlier, unless the Team renounces its exclusive rights to the player in accordance with Article X, Section 4(g). If, after such following July 1, or any subsequent July 1, the player signs another, or remains under, contract with a non-NBA team, the player's applicable Rookie Scale Cap Hold Amount will again be excluded from Team Salary beginning on the date of the contract signing or the first day of the Regular Season commencing after such July 1, whichever is later, and will again be included in Team Salary at the applicable Rookie Scale Cap Hold Amount on the following July 1 or the date the player's contract ends (or the player is released from his non-NBA contractual obligations), whichever is earlier, unless the Team renounces its exclusive rights to the player in accordance with Article X, Section 4(g).
  \item
    A Team that holds draft rights to a First Round Pick may elect to have the player's applicable Rookie Scale Cap Hold Amount excluded from its Team Salary at any time prior to the first day of any Regular Season by providing the NBA with a written statement that the Team will not sign the player during that Salary Cap Year accompanied by a written statement from the First Round Pick renouncing his right to accept any outstanding Required Tender made to him by the Team. After making such an election, (i) the Team shall be prohibited from signing the player during that Salary Cap Year, except in accordance with Section 5(c)(4)(ii) below, (ii) the Team shall continue to possess such rights with respect to the player that the Team possessed pursuant to Article X immediately prior to such election, and (iii) the player's applicable Rookie Scale Amount shall be included again in his Team's Team Salary at the applicable Rookie Scale Cap Hold Amount on the following July 1. When a First Round Pick provides a Team with a written statement renouncing his right to accept that year's outstanding Required Tender, the Player shall no longer be permitted to accept it.
  \item
    For purposes of this Section 4(e), in the event that a First Round Pick does not sign a Contract with the Team that holds his draft rights during the Salary Cap Year immediately following the Draft in which he was selected (or during the same Salary Cap Year in which he was drafted if the Draft occurs on or after July 1), the ``applicable Rookie Scale Amount'' for such First Round Pick means, with respect to any subsequent Salary Cap Year, the Rookie Scale Amount that would apply if the player were drafted in the Draft immediately preceding such Salary Cap Year at the same draft position at which he was actually selected.
  \end{enumerate}
\item
  \textbf{Incomplete Rosters.}

  \begin{enumerate}
  \def\labelenumii{(\arabic{enumii})}
  \item
    If at any time from July 1 through the day prior to the first day of the Regular Season a Team has fewer than twelve (12) players, determined in accordance with Section 4(f)(2) below, included in its Team Salary, then the Team's Team Salary shall be increased by an amount calculated as follows:

    \begin{itemize}
    \item
      STEP 1: Subtract from twelve (12) the number of players included in Team Salary.
    \item
      STEP 2: If the result in Step 1 is a positive number, multiply the result in Step 1 by the Minimum Annual Salary applicable to players with zero (0) Years of Service under the Minimum Annual Salary Scale for that Salary Cap Year.
    \end{itemize}
  \item
    In determining whether a Team has fewer than twelve (12) players included in its Team Salary for purposes of Section 4(f)(1) above only, the only players who shall be counted are (i) players under Contract with the Team who are included in Team Salary, (ii) Free Agents who are included in Team Salary pursuant to Section 4(a)(2) above, (iii) players to whom Offer Sheets have been given, and (iv) unsigned First Round Picks who are included in Team Salary pursuant to Section 4(e) above.
  \end{enumerate}
\item
  \textbf{Renouncing.}

  \begin{enumerate}
  \def\labelenumii{(\arabic{enumii})}
  \tightlist
  \item
    To renounce a Veteran Free Agent, a Team must provide the NBA with a written statement renouncing its right to re-sign the player, effective no earlier than the July 1 following the last Season covered by the player's Contract. (The NBA shall notify the Players Association of any such renunciation by email within two (2) business days following receipt of notice of such renunciation.) If a Team renounces a Veteran Free Agent, the player will no longer qualify as a Qualifying Veteran Free Agent, Early Qualifying Veteran Free Agent, or Non-Qualifying Veteran Free Agent, as the case may be, and the Team will only be permitted to re-sign such player with Room (i.e., the Team cannot sign such player pursuant to Section 6(b) below), pursuant to the Minimum Player Salary Exception, or to a Two-Way Contract. Notwithstanding the foregoing, in the event a Team renounces one or more players pursuant to this Section 6(g) (or, with respect to a First Round Pick, pursuant to Article X, Section 4(g)) in order to create Room for an Offer Sheet, and the offeree-player's Prior Team subsequently matches the Offer Sheet and enters into a Contract with that player, the Team may rescind the renunciation (in the case where a Team renounces only one player) or all such renunciations (in the case where the Team renounces more than one player) within two (2) business days of the date the Offer Sheet is matched (or, if the Prior Team conditions its match on the player reporting for and passing a physical, within two (2) business days of the player passing the physical), whereupon any such ``unrenounced'' player may again sign a Player Contract with the Team as a First Round Pick, Qualifying Veteran Free Agent, Early Qualifying Veteran Free Agent, or Non-Qualifying Veteran Free Agent, as the case may be, and will again be included in his Prior Team's Team Salary at his applicable Free Agent Amount; provided, however, that a Team may not rescind the renunciation of a player if (i) at the time the player was renounced, the Team's Team Salary was at or below the Salary Cap and ``unrenouncing'' the player would cause the Team's Team Salary to exceed the Salary Cap, or (ii) at the time the player was renounced, the Team's Team Salary was above the Salary Cap and ``unrenouncing'' the player would cause the Team's Team Salary to exceed the Salary Cap by more than the amount by which the Team's Team Salary exceeded the Salary Cap prior to the renunciation.
  \item
    A Team cannot renounce any player who is a Restricted Free Agent.
  \end{enumerate}
\item
  \textbf{Long-Term Injuries.} Any player who suffers a career-ending injury or illness, and whose Contract is terminated by the Team in accordance with the NBA waiver procedure, will be excluded from his Team's Team Salary as follows:

  \begin{enumerate}
  \def\labelenumii{(\arabic{enumii})}
  \tightlist
  \item
    Subject to Section 4(h)(5) below, a Team may apply to the NBA to have the player's Salary for each remaining Salary Cap Year covered by the Contract excluded from Team Salary beginning on the first anniversary of the date of the last Regular Season, Play-In, or playoff game in which the player played; provided, however, that if the player played in fewer than ten (10) Regular Season, Play-In, and playoff games in the last Season in which he played, then the earliest date upon which a Team may apply to the NBA to have the player's Salary excluded from its Team Salary in accordance with this Section 4(h) shall be the later of (A) sixty (60) days following the date during such Season in which the player last played in a Regular Season, Play-In, or playoff game, and (B) the first anniversary of the date during a prior Season in which the player last played in a Regular Season, Play-In, or playoff game under such Contract. Notwithstanding anything to the contrary in this Section 4(h)(1), a Team may not apply to have a player's Salary excluded from Team Salary prior to the first anniversary of the date of the first Regular Season game that the player is on the Team's roster under the Contract in question.
  \item
    The determination of whether a player has suffered a career-ending injury or illness shall be made by a physician selected jointly by the NBA and the Players Association or, upon agreement of the NBA and the Players Association, a Fitness-to-Play Panel established under Article XXII. A player shall be deemed to have suffered a career-ending injury or illness if it is determined (i) by such a physician or Fitness-to-Play Panel that the player has an injury or illness that (x) prevents him from playing skilled professional basketball at an NBA level for the duration of his career, or (y) substantially impairs his ability to play skilled professional basketball at an NBA level and is of such severity that continuing to play professional basketball at an NBA level would subject the player to medically unacceptable risk of suffering a life-threatening or permanently disabling injury or illness, or (ii) by such Fitness-to-Play Panel that the player has an injury or illness that would create a materially elevated risk of death, paralysis, or other permanent spinal injury for the player under the procedures set forth in Article XXII, Section 11.
  \item
    Notwithstanding Sections 4(h)(1) and (2) above, if after a player's Salary is excluded from Team Salary in accordance with this Section 4(h), the player plays in twenty-five (25) NBA Regular Season, Play-In, and playoff games in any Season for any Team, the excluded Salary for the Salary Cap Year covering such Season and each subsequent Salary Cap Year shall thereupon be included in Team Salary of the Team from which the Salary was previously excluded (and if the twenty-fifth (25th) game played is a Play-In or playoff game, then the excluded Salary shall be included in Team Salary retroactively as of the start of the Team's last Regular Season game); provided, however, that the foregoing sentence shall not apply in the event a player is determined to have suffered a career-ending injury or illness pursuant to Section 4(h)(2)(ii) above. After a player's Salary for one (1) or more Salary Cap Years has been included in Team Salary in accordance with this Section 4(h)(3), the Team shall be permitted to re-apply to have the player's Salary (for each Salary Cap Year remaining at the time of the re-application) excluded from Team Salary in accordance with the rules set forth in this Section 4(h) (including the waiting period criteria set forth in Section 4(h)(1) above).
  \item
    If a Team applies to have a player's Salary excluded from its Team Salary pursuant to this Section 4(h), the player shall cooperate in the processing of the application, including by appearing at the reasonably scheduled place and time for examination by the jointly-selected physician. The player shall not make any misrepresentation or fail to disclose any relevant information in connection with the processing of the application.
  \item
    Only the Team with which the player was under Contract at the time his career-ending injury or illness became known or reasonably should have become known shall be permitted to apply to have the player's Salary excluded from Team Salary pursuant to this Section 4(h). A Team may only apply to have a player's Salary excluded from its Team Salary pursuant to this Section 4(h) during the term covered by the player's Contract. For clarity, if a player's Salary is excluded from Team Salary pursuant to this Section 4(h), if, at the time of such exclusion, the Team has previously elected to stretch any Salary in respect of one or more current or future Salary Cap Years pursuant to Section 7(d)(6), such stretched Salary shall also be excluded.
  \item
    Notwithstanding anything to the contrary in this Agreement, (i) if a Team applies to have a player's Salary excluded from its Team Salary pursuant to this Section 4(h) and such application is granted, the Team will be prohibited from re-signing or re-acquiring that player at any time, and (ii) if a Team makes a request for an Exception to replace a Disabled Player pursuant to Section 6(c) below for a Salary Cap Year, then, whether such application is granted or denied, the Team will be precluded from applying to have that player's Salary excluded from its Team Salary pursuant to this Section 4(h) for the same Salary Cap Year.
  \end{enumerate}
\item
  \textbf{Summer Contracts.}

  \begin{enumerate}
  \def\labelenumii{(\arabic{enumii})}
  \tightlist
  \item
    Except as provided in Section 4(i)(2) below and subject to Article II, Section 15, from July 1 until the day prior to the first day of the next Regular Season, a Team may enter into Player Contracts that will not be included in Team Salary until the first day of such Regular Season (i.e., the player will be deemed not to have any Salary until the first day of such Regular Season), provided that such Contracts satisfy the requirements of this Section 4(i) (each such Contract, a ``Summer Contract''). Except as set forth in the following sentence, no Summer Contract may provide for (i) Compensation of any kind that is or may be paid or earned prior to the first day of the next Regular Season, or (ii) Compensation protection of any kind pursuant to Article II, Section 3(i) or 4. The only consideration that may be provided to a player signed to a Summer Contract, prior to the start of the Regular Season, is per diem, lodging, transportation, compensation in accordance with Paragraph 3(b) of the Uniform Player Contract, and a disability insurance policy covering disabilities incurred while such player participates in summer leagues or rookie camps for the Team. A Team that has entered into one or more Summer Contracts must terminate such Contracts no later than the day prior to the first day of a Regular Season, except to the extent the Team has Room for such Contracts or is entitled to use the Minimum Player Salary Exception.
  \item
    A Team may not enter into a Summer Contract with a Veteran Free Agent who last played for the Team unless the Contract is for one (1) Season only and provides for no more than the Minimum Player Salary applicable to such player.
  \end{enumerate}
\item
  \textbf{Two-Way Contracts.} Two-Way Player Salaries shall be excluded from Team Salary. Thus, for example, a Team is not required to have Room or an Exception to sign, acquire, or convert a player to a Two-Way Contract.
\item
  \textbf{Exhibit 10 Bonus.} Any amounts earned by a player pursuant to an Exhibit 10 Bonus shall be excluded from Team Salary.
\item
  \textbf{Second Round Pick Exception.} Subject to Article II, Section 15, each Salary Cap Year, from July 1 through July 30, if a Team signs a Player Contract pursuant to the Second Round Pick Exception, such Contract will not be included in Team Salary until July 31 of such Salary Cap Year (i.e., the player will be deemed not to have any Salary until such July 31).
\item
  \textbf{Team Salary Summaries.}

  \begin{enumerate}
  \def\labelenumii{(\arabic{enumii})}
  \tightlist
  \item
    The NBA shall provide the Players Association with Team Salary summaries and a list of current Exceptions twice a month during the Regular Season and once every week during the off-season.
  \item
    In the event that the NBA fails to provide the Players Association with any Team Salary summary or list of Exceptions as provided for in Section 4(m)(1) above, the Players Association shall notify the NBA of such failure, and the NBA, upon receipt of such notice, shall as soon as reasonably possible, but in no event later than two (2) business days following receipt of such notice, provide the Players Association with any such summary or list that should have been provided pursuant to Section 4(m)(1) above.
  \end{enumerate}
\end{enumerate}

\hypertarget{salary-cap-contract-structure-rules.}{%
\section{Salary Cap Contract Structure Rules.}\label{salary-cap-contract-structure-rules.}}

\begin{enumerate}
\def\labelenumi{(\alph{enumi})}
\tightlist
\item
  \textbf{Annual Salary Increases and Decreases.}

  \begin{enumerate}
  \def\labelenumii{(\arabic{enumii})}
  \tightlist
  \item
    The following rules apply to all Player Contracts other than Contracts between Qualifying Veteran Free Agents or Early Qualifying Veteran Free Agents and their Prior Team:

    \begin{enumerate}
    \def\labelenumiii{(\roman{enumiii})}
    \tightlist
    \item
      For each Salary Cap Year covered by a Player Contract after the first Salary Cap Year, the player's: (A) Salary, excluding Incentive Compensation, may increase or decrease in relation to the previous Salary Cap Year's Salary, excluding Incentive Compensation, by no more than five percent (5\%) of the Salary for the first Salary Cap Year covered by the Contract; and (B) Regular Salary may increase or decrease in relation to the previous Salary Cap Year's Regular Salary by no more than five percent (5\%) of the Regular Salary for the first Salary Cap Year covered by the Contract.
    \item
      In the event that the first Salary Cap Year covered by a Contract provides for Incentive Compensation, the amount of each bonus included in the first Salary Cap Year of the Contract may increase or decrease in each subsequent Salary Cap Year by up to five percent (5\%) of the amount of such bonus in the first Salary Cap Year of the Contract.
    \end{enumerate}
  \item
    The following rules apply to all Player Contracts between Qualifying Veteran Free Agents or Early Qualifying Veteran Free Agents and their Prior Team (except any such Contracts signed pursuant to Section 6(d)(4), Section 6(e)(4), Section 6(f)(3), Section 6(g)(4), or Section 8(e)(1) below, which shall be governed by Section 5(a)(1) above):

    \begin{enumerate}
    \def\labelenumiii{(\roman{enumiii})}
    \tightlist
    \item
      For each Salary Cap Year covered by a Player Contract after the first Salary Cap Year, the player's: (A) Salary, excluding Incentive Compensation, may increase or decrease in relation to the previous Salary Cap Year's Salary, excluding Incentive Compensation, by no more than eight percent (8\%) of the Salary for the first Salary Cap Year covered by the Contract; and (B) Regular Salary may increase or decrease in relation to the previous Salary Cap Year's Regular Salary by no more than eight percent (8\%) of the Regular Salary for the first Salary Cap Year covered by the Contract.
    \item
      In the event that the first Salary Cap Year covered by a Contract provides for Incentive Compensation, the amount of each bonus included in the first Salary Cap Year of the Contract may increase or decrease in each subsequent Salary Cap Year by up to eight percent (8\%) of the amount of such bonus in the first Salary Cap Year of the Contract.
    \end{enumerate}
  \item
    The following rules apply to all Extensions other than Extensions entered into in connection with a trade pursuant to Section 8(e)(2) below:

    \begin{enumerate}
    \def\labelenumiii{(\roman{enumiii})}
    \tightlist
    \item
      For each Salary Cap Year covered by an Extension after the first Salary Cap Year covered by the extended term, the player's: (A) Salary, excluding Incentive Compensation, may increase or decrease in relation to the previous Salary Cap Year's Salary, excluding Incentive Compensation, by no more than eight percent (8\%) of the Salary for the first Salary Cap Year covered by the extended term of the Contract; and (B) Regular Salary may increase or decrease in relation to the previous Salary Cap Year's Regular Salary by no more than eight percent (8\%) of the Regular Salary for the first Salary Cap Year covered by the Contract.
    \item
      In the event that the first Salary Cap Year covered by the extended term of a Contract provides for Incentive Compensation, the amount of each bonus included in the first Salary Cap Year of the extended term may increase or decrease in each subsequent Salary Cap Year by up to eight percent (8\%) of the amount of such bonus in the first Salary Cap Year of the extended term.
    \end{enumerate}
  \item
    The following rules apply to Extensions entered into in connection with a trade pursuant to Section 8(e)(2) below:

    \begin{enumerate}
    \def\labelenumiii{(\roman{enumiii})}
    \tightlist
    \item
      For each Salary Cap Year covered by an Extension after the first Salary Cap Year covered by the extended term, the player's: (A) Salary, excluding Incentive Compensation, may increase or decrease in relation to the previous Salary Cap Year's Salary, excluding Incentive Compensation, by no more than five percent (5\%) of the Salary for the first Salary Cap Year covered by the extended term of the Contract; and (B) Regular Salary may increase or decrease in relation to the previous Salary Cap Year's Regular Salary by no more than five percent (5\%) of the Regular Salary for the first Salary Cap Year covered by the Contract.
    \item
      In the event that the first Salary Cap Year covered by the extended term of a Contract provides for Incentive Compensation, the amount of each bonus included in the first Salary Cap Year of the extended term may increase or decrease in each subsequent Salary Cap Year by up to five percent (5\%) of the amount of such bonus in the first Salary Cap Year of the extended term.
    \end{enumerate}
  \item
    For purposes of Sections 5(a)(1)(ii), 5(a)(2)(ii), 5(a)(3)(ii), and 5(a)(4)(ii), in the event that the first Salary Cap Year covered by the Contract or extended term, as applicable, provides for Incentive Compensation, the criteria for earning any bonus included in such Salary Cap Year must be unchanged in any subsequent Salary Cap Year.(6) The foregoing rules set forth above in this Section 5 shall not apply to Two-Way Contracts, which are subject to the rules set forth in Article II, Section 11(a).
  \end{enumerate}
\item
  \textbf{Performance Bonuses.}

  \begin{enumerate}
  \def\labelenumii{(\arabic{enumii})}
  \tightlist
  \item
    Notwithstanding any other provision of this Agreement, no Player Contract may provide for Unlikely Bonuses in any Salary Cap Year that exceed fifteen percent (15\%) of the player's Regular Salary for such Salary Cap Year at the time the Contract is signed; provided, however, that: (i) with respect to Extensions, if the amount of Unlikely Bonuses in the Salary Cap Year in which the Extension is signed exceeds fifteen percent (15\%) of the player's Regular Salary for such Salary Cap Year, the Extension may provide for up to the same percentage of Unlikely Bonuses in the first year of the extended term; and (ii) no Renegotiation may provide for an increase in Unlikely Bonuses if, after the Renegotiation, the amount of Unlikely Bonuses in respect of any Salary Cap Year covered by the renegotiated Contract exceeds fifteen percent (15\%) of the player's Regular Salary for such Salary Cap Year.
  \item
    No Player Contract may provide for any Unlikely Bonus for the first Salary Cap Year covered by the Contract that, if included in the player's Salary for such Salary Cap Year, would result in the Team's Team Salary exceeding the Room under which it is signing the Contract. For the sole purpose of determining whether a Team has Room for a new Unlikely Bonus, the Team's Room shall be deemed reduced by all Unlikely Bonuses in Contracts approved by the Commissioner that may be paid to all of the Team's players that entered into Player Contracts (including Renegotiations) during that Salary Cap Year.
  \end{enumerate}
\item
  \textbf{No Futures Contracts.} Subject to Section 5(c)(4) below, but notwithstanding any other provision in this Agreement:

  \begin{enumerate}
  \def\labelenumii{(\arabic{enumii})}
  \item
    Every Player Contract must cover at least the then-current Season (or the upcoming Season in the case of a Contract entered into from July 1 through the day prior to the first day of the Season).
  \item
    No Team and player may enter into a Player Contract from the commencement of the Team's last game of the Regular Season through the following June 30. The preceding sentence shall not prohibit a Team and player from entering into an amendment to an existing Player Contract during such period if such amendment would otherwise be permitted under this Agreement.
  \item
    A Player Contract that covers more than one (1) Season must be for a consecutive period of Seasons.
  \item
    \begin{enumerate}
    \def\labelenumiii{(\roman{enumiii})}
    \tightlist
    \item
      A player who receives a Required Tender or a Qualifying Offer during the month of June may accept such Required Tender or Qualifying Offer beginning on the date he receives it.
    \item
      From February 1 through June 30 of any Salary Cap Year, a First Round Pick may enter into a Rookie Scale Contract commencing with the following Season, provided that as of or at any point following the first day of the then-current Regular Season (or the preceding Regular Season in the case of a Contract signed from the day following the last day of the Regular Season through June 30) the player was a party to a player contract with a professional basketball team or league not in the NBA covering such Regular Season. With respect to any Rookie Scale Contract entered into pursuant to this Section 5(e)(4)(ii) and subject to the provisions in Article VII and VIII: (i) the Rookie Salary Scale applicable to such Contract shall be the Rookie Salary Scale for the Salary Cap Year encompassing the first Season covered by the player's Contract; (ii) in lieu of providing for Compensation for each Season covered by the Contract as a specific dollar amount, teams must state in Exhibit 1 of the Contract that the player's Current Base Compensation and, if applicable, Incentive Compensation for each such Season shall be ``{[}\_\_\_{]}\% of the player's applicable Rookie Salary Scale''; and (iii) the player's Base Compensation protection shall be expressed in terms of a percentage of the player's Base Compensation.
    \end{enumerate}
  \end{enumerate}
\end{enumerate}

\hypertarget{exceptions-to-the-salary-cap.}{%
\section{Exceptions to the Salary Cap.}\label{exceptions-to-the-salary-cap.}}

There shall be the following exceptions to the rule that a Team's Team Salary may not exceed the Salary Cap:

\begin{enumerate}
\def\labelenumi{(\alph{enumi})}
\tightlist
\item
  \textbf{Existing Contracts.} A Team may exceed the Salary Cap to the extent of its current contractual commitments, provided that such contracts satisfied the provisions of this Agreement when entered into or were entered into prior to the effective date of this Agreement in accordance with the rules then in effect.
\item
  \textbf{Veteran Free Agent Exception.} Subject to the rules set forth in Section 6(n) below, beginning at 12:01 p.m. eastern time on the last day of the Moratorium Period following the last Season covered by a Veteran Free Agent's Player Contract, such player may enter into a new Player Contract with his Prior Team (or, in the case of a player selected in an Expansion Draft that year, with the Team that selected such player in an Expansion Draft) as follows:

  \begin{enumerate}
  \def\labelenumii{(\arabic{enumii})}
  \item
    If the player is a Qualifying Veteran Free Agent, the new Player Contract may provide for Salary plus Unlikely Bonuses in the first Salary Cap Year totaling up to the maximum amount provided for in Article II, Section 7. Annual increases and decreases in Salary and Unlikely Bonuses shall be governed by Section 5(a)(2) above.
  \item
    If the player is a Non-Qualifying Veteran Free Agent, then, subject to Article II, Section 7, the new Player Contract may provide in the first Salary Cap Year up to the greater of: (i) one hundred twenty percent (120\%) of the Regular Salary for the final Salary Cap Year of the player's prior Contract, plus one hundred twenty percent (120\%) of any Likely Bonuses and Unlikely Bonuses, respectively, called for in the final Salary Cap Year covered by the player's prior Contract; (ii) Salary plus Unlikely Bonuses totaling one hundred twenty percent (120\%) of the then-current Minimum Annual Salary applicable to the player; or (iii) in the case of a Contract between a Team and its Restricted Free Agent, the Salary and Unlikely Bonuses required to be provided in a Qualifying Offer. Annual increases and decreases in Salary and Unlikely Bonuses shall be governed by Section 5(a)(1) above.
  \item
    \begin{enumerate}
    \def\labelenumiii{(\roman{enumiii})}
    \tightlist
    \item
      If the player is an Early Qualifying Veteran Free Agent, the new Player Contract must cover at least two (2) Seasons (not including a Season covered by an Option Year) and, subject to Article II, Section 7, may provide in the first Salary Cap Year up to the greater of: (A) one hundred seventy-five percent (175\%) of the Regular Salary for the final Salary Cap Year covered by his prior Contract, plus one hundred seventy-five percent (175\%) of any Likely Bonuses and Unlikely Bonuses, respectively, called for in the final Salary Cap Year covered by the player's prior Contract, or (B) Salary plus Unlikely Bonuses totaling an amount equal to one hundred five percent (105\%) of the Average Player Salary for the prior Salary Cap Year (or if the Audit Report for the prior Salary Cap Year has not been completed, one hundred five percent (105\%) of the Average Player Salary for the prior Salary Cap Year as computed by substituting Estimated Total Salaries (as defined in Section 1(i) above) for Total Salaries). Annual increases and decreases in Salary and Unlikely Bonuses shall be governed by Section 5(a)(2) above.
    \item
      Notwithstanding anything to the contrary in Section 5(a)(2) above or this Section 6(b)(3), if an Early Qualifying Veteran Free Agent with two (2) Years of Service receives an Offer Sheet in accordance with the provisions of Article XI, Section 5(d), the player's Prior Team may use the Early Qualifying Veteran Free Agent Exception to match the Offer Sheet.
    \end{enumerate}
  \end{enumerate}
\item
  \textbf{Disabled Player Exception.}

  \begin{enumerate}
  \def\labelenumii{(\arabic{enumii})}
  \tightlist
  \item
    Subject to the rules set forth in Section 6(n) below, a Team may, in accordance with the rules set forth in this Section 6(c), sign or acquire one Replacement Player to replace a player who, as a result of a Disabling Injury or Illness (as defined below), is unable to render playing services (the ``Disabled Player'').

    \begin{enumerate}
    \def\labelenumiii{(\roman{enumiii})}
    \tightlist
    \item
      An application for a Disabled Player Exception in respect of a Salary Cap Year, regardless of when the Disabling Injury or Illness occurred, may be made at any time from July 1 through January 15 of such Salary Cap Year.
    \item
      If a Team wishes to sign a Replacement Player pursuant to this Section 6(c), such Replacement Player's Contract may be for one Season and provide Salary and Unlikely Bonuses for the Salary Cap Year in which the player is signed totaling up to the lesser of (A) fifty percent (50\%) of the Disabled Player's Salary for the then-current Salary Cap Year, or (B) an amount equal to Non-Taxpayer Mid-Level Salary Exception (as defined in Section 6(e) below) for such Salary Cap Year.
    \item
      If a Team wishes to acquire a Replacement Player pursuant to this Section 6(c), the Replacement Player must have only one Season remaining on his Player Contract and the Replacement Player's post-assignment Salary for the Salary Cap Year in which the Replacement Player is acquired may be up to the lesser of the amount described in Section 6(c)(1)(ii)(A) above or the amount described in Section 6(c)(1)(ii)(B) above, plus, in either case, \$100,000.
    \end{enumerate}
  \item
    For purposes of this Section 6(c), ``Disabling Injury or Illness'' means any injury or illness that, in the opinion of the physician described in Section 6(c)(4) below, makes it substantially more likely than not that the player would be unable to play through the following June 15.
  \item
    The Exception for a Disabling Injury or Illness shall expire on the March 10 following the date the Exception is granted.
  \item
    The determination of whether a player has suffered a Disabling Injury or Illness shall be made by a physician designated by the NBA, who shall review the relevant medical information and, if the physician deems it appropriate, examine the player. The NBA shall advise the Players Association of the determination of its physician within one (1) business day of such determination. In the event the Players Association disputes the NBA physician's determination, the parties will immediately refer the matter to a neutral physician (to be selected by the parties at the commencement of each Salary Cap Year) to review the relevant medical information and, if the neutral physician deems it appropriate, examine the player. Within three (3) business days of receipt of such information (and examination of the player, if requested), the neutral physician shall make a final determination, which will be final, binding, and unappealable. The cost of the NBA physician will be borne by the NBA. The cost of the neutral physician will be borne equally and jointly by the NBA and the Players Association.
  \item
    If a Team requests an Exception pursuant to this Section 6(c), the player with respect to whom the request is made shall cooperate in the processing of the request, including by appearing at the scheduled place and time for examination by the NBA-appointed physician and, if necessary, the neutral physician. The player shall not make any misrepresentation or fail to disclose any relevant information in connection with the processing of the application.
  \item
    Notwithstanding a Team's receipt of an Exception in respect of a Disabled Player pursuant to this Section 6(c), such player, upon recovering from his injury or illness, may resume playing for the Team. If the player resumes playing for the Team, or is traded, prior to the Team's use of its Exception, the Exception shall be extinguished.
  \item
    The Disabled Player Exception is available only to the Team with which the player was under Contract, and during the term of the Contract that the player was under, at the time his Disabling Injury or Illness became known or reasonably should have become known. In order for a Team to be granted a Disabled Player Exception pursuant to this Section 6(c), the Disabled Player must continue to be on the Team's roster from the time the Team makes an application for the Exception through the date upon which the Exception is granted.
  \item
    If a Team makes a request for an Exception to replace a Disabled Player pursuant to this Section 6(c) and such request is denied, the Team shall not be permitted to make any subsequent request for an Exception to replace the same player pursuant to this Section 6(c) unless ninety (90) days have passed since the first request was denied and the Team establishes that the subsequent request is based on a new injury or an aggravation of the same injury. If a Team makes a request for an Exception to replace a Disabled Player for a Season pursuant to this Section 6(c), then, whether such request is granted or denied, the Team shall be permitted to renew its request for an Exception to replace the Disabled Player for a subsequent Season(s) by applying for another Exception in respect of that player for such Season in accordance with the rules set forth in this Section 6(c).
  \end{enumerate}
\item
  \textbf{Bi-annual Exception.} Subject to the rules set forth in Section 2(e) above and Section 6(n) below:

  \begin{enumerate}
  \def\labelenumii{(\arabic{enumii})}
  \tightlist
  \item
    A Team may use the Bi-annual Exception during a Salary Cap Year to sign and/or acquire by assignment one (1) or more Player Contracts that, in the aggregate, provide for Salaries and Unlikely Bonuses (or in the case of assignment, post-assignment Salaries and Unlikely Bonuses) in the first Salary Cap Year totaling up to 3.32\% of the Salary Cap for such Salary Cap Year; provided, however, that, prior to the first day of the 2024-25 Salary Cap Year, a Team shall not be permitted to use the Bi-annual Exception to acquire a Player Contract by assignment.
  \item
    The term of a Player Contract signed pursuant to the Bi-annual Exception may not exceed two (2) Seasons in length, and the remaining term of a Player Contract acquired by assignment pursuant to the Bi-annual Exception may not exceed two (2) Seasons in length.
  \item
    A Team may not use all or any portion of the Bi-annual Exception (i) if at the time the Team proposes to use the Exception the Team has already used the Mid-Level Salary Exception for Room Teams in that same Salary Cap Year, or (ii) in any two (2) consecutive Salary Cap Years. The prohibition in the preceding sentence against using the Bi-annual Exception or any portion thereof in any two (2) consecutive Salary Cap Years shall apply to the 2022-23 Salary Cap Year (i.e., if a Team used all or any portion of the Bi-annual Exception during the 2022-23 Salary Cap Year, that Team shall not be permitted to use all or any portion of the Bi-annual Exception during the 2023-24 Salary Cap Year).
  \item
    Player Contracts signed pursuant to the Bi-annual Exception covering two (2) Seasons may provide for an increase or decrease in Salary and Unlikely Bonuses for the second Salary Cap Year in accordance with Section 5(a)(1) above.
  \item
    The Bi-annual Exception for a Team, if applicable, shall arise on the first day of a Salary Cap Year and shall expire at the start of the Team's last game of the Regular Season during that Salary Cap Year.
  \end{enumerate}
\item
  \textbf{Non-Taxpayer Mid-Level Salary Exception.} Subject to the rules set forth in Section 2(e) above and Section 6(n) below:

  \begin{enumerate}
  \def\labelenumii{(\arabic{enumii})}
  \tightlist
  \item
    A Team may use the Non-Taxpayer Mid-Level Salary Exception to sign and/or acquire by assignment one (1) or more Player Contracts during each Salary Cap Year that, in the aggregate, provide for Salaries and Unlikely Bonuses (or in the case of assignment, post-assignment Salaries and Unlikely Bonuses) in the first Salary Cap Year totaling up to 9.12\% of the Salary Cap for such Salary Cap Year; provided, however, that, prior to the first day of the 2024-25 Salary Cap Year, a Team shall not be permitted to use the Non-Taxpayer Mid-Level Salary Exception to acquire a Player Contract by assignment.
  \item
    The term of a Player Contract signed pursuant to the Non-Taxpayer Mid-Level Salary Exception may not exceed four (4) Seasons in length, and the remaining term of a Player Contract acquired by assignment pursuant to the Non-Taxpayer Mid-Level Salary Exception may not exceed four (4) Seasons in length.
  \item
    A Team may not use all or any portion of the Non-Taxpayer Mid-Level Salary Exception if at the time the Team proposes to use the Exception the Team has already used the Mid-Level Salary Exception for Room Teams in that same Salary Cap Year.
  \item
    Player Contracts signed pursuant to the Non-Taxpayer Mid-Level Salary Exception may provide for annual increases and decreases in Salary and Unlikely Bonuses in accordance with Section 5(a)(1) above.
  \item
    Notwithstanding anything to the contrary in Section 6(e)(3) above, if a Veteran Free Agent with one (1) or two (2) Years of Service receives an Offer Sheet in accordance with the provisions of Article XI, Section 5(d), the player's Prior Team may use the Non-Taxpayer Mid-Level Salary Exception to match the Offer Sheet.
  \item
    The Non-Taxpayer Mid-Level Salary Exception for a Team shall arise on the first day of each Salary Cap Year and shall expire at the start of the Team's last game of the Regular Season during that Salary Cap Year.
  \end{enumerate}
\item
  \textbf{Taxpayer Mid-Level Salary Exception.} Subject to the rules set forth in Section 2(e) above and Section 6(n) below:

  \begin{enumerate}
  \def\labelenumii{(\arabic{enumii})}
  \tightlist
  \item
    A Team may use the Taxpayer Mid-Level Salary Exception to sign one (1) or more Player Contracts during each Salary Cap Year not to exceed two (2) Seasons in length, that, in the aggregate, provide for Salaries and Unlikely Bonuses in the first Salary Cap Year totaling up to the amounts set forth below, provided that the Team's Apron Team Salary immediately following the Team's use of such Exception exceeds the First Apron Level:
  \end{enumerate}

  \begin{longtable}[]{@{}
    >{\raggedright\arraybackslash}p{(\columnwidth - 2\tabcolsep) * \real{0.2727}}
    >{\raggedright\arraybackslash}p{(\columnwidth - 2\tabcolsep) * \real{0.7273}}@{}}
  \toprule()
  \begin{minipage}[b]{\linewidth}\raggedright
  \end{minipage} & \begin{minipage}[b]{\linewidth}\raggedright
  Taxpayer Mid-Level Salary Exception
  \end{minipage} \\
  \midrule()
  \endhead
  For the 2023-24 Salary Cap Year: & \$5 million \\
  For each subsequent Salary Cap Year: & \$5 million multiplied by a fraction, the numerator of which is the Salary Cap for that Salary Cap Year and the denominator of which is the Salary Cap for the 2023-24 Salary Cap Year \\
  \bottomrule()
  \end{longtable}

  \begin{enumerate}
  \def\labelenumii{(\arabic{enumii})}
  \setcounter{enumii}{1}
  \tightlist
  \item
    A Team may not use all or any portion of the Taxpayer Mid-Level Salary Exception if at the time the Team proposes to use the Exception the Team has already used the Mid-Level Salary Exception for Room Teams in that same Salary Cap Year.
  \item
    Player Contracts signed pursuant to the Taxpayer Mid-Level Salary Exception may provide for annual increases and decreases in Salary and Unlikely Bonuses in accordance with Section 5(a)(1) above.
  \item
    The Taxpayer Mid-Level Salary Exception for a Team shall arise on the first day of each Salary Cap Year and shall expire at the start of the Team's last game of the Regular Season during that Salary Cap Year.
  \item
    In the event that, during a Salary Cap Year, a Team: (i) does not use the Non-Taxpayer Mid-Level Salary Exception to acquire any Player Contracts by assignment; (ii) uses the Non-Taxpayer Mid-Level Salary Exception in order to sign one (1) or more new Player Contracts during a Salary Cap Year, not to exceed two (2) Seasons in length that, in the aggregate, provide for Salaries and Unlikely Bonuses in the first Salary Cap Year of the Contract(s) totaling no more than the amounts set forth in Section 6(f)(1) above, and (iii) but for the Team's use of the Non-Taxpayer Mid-Level Salary Exception as described in clause (ii) above, the Team otherwise would be permitted to engage in a transaction that causes the Team's Apron Team Salary to exceed the First Apron Level for such Salary Cap Year in accordance with the rules set forth in Section 2(e) above, then the Team shall be permitted to engage in such transaction, whereupon the Team will be deemed to have used the Taxpayer Mid-Level Salary Exception instead of the Non-Taxpayer Mid-Level Salary Exception for all purposes under this Article VII, and the Team's ability to use the Non-Taxpayer Mid-Level Salary Exception during such Salary Cap Year shall thereupon be extinguished.
  \end{enumerate}
\item
  \textbf{Mid-Level Salary Exception for Room Teams.} Subject to the rules set forth in Section 6(n) below:

  \begin{enumerate}
  \def\labelenumii{(\arabic{enumii})}
  \tightlist
  \item
    In the event (i) a Team's Team Salary at any time during a Salary Cap Year is below the Salary Cap for such Salary Cap Year such that the Team is not entitled to use the Bi-annual Exception, Non-Taxpayer Mid-Level Salary Exception, or Taxpayer Mid-Level Salary Exception, and (ii) at the time the Team proposes to use the Mid-Level Salary Exception for Room Teams, the Team has not already used either the Bi-annual Exception, the Non-Taxpayer Mid-Level Salary Exception, or the Taxpayer Mid-Level Salary Exception in that same Salary Cap Year, then the Team may at such time use the Mid-Level Salary Exception for Room Teams to sign and/or acquire by assignment one (1) or more Player Contracts that, in the aggregate, provide for Salaries and Unlikely Bonuses (or in the case of assignment, post-assignment Salaries and Unlikely Bonuses) in the first Salary Cap Year totaling up to 5.678\% of the Salary Cap for such Salary Cap Year; provided however, that prior to the first day of the 2024-25 Salary Cap Year, a Team shall not be permitted to use the Mid-Level Salary Exception for Room Teams to acquire a Player Contract by assignment.
  \item
    The term of a Player Contract signed pursuant to the Mid-Level Salary Exception for Room Teams may not exceed three (3) Seasons in length, and the remaining term of a Player Contract acquired by assignment pursuant to the Mid-Level Salary Exception for Room Teams may not exceed three (3) Seasons in length.
  \item
    Once a Team uses the Mid-Level Salary Exception for Room Teams during a Salary Cap Year, the Team will be prohibited from using either the Non-Taxpayer Mid-Level Salary Exception, the Taxpayer Mid-Level Salary Exception, or the Bi-annual Exception at all times thereafter during such Salary Cap Year.
  \item
    Player Contracts signed pursuant to the Mid-Level Salary Exception for Room Teams may provide for annual increases and decreases in Salary and Unlikely Bonuses in accordance with Section 5(a)(1) above.
  \item
    The Mid-Level Salary Exception for Room Teams for a Team shall: (i) arise on the date upon which the Team's Team Salary falls below the Salary Cap for such Salary Cap Year such that the Team is not entitled to use the Bi-annual Exception, the Non-Taxpayer Mid-Level Salary Exception, and the Taxpayer Mid-Level Salary Exception; and (ii) expire at the start of the Team's last game of the Regular Season during that Salary Cap Year.
  \end{enumerate}
\item
  \textbf{Rookie Scale Exception.} A Team may enter into a Rookie Scale Contract in accordance with Article VIII, Section 1.
\item
  \textbf{Minimum Player Salary Exception.} A Team may sign a player to, or acquire by assignment, a Player Contract, not to exceed two (2) Seasons in length, that provides for a Salary for the first Season equal to the Minimum Player Salary applicable to that player (with no bonuses of any kind). A Player Contract signed or acquired pursuant to the Minimum Player Salary Exception covering two (2) Seasons must provide for a Salary for the second Season equal to the Minimum Player Salary applicable to the player for such Season (with no bonuses of any kind).
\item
  \textbf{Traded Player Exception.}

  \begin{enumerate}
  \def\labelenumii{(\arabic{enumii})}
  \item
    Subject to the rules set forth in Section 6(n) below and Section 6(j)(6) below, a Team may acquire one (1) or more players by assignment in accordance with the following:

    \begin{enumerate}
    \def\labelenumiii{(\roman{enumiii})}
    \tightlist
    \item
      Standard Traded Player Exception. Subject to the rules set forth in Section 2(e) above, a Team may use the ``Standard Traded Player Exception'' to replace one (1) Traded Player with one (1) or more Replacement Players whose Player Contracts are acquired simultaneously or non-simultaneously and whose post-assignment Salaries for the Salary Cap Year in which the Replacement Player(s) are acquired, in the aggregate, are no more than an amount equal to one hundred percent (100\%) of the pre-trade Salary of the Traded Player, plus \$250,000, provided that any Player Contract acquired non-simultaneously pursuant to this Exception must be acquired within one (1) year following the date on which the Traded Player was traded.
    \item
      Aggregated Standard Traded Player Exception. Subject to the rules set forth in Section 2(e) above, a Team may use the ``Aggregated Standard Traded Player Exception'' to replace two (2) or more Traded Players with one (1) or more Replacement Players whose Player Contracts are acquired simultaneously and whose post-trade Salaries for the then-current Salary Cap Year, in the aggregate, are no more than an amount equal to one hundred percent (100\%) of the aggregated pre-trade Salaries of the Traded Players, plus \$250,000.
    \item
      Transition Traded Player Exception. During the 2023-24 Salary Cap Year only, and subject to the rules set forth in Section 2(e) above: a Team may use the ``Transition Traded Player Exception'' to replace one (1) or more Traded Players with one (1) or more Replacement Players whose Player Contracts are acquired simultaneously and whose post-trade Salaries for the 2023-24 Salary Cap Year, in the aggregate, are no more than an amount equal to one hundred ten percent (110\%) of the pre-trade Salaries of the Traded Player(s), plus \$250,000.
    \item
      Expanded Traded Player Exception. Subject to the rules set forth in Section 2(e) above, a Team may use the ``Expanded Traded Player Exception'' to replace one (1) or more Traded Players with one (1) or more Replacement Players whose Player Contracts are acquired simultaneously and whose post-trade Salaries for the then-current Salary Cap Year, in the aggregate, are no more than an amount equal to the greater of: (y) the lesser of: (A) two hundred percent (200\%) of the aggregated pre-trade Salaries of the Traded Player(s), plus \$250,000; or (B) one hundred percent (100\%) of the aggregated pre-trade Salaries of the Traded Player(s), plus an amount equal to \$7.5 million multiplied by a fraction, the numerator of which is the Salary Cap for the then-current Salary Cap Year and the denominator of which is the Salary Cap for the 2023-24 Salary Cap Year; or (z) one hundred twenty-five percent (125\%) of the aggregated pre-trade Salaries of the Traded Player(s), plus \$250,000.
    \item
      Room Under Salary Cap Plus \$250,000. Except as provided in Section 6(j)(2) below, and notwithstanding Section 6(n) below, a Team with a Team Salary below the Salary Cap may acquire one (1) or more players by assignment whose post-assignment Salaries, in the aggregate, are no more than an amount equal to the Team's room under the Salary Cap plus \$250,000. For clarity, a Team that acquires one (1) or more players in accordance with this Section 6(j)(1)(v) (or with room under the Salary Cap (i.e., without making use of the additional \$250,000)) may not simultaneously acquire any players in accordance with Sections 6(j)(1)(i)-(iv) above.
    \end{enumerate}
  \item
    In lieu of conducting a trade in accordance with Section 6(j)(1)(v) above, and notwithstanding Section 6(n) below and subject to Section 2(e) above and Section 6(j)(6) below, a Team with a Team Salary below the Salary Cap may conduct a trade in accordance with Sections 6(j)(1)(iii)-(iv) above.
  \item
    Notwithstanding anything to the contrary in Section 6(j)(1) above, if a Team's post-assignment Apron Team Salary would exceed the First Apron Level, then the \$250,000 allowance referenced in each of Sections 6(j)(1)(i)-(v) above shall be reduced to \$0.
  \item
    Notwithstanding anything to the contrary in Section 6(j)(1) above, the following rules will apply when a Team is aggregating the Contracts of two (2) or more Traded Players in a trade pursuant to a Traded Player Exception set forth in Section 6(j)(1)(ii), 6(j)(1)(iii), or 6(j)(1)(iv) above:

    \begin{enumerate}
    \def\labelenumiii{(\roman{enumiii})}
    \tightlist
    \item
      No player whose Player Contract was acquired pursuant to an Exception in the two (2) month period preceding the trade may be among the Traded Players whose Contracts are being aggregated pursuant to Sections 6(j)(ii), 6(j)(iii), or 6(j)(iv) above (for example, if a player were traded to a Team pursuant to an Exception on November 20, 2023, then the player's Contract could not be aggregated with any other Contract for purposes of a trade until January 20, 2024); provided, however, that if a Team acquires a Player Contract pursuant to an Exception on or before December 16 of a Salary Cap Year, then the foregoing restriction shall not apply in the event the player is subsequently traded on or after the day prior to the NBA trade deadline of such Salary Cap Year; and
    \item
      Other than during the period beginning on December 15 of a Salary Cap Year through the NBA trade deadline of such Salary Cap Year, if a Team is aggregating the Contracts of three (3) or more Traded Players in a trade and the number of Replacement Players that the Team is acquiring in respect of such Traded Players is less than the number of such Traded Players, then no more than one (1) of such Traded Players may be a Minimum Traded Player (as defined below). For the purposes of this rule only, a ``Minimum Traded Player'' is a player whose Contract provides for his applicable Minimum Player Salary for the Salary Cap Year in which the trade of his Contract occurs or, if the trade occurs during the period beginning on the day after the last day of the Regular Season of a Salary Cap Year through the last day of such Salary Cap Year, a player whose Contract provides for his applicable Minimum Player Salary in the immediately following Salary Cap Year.
    \end{enumerate}
  \item
    If (x) a Qualifying Veteran Free Agent or Early Qualifying Veteran Free Agent and his prior Team enter into a Player Contract, in accordance with Section 6(b)(1) or (3) above, in connection with an agreement to trade the Contract in accordance with Section 8(e) below, (y) the Team's Team Salary immediately following such Contract signing is above the Salary Cap, and (z) the new Contract to be traded provides for Salary and Unlikely Bonuses for the first Season of such Contract in excess of the Salary and Unlikely Bonuses that could have been provided for by the Contract had the player been a Non-Qualifying Veteran Free Agent and the Contract had been signed pursuant to Section 6(b)(2) above, then for purposes of calculating the assignor Team's Traded Player Exception, the player's Salary shall be deemed equal to the greater of (i) the Salary for the last Season of his preceding Contract, or (ii) fifty percent (50\%) of the Salary for the first Season of his new Contract. For purposes of this Section 6(j)(5), if the player's immediately prior Contract was a one-year Contract that provided for Salary equal to the Minimum Player Salary (with no bonuses of any kind), the player's prior Salary shall include the portion of the Minimum Player Salary, if any, that was reimbursed out of the League-wide benefits fund described in Article IV, Section 6(h).
  \item
    For purposes of calculating a Team's Traded Player Exception under this Section 6(j), a Traded Player's Salary shall be deemed reduced by the amount of the player's unearned Base Compensation that, at the time of the trade, is not fully protected for lack of skill and injury or illness (or may become not fully protected for lack of skill and injury or illness due to additional conditions or limitations set forth in the Exhibit 2 of the player's Contract). For purposes of this Section 6(j)(6):

    \begin{enumerate}
    \def\labelenumiii{(\roman{enumiii})}
    \tightlist
    \item
      With respect to the assignment of Player Contracts occurring during the period from January 8 through the last day of the Regular Season, a Traded Player's Base Compensation for such Season shall be deemed fully protected for lack of skill and injury or illness;
    \item
      With respect to the assignment of a Player Contract that is a one-year Contract that provides for Salary equal to the Minimum Player Salary (with no bonuses of any kind), the player's unearned Base Compensation shall exclude the portion of the Minimum Player Salary, if any, that is reimbursed out of the League-wide benefits fund described in Article IV, Section 6(h); and
    \item
      With respect to the assignment of Player Contracts occurring during the period from the day following the last day of a Regular Season through June 30 of that Salary Cap Year, a Traded Player's Salary will equal the lesser of: (x) the player's Salary for the current Salary Cap Year; and (y) the player's Salary for the subsequent Salary Cap Year reduced by the amount of the player's unearned Base Compensation for the subsequent Salary Cap Year that, at the time of the trade, is not fully protected for lack of skill and injury or illness (or may become not fully protected for lack of skill and injury or illness due to additional conditions or limitations set forth in the Exhibit 2 of the player's Contract).
    \end{enumerate}

    \emph{To illustrate the foregoing, assume that a Team seeks to replace a Traded Player whose Contract provides for (i) Base Compensation and Salary for each of the 2023-24 and 2024-25 Seasons of \$8 million, and (ii) Base Compensation protection for lack of skill and injury or illness equal to \$1 million for each such Season. If the trade of such Traded Player occurs on:}

    \begin{enumerate}
    \def\labelenumiii{(\Alph{enumiii})}
    \setcounter{enumiii}{22}
    \item
      \emph{the day prior to the first day of the 2023-24 Regular Season, the Traded Player's Salary for purposes of calculating the Team's Traded Player Exception under this Section 6(j) would be \$1 million (\$8 million (the player's 2023-24 Salary) reduced by \$7 million (the amount of the player's unearned 2023-24 Base Compensation that is not fully protected for lack of skill and injury or illness at the time of the trade));}
    \item
      \emph{after one-quarter of the 2023-24 Regular Season has elapsed, the Traded Player's Salary for purposes of calculating the Team's Traded Player Exception under this Section 6(j) would be \$2 million (\$8 million (the player's 2023-24 Salary) reduced by \$6 million (\$8 million multiplied by 75\% -- the amount of the player's unearned 2023-24 Base Compensation that is not fully protected for lack of skill and injury or illness at the time of the trade));}
    \item
      \emph{on January 8, 2024, the Traded Player's Salary for purposes of calculating the Team's Traded Player Exception under this Section 6(j) would be \$8 million (\$8 million (the player's 2023-24 Salary) reduced by \$0 (pursuant to Section 6(j)(6)(i) above, the deemed amount of the player's unearned 2023-24 Base Compensation that is not fully protected for lack of skill and injury or illness at the time of the trade)); and}
    \item
      \emph{on the day following the last day of the 2023-24 Regular Season, the Traded Player's Salary for purposes of calculating the Team's Traded Player Exception under this Section 6(j) would be \$1 million (the lesser of: (i) \$8 million (\$8 million (the player's 2023-24 Salary) reduced by \$0 (the amount of the player's unearned 2023-24 Base Compensation that is not fully protected for lack of skill and injury or illness at the time of the trade)), and (ii) \$1 million (\$8 million (the player's 2024-25 Salary) reduced by \$7 million (the amount of the player's unearned 2024-25 Base Compensation that is not fully protected for lack of skill and injury or illness at the time of the trade)).}
    \end{enumerate}
  \item
    Notwithstanding anything to the contrary in this Section 6(j), no Traded Player Exception shall arise from trading a player during a Salary Cap Year if the Team has previously used (or simultaneously uses) a Disabled Player Exception in respect of such player during such Salary Cap Year.
  \item
    The foregoing rules in this Section 6(j) shall not apply to Two-Way Players. Accordingly, for example, a Traded Player Exception will not arise from trading a Two-Way Player.
  \end{enumerate}
\item
  \textbf{Second Round Pick Exception.} A Team that holds the draft rights to a Second Round Pick may use the Second Round Pick Exception to sign such player to a Player Contract in accordance with the following:

  \begin{enumerate}
  \def\labelenumii{(\arabic{enumii})}
  \tightlist
  \item
    The term of a Player Contract signed pursuant to the Second Round Pick Exception must be either: (i) two (2) Seasons with an Option in favor of the Team for a third Season; or (ii) three (3) Seasons with an Option in favor of the Team for a fourth Season.
  \item
    If a Player Contract signed pursuant to the Second Round Pick Exception has a term of two (2) Seasons with an Option in favor of the Team for a third Season, then such Contract must provide for:

    \begin{enumerate}
    \def\labelenumiii{(\roman{enumiii})}
    \tightlist
    \item
      Salary plus Unlikely Bonuses for the first Season of up to the Minimum Player Salary applicable to a player with one (1) Year of Service; and
    \item
      the player's applicable Minimum Player Salary for the second Season and the Option Year.
    \end{enumerate}

    For clarity, the foregoing amounts shall be those as set forth in the Minimum Annual Scale for the Salary Cap Year in which the Contract is signed.
  \item
    If a Player Contract signed pursuant to the Second Round Pick Exception has a term of three (3) Seasons with an Option in favor of the Team for a fourth Season, then such Contract must provide for:

    \begin{enumerate}
    \def\labelenumiii{(\roman{enumiii})}
    \tightlist
    \item
      Salary plus Unlikely Bonuses for the first Season of up to the Minimum Player Salary applicable to a player with two (2) Year of Service;
    \item
      Salary plus Unlikely Bonuses for the second Season of up to the amount shown in the ``Year 2'' column for a player with two (2) Years of Service in the Minimum Annual Salary Scale; and(iii) the player's applicable Minimum Player Salary for the third Season and the Option Year.
    \end{enumerate}

    For clarity, the foregoing amounts shall be those as set forth in the Minimum Annual Scale for the Salary Cap Year in which the Contract is signed.
  \item
    For Player Contracts signed in accordance with Section 6(k)(3) above, annual increases and decreases in Salary and Unlikely Bonuses from the first Season to the second Season shall be governed by Section 5(a)(1) above.
  \end{enumerate}
\item
  \textbf{Reinstatement.} If a player has been dismissed and disqualified from further association with the NBA and subsequently reinstated pursuant to Article XXXIII (Anti-Drug Agreement), the Team for which the player last played may enter into a Player Contract with such player in accordance with the applicable rules set forth in Article XXXIII, Section 13(f) or (g), even if the Team has a Team Salary at or above the Salary Cap or such Player Contract causes the Team to have a Team Salary above the Salary Cap. If, in accordance with the preceding sentence, a Team and a player enter into a Player Contract and such Contract covers more than one (1) Season, annual increases and decreases in Salary and Unlikely Bonuses shall be governed by Section 5(a)(1) above.
\item
  \textbf{Non-Aggregation.} Other than in accordance with Section 6(j) above, a Team may not aggregate or combine any of the Exceptions set forth above in order to sign or acquire one (1) or more players at Salaries greater than that permitted by any one of the Exceptions. If a Team has more than one (1) Exception available at the same time, the Team shall have the right to choose which Exception it wishes to use to sign or acquire a player.
\item
  \textbf{Other Rules.}

  \begin{enumerate}
  \def\labelenumii{(\arabic{enumii})}
  \tightlist
  \item
    A Team shall be entitled to use the Disabled Player Exception, Bi-annual Exception, Non-Taxpayer Mid-Level Salary Exception, Taxpayer Mid-Level Salary Exception, and a Traded Player Exception set forth in Sections 6(c), (d), (e), (f), and (j) above, respectively, except as set forth in Sections 6(j)(1)(v) and 6(j)(2) above, only if, at the time any such Exception would arise and at all times until it is used, the Team's Team Salary, excluding the amount(s) of such Exception and any other Exception that would be included in Team Salary pursuant to Section 6(n)(2) below, is (i) at or above the Salary Cap, or (ii) below the Salary Cap by less than the amount(s) of the Team's Exception(s) (excluding the amount of the Taxpayer Mid-Level Salary Exception unless the Team is no longer able to use the Non-Taxpayer Mid-Level Salary Exception but remains able to use the Taxpayer Mid-Level Salary Exception, in which case the amount of the Taxpayer Mid-Level Salary Exception shall be included).
  \item
    In the event that when a Disabled Player Exception, Bi-annual Exception, Non-Taxpayer Mid-Level Salary Exception (or the Taxpayer Mid-Level Salary Exception instead of the Non-Taxpayer Mid-Level Salary Exception if the Team is no longer able to use the Non-Taxpayer Mid-Level Salary Exception but remains able to use the Taxpayer Mid-Level Salary Exception), and/or a Traded Player Exception arises, the Team's Team Salary is below the Salary Cap (or in the event that, prior to the expiration of any such Exceptions, the Team's Team Salary falls below the Salary Cap) by less than the amount of such Exceptions, then (i) the Team's Team Salary shall include, until the Exceptions are actually used or until the Team no longer is entitled to use the Exceptions, the amount of the Exceptions (or any unused portion of the Exceptions), and (ii) the amount by which the Team's Team Salary is less than the Salary Cap shall thereby be extinguished. When the Disabled Player Exception is used to sign or acquire a player, the Replacement Player's Salary for the Season covered by his Contract, instead of the amount of the Exception, shall be included in Team Salary. When a Bi-annual Exception, Non-Taxpayer Mid-Level Salary Exception, Taxpayer Mid-Level Salary Exception, or Traded Player Exception is used to sign or acquire a player (as applicable), the Salary for the first Season of the signed Contract or the Salary for the then-current Salary Cap Year of the acquired Contract (as applicable), plus any then-unused portion of the Exception (instead of the full amount of the Exception), shall be included in Team Salary. A Team may at any time renounce its rights to use an Exception, in which case the Exception (or any unused portion of the Exception) will no longer be included in Team Salary.
  \item
    Notwithstanding anything to the contrary in this Agreement, if a player is a Veteran Free Agent following the second or third Season of his Rookie Scale Contract (where the first Option Year or second Option Year (as applicable) to extend such Contract was not exercised), then any new Player Contract between the player and the Team that signed him to his Rookie Scale Contract (and/or, if such Contract was subsequently assigned, any such assignee Team) may provide for Regular Salary, Likely Bonuses, and Unlikely Bonuses in the first Salary Cap Year of up to the Regular Salary, Likely Bonuses, and Unlikely Bonuses, respectively, that the player would have received for such Salary Cap Year had his first or second Option Year (as applicable) been exercised. Annual increases and decreases in Salary and Unlikely Bonuses shall be governed by Section 5(a)(2) above.
  \item
    Beginning on January 10 of each Season, each unused Exception, other than the Traded Player Exception, the Minimum Player Salary Exception (which is governed by Section 6(i) above and Article I, Section 1(kk)) and the Disabled Player Exception, shall be reduced daily throughout the remainder of the Regular Season by the amount of the unused Exception as of January 10 multiplied by a fraction, the numerator of which is one (1) and the denominator of which is the total number of days in such Regular Season; provided that the foregoing reduction shall not apply in the event a Team is using the applicable Exception:

    \begin{enumerate}
    \def\labelenumiii{(\roman{enumiii})}
    \tightlist
    \item
      During the period beginning on January 10 of a Salary Cap Year through the date of the NBA trade deadline of such Salary Cap Year; or
    \item
      For purposes of matching an Offer Sheet.
    \end{enumerate}
  \end{enumerate}
\end{enumerate}

\hypertarget{extensions-renegotiations-and-other-amendments.}{%
\section{Extensions, Renegotiations, and Other Amendments.}\label{extensions-renegotiations-and-other-amendments.}}

\begin{enumerate}
\def\labelenumi{(\alph{enumi})}
\tightlist
\item
  \textbf{Veteran Extensions.} No Player Contract, other than a Rookie Scale Contract, may be extended except in accordance with the following:

  \begin{enumerate}
  \def\labelenumii{(\arabic{enumii})}
  \item
    Subject to the rules set forth in Section 7(a)(2) below: (i) a Player Contract covering a term of three (3) or four (4) Seasons (including, for clarity, any Option Year) may be extended no sooner than the second anniversary of the signing (or, as applicable, the Extension) of the Contract; and (ii) a Player Contract covering a term of five (5) or six (6) Seasons (including, for clarity, any Option Year) may be extended no sooner than the third anniversary of the signing (or, as applicable, the Extension) of the Contract. A Player Contract covering a term of one (1) or two (2) Seasons (including, for clarity, any Option Year) may not be extended. If a player and Team seek to enter into an Extension pursuant to this Section 7(a) (other than a Designated Veteran Player Extension in accordance with Section 7(a)(3)(ii) below) more than one (1) year prior to the July 1 preceding the first Season covered by the extended term, then the Extension may only be negotiated and entered into during the off-season (i.e., during the period from July 1 through the day prior to the first day of a Regular Season). Notwithstanding the foregoing, a Player Contract may be extended pursuant to the Designated Veteran Player Extension rules set forth in Article II, Section 7 and Section 7(a)(3)(ii) below no sooner than the third anniversary of the signing of the Contract, and Designated Veteran Player Extensions may only be negotiated and entered into during the off-season. For purposes of this Section 7: (A) to determine the second or third anniversary of the signing of an Extension or Renegotiation, an Extension or Renegotiation entered into during the period from October 2 through the day prior to the first day of the Regular Season of a Salary Cap Year shall be deemed to have been signed on October 1 of such Salary Cap Year; and (B) the number of Seasons covered by a Player Contract that was previously extended shall be the number of Seasons covered by the most-recent Extension.
  \item
    \begin{enumerate}
    \def\labelenumiii{(\roman{enumiii})}
    \tightlist
    \item
      A Player Contract that has been renegotiated to provide for an increase in Salary in any Salary Cap Year covered by the Contract of more than ten percent (10\%) of the player's Salary prior to the Renegotiation, may not subsequently be extended until the third anniversary of the signing of such Renegotiation.
    \item
      A Team and a player shall not be permitted to extend any Player Contract with a term that has been shortened as a result of the player's exercise of an Early Termination Option.
    \item
      Subject to the rules set forth in this Section 7(a): (a) a Contract may be extended following the exercise of an Option by a player or Team; and (b) a Contract may be extended following the non-exercise of an Option by a player or Team only if the extended term covers a minimum of two (2) Seasons (excluding any new Option Year). In order to effectuate an Extension of the types described in this Section 7(a)(2)(iii), a Team and player may amend a Contract to provide simultaneously for the (i) exercise or non-exercise (as applicable) of the Option, and (ii) the Extension.
    \end{enumerate}
  \item
    \begin{enumerate}
    \def\labelenumiii{(\roman{enumiii})}
    \tightlist
    \item
      Subject to Article II, Section 7, a Player Contract extended in accordance with this Section 7(a) (other than an Extension entered into in connection with a trade pursuant to Section 8(e)(2) below or a Designated Veteran Player Extension) may, in the first Salary Cap Year covered by the extended term, provide for a Salary, excluding Incentive Compensation, of up to the greater of: (A) one hundred forty percent (140\%) of the Regular Salary in the last Salary Cap Year covered by the original term of the Contract; or (B) one hundred forty percent (140\%) of the Estimated Average Player Salary for the Salary Cap Year in which the Extension is signed (or, if the Extension provides for any Incentive Compensation in the first Salary Cap Year covered by the extended term, then one hundred forty percent (140\%) of the Estimated Average Player Salary for such Salary Cap Year less the amount of such Incentive Compensation). In the event that the last Salary Cap Year covered by the original term of the Contract provides for Incentive Compensation, the first Salary Cap Year covered by the extended term may provide for Likely Bonuses and Unlikely Bonuses of up to one hundred forty percent (140\%) of the Likely Bonuses and Unlikely Bonuses, respectively, in the last Salary Cap Year covered by the original term. Annual increases and decreases in Salary and Unlikely Bonuses shall be governed by Section 5(a)(3) above.
    \item
      Notwithstanding Section 7(a)(3)(i) above, a Designated Veteran Player Extension may provide for a Salary in the first Salary Cap year covered by the extended term totaling no more than the maximum amount provided for in Article II, Section 7. Annual increases and decreases in Salary shall be governed by Section 5(a)(3) above.
    \item
      Notwithstanding Section 7(a)(3)(i) or (ii) above, for an Extension entered into in connection with a trade pursuant to Section 8(e)(2) below:

      \begin{enumerate}
      \def\labelenumiv{(\Alph{enumiv})}
      \tightlist
      \item
        If such Extension is signed prior to the first day of the 2024-25 Salary Cap Year, then the Extension may, in the first Salary Cap Year covered by the extended term, provide for a Salary, excluding Incentive Compensation, of up to one hundred five percent (105\%) of the Regular Salary in the last Salary Cap Year covered by the original term of the Contract. In the event that the last Salary Cap Year covered by the original term of the Contract provides for Incentive Compensation, the first Salary Cap Year covered by the extended term may provide for Likely Bonuses and Unlikely Bonuses of up to one hundred five percent (105\%) of the Likely Bonuses and Unlikely Bonuses, respectively, in the last Salary Cap Year covered by the original term. Annual increases and decreases in Salary and Unlikely Bonuses shall be governed by Section 5(a)(4) above.
      \item
        If such Extension is signed on or after the first day of the 2024-25 Salary Cap Year, then the Extension may, in the first Salary Cap Year covered by the extended term, provide for a Salary, excluding Incentive Compensation, of up to the greater of: (A) one hundred twenty percent (120\%) of the Regular Salary in the last Salary Cap Year covered by the original term of the Contract; or (B) one hundred twenty percent (120\%) of the Estimated Average Player Salary for the Salary Cap Year in which the Extension is signed (or, if the Extension provides for any Incentive Compensation in the first Salary Cap Year covered by the extended term, then one hundred twenty percent (120\%) of the Estimated Average Player Salary for such Salary Cap Year less the amount of such Incentive Compensation). In the event that the last Salary Cap Year covered by the original term of the Contract provides for Incentive Compensation, the first Salary Cap Year covered by the extended term may provide for Likely Bonuses and Unlikely Bonuses of up to one hundred twenty percent (120\%) of the Likely Bonuses and Unlikely Bonuses, respectively, in the last Salary Cap Year covered by the original term. Annual increases and decreases in Salary and Unlikely Bonuses shall be governed by Section 5(a)(4) above.
      \end{enumerate}
    \item
      For purposes of determining the maximum allowable Salary in the first year of the extended term of an Extension pursuant to Sections 7(a)(3)(i) and 7(a)(3)(iii) above only, the amount of any bonuses that a player may receive pursuant to Article II, Sections 3(b)(iii) and 3(c) shall be added to the player's Regular Salary and excluded from his Incentive Compensation.
    \item
      Notwithstanding anything to the contrary in this Agreement, a player who will not be a Qualifying Veteran Free Agent at the conclusion of his Contract will not be eligible to enter into an Extension pursuant to this Section 7(a).
    \end{enumerate}
  \item
    Subject to Article II, Section 7, any Player Contract of a player who has played for his current Team for at least ten (10) Seasons and whose Salary in the last Salary Cap Year covered by the original term of the Contract is less than the Salary in the second-to-last Salary Cap Year covered by such Contract may, in the first Salary Cap Year covered by an extended term, provide for a Salary equal to one hundred seven and one-half percent (107.5\%) of the greater of (i) the average of the Regular Salaries for each Salary Cap Year covered by the original Contract beginning with the Salary Cap Year in which such Contract was entered into, or previously extended, as the case may be, or (ii) the Regular Salary in the last Salary Cap Year covered by his original Contract. In the event that the last Salary Cap Year covered by the original term of the Contract provides for Incentive Compensation, the first Salary Cap Year covered by the extended term may provide for Likely Bonuses and Unlikely Bonuses of up to one hundred seven and one-half percent (107.5\%) of the Likely Bonuses and Unlikely Bonuses, respectively, in the last Salary Cap Year covered by the original term. Annual increases and decreases in Salary and Unlikely Bonuses shall be governed by Section 5(a)(3) above. If, however, the Salary that may be included in the first year of an extended term pursuant to this Section 7(a)(4) is less than the Salary that may be included in the first year of an extended term pursuant to Section 7(a)(3) above, then the Contract may, in the first Salary Cap Year covered by an extended term, provide for a Salary of up to the amount permissible under Section 7(a)(3) above.
  \end{enumerate}
\item
  \textbf{Rookie Scale Extensions.} No Rookie Scale Contract may be extended except in accordance with the following:

  \begin{enumerate}
  \def\labelenumii{(\arabic{enumii})}
  \tightlist
  \item
    A First Round Pick who enters into a Rookie Scale Contract may enter into an Extension of such Rookie Scale Contract during the period from 12:01 p.m. eastern time on the last day of the Moratorium Period through 6:00 p.m. eastern time on the day prior to the first day of the Regular Season of the second Option Year provided for in such Contract (assuming the Team exercises such Option).
  \item
    An Extension of a Rookie Scale Contract may provide for Salary and Unlikely Bonuses in the first Salary Cap Year covered by the extended term totaling no more than the maximum amount provided for in Article II, Section 7. Annual increases and decreases in Salary and Unlikely Bonuses shall be governed by Section 5(a)(3) above.
  \item
    Notwithstanding anything to the contrary in this Agreement, a player who will not be a Qualifying Veteran Free Agent at the conclusion of his Rookie Scale Contract will not be eligible to enter into an Extension of a Rookie Scale Contract pursuant to this Section 7(b).
  \end{enumerate}
\item
  \textbf{Renegotiations.} No Player Contract may be renegotiated except in accordance with the following:

  \begin{enumerate}
  \def\labelenumii{(\arabic{enumii})}
  \tightlist
  \item
    Subject to Sections 7(c)(2) and (3) below, a Player Contract covering a term of four (4) or more Seasons may be renegotiated no sooner than the third anniversary of the signing of the Contract.
  \item
    Subject to Section 7(c)(3) below, any Player Contract that has been renegotiated in accordance with Section 7(c)(1) above to provide for an increase in Salary or Incentive Compensation in any Salary Cap Year covered by the Contract of more than five percent (5\%), or extended in accordance with Section 7(a) or (b) above, may not subsequently be renegotiated until the third anniversary of the signing of such Extension or Renegotiation.
  \item
    Assuming Section 7(c)(1) or (2) above are satisfied, a Team with a Team Salary below the Salary Cap may renegotiate a Player Contract in accordance with the following rules:

    \begin{enumerate}
    \def\labelenumiii{(\roman{enumiii})}
    \tightlist
    \item
      Subject to Article II, Section 7, the Renegotiation may provide for additional Regular Salary, Likely Bonuses, and/or Unlikely Bonuses for the then-current Salary Cap Year covered by the Contract (the ``Renegotiation Season'') that, in the aggregate, would not exceed the Team's Room at the time of the Renegotiation. (For clarity, a Renegotiation may provide for additional Likely Bonuses and/or Unlikely Bonuses even if such category (i.e., Likely Bonuses or Unlikely Bonuses) was not provided for by the Contract.)
    \item
      Every category (Regular Salary, Likely Bonuses and Unlikely Bonuses, respectively) that is increased for the Renegotiation Season must also be increased for each of the remaining Seasons of the Contract. For each Season of the Contract after the Renegotiation Season, the player's additional Regular Salary may increase or decrease over the previous Season's additional Regular Salary by no more than eight percent (8\%) of the additional Regular Salary provided for in the Renegotiation Season. In the event that the Renegotiation Season provides for additional Incentive Compensation, the amount of additional Likely Bonuses and Unlikely Bonuses provided for in each Season after the Renegotiation Season may increase or decrease by up to eight percent (8\%) of the amount of additional Likely Bonuses and Unlikely Bonuses, respectively, provided for in the Renegotiation Season.
    \item
      No Renegotiation may contain a signing bonus, unless the Renegotiation is accompanied by an Extension and the signing bonus would otherwise be permitted under the rules governing the inclusion of signing bonuses in Extensions.
    \end{enumerate}
  \item
    In no event may a Team with a Team Salary at or above the Salary Cap renegotiate a Player Contract.
  \item
    In no event may a Team and a player renegotiate a Player Contract from March 1 through June 30 of any Salary Cap Year.
  \end{enumerate}
\item
  \textbf{Other.}

  \begin{enumerate}
  \def\labelenumii{(\arabic{enumii})}
  \item
    In no event shall a Team and player negotiate a decrease in Salary or in any Incentive Compensation for any Salary Cap Year covered by a Player Contract.
  \item
    A Player Contract that is extended pursuant to Section 7(a) above may be renegotiated simultaneously, but only if and to the extent permitted by the rules set forth in Section 7(c) above. Notwithstanding anything to the contrary in this Agreement, if a Player Contract is extended pursuant to Section 7(a) above and renegotiated simultaneously, then the amount of the player's Salary, excluding Incentive Compensation, in the first Salary Cap Year covered by the extended term may decrease by no more than forty percent (40\%) of the player's Regular Salary (as renegotiated) in the last Salary Cap Year covered by the original term. In the event that the last Salary Cap Year covered by the original term provides for Incentive Compensation and such Incentive Compensation is also renegotiated, the amount of Likely Bonuses and Unlikely Bonuses in the first Salary Cap Year covered by the extended term may decrease by up to forty percent (40\%) of the player's Likely Bonuses and Unlikely Bonuses, respectively (as renegotiated), in the last Salary Cap Year covered by the original term.
  \item
    A Contract that is amended pursuant to Article XXIV, Section 2(a)(iii)(B)(3) to waive all or any portion of a trade bonus in connection with the trade of a Player Contract may not be subsequently renegotiated until the later of (i) six (6) months from the date of the trade, or (ii) the first date on which the Contract could otherwise be renegotiated pursuant to this Section 7.
  \item
    In connection with the trade of a Player Contract, notwithstanding anything to the contrary in Article XII, Section 2(a), a player and the assignor Team may agree upon an amendment to the Contract providing for the exercise or non-exercise of an Option contained in the Contract by a player or Team (as the case may be), provided that the amendment also provides that (i) the player will be traded to the assignee Team within a specified period of time not to exceed forty-eight (48) hours of the execution of the amendment, and (ii) such trade and the consummation of such trade are conditions precedent to the validity of the amendment.
  \item
    In the event that a Team and a player agree to amend a Player Contract in accordance with Article II, Section 3(p), then: (i) for purposes of calculating the player's Salary for the then-current and any remaining Salary Cap Year covered by the Contract, notwithstanding any stretch of the player's protected Compensation payment schedule, the aggregate reduction in the player's protected Compensation, if any, shall be allocated pro rata over the then-current and each remaining Salary Cap Year on the basis of the remaining unearned protected Base Compensation in each such Salary Cap Year; and (ii) the Team shall not be permitted to sign the player to a new Player Contract (or claim the player off of waivers) before the later of: (x) one (1) year following the date that the player's Player Contract with such Team was terminated; or (y) the July 1 following the last Season of such Player Contract.
  \item
    \begin{enumerate}
    \def\labelenumiii{(\roman{enumiii})}
    \tightlist
    \item
      For any Contract terminated on or after the first day of the 2023-24 Salary Cap Year, the following rules shall apply for purposes of determining a Team's Team Salary in circumstances where the Contract is terminated prior to the September 1 preceding the final Season covered by the Contract and, prior to such September 1, the Team elects to have the player's Salary for the then-current and any remaining Salary Cap Years stretched (i.e., re-attributed):

      \begin{enumerate}
      \def\labelenumiv{(\Alph{enumiv})}
      \tightlist
      \item
        in the event the Team so elects during the period from September 1 through the following June 30 of a Salary Cap Year, (i) the player's post-termination Salary for the then-current Salary Cap Year (after giving effect to the provisions of Section (d)(5) above, if applicable) shall remain unchanged, and (ii) the player's post-termination Salary for each remaining Salary Cap Year (after giving effect to the provisions of Section (d)(5) above, if applicable) shall be aggregated and allocated evenly over a number of Salary Cap Years equal to twice the number of Seasons (including any Player Option Year) remaining on the Contract following the Salary Cap Year in which the election occurred, plus one (1) Season; or
      \item
        in the event the Team so elects during the period from July 1 through August 31 of a Salary Cap Year, the player's post-termination Salary for the then-current and any remaining Salary Cap Years (after giving effect to the provisions of Section (d)(5) above, if applicable) shall be aggregated and allocated evenly over a number of Salary Cap Years equal to twice the number of Seasons (including any Player Option Year) remaining on the Contract following the date of the election (including the upcoming Season), plus one (1) Season.
      \end{enumerate}
    \item
      To make an election pursuant to Section 7(d)(6)(i) above, a Team must provide the NBA with a written statement electing to stretch the player's Salary. The NBA shall provide notice of such election to the Players Association by email within two (2) business days following the NBA's receipt of such notice.
    \item
      Notwithstanding anything to the contrary in this Section 7(d)(6): (A) in no event shall a Team be permitted to elect to stretch a waived player's Salary if the portion of the Team's Team Salary representing all of the Team's waived players (and any other former players) in any future Salary Cap Year exceeds or as a result of the proposed stretch would exceed fifteen percent (15\%) of the Salary Cap in effect during the Salary Cap Year in which the election occurs; (B) any Team that stretches a player's Salary for Salary Cap purposes may not subsequently sign or acquire such player prior to the July 1 following the end of the last Season of the player's Contract (including, for clarity, any Option Year); and (C) a Team that terminates a player's Contract and subsequently signs or acquires such player prior to July 1 following the end of the last Season of the player's Contract (including, for clarity, any Option Year) may not make an election to stretch the Salary of such terminated Contract pursuant to Section 7(d)(6)(i) above.
    \item
      In the event a Team makes an election pursuant to Section 7(d)(6)(i) above to stretch the Salary provided for in a Player Contract, the amount included in Total Salaries in respect of such Contract shall be calculated without regard to such election.
    \end{enumerate}
  \item
    In no event shall a Team and player amend a Contract for the purpose of terminating or shortening the term of the Contract, except in accordance with the NBA waiver procedure or Article XII, Section 2.
  \item
    A Team and player may negotiate the terms and conditions of an amendment to a Player Contract (including an Extension or Renegotiation) only during the period of time in which the Team and player are permitted to enter into such amendment. Notwithstanding the foregoing, if a Team and player would be permitted to enter into an amendment to a Player Contract as of the last day of the Moratorium Period immediately following a Season, then the Team and player may negotiate the terms and conditions of such amendment beginning on the day following the last day of such Season.
  \end{enumerate}
\end{enumerate}

\hypertarget{trade-rules.}{%
\section{Trade Rules.}\label{trade-rules.}}

\begin{enumerate}
\def\labelenumi{(\alph{enumi})}
\item
  Subject to the rules in Section 2(e) above, a Team shall be permitted to pay or receive in connection with one (1) or more trades occurring during a Salary Cap Year, directly or indirectly, up to an aggregate amount equal to 5.15\% of the Salary Cap for such Salary Cap Year in cash across all such trades, including cash received as reimbursement for Compensation obligations to players whom the Team is acquiring.

  For purposes of this Section 8(a), (i) if a Contract is signed and then traded pursuant to Section 8(e)(1) below, and the Contract contains a signing bonus, the payment of all or any portion of such bonus by the Team that signed the Contract shall be treated as a reimbursement of a Compensation obligation of the assignee Team and shall be subject to this Section 8(a), and (ii) the amounts paid or received by a Team in connection with one (1) or more trades occurring during a Salary Cap Year shall not be netted against each other (thus, for example, if the maximum allowable cash limit for the 2023-24 Salary Cap Year were \$6.5 million and Team A paid \$6.5 million in connection with one (1) trade occurring during such Salary Cap Year and received \$6.5 million from another Team in connection with a subsequent trade occurring during the same Salary Cap Year, Team A would be unable to either pay or receive any cash in connection with any subsequent trades that it makes during that Salary Cap Year).
\item
  A player (other than a Two-Way Player) with a one-year Contract (excluding any Option Year) who would be a Qualifying Veteran Free Agent or an Early Qualifying Veteran Free Agent upon completing the playing services called for under his Contract cannot be traded without the player's consent; provided, however, that in accordance with Article II, Section 3(h) above, the player and Team may agree at the time of signing such Contract that the player's right to consent to a trade pursuant to this Section 8(b) shall be eliminated. Should the player consent (or if the player and Team agreed at the time of signing to eliminate his right to consent) and the player is traded (except if the Contract has an Option for the second year that was exercised prior to the trade), then, for purposes of determining whether the player is a Qualifying Veteran Free Agent, Early Qualifying Veteran Free Agent, or Non-Qualifying Veteran Free Agent at the conclusion of the Contract or any subsequent Contract between the player and the assignee Team, the player shall be considered as having changed Teams by means of signing a Contract with the assignee Team as a Free Agent (and not by means of trade). For clarity, for any player who did not agree at the time of signing to eliminate his right to consent, such right under this Section 8(b) shall continue following the initial trade of the player's Contract and any proposed subsequent trade of such Contract during the term thereof (not including any Option Year).
\item
  A Team cannot trade any player after the NBA trade deadline occurring in the last Season of the player's Contract, or after the NBA trade deadline occurring in any Season that could be the last Season of the player's Contract based upon the exercise or non-exercise of an Option or Early Termination Option.
\item
  \begin{enumerate}
  \def\labelenumii{(\roman{enumii})}
  \tightlist
  \item
    No Draft Rookie who signs a Standard NBA Contract or player who signs a Two-Way Contract may be traded before thirty (30) days following the date on which the Contract is signed.
  \item
    No player who signs a Standard NBA Contract as a Free Agent (or who signs a Standard NBA Contract while under a Two-Way Contract) may be traded before the later of (A) three (3) months following the date on which such Contract was signed or (B) the December 15 of the Salary Cap Year in which such Contract was signed; provided, that if a Contract is signed in connection with an agreement to trade the Contract in accordance with Section 8(e) below, the foregoing rule shall not apply to the initial trade but shall instead be applicable if the Contract is traded a second time. For the purposes of this rule, a Two-Way Contract that is converted to a Standard NBA Contract pursuant to such Contract's Standard NBA Contract Conversion Option will be deemed to be a Standard NBA Contract signed by a Free Agent on the date of the conversion.
  \item
    Notwithstanding the rule set forth in Section (d)(ii) above, any player who signs a Standard NBA Contract with his prior Team meeting the following criteria may not be traded before the later of (x) three (3) months following the date on which such Contract was signed or (y) the January 15 of the Salary Cap Year in which such Contract was signed: the Team Salary of the player's Team is above the Salary Cap immediately following the Contract signing and the player is a Qualifying Veteran Free Agent or Early Qualifying Veteran Free Agent who, in accordance with Section 6(b)(1) or (3) above, enters into a new Player Contract with his prior Team that provides for a Salary for the first Season of such new Contract greater than one hundred twenty percent (120\%) of the Salary for the last Season of the player's immediately prior Contract. The rule set forth in this Section (d)(iii) shall not apply to a player if his new Contract provides for Salary equal to the Minimum Player Salary (with no bonuses of any kind). For purposes of the foregoing sentence, if the player's immediately prior Contract was a one-year Contract that provided for Salary equal to the Minimum Player Salary (with no bonuses of any kind), the player's prior Salary shall include the portion of the Minimum Player Salary, if any, that was reimbursed out of the League-wide benefits fund described in Article IV, Section 6(h).
  \end{enumerate}
\item
  \begin{enumerate}
  \def\labelenumii{(\arabic{enumii})}
  \tightlist
  \item
    Subject to the rules set forth in Section 2(e) above, a Veteran Free Agent and his Prior Team may enter into a Player Contract pursuant to an agreement between the Prior Team and another Team concerning the signing and subsequent trade of such Contract, but only if (i) the Veteran Free Agent finished the prior Season on his Prior Team's roster, (ii) the Contract is for at least three (3) Seasons (excluding any Option Year) but no more than four (4) Seasons in length, (iii) the Contract is not signed pursuant to the Non-Taxpayer Mid-Level Salary Exception or the Mid-Level Salary Exception for Room Teams, (iv) the first Season of the Contract is fully protected for lack of skill, (v) the Contract is entered into prior to the first day of the Regular Season, (vi) with respect to any 5th Year Eligible Player (as defined in Article II, Section 7) who met one of the Higher Max Criteria (as defined in Article II, Section 7), the Contract may not provide the player with Salary (plus Unlikely Bonuses) in excess of twenty-five percent (25\%) of the Salary Cap (as calculated pursuant to Article II, Section 7) in effect at the time the Contract is signed, and (vii) the acquiring Team has Room for the player's Salary plus any Unlikely Bonuses provided for in the first Season of the Contract.
  \item
    A player and his Team may amend a Player Contract (including by entering into an Extension but not by entering into a Renegotiation) pursuant to an agreement between such Team and another Team concerning the signing of the amendment and subsequent trade of the amended Contract; provided, however, that: (i) no such agreement may be made during the period from the last day of the last Regular Season covered by the Contract (or the last day of any Regular Season that could be the last Regular Season covered by the Contract based upon the exercise or non-exercise of an Option or ETO) through the following June 30; (ii) no such Extension entered into pursuant to this Section 8(e)(2) prior to the first day of the 2024-25 Salary Cap Year may cover more than three (3) Seasons from the date the Extension is signed; and (iii) no such Extension entered into pursuant to this Section 8(e)(2) on or after the first day of the 2024-25 Salary Cap Year may cover more than four (4) Seasons from the date the Extension is signed. The Salary and Unlikely Bonuses that may be provided in the first year of the extended term and annual increases and decreases in Salary and Unlikely Bonuses shall be governed by Section 7(a)(3)(iii) and Section 5(a)(4) above.
  \item
    A Player Contract or Extension entered into pursuant to Section 8(e)(1) or (2) above may not contain an Exhibit 6 thereto. However, the preceding sentence shall not prohibit the Teams involved in the trade from agreeing that the trade (and thus the validity of the Player Contract or Extension) will be conditional upon the passage of a physical examination to be performed by a physician designated by the assignee-Team in accordance with NBA procedures.
  \end{enumerate}
\item
  \begin{enumerate}
  \def\labelenumii{(\roman{enumii})}
  \tightlist
  \item
    In the event a player enters into (A) an Extension pursuant to Section 7(a) above (other than a Designated Veteran Player Extension governed by Section (f)(ii) below) that covers five (5) Seasons (or, for Extensions entered into prior to the first day of the 2024-25 Salary Cap Year, four (4) or more Seasons) and/or provides for Salary and Unlikely Bonuses or annual increases or decreases in the player's Salary and Unlikely Bonuses in excess of the amounts that, at the time such Extension was entered into, were permissible in Extensions entered into in connection with an agreement to trade the Contract pursuant to Section 8(e)(2) above, or (B) a Renegotiation pursuant to Section 7(c) above, then the player may not be traded before six (6) months following the date on which such Extension or Renegotiation was signed. If a team acquires a player in a trade, then, for a period of six (6) months following the date of the trade, the team may not enter into (X) an Extension with the player pursuant to Section 7(a) above that covers five (5) Seasons (or, if the trade occurred prior to the first day of the 2024-25 Salary Cap Year, four (4) or more Seasons) and/or provides for Salary and Unlikely Bonuses or annual increases or decreases in the player's Salary and/or Unlikely Bonuses in excess of the amounts that, at the time such trade occurred, were permissible in Extensions entered into in connection with an agreement to trade the Contract pursuant to Section 8(e)(2) above, or (Y) a Renegotiation pursuant to Section 7(c) above.
  \item
    In the event a player enters into a Designated Veteran Player Extension pursuant to Section 7(a)(3)(ii) above or a Designated Veteran Player Contract pursuant to Article II, Section 7, the player may not be traded before one (1) year following the date on which he entered into such Designated Veteran Player Extension or Designated Veteran Player Contract.
  \end{enumerate}
\item
  In the event a Rookie Scale Contract is extended pursuant to Section 7(b) above and a Team proposes to trade such Contract to another Team prior to the first day of the Salary Cap Year immediately following such Extension, then, only for purposes of determining whether the acquiring Team has Room for the Contract, the Salary for the last Salary Cap Year of the original term of the Contract shall be deemed to equal the average of the aggregate Salaries for such Salary Cap Year and each Salary Cap Year of the extended term. For purposes of this subsection (g), the Salary for each Salary Cap Year of the extended term of the Contract shall be the Salary as set forth in the Contract; provided, however, that:

  \begin{enumerate}
  \def\labelenumii{(\roman{enumii})}
  \tightlist
  \item
    If the Contract provides for Base Compensation in the first Salary Cap Year of the extended term that is expressed as a percentage of the Salary Cap in accordance with Article II, Section 7(d), then the Base Compensation in the extended term of the Contract shall be determined assuming that (a) the Salary Cap in the first Salary Cap Year covered by the extended term will equal one hundred four and one-half percent (104.5\%) of the Salary Cap in effect at the time that the proposed trade would occur, and (b) the player does not meet any of the applicable Higher Max Criteria during the fourth Season of his Rookie Scale Contract; or
  \item
    If the Contract provides for Salary plus Unlikely Bonuses in the first Salary Cap Year of the extended term that exceeds the applicable Maximum Annual Salary that would apply to such player assuming that the Salary Cap in the first Salary Cap Year covered by the extended term will equal one hundred four and one-half percent (104.5\%) of the Salary Cap in effect at the time that the proposed trade would occur, then the Salary plus Unlikely Bonuses in the extended term of the Contract shall be determined to be the Salary plus Unlikely Bonuses that would result from the deemed amendment(s) pursuant to Article II, Section 7(c) using the assumption described above in this subsection (ii).
  \end{enumerate}
\item
  If a Team trades a player and the assignee Team subsequently places the player on waivers, the assignor Team shall not be permitted to sign the player to a new Contract (or claim the player off of waivers) before the earlier of: (i) one (1) year following the date all conditions to the trade were satisfied; or (ii) the July 1 following the last Season of the player's Player Contract.
\item
  Prior to the assignment of any Player Contract, the Team from which such Player Contract is to be assigned and the player whose Player Contract is to be assigned shall be required to divest themselves, on terms mutually agreeable to the player and the Team, of any preexisting financial arrangements between such Team and such player. The foregoing shall not apply to Compensation earned by the player prior to the assignment or to loans.
\item
  As soon as is practicable following each trade (but in no event later than one (1) week from the date of the trade), the NBA shall send to the Players Association, by email, a summary of the principal terms of the trade; provided, however, that the NBA may omit from such summary any terms that the NBA or one (1) or more Teams involved in the trade reasonably deem confidential (other than such terms as may be necessary to verify the Teams' compliance with Section 8(a) above).
\item
  A ``trade'' of a player under this Agreement shall mean an assignment of a Player Contract pursuant to a negotiated exchange between two or more Teams following a trade conference call with the NBA league office. For clarity, the word ``trade'' shall not include an assignment of a player via the NBA's waiver procedures.
\end{enumerate}

\hypertarget{miscellaneous.}{%
\section{Miscellaneous.}\label{miscellaneous.}}

\begin{enumerate}
\def\labelenumi{(\alph{enumi})}
\tightlist
\item
  Except where this Agreement states otherwise, for purposes of any rule in this Agreement that limits, involves counting, or otherwise relates to, the number of Seasons covered by a Contract:

  \begin{enumerate}
  \def\labelenumii{(\arabic{enumii})}
  \tightlist
  \item
    If a Player Contract or Extension is signed after the beginning of a Season, the Season in which the Contract or Extension is signed shall be counted as one (1) full Season covered by the Contract or Extension; and in the case of an Extension that is signed during the period from the end of a Season through the immediately following June 30, the Season immediately preceding the signing of the Extension (i.e., the just-completed Season) shall be counted as one (1) full Season covered by the Extension.
  \item
    An Option Year shall be counted as one (1) Season covered by the Contract.
  \end{enumerate}
\item
  Except where this Agreement states otherwise, all of the rules in this Agreement that limit, affect the calculation of, or otherwise relate to, the Compensation or Salary provided for in a Player Contract shall apply to Option Years.
\end{enumerate}

\hypertarget{accounting-procedures.}{%
\section{Accounting Procedures.}\label{accounting-procedures.}}

\begin{enumerate}
\def\labelenumi{(\alph{enumi})}
\item
  \begin{enumerate}
  \def\labelenumii{(\arabic{enumii})}
  \tightlist
  \item
    The NBA and the Players Association shall jointly engage an independent auditor (the ``Accountants'') to provide the parties with an ``Audit Report'' (and a ``Draft Audit Report,'' and, if applicable, an ``Interim Audit Report'' and, if applicable, an ``Interim Designated Share Audit Report'') setting forth BRI, and Total Salaries and Benefits for the immediately preceding Salary Cap Year, and the information called for by Section 12 below (the ``Designated Share Information''). The audit reports provided for by this Section 10(a)(1) are to be prepared in accordance with the provisions and definitions contained in this Agreement. The engagement of the Accountants shall be deemed to be renewed annually unless they are discharged by either party during the period from the submission of an Audit Report up to January 1 of the following year. The parties agree to share equally the costs incurred by the Accountants in preparing the audit reports provided for by this Section 10(a)(1).
  \item
    The Accountants shall submit a ``Draft Audit Report'' for each Salary Cap Year to the NBA and the Players Association, along with relevant supporting documentation, two (2) weeks prior to the scheduled issuance of the final Audit Report.
  \item
    The final Audit Report shall be submitted by the Accountants to the parties by 6:00 p.m. eastern time on the last day of the Salary Cap Year under audit. The audit shall begin as needed to ensure there is no reduction in the audit duration compared to the 2011 CBA. The Audit Report shall not be deemed final until the parties have confirmed in writing their agreement (in a form acceptable to the parties) with such Report. The NBA, the Players Association, and the Teams shall use their best efforts to facilitate the Accountants' timely completion of the Audit Report.
  \item
    In the event that, for any reason, the Accountants fail to submit to the parties a final Audit Report by 6:00 p.m. eastern time on the last day of the Salary Cap Year under audit, the Accountants shall prepare an interim Audit Report (the ``Interim Audit Report'') by such time setting forth the Accountants' best estimate of BRI and Total Salaries and Benefits for the preceding Salary Cap Year and, based upon such best estimates, the Designated Share Information. Such Interim Audit Report shall include:

    \begin{enumerate}
    \def\labelenumiii{(\roman{enumiii})}
    \tightlist
    \item
      All amounts of BRI and Total Salaries and Benefits (or the portions thereof) and all Designated Share Information (or the portions thereof) for such Salary Cap Year as to which the Accountants have completed their review and, by written agreement of the Players Association and the NBA (waiving their respective rights to dispute such amounts), are not in dispute.
    \item
      With respect to any amounts of BRI or Total Salaries and Benefits (or portions thereof) as to which the Accountants have not completed their review or which are the subject of a good faith dispute between the parties, the NBA's good faith proposal as to the proper amount, if any, that should be included in the Audit Report.
    \item
      With respect to any items of Designated Share Information that are the subject of a good faith dispute between the parties, the Accountants' good faith determination as to such items, taking into account the provisions of Sections 10(a)(4)(i) and (ii).
    \end{enumerate}
  \end{enumerate}

  As soon as practicable after the Interim Audit Report is submitted to the parties, the Accountants shall submit the final Audit Report, including a description of the differences, if any, from the Interim Audit Report. The Audit Report shall not be deemed final until the parties have confirmed in writing their agreement (in a form acceptable to the parties) with such Report or all disputes with respect to such Report have been finally resolved by means of the dispute-resolution procedures provided for by this Agreement.

  If, at the conclusion of the Audit Report Challenge Period (as defined by Section 12(a)(11) below), the Accountants have not submitted or are unable to submit a final Audit Report (because, by way of example but not limitation, there are disputes or claims that have been asserted pursuant to Article XXXII, Section 9(c) and which remain pending), the Accountants shall prepare and submit to the parties, within five (5) business days following the completion of the Audit Report Challenge Period, an Interim Designated Share Audit Report that shall include the information set forth in the Interim Audit Report as adjusted or amended so as to reflect any final determinations made by the System Arbitrator or the Appeals Panel (as the case may be) in proceedings commenced pursuant to Article XXXII, Section 9(b) and involving disputes or claims with respect to such Interim Audit Report. The sole purposes for which any Interim Designated Share Audit Report is to be used under this Agreement are to perform or form the basis for the calculations to be made pursuant to Section 12 below and, if applicable, to perform the calculations that determine whether the conditions to the parties' mutual termination rights set forth in Article XXXIX, Sections 7-8 are satisfied.
\item
  For purposes of determining BRI, Total Salaries and Benefits, and the Designated Share Information, the Accountants shall perform at least such review procedures as shall be agreed upon by the parties. In connection with the preparation of Audit Reports for each Salary Cap Year, each Team and the NBA shall submit a report to the Accountants, the NBA, and the Players Association setting forth BRI, Team Salaries, and Benefits information for such Salary Cap Year, on forms agreed upon by the NBA, the Players Association, and the Accountants (the ``BRI Reports''). The NBA and the Players Association shall agree upon such forms no later than April 1 of each Salary Cap Year.
\item
  The Accountants shall review the reasonableness of any estimates of revenues or expenses for a Salary Cap Year included in the Teams' and the NBA's BRI Reports for such Salary Cap Year and may make such adjustments in such estimates as they deem appropriate. To the extent the actual amounts of revenues received or expenses incurred for a Salary Cap Year differ from such estimates, adjustments shall be made in BRI for the following Salary Cap Year in accordance with the provisions of Section 10(f) below.
\item
  With respect to deducted expenses, except for Newly-Deductible International Expenses, the NBA, League-related entities, Teams, and Related Parties shall report in BRI Reports only those expenses that are reasonable and customary in accordance with the provisions of Section 1(a) above. Subject to the terms of Section 1(a)(6) and Section 1(a)(14) above, and Section 11 below, all categories of expenses deducted in a BRI Report completed by the NBA or a Team shall be reviewed by the Accountants, but such categories shall be presumed to be reasonable and customary and the amount of the expenses deducted by the NBA or a Team that come within such expense categories shall also be presumed to be reasonable and customary, unless such categories or amounts are found by the Accountants to be either unrelated to the revenues involved or grossly excessive.
\item
  The Accountants shall notify designated representatives of the NBA and the Players Association: (1) if the Accountants have any questions concerning the amounts of revenues or expenses reported by the Teams and the NBA or any other information contained in the BRI Reports; or (2) if the Accountants propose that any adjustments be made to any revenue or expense item or any other information contained in the BRI Reports.
\item
  The Accountants shall indicate which amounts included in BRI for a Salary Cap Year, if any, represent estimates of revenues or expenses. With respect to any such estimated revenues or expenses, the Accountants shall, in preparing the Audit Report for the immediately succeeding Salary Cap Year (``Subsequent Audit Report''), or the Audit Report for the same Salary Cap Year in the event that an Interim Audit Report was previously issued for that Salary Cap Year, determine the actual revenues and expenses received for the prior Salary Cap Year and include as a credit or debit to BRI in such Subsequent Audit Report the amount of the aggregate difference, if any, between all such estimated revenues and expenses for the prior Salary Cap Year and the actual revenues and expenses received for such Salary Cap Year.
\item
  In the event that in the course of preparing an Audit Report for a Salary Cap Year the Accountants discover that they committed an error in computing BRI in the Audit Reports for either of the two (2) previous Salary Cap Years, which error resulted in a material understatement or overstatement of BRI for either of such Salary Cap Years, and the parties agree that such error was committed and agree as to the amount of the resulting understatement or overstatement (or, if they do not agree, an error (and the amount of such error) is established pursuant to the dispute resolution procedures provided for in this Agreement) the amount of such understatement or overstatement of BRI shall be added to or subtracted from BRI, as the case may be, with interest (at a rate equal to the one (1) year Treasury Bill rate as published in The Wall Street Journal on the date of the issuance of such Audit Report) accruing from the date of the Audit Report for the Salary Cap Year in which such understatement or overstatement occurred in equal annual amounts over the then-current and subsequent Salary Cap Years. Notwithstanding the foregoing, the parties will jointly instruct the Accountants that their audits shall not include procedures specifically designed to detect errors committed in prior audits.
\item
  In the event that there is an NHL players' strike or owners' lockout (``work stoppage'') resulting in the cancellation of all or part of any NHL season in any Salary Cap Year, and such work stoppage results in a refund being made to luxury suite-holders, premium seat license-holders, or to purchasers of fixed arena signage and/or naming rights in arenas in which both an NBA Team and an NHL team plays its home games, then the revenues for luxury suites, premium seat licenses, and fixed arena signage and/or naming rights in such arenas shall be determined as if such refunds were not made. If the work stoppage continues for a second year, then the NHL revenues shall be deemed to be the amount included for the prior year.
\item
  All disputes with respect to any Interim Audit Report shall be resolved exclusively in accordance with the procedures set forth in Article XXXII.
\item
  In the event of a trade that (i) occurs after the final Audit Report for a Salary Cap Year is submitted by the Accountants and before the conclusion of such Salary Cap Year, and (ii) results in a player earning a trade bonus, the final Audit Report shall be amended to reflect such trade bonus for purposes of calculating such player's Salary and the Team's Team Salary (and thus, for clarity, for the purposes of computing the amount of tax the Team owes pursuant to Article VII, Section 2(d)); however, the portion of such trade bonus that is included in the Team's Team Salary for such Salary Cap Year shall be excluded from Total Salaries for such Salary Cap Year, and included in Total Salaries for the immediately following Salary Cap Year.
\end{enumerate}

\hypertarget{players-association-audit-rights.}{%
\section{Players Association Audit Rights.}\label{players-association-audit-rights.}}

\begin{enumerate}
\def\labelenumi{(\alph{enumi})}
\tightlist
\item
  \textbf{Team Audits.} The Players Association shall have the right as part of the annual review of BRI Reports to retain its own accountants (the ``Players Association's Accountants''), at its own expense, after the submission of each Audit Report under this Agreement, to audit the books and records of NBA teams (of its choosing), with the number of such audits in each Salary Cap Year set forth below (the ``First Audit''); provided, however, that such review shall be limited to (i) revenue items (including in respect of equity transactions subject to Section 1(a)(13) above), and (ii) expense items, in each case that appear or should have appeared in the BRI Reports. In the event that, in the opinion of the Players Association's Accountants, such audit indicates misallocations or miscategorizations of revenues or expenses (other than with respect to matters that constituted Disputed Adjustments in connection with the prior Audit Report) resulting in an understatement of BRI, they shall submit to the NBA proposed adjustments to BRI consistent with their findings. In the event that the NBA disputes such proposed adjustments, such proposed adjustments shall be deemed to be ``Disputed Adjustments'' and shall be resolved in accordance with the procedures set forth in Article XXXII. In addition, in the event that First Audit Disputed Adjustments in excess of \$8 million are resolved in favor of the Players Association, the Players Association shall then have the right, that Season, to have the Players Association's Accountants audit up to an additional ten (10) NBA teams for the same Salary Cap Year, in accordance with the foregoing procedures (the ``Second Audit''). If, as a result of the Second Audit, additional Disputed Adjustments in excess of \$8 million are resolved in favor of the Players Association, the Players Association shall then have the right, that Season, to have the Players Association's Accountants audit all remaining NBA Teams for that Salary Cap Year. The amount of any and all Disputed Adjustments that are ultimately resolved in favor of the Players Association in accordance with this Section 11(a) shall be added to BRI in the Salary Cap Year in which such resolution is reached. The aggregate number of NBA Teams selected for the First Audit by the Players Association over the course of the first six (6) Salary Cap Years of the CBA will be ninety (90), to be distributed over the Salary Cap Years at the Players Association's option. For the seventh Salary Cap Year of the CBA, the Players Association will be entitled to select fifteen (15) NBA Teams for First Audits, supplemented by any of the ninety (90) audits that were not used in a previous Salary Cap Year.
\item
  \textbf{League Audit.} The Players Association shall have the right as part of the annual review of BRI Reports to retain the Players Association's Accountants to conduct an audit, at its own expense, of the books and records of the NBA, Properties, Media Ventures, and other League-related entities associated with generating BRI, provided, however, that such audit shall be limited to (i) revenue items (including in respect of equity transactions subject to Section 1(a)(13) above) and (ii) expense items, regardless of whether such expenses exceed the applicable BRI ratio of expenses to revenues set forth in Exhibit D, in each case that appear or should have appeared in the BRI Report. In the event that, in the opinion of the Players Association's Accountants, such audit indicates misallocations or miscategorizations of revenues or expenses (other than with respect to matters that constituted League Disputed Adjustments in connection with the prior Audit Report) resulting in an understatement of BRI, they shall submit proposed adjustments to the NBA consistent with their findings. In the event that the NBA disputes such proposed adjustments, such proposed adjustments shall be deemed to be League Disputed Adjustments and resolved in accordance with the procedures set forth in Article XXXII. The amount of any and all such League Disputed Adjustments that are resolved in the Players Association's favor shall be included in BRI in the Salary Cap Year in which such resolution is reached. In addition, in the event that any such League Disputed Adjustments are resolved in the Players Association's favor, the Accountants shall be directed to correct such expense misallocations and/or miscategorizations in the remaining Salary Cap Years covered by the Agreement.
\item
  \textbf{Confidentiality.} In connection with any audit conducted by the Players Association pursuant to this Section 11, the Players Association agrees to sign, and to cause its representatives to sign, a confidentiality agreement in the form annexed hereto as Exhibit J-1. The Players Association also agrees to sign, and to cause its representatives to sign, a similar confidentiality agreement with respect to information obtained in connection with the Accountants' audit pursuant to Section 10 above.
\item
  \textbf{Preceding Salary Cap Year Audit Adjustments.} Notwithstanding anything else in this Agreement or any release in the annual BRI letter agreement, if upward or downward adjustments are made in connection with a Players Association-initiated audit, an adjustment to BRI in respect of the same item can also be made for revenues or expenses related to the preceding Salary Cap Year, if applicable. For example, without limitation, if, based on the audit findings, the parties agree that a Team under-reported 2023-24 BRI by one million dollars (\$1,000,000), and that the same error in the same amount occurred in 2022-23, then 2024-25 BRI would be adjusted upward by two million dollars (\$2,000,000).
\item
  \textbf{Related Party Access.} The Players Association's Accountants shall have access to such portions of a Related Party's books and records that the accountants have a well-founded basis to believe have a meaningful impact on BRI. For purposes of the foregoing, (i) where a team plays in an arena owned or operated by a Related Party, the Players Association's Accountants will have access to that Related Party arena company's trial balance relating to all revenues and to such other portions of the trial balance that the Players Association's Accountants have a well-founded basis to believe have a meaningful impact on BRI; (ii) for other Related Parties, information requests should fit the circumstances to enable the Players Association's Accountants to verify the accuracy of BRI amounts (x) that cannot reasonably be verified through other means, and (y) without accessing financial and business information that there is no well-founded basis to believe have a meaningful impact on BRI; and (iii) the NBA, Players Association, and the Team will collectively consider any request for access to Related Party books and records while onsite and make their best efforts to resolve the access issue.
\item
  \textbf{Bilateral Adjustments.} Subject to the deadlines set forth in Section 11(g) below, the NBA may propose BRI adjustments with respect to any Team audited by the Players Association. The NBA's right to propose such adjustments may not adversely affect in any way the time and resources available to the Players Association under its audit rights. In the event that the Players Association disputes such proposed adjustments, such proposed adjustments shall be resolved in accordance with the procedures set forth in Article XXXII. The amount of any and all such proposed adjustments that are ultimately resolved in favor of the NBA in accordance with this Section 11(f), including any adjustments made pursuant to Section 11(d) above, shall be deducted from BRI in the Salary Cap Year in which such resolution is reached.
\item
  \textbf{Timing.} Audits conducted by the Players Association must be noticed within ninety (90) days after issuance of the final Audit Report in respect of the applicable Salary Cap Year. Any proposed adjustments by the Players Association and NBA relating to Team audits (and in the case of the Players Association, with respect to any League Office audits) will be resolved by April 30 of the following calendar year. Each party will provide its proposed adjustments by March 25.
\end{enumerate}

\hypertarget{designated-share-arrangement.}{%
\section{Designated Share Arrangement.}\label{designated-share-arrangement.}}

\begin{enumerate}
\def\labelenumi{(\alph{enumi})}
\item
  \textbf{Definitions.} As used in this Agreement, the following terms shall have the following meanings:

  \begin{enumerate}
  \def\labelenumii{(\arabic{enumii})}
  \tightlist
  \item
    ``Actual Reduction Percentage'' means, with respect to a Salary Cap Year, the lesser of (i) ten percent (10\%), and (ii) the percentage that, when multiplied by Adjusted Total Salaries, equals the Uncapped Reduction Amount.
  \item
    ``Adjusted Team Salary'' means for a Team, with respect to a Salary Cap Year, the portion of Adjusted Total Salaries for such Salary Cap Year for which the Team is financially responsible. For clarity, for purposes of this Section 12(a)(2), (i) with respect to a player that was employed by more than one (1) Team under the same Player Contract during the Salary Cap Year (i.e., in cases where a player's Contract is acquired by trade or pursuant to the NBA waiver procedure), the portion of the Adjustment Salary in respect of the Adjustment Contract for which each such Team is financially responsible shall be determined in accordance with NBA rules; and (ii) a Team shall be considered financially responsible for any signing bonus allocation (or the allocation of any amount treated as a signing bonus pursuant to Section 3(b)(1) above) that was the result of a signing bonus (or the result of any amount treated as an earned signing bonus pursuant to Section 3(b)(1) above) paid by the Team.
  \item
    ``Adjusted Total Benefits'' means, with respect to a Salary Cap Year, an amount equal to Total Benefits for such Salary Cap Year less any amounts reimbursed out of the League-wide benefits fund described in Article IV, Section 6(h) in respect of such Salary Cap Year.
  \item
    ``Adjusted Total Salaries'' means, with respect to a Salary Cap Year, an amount equal to Total Salaries for such Salary Cap Year plus any amounts reimbursed out of the League-wide benefits fund described in Article IV, Section 6(h) in respect of such Salary Cap Year.
  \item
    ``Adjusted Total Salaries and Benefits'' means, with respect to a Salary Cap Year, the sum of Adjusted Total Salaries and Adjusted Total Benefits for such Salary Cap Year.
  \item
    ``Adjustment Contract'' means, with respect to a Salary Cap Year, any Contract that provides for Salary that is included in Adjusted Total Salaries for such Salary Cap Year.
  \item
    ``Adjustment Salary'' means, with respect to a Salary Cap Year, the amount of Salary in respect of an Adjustment Contract that is included in Adjusted Total Salaries for such Salary Cap Year.
  \item
    ``Adjustment Schedules'' means the schedules prepared by the NBA with respect to each Salary Cap Year in advance of each semi-monthly payment date setting forth for each player the applicable Compensation adjustment to be applied in respect of the applicable Compensation payment.
  \item
    ``Aggregate Reduction Amount'' means, with respect to a Salary Cap Year, an amount equal to the sum of the Contract Reduction Amounts for all Adjustment Contracts for such Salary Cap Year.
  \item
    ``Aggregate Team Overage Balance'' means, with respect to a Salary Cap Year, an amount equal to the sum of all Team Overage Balances for such Salary Cap Year.
  \item
    ``Audit Report Challenge Period'' means the period beginning with the date on which an Interim Audit Report is issued by the Accountants and ending on the last date by which all challenges thereto brought pursuant to Article XXXII, Section 9(b) are resolved.
  \item
    ``Carryover Amount'' means, with respect to a Salary Cap Year, (i) the amount, if any, by which the Uncapped Reduction Amount in respect of the immediately preceding Salary Cap Year exceeded the Aggregate Reduction Amount in respect of such immediately preceding Salary Cap Year, less (ii) the amount, if any, by which the Shortfall Amount for the immediately preceding Salary Cap Year exceeded the amount distributed to players pursuant to Section 12(e)(3) below with respect to such immediately preceding Salary Cap Year.
  \item
    ``Carryover Interest Rate'' means, with respect to a Salary Cap Year, a rate equal to the Secured Overnight Financing Rate as published in The Wall Street Journal on the first day of such Salary Cap Year, plus 1.225\%.
  \item
    ``Contract Reduction Amount'' means for an Adjustment Contract, with respect to a Salary Cap Year:

    \begin{enumerate}
    \def\labelenumiii{(\roman{enumiii})}
    \tightlist
    \item
      Prior to the completion of the Governing Audit Report: an amount equal to the Adjustment Salary in respect of such Adjustment Contract for such Salary Cap Year multiplied by the Withholding Percentage; and
    \item
      For purposes of, and following the completion of, the Governing Audit Report: an amount equal to the Adjustment Salary in respect of such Adjustment Contract for such Salary Cap Year multiplied by the Actual Reduction Percentage.
    \end{enumerate}
  \item
    ``Designated Share'' means, with respect to a Salary Cap Year, fifty percent (50\%) of BRI for such Salary Cap Year, provided that the Designated Share for a Salary Cap Year shall be increased or decreased in accordance with the following:

    \begin{enumerate}
    \def\labelenumiii{(\roman{enumiii})}
    \tightlist
    \item
      in the event that BRI for a Salary Cap Year exceeds Forecasted BRI for such Salary Cap Year, then the Designated Share for such Salary Cap Year shall equal fifty percent (50\%) of Forecasted BRI for such Salary Cap Year, plus sixty and one-half percent (60.5\%) of the difference between BRI for such Salary Cap Year and Forecasted BRI for such Salary Cap Year; and
    \item
      in the event that Forecasted BRI for a Salary Cap Year exceeds BRI for such Salary Cap Year, then the Designated Share for such Salary Cap Year shall equal fifty percent (50\%) of Forecasted BRI for such Salary Cap Year, less sixty and one-half percent (60.5\%) of the difference between Forecasted BRI for such Salary Cap Year and BRI for such Salary Cap Year.
    \end{enumerate}
  \end{enumerate}

  Notwithstanding anything to the contrary in the foregoing, in no event shall the Designated Share for any Salary Cap Year be less than forty-nine percent (49\%) of BRI for such Salary Cap Year or greater than fifty-one percent (51\%) of BRI for such Salary Cap Year.

  \emph{To illustrate the foregoing:}

  \begin{enumerate}
  \def\labelenumii{(\Alph{enumii})}
  \setcounter{enumii}{23}
  \item
    \emph{if BRI for a Salary Cap Year were to equal \$10 billion, and Forecasted BRI for such Salary Cap Year were to equal \$9.5 billion, then the Designated Share for such Salary Cap Year would equal \$5.0525 billion (i.e., \$4.75 billion (i.e., Forecasted BRI of \$9.5 billion multiplied by 50\%) plus \$0.3025 billion (i.e., 60.5\% of \$0.5 billion -- the difference between BRI of \$10 billion and Forecasted BRI of \$9.5 billion)), which would equate to 50.525\% of BRI;}
  \item
    \emph{if BRI for a Salary Cap Year were to equal \$9.5 billion, and Forecasted BRI for such Salary Cap Year were to equal \$10 billion, then the Designated Share for such Salary Cap Year would equal \$4.6975 billion (i.e., \$5 billion (i.e., Forecasted BRI of \$10 billion multiplied by 50\%) less \$0.3025 billion (i.e., 60.5\% of \$0.5 billion -- the difference between Forecasted BRI of \$10 billion and BRI of \$9.5 billion)), which would equate to 49.4474\% of BRI; and}
  \item
    \emph{if BRI for a Salary Cap Year were to equal \$10 billion, and Forecasted BRI for such Salary Cap Year were to equal \$9 billion, then the Designated Share for such Salary Cap Year would equal \$5.1 billion or 51\% of BRI since the amount per the calculation would exceed 51\% of BRI (i.e., \$4.5 billion (i.e., Forecasted BRI of \$9 billion multiplied by 50\%) plus \$0.605 billion (i.e., 60.5\% of \$1 billion -- the difference between BRI of \$10 billion and Forecasted BRI of \$9 billion) would equal \$5.105 billion or 51.05\% of BRI).}
  \end{enumerate}

  \begin{enumerate}
  \def\labelenumii{(\arabic{enumii})}
  \setcounter{enumii}{15}
  \tightlist
  \item
    ``Distribution Amount'' means, with respect to a Salary Cap Year, an amount equal to the sum of the Aggregate Reduction Amount for such Salary Cap Year and any Shortfall Amount for such Salary Cap Year, allocated in accordance with Section 12(e) below.
  \item
    ``Forecasted BRI'' means:

    \begin{enumerate}
    \def\labelenumiii{(\roman{enumiii})}
    \tightlist
    \item
      With respect to the 2023-24 Salary Cap Year, ninety and forty-eight hundredths percent (90.48\%) of BRI for the 2022-23 Salary Cap Year; and
    \item
      With respect to each Salary Cap Year beginning with the 2024-25 Salary Cap Year, one hundred four and one-half percent (104.5\%) of Forecasted BRI for the immediately preceding Salary Cap Year.
    \end{enumerate}
  \item
    ``Governing Audit Report'' means, with respect to a Salary Cap Year, the Audit Report for such Salary Cap Year, or, if no final Audit Report has been submitted at the conclusion of the Audit Report Challenge Period, the Interim Designated Share Audit Report for such Salary Cap Year.
  \item
    ``Interest Amount'' means, with respect to a Salary Cap Year, the Carryover Interest Rate for such Salary Cap Year multiplied by the Carryover Amount with respect to such Salary Cap Year.
  \item
    ``Overage'' or ``Overage Amount'' means, with respect to a Salary Cap Year, the amount, if any, by which Adjusted Total Salaries and Benefits for such Salary Cap Year exceeds the Designated Share for such Salary Cap Year.
  \item
    ``Shortfall Amount'' means, with respect to a Salary Cap Year, the amount, if any, by which the Designated Share for such Salary Cap Year exceeds Adjusted Total Salaries and Benefits for such Salary Cap Year.
  \item
    ``Team Overage Balance'' means for a Team, with respect to a Salary Cap Year, the Overage Amount (if any) for such Salary Cap Year to which such Team is entitled (calculated in accordance with Section 12(d) below), adjusted in accordance with Section 12(e) below.
  \item
    ``Uncapped Reduction Amount'' means, with respect to a Salary Cap Year, the sum of the Overage Amount, the Carryover Amount, and the Interest Amount with respect to such Salary Cap Year.
  \item
    ``Withholding Percentage'' means, with respect to a Salary Cap Year, ten percent (10\%), provided that in the event that the Salary Cap for such Salary Cap Year is limited to one hundred ten percent (110\%) of the Salary Cap for the immediately preceding Salary Cap Year pursuant to Section 2(a)(5) above, then the NBA and Players Association shall discuss in good faith reducing the Withholding Percentage (i.e., to a percentage that is less than ten percent (10\%)), taking into account reasonable estimates of Team and League financial performance (accounting for attendant risks and the likely size of any Shortfall Amount).
  \end{enumerate}
\item
  \textbf{Benefit Adjustment.}

  \begin{enumerate}
  \def\labelenumii{(\arabic{enumii})}
  \tightlist
  \item
    In the event that, for a Salary Cap Year, prior to any reduction pursuant to this Section 12(b)(1), the Uncapped Reduction Amount less any Shortfall Amount would exceed ten percent (10\%) of Adjusted Total Salaries, the Additional Benefit Amount as provided for by Article IV, Section 4(d)(1) (i.e., the one percent (1\%) of BRI amount for additional benefits) shall be reduced by such excess amount (or, if such excess amount is greater than the Additional Benefit Amount, then the Additional Benefit Amount shall be reduced in full). Any reduction to the Additional Benefit Amount for a Salary Cap Year pursuant to this Section 12(b)(1) shall be deducted from Total Benefits (and, thus, Adjusted Total Benefits) for such Salary Cap Year (thus decreasing the Overage Amount and/or increasing the Shortfall Amount for such Salary Cap Year, as applicable).
  \item
    For purposes of calculating Projected Benefits (and, thus, the Salary Cap) for a Salary Cap Year, no reduction expected to be made pursuant to Section 12(b)(1) above shall be taken into account.
  \end{enumerate}
\item
  \textbf{Compensation Adjustments.}

  \begin{enumerate}
  \def\labelenumii{(\arabic{enumii})}
  \tightlist
  \item
    For each Salary Cap Year, each Compensation payment made to a player in respect of the Season encompassed by such Salary Cap Year pursuant to an Adjustment Contract shall be adjusted by the percentage reduction that, when applied to each remaining Compensation payment in respect of that Season pursuant to such Adjustment Contract, and taking into account any Compensation adjustments already made pursuant to this Section 12(c), would result in a reduction (pursuant to this Section 12(c)) of the total Compensation payable to such player pursuant to the Adjustment Contract in respect of that Season equal to the then-applicable Contract Reduction Amount.
  \item
    In the event that, as of the completion of the Governing Audit Report, the Compensation payable to a player pursuant to the Adjustment Contract has already been reduced pursuant to this Section 12(c) by an amount that exceeds the then-applicable Contract Reduction Amount, then such excess shall be paid to the player in equal installments over the remaining semi-monthly payment dates on which payments are due to such player for the applicable Season pursuant to the Adjustment Contract beginning with either the next semi-monthly payment date following the issuance of the Governing Audit Report or, if practicability warrants, the second semi-monthly payment date following the issuance of the Governing Audit Report (or, if there are no remaining payments due to such player for the applicable Season pursuant to the Adjustment Contract, such excess shall be paid to the player within sixty (60) days following the completion of the Governing Audit Report).
  \item
    If for any reason, in respect of a Salary Cap Year, the Contract Reduction Amount for an Adjustment Contract exceeds the amount by which the Compensation provided for by such Contract was decreased pursuant to this Section 12(c) above, then the Players Association shall make good faith efforts to facilitate the applicable Team's recovery of such excess from the applicable player via a direct payment. In the event any amount remains outstanding as of the first semi-monthly payment date for the immediately following Salary Cap Year, such outstanding amount shall be subtracted from the Contract Reduction Amount for such Adjustment Contract in respect of the Salary Cap Year with respect to which such amount remains outstanding (and the parties shall adjust (or, if necessary, deem amended) the Governing Audit Report to reflect such decrease).
  \item
    Within seven (7) days after receiving any set of Adjustment Schedules from the NBA, or within seven (7) days after any event that the Players Association believes warrants a change in any previously-issued Adjustment Schedules, the Players Association may bring a proceeding before the System Arbitrator, in accordance with Article XXXII, Section 10, contesting the NBA's calculation of any player's Compensation adjustment pursuant to this Section 12(c). Notwithstanding the commencement of any such proceeding, each Team shall continue making Compensation adjustments in accordance with this Section 12(c), and in no event shall any Team be prohibited from making such Compensation adjustments prior to a final determination in any such proceeding. In the event that the NBA makes a determination, or a final determination is made in a proceeding in accordance with this Section 12(c)(4), that an adjustment to a player's Compensation was erroneously calculated by the NBA, the sole remedy with respect to any amounts erroneously deducted from the player's Compensation shall be to modify, as soon as practicable, the deduction amounts applicable to such player so as to reduce, in equal amounts, all scheduled future deductions from post-determination payments of Compensation until the amount of any prior over-deduction is fully offset; provided, however, that to the extent that reducing the player's future deductions would not fully offset the prior over-deductions, the NBA shall instruct the applicable Team to pay the player as soon as practicable such additional amounts as are necessary to fully offset such over-deductions.
  \end{enumerate}
\item
  \textbf{Team Overage Balance.} In respect of each Salary Cap Year, each Team's Team Overage Balance shall (prior to any adjustments made in accordance with Section 12(e) below) equal:

  \begin{enumerate}
  \def\labelenumii{(\arabic{enumii})}
  \tightlist
  \item
    If there is an Overage Amount that is less than or equal to ten percent (10\%) of Adjusted Total Salaries: the Overage Amount divided by thirty (30).
  \item
    If there is an Overage Amount that is greater than ten percent (10\%) of Adjusted Total Salaries: an amount equal to the sum of (i) ten percent (10\%) of Adjusted Total Salaries divided by thirty (30), and (ii) the amount by which the Overage exceeds ten percent (10\%) of Adjusted Total Salaries, multiplied by a fraction, the numerator of which is the Team's Adjusted Team Salary for such Salary Cap Year, and the denominator of which is Adjusted Total Salaries for such Salary Cap Year.
  \item
    If the Overage Amount is zero (0): zero (0).
  \end{enumerate}
\item
  \textbf{Allocation of Distribution Amount.} Each Salary Cap Year, as part of the Governing Audit Report, the following processes will apply with respect to Team Overage Balances and the Distribution Amount for such Salary Cap Year (the ``Distribution Year''):

  \begin{enumerate}
  \def\labelenumii{(\arabic{enumii})}
  \tightlist
  \item
    Beginning with the earliest Salary Cap Year in respect of which the Aggregate Team Overage Balance is greater than zero (0):

    \begin{enumerate}
    \def\labelenumiii{(\roman{enumiii})}
    \tightlist
    \item
      If the Distribution Year is later than such Salary Cap Year, then each Team's Team Overage Balance in respect of such Salary Cap Year shall be increased by an amount equal to its Team Overage Balance in respect of such Salary Cap Year multiplied by the Carryover Interest Rate in respect of the Distribution Year (and thus the Aggregate Team Overage Balance in respect of such Salary Cap Year shall be increased by the sum of the increase to each Team's Team Overage Balance in accordance with this Section 12(e)(1)(i)); and
    \item
      With respect to such Salary Cap Year, the Distribution Amount for the Distribution Year shall be allocated as follows:

      \begin{enumerate}
      \def\labelenumiv{(\Alph{enumiv})}
      \tightlist
      \item
        If the Distribution Amount is greater than or equal to the Aggregate Team Overage Balance in respect of such Salary Cap Year, each Team shall be allocated a portion of the Distribution Amount equal to its Team Overage Balance in respect of such Salary Cap Year. As a result, each Team's Team Overage Balance in respect of such Salary Cap Year (and thus the Aggregate Team Overage Balance in respect of such Salary Cap Year) shall be reduced to zero (0); or
      \item
        If the Distribution Amount is less than the Aggregate Team Overage Balance in respect of such Salary Cap Year, each Team shall be allocated a portion of the Distribution Amount in proportion to its Team Overage Balance in respect of such Salary Cap Year. As a result, each Team's Team Overage Balance in respect of such Salary Cap Year shall be reduced by its respective allocated amount (and thus the Aggregate Team Overage Balance in respect of such Salary Cap Year shall be reduced by the sum of such allocated amounts).
      \end{enumerate}
    \end{enumerate}
  \item
    The process described in Section 12(e)(1) above shall then be repeated, in chronological order, for each successive Salary Cap Year for which the Aggregate Team Overage Balance is greater than zero (0), utilizing the portion of the Distribution Amount for the Distribution Year that has not yet been allocated to Teams in respect of an earlier Salary Cap Year.
  \item
    If, as a result of the existence of a Shortfall Amount for the Distribution Year and following the procedures described in Sections 12(e)(1)-(2) above, the portion of the Distribution Amount that has not been allocated to Teams is greater than zero (0), such unallocated portion of the Distribution Amount shall then be allocated to each player in proportion to the Adjustment Salary provided for by his Adjustment Contract(s) (as a percentage of Adjusted Total Salaries) for the Distribution Year. Notwithstanding the foregoing, if the portion of the Distribution Amount to be allocated to players pursuant to this Section 12(e)(3) in respect of the Distribution Year exceeds the Aggregate Reduction Amount in respect of such Salary Cap Year, then such excess, rather than being allocated to each player in proportion to the Adjustment Salary provided for by his Adjustment Contract(s), shall be allocated to each player on such proportional basis as may be reasonably determined by the Players Association.
  \end{enumerate}

  \emph{Example: Assume (i) as set forth in the Governing Audit Report for the 2024-25 Salary Cap Year (i.e., the Distribution Year), the 2024-25 Distribution Amount is \$500 million, and (ii) prior to any allocations of the 2024-25 Distribution Amount pursuant to this Section 12(e), there is a 2023-24 Aggregate Team Overage Balance of \$200 million and a 2024-25 Aggregate Team Overage Balance of \$400 million, and (iii) the 2024-25 Carryover Interest Rate is five percent (5\%).}

  \emph{The earliest Salary Cap Year in which the Aggregate Team Overage Balance is greater than zero (0) is the 2023-24 Salary Cap Year. Because the 2024-25 Distribution Year is later than the 2023-24 Salary Cap Year, each Team's 2023-24 Team Overage Balance (and hence the 2023-24 Aggregate Team Overage Balance) would be increased by five percent (5\%) (i.e., the 2024-25 Carryover Interest Rate). As a result of such increase, the 2023-24 Aggregate Team Overage Balance would be \$210 million (i.e., \$200 million increased by five percent (5\%)).}

  \emph{The 2024-25 Distribution Amount of \$500 million is greater than the 2023-24 Aggregate Team Overage Balance of \$210 million. Accordingly, \$210 million of the 2024-25 Distribution Amount would be allocated to Teams in amounts equal to each Team's 2023-24 Team Overage Balance. As a result of such allocation, each Team's 2023-24 Team Overage Balance (and hence the 2023-24 Aggregate Team Overage Balance) would be zero (0).}

  \emph{The next Salary Cap Year in which the Aggregate Team Overage Balance is greater than zero (0) is the 2024-25 Salary Cap Year. The Distribution Year is not later than the 2024-25 Salary Cap Year and thus neither each Team's 2024-25 Team Overage Balance nor the 2024-25 Aggregate Team Overage Balance would be increased pursuant to Section 12(e)(1)(i) above.}

  \emph{The unallocated portion of the 2024-25 Distribution Amount is \$290 million (\$500 million less the \$210 million amount that was allocated in respect of the 2023-24 Salary Cap Year), which is less than the 2024-25 Aggregate Team Overage Balance of \$400 million. Accordingly, \$290 million of the 2024-25 Distribution Amount would be allocated to Teams in proportion to each Team's 2024-25 Team Overage Balance. As a result of such allocation, each Team's 2024-25 Team Overage Balance would be decreased by its respective allocation of the 2024-25 Distribution Amount (and hence the 2024-25 Aggregate Team Overage Balance would be reduced by \$290 million), resulting in a 2024-25 Aggregate Team Overage Balance of \$110 million (\$400 million less the \$290 million amount allocated from the 2024-25 Distribution Amount).}

  \emph{In accordance with Section 12(e)(3) above, no portion of the 2024-25 Distribution Amount would be allocated to players in respect of 2024-25 Adjustment Contracts.}
\item
  \textbf{Team Reconciliation Payments.} The NBA shall facilitate the following payments to be made within sixty (60) days following the completion of the Governing Audit Report in respect of each Salary Cap Year, based on the allocations described in Section 12(e) above. Each Team shall be entitled to receive or, if the result of the following calculation is negative for such Team, required to pay, the following amount in respect of a Salary Cap Year:

  \begin{enumerate}
  \def\labelenumii{(\arabic{enumii})}
  \tightlist
  \item
    The sum of: (i) the total Distribution Amount allocated to the Team in accordance with Sections 12(e)(1)-(2) above; and (ii) the portion of the Distribution Amount allocated in accordance with Section 12(e)(3) above to each player in respect of an Adjustment Contract for which the Team is (or was) the last Team responsible for making payments to the player in respect of the Season encompassed by the Salary Cap Year for which the Governing Audit Report was just completed; less
  \item
    The sum of: (i) the total amount by which such Team reduced (or is scheduled to reduce) Compensation payments to players in respect of the Season encompassed by such Salary Cap Year pursuant to Section 12(c) above; and (ii) the Shortfall Amount (if any) in respect of such Salary Cap Year divided by the number of Teams that played in the NBA during such Salary Cap Year.
  \end{enumerate}
\item
  \textbf{Player Reconciliation Payments.} Following completion of the Governing Audit Report in respect of each Salary Cap Year, each player to whom a portion of the Distribution Amount is allocated in accordance with Section 12(e)(3) above shall be paid the amount of such allocation in respect of an Adjustment Contract by the final Team responsible for making payments to such player pursuant to such Adjustment Contract in respect of the Salary Cap Year for which the Governing Audit Report was just completed. Such payment shall be made to the player in equal installments over the remaining semi-monthly dates on which payments are due to the player for the applicable Season pursuant to the Adjustment Contract beginning with either the next semi-monthly payment date following the issuance of the Governing Audit Report or, if practicability warrants, the second semi-monthly payment date following the issuance of the Governing Audit Report (or, if there are no remaining payments due to the player for such Season pursuant to the Adjustment Contract, in one (1) payment to be made within sixty (60) days following the completion of the Governing Audit Report).
\item
  \textbf{Survival of Obligation and Terms.} In the event that, upon the expiration or termination of this Agreement, there is an Aggregate Team Overage Balance in respect of any Salary Cap Year (including the final Salary Cap Year of this Agreement) that is greater than zero (0), the sum total of any Aggregate Team Overage Balances shall be due and owing by the players to the Teams and shall be recouped in full by the Teams under a successor collective bargaining agreement via reductions to the Compensation otherwise payable to players (i) no later than the Salary Cap Years in which such Aggregate Team Overage Balances would have been recouped pursuant to this Section 12 had these provisions continued in effect, and (ii) using the method described in this Section 12 or such other method as is mutually agreed by the parties. Notwithstanding any other provision of this Agreement, the terms of this Section 12(h) shall survive the expiration or termination of this Agreement. For clarity, nothing in this Section 12(h) shall impact any right to engage in any strikes, lockouts, or cessations or other stoppages of work following the expiration or termination of this Agreement.
\item
  \textbf{Information to Be Included in Audit Report.} The parties shall cause the Accountants to include in the Interim Audit Report and the Governing Audit Report for each Salary Cap Year schedules setting forth, with respect to such Salary Cap Year:

  \begin{enumerate}
  \def\labelenumii{(\arabic{enumii})}
  \tightlist
  \item
    BRI, the Designated Share, Total Salaries and Benefits, and Adjusted Total Salaries and Benefits;
  \item
    the Overage Amount or the Shortfall Amount (as applicable);
  \item
    the Actual Reduction Percentage;
  \item
    the Aggregate Reduction Amount;
  \item
    the Distribution Amount;
  \item
    the Aggregate Team Overage Balance and each Team's Team Overage Balance, both before and after the application of Sections 12(e)(1)-(2) above;
  \item
    for each prior Salary Cap Year in respect of which the Aggregate Team Overage Balance is greater than zero (0) prior to the application of Sections 12(e)(1)-(2) above, the Aggregate Team Overage Balance and each Team's Team Overage Balance in respect of each such prior Salary Cap Year, both before and after the application of Sections 12(e)(1)-(2) above;
  \item
    a listing of each Team and the Distribution Amount allocated to each such Team in accordance with Sections 12(e)(1)-(2) above;
  \item
    a listing of each Adjustment Contract, the associated Contract Reduction Amount, and the Distribution Amount allocated to the player in respect of such Adjustment Contract in accordance with Section 12(e)(3) above;
  \item
    a summary of the reconciliation payments described in Sections 12(f)-(g) above;
  \item
    the amount (if any) by which each Team's Tax Team Salary (as computed pursuant to Section 2(d) above) exceeds the Tax Level, and the resulting tax payment due by the Team;
  \item
    the amount (if any) of any Minimum Team Salary payment owed by a team in accordance with Section 2(c) above; and
  \item
    the amount (if any) by which each Team's Apron Team Salary (as computed pursuant to Section 2(e) above) exceeds the Second Apron Level.
  \end{enumerate}
\item
  \textbf{Miscellaneous.}

  \begin{enumerate}
  \def\labelenumii{(\arabic{enumii})}
  \tightlist
  \item
    For all purposes under this Agreement, the computation of a player's Salary or Adjustment Salary shall be made without regard to any adjustment made (or to be made) to such player's Compensation in accordance with this Section 12.
  \item
    When (i) pursuant to Article VI, Section 1 or Article XLI, Section 4(e), a player has forfeited a portion of his Compensation for a Season (payable to him pursuant to an Adjustment Contract) (the ``forfeited amount'') and (ii) following the completion of the Governing Audit Report, the Contract Reduction Amount for such Adjustment Contract for the applicable Salary Cap Year is greater than zero (0), then the player shall be entitled to a refund of a portion of the forfeited amount. The refund shall be in an amount equal to the Contract Reduction Amount for the Adjustment Contract for the Salary Cap Year to which the forfeited amount related multiplied by a fraction, the numerator of which is the forfeited amount, and the denominator of which is the player's Base Compensation for such Season pursuant to the Adjustment Contract as of the date(s) the Compensation was forfeited. For clarity, the amount of the refund shall be less all amounts required to be withheld by any governmental authority. For purposes of the foregoing calculation, a player's Contract Reduction Amount shall be deemed to include only the portion of the player's Contract Reduction Amount that relates to the Base Compensation for the applicable Season set forth in the applicable Adjustment Contract. Such refund shall be made to the player within sixty (60) days following the completion of the Governing Audit Report for the Salary Cap Year in which the forfeited amount is collected.
  \end{enumerate}
\end{enumerate}

\hypertarget{rookie-scale}{%
\chapter{ROOKIE SCALE}\label{rookie-scale}}

\hypertarget{rookie-scale-contracts-for-first-round-picks.}{%
\section{Rookie Scale Contracts for First Round Picks.}\label{rookie-scale-contracts-for-first-round-picks.}}

\begin{enumerate}
\def\labelenumi{(\alph{enumi})}
\item
  Each Rookie Scale Contract between a Team and a First Round Pick shall cover a period of two (2) Seasons, but shall have an Option in favor of the Team for the player's third Season and a second Option in favor of the Team for the player's fourth Season. The Option for the player's third Season shall be exercisable during the period from the day following the last day of the first Season through the immediately following October 31. The Option for the player's fourth Season shall be exercisable during the period from the day following the last day of the second Season through the immediately following October 31. (For clarity, consistent with the rule set forth in Article XLII, Section 2, if October 31 in any year falls on a Saturday, Sunday, or Federal Holiday, then the deadline for exercising Options in Rookie Scale Contracts shall be deemed to fall on the following business day.) Such Options shall be exercisable by notice to the player that is either personally delivered to the player or his representative or sent by email or pre-paid certified, registered, or overnight mail to the last known address of the player or his representative, signed by the Team, informing the player that the Team has exercised such Option.
\item
  \begin{enumerate}
  \def\labelenumii{(\roman{enumii})}
  \tightlist
  \item
    The Rookie Salary Scale applicable to a First Round Pick is determined by the first Season to be covered by the player's Rookie Scale Contract. Accordingly, for example, if a player's Rookie Scale Contract commences with the 2023-24 Season, the 2023-24 Rookie Salary Scale shall apply. Within a particular Rookie Salary Scale, a First Round Pick's applicable Rookie Scale Amounts are determined by the player's selection number in the NBA Draft. Accordingly, for example, the Rookie Scale Amounts applicable to the eighth player selected in the first round of the NBA Draft shall be those specified in the applicable Rookie Salary Scale for the eighth pick. Notwithstanding anything to the contrary in this Section 1(b)(i) or in Section 1(b)(ii) below, beginning on January 10 of each Season, an unsigned First Round Pick's applicable Rookie Scale Amount for such Season shall be reduced daily through the end of the Regular Season by an amount equal to the applicable Rookie Scale Amount (as set forth in the applicable Rookie Salary Scale) multiplied by a fraction, the numerator of which is one (1) and the denominator of which is the total number of days in such Regular Season.
  \item
    Notwithstanding Section 1(b)(i) above, if, pursuant to any provision of this Agreement or the NBA Constitution and By-Laws, one (1) or more Teams is required to forfeit one (1) or more draft picks in the first round of a particular NBA Draft, then:

    \begin{enumerate}
    \def\labelenumiii{(\Alph{enumiii})}
    \tightlist
    \item
      the Rookie Salary Scale for the Salary Cap Year immediately following such Draft (or the Salary Cap Year of such Draft if the Draft occurs on or after July 1) shall be adjusted by removing one (1) or more Rookie Scale Amounts from the middle of the Rookie Salary Scale, as follows: if one (1) first round pick is forfeited, then the Rookie Scale Amounts that would have been applicable to the 15th player selected in the first round (absent any forfeiture of picks) (hereinafter, the ``15th Pick'') shall be removed from the Rookie Salary Scale; if two (2) first round picks are forfeited, then the Rookie Scale Amounts applicable to the 15th Pick and the pick immediately following the 15th Pick shall be removed from the Rookie Salary Scale; if three (3) first round picks are forfeited, then the Rookie Scale Amounts applicable to the 15th Pick and the picks immediately preceding and immediately following the 15th Pick shall be removed from the Rookie Salary Scale; and if more than three picks are forfeited, additional Rookie Scale Amounts shall be removed from the Rookie Salary Scale in accordance with the foregoing procedure; and
    \item
      the Rookie Scale Amounts applicable to players selected in such Draft shall be determined by their selection number under the Rookie Salary Scale as adjusted by Section 1(b)(ii)(A) above. Accordingly, for example, if one First Round Pick were forfeited in the first round of the 2024 Draft, the applicable Rookie Scale Amounts would remain unchanged for the first 14 picks, and the Rookie Scale Amounts applicable to the remaining 15 picks in the first round would be the Rookie Scale Amounts that (absent any forfeiture of picks) would have been applicable to picks 16 through 30.
    \end{enumerate}
  \end{enumerate}
\item
  \begin{enumerate}
  \def\labelenumii{(\roman{enumii})}
  \tightlist
  \item
    A Rookie Scale Contract shall provide in each of the two (2) Seasons covered by the Contract and the first Option Year for Current Base Compensation of at least the greater of: (A) eighty percent (80\%) of the applicable Rookie Scale Amount, and (B) the player's applicable Minimum Player Salary. Components of Salary in excess of the foregoing amount, if any, are subject to individual negotiation, except that (1) in no event may Salary plus Unlikely Bonuses for any Salary Cap Year exceed one hundred twenty percent (120\%) of the applicable Rookie Scale Amount, and (2) a Rookie Scale Contract may not provide for a signing bonus (except for an ``international player'' payment in excess of the Excluded International Player Payment Amount made in accordance with Article VII, Section 3(e)) or a loan. A Rookie Scale Contract may provide for a payment schedule in any Season that is more favorable to the player than that called for under Paragraph 3 of the Uniform Player Contract, subject to the other provisions of this Agreement.
  \item
    A Rookie Scale Contract must provide for Compensation protection for lack of skill and injury or illness in each of the two (2) Seasons covered by the Contract and the first Option Year of not less than eighty percent (80\%) of the applicable Rookie Scale Amount. To the extent permitted by Article II, Section 4(l), a Team and a First Round Pick may negotiate additional conditions or limitations applicable to the player's Base Compensation protection, except that lack of skill and injury or illness protection of at least eighty percent (80\%) of the applicable Rookie Scale Amount in each of the first two (2) Seasons and the first Option Year shall contain no such individually-negotiated additional conditions or limitations.
  \item
    The terms and conditions (other than with respect to the payment schedule for the player's Base Compensation) that apply to the second Option Year shall be unchanged from all terms and conditions that applied to the first Option Year (including, but not limited to, the percentage of Base Compensation that is protected), except that the Salary (excluding Incentive Compensation) and, if the Rookie Scale Contract provides for Incentive Compensation for the first Option Year, then the amount of each bonus, for the second Option Year shall be increased over the Salary (excluding Incentive Compensation) and amount of each bonus, respectively, for the first Option Year by the applicable percentage specified in the applicable Rookie Salary Scale.
  \end{enumerate}
\item
  Notwithstanding any other provision of this Agreement, if a trade of a Rookie Scale Contract would, by reason of a trade bonus contained in such Contract, cause the player's Salary plus Unlikely Bonuses for the Salary Cap Year in which such trade occurs to exceed one hundred twenty percent (120\%) of the player's applicable Rookie Scale Amount for such Salary Cap Year, such player's trade bonus shall be deemed amended to the extent necessary to reduce the player's Salary plus Unlikely Bonuses for such Salary Cap Year to one hundred twenty percent (120\%) of the applicable Rookie Scale Amount.
\end{enumerate}

\hypertarget{rookie-contracts-for-later-signed-first-round-picks.}{%
\section{Rookie Contracts for Later-Signed First Round Picks.}\label{rookie-contracts-for-later-signed-first-round-picks.}}

Except as provided in Section 3 below, a First Round Pick who does not sign with the Team that holds his draft rights for any portion of the three (3) Seasons following the NBA Draft in which he was selected (and who did not play intercollegiate basketball during such period) may enter into either (a) a Rookie Scale Contract in accordance with Section 1 above, or (b) if the Team has Room in excess of the applicable first-year Rookie Scale Amount and subject to the provisions of Article VII, a Contract covering no fewer than three (3) Seasons (not including any Option Year) that provides for Base Compensation in the first Season greater than one hundred twenty percent (120\%) of the applicable first-year Rookie Scale Amount.

\hypertarget{loss-of-draft-rights.}{%
\section{Loss of Draft Rights.}\label{loss-of-draft-rights.}}

If for any reason a Team fails to make a Required Tender to a First Round Pick in accordance with Article X, withdraws a Required Tender to a First Round Pick in accordance with Article X, or renounces a First Round Pick in accordance with Article X, or if a First Round Pick selected in a Subsequent Draft does not sign a Contract for a period of one (1) year following such Subsequent Draft in accordance with Article X, then the rules set forth in Sections 1 and 2 above shall not apply, and such First Round Pick shall become a Rookie Free Agent. In addition, any Team that fails to make a Required Tender to a First Round Pick, withdraws a Required Tender to a First Round Pick, renounces a First Round Pick, or fails to sign within one (1) year a First Round Pick selected in a Subsequent Draft shall be prohibited from signing such player until after he has signed a Player Contract with another NBA Team, and either (a) the player completes the playing services called for under the Contract, or (b) the Contract is terminated in accordance with the NBA waiver procedure.

\hypertarget{length-of-player-contracts}{%
\chapter{LENGTH OF PLAYER CONTRACTS}\label{length-of-player-contracts}}

\hypertarget{maximum-term.}{%
\section{Maximum Term.}\label{maximum-term.}}

Except where a shorter term is expressly provided for elsewhere in this Agreement, a Player Contract entered into after the effective date of this Agreement may cover, in the aggregate, up to but no more than four (4) Seasons from the date such Contract is signed; provided, however, that (a) a Player Contract between a Qualifying Veteran Free Agent and his Prior Team may cover, in the aggregate, up to but no more than five (5) Seasons from the date such Contract is signed, (b) an Extension of a Rookie Scale Contract may cover, in the aggregate, up to but no more than six (6) Seasons from the date such Extension is signed, (c) a Veteran Extension signed pursuant to Article VII, Section 7(a) (other than a Designated Veteran Player Extension) may cover, in the aggregate, up to but no more than five (5) Seasons from the date such Extension is signed, and (d) a Designated Veteran Player Extension with a Team's Designated Veteran Player must cover six (6) Seasons from the date such Extension is signed. For the avoidance of doubt and consistent with Article VII, Section 9(a)(2), the maximum Contract and Extension lengths described herein are inclusive of any Option Year contained in a Contract or Extension.

\hypertarget{computation-of-time.}{%
\section{Computation of Time.}\label{computation-of-time.}}

For purposes of Section 1 above and consistent with Article VII, Section 9(a)(1), if a Player Contract or Extension is signed after the beginning of a Season, the Season in which the Contract or Extension is signed shall be counted as one (1) full Season covered by the Contract or Extension; and in the case of an Extension that is signed during the period from the end of a Season through the immediately following June 30, the Season immediately preceding the signing of the Extension (i.e., the just-completed Season) shall be counted as one (1) full Season covered by the Extension.

\hypertarget{player-eligibility-and-nba-draft}{%
\chapter{PLAYER ELIGIBILITY AND NBA DRAFT}\label{player-eligibility-and-nba-draft}}

\hypertarget{player-eligibility.}{%
\section{Player Eligibility.}\label{player-eligibility.}}

\begin{enumerate}
\def\labelenumi{(\alph{enumi})}
\tightlist
\item
  No player may sign a Contract or play in the NBA unless he has been eligible for selection in at least one (1) NBA Draft. No player shall be eligible for selection in more than two (2) NBA Drafts.
\item
  A player shall be eligible for selection in the first NBA Draft with respect to which he has satisfied all applicable requirements of Section 1(b)(i) below and one of the requirements of Section 1(b)(ii) below:

  \begin{enumerate}
  \def\labelenumii{(\roman{enumii})}
  \item
    The player (A) is or will be at least nineteen (19) years of age during the calendar year in which the Draft is held, and (B) with respect to a player who is not an ``international player'' (defined below), at least one (1) NBA Season has elapsed since the player's graduation from high school (or, if the player did not graduate from high school, since the later of the graduation of the class with which the player would have graduated based on the high school class he was in when he (i) first enrolled in high school, or (ii) was last enrolled in high school); and
  \item
    \begin{enumerate}
    \def\labelenumiii{(\Alph{enumiii})}
    \tightlist
    \item
      The player has graduated from a four-year college or university in the United States (or is to graduate in the calendar year in which the Draft is held) and has no remaining intercollegiate basketball eligibility; or
    \item
      The player is attending or previously attended a four-year college or university in the United States, his original class in such college or university has graduated (or is to graduate in the calendar year in which the Draft is held), and he has no remaining intercollegiate basketball eligibility; or
    \item
      The player has graduated from high school in the United States, did not enroll in a four-year college or university in the United States, and four (4) calendar years have elapsed since such player's high school graduation; or
    \item
      The player did not graduate from high school in the United States, and four (4) calendar years have elapsed since the graduation of the class with which the player would have graduated had he graduated from high school; or
    \item
      The player is or will be at least twenty-two (22) years of age during the calendar year of the Draft, has signed a ``non-NBA professional basketball contract'' (defined below), and has rendered services under such contract prior to the January 1 immediately preceding such Draft; or
    \item
      The player is or will be twenty-two (22) years of age during the calendar year of the Draft and is an international player; or
    \item
      The player has expressed his desire to be selected in the Draft in a writing received by the NBA at least sixty (60) days prior to such Draft (an ``Early Entry'' player).
    \end{enumerate}
  \end{enumerate}
\item
  For purposes of this Article X, an ``international player'' is a player: (i) who has maintained a permanent residence outside of the United States for at least the three (3) years prior to the Draft, while participating in the game of basketball as an amateur or as a professional outside of the United States; (ii) who has never previously enrolled in a college or university in the United States; and (iii) who did not complete high school in the United States.
\item
  For purposes of this Article X:

  \begin{enumerate}
  \def\labelenumii{(\roman{enumii})}
  \tightlist
  \item
    A ``non-NBA professional basketball contract'' means a contract between a player and any non-NBA basketball team or league pursuant to which the team or league pays money or compensation of any kind -- in excess of a stipend for living expenses -- to the player for rendering services to a basketball team.
  \item
    A ``professional basketball team or league not in the NBA'' means any team or league that pays money or compensation of any kind -- in excess of a stipend for living expenses -- to a basketball player for rendering services to such team and/or league.
  \end{enumerate}
\end{enumerate}

\hypertarget{term-and-timing-of-draft-provisions.}{%
\section{Term and Timing of Draft Provisions.}\label{term-and-timing-of-draft-provisions.}}

An NBA Draft will be held prior to the commencement of each NBA Season covered by the term of this Agreement and, despite the expiration of the other terms of this Agreement pursuant to Article XXXIX, prior to the commencement of the NBA Season immediately following the final Season covered by the term of this Agreement. Each such Draft will be held prior to the July 10 preceding the commencement of the NBA Season on a date to be designated by the Commissioner.

\hypertarget{number-of-choices.}{%
\section{Number of Choices.}\label{number-of-choices.}}

\begin{enumerate}
\def\labelenumi{(\alph{enumi})}
\tightlist
\item
  The NBA Draft shall consist of two (2) rounds, with each round consisting of the same number of selections as there will be Teams in the NBA the following Season. Each Team shall be required to exercise any and all draft selections in its possession during each round of the Draft.
\item
  If, pursuant to any provision of this Agreement or the NBA Constitution and By-Laws, any Team is required to forfeit one or more draft pick(s) in a particular NBA Draft, the number of players selected in the applicable round of the Draft will be reduced by the number of such forfeitures. (Thus, for example, if Team A is required to forfeit the ninth pick in the first round of the Draft (at a time when there are thirty (30) NBA Teams), there will only be twenty-nine (29) players selected in the first round of such Draft.) In the event the forfeiture relates to one or more first round picks, the Rookie Salary Scale will be adjusted as set forth in Article VIII, Section 1(b)(ii). Other than as specifically agreed to herein, nothing contained in this Agreement shall be deemed to be an agreement of the Players Association to any provision of the NBA Constitution and By-Laws.
\end{enumerate}

\hypertarget{negotiating-rights-to-draft-rookies.}{%
\section{Negotiating Rights to Draft Rookies.}\label{negotiating-rights-to-draft-rookies.}}

\begin{enumerate}
\def\labelenumi{(\alph{enumi})}
\tightlist
\item
  A Team that drafts a player shall, during the period from the date of such NBA Draft (hereinafter, the ``Initial Draft'') to the date of the next Draft (hereinafter, the ``Subsequent Draft''), be the only Team with which such player may negotiate or sign a Player Contract, provided that, (i) on or before the July 15 immediately following the Initial Draft (for a First Round Pick), (ii) in the two (2) weeks before the September 5 immediately following the Initial Draft (for a Second Round Pick selected in an NBA Draft prior to the 2024 NBA Draft), or (iii) on or before the August 5 immediately following the Initial Draft (for a Second Round Pick selected in the 2024 NBA Draft or any subsequent NBA Draft), such Team has made a Required Tender to such player. If a Team has made a Required Tender to such a player and the player has not signed a Player Contract within the period between the Initial Draft and the Subsequent Draft, the Team that drafted the player shall lose its exclusive right to negotiate with the player and the player will then be eligible for selection in the Subsequent Draft.
\item
  A Team that, in the Subsequent Draft, drafts a player who (i) was drafted in the Initial Draft, (ii) received a Required Tender from the Team that drafted him in the Initial Draft, and (iii) did not sign a Player Contract with such first Team prior to the Subsequent Draft, shall, during the period from the date of the Subsequent Draft to the date of the next NBA Draft, be the only Team with which such player may negotiate or sign a Player Contract, provided such Team has made a Required Tender to such player by the applicable date specified in Section 4(a) above. If such player has not signed a Player Contract within the period between the Subsequent Draft and the next NBA Draft with the Team that drafted him in the Subsequent Draft, that Team shall lose its exclusive right, which it obtained in the Subsequent Draft, to negotiate with the player, and the player will become a Rookie Free Agent as of the date of the next NBA Draft.
\item
  If a player is drafted in an Initial Draft and (i) receives a Required Tender, (ii) does not sign a Player Contract with a Team prior to the Subsequent Draft, and (iii) is not drafted by any Team in such Subsequent Draft, the player will become a Rookie Free Agent immediately upon the conclusion of the Subsequent Draft.
\item
  If a Second Round Pick receives and signs a Required Tender and is subsequently waived by the Team after signing such Required Tender, then the Team that made the Required Tender to the player shall have exclusive rights to negotiate with and sign (or convert) the player to a Two-Way Contract for the Season covered by the Required Tender.
\item
  If a player is drafted by a Team in either an Initial or Subsequent Draft and that Team does not sign such player to a Player Contract or make a Required Tender to such player, the player will become a Rookie Free Agent on (i) the July 16 following such Draft (for a First Round Pick), (ii) on the September 6 following such Draft (for a Second Round Pick selected in an NBA Draft prior to the 2024 NBA Draft), or (iii) on the August 6 following such Draft (for a Second Round Pick selected in the 2024 NBA Draft or any subsequent NBA Draft).
\item
  A Team may at any time withdraw a Required Tender it has made to a player, provided that the player agrees in writing to the withdrawal. In the event that a Required Tender is withdrawn, the player shall thereupon become a Rookie Free Agent.
\item
  A Team that holds the exclusive rights to negotiate with and sign a drafted player may at any time renounce such exclusive rights, except that, if the Team has made a Required Tender to the player, a renunciation shall not be permitted during the time the player has to accept the Required Tender under Article I, Section 1(ddd). In order to renounce its exclusive rights with respect to a drafted player, a Team shall provide the NBA with an express, written statement renouncing such exclusive rights. The NBA shall provide a copy of such statement to the Players Association within three (3) business days following its receipt thereof.
\item
  Subject to the provisions of Article VII, and subject further to Article II, Section 15, a Team is free at any time beginning immediately following the conclusion of the NBA Draft to negotiate, and free at any time after the conclusion of the Moratorium Period to enter into, a Player Contract with a Draft Rookie who is subject to that Team's exclusive negotiating rights.
\end{enumerate}

\hypertarget{effect-of-contracts-with-other-professional-teams.}{%
\section{Effect of Contracts with Other Professional Teams.}\label{effect-of-contracts-with-other-professional-teams.}}

If a player is drafted by a Team in either an Initial or Subsequent Draft and, during a period in which he may negotiate and sign a Player Contract with only the Team that drafted him, either (x) is a party to a previously existing (i) non-NBA professional basketball contract or (ii) player contract with a professional basketball team or league not in the NBA that, in either case, covers all or any part of the NBA Season immediately following said Initial or Subsequent Draft, or (y) signs either such a player contract (either (x) or (y), a ``Non-NBA Signing''), then the following rules will apply:

\begin{enumerate}
\def\labelenumi{(\alph{enumi})}
\item
  Subject to Section 5(b) below, the Team that drafts the player shall retain the exclusive NBA rights to negotiate with and sign him for the period ending one (1) year from the earlier of the following two dates: (i) the date the player notifies such Team that he is available to sign a Player Contract with such Team immediately, provided that such notice will not be effective until the player is under no contractual or other legal impediment to sign and play with such Team for the then-current Season (if applicable) and any future Season; or (ii) the date of the NBA Draft occurring in the twelve-month period from August 1 to July 30 in which the player notifies such Team of his availability and intention to play in the NBA during the Season immediately following said twelve-month period, provided that such notice will not be effective until the player is under no contractual or other legal impediment to sign and play with such Team for the then-current Season (if applicable) and any future Season.
\item
  \begin{enumerate}
  \def\labelenumii{(\roman{enumii})}
  \tightlist
  \item
    If, by July 1, 2023, the player notifies the Team that has drafted him that by September 1, 2023 he will, immediately thereafter and for any future Season, be under no contractual or other legal impediment to sign and play with such Team, and provided that on September 1, 2023 the player is in fact under no such contractual or other legal impediment, then, in order to retain the exclusive NBA rights to negotiate with and sign the player as provided in Section 5(a), such Team must make a Required Tender to the player by September 10, 2023.
  \item
    If, by July 1 of any year following 2023, the player notifies the Team that has drafted him that by August 1 of such year he will, immediately thereafter and for any future Season, be under no contractual or other legal impediment to sign and play with such Team, and provided that on such August 1 the player is in fact under no such contractual or other legal impediment, then, in order to retain the exclusive NBA rights to negotiate with and sign the player as provided in Section 5(a), such Team must make a Required Tender to the player by August 10 of such year.
  \end{enumerate}
\item
  If the player gives the required notice by July 1 of any year, and the Team that drafted him fails to make a Required Tender by September 10 of such year (if such notice was provided by July 1, 2023) or August 10 of such year (if such notice was provided by July 1 of any year following 2023), the player shall thereupon become a Rookie Free Agent.
\item
  If, during the one-year period of exclusive NBA negotiating rights set forth in Section 5(a) above, the player signs a non-NBA professional basketball contract or a player contract with a professional basketball team or league not in the NBA and the player has not made a bona fide effort to negotiate a Player Contract with the Team possessing his exclusive NBA rights or such bona fide effort is made and such Team makes a Required Tender to such player in accordance with Section 5(b) above, then such Team shall retain the exclusive NBA rights to negotiate with and sign the player for additional one-year periods as measured in and in accordance with the provisions of Section 5(a) above.
\item
  If, during the one-year period of exclusive NBA negotiating rights set forth in subsection (a) above, (i) the player signs (x) a non-NBA professional basketball contract or (y) a player contract with a professional basketball team or league not in the NBA, (ii) the player has made a bona fide effort to negotiate a Player Contract with the Team possessing his exclusive NBA rights, and (iii) such Team fails to make a Required Tender to such player in accordance with Section 5(b) above, then the player shall thereupon become a Rookie Free Agent.
\item
  If, during the one-year period of exclusive NBA negotiating rights set forth in Section 5(a) above, the Team makes or has made a Required Tender to the player and the player does not sign (x) a non-NBA professional basketball contract or (y) a player contract with a professional basketball team or league not in the NBA, then (i) in the case of a player who was previously drafted in an Initial Draft, the next NBA Draft following such one-year period shall be deemed the Subsequent Draft as to such player, and the rules applicable to a player who is subject to a Subsequent Draft will apply, or (ii) in the case of a player who was previously drafted in a Subsequent Draft, such player shall become a Rookie Free Agent at the end of such one-year period.
\item
  Notice under this Section 5 shall be provided in writing by email, personal delivery, or pre-paid certified, registered, or overnight mail sent to the Team's principal address or principal office (as then listed in the NBA's records), to the attention of the Team's general manager and to the League Office (attention: General Counsel).
\end{enumerate}

\hypertarget{application-to-early-entry-players.}{%
\section{Application to ``Early Entry'' Players.}\label{application-to-early-entry-players.}}

If a player who is eligible for the Draft pursuant to Section 1(b)(ii)(G) above (an ``Early Entry'' player) is selected in such Draft by a Team, the following rules apply:

\begin{enumerate}
\def\labelenumi{(\alph{enumi})}
\tightlist
\item
  Subject to Section 5 above, if the player does not thereafter play intercollegiate basketball, then the Team that drafted him shall, during the period from the date of such Draft to the date of the Draft in which the player would, absent his becoming an Early Entry player, first have been eligible to be selected, be the only Team with which the player may negotiate or sign a Player Contract, provided that such Team makes a Required Tender to the player each year by the date specified in Section 4(a) above. For purposes hereof, the Draft in which such player would, absent his becoming an Early Entry player, first have been eligible to be selected, will be deemed the ``Subsequent Draft'' as to that player, and the rules applicable to a player who has been drafted in a Subsequent Draft will apply. If the player, having been selected in a Draft for which he was eligible as an Early Entry player, has not signed a Player Contract with the Team that drafted him in such Draft following a Required Tender by that Team and is not drafted in the Subsequent Draft (as defined in the previous sentence), he shall become a Rookie Free Agent.
\item
  Subject to Section 5 above, if the player does thereafter play intercollegiate basketball, then the Team that drafted him shall retain the exclusive NBA rights to negotiate with and sign the player for the period ending one (1) year from the date of the Draft in which the player would, absent his becoming an Early Entry player, first have been eligible to be selected, provided that such Team makes a Required Tender to the player each year by the date specified in Section 4(a) above. For purposes hereof, the Draft in which such player would, absent his becoming an Early Entry player, first have been eligible to be selected, will be deemed the ``Initial Draft'' as to that player. The next NBA Draft shall be deemed the ``Subsequent Draft'' as to that player, and the rules applicable to a player who has been drafted in a Subsequent Draft will apply.
\item
  Notwithstanding anything to the contrary in this Section 6 or in Section 5 above, a Non-NBA Signing by an Early Entry player shall never shorten the period of time during which such player may negotiate and sign a Player Contract only with the Team that drafted him.
\end{enumerate}

\hypertarget{assignment-of-draft-rights-and-effect-of-void-contracts.}{%
\section{Assignment of Draft Rights and Effect of Void Contracts.}\label{assignment-of-draft-rights-and-effect-of-void-contracts.}}

\begin{enumerate}
\def\labelenumi{(\alph{enumi})}
\tightlist
\item
  In the event that the exclusive right to negotiate with a player obtained in any NBA Draft is assigned by a Team to another Team, in accordance with NBA procedures, the Team to which such right has been assigned shall have the same, but no greater, right to negotiate with and sign such player as is possessed by the Team assigning such right, and such player shall have the same, but no greater, obligation to the Team to which such right has been assigned as he had to the Team assigning such right.
\item
  Notwithstanding anything to the contrary in Section 7(a) above, in the event that:

  \begin{enumerate}
  \def\labelenumii{(\roman{enumii})}
  \tightlist
  \item
    Pursuant to Section 4 or 5 above, a Team must make a Required Tender to a player in order to retain the exclusive right to negotiate with and sign such player to a Player Contract, but has not yet made such Required Tender; and
  \item
    On or before the applicable date set forth in Section 4(a) or 5(b) above, such Team engages in a trade conference call pursuant to which the Team assigns, subject to any applicable trade conditions, the exclusive right to another Team; then:

    \begin{enumerate}
    \def\labelenumiii{(\Alph{enumiii})}
    \tightlist
    \item
      if the trade is consummated, the Team to which such rights are assigned shall have the exclusive right to negotiate with and sign such player to a Player Contract, provided that such assignee Team makes a Required Tender to such player on or before the later of (1) the applicable date set forth in Section 4(a) or 5(b) above, and (2) the date that is three (3) days following the date on which the trade is consummated (i.e., the date that all conditions (if any) to the trade are satisfied); or
    \item
      if the trade is voided (e.g., due to the failure of a condition of a trade), the assignor Team shall retain the exclusive right to negotiate with and sign such player to a Player Contract, provided that such Team makes a Required Tender to such player on or before the later of (1) the applicable date set forth in Section 4(a) or 5(b) above, and (2) the date that is three (3) days following the date on which the trade is voided.
    \end{enumerate}
  \end{enumerate}
\item
  In the event that:

  \begin{enumerate}
  \def\labelenumii{(\roman{enumii})}
  \tightlist
  \item
    Pursuant to Section 4 or 5 above, a Team must make a Required Tender to a player in order to retain the exclusive right to negotiate with and sign such player to a Player Contract, but has not yet made such Required Tender;
  \item
    such Team signs such player to a Player Contract prior to the applicable date set forth in Section 4(a) or 5(b) above; and
  \item
    such Contract becomes void as a result of a Commissioner disapproval;
  \end{enumerate}

  then such Team shall have the exclusive right to negotiate with and sign such player to a Player Contract, provided that it makes a Required Tender to such player on or before the later of (A) the applicable date set forth in Section 4(a) or 5(b) above, and (B) the date that is three (3) days following the date of the Commissioner's disapproval.
\end{enumerate}

\hypertarget{general.-1}{%
\section{General.}\label{general.-1}}

\begin{enumerate}
\def\labelenumi{(\alph{enumi})}
\tightlist
\item
  The placement of a Rookie on the Armed Services List, or on any of the other lists described in the NBA By-Laws, or on any other list created by the NBA, shall not extend the period of exclusive negotiating rights which a Team has to any Draft Rookie beyond the period specified in this Agreement.
\item
  Nothing contained herein shall prevent the NBA, in accordance with the applicable provisions of the NBA Constitution and By-Laws, from prohibiting or otherwise responding to violations by Teams of the exclusive NBA rights obtained in any NBA Draft, as set forth or referred to in this Article. Other than as specifically agreed to herein, nothing contained in this Agreement shall be deemed to be an agreement by the Players Association to any provision of the NBA Constitution and By-Laws.
\item
  An Early Entry player who is eligible to be selected in the next NBA Draft pursuant to Section 1(b)(ii)(G) above shall be entitled to withdraw from such Draft by providing written notice that is received by the NBA ten (10) days prior to such Draft. A player shall not be entitled to withdraw from more than two (2) NBA Drafts.
\item
  Any claim by a player that a Contract offered as a Required Tender pursuant to this Article X fails to meet one or more of the criteria for a Required Tender shall be made by written notice to the Team (with copies sent to the NBA and the Players Association), no later than ten (10) days after the receipt of such Contract by the Players Association. Such notice must set forth the specific changes that the player asserts must be made to the offered Contract in order for it to constitute a Required Tender. Upon receipt of such notice, if the requested changes are necessary to satisfy the requirements of a Required Tender, the Team may, within five (5) business days, offer the player an amended Contract incorporating the requested changes. If the Team offers such an amended Contract, the player and the Players Association shall be precluded from asserting that such Contract does not constitute a timely and valid Required Tender.
\item
  For purposes of this Article X, any rights afforded to ``a Team that drafts a player'' shall also be afforded to any Team to which such rights are subsequently assigned.
\end{enumerate}

\hypertarget{nba-draft-combine.}{%
\section{NBA Draft Combine.}\label{nba-draft-combine.}}

\begin{enumerate}
\def\labelenumi{(\alph{enumi})}
\tightlist
\item
  Notwithstanding any other provision of this Agreement, any player invited by the NBA to attend the Draft Combine who is reasonably determined by the NBA in consultation with the Players Association to have failed to fulfill his obligation to fully participate in the Draft Combine in accordance with Article XXII, Section 14 shall be ineligible for selection in the NBA Draft immediately following such Draft Combine.
\item
  If a player is ineligible for selection in a Draft pursuant to Section 9(a) above and such player is:

  \begin{enumerate}
  \def\labelenumii{(\roman{enumii})}
  \tightlist
  \item
    an Early Entry player, then he will be deemed to have withdrawn from such Draft in accordance with Section 8(c) above (even if such player had previously withdrawn, or had previously been deemed to have withdrawn, from two (2) or more NBA Drafts); or
  \item
    not an Early Entry player, then he will be deemed to meet the criteria set forth in Section 1(b) or Section 4(a) above, as applicable, in respect of the Draft immediately following such Draft.
  \end{enumerate}

  For clarity, (x) consistent with Section 1(a) above, any player ineligible for selection in a Draft pursuant to Section 9(a) above may not sign a Contract or play in the NBA at any time following such Draft until he has been eligible for selection in at least one (1) NBA Draft following the Draft for which the player was ineligible for selection pursuant to Section 9(a) above, and (y) any player ineligible for selection in a Draft pursuant to Section 9(a) above shall remain subject to the provisions of Article XXII, Section 14, and Section 9(a) above, in respect of any future Draft Combine in which the player is invited by the NBA to participate following the Draft for which the player was ineligible for selection pursuant to Section 9(a) above.
\end{enumerate}

\hypertarget{combine-related-eligibility-disputes.}{%
\section{Combine-Related Eligibility Disputes.}\label{combine-related-eligibility-disputes.}}

\begin{enumerate}
\def\labelenumi{(\alph{enumi})}
\tightlist
\item
  Notwithstanding any other provision of this Agreement, the procedures set forth in this Section 10 shall apply to the resolution of a dispute regarding a determination by the NBA, in accordance with Section 9(a) above, that a player invited by the NBA to attend the Draft Combine has failed to fulfill his obligation to fully participate in the Draft Combine in accordance with Article XXII, Section 14 and therefore is ineligible for selection in the NBA Draft immediately following such Draft Combine (any such dispute, a ``Combine-Related Eligibility Dispute''). If in connection with any such Combine-Related Eligibility Dispute, there is any conflict between the procedures set forth in this Section 10 and those set forth elsewhere in this Agreement, the procedures set forth in this Section shall control.
\item
  Any Combine-Related Eligibility Dispute may be initiated, as set forth below, only by the Players Association, except that the Players Association may not initiate a Combine-Related Eligibility Dispute without the approval of the player concerned. Combine-Related Eligibility Disputes shall be heard by the System Arbitrator.
\item
  Any determination that a player is ineligible for selection in an NBA Draft in accordance with Section 9(a) above must be made no later than the day that is ten (10) days prior to the date of such Draft.
\item
  A Combine-Related Eligibility Dispute must be brought by the Players Association within two (2) days of the date of the eligibility determination by the NBA. The Players Association may initiate a Combine-Related Eligibility Dispute by serving a written notice thereof on the NBA, with a copy of such written notice to be filed with the System Arbitrator. Such written notice shall be accompanied by a witness list, relevant documents, and other evidentiary materials on which the Players Association intends to rely in its affirmative case. No later than the second day following the date on which the NBA received written notice of the Dispute, the NBA shall provide to the Players Association a witness list, relevant documents, and other evidentiary materials on which the NBA intends to rely in its affirmative case. Absent a showing of good cause, neither the Players Association nor the NBA may proffer, refer to, or rely on the testimony of any witness, document, or other evidentiary material in its affirmative case that has not been identified to the other side as required by this Section 10(d).
\item
  The System Arbitrator shall convene a hearing within three (3) days of the System Arbitrator's receipt of the NBA's submission of its witness list, relevant documents, and other evidentiary materials. The hearing shall take place by videoconference and shall last no longer than one (1) day. The Players Association and the NBA shall each have the right to participate in the hearing. The player whose eligibility is the subject of the proceeding shall have the right to attend the hearing.
\item
  The System Arbitrator shall render a decision within one (1) day following the date of the hearing, and the decision shall be accompanied by a written opinion. Notwithstanding the foregoing, if the System Arbitrator determines that expedition so requires, he/she shall accompany the decision with a written summary of the grounds upon which the decision is based, and a full written opinion may follow within a reasonable time thereafter. The decision of the System Arbitrator shall constitute full, final, and complete disposition of the dispute and shall be binding upon the parties to this Agreement and the player, and there shall be no appeal to the Appeals Panel.
\item
  If the Players Association prevails in the proceeding, the sole remedy shall be that the player is deemed eligible for the Draft in respect of which the dispute was brought.
\item
  For clarity, any ongoing dispute regarding a player's Draft eligibility shall not affect the NBA's scheduling or operation of, or right to hold, the Draft.
\item
  Should circumstances warrant, each of the deadlines set forth in this Section 10 may be reasonably modified by agreement of the NBA and Players Association.
\end{enumerate}

\hypertarget{free-agency}{%
\chapter{FREE AGENCY}\label{free-agency}}

\hypertarget{general-rules.}{%
\section{General Rules.}\label{general-rules.}}

\begin{enumerate}
\def\labelenumi{(\alph{enumi})}
\item
  Subject to the provisions of Article VII and this Article XI, and subject further to Article II, Section 15:

  \begin{enumerate}
  \def\labelenumii{(\roman{enumii})}
  \tightlist
  \item
    other than as provided in Sections 1(a)(iii) and (iv) below, a player who is an Unrestricted Free Agent, or will become an Unrestricted Free Agent on the immediately following July 1, is free at any time beginning at 6:00 p.m. eastern time on June 30 to negotiate, and free at any time after the conclusion of the applicable Moratorium Period to enter into, a Player Contract with any Team;
  \item
    other than as provided for in Section 1(a)(iii) below, a player who will become a Restricted Free Agent on the immediately following July 1 is (1) free at any time beginning at 6:00 p.m. eastern time on June 30 to negotiate a Player Contract with his Prior Team and to negotiate an Offer Sheet (as defined in Section 5(b) below) with any Team other than his Prior Team; (2) free beginning at 12:01 p.m. eastern time on the first day of the Moratorium Period to enter into an Offer Sheet (as defined in Section 5(b) below) with any Team other than his Prior Team; and (3) free at any time after the conclusion of the Moratorium Period to enter into a Player Contract with his Prior Team;
  \item
    a player who (1) will (or could as a result of the non-exercise of an Option or the exercise of an ETO) become an Unrestricted Free Agent or a Restricted Free Agent on the immediately following July 1, and (2) finished the Season on a Team's roster, is free at any time beginning on the day following the last day of such Season to negotiate a Player Contract with such Team; and
  \item
    a Non-Draft Rookie is free to negotiate a Player Contract with any Team beginning immediately following the conclusion of the NBA Draft for which he was first eligible and was not selected.
  \end{enumerate}

  For clarity, the rules set forth in Sections 1(a)(iii) and (iv) allowing the Free Agents described therein to begin negotiating Player Contracts at the times specified therein do not affect the time at which such Free Agents may begin entering into Player Contracts. Subject to Article II, Section 15, the time at which such Free Agents may begin entering into Player Contracts is as set forth in Sections 1(a)(i) and (ii) above.
\item
  Upon a finding by the Commissioner of a violation of the rules set forth in Section 1(a) above regarding the timing of free agency discussions, the Commissioner shall be authorized to:

  \begin{enumerate}
  \def\labelenumii{(\roman{enumii})}
  \tightlist
  \item
    impose a fine up to \$2,000,000 on any Team found to have committed such violation;
  \item
    direct the forfeiture of draft picks; and/or
  \item
    suspend any Team personnel found to have engaged in such violation.
  \end{enumerate}

  For clarity, (1) the Commissioner's authority described above is without limitation to any other penalties, remedies, or actions the Commissioner is otherwise authorized to impose or take under Article XIII, (2) any discipline imposed pursuant to this Section 1(b) shall not require as a predicate any finding of, or proceeding before, the System Arbitrator, and (3) any such discipline may be appealed by the Players Association to the System Arbitrator.
\item
  Prior to the conclusion of the Moratorium Period, players (or, for clarity, any person or entity acting with authority on behalf of a player) and Teams shall each be prohibited from stating publicly that the player and Team have reached agreement on the terms of a Player Contract (or amendment to a Player Contract) that, pursuant to Article II, Section 15, cannot be entered into until after the conclusion of such Moratorium Period; provided, however, that the foregoing prohibition shall not apply to players with respect to the Moratorium Period of the 2023-24 Salary Cap Year.
\item
  No compensation obligation of any kind to another Team shall be applicable to any Free Agent. No right of first refusal (``Right of First Refusal'') of any kind shall be applicable to any Free Agent other than a Restricted Free Agent.
\item
  \begin{enumerate}
  \def\labelenumii{(\roman{enumii})}
  \tightlist
  \item
    For purposes of this Agreement, ``Qualifying Offer'' means an offer of a Uniform Player Contract, signed by the Team, that:

    \begin{enumerate}
    \def\labelenumiii{(\arabic{enumiii})}
    \tightlist
    \item
      is either personally delivered to the player or his representative or sent by email or pre-paid certified, registered, or overnight mail to the last known address of the player or his representative (if sent by email with a copy to the Players Association);
    \item
      is for a period of one (1) year;
    \item
      provides for Salary (excluding Incentive Compensation), Likely Bonuses, and Unlikely Bonuses in the amounts described in (ii), (iii), and (iv) below;
    \item
      provides for one hundred percent (100\%) of the Base Compensation to be protected for lack of skill and injury or illness (with no individually-negotiated conditions or limitations on such protection and no other types of protection); provided, however, that Qualifying Offers for players finishing Two-Way Contracts shall not be subject to this Section 1(e)(i)(4) and shall instead be subject to the rules set forth in Section 1(e)(iii) below; and
    \item
      provides for one hundred percent (100\%) of the Base Compensation to be payable in accordance with Paragraph 3 of the Uniform Player Contract.
    \end{enumerate}
  \item
    For First Round Picks finishing their Rookie Scale Contracts, the Salary (excluding Incentive Compensation), Likely Bonuses, and Unlikely Bonuses contained in a Qualifying Offer shall be equal to the Salary (excluding Incentive Compensation), Likely Bonuses, and Unlikely Bonuses, respectively, provided in the fourth Salary Cap Year of the Rookie Scale Contract (``Fourth Year Salary'') increased by the percentage called for in the ``Qualifying Offer: Percentage Increase Over 4th Year Salary'' column in the Rookie Salary Scale applicable to the First Round Pick's Rookie Scale Contract; provided that:

    \begin{enumerate}
    \def\labelenumiii{(\Alph{enumiii})}
    \tightlist
    \item
      For any First Round Pick finishing his Rookie Scale Contract who was not selected with one of the first nine (9) picks in the Draft and who, (1) during the third and fourth Seasons of his Rookie Scale Contract, either started an average of forty-one (41) or more Regular Season games per Season or averaged two thousand (2,000) or more minutes of playing time per Regular Season, or (2) in the fourth Season of his Rookie Scale Contract either started forty-one (41) or more Regular Season games or played two thousand (2,000) or more minutes (collectively, the ``Starter Criteria''), the Qualifying Offer shall instead contain Base Compensation (with no bonuses of any kind) equal to the amount of the Qualifying Offer applicable to the ninth player selected in the first round of the Draft (the ``ninth player'') as called for by the Rookie Salary Scale applicable to the First Round Pick's Rookie Scale Contract. For purposes of calculating such Qualifying Offer amount, the Fourth Year Salary of the ninth player shall be deemed to equal one hundred twenty percent (120\%) of the Rookie Scale Amount applicable to the ninth player.
    \item
      For any First Round Pick finishing his Rookie Scale Contract who was selected with one of the first through fourteenth picks in the Draft and who failed to meet the Starter Criteria, the player's Qualifying Offer shall contain the lesser of: (x) the Salary (excluding Incentive Compensation), Likely Bonuses, and Unlikely Bonuses, respectively, provided in the Fourth Year Salary increased by the percentage called for in the ``Qualifying Offer: Percentage Increase Over 4th Year Salary'' column in the Rookie Salary Scale applicable to the First Round Pick's Rookie Scale Contract; or (y) Base Compensation (with no bonuses of any kind) equal to the amount of the Qualifying Offer applicable to the fifteenth player selected in the first round of the Draft (the ``fifteenth player'') as called for by the Rookie Salary Scale applicable to the First Round Pick's Rookie Scale Contract. For purposes of calculating such Qualifying Offer amount, the Fourth Year Salary of the fifteenth player shall be deemed to equal one hundred twenty percent (120\%) of the Rookie Scale Amount applicable to the fifteenth player.
    \end{enumerate}
  \item
    With respect to Qualifying Offers for players finishing Two-Way Contracts:

    \begin{enumerate}
    \def\labelenumiii{(\Alph{enumiii})}
    \tightlist
    \item
      For any player who (x) finished a Two-Way Contract with a term of two (2) Seasons in the current Salary Cap Year or (y) finished a Two-Way Contract with the same Team in each of the current Salary Cap Year and the immediately preceding Salary Cap Year (or if he finished a Two-Way Contract in the current Salary Cap Year with a Team that is different from the Team with which he finished a Two-Way Contract in the immediately preceding Salary Cap Year, he did so solely because he changed Teams during the current Salary Cap Year only by means of trade or an assignment via the NBA's waiver procedures), the Qualifying Offer shall be an offer of a Standard NBA Contract and shall provide for (i) Base Compensation in an amount equal to the Minimum Player Salary applicable to the player for the next Salary Cap Year (with no bonuses of any kind), and (ii) Base Compensation protection for lack of skill and injury or illness (with no individually-negotiated conditions or limitations on such protection and no other types of protection) in an amount equal to the ``Standard/Two-Way QO Protection Amount'' for the Season covered by the Qualifying Offer. The ``Standard/Two-Way QO Protection Amount'' shall equal (1) for a Qualifying Offer that covers the 2023-24 Season, \$90,000, and (2) for a Qualifying Offer that covers a subsequent Season, \$90,000 multiplied by a fraction, the numerator of which is the Salary Cap for the Salary Cap Year encompassing the applicable Season and the denominator of which is the Salary Cap for the 2023-24 Salary Cap Year.
    \item
      For all other players finishing Two-Way Contracts, the Qualifying Offer shall be an offer of a Two-Way Contract and shall provide for (i) the Two-Way Player Salary for the next Salary Cap Year, and (ii) Base Compensation protection for lack of skill and injury or illness (with no individually-negotiated conditions or limitations on such protection and no other types of protection) in an amount equal to the Maximum Two-Way Protection Amount for the next Salary Cap Year (``Two-Way Qualifying Offer'').
    \item
      Notwithstanding Sections (1)(e)(iii)(A) and (B) above, for any player finishing a Two-Way Contract who is not eligible to enter into another Two-Way Contract with the Team pursuant to Article II, Section 11(e), the Qualifying Offer (regardless of whether the prior Contract was a Two-Way Contract for a term of one (1) or two (2) Seasons) shall be the Qualifying Offer described in Section 1(e)(iii)(A) above.
    \end{enumerate}
  \item
    For all other players subject to a Right of First Refusal in accordance with this Article XI, the Salary (excluding Incentive Compensation), Likely Bonuses, and Unlikely Bonuses contained in a Qualifying Offer shall be one hundred thirty-five percent (135\%) (or, if the player's prior Contract was signed prior to the start of the 2023-24 Salary Cap Year, one hundred twenty-five percent (125\%)) of the player's Salary (excluding Incentive Compensation), Likely Bonuses, and Unlikely Bonuses, respectively, for the last Salary Cap Year covered by the player's prior Contract (the ``Prior Salary Qualifying Offer Amount''), provided that if on the July 1 immediately following the date on which such a Qualifying Offer was made, the sum of the Minimum Annual Salary applicable to the player (for the Season covered by the Qualifying Offer) plus \$200,000 (the ``Minimum-Plus Qualifying Offer Amount'') is greater than the Prior Salary Qualifying Offer Amount, then such a Qualifying Offer shall be deemed amended to provide for Base Compensation equal to the Minimum-Plus Qualifying Offer Amount (with no bonuses of any kind); provided, however, that, for any second round pick or undrafted player with two (2) or three (3) Years of Service who met the Starter Criteria in respect of the prior two (2) Seasons of his Contract(s) (i.e., who either averaged the games started or minutes played amounts described in Section 1(e)(ii)(A)(1) above during his prior two (2) Seasons, or achieved the games started or minutes played amounts described in Section 1(e)(ii)(A)(2) above in his prior Season only), the Qualifying Offer shall instead contain, if such amount exceeds the greater of the Prior Salary Qualifying Offer Amount or the Minimum-Plus Qualifying Offer Amount, Base Compensation equal to the amount of the Qualifying Offer applicable to the twenty-first player selected in the first round of the Draft (the ``twenty-first player'') as called for by the Rookie Salary Scale applicable to Rookie Scale Contracts finishing in the same Season as the last Season of the player's Contract. For purposes of calculating such Qualifying Offer amount, the Fourth Year Salary of the twenty-first player shall be deemed to equal one hundred percent (100\%) of the Rookie Scale Amount applicable to the twenty-first player.
  \item
    All other terms and conditions in a Qualifying Offer must be unchanged from those that applied to the last year of the player's prior Contract to the extent that such terms and conditions are allowable amendments under this Agreement at the time the Qualifying Offer is made. In addition, a Team shall be permitted to include in any Qualifying Offer an Exhibit 6 to the Uniform Player Contract requiring that the player, if he signs the Qualifying Offer, pass a physical examination to be performed by a physician designated by the Team as a condition precedent to the validity of the Contract. For purposes of the foregoing, the Starter Criteria shall be determined based upon Official NBA statistics.
  \end{enumerate}
\item
  No Team or any of its employees or agents shall make a public statement that the Team would match any future Offer Sheet for one of the Team's players or offer an impending or current Restricted Free Agent a particular Player Contract in free agency (e.g., a Contract providing for the player's maximum allowable Salary). The foregoing does not limit a Team's ability to express its desire to retain an impending or current Restricted Free Agent or to make general statements praising such a player (e.g., that the player is an important or essential part of the Team, that the Team wants or hopes to retain the player's services, and other similar statements).
\end{enumerate}

\hypertarget{no-individually-negotiated-right-of-first-refusal.}{%
\section{No Individually-Negotiated Right of First Refusal.}\label{no-individually-negotiated-right-of-first-refusal.}}

\begin{enumerate}
\def\labelenumi{(\alph{enumi})}
\tightlist
\item
  No Player Contract may include any individually-negotiated Right of First Refusal or other limitation on player movement following the last Salary Cap Year covered by such Player Contract.
\item
  No Right of First Refusal rule, practice, policy, regulation, or agreement providing for a Right of First Refusal shall be applied to any player as a result of that player's entry into a player contract with (or for otherwise playing with) any team in any professional basketball league other than the NBA.
\end{enumerate}

\hypertarget{withholding-services.}{%
\section{Withholding Services.}\label{withholding-services.}}

A player who withholds playing services called for by a Player Contract for more than thirty (30) days after the start of the last Season covered by his Player Contract shall be deemed not to have ``complet{[}ed{]} his Player Contract by rendering the playing services called for thereunder.'' Accordingly, such a player shall not be a Veteran Free Agent and shall not be entitled to negotiate or sign a Player Contract with any other professional basketball team unless and until the Team for which the player last played expressly agrees otherwise.

\hypertarget{qualifying-offers-to-make-certain-players-restricted-free-agents.}{%
\section{Qualifying Offers to Make Certain Players Restricted Free Agents.}\label{qualifying-offers-to-make-certain-players-restricted-free-agents.}}

\begin{enumerate}
\def\labelenumi{(\alph{enumi})}
\item
  \begin{enumerate}
  \def\labelenumii{(\roman{enumii})}
  \tightlist
  \item
    From the day following the Season covered by the second Option Year of a First Round Pick's Rookie Scale Contract through 5:00 p.m. eastern time on the immediately following June 29, the player's Team may make a Qualifying Offer to the player. If such a Qualifying Offer is made, then, on the July 1 following such Season, the player shall become a Restricted Free Agent, subject to a Right of First Refusal in favor of the Team (``ROFR Team''), as set forth in Section 5 below. If such a Qualifying Offer is not made, then the player shall become an Unrestricted Free Agent on such July 1. If a Team does not timely exercise its Option with respect to the first Option Year or second Option Year of a player's Rookie Scale Contract in accordance with Article VIII, the player shall, following his second or third Season (as the case may be) become an Unrestricted Free Agent.
  \item
    A Team that makes a Qualifying Offer to a player following the second Option Year of his Rookie Scale Contract may elect simultaneously to offer the player an alternative Contract covering five (5) Seasons that provides Salary for the first Salary Cap Year equal to the Maximum Annual Salary under Article II, Section 7(a), with annual increases in Salary equal to eight percent (8.0\%) of the Salary for the first Salary Cap Year (a ``Maximum Qualifying Offer''). Providing a player with a Maximum Qualifying Offer shall have the consequence described in Section 5(b) below. A Maximum Qualifying Offer shall be subject to the following:

    \begin{enumerate}
    \def\labelenumiii{(\Alph{enumiii})}
    \tightlist
    \item
      A Maximum Qualifying Offer shall contain only Base Compensation and no bonuses of any kind.
    \item
      A Maximum Qualifying Offer shall state that the player's Base Compensation for the first Season shall equal ``the Maximum Annual Salary applicable to the player in the first Season of the Contract,'' and that the Base Compensation in each of the four (4) subsequent Seasons shall ``be increased by eight percent (8.0\%) of the Base Compensation for the first Season.'' Such a Contract, if timely accepted by the player in accordance with Section 4(a)(ii)(D) below, shall be deemed amended to provide for specific Base Compensation for each Season covered by the Contract, based on the Maximum Annual Salary applicable to the player in the first Season.
    \item
      A Maximum Qualifying Offer cannot contain an Option or ETO, and must provide full Base Compensation protection in each Season for lack of skill and injury or illness (with no individually-negotiated conditions or limitations on such protection).
    \item
      The Team's offer of a Maximum Qualifying Offer must remain open for the same period that the player's Qualifying Offer remains open and cannot be withdrawn, except that if the Team withdraws its Qualifying Offer, the Maximum Qualifying Offer shall be deemed to be withdrawn simultaneously.
    \item
      A player may accept either his Qualifying Offer or his Maximum Qualifying Offer, but not both.
    \end{enumerate}
  \end{enumerate}
\item
  Any Veteran Free Agent (other than a First Round Pick whose first Option Year or second Option Year was not exercised) who (i) will have three (3) or fewer Years of Service as of the June 30 following the end of the last Season covered by his Player Contract, or (ii) is completing a Two-Way Contract will be a Restricted Free Agent if his Prior Team makes a Qualifying Offer to the player at any time from the day following such Season through 5:00 p.m. eastern time on the immediately following June 29. If such a Qualifying Offer is made, then, on the July 1 following the last Season covered by the player's Player Contract, the player shall become a Restricted Free Agent, subject to a Right of First Refusal in favor of the ROFR Team, as set forth in Section 5 below. If such a Qualifying Offer is not made, then the player shall become an Unrestricted Free Agent on such July 1.
\item
  \begin{enumerate}
  \def\labelenumii{(\roman{enumii})}
  \tightlist
  \item
    A player who receives a Qualifying Offer must be given until the October 1 following its issuance to accept it. Notwithstanding the preceding sentence, a Qualifying Offer may be withdrawn by the Team at any time through the July 13 following its issuance. If the Qualifying Offer is not withdrawn on or before July 13, it may be withdrawn thereafter but only if the player agrees in writing to the withdrawal. If a Qualifying Offer is withdrawn, the player shall immediately become an Unrestricted Free Agent. If a Qualifying Offer is withdrawn on or after July 14, the Team also shall be deemed to have renounced the player in accordance with Article VII, Section 4(g). A player may not accept a Qualifying Offer after the October 1 following the issuance thereof, unless the Team, prior to October 1, extends the date by which the player may accept the Qualifying Offer. In order to extend the date by which a player may accept his Qualifying Offer, a Team shall provide the player with written notice of the extension, which shall be either personally delivered to the player or his representative or sent by email or pre-paid certified, registered, or overnight mail to the last known address of the player or his representative. For clarity, there shall be no limit on the number of times a Team may extend the date by which a player may accept a Qualifying Offer. In no event may the acceptance date for a Qualifying Offer be extended beyond, or may a player accept a Qualifying Offer beyond, the March 1 following its issuance.
  \item
    If a Qualifying Offer is neither withdrawn nor accepted and the deadline for accepting it passes, the Team's Right of First Refusal shall continue, subject to Section 5(a) below.
  \item
    A player who knows that he has a medical disability that would render him unable to perform the playing services required under a Player Contract the following Season may not validly accept a Qualifying Offer received under this Section 4 or Section 5 below, unless the ROFR Team consents after disclosure of such medical disability. Notwithstanding the immediately preceding sentence, a player who knows that he has a medical disability that would render him unable to perform the playing services required under a Player Contract the following Season remains subject to the ROFR Team's Right of First Refusal.
  \end{enumerate}
\item
  Any claim that a Contract offered as a Qualifying Offer or a Maximum Qualifying Offer fails to meet one or more of the criteria for a Qualifying Offer or a Maximum Qualifying Offer shall be made by notice to the Team, in writing, no later than ten (10) days after a copy of the Qualifying Offer or Maximum Qualifying Offer was given by the Team or the NBA to the Players Association. Such notice must set forth the specific changes that allegedly must be made to the offered Contract in order for it to constitute a Qualifying Offer or a Maximum Qualifying Offer. Upon receipt of such notice, if the requested changes are necessary to satisfy the requirements of a Qualifying Offer or a Maximum Qualifying Offer, the Team may, within five (5) business days, offer the player an amended Contract incorporating the requested changes. If the Team offers such an amended Contract, the player and the Players Association shall be precluded from asserting that such Contract does not constitute a timely and valid Qualifying Offer or Maximum Qualifying Offer.
\end{enumerate}

\hypertarget{restricted-free-agency.}{%
\section{Restricted Free Agency.}\label{restricted-free-agency.}}

\begin{enumerate}
\def\labelenumi{(\alph{enumi})}
\item
  If a Restricted Free Agent does not sign an Offer Sheet with any Team by March 1 of the Season for which the Qualifying Offer is made, and does not sign a Player Contract with the ROFR Team before that Season ends, then his ROFR Team may reassert its Right of First Refusal for the following Season by extending another Qualifying Offer (with the same terms, including the amount of Salary (excluding Incentive Compensation), Likely Bonuses, and Unlikely Bonuses, respectively, that were included in the prior Qualifying Offer) by 5:00 p.m. eastern time on the next June 29. A ROFR Team may continue to reassert its Right of First Refusal by following the foregoing procedure in each subsequent year in which that Restricted Free Agent does not sign an Offer Sheet with any Team by March 1 of the Season for which the Qualifying Offer is made, and does not sign a Player Contract with the ROFR Team before that Season ends. In each Season in which a Team reasserts its Right of First Refusal by extending another Qualifying Offer in accordance with this Section 5(a), the Team may also elect to simultaneously provide the player with a Maximum Qualifying Offer (with the same terms that were included in the prior Maximum Qualifying Offer). Any such Qualifying Offer and Maximum Qualifying Offer shall be governed by the provisions of Section 4 above.
\item
  When a Restricted Free Agent receives an offer to sign a Player Contract from a Team other than the ROFR Team (the ``New Team''), which he desires to accept, he shall give to the ROFR Team a completed certificate substantially in the form of Exhibit G annexed hereto (the ``Offer Sheet''), signed by the Restricted Free Agent and the New Team, which shall have attached to it a Uniform Player Contract separately specifying: (i) the ``Principal Terms'' (as defined in Section 5(e) below) of the New Team's offer; and (ii) any non-Principal Terms of the New Team's offer that the ROFR Team is not required to match (as specified in Section 5(e) below) but which would be included in the player's Player Contract with the New Team if the ROFR Team does not exercise its Right of First Refusal. The player's obligation in the foregoing sentence to give to the ROFR Team a completed Offer Sheet shall be deemed satisfied if the Offer Sheet is given to the ROFR Team by the New Team. The Offer Sheet must be for a Player Contract with a term of more than one (1) Season (not including any Option Year), unless the ROFR team has tendered the player both a Qualifying Offer and a Maximum Qualifying Offer, in which case the Offer Sheet must be for a Player Contract with a term of more than two (2) Seasons (not including any Option Year). The Offer Sheet cannot be for a Two-Way Contract. In order to extend an Offer Sheet, the New Team must have Room for the player's Player Contract at the time the Offer Sheet is signed and must continue to have such Room at all times while the Offer Sheet is outstanding.
\item
  The ROFR Team, upon receipt of the Offer Sheet, may exercise its Right of First Refusal, which shall have the consequences hereinafter set forth below in this Section 5. The ROFR Team may match an Offer Sheet by using, as applicable, Room, a Veteran Free Agent Exception set forth in Article VII, Section 6(b), or the Minimum Player Salary Exception. In order to match an Offer Sheet, the ROFR Team must have, as applicable, Room, a Veteran Free Agent Exception, or Minimum Player Salary Exception in an amount equal or greater to the Salary plus any Unlikely Bonuses provided for in the first Salary Cap Year of the player's Contract at the time notice of the Team's exercise of its Right of First Refusal is given and must continue to have such Room or the applicable Exception at all times the First Refusal Exercise Notice remains in effect.
\item
  The following rules shall govern the signing of an Offer Sheet by a Restricted Free Agent who has one (1) or two (2) Years of Service:

  \begin{enumerate}
  \def\labelenumii{(\roman{enumii})}
  \tightlist
  \item
    Notwithstanding any other provision of this Agreement, no such Offer Sheet may provide for Salary plus Unlikely Bonuses in the first Salary Cap Year totaling more than the amount of the Non-Taxpayer Mid-Level Salary Exception for such Salary Cap Year. Annual increases or decreases in Salary and Unlikely Bonuses shall be governed by Article VII, Section 5(a)(1).
  \item
    If an Offer Sheet provides for the maximum allowable amount of Salary for the first two (2) Salary Cap Years pursuant to Section 5(d)(i) above, then, subject to Section 5(d)(iii) below, the Offer Sheet may provide for Salary for the third Salary Cap Year of up to the maximum amount that the player would have been eligible to receive for the third Salary Cap Year absent the restriction in the first sentence of Section 5(d)(i) above and had the player's Salary for the first two (2) Salary Cap Years been the maximum amount permitted under Article II, Section 7(a) and Article VII, Section 5(b)(1). If the Offer Sheet provides for Salary for the third Salary Cap Year in accordance with the foregoing sentence, then, subject to Section 5(d)(iii) below, (A) the player's Salary for the fourth Salary Cap Year may increase or decrease in relation to the third Salary Cap Year's Salary by no more than four and five tenths percent (4.5\%) of the Salary for the third Salary Cap Year, (B) the Offer Sheet cannot contain bonuses of any kind, and (C) the Offer Sheet must provide for one hundred percent (100\%) of the Base Compensation in each Season to be protected for lack of skill and injury or illness (with no individually-negotiated conditions or limitations on such protection).
  \item
    If a Team extends an Offer Sheet in accordance with Section 5(d)(ii) above, then, for purposes of determining whether the Team has Room for the Offer Sheet, the Salary for the first Salary Cap Year covered by the Offer Sheet shall be deemed to equal the average of the aggregate Salaries for such Salary Cap Year and each subsequent Salary Cap Year covered by the Offer Sheet. If the ROFR Team does not exercise its Right of First Refusal, the player's Salary for each Salary Cap Year covered by the Contract with the Team that extended the Offer Sheet shall be deemed to equal the average of the aggregate Salaries for each such Salary Cap Year. If the ROFR Team exercises its Right of First Refusal, the player's Salary for each Salary Cap Year covered by the Contract with the ROFR Team shall be the Salary for such Salary Cap Year as set forth in the Contract. Notwithstanding the preceding sentence, if the sum of (A) the ROFR Team's Team Salary at the time it exercises its Right of First Refusal, and (B) the average of the aggregate Salaries for each Salary Cap Year of the Offer Sheet, is less than or equal to the Salary Cap for the then-current Salary Cap Year, then the ROFR Team may, in connection with exercising its Right of First Refusal, elect to have the player's Salary for each Salary Cap Year covered by the Contract equal the average of such aggregate Salaries for each such Salary Cap Year. If the ROFR Team wishes to make such an election, it must do so by providing the NBA with a written statement on the same day that it gives the First Refusal Exercise Notice to the Restricted Free Agent and New Team pursuant to Section 5(t) below, and the NBA shall provide a copy of this notice to the Players Association within one (1) business day following its receipt thereof.
  \end{enumerate}
\item
  The Principal Terms of an Offer Sheet are only:

  \begin{enumerate}
  \def\labelenumii{(\roman{enumii})}
  \tightlist
  \item
    the term of the Contract;
  \item
    the fixed and specified Compensation that the New Team will pay or lend to the Restricted Free Agent as a signing bonus, Current Base Compensation, and/or Deferred Base Compensation in specified installments on specified dates;
  \item
    Incentive Compensation; provided, however, that the only elements of such Incentive Compensation that shall be included in the Principal Terms are the following: (A) bonuses that qualify as Likely Bonuses based upon the performance of the Team extending the Offer Sheet and the ROFR Team; and (B) Generally Recognized League Honors; and
  \item
    Any allowable amendments to the terms contained in the Uniform Player Contract (e.g., Base Compensation protection, a trade bonus, etc.).
  \end{enumerate}
\item
  In the event that an Offer Sheet includes an Exhibit 6 requiring that the player pass a physical examination to be performed by a physician designated by the New Team, the Exhibit 6 language must be replaced with the following: ``This Offer Sheet will be deemed invalid and of no force and effect (except as described in Article XI, Section 5(l) of the CBA) unless the player passes, in the sole discretion of the Team (exercised in good faith), a physical examination in accordance with Article II, Section 13(h) of the CBA that is (i) conducted within two (2) days of the execution of this Offer Sheet, and (ii) the results of which are reported by the Team to the player within three (3) days of the execution of this Offer Sheet. The player agrees to supply complete and truthful information in connection with any such examinations.'' The New Team must notify the player and the ROFR Team within the three (3)-day period set forth in the Exhibit 6 in the Offer Sheet whether the player has passed the physical. In the event that the New Team fails to timely provide such notice, the player shall be deemed to have passed the physical with the New Team.
\item
  If the ROFR Team gives to the Restricted Free Agent a ``First Refusal Exercise Notice'' substantially in the form of Exhibit H annexed hereto (i) for an Offer Sheet received by the ROFR Team prior to 12:00 p.m. eastern time on a day, by 11:59 p.m. eastern time on the day immediately following such day, or (ii) for an Offer Sheet received by the ROFR Team on or after 12:00 p.m. eastern time on a day, by 11:59 p.m. eastern time on the day that is two (2) days following such day, then, subject to Section 5(k) below, such Restricted Free Agent and the ROFR Team shall be deemed to have entered into a Player Contract containing all the Principal Terms (but not any terms other than the Principal Terms) included in the Uniform Player Contract attached to the Offer Sheet (except that if the Contract contains an Exhibit 6, such Exhibit 6 shall be deemed deleted). Such Contract may not thereafter be amended in any manner for a period of one (1) year.
\item
  If the ROFR Team does not give the First Refusal Exercise Notice within the applicable period specified in Section 5(g) above, or if during such period the ROFR Team provides written notice to the player that the Team declines to exercise its Right of First Refusal, then the player and the New Team shall be deemed to have entered into a Player Contract containing all of the terms and conditions included in the Uniform Player Contract attached to the Offer Sheet (including, if the Offer Sheet contains an Exhibit 6, that the player pass a physical examination to be conducted by the Team as a condition precedent to the validity of the Contract). Such Contract may not thereafter be amended in any manner for a period of one (1) year.
\item
  Notwithstanding anything contained herein to the contrary, for any Offer Sheet received by the ROFR Team during the Moratorium Period, the ROFR Team shall have until 11:59 p.m. eastern time on the July 7 immediately following such Moratorium Period to give the First Refusal Exercise Notice.
\item
  After exercising its Right of First Refusal as described in this Section 5, the ROFR Team may not trade the Restricted Free Agent for one (1) year, without the player's consent. Even with the player's consent, for one (1) year, neither the ROFR Team exercising its Right of First Refusal nor any other Team may trade the player to the Team whose Offer Sheet was matched.
\item
  \begin{enumerate}
  \def\labelenumii{(\roman{enumii})}
  \tightlist
  \item
    Any Team may condition its First Refusal Exercise Notice on the player reporting for and passing, in the sole discretion of the Team (exercised in good faith), a physical examination to be conducted by a physician designated by the Team within two (2) days from its exercise of the Right of First Refusal. In connection with the physical examination, the player must supply all information reasonably requested of him, provide complete and truthful answers to all questions posed to him, and submit to all examinations and tests requested of him.
  \item
    If the player does not submit to the requested physical examination within two (2) days of the exercise of the Right of First Refusal, then the First Refusal Exercise Notice shall be deemed no longer to be conditioned on the player reporting for and passing a physical examination and the following rules shall apply:

    \begin{enumerate}
    \def\labelenumiii{(\arabic{enumiii})}
    \tightlist
    \item
      In its sole discretion, the ROFR Team shall have two (2) days following the conclusion of the two (2) day period referenced in Section 5(k)(i) above to withdraw its First Refusal Exercise Notice. If the First Refusal Exercise Notice is timely withdrawn in accordance with the foregoing, such withdrawal shall have the following effects:

      \begin{enumerate}
      \def\labelenumiv{(\Alph{enumiv})}
      \item
        The Offer Sheet shall be deemed invalid and the Team that issued the Offer Sheet shall be prohibited from signing or acquiring the player for a period of one (1) year from the date the First Refusal Exercise Notice was withdrawn; provided, however, that if the ROFR Team subsequently relinquishes its Right of First Refusal (resulting in the player becoming an Unrestricted Free Agent) in accordance with Section 5(o) below, then the aforementioned one-year prohibition shall be deemed to expire on the last day of the Salary Cap Year in which the invalidated Offer Sheet was signed by the player and the New Team;
      \item
        The player shall be prohibited from entering into an Offer Sheet with any other Team prior to the first day of the Salary Cap Year immediately following the Salary Cap Year in which the invalidated Offer Sheet was signed by the player and the New Team;
      \item
        If the deadline by which the player was given to accept the Qualifying Offer (and, if applicable, Maximum Qualifying Offer) provided to him by the ROFR Team has not already expired, then such deadline shall be deemed to have expired (which shall have the effect of the player no longer being able to accept the Qualifying Offer (or, if applicable, Maximum Qualifying Offer) given to him by the ROFR Team); and
      \item
        \begin{enumerate}
        \def\labelenumv{(\roman{enumv})}
        \tightlist
        \item
          The player shall be prohibited from entering into any Player Contract with the ROFR Team other than a Player Contract containing all of the Principal Terms (but not any terms other than the Principal Terms) included in the Uniform Player Contract attached to the invalidated Offer Sheet (except that such Player Contract may contain an Exhibit 6) before the later of (A) three (3) months following the date the ROFR Team withdrew its First Refusal Exercise Notice, and (B) the January 15 of the Salary Cap Year in which the invalidated Offer Sheet was signed by the player and the New Team.
        \item
          In circumstances in which the invalidated Offer Sheet was signed pursuant to the rules set forth in Section 5(d)(ii) above (governing the signing of certain Offer Sheets by Restricted Free Agents who have one (1) or two (2) Years of Service) and the player and the ROFR Team sign a subsequent Player Contract during the time period referenced in Section 5(k)(ii)(1)(D)(i) above:

          \begin{enumerate}
          \def\labelenumvi{(\Alph{enumvi})}
          \tightlist
          \item
            solely for purposes of determining whether the ROFR Team is eligible to sign the subsequent Player Contract with the Early Qualifying Veteran Free Agent Exception pursuant to Article VII, Section 6(b)(3)(ii) or the Non-Taxpayer Mid-Level Salary Exception pursuant to Article VII, Section 6(e)(5), the signing of such Player Contract shall be regarded as the matching of an Offer Sheet provided in accordance with Article XI, Section 5(d)(ii); and
          \item
            if, at the time the ROFR Team matched the invalidated Offer Sheet, the ROFR Team made an election pursuant to Section 5(d)(iii) above to have the player's Salary for each Salary Cap Year covered by the Contract equal the average of the aggregate Salaries for each Salary Cap Year covered by the Contract, then the ROFR Team shall be permitted to have the player's Salary for each Salary Cap Year covered by the subsequent Contract equal the average of such aggregate Salaries for each such Salary Cap Year covered by the subsequent Contract, provided that the sum of: (i) the ROFR Team's Team Salary at the time the subsequent Contract is signed; and (ii) the average of the aggregate Salaries for each Salary Cap Year of the subsequent Contract is less than or equal to the Salary Cap for the then-current Salary Cap Year. If the ROFR Team wishes to make such an election, it must do so by providing the NBA with a written statement on the same day that it signs the subsequent Contract. The NBA shall provide a copy of any such notice to the Players Association within one (1) business day following its receipt thereof.
          \end{enumerate}
        \end{enumerate}
      \end{enumerate}
    \item
      If the First Refusal Exercise Notice is not withdrawn by the ROFR Team within the applicable two (2) day period, then the player and the ROFR Team shall be deemed to have entered into a Player Contract in accordance with the provisions of Section 5(g) above.
    \end{enumerate}
  \item
    If the player submits to the requested physical examination within two (2) days of the exercise of the Right of First Refusal but does not pass such physical examination, then in its sole discretion, the ROFR Team may withdraw its First Refusal Exercise Notice within two (2) days following the date upon which such physical examination is conducted. If the First Refusal Exercise Notice is withdrawn, the player and the New Team shall be deemed to have entered into a Player Contract in accordance with the provisions of Section 5(h) above. If the First Refusal Exercise Notice is not withdrawn, then the ROFR Team shall be deemed to have waived its right to have the player pass a physical examination and the player and ROFR Team will be deemed to have entered into a Player Contract in accordance with the provisions of Section 5(g) above.
  \end{enumerate}
\item
  In the event that (1) the Offer Sheet includes an Exhibit 6 and the New Team determines that the player has not passed the physical, and (2) either (A) the ROFR Team declines to exercise its Right of First Refusal, (B) the period for the ROFR Team to exercise its Right of First Refusal expires, or (C) the ROFR Team exercises its Right of First Refusal conditioned on the player reporting for and passing a physical and timely determines that the player has not passed his physical and withdraws its First Refusal Exercise Notice pursuant to Section 5(k) above, then the ROFR Team must within two (2) days from the later of the date (x) that the ROFR Team receives notice from the New Team that the player has not passed the physical examination administered by the New Team, or (y) on which the ROFR Team timely notifies the player that he has not passed the physical, pursuant to Section 5(k) above:

  \begin{enumerate}
  \def\labelenumii{(\roman{enumii})}
  \tightlist
  \item
    Elect to continue to possess such rights with respect to the player as the ROFR Team possessed at the time of the execution of the Offer Sheet, provided that the ROFR Team can only make this election if the ROFR Team has not engaged in any transaction since the Offer Sheet was given that the ROFR Team would not have been able to engage in if the player's Free Agent Amount (or the amount of a Qualifying Offer or Maximum Qualifying Offer made to the player, if applicable) at the time the Offer Sheet was given had remained included in the ROFR Team's Team Salary; or
  \item
    Decline to continue to possess such rights with respect to the player as the ROFR Team possessed at the time of the execution of the Offer Sheet, in which case any Qualifying Offer given to the player by the Team shall be deemed withdrawn pursuant to Section 4(c)(i) above, and the Team's Right of First Refusal shall be deemed relinquished pursuant to Section 5(o) below.
  \end{enumerate}

  If at the time the New Team notifies the ROFR Team that the player has not passed the physical administered by the New Team, the ROFR Team has not yet exercised its Right of First Refusal or has not yet provided written notice to the player that the ROFR Team declines to exercise its Right of First Refusal, nothing in this Section 5(l) shall prohibit a ROFR Team from: (a) exercising its Right of First Refusal; or (b) making one of the elections set forth in Section 5(l)(i) or 5(l)(ii) above.
\item
  A Team shall not be permitted to exercise its Right of First Refusal pursuant to an agreement to trade the Player Contract to another Team pursuant to Article VII, Section 8(e).
\item
  There may be only one (1) Offer Sheet signed by a Restricted Free Agent outstanding at any one time, provided that the Offer Sheet has also been signed by a Team. An Offer Sheet, both before and after it is given to the ROFR Team, may be revoked or withdrawn only upon the written consent of the ROFR Team, the New Team, and the Restricted Free Agent. In such event, a Restricted Free Agent shall again be free to negotiate and sign an Offer Sheet with any Team, and any Team shall again be free to negotiate and sign an Offer Sheet with such Restricted Free Agent, subject only to the ROFR Team's renewed Right of First Refusal.
\item
  A Team that holds the Right of First Refusal with respect to a Restricted Free Agent may relinquish such Right of First Refusal at any time except during the period that the player has been given to accept a Qualifying Offer. If a Team relinquishes its Right of First Refusal with respect to a Restricted Free Agent, the player shall immediately become an Unrestricted Free Agent and the Team shall be deemed to have renounced the player in accordance with Article VII, Section 4(g) hereof. In order to relinquish its Right of First Refusal with respect to a Restricted Free Agent, a Team shall provide the NBA with a written statement relinquishing such Right of First Refusal. The NBA shall provide a copy of such statement to the Players Association by email within two (2) business days following its receipt thereof.
\item
  An expedited arbitration before the System Arbitrator, whose decision shall be final and binding upon all parties, shall be the exclusive method for resolving any disputes concerning this Section 5. If a dispute arises between the player and either the ROFR Team or the New Team, as the case may be, relating to the contents of an Offer Sheet, and/or whether the binding agreement is between the Restricted Free Agent and the New Team or the Restricted Free Agent and the ROFR Team, such dispute shall immediately be submitted to the System Arbitrator, who shall resolve such dispute within five (5) days.
\item
  A Restricted Free Agent may not give an Offer Sheet to the ROFR Team at any time after the March 1 of the Season for which he has been made a Qualifying Offer.
\item
  On the same day as the giving of an Offer Sheet to the ROFR Team, the ROFR Team shall cause a copy thereof to be given to the NBA, which shall cause a copy thereof to be promptly given to the Players Association. On the same day as the giving of a First Refusal Exercise Notice to the Restricted Free Agent, the ROFR Team shall cause a copy thereof to be given to the New Team, which shall cause a copy thereof to be promptly given to the NBA, which shall cause a copy thereof to be promptly given to the Players Association.
\item
  There may be no consideration of any kind given by one Team to another Team in exchange for a Team's decision to exercise or not to exercise its Right of First Refusal, or in exchange for a Team's decision to submit or not to submit an Offer Sheet to a Restricted Free Agent.
\item
  Any Offer Sheet, First Refusal Exercise Notice, or other writing required or permitted to be given under this Article XI, shall be either by personal delivery, email, or by pre-paid certified, registered, or overnight mail addressed as follows:

  To any Team: addressed to the Team at the principal address of such Team as then listed on the records of the NBA or at the Team's principal office, to the attention of the Team's general manager (and if by email, then to the general manager's email address with the Team and any such other email address as the Team may designate in writing);

  To the NBA: National Basketball Association, Olympic Tower, 645 Fifth Avenue, New York, NY 10022, Attn: General Counsel (and if by email, then to the General Counsel's email address with the NBA and any such other email address as the NBA may designate in writing);

  To the Players Association: National Basketball Players Association, 1133 Avenue of the Americas, 5th Floor, New York, NY 10036, Attn: General Counsel (and if by email, then to the General Counsel's email address with the NBPA and any such other email address as the NBPA may designate in writing);

  To a Restricted Free Agent: (i) for Qualifying Offers and other writings relating to Qualifying Offers (e.g., withdrawal of a Qualifying Offer), to the last known email address or address of the player or his representative; and (ii) for Offer Sheets and other writings relating to Offer Sheets (e.g., First Refusal Exercise Notice), to his address listed on the Offer Sheet, and, if the Restricted Free Agent designates a representative on the Offer Sheet and lists such representative's address thereon, then such representative's address (and if by email to the player or his representative, then with a copy to the General Counsel's email address with the NBPA and any such other email address as the NBPA may designate in writing).
\item
  Notwithstanding anything contained herein to the contrary:

  \begin{enumerate}
  \def\labelenumii{(\roman{enumii})}
  \tightlist
  \item
    In addition to personal delivery or delivery by pre-paid certified, registered, or overnight mail, any Offer Sheet, notice revoking or withdrawing an Offer Sheet, First Refusal Exercise Notice, notice declining to exercise a Right of First Refusal, notice relinquishing a Right of First Refusal, or notice withdrawing a First Refusal Exercise Notice (collectively ``Offer Sheet-Related Notices'') may be given by email as follows:

    \begin{enumerate}
    \def\labelenumiii{(\arabic{enumiii})}
    \tightlist
    \item
      To any Team: to the attention of each of the Team's specified representatives' email address (as set forth in subsection (iii) below).
    \item
      To the NBA: to the attention of the email address used for that purpose under the 2017 CBA or such other email address as agreed to by the parties.
    \item
      To the Players Association: to the attention of the email address used for that purpose under the 2017 CBA or such other email address as agreed by the parties.
    \item
      To a Restricted Free Agent: to his email address listed on the Offer Sheet, and, if the Restricted Free Agent designates a representative on the Offer Sheet and lists such representative's email address thereon, a copy shall be sent to such representative at such email address.
    \end{enumerate}
  \item
    Any Offer Sheet-Related Notice given by email must be sent to the NBA, the Players Association, the applicable Restricted Free Agent (including such Restricted Free Agent's representative if required pursuant to Section 5(t) above), the ROFR Team, and the New Team. If an Offer Sheet fails to list a player's email address, delivery of any Offer Sheet-Related Notice to the player shall be deemed satisfied by email delivery to the Players Association.
  \item
    By the June 10 prior to each Salary Cap Year, each Team shall provide to the NBA the names and email addresses of three (3) representatives designated by the Team who shall be, for such Salary Cap Year, (i) the only representatives of the Team permitted to give any Offer Sheet-Related Notice on behalf of the Team via the email notification procedures set forth herein, and (ii) the required recipients of any Offer Sheet-Related Notice sent to the Team via the email notification procedures set forth herein. In each Salary Cap Year, the NBA shall provide to the Players Association (and all Teams) the list of Team representatives (and such representatives' email addresses) by June 15.
  \end{enumerate}

  Any Offer Sheet, First Refusal Exercise Notice, or other writing required or permitted to be given under this Article XI that is sent by email shall be deemed given when sent. For delivery by any other means allowed by this Article XI, the following shall apply: (i) an Offer Sheet shall be deemed given only when received by the ROFR Team; (ii) a First Refusal Exercise Notice shall be deemed given when sent by the ROFR Team; (iii) a Qualifying Offer, a Maximum Qualifying Offer, an amended Qualifying Offer (i.e., pursuant to Section 4(d) above), and a notice of extension of the date by which a Qualifying Offer can be accepted shall be deemed given when sent by the ROFR Team; and (iv) other writings required or permitted to be given under this Article XI (e.g., notice relinquishing a Right of First Refusal, an acceptance of a Qualifying Offer, a withdrawal of a Qualifying Offer, notice that a Qualifying Offer fails to meet one or more of the criteria for a Qualifying Offer, etc.) shall be deemed given only when received by the party to whom it is addressed.
\end{enumerate}

\hypertarget{option-clauses}{%
\chapter{OPTION CLAUSES}\label{option-clauses}}

\hypertarget{team-options.}{%
\section{Team Options.}\label{team-options.}}

Except as provided by Article VIII, Section 1, a Player Contract shall not contain any option in favor of the Team, except an Option (as defined in Article I, Section 1(ss)) that: (i) is specifically negotiated between a Veteran and a Team or (except in the case of a Rookie Scale Contract) a Rookie and a Team; (ii) authorizes the extension of such Contract for no more than one (1) year beyond the stated term; (iii) is exercisable only once; (iv) provides that the Salary (excluding Incentive Compensation), Likely Bonuses, and Unlikely Bonuses payable with respect to the Option Year are no less than one hundred percent (100\%) of the Salary (excluding Incentive Compensation), Likely Bonuses, and Unlikely Bonuses, respectively, payable with respect to the last year of the stated term of such Contract; and (v) except in the case of a Contract signed pursuant to the Second Round Pick Exception, provides that all other terms and conditions (other than with respect to the payment schedule for the player's Base Compensation) in the Option Year shall be unchanged from those that applied to the last year of the stated term of such Contract (including, but not limited to, the percentage of Base Compensation that is protected).

\hypertarget{player-options.}{%
\section{Player Options.}\label{player-options.}}

A Player Contract shall not contain any option in favor of the player, except:

\begin{enumerate}
\def\labelenumi{(\alph{enumi})}
\tightlist
\item
  an Option that: (i) is specifically negotiated between a Veteran and a Team or (except in the case of a Rookie Scale Contract) a Rookie and a Team; (ii) authorizes the extension of such Contract for no more than one (1) year beyond the stated term; (iii) is exercisable only once; (iv) provides that the Salary (excluding Incentive Compensation), Likely Bonuses, and Unlikely Bonuses payable with respect to the Option Year are no less than one hundred percent (100\%) of the Salary (excluding Incentive Compensation), Likely Bonuses, and Unlikely Bonuses, respectively, payable with respect to the last year of the stated term of such Contract; and (v) that all other terms and conditions (other than with respect to the payment schedule for the player's Base Compensation) in the Option Year shall be unchanged from those that applied to the last year of the stated term of such Contract (including, but not limited to, the percentage of Base Compensation that is protected). If a Player Contract contains an Option in favor of the player and provides, in whole or in part, for Base Compensation protection in the Option Year, such Contract must also contain, in Exhibit 2 of the Contract under the heading ``Additional Conditions or Limitations,'' either the language set forth in subsection (A) below or the language set forth in subsection (B) below, but not both, and such language shall define the respective rights and obligations of the player and Team with respect to the subject matter thereof:

  \begin{enumerate}
  \def\labelenumii{(\Alph{enumii})}
  \tightlist
  \item
    ``If this Contract is terminated by Team prior to Player's exercise of the Option described in Exhibit 1 of the Contract, then Player shall be entitled to benefit from the Base Compensation protection provisions of this Exhibit 2 to the same extent as if the exercise of the Option by Player had occurred prior to Team's termination of the Contract.''
  \item
    ``If this Contract is terminated by Team prior to Player's exercise of the Option described in Exhibit 1 of the Contract, then Team shall be relieved of any obligation to pay Player any Base Compensation with respect to the Option Year.''
  \end{enumerate}

  No Player Contract that contains the language set forth in subsection (B) above may provide for the Option in favor of the player to be exercisable earlier than the day following the date of the Team's last game of the Season prior to the Option Year; and/or
\item
  an Early Termination Option (or ``ETO'') (as defined in Article I, Section 1(u)), provided that such ETO is exercisable only once and takes effect no earlier than the end of the fourth Season of the Contract. A Contract that does not provide for an ETO when signed may not be amended to provide for an ETO during the original term of the Contract. If a Team and a player enter into an Extension (other than an Extension of a Rookie Scale Contract), the Contract may not be amended to provide for an ETO and any previously-existing ETO must be eliminated. If a Team and player enter into an Extension of a Rookie Scale Contract, the Contract may simultaneously be amended to provide for an ETO, provided that such ETO is exercisable only once and takes effect no earlier than the end of the fourth Season of the extended term of the Contract.
\end{enumerate}

\hypertarget{no-conditional-options.}{%
\section{No Conditional Options.}\label{no-conditional-options.}}

If a Contract contains any Option or ETO, the right of the Team or player to exercise such Option or ETO must be fixed at the time the Contract (or Extension) is entered into and may not be contingent upon the satisfaction of any individually-negotiated condition. In the case of an ETO, the Effective Season of such ETO also must be fixed at the time the Contract (or Extension) is entered into and may not be contingent upon the satisfaction of any individually-negotiated condition.

\hypertarget{exercise-period.}{%
\section{Exercise Period.}\label{exercise-period.}}

Any ETO must be exercised by 5:00 p.m. eastern time on the June 29 immediately prior to the Effective Season of such ETO. Any Option must be exercised by 5:00 p.m. eastern time on the June 29 immediately prior to the Season covered by the Option, except that an Option in favor of a player who would become a Restricted Free Agent if the Option were not exercised must be exercised prior to the June 25 immediately prior to the Season covered by such Option.

\hypertarget{option-exercise-notices.}{%
\section{Option Exercise Notices.}\label{option-exercise-notices.}}

The NBA shall provide the Players Association with copies of any Option or ETO exercise or non-exercise notice received by the NBA within two (2) business days of the NBA's receipt of such notice from the Team.

\hypertarget{circumvention}{%
\chapter{CIRCUMVENTION}\label{circumvention}}

\hypertarget{general-prohibitions.}{%
\section{General Prohibitions.}\label{general-prohibitions.}}

\begin{enumerate}
\def\labelenumi{(\alph{enumi})}
\tightlist
\item
  It is the intention of the parties that the provisions agreed to herein, including, without limitation, those relating to the Salary Cap, the Exceptions to the Salary Cap, the scope of Basketball Related Income, the Escrow and Tax Arrangement, the Rookie Scale, the Right of First Refusal, the Maximum Player Salary, and free agency, be interpreted so as to preserve the essential benefits achieved by both parties to this Agreement. Neither the Players Association, the NBA, nor any Team (or Team Affiliate) or player (or person or entity acting with authority on behalf of such player), shall enter into any agreement, including, without limitation, any Player Contract (including any Renegotiation, Extension, or amendment of a Player Contract), or undertake any action or transaction, including, without limitation, the assignment or termination of a Player Contract, which is, or which includes any term that is, designed to serve the purpose of defeating or circumventing the intention of the parties as reflected by all of the provisions of this Agreement.
\item
  It shall constitute a violation of Section 1(a) above for a Team (or Team Affiliate) to enter into an agreement or understanding with any sponsor or business partner or third party under which such sponsor, business partner, or third party pays or agrees to pay compensation for basketball services (even if such compensation is ostensibly designated as being for non-basketball services) to a player under Contract to the Team. Such an agreement with a sponsor or business partner or third party may be inferred where: (i) such compensation from the sponsor or business partner or third party is substantially in excess of the fair market value of any services to be rendered by the player for such sponsor or business partner or third party; and (ii) the Compensation in the Player Contract between the player and the Team is substantially below the fair market value of such Contract.
\item
  It shall constitute a violation of Section 1(a) above for a Team (or Team Affiliate) to have a financial arrangement with or offer a financial inducement to any player (not including retired players) not signed to a current Player Contract, except as permitted by this Agreement.
\item
  Nothing contained in Section 1(c) above shall interfere with a Team's obligation to pay a player Deferred Compensation earned under a prior Player Contract.
\end{enumerate}

\hypertarget{no-unauthorized-agreements.}{%
\section{No Unauthorized Agreements.}\label{no-unauthorized-agreements.}}

\begin{enumerate}
\def\labelenumi{(\alph{enumi})}
\tightlist
\item
  At no time shall there be any agreements or transactions of any kind (whether disclosed or undisclosed to the NBA), express or implied, oral or written, or promises, undertakings, representations, commitments, inducements, assurances of intent, or understandings of any kind (whether disclosed or undisclosed to the NBA), between a player (or any person or entity controlled by, related to, or acting with authority on behalf of, such player) and any Team (or Team Affiliate):

  \begin{enumerate}
  \def\labelenumii{(\roman{enumii})}
  \tightlist
  \item
    concerning any future Renegotiation, Extension, or other amendment of an existing Player Contract, or entry into a new Player Contract; or
  \item
    except as permitted by this Agreement or as set forth in a Uniform Player Contract (provided that the Team has not intentionally delayed submitting such Uniform Player Contract for approval by the NBA), involving compensation or consideration of any kind or anything else of value, to be paid, furnished, or made available by, to, or for the benefit of the player, or any person or entity controlled by, related to, or acting with authority on behalf of the player; or
  \item
    except as permitted by this Agreement, involving an investment or business opportunity to be furnished or made available by, to, or for the benefit of the player, or any person or entity controlled by, related to, or acting with authority on behalf of the player.
  \end{enumerate}
\item
  In addition to the foregoing, it shall be a violation of this Section 2 for any Team (or Team Affiliate) or any player (or any person or entity controlled by, related to, or acting with authority on behalf of, such player) to attempt to enter into or to intentionally solicit any agreement, transaction, promise, undertaking, representation, commitment, inducement, assurance of intent, or understanding that would be prohibited by Section 2(a) above.
\item
  Notwithstanding the foregoing, it shall not be a violation of Section 2(a) or 2(b) above solely for a Team Affiliate and a player (or any person or entity controlled by, related to, or acting with authority on behalf of, such player) to each passively invest (i.e., invest with no management, governance, voting, or executive role or other operational rights or role) in the same third-party entity, provided that (i) neither such Team Affiliate or such player holds more than a twelve and one-half percent (12.5\%) interest in such third-party entity, (ii) the Team Affiliate's investment and player's investment are not made in coordination or in consultation with each other, and (iii) the investment opportunity was not furnished or made available to the player by the Team Affiliate (or vice versa).
\item
  A violation of Section 2(a) or 2(b) above may be proven by direct or circumstantial evidence, including, but not limited to, evidence that a Player Contract or any term or provision thereof cannot rationally be explained in the absence of conduct violative of Section 2(a) or 2(b).
\item
  In any proceeding brought before the System Arbitrator pursuant to this Section 2, no adverse inference shall be drawn against the party initiating such proceeding because that party, when it first suspected or believed that a violation of Section 2 may have occurred, deferred the initiation of such proceeding until it had further reason to believe that such a violation had occurred.
\item
  A player will not be found to have committed a violation of Section 2(a)(ii) above if the violation is the Team's intentional delay in submitting a Uniform Player Contract to the NBA and this was done without the player's knowledge.
\end{enumerate}

\hypertarget{penalties.}{%
\section{Penalties.}\label{penalties.}}

\begin{enumerate}
\def\labelenumi{(\alph{enumi})}
\tightlist
\item
  Upon a finding of a violation of Section 1 above by the System Arbitrator, but only following the conclusion of any appeal to the Appeals Panel, the Commissioner shall be authorized to:

  \begin{enumerate}
  \def\labelenumii{(\roman{enumii})}
  \tightlist
  \item
    impose a fine of up to \$4,500,000 (fifty percent (50\%) of which shall be payable to the NBA, and fifty percent (50\%) of which shall be payable to the NBPA-Selected Charitable Organization (as defined in Article VI, Section 6(a))) on any Team found to have committed such violation for the first time;
  \item
    impose a fine of up to \$5,500,000 (fifty percent (50\%) of which shall be payable to the NBA, and fifty percent (50\%) of which shall be payable to the NBPA-Selected Charitable Organization) on any Team found to have committed such violation for at least the second time;
  \item
    direct the forfeiture of one First Round Draft Pick;
  \item
    void any Player Contract, or any Renegotiation, Extension, or amendment of a Player Contract, between any player and any Team when both the player (or any person or entity acting with authority on behalf of such player) and the Team (or Team Affiliate) are found to have committed such violation; and/or
  \item
    void any other transaction or agreement found to have violated Section 1 above.
  \end{enumerate}
\item
  Upon a finding of a violation of Section 2 above by the System Arbitrator, but only following the conclusion of any appeal to the Appeals Panel, the Commissioner shall be authorized to:

  \begin{enumerate}
  \def\labelenumii{(\roman{enumii})}
  \tightlist
  \item
    impose a fine of up to \$7,500,000 on any Team found to have committed such violation (fifty percent (50\%) of which shall be payable to the NBA, and fifty percent (50\%) of which shall be payable to the NBPA-Selected Charitable Organization);
  \item
    direct the forfeiture of draft picks;
  \item
    when both the player (or any person or entity acting with authority on behalf of such player) and the Team (or Team Affiliate) are found to have committed such violation, (A) void any Player Contract, or any Renegotiation, Extension, or amendment of a Player Contract, between such player and such Team, (B) impose a fine of up to \$350,000, on any player (fifty percent (50\%) of which shall be payable to the NBA, and fifty percent (50\%) of which shall be payable to the NBPA-Selected Charitable Organization), and/or (C) prohibit any future Player Contract, or any Renegotiation, Extension, or amendment of a Player Contract, between such player and such Team;
  \item
    suspend for up to one (1) year any Team personnel found to have willfully engaged in such violation; and/or
  \item
    void any transaction or agreement found to have violated Section 2 above and direct the disgorgement by the player of anything of value received in connection with such transaction or agreement (except Compensation received for services already performed pursuant to a Player Contract), unless the player establishes by a preponderance of the evidence that he was unaware of the violation.
  \end{enumerate}
\item
  In any proceeding before the System Arbitrator in which it is alleged that a player agent or other person or entity acting with authority on behalf of a player has violated Section 2 above, the System Arbitrator shall make a specific determination with respect to such allegation. If the System Arbitrator finds such violation and such finding, if appealed, is affirmed by the Appeals Panel, the System Arbitrator shall refer such finding to the Players Association, which shall accept as binding and conclusive the finding(s) of the System Arbitrator (or, in the case of an appeal, the Appeals Panel) that a violation of Section 2(a) or 2(b) has occurred and shall consider such finding(s) as establishing a violation of the Players Association's regulations applicable to such person or entity. The Players Association represents that it will impose such discipline as is appropriate under the circumstances on the person or entity found to have violated Section 2 above, and that it will promptly notify the NBA of the discipline imposed; provided, however, that in no event shall the penalty imposed upon a player agent found to have violated Section 2 above be less than a one-year suspension of that player agent's certification by the Players Association.
\item
  In addition to the authority conferred on the Commissioner pursuant to Sections 3(a) and 3(b) above, the Commissioner shall be authorized to impose a fine of up to \$1,000,000 on any Team or Team personnel found by the Commissioner to have violated Section 2 above. Any fine imposed pursuant to this Section 3(d) shall not require as a predicate any finding of, or proceeding before, the System Arbitrator. In the event the Commissioner imposes such a fine, the Players Association has the right to de novo review of the Commissioner's finding that a Section 2 violation occurred under the System Arbitration provisions of Article XXXII. With respect to any fine imposed under this Section 3(d), fifty percent (50\%) shall be payable to the NBA and fifty percent (50\%) shall be payable to the NBPA-Selected Charitable Organization (as defined in Article VI, Section 6(a)).
\end{enumerate}

\hypertarget{production-of-tax-materials.}{%
\section{Production of Tax Materials.}\label{production-of-tax-materials.}}

In any proceeding to enforce Section 1 or 2 above, the System Arbitrator shall have the authority, upon good cause shown, to direct any Team, Team Affiliate, or player to produce any tax returns or other relevant tax materials disclosing income figures for the player (non-income figures may be redacted), or disclosing expense figures by the Team or Team Affiliate (non-expense figures may be redacted), which materials shall not be released to the general public or the media and shall be treated as strictly confidential by all parties.

\hypertarget{transactions-with-retired-players.}{%
\section{Transactions with Retired Players.}\label{transactions-with-retired-players.}}

\begin{enumerate}
\def\labelenumi{(\alph{enumi})}
\item
  If (i) a Team or Team Affiliate enters into a transaction after the date of this Agreement with a retired player who played for the Team within the five-year (5) period preceding such transaction and the terms of such transaction provide for the retired player to be paid compensation or consideration in excess of \$10,000 or to be provided with an investment or business opportunity, and, if (ii) the compensation the retired player received from the Team when he was a player was substantially below the then fair market value of such player's basketball services under the Salary Cap system, then the NBA may challenge the transaction, pursuant to the procedures set forth in Section 5(b) below, on the ground that: (A) the compensation or consideration to the retired player substantially exceeds the then fair market value of the services or other consideration provided by the retired player in the transaction; or that (B) the amount of the retired player's investment or the benefit conferred upon the retired player as a result of the investment or business opportunity is not commercially reasonable, given the relative risks and rewards of such investment.
\item
  \begin{enumerate}
  \def\labelenumii{(\roman{enumii})}
  \tightlist
  \item
    Any challenge under this Section 5 shall be filed in writing with a business valuation expert jointly selected by the NBA and the Players Association. In the event the parties cannot agree on the identity of a business valuation expert, a business valuation expert shall be selected in the same manner set forth in Article XXXI, Section 7 for the selection of a Grievance Arbitrator in the absence of an agreement between the parties. The business valuation expert shall conduct a hearing in which the retired player, the Team and/or Team Affiliate, the Players Association, and the NBA are afforded the opportunity to appear and participate. The NBA shall have the burden of proof in the proceeding. The business valuation expert may permit discovery of relevant documents necessary to undertake the valuation, and shall render a decision within fifteen (15) days following the conclusion of the hearing. Within ten (10) days of any decision by the business valuation expert, any of the parties may file an appeal with the System Arbitrator, who shall conduct a hearing and render a decision within twenty (20) days of the filing of the appeal. In any such proceeding, the System Arbitrator shall apply an ``arbitrary and capricious'' standard of review. There shall be no right of further appeal to the Appeals Panel.
  \item
    If the NBA prevails in its challenge under this Section 5, the difference between (A) the compensation or consideration received by the retired player, or the value of the investment or business opportunity received by the retired player (net of any contribution by the retired player), and (B) a reasonable estimate of the fair market value of the services or other consideration provided by the retired player, or a reasonable estimate of the fair market value of the investment or business opportunity, in each case as determined by the business valuation expert or the System Arbitrator, as the case may be, shall be included in the Team's Team Salary, subject to the Team's Room and other Salary Cap rules, and further subject to any allocation over time that the business valuation expert or System Arbitrator determines is appropriate. In the event that any amount required to be included in the Team's Team Salary pursuant to this subsection exceeds the Team's Room, the challenged transaction or arrangement shall be rescinded and of no further force and effect.
  \item
    If the NBA prevails in its challenge under this Section 5, and the retired player and the Team and/or Team Affiliate renegotiate or terminate the transaction, any revised terms of the transaction shall be promptly disclosed to the NBA and the Players Association, and may, at the request of the NBA, be re-subjected to the procedures of this Section 5(b).
  \end{enumerate}
\item
  Any information disclosed to the NBA and the Players Association pursuant to the procedures of this Section 5 shall be treated strictly confidential, and shall not be released to the general public or the media.
\end{enumerate}

\hypertarget{charitable-contributions.-1}{%
\section{Charitable Contributions.}\label{charitable-contributions.-1}}

\begin{enumerate}
\def\labelenumi{(\alph{enumi})}
\tightlist
\item
  Notwithstanding any other provision in this Article XIII, a Team is permitted to make charitable contributions in respect of players on the Team so long as the combined value of all donations by a Team in respect of any one player on the Team does not exceed \$20,000 per Salary Cap Year, and the combined value of all donations in respect of all players on the Team does not exceed \$75,000 per Salary Cap Year. For purposes of this Section 6, a donation in respect of a player means a donation to a bona fide charity that qualifies as a tax exempt organization under the Internal Revenue Code and is either (i) a player's own charitable foundation or another charity with which the player is affiliated, or (ii) a charity to which a Team makes a donation on behalf of, or at the request of, a player or for the purpose of demonstrating support for a player.
\item
  The combined value of all charitable donations by a player to any Team-related charity may not exceed \$20,000 per Salary Cap Year. For purposes of this Section 6, a ``Team-related charity'' means a bona fide charity that qualifies as a tax exempt organization under the Internal Revenue Code and is either (i) the charitable foundation of the player's Team or other charity with which the Team or a Team Affiliate is affiliated, or (ii) a charity to which a player makes a donation on behalf of, or at the request of, his Team or a Team Affiliate or for the purpose of demonstrating support for the Team or Team Affiliate.
\end{enumerate}

\hypertarget{anti-collusion-provisions}{%
\chapter{ANTI-COLLUSION PROVISIONS}\label{anti-collusion-provisions}}

\hypertarget{no-collusion.}{%
\section{No Collusion.}\label{no-collusion.}}

Subject to Section 2 below, no NBA Team, its employees or agents, will enter into any contracts, combinations, or conspiracies, express or implied, with the NBA or any other NBA Team, their employees or agents: (a) to negotiate or not to negotiate with any Veteran or Rookie; (b) to submit or not to submit an Offer Sheet to any Restricted Free Agent; (c) to offer or not to offer a Player Contract to any Free Agent; (d) to exercise or not to exercise a Right of First Refusal; or (e) concerning the terms or conditions of employment offered to any Veteran or Rookie.

\hypertarget{non-collusive-conduct.}{%
\section{Non-Collusive Conduct.}\label{non-collusive-conduct.}}

The following is a non-exhaustive list of conduct that shall not be deemed a violation of Section 1 above:

\begin{enumerate}
\def\labelenumi{(\alph{enumi})}
\tightlist
\item
  the formulation and negotiation of collective bargaining proposals;
\item
  agreements between NBA Teams necessary to the assignment of a Player Contract of a Veteran or the assignment of the exclusive negotiating rights to a Draft Rookie, where such assignment is contingent upon (i) the signing by the Veteran of an amendment to an existing Player Contract (including, for example, an Extension), or (ii) the signing by the Draft Rookie of a new Player Contract; provided, however, that if such contingency is fulfilled by the Veteran entering into an amended Player Contract (including, for example, an Extension) or the Draft Rookie entering into a new Player Contract, this subsection shall only apply if the assignment is actually consummated;
\item
  an agreement between NBA Teams concerning the signing of a new Player Contract by a Veteran Free Agent with his Prior Team, where such agreement is necessary for the subsequent assignment of the new Player Contract between the agreeing Teams; provided, however, that this Section 2(c) shall apply only if the subsequent assignment is consummated, and only if the agreement and the new Player Contract comply with the provisions of Article VII, Section 8(e);
\item
  conduct authorized by the terms and conditions of the NBA Draft (as set forth in Article X above);
\item
  conduct authorized by any provision of this Agreement or conduct by the NBA League Office, undertaken in good faith, that reflects a reasonable interpretation of this Agreement or a Player Contract;
\item
  any action taken by the NBA League Office to exclude from the NBA, suspend, or discipline any player for any reason authorized or permitted by any provision of this Agreement (this Section 2(f), however, shall not affect any other rights of any player or the Players Association to contest such action); or
\item
  any disapproval by the NBA Commissioner of a Player Contract, Extension, Renegotiation, or other amendment.
\end{enumerate}

\hypertarget{individual-negotiations.}{%
\section{Individual Negotiations.}\label{individual-negotiations.}}

No NBA Team shall fail or refuse to negotiate with, or enter into a Player Contract with, any player who is free to negotiate and sign a Player Contract with any NBA Team, on any of the following grounds:

\begin{enumerate}
\def\labelenumi{(\alph{enumi})}
\tightlist
\item
  that the player has previously been subject to the exclusive negotiating rights obtained by another NBA Team in an NBA Draft; or
\item
  that the player has previously refused or failed to enter into a Player Contract containing an Option; or
\item
  that the player has become a Restricted Free Agent or an Unrestricted Free Agent; or
\item
  that the player is or has been subject to a Right of First Refusal.
\end{enumerate}

The fact that a Team has not negotiated with, made any offers to, or entered into any Player Contracts with players who are free to negotiate and sign Player Contracts with any Team, shall not, by itself, be deemed proof that such Team failed or refused to negotiate with, make any offers to, or enter into any Player Contracts with any players on any of the prohibited grounds referred to in this Section 3.

\hypertarget{league-disclosures.}{%
\section{League Disclosures.}\label{league-disclosures.}}

The NBA League Office shall not knowingly communicate or disclose, directly or indirectly, to any NBA Team that another NBA Team has negotiated with or is negotiating with any Restricted Free Agent, unless and until an Offer Sheet (as defined in Article XI, Section 5(b)) shall have been given to the ROFR Team (as defined in Article XI, Section 4(a)), or any Free Agent prior to the execution of a Player Contract with that player.

\hypertarget{enforcement-of-anti-collusion-provisions.}{%
\section{Enforcement of Anti-Collusion Provisions.}\label{enforcement-of-anti-collusion-provisions.}}

\begin{enumerate}
\def\labelenumi{(\alph{enumi})}
\tightlist
\item
  Any player, or the Players Association acting on behalf of a player or players, may bring an action before the System Arbitrator alleging a violation of Section 1 above. Issues of relief and liability shall be determined in the same proceeding (including the amount of damages, pursuant to Section 9 below, if any). The complaining party will bear the burden of demonstrating by a clear preponderance of the evidence that the challenged conduct was in violation of Section 1 above and caused economic injury to such player(s); provided, however, that the Players Association may, in the absence of economic injury to any player, bring an action before the System Arbitrator claiming a violation of Section 1 above (which must be proved by a clear preponderance of the evidence) and seeking only declaratory relief or a direction to cease and desist from the challenged conduct.
\item
  The provisions of this Agreement are not intended to create any substantive rights in any party, other than as provided for herein. This Agreement may be enforced, and any alleged violations may be remedied, only as provided for herein.
\end{enumerate}

\hypertarget{satisfaction-of-burden-of-proof.}{%
\section{Satisfaction of Burden of Proof.}\label{satisfaction-of-burden-of-proof.}}

The failure by a Team or Teams to submit Offer Sheets to Restricted Free Agents, or to make offers or sign Contracts for the playing services of Free Agents, shall not, by itself or in combination only with evidence about the playing skills of the player(s) not receiving such offers or Contracts, satisfy the burden of proof set forth in Section 5 above. However, such evidence may support a finding of a violation of Section 1 above, but only in combination with other evidence that either by itself or in combination with the evidence referred to in the immediately preceding sentence indicates that the challenged conduct was in violation of Section 1 above and, except in cases where the Players Association seeks only declaratory relief or a direction to cease and desist from the challenged conduct, caused economic injury to such player(s).

\hypertarget{summary-judgment.}{%
\section{Summary Judgment.}\label{summary-judgment.}}

The System Arbitrator may, at any time following the conclusion of any permitted discovery, determine whether or not the complainant's evidence is sufficient to raise a genuine issue of material fact capable of satisfying the standards imposed by Sections 5 and 6 above. If the System Arbitrator determines that complainant's evidence is not so sufficient, he shall dismiss the action. In considering the sufficiency of the complainant's evidence, the System Arbitrator may consider documentary evidence and affidavits submitted by the parties.

\hypertarget{remedies-for-economic-injury.}{%
\section{Remedies for Economic Injury.}\label{remedies-for-economic-injury.}}

In the event that an individual player or players, or the Players Association acting on his or their behalf, successfully proves a violation of Section 1 above that has caused economic injury, the player or players determined by the System Arbitrator to have suffered economic injury as a result of the violation will have the right:

\begin{enumerate}
\def\labelenumi{(\alph{enumi})}
\tightlist
\item
  to terminate his (or their) existing Player Contract(s) at his (or their) option (however, such termination shall not take effect until the conclusion of a then-ongoing NBA Season, if any). Such right of termination shall not arise until the recommendation of the System Arbitrator finding a violation is no longer subject to further appeal and must be exercised by the player within thirty (30) days therefrom. If, at the time the Player Contract is terminated, such player would have been an Unrestricted Free Agent pursuant to the provisions of this Agreement, he shall immediately become an Unrestricted Free Agent upon such termination. If, at the time the Player Contract is terminated, such player would have been a Restricted Free Agent pursuant to the provisions of this Agreement, such player shall immediately become a Restricted Free Agent upon such termination; however, any such player may choose to reinstate his Player Contract at any time up until September 15 of that year; and
\item
  to recover damages as described in Section 9 below. However, if the player terminates his Player Contract under Section 8(a) above and does not reinstate it pursuant thereto, he may not recover damages for the period after such termination takes effect. A player who does not terminate his Contract, or who reinstates it pursuant to Section 8(a) above, may recover damages for the entire period of his injury.
\end{enumerate}

\hypertarget{calculation-of-damages.}{%
\section{Calculation of Damages.}\label{calculation-of-damages.}}

Upon any finding of a violation of Section 1 above that has caused economic injury, compensatory damages (i.e., the amount by which any player has been injured as a result of such violation) and non-compensatory damages (i.e., the amount exceeding compensatory damages) shall be awarded as follows:

\begin{enumerate}
\def\labelenumi{(\alph{enumi})}
\tightlist
\item
  Two (2) times the amount of compensatory damages, in the event that all of the Teams found to have violated Section 1 above have committed such a violation for the first time. Any Team found to have committed such a violation for the first time shall be jointly and severally liable for two (2) times the amount of compensatory damages.
\item
  Three (3) times the amount of compensatory damages, in the event that any of the Teams found to have violated Section 1 above have committed such a violation for the second time during the term of this Agreement. In the event that damages are awarded pursuant to this Section 9(b): (i) any Team found to have committed such a violation for the first time shall be jointly and severally liable for two (2) times the amount of compensatory damages; and (ii) any Team found to have committed such a violation for the second time during the term of this Agreement shall be jointly and severally liable for three (3) times the amount of compensatory damages.
\item
  Three (3) times the amount of compensatory damages, plus, for each Team found to have violated Section 1 above for at least the third time during the term of this Agreement, four million dollars (\$4,000,000), in the event that any of the Teams found to have violated Section 1 above have committed such violation for at least the third time during the term of this Agreement. In the event that damages are awarded pursuant to this Section 9(c): (i) any Team found to have committed such a violation for the first time shall be jointly and severally liable for two (2) times the amount of compensatory damages; (ii) any Team found to have committed such a violation for at least the second time during the term of this Agreement shall be jointly and severally liable for three (3) times the amount of compensatory damages; and (iii) any Team found to have committed such a violation for at least the third time during the term of this Agreement shall, in addition, pay a fine of four million dollars (\$4,000,000).
\end{enumerate}

\hypertarget{payment-of-damages.}{%
\section{Payment of Damages.}\label{payment-of-damages.}}

In the event damages are awarded pursuant to Section 9 above, the amount of compensatory damages shall be paid to the injured player or players. The amount of non-compensatory damages, including any fines, shall be paid to the Players Association, which may use it for any purpose other than to pay it to any player who has received compensatory damages, except that any such player may receive some portion of a non-compensatory damage award as part of a proportional distribution to Players Association members.

\hypertarget{effect-of-damages-on-salary-cap.}{%
\section{Effect of Damages on Salary Cap.}\label{effect-of-damages-on-salary-cap.}}

In the event damages are awarded pursuant to Section 9 above, the amount of non-compensatory damages, including any fines, will not be included in any of the computations described in Article VII above. The amount of compensatory damages awarded will be included in such computations.

\hypertarget{contribution.}{%
\section{Contribution.}\label{contribution.}}

Any Team found liable under Section 1 above shall have the right to seek contribution from any other Team found liable for the same violation in a proceeding before the Commissioner who shall determine what contribution, if any, is fair and equitable. The Commissioner's determination with regard to contribution shall be final and binding upon and unappealable by any Team. A contribution determination by the Commissioner may be appealed by the Players Association to the System Arbitrator, except that if such a determination involves fewer than four (4) Teams found to have committed a violation of Section 1 above and allocates damages equally among the Teams found liable, there shall be no appeal to the System Arbitrator. In the event of a contribution determination by the Commissioner, the NBA shall provide the Players Association with the data and information that the Commissioner used or relied upon in making his determination. Any contribution determination appealed by the Players Association to the System Arbitrator shall be upheld unless it is clearly erroneous.

\hypertarget{no-reimbursement.}{%
\section{No Reimbursement.}\label{no-reimbursement.}}

Any damages awarded pursuant to Section 9 above must be paid by the individual Teams found liable and those Teams may not be reimbursed or indemnified by any other Team or the NBA, except to the extent of any award of contribution made pursuant to Section 12 above.

\hypertarget{costs.}{%
\section{Costs.}\label{costs.}}

In any action brought for an alleged violation of Section 1 above, the System Arbitrator shall order the payment of reasonable attorneys' fees by any party found to have brought such an action or to have asserted a defense to such an action without any reasonable basis for asserting such a claim or defense.

\hypertarget{termination-of-agreement.}{%
\section{Termination of Agreement.}\label{termination-of-agreement.}}

The Players Association shall have the right to terminate this Agreement (pursuant to the procedure set forth in Article XXXIX, Section 3 of this Agreement), under the following circumstances:

\begin{enumerate}
\def\labelenumi{(\alph{enumi})}
\tightlist
\item
  Where there has been a finding or findings of one (1) or more instances of a violation of Section 1 above with respect to any one NBA Season during the term of this Agreement which, either individually or in total, involved five (5) or more Teams and caused injury to five (5) or more players; or
\item
  Where there has been a finding or findings of one (1) or more instances of a violation of Section 1 above with respect to any two (2) consecutive NBA Seasons during the term of this Agreement which, either individually or in total, involved seven (7) or more Teams and caused economic injury to seven (7) or more players. For purposes of this Section 15(b), a player found to have been injured by a violation of Section 1 above in each of two (2) consecutive Seasons shall be counted as an additional player injured by such a violation for each such NBA Season; or
\item
  Where, in a proceeding brought by the Players Association, it is shown by clear and convincing evidence that during the term of this Agreement ten (10) or more Teams have engaged in a violation or violations of Section 1 above, causing economic injury to one or more NBA players. In order to terminate this Agreement pursuant to this Section 15(c) and Article XXXIX, Section 3 of this Agreement:

  \begin{enumerate}
  \def\labelenumii{(\roman{enumii})}
  \tightlist
  \item
    the proceeding must be brought by the Players Association; and
  \item
    the NBA and the System Arbitrator must be informed at the outset of any such proceeding that the Players Association is proceeding under this Section 15(c) for the purpose of establishing its entitlement to terminate this Agreement.
  \end{enumerate}
\end{enumerate}

\hypertarget{discovery.}{%
\section{Discovery.}\label{discovery.}}

\begin{enumerate}
\def\labelenumi{(\alph{enumi})}
\tightlist
\item
  In any of the actions described in this Article XIV, the System Arbitrator shall grant reasonable and expedited discovery upon the application of any party where, and to the extent, he or she determines it is reasonable to do so. Such discovery may include the production of documents and the taking of depositions.
\item
  Notwithstanding Section 16(a) above, the Players Association and the NBA shall each have the right to obtain discovery upon request in any three (3) proceedings brought under this Article XIV during the term of this Agreement. The scope and extent of such discovery shall be determined by the System Arbitrator.
\end{enumerate}

\hypertarget{time-limits.}{%
\section{Time Limits.}\label{time-limits.}}

Any action under Section 1 above must be brought within ninety (90) days of the time when the player knows or reasonably should have known that he had a claim, or within ninety (90) days of the start of the NBA Season in which a violation of Section 1 above is claimed, whichever is later. In the absence of a System Arbitrator, the complaining party shall file such claim for breach of this Agreement pursuant to Section 301 of the Labor Management Relations Act in either the U.S. District Court for the Southern District of New York or the U.S. District Court for the District of New Jersey. Any party alleged to have violated Section 1 shall have the right, prior to any proceedings on the merits, to make an initial motion to dismiss any complaint that does not comply with the timeliness requirement of this Section 17.

\hypertarget{certifications}{%
\chapter{CERTIFICATIONS}\label{certifications}}

\hypertarget{contract-certification.}{%
\section{Contract Certification.}\label{contract-certification.}}

\begin{enumerate}
\def\labelenumi{(\alph{enumi})}
\tightlist
\item
  Every Player Contract (other than a 10-Day Contract), or any Renegotiation, Extension, or other amendment of a Player Contract, entered into during the term of this Agreement shall be accompanied by a certification, sworn to separately by (i) the person who executed the Player Contract on behalf of the Team, (ii) the player, and (iii) any player agent who negotiated the Contract on behalf of the player, under penalties of perjury, that the Player Contract, Renegotiation, Extension, or other amendment sets forth all components of a player's Compensation from the Team or any Team Affiliate, and that there are no agreements or transactions of any kind (whether disclosed or undisclosed to the NBA), express or implied, oral or written, or promises, undertakings, representations, commitments, inducements, assurances of intent, or understandings of any kind (whether disclosed or undisclosed to the NBA):

  \begin{enumerate}
  \def\labelenumii{(\roman{enumii})}
  \tightlist
  \item
    concerning any future Renegotiation, Extension, or other amendment of an existing Player Contract, or entry into a new Player Contract; or
  \item
    except as permitted by this Agreement or contained in such Uniform Player Contract, involving compensation or consideration of any kind or anything else of value to be paid, furnished, or made available by, to, or for the benefit of the player, or any person or entity controlled by, related to, or acting with authority on behalf of the player; or
  \item
    except as permitted by this Agreement, involving an investment or business opportunity to be furnished or made available by, to, or for the benefit of the player, or any person or entity controlled by, related to, or acting with authority on behalf of the player.
  \end{enumerate}
\item
  Prior to the assignment of any Player Contract of a player who is in the last Salary Cap Year of the Contract (or the last Salary Cap Year before the player or the Team has the right to terminate the Contract), the player, the player's agent, and the Team to which such Contract is to be assigned shall each submit to the NBA a certification, sworn to under penalties of perjury, that other than the Player Contract that has been assigned, or as permitted by this Agreement, there are no agreements or transactions of any kind (whether disclosed or undisclosed to the NBA), express or implied, oral or written, or promises, undertakings, representations, commitments, inducements, assurances of intent, or understandings of any kind (whether disclosed or undisclosed to the NBA), between the player (or the player's agent or any person or entity controlled by or related to the player) and the Team to which the Player Contract is to be assigned or a Team Affiliate of the Team to which the Player Contract is to be assigned concerning (i) any future Renegotiation, Extension, or other amendment of the Player Contract that has been assigned, (ii) any future Player Contract, or (iii) an investment or business opportunity or compensation or consideration of any kind or anything else of value to be paid, furnished, or made available by, to, or for the benefit of the player or any person or entity controlled by, related to, or acting with authority on behalf of the player.
\item
  If a player, within two (2) years after the assignment of such player's Player Contract, enters into a new Player Contract, or any Renegotiation, Extension, or other amendment of the Player Contract that had been assigned, the Team, the player, and the player's agent shall each submit to the NBA a certification, sworn to under penalties of perjury, that, at the time of the assignment, other than the Player Contract that has been assigned, or as permitted by this Agreement, there were no agreements or transactions of any kind (whether disclosed or undisclosed to the NBA), express or implied, oral or written, or promises, undertakings, representations, commitments, inducements, assurances of intent, or understandings of any kind (whether disclosed or undisclosed to the NBA), between the player (or the player's agent or any person or entity controlled by or related to the player) and the Team to which the Player Contract has been assigned or a Team Affiliate of the Team to which the Player Contract has been assigned concerning (i) any future Renegotiation, Extension, or other amendment of the Player Contract that has been assigned, (ii) any future Player Contract, or (iii) an investment or business opportunity or compensation or consideration of any kind or anything else of value to be paid, furnished, or made available by, to, or for the benefit of the player or any person or entity controlled by, related to, or acting with authority on behalf of the player. Such certification shall be submitted to the NBA no later than sixty (60) days following the execution of such new Player Contract, or any Renegotiation, Extension, or other amendment of the Player Contract.
\item
  If an agent, player, or Team fails or refuses to provide a certification called for under this Article XV, the NBA shall have the option, in its sole discretion, to approve or disapprove the transaction in question. In the case of a failure or refusal by an agent, and whether the transaction in question is approved or disapproved, the Players Association shall take appropriate disciplinary action against the agent.
\end{enumerate}

\hypertarget{end-of-season-certification.}{%
\section{End of Season Certification.}\label{end-of-season-certification.}}

\begin{enumerate}
\def\labelenumi{(\alph{enumi})}
\tightlist
\item
  At the conclusion of each NBA Season, a Governor (or Alternate Governor) and the executive primarily responsible for basketball operations on behalf of the Team shall each submit to the NBA a certification, sworn to under penalties of perjury, that the Team has not, to the extent of their knowledge after reasonable inquiry, (i) violated the terms of Article XIV, Section 1, (ii) violated the terms of Article XIII, Section 2, nor (iii) received from the NBA League Office any communication disclosing that an NBA Team has negotiated with any Free Agent prior to the execution of a Player Contract with that player. Upon receipt of each such certification, the NBA shall forward a copy of the certification to the Players Association.
\item
  A violation of this Section 2 may be deemed evidence of a violation of Article XIV, Section 1 or Article XIII, Section 2.
\end{enumerate}

\hypertarget{false-certification.}{%
\section{False Certification.}\label{false-certification.}}

Any criminal complaint of perjury filed by the NBA or any Team based upon a certification required pursuant to Section 1 above shall be against the player, the player's agent, and the Team official making such certification.

\hypertarget{mutual-reservation-of-rights}{%
\chapter{MUTUAL RESERVATION OF RIGHTS}\label{mutual-reservation-of-rights}}

Upon the expiration or termination of this Agreement, no person shall be deemed to have waived, by reason of the entry into or effectuation of this Agreement, any other collective bargaining agreement, or any Player Contract, or any of the terms of any of them, or by reason of any practice or course of dealing, their respective rights under law with respect to any issue or their ability to advance any legal argument.

\hypertarget{procedure-with-respect-to-playing-conditions-at-various-facilities}{%
\chapter{PROCEDURE WITH RESPECT TO PLAYING CONDITIONS AT VARIOUS FACILITIES}\label{procedure-with-respect-to-playing-conditions-at-various-facilities}}

\chaptermark{PROCEDURE WITH RESPECT TO PLAYING CONDITIONS AT VARIOUS \ldots}

When a new franchise is granted, or when an existing franchise moves to another city or a new or different arena, the Players Association shall, upon request and within a reasonable period of time, have the right to inspect the facility to be used by such franchise. Similarly, the Players Association shall, upon reasonable notice to the Team(s) involved and the NBA, have the right to inspect the training camp and practice facilities used by such Team(s). If, following such inspection, the Players Association is of the opinion that the playing conditions at such facility will endanger the health and safety of NBA players, it shall promptly notify the Commissioner and the Team involved in writing. Promptly following the receipt of such notice, representatives of the Players Association and of the Team(s) involved, and the Commissioner or his designee shall meet in an effort to resolve the matter. It is agreed that the failure of the parties to resolve the matter shall not impair the legally binding effect of this Agreement or create any right, during the term of this Agreement, to (a) unilaterally implement any provision concerning such unresolved matter, (b) lockout, or (c) strike. If no resolution satisfactory to the Players Association, the Team(s) involved, and the Commissioner is reached, the issue of whether the playing conditions at the facility in question will endanger the health and safety of NBA players will, without interruption of the schedule or training camp or practice activities, immediately be submitted to and determined by the Grievance Arbitrator in accordance with the provisions of Article XXXI; provided, however, that the Grievance Arbitrator need not render an award within twenty-four (24) hours of the conclusion of the hearing, but shall issue his award as expeditiously as possible under the circumstances.

\hypertarget{travel-accommodations-locker-room-facilities-and-parking}{%
\chapter{TRAVEL ACCOMMODATIONS, LOCKER ROOM FACILITIES, AND PARKING}\label{travel-accommodations-locker-room-facilities-and-parking}}

\hypertarget{hotel-arrangements.}{%
\section{Hotel Arrangements.}\label{hotel-arrangements.}}

\begin{enumerate}
\def\labelenumi{(\alph{enumi})}
\tightlist
\item
  Each Team agrees to use its best efforts to make the following arrangements for its players while they are ``on the road'':

  \begin{enumerate}
  \def\labelenumii{(\roman{enumii})}
  \tightlist
  \item
    to have their baggage picked up by porters;
  \item
    to have them stay in first class hotels; and
  \item
    to have extra-long beds available to them in each hotel.
  \end{enumerate}

  If there is a finding that a Team has committed a willful violation of this Section 1(a), the NBA shall impose a \$5,000 fine on such Team.
\item
  When its players are ``on the road,'' each Team shall provide an individual hotel room for each player.
\end{enumerate}

\hypertarget{first-class-travel.}{%
\section{First Class Travel.}\label{first-class-travel.}}

\begin{enumerate}
\def\labelenumi{(\alph{enumi})}
\tightlist
\item
  Each Team shall provide first class travel accommodations on all trips in excess of one (1) hour, except when such accommodations are not available; provided, however, that a Team's head coach may fly first class in place of a player when eight (8) or more first class seats are provided to players. In the event a Team's head coach flies first class in place of a player, one (1) player, designated by the Players Association, shall be paid the difference between the amount paid by such Team for a first class seat on the flight involved and the cost of the seat purchased for such designated player on that flight.
\item
  If there is a finding that a Team has committed a willful violation of Section 2(a) above, the NBA shall impose a \$5,000 fine on such Team.
\end{enumerate}

\hypertarget{locker-room-facilities.}{%
\section{Locker Room Facilities.}\label{locker-room-facilities.}}

Each Team agrees to provide suitable locker room facilities and to use its best efforts to stabilize the temperature in locker rooms to make it consistent with the temperature on playing courts.

\hypertarget{parking-facilities.}{%
\section{Parking Facilities.}\label{parking-facilities.}}

Each Team agrees to make parking facilities available to its players without charge in connection with games and practices conducted at the facility regularly used by such Team for home games and/or practices.

\hypertarget{hotel-incidentals.}{%
\section{Hotel Incidentals.}\label{hotel-incidentals.}}

In the event that a player fails or refuses to pay any incidental charges he has incurred in connection with a hotel room provided to him by his Team while the Team is ``on the road,'' he shall be subject to the following discipline: (i) for each of the first two (2) occasions during the Season -- a maximum fine of \$100; and (ii) for any subsequent occasion during such Season, such discipline as is reasonable under the circumstances.

\hypertarget{two-way-players.}{%
\section{Two-Way Players.}\label{two-way-players.}}

The foregoing requirements and obligations set forth in Sections 1, 2, and 5 above shall not apply to any Two-Way Player traveling between his NBA Team and NBAGL team.

\hypertarget{union-security-dues-and-check-off}{%
\chapter{UNION SECURITY, DUES, AND CHECK-OFF}\label{union-security-dues-and-check-off}}

\hypertarget{membership.}{%
\section{Membership.}\label{membership.}}

As a condition of employment commencing with the execution of this Agreement, for the duration of this Agreement only, and wherever legal: (a) any active player who is or later becomes a member in good standing of the Players Association must maintain his membership in good standing in the Players Association; and (b) any active player (including a player in the future) who is not a member in good standing of the Players Association must, on the 30th day following the beginning of his employment or the 30th day following the execution of this Agreement, whichever is later, pay, pursuant to Section 2 below or otherwise, financial core obligations to the Players Association related to collective bargaining and the administration of collective bargaining agreements (hereinafter referred to as ``financial core fees'').

\hypertarget{check-off.}{%
\section{Check-off.}\label{check-off.}}

Commencing with the execution of this Agreement and for the duration of this Agreement only, each Team, following its receipt of the requisite authorization form, will check-off the initiation fee and annual dues, assessments, and financial core fees, as the case may be, in equal installments from the first four (4) payments made thereafter to the player pursuant to Paragraph 3(a) of the Uniform Player Contract or from such lesser number of payments made thereafter as provided for by Exhibit 1 to such Contract, for each player for whom a current check-off authorization has been provided to the Team. The Team will forward the check-off monies to the Players Association within fourteen (14) days of each check-off. If the Team fails to do so, interest at seven percent (7\%) per annum, payable to the Players Association, shall begin to accrue on such check-off monies upon the conclusion of such fourteen (14) day period.

\hypertarget{enforcement.}{%
\section{Enforcement.}\label{enforcement.}}

\begin{enumerate}
\def\labelenumi{(\alph{enumi})}
\item
  Upon written notification to the NBA by the Players Association that a player has not paid any initiation fee, dues, or financial core fees in violation of Section 1 above, the NBA will raise the matter for discussion with the player and his Team. If there is no resolution of the matter within seven (7) days, then the Team will, upon the written request of the Players Association, suspend the player without pay, wherever legal. Such suspension will continue until the Players Association has notified the Team in writing that the suspended player has satisfied his obligation as contained in Section 1 above. The parties hereby agree that suspension without pay is adopted as a substitute for and in lieu of discharge as the penalty for a violation of the union security clause of this Agreement and that no player will be discharged for a violation of that clause.

  A copy of all notices required by this Section 3(a) will be simultaneously mailed to the player involved and the NBA.
\item
  The term ``member in good standing'' as used in this Article XIX applies only to the payment of dues or any initiation fee and not to any other factors involved in union discipline.
\item
  Other than pursuant to Section 2 above, no Team shall pay any initiation fees, dues, or financial core fees on behalf of any player.
\end{enumerate}

\hypertarget{no-liability.}{%
\section{No Liability.}\label{no-liability.}}

Neither the NBA nor any Team shall be liable for any salary, bonus, or other monetary or non-monetary claims that result from a player being suspended pursuant to the terms of Section 3 above. The Players Association indemnifies, saves, and holds harmless the NBA and each Team against any and all claims, demands, suits, or other forms of liability that may arise, directly or indirectly, in connection with the enforcement or application of any term or provision of this Article XIX, including, without limitation, claims relating to any action taken by the NBA or any Team in reliance upon any written authorization provided hereunder.

\hypertarget{scheduling}{%
\chapter{SCHEDULING}\label{scheduling}}

\hypertarget{training-camp.}{%
\section{Training Camp.}\label{training-camp.}}

\begin{enumerate}
\def\labelenumi{(\alph{enumi})}
\item
  Veteran Players will not be required to attend training camp earlier than 11 a.m. (local time) on the twenty-second day prior to the first game of any Regular Season. On such twenty-second day, Veterans may only be required to attend a Team dinner and Team meetings, participate in photograph and media sessions, and submit to a physical examination.
\item
  Notwithstanding Section 1(a) above, if a Veteran Player is under contract to a Team that is scheduled during a particular NBA Season to participate outside North America in one (1) or more Exhibition or Regular Season games during the first ten (10) days of the Regular Season (each such Team, a ``Global Games Team''), such Veteran Player may be required to attend the training camp conducted in advance of that Regular Season by 11 a.m. (local time) on the earlier of (i) if any such game is scheduled to be held in South America, the twenty-sixth day prior to the first game of the Regular Season; (ii) if any such game is scheduled to be held in Europe, the twenty-seventh day prior to the first game of the Regular Season; and (iii) if any such game is scheduled to be held in Africa, Asia, or the Oceania region, the twenty-eighth day prior to the first game of the Regular Season. If a Global Games Team requires a Veteran Player to attend training camp earlier than the twenty-fifth day prior to the first game of the Regular Season in accordance with the foregoing, then, beginning on the day immediately following the date on which the Global Games Team lands at its destination airport in North America after the game(s) outside North America, such Veteran Player shall be provided one (1) Day Off for each day earlier than the twenty-fifth day prior to the first game of the Regular Season that such Global Games Team required the Player to attend training camp.
\item
  ``First-Year Players'' (defined below) may be required to attend training camp on a date earlier than the date(s) specified in Sections 1(a) and 1(b) above, but no earlier than ten (10) days prior to the date that Veterans on such Team are required to attend. If a Global Games Team requires Veteran Players under contract to the Team to attend training camp earlier than the twenty-fifth day prior to the first game of the Regular Season pursuant to Section 1(b) above, and further requires First-Year Players to attend training camp on a date that is ten (10) days prior to the date that Veterans on such Team are required to attend, then, beginning on the day immediately following the date on which the Global Games Team lands at its destination airport in North America after the game(s) outside North America, each such First-Year Player shall be provided one (1) Day Off for each day earlier than the thirty-fifth day prior to the first game of the Regular Season that such Global Games Team required the First-Year Player to attend training camp.

  For purposes of this Section 1(c), ``First-Year Player'' means a player with zero (0) Years of Service who is under Contract to a Team.
\item
  \begin{enumerate}
  \def\labelenumii{(\roman{enumii})}
  \tightlist
  \item
    Team training camps may be held at any location, within or outside the United States and Canada. The NBA shall oversee the arrangements made with respect to any training camp held outside the United States and Canada and the accommodations provided to participating players.
  \item
    The NBA shall be required to notify the Players Association of its intention to conduct a team training camp outside the United States and Canada. Within three (3) business days of its receipt of such notification, the Players Association shall have the right to disapprove such plans, provided that such disapproval may be based solely on a reasonable and well-founded concern that the location of such training camp would be unsafe for players.
  \item
    No Team shall hold its training camp outside the United States and Canada in any two (2) successive Seasons, it being understood that limited practice sessions held in connection with one (1) or more exhibition games outside of the United States or Canada shall not be considered training camp for the purposes of this Section 1(d)(iii).
  \item
    Players on a Team that holds its training camp outside of the United States and Canada shall have at least one (1) day off following the travel day during which they travel back to the United States or Canada from such training camp.
  \end{enumerate}

  For purposes of this Section 1(d), the U.S. Territories and Caribbean islands shall not be considered ``outside the United States and Canada.''
\item
  \begin{enumerate}
  \def\labelenumii{(\roman{enumii})}
  \tightlist
  \item
    During any six (6) days beginning on the day after the first day of training camp and ending on the fourteenth (14th) day of training camp (the ``Two-a-Day Period''): (A) a Team shall be permitted to conduct no more than two (2) regular practice sessions per day; (B) such session(s) may last an aggregate of no longer than 3.5 hours (excluding time -- not to exceed 30 minutes -- spent stretching and participating in aerobic warm-ups and cool-downs); (C) there must be at least a two (2) hour interval between the two (2) practice sessions; and (D) if a Team elects to conduct two (2) regular practice sessions during a day, one (1) of the two (2) sessions must be limited to non-contact activities. For the remainder of training camp, a Team shall be permitted to conduct no more than one (1) regular practice session per day and such session may last no longer than 3.5 hours (excluding time -- not to exceed 30 minutes -- spent stretching and participating in aerobic warm-ups and cool-downs); provided, however, that any Team that is unable due to international travel for pre-season events to conduct two (2) practice sessions per day during the Two-a-Day Period may make up any missed practice sessions (up to a maximum of two (2)) during the first five (5) days upon the Team's return from such international travel.
  \item
    If a Team conducts one (1) or two (2) regular practice sessions during a day in accordance with Section 1(e)(i) above, then except as provided in clause (A) of Section 1(e)(iii) below, the Team shall not, at a separate time during the day, conduct, organize or supervise any additional basketball activity on the basketball court.
  \item
    Nothing in Sections 1(e)(i) and (ii) above shall be construed to prohibit a Team, on any day of training camp, from conducting one (1) or two (2) regular practice sessions in accordance with Section 1(e)(i) above, plus:

    \begin{enumerate}
    \def\labelenumiii{(\Alph{enumiii})}
    \tightlist
    \item
      on-court skills development sessions (e.g., pick-and-roll situations, shooting, passing, etc.) not involving the playing of live defense (i.e., only ``dummy'' defense may be played) and not involving the practicing of four-man or five-man offenses or defenses; and
    \item
      team-related or training-related activities (including, but not limited to, weight training, other conditioning sessions (excluding high-impact conditioning drills that are normally conducted during regular practice sessions), video sessions, meetings, and promotional appearances), so long as such additional activities do not include any basketball activity on the basketball court that is organized, supervised, or conducted by the Team.
    \end{enumerate}
  \end{enumerate}
\end{enumerate}

\hypertarget{exhibition-games.}{%
\section{Exhibition Games.}\label{exhibition-games.}}

\begin{enumerate}
\def\labelenumi{(\alph{enumi})}
\tightlist
\item
  Exhibition games prior to any Regular Season shall not exceed six (6) (including intra-squad games for which admission is charged), and Exhibition games during any Regular Season shall not exceed three (3).
\item
  Exhibition games shall not be played on the three (3) days prior to the Team's first Regular Season game in the United States or Canada, on the day prior to a Regular Season game, or on the day prior to and the day following the All-Star Game.
\end{enumerate}

\hypertarget{regular-season-games.}{%
\section{Regular Season Games.}\label{regular-season-games.}}

Each Team agrees that in no event will it play more than eighty-two (82) Regular Season games.

\hypertarget{in-season-tournament.}{%
\section{In-Season Tournament.}\label{in-season-tournament.}}

Each Season, the NBA shall determine and supervise the arrangements made with respect to an In-Season Tournament, which shall consist of two stages: (a) the group stage and (b) the knockout stage.

\begin{enumerate}
\def\labelenumi{(\alph{enumi})}
\tightlist
\item
  \textbf{Group Stage.} All Teams shall participate in the group stage. Each Team shall play a total of four (4) group stage games. Such games shall be scheduled by the NBA to take place in the first two (2) months of the Regular Season on two (2) designated days of the week.

  \begin{enumerate}
  \def\labelenumii{(\roman{enumii})}
  \tightlist
  \item
    To determine the schedule of group stage games, the NBA shall divide the Teams in each Conference into three (3) groups of five (5) Teams each (each group, a ``Group Stage Group'') via random drawings. In the group stage, each Team shall play one (1) game against each of the other four (4) teams in its Group Stage Group.
  \item
    Each Group Stage Group shall include one team from each of the following subgroups, which are based on the Teams' winning percentage in the prior Regular Season:

    \begin{enumerate}
    \def\labelenumiii{(\arabic{enumiii})}
    \tightlist
    \item
      First- through third-highest in the Conference,
    \item
      Fourth- through sixth-highest in the Conference,
    \item
      Seventh- through ninth-highest in the Conference,
    \item
      Tenth- through twelfth-highest in the Conference, and
    \item
      Thirteenth- through fifteenth-highest in the Conference.
    \end{enumerate}
  \end{enumerate}
\item
  \textbf{Knockout Stage.}

  \begin{enumerate}
  \def\labelenumii{(\roman{enumii})}
  \tightlist
  \item
    Eight (8) Teams shall participate in the knockout stage:

    \begin{enumerate}
    \def\labelenumiii{(\arabic{enumiii})}
    \tightlist
    \item
      The Team with the best winning percentage in group stage games in each of the Group Stage Groups; and
    \item
      One (1) ``wildcard'' Team from each Conference, which shall be the Team from each Conference with the best winning percentage in group stage games that finished second in the standings in group stage games in its Group Stage Group.
    \end{enumerate}
  \item
    Each game in the knockout stage shall be a single elimination game (i.e., the Team that wins such game shall advance to the next round of the knockout stage and the Team that loses shall be eliminated from the In-Season Tournament).
  \item
    For the first round of the knockout stage (the ``IST Quarterfinals''), in each Conference, (A) the Team with the highest winning percentage in group stage games shall host the ``wildcard'' Team, and (B) the Team with the second-highest winning percentage in group stage games shall host the Team with the third-highest winning percentage in group stage games.
  \item
    The games in the second round of the knockout stage (the ``IST Semifinals Games'') and the In-Season Tournament championship game (the ``IST Finals Game'') shall be played at a neutral site (i.e., not the home arena for any participating Team).
  \item
    A player (including a Two-Way Player) shall not be eligible to participate in the IST Finals Game with a participating Team if such player was not on such Team's roster as of the start of the first scheduled IST Semifinals Game.
  \item
    The twenty-two (22) teams that do not participate in the knockout stage shall each play two (2) additional Regular Season games during the knockout stage on days on which knockout stage games are not scheduled. The four (4) Teams that play in the IST Quarterfinals but do not qualify for an IST Semifinals Game shall each play one (1) additional Regular Season game during the knockout stage on days on which knockout stage games are not scheduled.
  \end{enumerate}
\item
  Each game played as part of the In-Season Tournament other than the IST Finals Game shall be a Regular Season game. Notwithstanding the foregoing, the IST Finals Game shall be considered a Regular Season game for all purposes under this Agreement except: (i) a Team's Regular Season winning percentage or standings; (ii) Article II, Sections 11(b)(ii)-(iii), 12(b), and 13(j); (iii) Article IV, Sections 1(b) and 3(a)(7); (iv) Article XI, Sections 1(e)(ii) and 1(e)(iv); (v) Article XX, Sections 3 and 9(e); and (vi) NBA By-Laws Section 5.05(b) (the provisions of which are referenced in and attached to the Uniform Player Contract).
\item
  \textbf{League Honors.} Each Season, players will be selected for Most Valuable In-Season Tournament Player and All-Tournament Team honors based on their performance in group stage games and knockout stage games in the In-Season Tournament that Season.
\item
  \textbf{Players Association Event.} The Players Association may schedule and hold a public event to be included on the official NBA calendar for the In-Season Tournament, with such event subject to approval by the NBA. The NBA and Players Association shall work together in good faith to avoid scheduling such Players Association event at the same time as a public NBA event. For clarity, this Section 4(e) shall not preclude the Players Association from holding other events during the In-Season Tournament that are not included on the official NBA calendar for the In-Season Tournament.
\end{enumerate}

\hypertarget{location-and-scheduling-of-games.}{%
\section{Location and Scheduling of Games.}\label{location-and-scheduling-of-games.}}

\begin{enumerate}
\def\labelenumi{(\alph{enumi})}
\tightlist
\item
  Exhibition and Regular Season games may be conducted at any location, within or outside the United States and Canada. The NBA shall supervise the arrangements made with respect to games conducted outside the United States and Canada and the accommodations provided to participating players.
\item
  Each year the NBA shall establish the schedule of Regular Season, In-Season Tournament, Play-In, and playoff games in its discretion (subject to Article XXXIX, Section 5), provided that the number of days beginning on the date of the first Regular Season game and continuing through the date of the last Regular Season game each Season shall equal approximately one hundred seventy-four (174). Notwithstanding the foregoing, if any such games are cancelled due to one or more events set forth in Article XXXIX, Section 5 (e.g., weather or natural disasters) or any other unexpected game cancellation (e.g., due to unexpected unavailability of a Team's arena or transportation), the NBA may reschedule any such cancelled game(s) in its discretion, after consulting with the Players Association.
\item
  Prior to the NBA's public announcement of the Regular Season game schedule each year, the NBA shall provide the Players Association with an initial draft of such schedule (no later than the date that such draft is provided to all NBA teams), and the Players Association shall have an opportunity to provide the NBA with comments (within at least as many days as NBA teams are given by the NBA to provide such comments). The NBA shall identify for the Players Association any game(s) included in such draft schedule in which a Team is scheduled to play on the same day that such Team has traveled across two (2) time zones. The NBA shall consider, but shall have no obligation to make any changes in respect of, the Players Association's comments regarding the draft schedule. The Players Association shall keep the draft schedule confidential, including by maintaining the confidentiality of any differences between the final schedule publicly announced by the NBA and the draft schedule previously received by the Players Association.
\end{enumerate}

\hypertarget{holidays.}{%
\section{Holidays.}\label{holidays.}}

\begin{enumerate}
\def\labelenumi{(\alph{enumi})}
\tightlist
\item
  No Team will be required to play a game on December 25, unless such game is to be telecast or cablecast nationally.
\item
  Games scheduled to be played on January 1 and Good Friday shall not commence prior to 6 p.m. (local time), unless the Players Association consents thereto, which consent shall not be unreasonably withheld. The Players Association will, upon request, consent to the earlier commencement of two (2) games on Good Friday and four (4) games on January 1 if such games are to be broadcast or cablecast nationally, and provided that the Teams involved are in the same time zone or otherwise in close geographic proximity.
\item
  Teams at home on December 25 and January 1 (each, a ``Holiday'') may, but shall not be required to, conduct a practice on either (or both) of such Holidays, provided: (i) the Team's players have requested that they practice on the Holiday, as communicated to the Team by the Team's player representative; and (ii) within seven (7) days before or after the Holiday, the Team's players are provided with a ``day off'' -- i.e., the Team will not conduct any practice, including any optional practice, on such date, and the Team will not have a scheduled game on such date.
\item
  Teams shall not depart for an away game or series of away games prior to 3 p.m. (local time) on December 25 or January 1, unless reasonable transportation arrangements for such game or games cannot be made at or after 3 p.m. (local time).
\end{enumerate}

\hypertarget{all-star.}{%
\section{All-Star.}\label{all-star.}}

No Team that plays a game on the Thursday prior to the All-Star Game shall play a game on the Tuesday following the All-Star Game or conduct a practice session prior to such Tuesday at 2 p.m. (local time).

\hypertarget{travel.}{%
\section{Travel.}\label{travel.}}

The NBA and its Teams shall use their best efforts to devise reasonable travel schedules when Team training camps, Exhibition games, and Regular Season games are conducted or played outside the United States and Canada.

\hypertarget{days-off.}{%
\section{Days Off.}\label{days-off.}}

\begin{enumerate}
\def\labelenumi{(\alph{enumi})}
\tightlist
\item
  Each Team will provide a minimum of eighteen (18) Days Off during each Regular Season for each of its players on dates to be determined by the Team. A ``Day Off'' means a calendar day on which a player is not required or permitted to participate in any Team directed activities, including, but not limited to, games, practices, travel, or promotional activities. Without limitation, Days Off shall include days that satisfy the foregoing definition and are provided: (i) during All-Star Weekend pursuant to Article XXI, Section 4 (only with respect to players not participating in All-Star activities); and (ii) in locations other than the Team's home city (such as when the Team is ``on the road''). Under no circumstances shall a Team pressure or coerce a player into providing services for the Team on a player's Day Off. Nothing contained herein shall prevent any player on his Day Off from voluntarily engaging in individual basketball related activity at the Team's facility or elsewhere (including, but not limited to, individual activity with Team coaches, trainers, or medical personnel). Each Team shall maintain a list of the Days Off provided to each player on such Team during the Regular Season.
\item
  A calendar day shall not fail to meet the definition of a Day Off because the Team is traveling on such day, provided the Team lands at its destination point (i.e., lands at its destination airport or, if the Team has not flown and is instead traveling by train or bus, arrives at the final destination of such train or bus) before:

  \begin{enumerate}
  \def\labelenumii{(\roman{enumii})}
  \tightlist
  \item
    1:00 a.m. (local time at the destination point) on such day if, at the time of departure, the local time at the departure point (i.e., the airport from which the Team departs, or if the Team has not flown, the point from which the Team's form of transportation, such as a train or bus, departs) is the same or later than the local time at the destination point, or
  \item
    2:00 a.m. (local time at the destination point) on such day if, at the time of departure, the local time at the departure point is earlier than the local time at the destination point.
  \end{enumerate}

  For any calendar day on which the Team arrives at its destination point at or after 1:00 a.m. (local time at the destination point) that (a) could not meet the definition of a Day Off in accordance with subsection (i) above, and (b) could meet the definition of a Day Off in accordance with subsection (ii) above, in order for such calendar day to meet the definition of a Day Off, the Team must, before concluding traveling as a team (i.e., before the players who traveled with the Team disembark from the final plane, train, or bus), notify the one or more players who traveled with the Team and will be provided a Day Off on that calendar day that they will be provided a Day Off on that calendar day.
\item
  For a player whose Player Contract is entered into after the first day of the Regular Season, the Team will provide a minimum number of Days Off during such Regular Season, rounded up or down to the nearest whole Day Off, calculated by multiplying 18 by a fraction, the numerator of which is the number of days covered by the Player Contract during such Regular Season (including the day on which the Player Contract is entered into), and the denominator of which is the total number of days in such Regular Season; provided, however, that:

  \begin{enumerate}
  \def\labelenumii{(\roman{enumii})}
  \tightlist
  \item
    A Team is not required to provide any Day Off to a player during a Regular Season if the term of his Player Contract covers fewer than 25 days during such NBA Regular Season (including the day on which the Player Contract is entered into). Teams are also not required to provide any Day Off to a player whose Player Contract is a Two-Way Contract; and
  \item
    A player who signs a Rest-of-Season Contract after March 1 of a Regular Season may waive his right to receive Days Off pursuant to this Section 9 for such Regular Season. Such waiver must be in writing, signed by the player, and approved by the Players Association.
  \end{enumerate}
\item
  For a player whose Player Contract is assigned by one Team to another Team during a Regular Season via trade or the NBA's waiver procedure, the assignor Team's obligation pursuant to Article XX, Section 9(a) shall be deemed satisfied with respect to the player for such Regular Season, and the acquiring Team will provide the player a minimum number of Days Off during such Regular Season calculated as if the player had entered into a Rest-of-Season Contract: (i) in the case of a trade, on the date that all conditions to the trade are satisfied; or (ii) in the case of a waiver claim, on the date that the acquiring Team acquires the player's Contract pursuant to the NBA waiver procedure.
\item
  In the event that any Season does not include at least an eighty-two (82) game Regular Season schedule, the requirements of Sections 9(a)-(d) above shall not apply and the NBA and Players Association will negotiate an alternate Days Off rule for such Season.
\end{enumerate}

\hypertarget{nba-all-star-game}{%
\chapter{NBA ALL-STAR GAME}\label{nba-all-star-game}}

\hypertarget{participation.}{%
\section{Participation.}\label{participation.}}

\begin{enumerate}
\def\labelenumi{(\alph{enumi})}
\tightlist
\item
  Any player selected (by any method designated by the NBA) to play in an All-Star Game shall be required to:

  \begin{enumerate}
  \def\labelenumii{(\roman{enumii})}
  \tightlist
  \item
    attend and participate in such Game;
  \item
    attend and participate in one (1) All-Star Skills Competition (but not including the Slam Dunk Competition) designated by the NBA that is conducted during the All-Star Weekend on which such Game is held; and
  \item
    attend and participate in every other event conducted in association with such All-Star Weekend, including, but not limited to, a reasonable number of media sessions, television appearances, and promotional appearances.
  \end{enumerate}
\item
  Any player selected (by any method designated by the NBA) to play in a Rookie-Sophomore Game (e.g., Rookies vs.~Sophomores, captains-selected mix of Rookies and Sophomores on each team, or U.S. players vs.~international players) shall be required to:

  \begin{enumerate}
  \def\labelenumii{(\roman{enumii})}
  \tightlist
  \item
    attend and participate in such Game;
  \item
    attend and participate in any All-Star Skills Competition designated by the NBA that is conducted during the All-Star Weekend on which such Game is held; and
  \item
    attend and participate in every other event conducted in association with such All-Star Weekend, including, but not limited to, a reasonable number of media sessions, television appearances, and promotional appearances.
  \end{enumerate}
\item
  Any player who has not been selected to play in the All-Star Game or the Rookie-Sophomore Game, but has been selected (by any method designated by the NBA) to participate in an All-Star Skills Competition (but not including the Slam Dunk Competition) shall be required to attend and participate in such Skills Competition. Notwithstanding the foregoing, no player will be required to attend and participate in such All-Star Skills Competition for more than two (2) consecutive years, unless he is the prior year's winner of such All-Star Skills Competition. Any player who, at the request of the NBA, voluntarily agrees to participate in the Slam Dunk Competition, shall be required to attend and participate in such Slam Dunk Competition.
\item
  Nothing in this Article XXI shall preclude a player who is an officer or a representative of the Players Association from attending the Players Association's annual meeting during All-Star Weekend or preclude any player from attending the Players Association's All-Star party.
\item
  Notwithstanding anything to the contrary in Section 1(a), (b), or (c) above, a player will not be required to participate in a particular All-Star Game, Rookie-Sophomore Game, or All-Star Skills Competition if he has been excused from participation in the particular event by the Commissioner because (i) he has an injury or illness that renders him physically unable to participate in such Game or Skills Competition, or (ii) for such other reason as the Commissioner may determine in his sole discretion. If the player asserts, or the player's Team asserts in respect of the player, that he should be excused from participation in a particular All-Star Game or event under Section 1(e)(i) above, the Commissioner shall be authorized to require the player to submit to a medical examination to be performed by a physician designated by the NBA, and the determination of whether Section 1(e)(i) is satisfied shall be made by such physician in his sole discretion. In the event that a player is excused from participation in an All-Star Game or event under Section 1(e)(i) above, he shall thereafter remain on his Team's Inactive List until he is cleared to return to the Active List by the NBA.
\item
  Any player who is selected to play in an All-Star Game but is excused from participation under Section 1(e) above shall not receive the All-Star award due to him under Section 2(a) below unless (i) he does not play in his Team's last Regular Season game prior to that All-Star Game or (ii) he does not play in his Team's first Regular Season game following that All-Star Game.
\end{enumerate}

\hypertarget{awards.}{%
\section{Awards.}\label{awards.}}

\begin{enumerate}
\def\labelenumi{(\alph{enumi})}
\tightlist
\item
  For their participation in an All-Star Game, players on the winning team shall each receive \$100,000 and players on the losing team shall each receive \$25,000.
\item
  For their participation in a Rookie-Sophomore Game, players on the winning team shall each receive \$25,000 and players on the losing team shall each receive \$10,000 (or, if there are more than two teams of players that participate in the Rookie-Sophomore Game, then players on the winning team shall each receive \$25,000, players on the second place team shall each receive \$15,000, and players on the remaining teams shall each receive \$10,000).
\item
  For their participation in an All-Star Skills Competition, players shall receive the following amounts:
\end{enumerate}

\begin{longtable}[]{@{}lclc@{}}
\toprule()
Slam Dunk & & Three-Point Shootout & \\
\midrule()
\endhead
1st Place: & \$105,000 & 1st Place: & \$60,000 \\
2nd Place: & \$55,000 & 2nd Place: & \$40,000 \\
3rd Place: & \$20,000 & 3rd Place: & \$25,000 \\
4th Place: & \$20,000 & 4th Place: & \$15,000 \\
& & 5th Place: & \$15,000 \\
& & 6th Place: & \$15,000 \\
& & 7th Place: & \$15,000 \\
& & 8th Place: & \$10,000 \\
\bottomrule()
\end{longtable}

\begin{longtable}[]{@{}lc@{}}
\toprule()
Skills Challenge & \\
\midrule()
\endhead
1st Place: & \$55,000 \\
2nd Place: & \$40,000 \\
3rd Place: & \$20,000 \\
4th Place: & \$20,000 \\
5th Place: & \$15,000 \\
6th Place: & \$15,000 \\
7th Place: & \$15,000 \\
8th Place: & \$15,000 \\
\bottomrule()
\end{longtable}

\hypertarget{player-guests.}{%
\section{Player Guests.}\label{player-guests.}}

Each player who participates in the All-Star Game, Rookie-Sophomore Game, or any All-Star Skills Competition may invite two (2) guests, who shall be reimbursed for the cost of round-trip first-class air transportation between the home city of the Team by which such player is employed and the site of the All-Star Game, Rookie-Sophomore Game, or All-Star Skills Competition.

\hypertarget{players-not-participating-in-all-star-activities.}{%
\section{Players Not Participating in All-Star Activities.}\label{players-not-participating-in-all-star-activities.}}

Players who do not attend or participate in the All-Star Game, Rookie-Sophomore Game, an All-Star Skills Competition, or NBAGL All-Star activities shall have three (3) days off during the All-Star Weekend break.

\hypertarget{all-star-skills-competitions.}{%
\section{All-Star Skills Competitions.}\label{all-star-skills-competitions.}}

The All-Star Skills Competitions that take place during any All-Star Weekend shall be selected by the NBA; provided, however, that before adding any new event to the All-Star Skills Competitions that take place during any All-Star Weekend (i.e., an event different from any conducted by the NBA during any All-Star Weekend held prior to the 2023-24 Season), the NBA shall obtain the consent of the Players Association, which consent shall not be unreasonably withheld. The rule relating to mandatory participation in Section 1(c) above shall apply only to current All-Star Skills Competitions (with the exception of the Slam Dunk Competition), unless the player is the prior year's winner of an All-Star Skills Competition (with the exception of the Slam Dunk Competition), and the new event is consented to by the Players Association under this Section 5.

\hypertarget{all-star-committee.}{%
\section{All-Star Committee.}\label{all-star-committee.}}

The NBA and the Players Association shall continue to discuss in good faith matters relating to All-Star Weekend, including the nature, schedule, and format of All-Star events, player participation therein, and award amounts.

\hypertarget{player-health-and-wellness}{%
\chapter{PLAYER HEALTH AND WELLNESS}\label{player-health-and-wellness}}

\hypertarget{requirements-for-certain-team-player-health-professionals.}{%
\section{Requirements for Certain Team Player Health Professionals.}\label{requirements-for-certain-team-player-health-professionals.}}

\begin{enumerate}
\def\labelenumi{(\alph{enumi})}
\tightlist
\item
  Each Team must secure the services of at least two (2) physicians as lead team physicians, at least one (1) of whom must be board certified in orthopedic surgery and at least one (1) of whom must be board certified in internal medicine, family medicine, or emergency medicine. Beginning with the 2017-18 Season, each individual hired for the first time to perform services as a team physician must be a duly licensed physician who as of the hiring date: (i) is board certified in his/her field of medical expertise; (ii) has successfully completed a fellowship in sports medicine, has a Certification of Added Qualification (CAQ) in sports medicine, or has other ``sports medicine'' qualifications as the parties may agree; and (iii) has at least five (5) years of clinical experience following the completion of such fellowship or CAQ (or of such other ``sports medicine'' qualifications as agreed by the parties). Each individual who performs services as a team physician additionally must be trained and hold a current certification in Basic Life Support, Basic Trauma Life Support, Advanced Cardiac Life Support, or Advanced Trauma Life Support. The NBA will issue additional rules regarding game coverage by team physicians, which shall include, among other requirements, that each Team ensure attendance at each home game of at least one (1) team physician who is board certified in orthopedic surgery and at least one (1) team physician who is board certified in internal medicine, family medicine, or emergency medicine.
\item
  Each Team must secure the services of at least one (1) athletic trainer to serve as the Head Athletic Trainer and one (1) athletic trainer to serve as an Assistant Athletic Trainer on a full-time basis. Beginning with the 2017-18 Regular Season: (i) each individual hired for the first time to perform services as an athletic trainer for a Team must as of the hiring date: (a) be certified by the National Athletic Trainers Association (NATA) or the Canadian Athletic Therapists Association (CATA) (or a similar organization as the parties may agree), and (b) be trained and hold a current certification in Basic Life Support, Basic Trauma Life Support, Advanced Cardiac Life Support, or Advanced Trauma Life Support; and (ii) each individual hired for the first time to perform services as a Head Athletic Trainer for a Team must, as of the hiring date, have at least three (3) years of experience as an athletic trainer since he/she first received such foregoing athletic training certification. The NBA will issue additional rules regarding game coverage by athletic trainers.
\item
  Each Team must secure the services of at least one (1) strength and conditioning coach on a full-time basis and designate one (1) strength and conditioning coach as the Head Strength and Conditioning Coach. Beginning with the 2017-18 Regular Season: (i) each individual hired for the first time to perform services as a strength and conditioning coach for a Team must, as of the hiring date, have a degree from an accredited four-year college or university and a certification from the National Strength and Conditioning Association (NSCA) (which, for each individual hired for the first time beginning with the 2023-24 Season, must be a Registered Strength and Conditioning Coach (RSCC) or Certified Strength and Conditioning Specialist (CSCS) certification from the NSCA) (or a certification from a similar organization as the parties may agree), and (ii) each individual hired for the first time to perform services as a Head Strength and Conditioning Coach for a Team must, as of the hiring date, have at least three (3) years of experience as a strength and conditioning coach since he/she first received such foregoing strength and conditioning certification. In addition, all individuals who perform services as a strength and conditioning coach for a Team must be trained and hold a current certification in Basic Life Support, Basic Trauma Life Support, Advanced Life Support, or Advanced Trauma Life Support.
\end{enumerate}

\hypertarget{one-surgeon.}{%
\section{One Surgeon.}\label{one-surgeon.}}

Each Team agrees that a player requiring the care and treatment of an orthopedic surgeon will, so far as practicable, be referred to and treated by one (1) orthopedic surgeon (rather than several).

\hypertarget{nba-physicians-association.}{%
\section{NBA Physicians Association.}\label{nba-physicians-association.}}

Representatives designated by the Players Association shall participate in meetings of the NBA Physicians Association for the purpose of discussing matters related to the medical care and treatment of players.

\hypertarget{disclosure-of-medical-information.}{%
\section{Disclosure of Medical Information.}\label{disclosure-of-medical-information.}}

\begin{enumerate}
\def\labelenumi{(\alph{enumi})}
\tightlist
\item
  A Team physician may disclose all relevant medical information concerning a player to (i) the General Manager, coaches, and trainers of the Team by which such player is employed, (ii) any entity from which any such Team seeks to procure, or has procured, an insurance policy covering such player's life or any disability, injury, illness, or other medical condition such player may suffer or sustain, and (iii) subject to the terms of Sections 4(d)-(e) below, the media or public on behalf of the Team.
\item
  Should it be requested in connection with the contemplated assignment of a player's Uniform Player Contract to one or more NBA Teams, a Team's physician may furnish all relevant medical information relating to the player to (i) the physicians and General Manager, coaches, and trainers of such other Team or Teams, and (ii) any entity from which any such other Team seeks to procure, or has procured, an insurance policy covering such player's life or any disability, injury, illness, or other medical condition such player may suffer or sustain.
\item
  Should a Team assign a player to the NBAGL, such Team's physician may furnish all relevant medical information relating to the player to (i) the physicians and General Manager, head coaches, and trainers of the player's NBAGL team, and (ii) any entity from which the Team, the NBAGL, or the player's NBAGL team seeks to procure, or has procured, an insurance policy covering such player's life or any disability, injury, illness, or other medical condition such player may suffer or sustain. In addition, an NBAGL team physician may furnish all relevant medical information relating to the player to the physicians and General Manager, coaches, and trainers of the player's Team.
\item
  Subject to Section 4(e) below, each Team may make public medical information relating to the players in its employ, provided that such information relates solely to the reasons why any such player has not been or is not rendering services as a player. If a player, in the judgment of the Team, is expected to be unable to participate in any basketball practice or game due to an injury, illness, or other medical condition for a period of two or more weeks, the Team's first public statement regarding such player's injury, illness, or other medical condition may only describe such injury, illness, or other medical condition and the anticipated date when such player will be re-evaluated by the Team. The Team may make subsequent public statement(s) with all relevant medical information only after such re-evaluation has occurred.
\item
  A player or his immediate family (where appropriate) shall have the right to approve the terms and timing of any public release of medical information relating to any injuries, illnesses, or other medical conditions suffered by that player that are potentially life- or career-threatening, or that do not arise from the player's participation in NBA games or practices. If a Team or the NBA requests such approval and the player or his immediate family (where appropriate) does not provide it, then the Team is limited to disclosing that an injury, illness, or other medical condition is preventing a player from rendering services to the Team and that the anticipated length of the player's absence from rendering services to the Team is unknown.
\item
  Nothing in Sections 4(d)-(e) shall limit a Team from disclosing medical information related to an injury, illness, or other medical condition with respect to any player who has made medical information available publicly that is inconsistent with the written opinion of a Team physician.
\item
  In addition to the access set forth in Article XXII, Section 8 of the CBA below, a player is entitled access to his own medical records and the Team shall use best efforts to provide such information on or before forty-eight (48) business hours of a player request.
\end{enumerate}

\hypertarget{vaccination-education-and-recommendations.}{%
\section{Vaccination Education and Recommendations.}\label{vaccination-education-and-recommendations.}}

The NBA and the Players Association shall, at least annually, jointly recommend, and issue educational materials to players (in connection with the Rookie Transition Program and Team Awareness Meetings described in Article VI, Section 4 of the CBA and via written materials provided to all players) regarding, the health benefits of vaccinations recommended by the CDC (i.e., as of the effective date of this Agreement, COVID-19, measles, mumps, and rubella (MMR), influenza, tetanus and pertussis, varicella (chicken pox), Hepatitis B) and the meningococcal vaccine.

\hypertarget{selection-of-team-physician-and-other-health-care-providers.}{%
\section{Selection of Team Physician and Other Health Care Providers.}\label{selection-of-team-physician-and-other-health-care-providers.}}

Each Team has the sole and exclusive discretion to select any doctors, hospitals, clinics, health consultants, or other health care providers (``Health Care Providers'') to examine and/or treat players pursuant to the terms of this Agreement and the Uniform Player Contract; provided, however, no Team will engage any such Health Care Provider based primarily on a sponsorship relationship (or lack thereof) with the Team, and without considering the Health Care Provider's qualifications (including, e.g., medical experience and credentials) and the goal of providing high quality care to all of its players.

\hypertarget{health-and-performance-screenings.}{%
\section{Health and Performance Screenings.}\label{health-and-performance-screenings.}}

Players shall undergo reasonable screening and baseline testing (e.g., pursuant to NBA cardiac and concussion protocols) and, in connection with such screening and testing, shall accurately and completely answer all reasonable health questions (including, upon request, providing accurate and complete medical histories). Players additionally shall participate in any league-wide biomechanics screening and assessment program upon request and direction by the NBA, provided that, prior to implementing any such program, the NBA shall consult with the Players Association, and provided further that any such assessment program shall require no more than four (4) assessments for any one Season. Any other new league-wide performance screening and assessment program directed by the NBA and required for players shall require prior agreement of the NBA and the Players Association.

\hypertarget{electronic-medical-records.}{%
\section{Electronic Medical Records.}\label{electronic-medical-records.}}

\begin{enumerate}
\def\labelenumi{(\alph{enumi})}
\tightlist
\item
  The NBA will use, during the Term, an electronic medical records system (``EMR'') that will provide a secure, searchable, centralized database of player health information. To the extent health information disclosures are permitted by this Agreement (including the Uniform Player Contract), such disclosures may be made via secure systems within the EMR. In addition, the EMR will: (i) allow for the NBA (but not the Teams) to conduct player health and safety reviews; (ii) allow for authorized academic researchers to access the data (on a de-identified basis) and conduct studies designed to improve player health and broaden medical knowledge (provided that the Players Association will be provided with notice prior to any such access and gives its consent, such consent not to be unreasonably withheld); and (iii) give players the ability to easily access their own health information and to grant access to such information to physicians of their choice both during and after their careers.
\item
  To satisfy the requirement in Section 8(a)(iii) above, by no later than the end of the 2023-24 Season, the NBA shall make available a mobile app for exclusive use by players to facilitate direct access for each player to such information in the EMR. The NBA shall also provide the same or similar access through the app for exclusive use by former players in respect of whom the EMR contains medical information. Following the 2023-24 Season, and annually following each Season thereafter, the NBA shall provide a Players Association-designated physician with a summary report for each player, which will summarize information on such player from the EMR (identified by player name) regarding such player's injuries, illnesses, or medical conditions, imaging studies, prescription medications, surgeries, vaccinations, concussions and concussion evaluation, and cardiac screening. In order to confirm player consent for the NBA to provide the above summary and related information to the Players Association, the NBA shall include in the NBA's health information authorization, which in accordance with Paragraph 7(i) of the UPC each player is required to sign annually, an authorization for the NBA to provide medical records to the Players Association.
\end{enumerate}

\hypertarget{concussion-cardiac-and-emergency-medical-preparedness-policies.}{%
\section{Concussion, Cardiac, and Emergency Medical Preparedness Policies.}\label{concussion-cardiac-and-emergency-medical-preparedness-policies.}}

\begin{enumerate}
\def\labelenumi{(\alph{enumi})}
\tightlist
\item
  A concussion policy designed to maximize the neurological health of players shall be in effect during the Term. The concussion policy will be reviewed and updated periodically by the NBA in conjunction with the NBA Physicians Association and the NBA's Concussion Advisory Committee in order to keep the policy current and consistent with the evolving science of concussion management. Prior to any update to the concussion policy, the NBA shall consult with the Players Association.
\item
  A cardiac screening policy designed to identify cardiovascular risks for players shall be in effect during the Term. The cardiac screening policy will be reviewed and updated periodically by the NBA in conjunction with the NBA Physicians Association and the NBA's Cardiac Advisory Committee in order to keep the policy current and consistent with the evolving science of sports cardiology. Prior to any update to the cardiac screening policy, the NBA shall consult with the Players Association.
\item
  A policy for response to medical emergencies designed to provide a framework for a rapid response to on-court emergencies shall be in effect during the Term. The emergency medical preparedness policy will be reviewed and updated periodically by the NBA in conjunction with the NBA Physicians Association and the NBA's Emergency Medical Preparedness Committee in order to keep the policy current and consistent with recommendations from organizations and experts with emergency response expertise. Prior to any update to the emergency medical preparedness policy, the NBA shall consult with the Players Association.
\end{enumerate}

\hypertarget{second-opinion.}{%
\section{Second Opinion.}\label{second-opinion.}}

\begin{enumerate}
\def\labelenumi{(\alph{enumi})}
\tightlist
\item
  Subject to the additional terms in subsections (b) through (e) below, players shall have the right to receive a second medical opinion at the Team's expense regarding the course of treatment for an injury, illness, or other medical condition that either: (i) has prevented the player from participating in a Regular Season, Play-In, or playoff game for two (2) weeks or more; (ii) in the opinion of a Team physician for the player's Team, is more likely than not to prevent the player from being able to participate in an NBA game for two (2) weeks or more (or during the off-season, from participating in competitive basketball without restriction for two weeks or more); (iii) in the opinion of the Team physician will not be significantly aggravated by the player continuing to participate in NBA games (or during the offseason participating in basketball without restriction) when the player reasonably believes that continued participation will significantly aggravate his injury, illness, or other medical condition; (iv) results in direction from the Team physician that the player should undergo surgery; or (v) results in direction from the Team physician that the player should not undergo surgery when the player reasonably believes that surgery is necessary for the injury, illness, or other medical condition. The foregoing shall not limit a player's ability to obtain a second medical opinion in circumstances other than those set forth in Sections 10(a)(i)-(v) above, provided that the Team shall not be obligated to pay for or consider any such second opinion.
\item
  The parties will maintain and publish annually a list (the ``Second Opinion List'') of jointly-appointed medical specialists, including one or more psychiatrist(s) (each a ``Second Opinion Physician''), by specialty and by geographic region in the United States and Canada, to provide players with the second medical opinions described in subsection (a) above. At least two (2) board-certified physicians shall be designated as Second Opinion Physicians for each specialty in each of the geographic regions.
\item
  Each Second Opinion Physician will be included on the Second Opinion List for the duration of this Agreement, unless either the NBA or the Players Association has provided written notice to the other party that a physician should be removed from the Second Opinion List (i) by December 1 of any year covered by this Agreement; or (ii) at any time of any year covered by this Agreement, for failure to provide a player's Team with all information relating to a consultation with the player within two (2) business days following the consultation; provided that, for the first such failure, a party is required to issue a warning to the Second Opinion Physician (following written notice to the other party), with removal permitted thereafter if the Second Opinion Physician does not provide the player's Team with all information relating to such consultation within two (2) business days following the warning, or for the second or any additional instances in which the Second Opinion Physician does not timely provide a player's Team with all information relating to a consultation with the player. Such removal shall be effective immediately, provided that, unless otherwise agreed by the parties, such removal shall not affect any second opinion process involving such Physician that has previously been requested by a player.
\item
  Prior to obtaining a second opinion, a player shall notify the Team in writing of his decision to seek such second opinion, the name of the physician who will be performing the evaluation, and the date and location of the evaluation. Upon receiving such notice and prior to the player's evaluation, the Team will make available to the physician relevant medical information regarding the player.
\item
  If, pursuant to subsections (a) through (d) above, a player obtains a second opinion from a Second Opinion Physician, the Team will pay the medical costs associated with the second opinion provided such cost is reasonable for the consultation.
\item
  In connection with obtaining a second opinion from a Second Opinion Physician pursuant to subsections (a) through (e) above, a player may not be absent from the Team for an unreasonable period of time or miss any games without authorization of the Team.
\item
  If the Second Opinion Physician provides the Team with a written opinion, and the player has otherwise complied with Paragraph 7(h) of the UPC, the Team will be required to consider the second opinion in connection with diagnosis or treatment. For clarity, nothing in this Section 10 shall be construed to alter or limit in any way the rights of any Team or the obligation of any player under the CBA or Uniform Player Contract, including without limitation pursuant to the provisions of Paragraph 7 of the Uniform Player Contract.
\end{enumerate}

\hypertarget{fitness-to-play.}{%
\section{Fitness-to-Play.}\label{fitness-to-play.}}

\begin{enumerate}
\def\labelenumi{(\alph{enumi})}
\item
  The parties shall establish panels of physicians (each a ``Fitness-to-Play Panel'') for the purpose of determining, as set forth in this Section 11, whether players with potentially life-threatening injuries, illnesses, or other medical conditions (or any of the foregoing that have the potential to result in paralysis or other permanent spinal injury) are medically able and medically fit to practice and play basketball in the NBA. Each Fitness-to-Play Panel shall consist of one (1) physician appointed by the NBA, one (1) physician appointed by the Players Association, and one (1) physician appointed by agreement of the first two (2) physicians. Each member of each Panel shall: (i) be board certified and fellowship trained in his/her field of medical expertise; (ii) be a specialist in the subject matter of the applicable Fitness-to-Play Panel; and (iii) have at least ten (10) years of post-fellowship clinical experience. Each Panel will operate by majority vote, including, but not limited to, its fitness to play determinations. Once appointed, each physician on a Fitness-to-Play Panel shall be included on such Panel for the duration of this Agreement, unless either the NBA or the Players Association has, by December 1 of any year covered by this Agreement, served written notice to the other party that a physician has been removed from such Panel. A party may not remove the physician that the other party appointed to a Fitness-to-Play Panel. In the event that either party removes a physician from a Fitness-to-Play Panel pursuant to the foregoing, such removal shall be effective immediately, provided that, unless otherwise agreed to by the parties, a physician will continue to serve on the Fitness-to-Play Panel in respect of any determination on a player's injury, illness, or other medical condition that has been referred to the Panel but for which the Panel has not yet issued its written determination.
\item
  The parties shall create one or more Fitness-to-Play Panels as are necessary to address injuries, illnesses, or other medical conditions that are potentially life-threatening or have the potential to result in paralysis or other permanent spinal injury for the player (e.g., cardiac illnesses and conditions, blood clots, and other blood conditions and disorders).
\item
  If the NBA, a Team, or the Players Association has been advised by a physician that a player is medically unable and/or medically unfit to perform his duties as a professional basketball player as a result of a potentially life-threatening injury, illness, or other medical condition and/or that performing such duties would likely create a materially elevated risk of death, paralysis, or other permanent spinal injury for the player, then the NBA, a Team, or the Players Association may refer the player to a Fitness-to-Play Panel by making such a referral in writing to the player and to the NBA, Team, and Players Association, as applicable. Once so referred, the player will not be permitted to play or practice in the NBA until he is cleared to do so by the Panel as set forth below.
\item
  \begin{enumerate}
  \def\labelenumii{(\arabic{enumii})}
  \tightlist
  \item
    Upon the referral described in subsection (c) above, the Panel will be provided with all medical information in the player's medical file that any member of the Panel deems relevant to the injury, illness, or other medical condition for which the player was referred. The Panel will review the player's injury, illness, or other medical condition (which review shall include an in-person examination of the player by each member of the Panel unless such member determines that an examination by him/her would serve no useful purpose). Upon conclusion of its review, the Panel shall provide a report to the NBA, the player's Team, and the Players Association setting forth its determination and the reasons therefor.
  \item
    The determination to be made by the Panel is whether, in the Panel's reasonable medical judgment and experience, and having considered current medical knowledge and the best available objective evidence: (i) the player is medically able and medically fit to perform his duties as a professional basketball player; and (ii) performing such duties would not create a materially elevated risk of death, paralysis, or other permanent spinal injury for the player. Where there are authoritative medical guidelines on fitness for athletic participation and a particular injury, illness, or other medical condition (e.g., the American Heart Association/American College of Cardiology Scientific Statements on Eligibility and Disqualification -- Recommendations for Competitive Athletes with Cardiovascular Abnormalities), the Panel will consider such guidelines in making its determination.
  \item
    Subsequent to the player being referred to a Fitness-to-Play Panel, and prior to the Panel's review of the player's injury, illness, or other medical condition, the player (on behalf of himself, his heirs, and assigns) shall be required to sign a release and covenant not to sue agreement in the form agreed upon by the parties; provided that this agreement shall not apply to any claim of medical malpractice against a Team-affiliated physician or any physician retained by the NBA or Players Association for the medical evaluation process.
  \end{enumerate}
\item
  In the event that the Fitness-to-Play Panel determines that the player is medically able and medically fit to play professional basketball pursuant to the standard in subsection (d) above: (i) the player will be required to sign an informed consent and assumption of risk agreement in the form agreed upon by the parties before he is able to play or practice in the NBA; and (ii) upon satisfying the prior clause, shall be deemed at that time medically able and fit to play basketball in the NBA and permitted to do so.
\item
  If the Fitness-to-Play Panel does not determine that the player is medically able and medically fit to play professional basketball pursuant to the standard in subsection (d) above, the NBA, a Team, or the Players Association may again refer the player to the Fitness-to-Play Panel beginning on the later of the first day of the Season that begins immediately following the date on which the Panel issued its report or nine (9) months after such date. The party making such referral must have been advised in writing by a physician that there have been materially changed circumstances since the Panel issued its report (e.g., medical advances or a material change in the player's medical condition) such that the Panel should reconsider its determination. If a player is referred under this subsection (f), the Fitness-to-Play Panel shall be comprised of the same members that reviewed and determined the player's initial referral, provided that the physicians on such panel are available.
\item
  Nothing in this Section 11 shall obligate a Team to permit a player to play or practice for the Team, even if a Fitness-to-Play Panel determines that the player is medically able to do so. If the Team disagrees with the Fitness-to-Play Panel's conclusion and refuses to permit the player to play and practice with the Team due to the injury, illness, or other medical condition for which the player was referred to the Fitness-to-Play Panel, then the Team will be required, within sixty (60) days of the Panel's issuance of its report (or, if the report is issued during the period from the date that is sixty (60) days prior to the date of the NBA trade deadline through May 31, then by August 1) (the ``Evaluation Period''), to either trade the player, agree to amend the player's Contract in accordance with Article II, Section 3(p) of the CBA, waive the player pursuant to Paragraph 16 of the Uniform Player Contract, or waive the player pursuant to the ``Partial Waiver Procedure'' described in Section 11(i) below (a ``Partial Waiver''); provided, however, that the foregoing shall not apply to any player who is in the last year of his Contract (excluding any Option Year) at the time that the Panel provides its report to the NBA, the player's Team, and the Players Association pursuant to Section 11(d)(1) above. During the Evaluation Period, the player, shall cooperate with the Team in connection with the Team's efforts to evaluate the player's injury, illness, or other medical condition, including by, among other things, in a prompt and diligent manner supplying all information requested of him, completing medical forms, and submitting to all examinations, tests, and workouts requested of him by or on behalf of the Team.
\item
  If a player referred to a Fitness-to-Play Panel satisfies the waiting period set forth in Article VII, Section 4(h)(1) of the CBA at the time of such referral (or any time thereafter prior to the Panel issuing its report), then the Team may request that such Panel, acting by majority vote, also serve as the physician described in Article VII, Section 4(h)(2) of the CBA, and accordingly provide in the Panel's report a determination for the purposes of Article VII, Section 4(h) of the CBA.
\item
  In order for an eligible Team, pursuant to Section 11(g) above, to designate an eligible player's Contract for a Partial Waiver, the Team must provide written notice of such waiver and designation to the NBA. Once a Team duly invokes the Partial Waiver Procedure, such procedure shall operate as follows:

  \begin{enumerate}
  \def\labelenumii{(\roman{enumii})}
  \tightlist
  \item
    The waiver period shall be the same as the period for other waivers.
  \item
    Any Team other than the Team requesting the waiver may submit either a Full Waiver Claim or a Partial Waiver Claim for the player. A ``Full Waiver Claim'' is a claim for the full value of the remaining term of the Contract pursuant to Section 5 of the NBA By-Laws. A ``Partial Waiver Claim'' is a discount bid of a specified dollar amount (rounded to the nearest dollar) for a portion of the value of the remaining term of the Contract. A Partial Waiver Claim can be for any amount equal to or greater than the total of the applicable Minimum Player Salary for all of the Remaining Protected Years (as defined below) of the Contract and less than the total of the full Base Compensation provided for in all of the Remaining Protected Years of the Contract, provided that a Partial Waiver Claim may never be less than the total of the unprotected Base Compensation provided for in all of the Remaining Protected Years of the Contract. A ``Remaining Protected Year'' means any remaining year of the Contract that contains any amount of Base Compensation protection that is not contingent on some event occurring on a date after the request for waivers; any remaining years of the Contract that are not Remaining Protected Years shall hereinafter be referred to as ``Remaining Unprotected Years.'' For clarity, any Player Option Year in which the Contract includes the language in Article XII, Section 2(a)(A) and the Effective Season of an ETO shall be a Remaining Protected Year, and any Player Option Year in which the Contract that includes the language in Article XII, Section 2(a)(B) and any Team Option Year shall be a Remaining Unprotected Year.
  \item
    In order to submit a Partial Waiver Claim, the Team must have a Team Salary below the Salary Cap and room equal to at least the portion of the Claiming Team Base Compensation Obligation (as defined in subsection (vi)(A) below) plus any Likely Bonuses applicable to the first Year of the Remaining Protected Years of the Contract. For purposes of the preceding sentence, ``room'' includes room that can be unilaterally created by the claiming Team (e.g., via renouncements or waivers, but not via trades) and such room must be created immediately upon the awarding of the player pursuant to this waiver procedure.
  \item
    If at least one (1) Full Waiver Claim is submitted during the waiver period, the Contract shall be awarded to the Team submitting a Full Waiver Claim that is entitled to the highest order of preference in accordance with the waiver procedures set forth in the NBA Constitution and By-Laws. If no Full Waiver Claim is submitted and at least one (1) Partial Waiver Claim is submitted, the Contract shall be awarded to the Team submitting the highest Partial Waiver Claim in total dollars (or, if more than one (1) Team submits the highest Partial Waiver Claim in total dollars, to the Team submitting the highest Partial Waiver Claim in total dollars that is entitled to the highest order of preference in accordance with the waiver procedures set forth in the NBA Constitution and By-Laws).
  \item
    If there is no Full Waiver Claim or Partial Waiver Claim submitted for the Contract during the waiver period, the Contract shall be terminated.
  \item
    In the event that the Contract is awarded to a Team (the ``Claiming Team'') as the result of a Partial Waiver Claim:

    \begin{enumerate}
    \def\labelenumiii{(\Alph{enumiii})}
    \tightlist
    \item
      The Claiming Team shall be responsible for payment of the player's Base Compensation in an amount equal to the total dollar amount of the Partial Waiver Claim allocated over the Remaining Protected Years of the Contract in proportion to the Base Compensation amounts provided for in each Remaining Protected Year of the Contract (e.g., if the player has two (2) years remaining on his Contract with \$10 million of Base Compensation in year one that is fully protected and \$11 million of Base Compensation in year two that is fifty percent (50\%) protected and the winning Partial Waiver Claim was for \$6 million, the Claiming Team shall be responsible for \$2.86 million of the player's Base Compensation in year one and \$3.14 million in year two) (the ``Claiming Team Base Compensation Obligation''). The waiving Team shall be responsible for paying the total Base Compensation in each Remaining Protected Year of the Contract less the Claiming Team Base Compensation Obligation for each Remaining Protected Year of the Contract (the ``Waiving Team Base Compensation Obligation''). In addition to the Claiming Team Base Compensation Obligation, the Claiming Team shall also be responsible for the total amount of all other Compensation obligations contained in the Contract other than Base Compensation (including, but not limited to, the full amount of any Incentive Compensation) and the total Base Compensation for any Remaining Unprotected Year.
    \item
      The Claiming Team Base Compensation Obligation plus any Likely Bonuses applicable to each Remaining Protected Year of the Contract and the total Base Compensation plus any Likely Bonuses of any Remaining Unprotected Year shall be included in the Team Salary of the Claiming Team immediately upon the awarding of the player to the Claiming Team pursuant to this waiver procedure.
    \item
      The Claiming Team may not trade a player awarded as a result of a Partial Waiver Claim until the July 1 following the award of the player's Contract to the Claiming Team pursuant to this waiver procedure. If a Claiming Team proposes to trade to another Team a player awarded as a result of a Partial Waiver Claim (after the waiting period set forth in the preceding sentence) or if the Claiming Team subsequently waives the player and another Team proposes to acquire such player in accordance with the NBA waiver procedure, then: (i) for purposes of determining (a) whether the acquiring Team has Room for the Contract, and (b) in the case of a trade, the amount of any Traded Player Exception in respect of such player's Contract, the player's Salary shall be deemed to equal the Claiming Team Base Compensation Obligation plus any Likely Bonuses applicable to the then-current Salary Cap Year; and (ii) the acquiring Team shall thereafter be deemed the Claiming Team for the purposes of this Section 11(i).
    \item
      The Claiming Team shall be responsible for making all payments to the player (and paying all related payroll taxes) other than Compensation due with respect to any Season prior to the waiver. The waiving Team shall reimburse the Claiming Team for the portion of the Waiving Team Base Compensation Obligation applicable to each pay period on or before each applicable pay date.
    \end{enumerate}
  \item
    In the event that the Contract is awarded to the Claiming Team as a result of a Partial Waiver Claim and the Claiming Team subsequently waives the player (a ``Subsequent Waiver'') resulting in the termination of the Contract:

    \begin{enumerate}
    \def\labelenumiii{(\Alph{enumiii})}
    \tightlist
    \item
      Without taking into consideration any conditional Base Compensation protection triggered after the date of the initial request for waivers but before the Subsequent Waiver (hereinafter referred to as ``Triggered Base Compensation Protection''), if the Contract contains full Base Compensation protection in each of the Remaining Protected Years or if the Contract contains no Remaining Protected Years, the Claiming Team Base Compensation Obligation and the Waiving Team Base Compensation Obligation shall remain unchanged.
    \item
      Without taking into consideration any Triggered Base Compensation Protection, if the Contract contains partial protection in one (1) or more of the Remaining Protected Years, the Claiming Team Base Compensation Obligation and Waiving Team Base Compensation Obligation for each such year shall be adjusted as follows upon the termination of the Contract:

      \begin{enumerate}
      \def\labelenumiv{(\arabic{enumiv})}
      \tightlist
      \item
        The Claiming Team Base Compensation Obligation for any Remaining Protected Year that contains only partial Base Compensation protection shall be reduced by a number equal to the Claiming Team Base Compensation Obligation for that year, divided by the total Base Compensation obligation for that year, multiplied by the unprotected Base Compensation remaining to be paid that year (the ``Adjusted Claiming Team Base Compensation Obligation'').
      \item
        The Waiving Team Base Compensation Obligation for any Remaining Protected Year that contains only partial Base Compensation protection shall be reduced by a number equal to the Waiving Team Base Compensation Obligation for that year, divided by the total Base Compensation obligation for that year, multiplied by the unprotected Base Compensation remaining to be paid for that year.
      \end{enumerate}
    \item
      The full amount of any Triggered Base Compensation Protection shall be added to the Adjusted Claiming Team Base Compensation Obligation in each remaining year of the Contract that contains Triggered Base Compensation Protection.
    \end{enumerate}
  \end{enumerate}
\item
  The costs associated with the Fitness-to-Play Panels will be borne equally by the NBA and the Players Association, and the Players Association's share shall be paid by the NBA and included in Player Benefits under Article IV, Section 6(l) of this Agreement.
\end{enumerate}

\hypertarget{player-care-survey.}{%
\section{Player Care Survey.}\label{player-care-survey.}}

The NBA and the Players Association will jointly conduct a confidential player survey during the 2023-24 Season (and during one or more subsequent Seasons during the term of this Agreement as determined by the parties) to solicit the players' input and opinion regarding the adequacy of medical care provided by their respective medical and training staffs and commission independent analyses of the results of such surveys. The costs of such surveys and analyses will be borne equally by the NBA and the Players Association, and the Players Association's share shall be paid by the NBA and included in Player Benefits under Article IV, Section 6(l) of this Agreement.

\hypertarget{wearables.}{%
\section{Wearables.}\label{wearables.}}

\begin{enumerate}
\def\labelenumi{(\alph{enumi})}
\tightlist
\item
  The wearables joint advisory committee formed by the NBA and the Players Association (the ``Wearables Committee'') shall continue to review and approve wearable devices for use by players. ``Wearables'' shall mean a device worn by an individual that measures movement information (such as distance, velocity, acceleration, deceleration, jumps, changes of direction, and player load calculated from such information and/or height/weight), physiological information (such as heart rate, heart rate variability, skin temperature, blood oxygen, hydration, lactate, and/or glucose), or other health, fitness, and performance information.
\item
  The Wearables Committee shall consist of three (3) representatives appointed by the NBA and three (3) representatives appointed by the Players Association. At least one of the members appointed by each of the NBA and the Players Association must have at least three (3) years of experience in sports medicine (such as a physician, athletic trainer, strength and conditioning coach, or sports scientist) in the NBA or with an NCAA Division I collegiate basketball team (or other relevant experience and expertise as agreed upon by the parties). Unless otherwise agreed by the parties, Committee members may not have an ownership or other financial interest in any company that produces or sells any wearable device.
\item
  The Wearables Committee shall be responsible for: (i) reviewing all requests by Teams, the NBA, or the NBPA to approve a wearable device for use by players, with the standard being whether the wearable device would be potentially harmful to anyone (including the player) if used as intended, and whether the wearable's functionality has been validated; and (ii) setting cybersecurity standards for the storage of data collected from Wearables.
\item
  The Wearables Committee will jointly retain such experts as it deems necessary in order to conduct its work (e.g., to validate a wearable device or to set cybersecurity standards), which the parties expect to include professionals in areas such as engineering, data science, and cybersecurity. The costs of such experts will be borne equally by the NBA and the Players Association, and the Players Association's share shall be paid by the NBA and included in Player Benefits under Article IV, Section 6(l) of this Agreement.
\item
  No Team may request a player to use any Wearable unless such device is one of the devices currently in use as set forth in Section 13(f) below or the device and the Team's cybersecurity standards have been approved by the Committee pursuant to Section 13(c) above.
\item
  Teams may request that, on a voluntary basis, players use the following devices: the FirstBeat Sport system; the Catapult Sports OptimEye, ClearSky, and Vector systems (including with a Polar chest strap but not with the Catapult heart rate vest); the iMeasureU Step Trident system; Kinexon Sports systems (including with a Kinexon heart rate vest paired with a Polar sensor); the ShotTracker system; the Strive Sense3 systems; the WHOOP Performance Strap 2.0; the Zephyr Performance System; and the Oura Ring (collectively, the ``Approved Wearables''). Wearables (whether Approved or otherwise) may not be used in games. Use of any wearable that is not among the Approved Wearables is prohibited. In addition: (i) the only metric categories and/or system variables that Teams can use from Approved Wearables are those that were designated as ``Pass'' in the wearables validation reports provided to the parties by their jointly retained experts; and (ii) Teams must follow the safety directions of the jointly retained experts as provided to Teams in the Wearable Device Validation Reports. With respect to raw or unprocessed data exports or APIs from Approved Wearables (``Raw Data''), so long as such Raw Data are not provided through a dashboard or other visual within an Approved Wearable's software platform, the foregoing shall not prohibit Teams from (i) using Raw Data so long as the Raw Data is used in metric categories and/or system variables that were designated as ``Pass'' in the Wearable Device Validation Reports, or (ii) receiving Raw Data. If upon evaluation by the Committee, any of the foregoing devices are reviewed and are not approved by the Committee, Teams will be required to discontinue the use of such Wearables.
\item
  A Team may request a player to use in practice (or otherwise not in a game) on a voluntary basis a Wearable that has been approved by the Committee. A player may decline to use (or discontinue use of) a Wearable at any time. Before a Team could request that a player use an approved Wearable, the Team shall be required to provide the player a written, confidential explanation of: (i) what the device will measure; (ii) what each such measurement means; and (iii) the benefits to the player in obtaining such data.
\item
  A player will have full access to all data collected on him from approved Wearables. Members of the Team's staff may also have access to such data but it can be used only for limited purposes as set forth below. Data collected from a Wearable worn at the request of a Team may be used for player health and performance purposes and Team on-court tactical and strategic purposes only. The data may not be considered, used, discussed, or referenced for any other purpose such as in negotiations regarding a future Player Contract or other Player Contract transaction (e.g., a trade or waiver) involving the player. In a proceeding brought by the Players Association under the procedures set forth in Article XXXI, the Grievance Arbitrator will have authority to impose a fine of up to \$250,000 on any Team shown to have violated this provision.
\item
  To advance the shared goal of the NBA and the Players Association to promote player health and reduce injuries, and in light of the preference of the NBA that game use of Wearables be required, and the preference of the Players Association that Wearables not be required in games and instead be allowed to be worn in games on a voluntary basis only in connection with modified rules regarding commercialization, the NBA and Players Association will continue to discuss in good faith matters related to the use of wearable devices. Pending an agreement between the parties, Wearables may not be used in games, and no player data collected from a Wearable worn at the request of a Team may be made available to the public in any way or used for any commercial purpose.
\end{enumerate}

\hypertarget{nba-draft-combine.-1}{%
\section{NBA Draft Combine.}\label{nba-draft-combine.-1}}

\begin{enumerate}
\def\labelenumi{(\alph{enumi})}
\item
  Each year, the NBA shall organize and operate a Draft Combine prior to the NBA Draft. All players invited by the NBA to attend the Draft Combine shall be required to attend and participate in the following components of the Draft Combine (``Combine Components''):

  \begin{enumerate}
  \def\labelenumii{(\roman{enumii})}
  \tightlist
  \item
    Strength and agility tests, shooting drills, performance testing, and anthropometric measurements (e.g., height, wingspan) (five-on-five scrimmages or any other live action offense versus defense drill (e.g., half-court four-on-four or two-on-one) shall be optional for all players);
  \item
    League-directed: medical history information, medical testing (e.g., MRIs, echocardiograms, and laboratory tests, other than tests for controlled substances), medical examinations, and biomechanical and functional movement testing, including, for clarity, with respect to any of the foregoing in this subsection (ii), any medical examination in accordance with subsection (d) below and/or follow-up in accordance with subsection (e) below;
  \item
    Media circuit;
  \item
    Player development sessions;
  \item
    Team interviews; and
  \item
    Other tests and/or assessments.
  \end{enumerate}

  The NBA shall determine and establish the Combine Components above, provided that the performance testing contemplated in the foregoing Component (a)(i), the medical testing contemplated in the foregoing Component (a)(ii), and the foregoing Components (a)(iii)-(vi) shall be determined in consultation with the Players Association.
\item
  Notwithstanding the foregoing requirement to attend and participate in the Combine Components, any invited player who is physically unable to participate in one or more basketball activities (as set forth in subsection (a)(i) above), medical testing or biomechanical or functional movement testing (as set forth in subsection (a)(ii) above), or any other tests or assessments at the Draft Combine (as set forth in subsection (a)(vi) above) shall be (i) excused from participation in some or all of the applicable activities or tests at the time of the Draft Combine, (ii) required at the Draft Combine to complete the Combine Components that he is able to complete, and (iii) required subsequently to complete, by no later than the eleventh day before the Draft, the remaining Combine Components, as reasonably determined by the NBA unless he remains physically unable to do so. Any determination with respect to this subsection (b) shall be made by the NBA's medical director for the Draft Combine, who shall be required to consider any opinion timely provided by the player's treating physician.
\item
  The NBA may excuse an invited player from attending one or more days of the Draft Combine due to a reasonable excuse, as reasonably determined by the NBA (e.g., family tragedy, birth of a child, playing with a FIBA club that is still in season at the time of the Draft Combine). Any such player may be required subsequently to complete, by no later than the eleventh day before the Draft, Combine Components as reasonably determined by the NBA in consultation with the Players Association (e.g., by attending an NBA Global Camp or via individual assessments and examinations arranged by the NBA).
\item
  The NBA and Players Association shall agree annually on certain jointly-selected orthopedic medical specialists with expertise in foot and ankle, knee, spine, hip, and wrist/hand injuries to attend the Draft Combine, conduct medical examinations of particular players at the request of either the player or a Team, and prepare a report for each such player. Subject to the limits on Teams accessing information on certain players in accordance with subsection (g) below, as with other medical history information, testing, and examinations from Combine Components, the NBA shall make any such report(s) from an orthopedic medical specialist available to the player and to Teams via the file in respect of the player in the EMR.
\item
  Based on available medical information, including the results of medical testing at the Draft Combine, the NBA may require any player who was invited to the Combine to undergo, by no later than the eleventh day before the date of the Draft, reasonable and appropriate follow-up testing or examination after the Draft Combine, as determined by the NBA's medical director for the Draft Combine in consultation with the player's treating physician (if any).
\item
  A player shall fail to fulfill his obligation to participate in the Draft Combine in respect of a Draft, and shall therefore be ineligible to be selected in such Draft in accordance with Article X, Section 9 of the CBA, if he is invited by the NBA to attend the Draft Combine and, as reasonably determined by the NBA in consultation with the Players Association, fails to fully participate in the Combine Components in which the player is required to participate pursuant to subsections (a)-(e) above.
\item
  The NBA shall organize and operate an annual process that utilizes the following method (the ``Top-10 Formula'') for the purpose of developing a ranking of the top-10 players eligible in that year's Draft:

  \begin{enumerate}
  \def\labelenumii{\roman{enumii}.}
  \tightlist
  \item
    The NBA (after consultation with the Players Association) shall annually select no fewer than (a) two (2) publications with publicly-available pre-Draft rankings and (b) two (2) individuals with relevant basketball experience (each such individual, a ``Combine Player Ranker'') for the purpose of generating the composite ranking described below.
  \item
    The NBA shall utilize (a) from each such publication referenced in subsection (i) above, its publicly-available pre-Draft rankings, and (b) from each Combine Player Ranker, his or her ranking of the top-fifteen players eligible in that year's Draft.
  \item
    Any player who is ranked within the top-fifteen by one (1) publication and/or Combine Player Ranker, but not ranked within the top-fifteen by another publication and/or Combine Player Ranker shall, for purposes of computing the ranking of such other publication(s) and/or Combine Player Ranker(s), be given a ranking of sixteen (16). For clarity, pursuant to the foregoing sentence, multiple players may be given such ranking of sixteen (16) for any publication and/or Combine Player Ranker that does not rank the player within the top-fifteen (e.g., if two (2) or more players are ranked within the top-fifteen by one (1) publication and/or Combine Player Ranker, but not ranked as such by one (1) or more of the other publication(s) and/or Combine Player Ranker(s)).
  \item
    A composite ranking shall be determined by taking, for each player, the median ranking of each such publication's publicly-available ranking and each such Combine Player Ranker's individual ranking provided to the NBA (such median ranking, the ``Combine Player Ranking'').
  \item
    Each player whose Combine Player Ranking equals one of the ten (10) lowest numbers (i.e., where 1 is the lowest possible sum that can be generated via the Top-10 Formula and thus the highest possible ranking for any player) shall be considered a top-10 player eligible in that year's Draft. If two (2) or more players ranked first through tenth have the same Combine Player Ranking, then the median ranking of the publicly-available rankings from each publication referenced in subsection (g)(i) above shall be used to determine each such player's ranking (e.g., if two (2) players have the same Combine Player Ranking, and such Ranking is higher than that of five (5) other players, then the player with the lower median ranking of each such publication's publicly-available ranking will be ranked sixth and the player with the higher median ranking will be ranked seventh). If still tied, then each such publication's publicly-available ranking and each Combine Player Ranker's individual ranking for each such player shall be aggregated, and the player with the lower total sum will be considered the higher-ranked player, followed by the player with the higher total sum. If still tied, the NBA shall conduct a random drawing to determine each such player's ranking.
  \item
    If in any particular year covered by this Agreement the NBA determines that it is impracticable to calculate one or more of the rankings set forth in subsection (g)(i) above (e.g., due to the unavailability of a Combine Player Ranker, or if there are fewer than two (2) publications whose publicly-available pre-Draft rankings are determined by the NBA (after consultation with the Players Association) to be suitable for this purpose), the NBA may, after consultation with the Players Association, generate an output of the Top-10 Formula using as many of such publication(s)' and/or Combine Player Ranker(s)' individual ranking(s) as is reasonably practicable.
  \end{enumerate}

  The NBA shall finalize and provide to the Players Association the list of players eligible in that year's Draft who are ranked first through tenth per the Top-10 Formula. The NBA shall finalize such list during the period beginning with the deadline established by the NBA under Article X, Section 1(b)(ii)(G) of the CBA to qualify as an Early Entry player and prior to the earlier of the date on which (x) the NBA conducts a drawing among the Teams that did not participate in the playoffs in the Season immediately preceding that year's NBA Draft to determine the order of selection positions in such NBA Draft, or (y) the on-site process to gather players' medical history commences (i.e., the medical intake portion of the medical history referenced at subsection (a)(ii) above).

  The information gathered from such players' Combine Components set forth in subsection (a)(ii) above (and (a)(vi) above to the extent that any such test or assessment involves medical information in respect of a player) shall be made available by the NBA after it is gathered: (1) for the player ranked first, to Teams selecting first through tenth in that year's Draft; (2) for the players ranked second through sixth, to Teams selecting first through fifteenth in that year's Draft; (3) for the players ranked seventh through tenth, to Teams selecting first through twenty-fifth in that year's Draft; and (4) for all other players invited to the Draft Combine, all Teams. For clarity, any assignee Team that trades for a Draft selection position within the top-25 selection positions in that year's Draft shall subsequently be given access to the information gathered from the Combine Components set forth in subsection (a)(ii) above for each player associated with such Draft selection position. Beginning on the day after the conclusion of the Draft, Teams will no longer have access to such information for any player whom a Team did not select in the Draft (or whose Draft rights the Team does not hold).
\item
  Nothing in this Section 14 shall limit the right of the NBA, the Players Association, or a Team to refer a player eligible for a Draft, prior to that year's Draft, to a Fitness-to-Play Panel, in accordance with Article XXII, Section 11 of the CBA, if advised by a physician that the player is medically unable and/or medically unfit to perform his duties as a professional basketball player as a result of a potentially life-threatening injury, illness, or other medical condition and/or that performing such duties would likely create a materially elevated risk of death, paralysis, or other permanent spinal injury for the player. In any such case, the fact that the player was referred, the Panel's determination, and all medical information in the player's medical file that any member of the Panel deemed relevant to the injury, illness, or other medical condition for which the player was referred shall be made available to all Teams following the Panel's determination.
\item
  For clarity, nothing in this Section 14 shall be construed to limit in any way (i) the right of a Team to request that a player eligible for a Draft voluntarily participate in the administration of such activities described in subsection (a) above (e.g., in connection with visiting a Team's practice facility during the period between the Draft Combine and the Draft) or provide the Team with information, including information from the Combine Components described in subsection (a)(ii) above, or (ii) the right of a player to supplement medical information gathered from those Combine Components set forth in subsection (a)(ii) above with additional information that the NBA shall make available to all Teams via the EMR, subject to subsection (g) above.
\item
  \begin{enumerate}
  \def\labelenumii{(\roman{enumii})}
  \tightlist
  \item
    Teams may use the results of information gathered from such Combine Components set forth in subsection (a)(ii) above for Draft evaluation purposes only, and may not discuss any such results with representatives of any other Team (regardless of whether the other Team would otherwise have access to the same results); provided, however, that (1) medical staff from Teams entitled to access the player's information in accordance with subsection (g) above may discuss such results with medical staff of other Teams who performed the medical examination of the player at the Draft Combine or in connection with subsections (b), (c), or (e) above (for clarity, the only Team personnel who will be involved in such examinations will be those who are entitled to access the player's information in accordance with subsection (g) above); (2) a Team may discuss such results with representatives from other Teams entitled to access the player's information in accordance with subsection (g) above if the player provides written consent (with notice to the Players Association); and (3) nothing in this Section 14 shall limit any rights a Team has to use or disclose such results in respect of a player who is under contract with the Team or as to whom the Team holds exclusive Draft rights (e.g., following the Draft, a team disclosure of medical information in connection with a trade of a player's Draft rights). For clarity, a Team will not have violated this subsection (j)(i) to the extent its discussion involves only public information regarding a player.
  \item
    If the NBA has reason to believe that the confidentiality restriction set forth in subsection (j)(i) above has been violated, it shall advise the Players Association in a timely manner.
  \item
    If the Players Association determines that the confidentiality restriction set forth in subsection (j)(i) above has been violated, it may bring a proceeding under Article XXXII, Section 1 of the CBA before the System Arbitrator. Upon a finding by the System Arbitrator of a material violation, the System Arbitrator shall have the authority to impose on any Team found to have committed such violation a fine of up to \$1,000,000. In considering appropriate discipline for a violation, the System Arbitrator shall take into account all relevant factors, including, but not limited to, the impact of the violation on the player, the degree of care demonstrated by the Team, and any ill intent regarding the player.
  \end{enumerate}
\item
  The NBA will consult with the Players Association in good faith on (i) creating a list of non-exclusive jointly-recommended interview questions to provide to Teams each year prior to the Draft Combine, and (ii) any issues the Players Association raises relating to scheduling or operational details of the Draft Combine (e.g., setting the dates and location of the Draft Combine, elements of the player experience at the Draft Combine).
\item
  \begin{enumerate}
  \def\labelenumii{(\roman{enumii})}
  \tightlist
  \item
    Each player invited to the Draft Combine will be provided one (1) complimentary first class travel accommodations (except when such accommodations are not available) for himself and one family member to the market in which such Draft Combine is held, and (2) one (1) complimentary individual room in a group of hotel rooms reserved by the NBA for the Draft Combine. One certified agent who represents each such player participating in the Combine shall (x) be permitted to reserve one (1) room in such group of hotel rooms (at such agent's expense); and (y) receive an NBA credential at the Draft Combine to attend the on-court activities set forth in subsection (a)(i) above; provided that if the Players Association notifies the NBA that a player participating in the Draft Combine is represented by a second certified agent who does not already have an NBA credential to attend the on-court activities set forth in subsection (a)(i) above, such second agent shall receive such credential.
  \item
    Each player invited to the Draft Combine will be offered an NBAGL contract covering the season immediately following the Draft Combine. To benefit players, the NBA shall also undertake to provide new media opportunities at the Draft Combine for players who attend and participate in the Combine Components.
  \item
    Mental health and wellness programming, jointly created by the NBA and Players Association for players, will be included as part of the Pre-Draft Information Program presented at the Draft Combine.
  \end{enumerate}
\end{enumerate}

\hypertarget{exhibition-games-and-off-season-games-and-events}{%
\chapter{EXHIBITION GAMES AND OFF-SEASON GAMES AND EVENTS}\label{exhibition-games-and-off-season-games-and-events}}

\hypertarget{exhibition-games.-1}{%
\section{Exhibition Games.}\label{exhibition-games.-1}}

Subject to the provisions of Paragraph 2 of the Uniform Player Contract, players shall be required to participate in Exhibition games between an NBA Team and a non-member of the NBA at any location, within or outside the United States, subject to the following conditions:

\begin{enumerate}
\def\labelenumi{(\alph{enumi})}
\tightlist
\item
  The NBA shall supervise the arrangements made with respect to tournaments or series conducted outside the United States and the accommodations provided to NBA players participating in such foreign tournaments or series.
\item
  The NBA shall use its best efforts to establish an Exhibition game schedule pursuant to which excessive travel will be avoided and reasonable periods of time between games will be allotted.
\item
  In any year in which it is played, the annual Basketball Hall of Fame Exhibition game shall be considered as one of the six (6) Exhibition games prior to the Regular Season referred to in Paragraph 2 of the Uniform Player Contract.
\end{enumerate}

\hypertarget{inter-squad-scrimmage.}{%
\section{Inter-squad Scrimmage.}\label{inter-squad-scrimmage.}}

In addition to the Exhibition games provided for by Paragraph 2 of the Uniform Player Contract, and during each of the playoff series conducted during the term of this Agreement, any Team that qualifies for the playoffs but is not required to participate in the first round thereof may arrange and require its players to participate in one inter-squad game or scrimmage with another similarly-situated Team, provided that such game or scrimmage is not open to members of the general public.

\hypertarget{off-season-basketball-events.}{%
\section{Off-Season Basketball Events.}\label{off-season-basketball-events.}}

\begin{enumerate}
\def\labelenumi{(\alph{enumi})}
\tightlist
\item
  No player may play in any public off-season basketball game, summer league, or public exhibition or competition of basketball skills (e.g., a slam dunk contest or a ``tour'' organized by an NBA business partner) (each, a ``Basketball Event'') unless such Basketball Event is approved in writing by the NBA for NBA player participation and complies with the terms and conditions of this Section 3. The NBA will consider an off-season Basketball Event for approval only if a request for such approval is submitted in writing to the NBA, and only if the arrangements made with respect to any such off-season Basketball Event are confirmed in writing to the NBA and satisfy the following requirements, in addition to such other reasonable requirements as the NBA may impose:

  \begin{enumerate}
  \def\labelenumii{(\roman{enumii})}
  \tightlist
  \item
    \textbf{\emph{General Requirements .}}

    \begin{enumerate}
    \def\labelenumiii{(\arabic{enumiii})}
    \tightlist
    \item
      The Basketball Event takes place on or after July 1, but in no event later than September 15 (or, in the case of a summer league, September 1);
    \item
      Prior to the Basketball Event, each participating player receives the express written consent of his Team to participate in the Basketball Event;
    \item
      The person(s) organizing the Basketball Event obtains disability insurance for the benefit of each participating player's Team, in an amount acceptable to the NBA (provided, however, that this requirement shall not apply to summer leagues); and
    \item
      The names and logos of the NBA and/or any NBA Team are not used or referred to in connection with the Basketball Event, unless the NBA provides express written authorization for such use.
    \end{enumerate}
  \item
    \textbf{\emph{Additional Charitable Game Requirements .}} The NBA will consider an off-season charitable game for approval only if, in addition to the general requirements set forth in Section 3(a)(i) above and such other reasonable requirements as the NBA may impose, the arrangements made with respect to such charitable game also satisfy the following:

    \begin{enumerate}
    \def\labelenumiii{(\arabic{enumiii})}
    \tightlist
    \item
      The Players Association approves the game (which approval shall not be unreasonably withheld);
    \item
      All proceeds from the sale of tickets to the game and other sources of revenue from the game (e.g., sponsorship revenue) less reasonable expenses incurred to conduct the game are used for charitable purposes;
    \item
      The game is officiated by NBA referees assigned by the NBA to officiate the game. The person or entity organizing the game will be responsible for paying the officiating fees and the actual expenses incurred for the referees' lodging and transportation to and from the referees' homes to the site of the game;
    \item
      There is at least one (1) NBA Team trainer and at least one (1) physician present at the game;
    \item
      The name or likeness of an NBA player is not used, or referred to, in advertisements or promotions for or related to the game, except that if the organizer of the game is an NBA player, such organizer-player's name or likeness may be used, or referred to, in such advertisements or promotions;
    \item
      Only current or former professional basketball players participate in the game;
    \item
      The game is not accompanied by an exhibition or competition of basketball skills (such as a slam dunk contest), unless such exhibition or competition has been separately approved in writing by the NBA and the Players Association;
    \item
      Participating players are not paid or compensated (in excess of per diem and actual reasonable expenses incurred in traveling to and participating in the game);
    \item
      The organizer guarantees that the game will produce at least \$100,000 for charity, and, if directed by the NBA and the Players Association, the organizer (or a third party if the organizer itself is a charity) posts security for such amount in a form satisfactory to the NBA and the Players Association which grants the NBA and/or the Players Association the right to sue to recover such amount for the benefit of the charity;
    \item
      The game is played in the United States or Canada; and
    \item
      The organizer agrees to provide the NBA and the Players Association with an audited statement of revenues and expenses, in a form acceptable to the NBA and the Players Association, within sixty (60) days following the game.
    \end{enumerate}
  \item
    \textbf{\emph{Additional Summer League Requirements .}} The NBA will consider an off-season summer league for approval only if, in addition to the general requirements set forth in Section 3(a)(i) above and such other reasonable requirements as the NBA may impose, the arrangements made with respect to each summer league game in which an NBA player participates also satisfy the following:

    \begin{enumerate}
    \def\labelenumiii{(\arabic{enumiii})}
    \tightlist
    \item
      Participating players are not paid or compensated (except as provided under Section 4(c) below);
    \item
      NBA players do not participate in an exhibition or competition of basketball skills (such as a slam dunk contest), unless such exhibition or competition has been separately approved in writing by the NBA;
    \item
      There is at least one (1) trainer or at least one (1) physician or other emergency medical personnel present at the game; and
    \item
      The game is played in the United States or Canada.
    \end{enumerate}
  \end{enumerate}
\item
  Notwithstanding any other terms of this Section 3, and without limiting the right of the NBA to approve all arrangements of a proposed Basketball Event, the NBA may, in its sole discretion, require, as a condition of its approval of a Basketball Event (other than a charitable game or summer league), that the Basketball Event organizer pay an appropriate fee to the NBA prior to the commencement of the Basketball Event.
\item
  For purposes of this Section 3, off-season games in which an NBA player participates on behalf of his national basketball federation as part of an international FIBA competition (e.g., the Olympics and FIBA Basketball World Cup), and the preparatory Exhibition games in connection therewith, are excluded from the definition of ``Basketball Event''; provided, however, that such exclusion shall not apply to any preparatory Exhibition game (other than games involving the U.S. national team) played and/or telecast in the United States.
\item
  Notwithstanding anything to the contrary in this Agreement, a Veteran Free Agent remains subject to the provisions of this Section 3 until the September 1 following the last Season of his Player Contract; provided, however, that any such Veteran Free Agent shall be permitted to sign a contract with and play in basketball games for a team in a professional basketball league other than the NBA beginning on the July 1 immediately following such Season (or prior to July 1 if approved in writing by the NBA).
\item
  The NBA shall have the exclusive right to (and to authorize third parties to) telecast or broadcast by radio any Basketball Event (in whole or in part) that is approved for NBA player participation in accordance with this Section 3.
\item
  Notwithstanding anything else in this Article XXIII, the NBA, in considering and acting upon a request for approval of a summer league, charity game, or other Basketball Event, does not consider or apply safety requirements for such leagues, games, or events.
\end{enumerate}

\hypertarget{summer-leagues.}{%
\section{Summer Leagues.}\label{summer-leagues.}}

\begin{enumerate}
\def\labelenumi{(\alph{enumi})}
\tightlist
\item
  No NBA Team may simultaneously enroll more than four (4) Veterans in any summer basketball league during an off-season. For purposes of this Section 4(a), the following players are not considered Veterans:

  \begin{enumerate}
  \def\labelenumii{(\roman{enumii})}
  \tightlist
  \item
    a player who has never signed a Player Contract or whose first Player Contract begins with the Season immediately following the off-season in which such summer league is to be conducted;
  \item
    a player not under contract to an NBA Team at the time he enrolls in such summer league;
  \item
    a player under contract to an NBA Team but who missed twenty-five (25) or more of the Team's games during the Regular Season immediately preceding such off-season due to injury or illness; and
  \item
    a player who played for a team in the NBAGL or any other U.S.-based professional league during all, or any portion, of the Regular Season immediately preceding such off-season.
  \end{enumerate}
\item
  Prior to playing in a summer basketball league, each player who is under contract with a Team for the following Season shall be provided by his Team, and requested to sign a ``Notice to Veteran Players Concerning Summer Leagues'' in the form attached hereto as Exhibit E.
\item
  The only compensation that may be paid by a Team or any person or entity affiliated with a Team to a player participating in a summer basketball league is a reasonable expense allowance for: (i) meals, but no greater than that set forth in Article III, Section 2; (ii) lodging; and (iii) transportation to and from the player's home to the site of the summer league, and to and from the site of the player's lodging during the summer league to the site of summer-league-related activities. In addition, the Team may purchase a disability insurance policy for the player covering the term of the applicable summer league.
\item
  No Team shall schedule, and no player shall participate in, a summer basketball league that is scheduled to extend, or does in fact extend, past September 1 of any calendar year.
\end{enumerate}

\hypertarget{prohibition-of-no-trade-contracts}{%
\chapter{PROHIBITION OF NO-TRADE CONTRACTS}\label{prohibition-of-no-trade-contracts}}

\hypertarget{general-limitation.}{%
\section{General Limitation.}\label{general-limitation.}}

No Player Contract may contain any prohibition or limitation of an NBA Team's right to assign such Contract to another NBA Team.

\hypertarget{exceptions-to-general-limitation.}{%
\section{Exceptions to General Limitation.}\label{exceptions-to-general-limitation.}}

Notwithstanding the provisions of Section 1 of this Article XXIV:

\begin{enumerate}
\def\labelenumi{(\alph{enumi})}
\tightlist
\item
  A Player Contract may contain (in Exhibit 4 to such Player Contract) a provision entitling a Player to earn Compensation if the player's Uniform Player Contract is traded (``trade bonus'') subject to the following:

  \begin{enumerate}
  \def\labelenumii{(\roman{enumii})}
  \tightlist
  \item
    A trade bonus shall be payable only the first time that the Contract is traded; provided, however, that if a Contract is signed in connection with an agreement to trade the Contract in accordance with Article VII, Section 8(e) and the Contract contains a trade bonus, the bonus shall not apply to such initial trade but shall instead be payable only the second time the Contract is traded.
  \item
    A trade bonus shall not exceed fifteen percent (15\%) of the Base Compensation remaining to be earned by the player pursuant to the Contract at the time of the trade (excluding an Option Year if not yet exercised).
  \item
    The only allowable amendments to Exhibit 4 to a Uniform Player Contract shall be as follows:

    \begin{enumerate}
    \def\labelenumiii{(\Alph{enumiii})}
    \tightlist
    \item
      The specification of the amount of the trade bonus to be paid to the player, expressed as either (1) a specified percentage of the Base Compensation remaining to be earned under the Contract at the time of the trade (excluding an Option Year if not yet exercised), or (2) a specified dollar amount not to exceed a specified percentage of Base Compensation remaining to be earned under the Contract at the time of the trade (excluding an Option Year if not yet exercised).
    \item
      If a Player Contract contains a trade bonus that has not previously been earned:

      \begin{enumerate}
      \def\labelenumiv{(\arabic{enumiv})}
      \tightlist
      \item
        In connection with an Extension (other than pursuant to an agreement to trade the extended Contract in accordance with Article VII, Section 8(e)), to: (a) modify the amount of the trade bonus to be paid to the player (subject to Sections 2(a)(ii) and 2(a)(iii)(A) above), or (b) provide that the trade bonus provision will not be applicable to the extended term of the Contract;
      \item
        In connection with an Extension pursuant to an agreement to trade the extended Contract in accordance with Article VII, Section 8(e), to: (a) reduce the amount of the trade bonus to be paid to the player (subject to Sections 2(a)(ii) and 2(a)(iii)(A) above); or (b) provide that the trade bonus provision will not be applicable to the extended term of the Contract; or
      \item
        In connection with the trade of a Player Contract (other than pursuant to an agreement to trade an extended Contract in accordance with Article VII, Section 8(e)), to reduce the amount of the trade bonus to be paid to the player (subject to Sections 2(a)(ii) and 2(a)(iii)(A) above).
      \end{enumerate}
    \end{enumerate}
  \item
    A Contract that does not contain a trade bonus when signed cannot be amended to add one except that: (A) if the Contract is extended (other than pursuant to an agreement to trade the extended Contract in accordance with Article VII, Section 8(e)), the Contract may be amended simultaneously to provide for a trade bonus that will be payable only the first time that the Contract is traded following the signing of the Extension (and not as a result of any subsequent trade), and (B) if the Contract is extended pursuant to an agreement to trade the extended Contract in accordance with Article VII, Section 8(e), the Contract may be amended simultaneously to provide for a trade bonus that shall not apply to such initial trade but shall instead be payable only if the extended Contract is traded a second time (and not as a result of any subsequent trade).
  \item
    If, in connection with an Extension, a Contract is amended to provide that a trade bonus that has not been previously earned will not be applicable to the extended term (pursuant to Section 2(a)(iii)(B)(1)(b) or 2(a)(iii)(B)(2)(b) above), the extension must include a replacement Exhibit 4 to the Contract with the same terms as the original Exhibit 4, but also providing that ``The foregoing trade bonus shall not be applicable with respect to the extended term of this Contract.'' To illustrate the foregoing, assume that a player and Team agree at the time of signing of an Extension that the trade bonus contained in the original Contract shall not be applicable to the extended term. In such case: (A) if the player is first traded under the Contract during the remainder of the original term of the Contract (i.e., prior to the first year of the extended term), then the player's trade bonus shall be calculated based solely on the Base Compensation remaining to be earned by the player pursuant to the original term of the Contract (and not on any Base Compensation payable to the Player in respect of the extended term); and (B) if the player is first traded under the Contract at any time during the extended term, then the trade bonus would not apply to such initial trade or any subsequent trade of the Contract during the extended term.
  \item
    In no event shall a trade bonus in a Contract be payable more than once.
  \end{enumerate}
\item
  A Player Contract entered into by a player who has eight (8) or more Years of Service in the NBA and who has rendered four (4) or more Years of Service for the Team entering into such Contract may contain a prohibition or limitation of such Team's right to trade such Contract to another NBA Team.
\end{enumerate}

\hypertarget{limitation-on-deferred-compensation}{%
\chapter{LIMITATION ON DEFERRED COMPENSATION}\label{limitation-on-deferred-compensation}}

\hypertarget{general-limitation.-1}{%
\section{General Limitation.}\label{general-limitation.-1}}

No Uniform Player Contract may provide for Deferred Compensation for any Season that exceeds twenty-five percent (25\%) of the player's Compensation for such Season.

\hypertarget{attribution.}{%
\section{Attribution.}\label{attribution.}}

All Player Contracts shall specify the Season(s) to which any Deferred Compensation is attributable.

\hypertarget{team-rules}{%
\chapter{TEAM RULES}\label{team-rules}}

\hypertarget{establishment-of-team-rules.}{%
\section{Establishment of Team Rules.}\label{establishment-of-team-rules.}}

Each Team may maintain or establish rules with which its players shall comply at all times, whether on or off the playing floor; provided, however, that such rules are in writing, are reasonable, and do not violate the provisions of this Agreement or the Uniform Player Contract.

\hypertarget{notice.}{%
\section{Notice.}\label{notice.}}

Any rule(s) established by a Team pursuant to Section 1 above shall be provided to the Players Association prior to the distribution of such rule(s) to that Team's players.

\hypertarget{grievances-challenging-team-rules.}{%
\section{Grievances Challenging Team Rules.}\label{grievances-challenging-team-rules.}}

The Players Association may file a Grievance challenging the reasonableness of a rule established by a Team pursuant to Section 1 above, and the Team's imposition of discipline on a player for a violation of such rule, within thirty (30) days from the date upon which the imposition of such discipline on the player became known or reasonably should have become known to the player. No ruling by the Grievance Arbitrator finding a Team rule unreasonable may be applied retroactively as to any player other than the player on whose behalf the Grievance was filed.

\hypertarget{right-of-set-off}{%
\chapter{RIGHT OF SET-OFF}\label{right-of-set-off}}

\hypertarget{set-off-calculation.}{%
\section{Set-off Calculation.}\label{set-off-calculation.}}

\begin{enumerate}
\def\labelenumi{(\alph{enumi})}
\item
  When a Team (``First Team'') terminates a Player Contract (``First Contract'') in circumstances where the First Team, following the termination, continues to be liable for unearned Base Compensation (i.e., unearned as of the date of the termination) called for by the First Contract (including any unearned Deferred Base Compensation), the First Team's liability for such unearned Base Compensation shall be reduced pro rata by a portion of the compensation earned by the player (for services as a player) from any professional basketball team(s) (the ``Subsequent Team(s)'') during each Salary Cap Year covered by the term of the First Contract (including, but not limited to, compensation earned but not paid during such period). The amount of the reduction in the First Team's liability (the ``set-off'' amount) shall be calculated for each Salary Cap Year covered by the term of the First Contract as follows:

  \begin{itemize}
  \item
    STEP 1: Calculate the total compensation earned by the player (for services as a player) from the Subsequent Team(s) during the Salary Cap Year.
  \item
    STEP 2: Subtract from the result in Step 1 (i) if the player had zero (0) Years of Service at the time the First Contract was terminated, the Minimum Annual Salary applicable to such player for the Salary Cap Year in which the First Contract was terminated, or (ii) if the player had one (1) or more Years of Service at the time the First Contract was terminated, the Minimum Annual Salary applicable to a player with one (1) Year of Service for the Salary Cap Year in which the First Contract was terminated.
  \item
    STEP 3: If the result in Step 2 is zero (0) or a negative amount, there is no reduction in the First Team's liability for unearned Base Compensation in respect of the relevant Salary Cap Year. If the result in Step 2 is a positive amount, the reduction in the First Team's liability for unearned Base Compensation in respect of the relevant Salary Cap Year shall equal fifty percent (50\%) of such amount.
  \end{itemize}

  Notwithstanding anything to the contrary in this Article XXVII, a Team shall not be required to enforce its set-off right against a player in respect of compensation earned by the player from any non-NBA Subsequent Team(s). The First Team may require that the player provide the First Team with evidence (such as a copy of the player's new contract) of the compensation to be earned by the player in connection with his services for any Subsequent Team(s).
\item
  For the purposes of this Article XXVII, (i) a ``professional basketball team'' shall mean any team in any country that pays money or compensation of any kind to a basketball player for rendering services to such team (other than a reasonable stipend limited to basic living expenses); and (ii) ``compensation'' earned by a player shall include all forms of compensation (including, without limitation, any non-cash compensation) other than benefits comparable to the type of benefits (e.g., medical and dental insurance) provided to an NBA player in accordance with Article IV above, travel and moving expenses, and any car and housing provided temporarily by a professional basketball team to the player during the period of time for which the player renders services to such team. Notwithstanding anything to the contrary in this Article XXVII, when a player receives compensation from a non-NBA Subsequent Team on a net-of-tax basis, then for purposes of calculating the amount of set-off to which the NBA Team is entitled pursuant to this Article XXVII, such compensation from the non-NBA Subsequent Team shall be deemed to equal the net-of-tax compensation divided by 0.65 (reflecting a deemed thirty-five percent (35\%) tax rate); provided, however, that such adjustment to the player's compensation from the non-NBA Subsequent Team shall not be made, or shall be modified accordingly, if the player can establish that taxes in respect of the player's compensation calculated under this provision were not paid, or exceed the actual amount paid, by the player's non-NBA Subsequent Team.
\item
  Without limiting any other rights the First Team has, in the event a player's Compensation is reduced pursuant to this Article XXVII and the Team is unable to effect all or a portion of the reduction through payroll deductions, the NBA shall have the right to direct any Subsequent Team that is an NBA Team to withhold any unrecouped amounts from the player's Compensation under his new Uniform Player Contract and remit such amounts to the First Team. To the extent such remedy is insufficient to effect a full recoupment of the set-off amount, the NBA and Players Association shall negotiate in good faith to agree on such supplemental measures as are appropriate to effect such recoupment.
\end{enumerate}

\hypertarget{successive-terminations.}{%
\section{Successive Terminations.}\label{successive-terminations.}}

In the event of successive terminations by NBA Teams of Player Contracts involving the same player, the Team first to terminate shall be entitled to the right of set-off provided for by this Article XXVII until its Compensation liability has been eliminated in its entirety, and the right of set-off shall then pass in order to the Team(s) terminating any subsequent Contract(s).

\hypertarget{deferred-compensation.}{%
\section{Deferred Compensation.}\label{deferred-compensation.}}

In calculating the amount of set-off to which a Team may be entitled pursuant to this Article XXVII, the unearned Deferred Compensation payable to a player for or with respect to a period covered by the terminated Contract shall be discounted on an annual basis by a percentage equal to the prime rate reported in the ``Money Rates'' column or any successor column of \emph{The Wall Street Journal} and in effect at the time the agreement providing for such Deferred Compensation was made.

\hypertarget{waiver-of-set-off-right.}{%
\section{Waiver of Set-off Right.}\label{waiver-of-set-off-right.}}

A Team and a player may agree in an amendment to an already-existing Player Contract to modify or eliminate the set-off right provided in this Article XXVII, but only pursuant to and to the extent allowed by Article II, Section 3(p).

\hypertarget{stretched-protected-salary.}{%
\section{Stretched Protected Salary.}\label{stretched-protected-salary.}}

\begin{enumerate}
\def\labelenumi{(\alph{enumi})}
\tightlist
\item
  In the event (i) a Team terminates a Player Contract and the payment of the player's protected Compensation for any remaining Salary Cap Year(s) under the First Contract is stretched in accordance with Article II, Section 4(k) (the ``mandatory stretch provision''), and (ii) the player subsequently earns compensation from another professional basketball team triggering a right of set-off under this Article XXVII, the amount of set-off to which the First Team may be entitled shall be calculated based on the unearned Base Compensation in respect of each Salary Cap Year covered by the term of the First Contract as provided in such Contract (and not with regard to how such protected Base Compensation amounts are payable to the player pursuant to the mandatory stretch provision). The set-off amount in respect of each remaining Salary Cap Year under the First Contract in which the related unearned Base Compensation is stretched in accordance with the mandatory stretch provision shall be allocated such that each of the player's stretched protected Compensation payments in respect of the applicable Salary Cap Year are reduced on an equal basis over the applicable stretch period (i.e., for the first Salary Cap Year with respect to which a player's protected Compensation is stretched, over the entire stretch period, and for any subsequent Salary Cap Years, over the remaining stretch period). In no event shall a Team be entitled to set-off under a First Contract in respect of compensation earned by a player (for services as a player) from a Subsequent Team(s) during a Salary Cap Year occurring after the term of the First Contract.
\item
  In the event the First Team elects to stretch the player's Salary under the First Contract for Salary Cap purposes in accordance with Article VII, Section 7(d)(6) (or had, prior to the effective date of this Agreement, made such election in accordance with Article VII, Section 7(d)(6) of the 2017 CBA), then the set-off amount in respect of each remaining Salary Cap Year covered by the term of the First Contract that is stretched for Salary Cap purposes in accordance with Article VII, Section 7(d)(6) shall be allocated equally to reduce the player's re-attributed Salary amounts over the applicable stretch period in the manner described in Section 5(a) above.
\item
  The following examples are for clarity:

  \begin{enumerate}
  \def\labelenumii{(\Alph{enumii})}
  \setcounter{enumii}{23}
  \tightlist
  \item
    Assume (i) a player has protected Compensation of \$3 million in respect of the 2023-24 Season and is being paid by the First Team at a rate of \$1 million over three (3) Seasons in accordance with the mandatory stretch provision and (ii) the amount of set-off to which the First Team is entitled under this Article XXVII with respect to the 2023-24 Season is \$600,000, then (1) the \$600,000 set-off amount would be allocated to each of the three (3) Seasons at a rate of \$200,000 per Season and (2) the \$200,000 set-off amount for each Season would be deducted in equal installments from each of the player's protected Compensation payments each Season.
  \item
    Assume: (i) a player has remaining protected Compensation of \$9 million (\$3 million each for the 2023-24, 2024-25, and 2025-26 Seasons, respectively) under the First Contract; (ii) the First Team requests waivers on the player on September 5, 2023 and the First Contract is terminated on September 7, 2023; (iii) the player later signs a Player Contract with Subsequent Team A that provides for a term covering the 2023-24 through 2024-25 Seasons; and (iv) the set-off amount to which the First Team is entitled under this Article XXVII in respect of the player's Contract with Subsequent Team A is \$600,000 in respect of the 2023-24 Season and \$600,000 in respect of the 2024-25 Season. (There is no set-off amount under the First Contract in respect of the 2025-26 Season given these facts because the term of the Contract with Subsequent Team A does not cover the 2025-26 Season.) Under these facts: (1) with respect to the 2023-24 Season, the player's \$3 million of protected Compensation under the First Contract would be reduced by the applicable \$600,000 set-off amount and his reduced protected Compensation amount of \$2.4 million would be payable in accordance with the payment schedule set forth in the First Contract; (2) the player's \$6 million of protected Compensation under the First Contract in respect of the 2024-25 and 2025-26 Seasons would (absent a right of set-off pursuant to this Article XXVII) be paid by the First Team at a rate of \$1.2 million per Season over five (5) Seasons in accordance with the mandatory stretch provision; and (3) as a result of the \$600,000 set-off amount to which the First Team is entitled in respect of the 2024-25 Season, the \$1.2 million stretched protected Compensation payments described in (2) above would each be reduced to \$1.08 million (i.e., by \$120,000 per Season over the five-Season stretch period covering the 2024-25 through 2028-29 Seasons).
  \item
    Assume the same facts as in example (Y) above and that on October 1, 2025, the player signs a Player Contract with Subsequent Team B covering the 2025-26 Season and the set-off amount to which the First Team is entitled under this Article XXVII in respect of such Contract is \$500,000. In such case, the player's aggregate then-remaining protected Compensation would be further reduced by the additional \$500,000 set-off amount such that the player's remaining stretched protected Compensation payments that would (absent this additional right of set-off pursuant to this Article XXVII) be paid by the First Team at a rate of \$1.08 million per Season over the remaining four (4) Seasons of the mandatory stretch period would each be reduced to \$955,000 (i.e., by \$125,000 per Season over the four-Season remaining stretch period covering the 2025-26 through 2028-29 Seasons.
  \end{enumerate}
\end{enumerate}

\hypertarget{media-rights}{%
\chapter{MEDIA RIGHTS}\label{media-rights}}

\hypertarget{league-rights.}{%
\section{League Rights.}\label{league-rights.}}

The Players Association agrees that the NBA, all League-related entities (including, but not limited to, NBA Properties, Inc.~and NBA Media Ventures, LLC) that generate BRI, and NBA Teams have the right during and after the term of this Agreement to use, exhibit, distribute, or license any performance by the players, under this Agreement or the Uniform Player Contract, in any or all media, formats or forms of exhibition and distribution, whether analog, digital, or other, now known or hereafter developed, including, but not limited to, print, tape, disc, computer file, radio, television, motion pictures, other audio-visual and audio works, Internet, broadband platforms, mobile platforms, applications, and other distributions platforms (collectively, ``Media'').

\hypertarget{no-suit.}{%
\section{No Suit.}\label{no-suit.}}

The Players Association, for itself and present and future NBA players, covenants not to sue (or finance any suit against) the NBA, all League related entities (including NBA Properties, Inc.~and NBA Media Ventures, LLC) that generate BRI, and all NBA Teams, or, any of their respective past, present, and future owners (direct and indirect) acting in their capacity as owners of any of the foregoing entities, officers, directors, trustees, employees, agents, attorneys, licensees, successors, heirs, administrators, executors, and assigns, with respect to the use, exhibition, distribution, or license, in any or all Media, of any performances by any player rendered under this Agreement or prior collective bargaining agreements, or under Player Contracts made pursuant thereto; provided, however, that this Section 2 shall not apply to any Endorsement, as defined in Section 3 below, any Unauthorized Sponsor Promotion, as defined in Paragraph 14(e) of the Uniform Player Contract, or any action of the Players Association pursuant to Section 3 (f) below.

\hypertarget{unauthorized-endorsementsponsor-promotion.}{%
\section{Unauthorized Endorsement/Sponsor Promotion.}\label{unauthorized-endorsementsponsor-promotion.}}

\begin{enumerate}
\def\labelenumi{(\alph{enumi})}
\tightlist
\item
  Section 1 above does not confer any right or authority for the NBA, any League related entity or any NBA Team to (i) use, or authorize any third party to use, any performance by a player in any way that constitutes an unauthorized endorsement by such player of a third party brand, product or service (``Endorsement''), or (ii) authorize any third party to use any performance by a player in any way that constitutes an Unauthorized Sponsor Promotion as defined in Paragraph 14(e) of the Uniform Player Contract.
\item
  For purposes of clarity, and without limitation: (i) it shall not be an Endorsement for the NBA, a League-related entity, or an NBA Team to use, or authorize others to use, including, without limitation, in third party advertising and promotional materials, footage and photographs of a player's participation in NBA games or other NBA events that do not unduly focus on, feature, or highlight, such player in a manner that leads the reasonable consumer to believe that such player is a spokesman for, or promoter of, a third-party commercial product or service; provided that the preceding sentence is independent of and is not relevant to determining whether a use is or is not an Unauthorized Sponsor Promotion; and (ii) any use of a player's Player Attributes that has been expressly authorized by the player (not including the Uniform Player Contract) shall not be an unauthorized Endorsement or an Unauthorized Sponsor Promotion.
\item
  Any dispute regarding whether a use of any performance by a player is or is not an Unauthorized Sponsor Promotion shall be determined by the expedited System Arbitration process described in Paragraph 14(f) of the Uniform Player Contract.
\item
  For purposes of clarity, nothing in this Agreement or the Uniform Player Contract shall limit the rights of the NBA, all League-related entities that generate BRI, and NBA Teams to provide, and authorize others to provide, advertising and promotional opportunities within NBA games or NBA or Team events and NBA-related or Team- related content; it being understood that nothing in this sentence is intended to authorize the NBA, any League-related entity or any NBA Team to use, or authorize any third party to use, any Player Attributes in any way (w) that constitutes an unauthorized Endorsement, (x) in the creative elements incorporated into such advertising executions that constitute an Unauthorized Sponsor Promotion, or (y) in the creative elements in promotional opportunities that are not Promotional Enhancements that are Unauthorized Sponsor Promotion. For purposes of the foregoing, examples of ``advertising'' include 30-second commercials, video pre-rolls and courtside signage.
\item
  Nothing in Section 3(d)(x) or (y) above shall limit the right of a telecaster or distributor of NBA games, NBA or Team events, or NBA-related or Team-related content to use, or authorize others to use, third party Promotional Enhancements in telecasts or other distribution of such games, events, or content in accordance with this Section 3(e). For purposes of this Section 3(e), ``Promotional Enhancements'' means: (i) virtual images, graphics, and/or text that are superimposed on the video and/or audio depiction of the NBA game, NBA or Team event, or NBA-related or Team-related content; (ii) non-virtual signage or other physical displays otherwise visible in the telecast or other distribution of the NBA game, NBA or Team event, or NBA-related or Team-related content (for purposes of clarity, clauses (i) and (ii) above do not include still images except in game and program telecasts); and (iii) other promotional opportunities for Telecasters (as defined in Paragraph 14(f) of the Uniform Player Contract) consistent with past practice as permitted under the 2017 CBA. Examples of Promotional Enhancements include branded backboard slide-outs, branded feature trackers, sponsored starting lineups, branded virtual lineups, virtual courtside signage, virtual court signage, branded statistical presentations, studio show backdrops, branded halftime desk signage, a sponsored ``Top Plays'' feature, and a sponsored ``audio drop-in'' mention. Creative elements incorporated into virtual signage Promotional Enhancements are not authorized under this Section if they otherwise are Unauthorized Sponsor Promotion.
\item
  Notwithstanding the foregoing, in addition to any other rights the Players Association may have, (A) if a telecaster or other distributor of NBA games, NBA or Team events, or NBA-related or Team-related content uses, or authorizes others to use, Player Attributes in creative elements within promotional opportunities in telecasts or other distribution of such games, events or content (i) in a manner that (x) is not covered by Section 3(e)(iii) above, and (y) unduly promotes the products or services of a sponsor, and (ii) the promotion of the sponsor's products or services within such promotional opportunity is more prominent than the NBA content, to which it relates, taken as a whole, then the Players Association shall notify the NBA in writing, and (B) the NBA shall have a period of fifteen (15) days to cause the telecaster or distributor to cease or modify such creative elements (``Cure''). If the NBA fails to Cure pursuant to the preceding sentence, then the Players Association may sue the NBA for any resulting damages to the Players Association's commercial group licensing business, with the NBA responsible for the violation and such damages even if the NBA did not authorize such promotional opportunity.
\item
  For purposes of clarity, nothing contained in this Article XXVIII or in Paragraph 14 of the Uniform Player Contract shall prohibit the inclusion of a sponsor's name and/or logo on a jersey patch, and any depiction of a player wearing a jersey that includes such a jersey patch shall not, by reason of the jersey patch alone, constitute an unauthorized Endorsement, an Unauthorized Sponsor Promotion, or a violation of Section 3(f) above.
\end{enumerate}

\hypertarget{reservation-of-rights.}{%
\section{Reservation of Rights.}\label{reservation-of-rights.}}

The Players Association expressly reserves its rights to bargain collectively on the subject described in Section 1 above at the expiration of this Agreement. Such reservation shall not, however, preclude the NBA from contending that the subject described in Section 1 above is not a mandatory subject of collective bargaining. The right of the NBA, League related entities, and NBA Teams described in Section 1 above is in addition to, and shall not limit nor be deemed to limit, derogate from or otherwise prejudice, any and all rights that any one or all of them have heretofore possessed or enjoyed, do now possess or enjoy or may hereafter possess or enjoy.

\hypertarget{miscellaneous}{%
\chapter{MISCELLANEOUS}\label{miscellaneous}}

\hypertarget{active-list.}{%
\section{Active List.}\label{active-list.}}

Each Team agrees to have at least twelve (12) and no more than fifteen (15) players on its Active List and to have a minimum of eight (8) players on the bench for all Regular Season games; provided, however, that for no more than (a) two (2) consecutive weeks at a time, and (b) a total of twenty-eight (28) days, a Team may have eleven (11) players on its Active List. During the period from the day following the last day of the Season until the day before the first day of the following Regular Season, the maximum number of players (including Two-Way Players) that a Team may carry on its Active List shall be twenty-one (21). Players on the Inactive List and Two-Way List shall be transferred to the Active List on the day following the last day of the Season.

\hypertarget{roster-size.}{%
\section{Roster Size.}\label{roster-size.}}

\begin{enumerate}
\def\labelenumi{(\alph{enumi})}
\tightlist
\item
  During the period from the first day of the Regular Season through the last day of the Regular Season (or, for Teams that qualify for the ``postseason'' (as defined below), through the Team's last game of the Season), each Team agrees to have either fourteen (14) or fifteen (15) players, in aggregate, on its Active List and Inactive List.
\item
  Notwithstanding Section 2(a), during the Regular Season a Team may have:

  \begin{enumerate}
  \def\labelenumii{(\roman{enumii})}
  \tightlist
  \item
    Twelve (12) or thirteen (13) players, in aggregate, on its Active List and Inactive List for no more than (A) two (2) consecutive weeks at a time, and (B) a total of twenty-eight (28) days; and
  \item
    More than fifteen (15) players, in aggregate, on its Active and Inactive List as a result of:

    \begin{enumerate}
    \def\labelenumiii{(\Alph{enumiii})}
    \tightlist
    \item
      The NBA authorizing the Team to sign a Player Contract pursuant to the NBA's hardship rules; and/or
    \item
      A player on the Team's Active or Inactive List who (1) is unable to perform the playing services required under his Player Contract during a period in which he is subject to in-patient treatment prescribed by the Medical Director of the Anti-Drug Program, and (2) has missed at least three (3) consecutive Regular Season games because of such treatment.
    \end{enumerate}
  \end{enumerate}

  A day shall count toward the limits set forth in Section 2(b)(i) above if the Team had fewer than fourteen (14) players, in aggregate, on its Active List and Inactive List at the end of such day.
\item
  For each Two-Way Player that a Team places on the Active List or Inactive List, the minimum and maximum roster size limits set forth in Sections 2(a) and 2(b) above shall be increased by one (1) player for that Team.
\item
  Other than during the period set forth in Section 2(a) above, each Team agrees to have no more than twenty-one (21) players, in aggregate, on its Active List, Inactive List, and Two-Way List.
\item
  For purposes of this Article XXIX, ``postseason'' means Play-In Games and/or the playoffs.
\end{enumerate}

\hypertarget{two-way-roster.}{%
\section{Two-Way Roster.}\label{two-way-roster.}}

\begin{enumerate}
\def\labelenumi{(\alph{enumi})}
\tightlist
\item
  During the period from the first day of the Regular Season through the last day of the Season, a Two-Way Player shall be placed on his Team's (i) Active List or Inactive List (as applicable) while the Two-Way Player is providing services to the NBA Team, and (ii) Two-Way List at all other times.
\item
  A Two-Way Player is not eligible to be designated on an NBA Team's postseason roster or participate in NBA postseason games, but is permitted to travel and practice with the Team and remain on the Team's Inactive List during the NBA postseason; provided, however, that subject to Section 4 below, a player who was previously a Two-Way Player but who, prior to the start of the Team's last Regular Season game, either signs a Standard NBA Contract in accordance with Article II, Section 11(h) or has his Two-Way Contract converted by the Team to a Standard NBA Contract pursuant to Article II, Section 11(g), is eligible to be designated on an NBA Team's postseason roster and participate in NBA postseason games.
\end{enumerate}

\hypertarget{postseason-eligibility-waiver-deadline.}{%
\section{Postseason Eligibility Waiver Deadline.}\label{postseason-eligibility-waiver-deadline.}}

Any player (including any Two-Way Player) with respect to whom a request for waivers has been made after 11:59 p.m. eastern time on March 1 is not eligible to participate in postseason games during the then-current Season unless the player has been acquired by a Team whose Active List is reduced to eight (8) players due to injury or illness.

\hypertarget{minimum-league-wide-roster.}{%
\section{Minimum League-Wide Roster.}\label{minimum-league-wide-roster.}}

\begin{enumerate}
\def\labelenumi{(\alph{enumi})}
\item
  If for two consecutive Regular Seasons, NBA Teams in the aggregate employ an average of less than fourteen and one-quarter (14.25) players (excluding Two-Way Players) per Team, then for each Regular Season covered by this Agreement that follows such consecutive two-year period:

  \begin{enumerate}
  \def\labelenumii{(\roman{enumii})}
  \tightlist
  \item
    The requirement set forth in Section 2(a) above that each Team have either fourteen (14) or fifteen (15) players, in aggregate, on its Active and Inactive List shall be modified so that each Team would for the remainder of the term of this Agreement be required to have fifteen (15) players, in aggregate, on its Active and Inactive List; and
  \item
    The rule set forth in Section 2(b)(i) above allowing a Team to have twelve (12) or thirteen (13) players, in aggregate, on its Active List and Inactive List for no more than (A) two (2) consecutive weeks at a time, and (B) a total of twenty-eight (28) days shall be modified so that each Team would for the remainder of the term of this Agreement be permitted to have thirteen (13) or fourteen (14) players, in aggregate, on its Active List and Inactive List for such time periods (Sections 5(a)(i) and 5(a)(ii) together, the ``League-Wide Roster Increase'').
  \end{enumerate}
\item
  If for two consecutive Regular Seasons, NBA Teams in the aggregate employ an average of less than fourteen and one-half (14.5) players (excluding Two-Way Players) per Team, then the Players Association shall have the option, exercisable within forty-five (45) days following the last day of the second of such two consecutive Regular Seasons, to amend Article II, Section 11(b)(i) above such that, beginning on the first day of the immediately following Salary Cap Year and continuing through the remaining term of this Agreement, no Team would be permitted to have on its roster at any one time more than two (2) Two-Way Players (``PA Third Two-Way Option'').
\item
  The rules set forth in Sections 5(a) and 5(b) above shall be measured following each Regular Season as follows:

  \begin{itemize}
  \item
    STEP 1: For each player signed to a Standard NBA Contract (including a Rest-of-Season or 10-Day Contract) during a Regular Season, determine the number of days during such Regular Season that such player was carried on his Team's Active List or Inactive List (hereinafter ``Duty Days'').
  \item
    STEP 2: Determine the total Duty Days for all players for such Regular Season by adding together the results for each player from Step 1.
  \item
    STEP 3: Multiply (x) the number of NBA Teams that played games during the applicable Regular Season, by (y) the number of days during the Regular Season, by (z) fourteen and one-quarter (14.25) or fourteen and one-half (14.5), as applicable.
  \item
    STEP 4: If, for two consecutive Regular Seasons, the result in Step 2 above is less than the applicable result in Step 3 above, then the League-Wide Roster Increase and/or PA Third Two-Way Option, as applicable, will be triggered.
  \end{itemize}
\end{enumerate}

\hypertarget{games-played-requirement-for-certain-league-honors.}{%
\section{Games Played Requirement for Certain League Honors.}\label{games-played-requirement-for-certain-league-honors.}}

\begin{enumerate}
\def\labelenumi{(\alph{enumi})}
\tightlist
\item
  \textbf{Award Eligibility.} No player shall be eligible for NBA Most Valuable Player, NBA Defensive Player of the Year, NBA Most Improved Player, All-NBA Team (First, Second, or Third), or NBA All-Defensive Team (First or Second) honors (the ``Applicable Generally Recognized League Honors'') for a Season unless the player has satisfied at least one of the following criteria (the ``Award Eligibility Criteria'') in respect of such Season: (1) the player played in at least sixty-five (65) Regular Season games; or (2) the player (A) played in at least sixty-two (62) Regular Season games, (B) suffered a ``season-ending injury'' (as defined below), and (C) played in at least eighty-five percent (85\%) of the Regular Season games played by his Team prior to the player suffering such injury.

  \begin{enumerate}
  \def\labelenumii{(\roman{enumii})}
  \tightlist
  \item
    For purposes of this Section 6, and notwithstanding anything to the contrary in this Agreement:

    \begin{enumerate}
    \def\labelenumiii{(\Alph{enumiii})}
    \tightlist
    \item
      A ``season-ending injury'' is an injury that, in the opinion of a physician jointly selected by the NBA and Players Association, makes it substantially more likely than not that the player would be unable to play through the May 31 following the date of such injury; and
    \item
      A player shall be considered to have played in a Regular Season game if he played at least twenty (20) minutes of such game, provided that in respect of no more than two (2) Regular Season games per Season, such player will be considered to have played in a Regular Season game if he played at least fifteen (15) minutes and fewer than twenty (20) minutes in such game.
    \end{enumerate}
  \item
    A player who fails to satisfy the Award Eligibility Criteria in a Season may nonetheless be deemed eligible for the Applicable Generally Recognized League Honors if he prevails in either an Award Eligibility Grievance or an Extraordinary Circumstances Challenge in respect of such Season in accordance with the procedures set forth in Sections 6(b) and 6(c) below. A player may not bring both an Award Eligibility Grievance and an Extraordinary Circumstances Challenge in respect of the same Season.
  \end{enumerate}
\item
  \textbf{Award Eligibility Grievance.}

  \begin{enumerate}
  \def\labelenumii{(\roman{enumii})}
  \tightlist
  \item
    To prevail in an Award Eligibility Grievance in respect of a Season, the player bears the burden of proving, by clear and convincing evidence, that the Team willfully limited the player's number of minutes played or games played during the Regular Season with the intention of depriving the player of eligibility for one or more of the Applicable Generally Recognized League Honors for such Season. If the player satisfies the burden and prevails in the proceeding, the sole remedy shall be that the player is deemed eligible for the Applicable Generally Recognized League Honors. For clarity, neither the foregoing sentence nor anything else in this Agreement shall limit or otherwise affect the right of the NBA to impose discipline on a Team for conduct prejudicial or detrimental to the best interests of the NBA in the event a player prevails in an Award Eligibility Grievance against such Team.
  \item
    Award Eligibility Grievances shall be heard by the System Arbitrator.
  \item
    Notwithstanding any of the other provisions of this Agreement, the procedures set forth in this Section 6(b) shall apply to the resolution of Award Eligibility Grievances. If in connection with such disputes, there is any conflict between the procedures set forth in this Section 6(b) and those set forth elsewhere in this Agreement, the procedures set forth in this Section shall control.
  \item
    A player may only bring an Award Eligibility Grievance in respect of a Season if being awarded one of the Applicable Generally Recognized League Honors in such Season could impact whether the player is or could become eligible:

    \begin{enumerate}
    \def\labelenumiii{(\Alph{enumiii})}
    \tightlist
    \item
      Pursuant to Article II, Section 7(a)(i), to enter into a Contract or Extension with a Maximum Annual Salary in the first year covered by the Contract or extended term, as applicable, in excess of twenty-five percent (25\%) of the Salary Cap by virtue of satisfying the Higher Max Criteria;
    \item
      Pursuant to Article II, Section 7(a)(ii), to enter into a Designated Veteran Player Contract; or
    \item
      Pursuant to Article II, Section 7(e), to enter into a Designated Veteran Player Extension.
    \end{enumerate}
  \item
    An Award Eligibility Grievance must be brought by a player within two (2) days of the date on which it becomes mathematically impossible for the player to play sixty-five (65) Regular Season games in a Season; provided, however, that any such Grievance must be initiated no later than 11:59 p.m. eastern time on the day following the last day of the Regular Season.
  \item
    A player may initiate an Award Eligibility Grievance against a Team by serving a written notice thereof on the Team, with a copy of such written notice to be filed with the System Arbitrator, the Players Association, and the NBA. Such written notice shall be accompanied by a witness list, relevant documents, and other evidentiary materials on which the player intends to rely in his affirmative case. No later than 11:59 p.m. eastern on the date following the date on which the Team received written notice of the Award Eligibility Grievance, the Team shall provide to the player, the NBA, and the Players Association a witness list, relevant documents, and other evidentiary materials on which the Team intends to rely in its affirmative case. Absent a showing of good cause, no party may proffer, refer to, or rely on the testimony of any witness, document, or other evidentiary material in its affirmative case that has not been identified to the other side as required by this Section 6(b)(vi).
  \item
    The System Arbitrator shall convene a hearing at the earliest possible time, but in no event later than two (2) days following the System Arbitrator's receipt of notice of the Award Eligibility Grievance. The hearing shall take place by videoconference and shall last no longer than one (1) day. The Players Association, the NBA, and the player and Team that are parties to the proceeding shall each have the right to participate in the hearing.
  \item
    The System Arbitrator shall render a decision not later than the day following the date of the hearing, and the decision shall be accompanied by a written opinion. Notwithstanding the foregoing, if the System Arbitrator determines that expedition so requires, he/she shall accompany the decision with a written summary of the grounds upon which the decision is based, and a full written opinion may follow within a reasonable time thereafter. The decision of the System Arbitrator shall constitute full, final, and complete disposition of the dispute and shall be binding upon the parties to this Agreement, and the player and Team that are parties to the proceeding, and there shall be no appeal to the Appeals Panel.
  \item
    Should circumstances warrant, each of the deadlines set forth in this Section 6(b) may be reasonably modified by agreement of the NBA and Players Association.
  \end{enumerate}
\item
  \textbf{Extraordinary Circumstances Challenge.}

  \begin{enumerate}
  \def\labelenumii{(\roman{enumii})}
  \tightlist
  \item
    To prevail in an Extraordinary Circumstances Challenge in respect of a Season, the player bears the burden of proving that:

    \begin{enumerate}
    \def\labelenumiii{(\Alph{enumiii})}
    \tightlist
    \item
      Due to extraordinary circumstances, it was impracticable for him to play in one (1) or more of the Regular Season game(s) that he missed during such Season;
    \item
      He would have satisfied the Award Eligibility Criterion set forth in Section 6(a)(1) above if he had played in every game that he missed due to the extraordinary circumstances (i.e., assuming that he would have played twenty (20) minutes in each such missed game); and
    \item
      As a result of the extraordinary circumstances, and taking into account the totality of the circumstances, including whether the player did not play in other Regular Season games in which he could have played during such Season, it would be unjust to exclude the player from eligibility for the Applicable Generally Recognized League Honors for such Season.
    \end{enumerate}
  \item
    If the player satisfies the burden and prevails in the proceeding, the sole remedy shall be that the player is deemed eligible for the Applicable Generally Recognized League Honors.
  \item
    Extraordinary Circumstances Challenges shall be heard by an independent expert jointly selected by the NBA and Players Association (the ``Challenge Expert'').
  \item
    Notwithstanding any of the other provisions of this Agreement, the procedures set forth in this Section 6(c) shall apply to the resolution of Extraordinary Circumstances Challenges. If in connection with such disputes, there is any conflict between the procedures set forth in this Section 6(c) and those set forth elsewhere in this Agreement, the procedures set forth in this Section shall control.
  \item
    An Extraordinary Circumstances Challenge must be brought by a player in respect of a Season no earlier than 12:00 p.m. eastern time on the last day of the Regular Season and no later than 11:59 p.m. eastern time on the day following the last day of the Regular Season.
  \item
    A player may initiate an Extraordinary Circumstances Challenge by serving a written notice thereof on the NBA and his Team, with a copy of such written notice to be filed with the Challenge Expert and the Players Association. The NBA may provide notice thereof to any Team with which the player was under contract during the Season. The player's written notice shall be accompanied by a witness list, relevant documents, and other evidentiary materials on which the player intends to rely in his affirmative case. No later than 11:59 p.m. eastern on the date following the date on which the Team received written notice of the Extraordinary Circumstances Challenge, the Team shall provide to the player, the NBA, and the Players Association a witness list, relevant documents, and other evidentiary materials (if any) on which the Team intends to rely in its affirmative case (if any). Absent a showing of good cause, neither the player nor the Team may proffer, refer to, or rely on the testimony of any witness, document, or other evidentiary material in its affirmative case that has not been identified as required by this Section 6(c)(vi).
  \item
    The Challenge Expert shall convene a hearing at the earliest possible time, but in no event later than two (2) days following the Expert's receipt of notice of the Extraordinary Circumstances Challenge. The hearing shall take place by videoconference and shall last no longer than one (1) day. The player, the Players Association, the NBA, and any Team for which the player played during the Season shall have the right to participate in the hearing.
  \item
    The Challenge Expert shall render a decision not later than the day following the date of the hearing, and the decision shall be accompanied by a written opinion. Notwithstanding the foregoing, if the Challenge Expert determines that expedition so requires, he/she shall accompany the decision with a written summary of the grounds upon which the decision is based, and a full written opinion may follow within a reasonable time thereafter. The decision shall constitute full, final, and complete disposition of the matter.
  \item
    Should circumstances warrant, each of the deadlines set forth in this Section 6(c) may be reasonably modified by agreement of the NBA and Players Association.
  \end{enumerate}
\end{enumerate}

\hypertarget{playing-rules-and-officiating.}{%
\section{Playing Rules and Officiating.}\label{playing-rules-and-officiating.}}

\begin{enumerate}
\def\labelenumi{(\alph{enumi})}
\tightlist
\item
  Up to four (4) representatives of the Players Association, three (3) of whom shall be active or recently retired players selected by the Players Association, shall be permitted to attend the meetings of and have a vote on the NBA Competition Committee with respect to issues relating to the NBA playing rules and officiating.
\item
  The Players Association may, on behalf of the players, submit to the Commissioner monthly reports as to the conduct of referees, including identifying individual referees by name. The NBA will consider, but is not required to act, on such reports.
\item
  The NBA and Players Association shall meet on a quarterly basis to discuss the relationship and interactions between players and referees, including any discipline imposed by the NBA on referees for conduct on the playing court. Each party shall consider in good faith any recommendations made by the other party at such meetings regarding referee-player interactions.
\item
  Upon a request from the Players Association, representatives of the NBA Basketball Operations and Referee Operations Departments shall meet annually with the Players Association and/or players to discuss issues relating to NBA playing rules and officiating. The NBA will request that representatives from the National Basketball Referees Association, including current referees, attend any such meeting.
\end{enumerate}

\hypertarget{postseason.}{%
\section{Postseason.}\label{postseason.}}

\begin{enumerate}
\def\labelenumi{(\alph{enumi})}
\tightlist
\item
  The number of Teams participating in the playoffs shall equal sixteen (16). Notwithstanding the foregoing, the NBA shall have the right to increase the number of Teams participating in the playoffs.
\item
  Each round of the playoffs shall be played in a best-of-seven-games format.
\item
  To determine which Teams qualify to participate as the seventh and eighth seeds in the playoffs for each Conference, each Season shall include six (6) Play-In Games, to be played after the conclusion of the Regular Season and prior to the first round of the playoffs. The determination of which Teams shall participate in any Play-In Games shall be made based on each Team's Regular Season winning percentage.
\item
  The Team with the seventh-highest winning percentage in each Conference shall play the Team with the eighth-highest winning percentage in its Conference in a Play-In Game (the ``Seven-Eight Game''). The winner of the Seven-Eight Game in each Conference shall participate in the playoffs as the seventh seed in its Conference. The Team with the ninth-highest winning percentage in each Conference shall play the Team with the tenth-highest winning percentage in its Conference in a Play-In Game (the ``Nine-Ten Game''). The winner of the Nine-Ten Game shall play the loser of the Seven-Eight Game in a Play-In Game, and the winner of that game shall participate in the playoffs as the eighth seed in its Conference.
\end{enumerate}

\hypertarget{game-tickets.}{%
\section{Game Tickets.}\label{game-tickets.}}

\begin{enumerate}
\def\labelenumi{(\alph{enumi})}
\tightlist
\item
  In the event that a Team provides complimentary tickets to its players, the Team may provide up to four (4) tickets per home game and up to two (2) tickets per road game. Teams may sell additional tickets to players, provided that such sales shall be no less than the season ticket holder prices for the applicable game. Seat locations for complimentary tickets provided by a Team under this Section 9 must be in the lower bowl of the arena and may not be on the floor (i.e., in front of the risers or permanent bowl seating or inside the dashers) or in a luxury suite (i.e., a private, enclosed area that is separate from the arena bowl, including, but not limited to, traditional enclosed suites, event level (bunker) suites, and party suites).
\item
  In the event that a Team provides complimentary tickets to its players for road games, each player on the roster who travels with the Team shall be provided the same number of tickets (i.e., either zero (0), one (1), or two (2)).
\item
  Teams are prohibited from providing tickets to players on other Teams, and players are only permitted to accept tickets from their own Team.
\item
  Any player found to be re-selling complimentary or reduced-price tickets will be prohibited from subsequently receiving such tickets from his Team.
\item
  In the event that a Team provides home-game tickets to its players, seat locations must be allocated to players based on seniority, with the most senior players (based on Years of Service) receiving the most favorable seat locations.
\item
  NBA Teams shall provide four (4) tickets to authorized representatives of the Players Association to any home game at box office prices, provided notice of such request is given at least forty-eight (48) hours before the game.
\item
  Each Team agrees to provide retired players with three (3) or more Years of Service with the opportunity to purchase two (2) tickets at box office prices to its NBA home games, and to hold such tickets for such players, provided tickets are available and the retired players provide the Team with forty-eight (48) hours advance notice of their desire for such tickets.
\end{enumerate}

\hypertarget{league-pass.}{%
\section{League Pass.}\label{league-pass.}}

Any player who is under a Uniform Player Contract, with the exception of 10-Day Contracts or Two-Way Contracts, shall receive a free League Pass Broadband account in each Season of his Player Contract.

\hypertarget{release-for-fighting.}{%
\section{Release for Fighting.}\label{release-for-fighting.}}

Each NBA Team (hereinafter ``such Team'') hereby releases and waives every claim it may have against any player employed by other NBA Teams for injuries sustained by any player in the employ of such Team which arise out of, or in connection with, any fighting or other form of violent and/or unsportsmanlike conduct during the course of any Exhibition, Regular Season, Play-In, or playoff game.

\hypertarget{limitation-on-player-ownership.}{%
\section{Limitation on Player Ownership.}\label{limitation-on-player-ownership.}}

\begin{enumerate}
\def\labelenumi{(\alph{enumi})}
\tightlist
\item
  During the term of this Agreement, no NBA player may acquire or hold a direct or indirect interest in the ownership of any NBA Team or in any company or entity, whether privately or publicly owned, that owns any interest in any NBA Team; provided, however, that any player may have an ownership of publicly-traded securities constituting less than one percent (1\%) of the ownership interests in a company or entity that directly or indirectly owns an NBA Team.
\item
  Notwithstanding Section 12(a) above, and subject to Sections 12(c) and 12(d) below, during the term of this Agreement, the Players Association or an affiliate of the Players Association may invest on behalf of all NBA players in one or more private investment funds approved by the NBA to acquire passive, non-voting minority interests in one or more NBA Teams in accordance with all applicable NBA rules and regulations (each, a ``Private Investment Fund''); provided, however, that any such investments shall be subject to the following conditions and limitations:

  \begin{enumerate}
  \def\labelenumii{(\roman{enumii})}
  \tightlist
  \item
    Any such investment must be passive and non-voting and may not, at any time, exceed five percent (5\%) of the aggregate committed capital of such Private Investment Fund;
  \item
    The Players Association shall be subject to the same general restrictions and rules as applicable to other investors in such Private Investment Fund (e.g., compliance with applicable ``accredited investor'' requirements and minimum investment thresholds);
  \item
    Notwithstanding subparagraph (ii) above, the Players Association shall be prohibited from holding or exercising any active participation rights or roles, and from receiving any enhanced information, with respect to a Private Investment Fund, including serving on limited partner or other advisory committees of a Private Investment Fund; and
  \item
    The Players Association shall be required to divest or reduce its ownership interest in a Private Investment Fund if any of the conditions set forth in subparagraphs (i)-(iii) above cease to be met.
  \end{enumerate}
\item
  Any Players Association investment in one or more Private Investment Funds shall be subject to compliance with all applicable laws, including but not limited to securities laws and federal labor law. If, at any point, any Players Association investment or provision of this Section 12 does not comply with applicable laws, then the Players Association shall use best efforts to, and cause the Private Investment Fund to, modify the terms of such investment and/or the parties will negotiate in good faith to modify the terms of this Section 12, in each case, to the extent necessary to fully comply with law and, if such modifications are not or cannot be negotiated, then (i) the applicable Players Association investment shall be divested in full and (ii) the applicable terms of this Section 12 shall be rendered void and of no further force and effect.
\item
  Immediately following the execution of this Agreement, the NBA and the Players Association shall form a joint advisory committee (the ``Investment Committee'') to study and discuss in good faith any issues relating to (x) the Players Association's or an affiliate of the Players Association's investment in Private Investment Funds in accordance with Sections 12(b) and 12(c) above, as well as (y) a potential CBA modification pertaining to (i) investment in Private Investment Funds by individual players investing collectively through a pooled investment vehicle (a ``Players Vehicle''), and/or (ii) investment by individual players in NBA affiliated businesses, in each case taking into account all appropriate legal, business, and other considerations. Prior to any such investment, the members of the Investment Committee, either jointly or independently through each respective party, shall obtain the advice of counsel to the satisfaction of both the NBA and the Players Association stating that such investment complies with applicable labor laws, including, without limitation, Section 302 of the Taft-Hartley Act of 1947 (Labor Management Relations Act of 1947). For clarity, (1) no investment contemplated under this Section 12(d) shall be permitted unless and until such time as the NBA and the Players Association confirm to their satisfaction that such investment would comply with all applicable laws, and (2) no investment by a Players Vehicle in a Private Investment Fund or investment by players in NBA affiliated businesses shall be permitted unless and until such time as the NBA and the Players Association agree upon any new structures and/or rules required for such investment.

  \begin{enumerate}
  \def\labelenumii{(\roman{enumii})}
  \tightlist
  \item
    The Investment Committee shall consist of three (3) representatives appointed by the NBA and three (3) representatives appointed by the Players Association. At least one of the members appointed by each of the NBA and the Players Association must be knowledgeable of private investment funds and their structures. Unless otherwise mutually agreed by the parties, Investment Committee members may not have an ownership or other financial interest in any Private Investment Fund.
  \item
    The Investment Committee may jointly retain such experts as it deems necessary in order to conduct its work, which the parties expect to include investment and legal professionals. The costs of such experts will be borne equally by the NBA and the Players Association.
  \end{enumerate}
\end{enumerate}

\hypertarget{player-ownership-in-independent-wnba-teams.}{%
\section{Player Ownership in Independent WNBA Teams.}\label{player-ownership-in-independent-wnba-teams.}}

\begin{enumerate}
\def\labelenumi{(\alph{enumi})}
\tightlist
\item
  Subject to Section 13(b) below and also subject to WNBA approval in each case, and notwithstanding anything to the contrary in Article XIII, Section 2(c), an NBA player may invest in a WNBA Team in which no Team Owner (or family member of a Team Owner) has a direct or indirect beneficial ownership interest (each such team, an ``Independent WNBA Team'') on substantially similar terms to other third-party investors, subject to the following conditions and limitations:

  \begin{enumerate}
  \def\labelenumii{(\roman{enumii})}
  \tightlist
  \item
    An NBA player's ownership interest in an Independent WNBA Team may not, at any time, exceed four percent (4\%) of such Independent WNBA Team;
  \item
    NBA players may not in the aggregate hold more than an eight percent (8\%) ownership interest in any Independent WNBA Team;
  \item
    An NBA player may hold an ownership interest in only one (1) Independent WNBA Team at any one time;
  \item
    Any NBA player investing in an Independent WNBA Team shall be subject to WNBA restrictions, rules, and penalties, as imposed and enforced by the WNBA, applicable to other WNBA team owners (e.g., relating to tampering, public comments on collectively bargained matters, and penalties for misconduct);
  \item
    Notwithstanding subparagraph (iv) above, an NBA player shall be prohibited from holding any governance rights or roles with respect to an Independent WNBA Team in which he holds an ownership interest or with respect to the WNBA, including participation on WNBA team or WNBA league governing bodies;
  \item
    NBA players' receipt of information with respect to the Independent WNBA Team and the WNBA shall be limited to annual audited team financials and any required tax information;
  \item
    An NBA player may be required by the NBA to divest or reduce his ownership interest in an Independent WNBA Team if any of the conditions set forth in subparagraphs (i)-(vi) above cease to be met; and
  \item
    In the event an Independent WNBA Team in which an NBA player is invested proposes to sell a ``controlling ownership interest'' to a Team Owner (or family member of a Team Owner), any NBA player investor in such Independent WNBA Team will be required to dispose of its entire ownership interest in such Independent WNBA Team, including through the exercise of any tag-along or drag-along rights applicable to such ownership interest. For purposes of this Section 13, a ``controlling ownership interest'' means a majority of the voting or equity interests in, or contractual control of, the Independent WNBA Team. If the Independent WNBA Team's existing governing agreements do not contain provisions with such rights, then such agreements shall be amended in connection with an NBA player's investment to create customary tag-along or drag-along rights with respect to the player's interests, and such amendment shall be a condition to such NBA player's investment. For clarity, in the event an Independent WNBA Team in which an NBA player is invested proposes to sell less than a controlling ownership interest to a Team Owner (or a family member of a Team Owner), such proposed transaction will not be in compliance with Section 13(a)(1) above unless such NBA player disposes of his entire ownership interest in such Independent WNBA Team by selling to a person or entity that is not a Team Owner (or a family member of a Team Owner), prior to such proposed sale.
  \end{enumerate}
\item
  NBA player investment in Independent WNBA Teams shall be subject to compliance with all applicable laws, including but not limited to securities laws and federal labor law. If, at any point, any player investment or provision of this Section 13 does not comply with applicable laws, then the player shall use best efforts to, and cause the Independent WNBA Team to, modify the terms of such investment and/or the parties will negotiate in good faith to modify the terms of this Section 13, in each case, to the extent necessary to fully comply with law and, if such modifications are not or cannot be negotiated, then (i) the applicable player investment shall be divested in full and (ii) the applicable terms of this Section 13 shall be rendered void and of no further force and effect.
\item
  For purposes of this Section 13, a ``player'' shall include any person or entity controlled by, related to, or acting with authority on behalf of a player. For clarity, and notwithstanding anything to contrary in this Section 13, no agent or representative of a player may invest in a WNBA team.
\end{enumerate}

\hypertarget{nondisclosure.}{%
\section{Nondisclosure.}\label{nondisclosure.}}

The parties agree that (a) the economic terms of any individual Uniform Player Contract entered into by a Team and a player, and (b) any information contained in, or disclosed to the Players Association in connection with an Audit Report, Draft Audit Report, Interim Audit Report, Interim Designated Share Audit Report, or BRI Report, shall not be disclosed to the media by (i) the NBA, its Teams, or their respective employees, or (ii) the Players Association, NBA players, or their respective employees, agents, or representatives.

\hypertarget{implementation-of-agreement.}{%
\section{Implementation of Agreement.}\label{implementation-of-agreement.}}

\begin{enumerate}
\def\labelenumi{(\alph{enumi})}
\tightlist
\item
  The NBA and the Players Association will use their respective best efforts to have NBA Teams and NBA players comply with the terms and provisions of this Agreement.
\item
  The NBA and the Players Association shall use their respective best efforts and take all reasonable steps to cooperate to defend the enforceability of this Agreement against any challenge thereto.
\end{enumerate}

\hypertarget{additional-canadian-provisions.}{%
\section{Additional Canadian Provisions.}\label{additional-canadian-provisions.}}

\begin{enumerate}
\def\labelenumi{(\alph{enumi})}
\tightlist
\item
  The bases upon which a player may be disciplined or discharged or a Player Contract terminated, as set forth in this Agreement and/or in the Uniform Player Contract, shall constitute just and reasonable cause within the meaning of any applicable Canadian law or statute (federal or provincial) and, to the extent this Agreement or the Uniform Player Contract provides specific penalties for such conduct, those penalties shall apply.
\item
  During the term of this Agreement, the NBA and Players Association shall consult regularly about issues relating to the workplace which affect the parties or any player bound by this Agreement.
\item
  If and to the extent Sections 48 and 49 of the Ontario Labour Relations Act are or may be found applicable to this Agreement, the parties agree that the provisions thereof shall apply only to disputes between the Toronto Raptors and players for the Toronto Raptors. Furthermore, the parties agree and acknowledge that any termination and severance benefits provided to players pursuant to this Agreement (including the provisions of Player Contracts that provide, in certain circumstances, for the continued payment of Salary to a player following the termination of a Player Contract) constitute and/or shall be deemed to constitute a greater right or benefit to the Player pursuant to Section 5(2) of the Employment Standards Act, 2000 (Ontario) and the provisions of Sections 54-66 of such Act do not apply.
\item
  The parties acknowledge and agree that a player employed by an NBA Team pursuant to the provisions of a Uniform Player Contract, a 10-Day Contract, a Rest-of-Season Contract, or a Two-Way Contract is and/or shall be deemed to be an ``employee hired on the basis that his employment is to terminate on the expiry of a definite term or the completion of a specific task'' within the meaning of paragraph 1 of Section 2(1) of Ontario Regulation 288/01 under the Ontario Employment Standards Act, 2000, so as to render inapplicable to NBA players the provisions of Sections 54-62 of such Act.
\item
  The parties acknowledge and agree that the severance benefits provided to players pursuant to this Agreement (including the provisions of Player Contracts that provide, in certain circumstances, for the continued payment of Salary to a player following the termination of a Player Contract) constitute and/or shall be deemed to constitute a settlement binding on the player within the meaning of Section 6 of the Ontario Employment Standards Act, 2000, and/or ``an amount paid to an employee for loss of employment under a provision of an employment contract based upon length of employment, length of service or seniority'' within the meaning of paragraph 2 of Section 65(8) of the Ontario Employment Standards Act, 2000, so as to render inapplicable to NBA players the provisions of Sections 63-66 of such Act.
\item
  Upon the NBA's request, the Players Association shall cooperate with the NBA in a reasonable manner in connection with any effort the NBA may make to seek an exemption from any Canadian (federal or provincial) law or regulation affecting the employment relationship that is inconsistent with the provisions of this Agreement or any other agreement between the Players Association and the NBA (or NBA Properties) or between any player and any NBA Team.
\item
  All players employed by NBA Teams shall be paid in U.S. dollars, regardless of where such Teams are located.
\end{enumerate}

\hypertarget{gate-reports.}{%
\section{Gate Reports.}\label{gate-reports.}}

The NBA shall provide the Players Association with reports regarding each Team's gate receipts and paid attendance (including season ticket sale summaries) as of the date two (2) weeks prior to the date of each report. The reports shall be provided on or before the following dates in respect of each Season: December 31; February 28; April 30; and July 31; provided, however, that with respect to season ticket sale summaries, the NBA shall not provide a report on or before December 31 and shall instead provide a report on or before September 30.

\hypertarget{league-wide-public-service-campaigns.}{%
\section{League-Wide Public Service Campaigns.}\label{league-wide-public-service-campaigns.}}

The NBA will notify the Players Association of any league-wide public service campaign to be implemented by the NBA at least two (2) weeks before any player is requested to appear on behalf of such campaign.

\hypertarget{fines-imposed-on-teams.}{%
\section{Fines Imposed on Teams.}\label{fines-imposed-on-teams.}}

In the event that (a) a fine is imposed on a Team, Governor, or Team personnel (in each case, a ``Team Fine'') by the NBA for violation of a league rule regarding (i) injury, illness, rest, or game status reporting, (ii) timing of free agency discussions, (iii) tampering, (iv) leaving the bench area during a game, or (v) team criticism of game officials, and (b) such Team Fine amount is collected by the NBA, then the NBA shall remit fifty percent (50\%) of the amount collected to an NBPA-Selected Charitable Organization (as defined in Article VI, Section 6(a)), provided that the maximum amount that shall be remitted to an NBPA-Selected Charitable Organization in respect of any Team Fine shall be fifty percent (50\%) of the amount of the maximum fine that may be imposed on a player for engaging in the conduct at issue, and provided further that, where there is no specified maximum fine for a player for the relevant conduct, the maximum amount that shall be remitted to the NBPA-Selected Charitable Organization in respect of such Team Fine shall be \$50,000. The remittances made by the NBA pursuant to this Section 19 shall be made annually, ninety (90) days following the Accountants' (as defined in Article VII, Section 10(a)) submission to the NBA and the Players Association of a final Audit Report or an Interim Designated Share Audit Report (as defined in Article VII, Section 10(a)) for the Salary Cap Year during which the fine amounts are collected by the NBA.

\hypertarget{quarterly-fiba-meetings.}{%
\section{Quarterly FIBA Meetings.}\label{quarterly-fiba-meetings.}}

The NBA and Players Association shall meet at least quarterly to discuss FIBA matters that relate to NBA players (e.g., players' participation in international FIBA competitions during the off-season).

\hypertarget{pro-days.}{%
\section{Pro Days.}\label{pro-days.}}

\begin{enumerate}
\def\labelenumi{(\alph{enumi})}
\tightlist
\item
  Prior to any NBA Draft, Teams shall be prohibited from attending any practice or workout involving one (1) or more players eligible for such Draft if such practice or workout is conducted, arranged, or organized by such player or any person or entity acting with authority on behalf of such player (each such practice or workout, a ``pro day''). Notwithstanding the foregoing, Teams shall be permitted to attend a pro day that is conducted as a part of:

  \begin{enumerate}
  \def\labelenumii{(\roman{enumii})}
  \tightlist
  \item
    the NBA Draft Combine (with such pro days to be scheduled by the NBA in coordination with the Players Association); or
  \item
    a series of pro days facilitated and scheduled by the Players Association (in coordination with the NBA). Each Salary Cap Year, there shall be no more than two (2) such series of Players Association-facilitated pro days, with one such series to take place in California and the other such series to take place in a city (or geographic vicinity thereof) located within the eastern time zone. Each such series shall take place over a period of no longer than two (2) days; provided, however, that in circumstances where conducting the workouts in a two-day period is impracticable, such workouts may, following discussion by the NBA and Players Association, be conducted over a three-day period.
  \end{enumerate}
\item
  Pro days conducted in accordance with Sections 21(a)(i) and (ii) above shall, in each case, take place in a single athletic facility.
\end{enumerate}

\hypertarget{no-strike-and-no-lockout-provisions-and-other-undertakings}{%
\chapter{NO-STRIKE AND NO-LOCKOUT PROVISIONS AND OTHER UNDERTAKINGS}\label{no-strike-and-no-lockout-provisions-and-other-undertakings}}

\hypertarget{no-strike.}{%
\section{No Strike.}\label{no-strike.}}

During the term of this Agreement, neither the Players Association nor its members shall engage in any strikes, cessations or stoppages of work, or any other similar interference with the operations of the NBA or any of its Teams. Notwithstanding the foregoing, nothing in this Section 1 shall impair the rights accorded the Players Association by Article XXXIX, Section 3 (Termination by Players Association/Anti-Collusion), Section 6 (Mutual Right of Termination), Section 7 (Mutual Right of Termination -- League Financial Results), Section 8 (Mutual Right of Termination -- Designated Share), or Section 9 (Mutual Right of Termination -- League Entity Transaction).

\hypertarget{no-lockout.}{%
\section{No Lockout.}\label{no-lockout.}}

During the term of this Agreement, neither the NBA nor its Teams shall engage in any lockouts, cessations or stoppages of work or any other similar interference with the employment of NBA players by NBA Teams. Notwithstanding the foregoing, nothing in this Section 2 shall impair the rights accorded the NBA by Article XXXIX, Section 4 (Termination by NBA/National TV Revenues), Section 5 (Termination by NBA/Force Majeure), Section 6 (Mutual Right of Termination), Section 7 (Mutual Right of Termination -- League Financial Results), Section 8 (Mutual Right of Termination -- Designated Share), or Section 9 (Mutual Right of Termination -- League Entity Transaction).

\hypertarget{no-breach-of-player-contracts.}{%
\section{No Breach of Player Contracts.}\label{no-breach-of-player-contracts.}}

The Players Association agrees that it will not engage in any concerted activities to breach, induce the breach of, or threaten to breach or induce the breach of, any Player Contract.Section 4. Best Efforts of Players Association. The Players Association will use its best efforts: (a) to prevent each player from rendering, or threatening to render, services as a professional basketball player for another professional basketball team during the term of a Player Contract between such player and the Team for which he plays (except as said Player Contract may be assigned, sold, or transferred in accordance with the provisions of such Player Contract or this Agreement); (b) to prevent each player from refusing, or threatening to refuse, to participate in any scheduled Exhibition game, Regular Season game, All-Star Game, Rookie-Sophomore Game, All-Star Skills Competition, Play-In, or playoff game; (c) to prevent each player from refusing, or threatening to refuse, to report, within the time required, to a team in the NBAGL when the player has been assigned to or is providing NBAGL Two-Way Service with an NBAGL team in accordance with the provisions of this Agreement, and to prevent each such player from refusing, or threatening to refuse, to participate in any scheduled NBAGL game; (d) to prevent each player from otherwise breaching, or threatening to breach, his Player Contract; and (e) to prevent each player from making any demand upon the NBA or any of its Teams, including, but not limited to, a demand (accompanied by threats that the player will render services as a professional basketball player for another professional basketball team during the term of his Player Contract) that such Player Contract be renegotiated during the term thereof; provided, however, that this provision is not intended to prevent any player from entering into negotiations with a Team, in accordance with Article VII, with respect to the compensation to be paid to said player for the Season(s) following the last playing Season covered by any Player Contract, or renewal or extension thereof.

\hypertarget{no-discrimination.}{%
\section{No Discrimination.}\label{no-discrimination.}}

Neither the NBA, any Team, nor the Players Association shall discriminate in the interpretation or application of this Agreement against or in favor of any Player because of religion, race, national origin, sexual orientation, or activity or lack of activity on behalf of the Players Association.

\hypertarget{grievance-and-arbitration-procedure-and-special-procedures-with-respect-to-disputes-involving-player-discipline}{%
\chapter{GRIEVANCE AND ARBITRATION PROCEDURE AND SPECIAL PROCEDURES WITH RESPECT TO DISPUTES INVOLVING PLAYER DISCIPLINE}\label{grievance-and-arbitration-procedure-and-special-procedures-with-respect-to-disputes-involving-player-discipline}}

\chaptermark{GRIEVANCE AND ARBITRATION PROCEDURE AND SPECIAL \ldots}

\hypertarget{scope.}{%
\section{Scope.}\label{scope.}}

\begin{enumerate}
\def\labelenumi{(\alph{enumi})}
\item
  \begin{enumerate}
  \def\labelenumii{(\roman{enumii})}
  \tightlist
  \item
    Except as provided otherwise by this Agreement or by Paragraph 9 of the Uniform Player Contract, the Grievance Arbitrator shall have exclusive jurisdiction to determine, in accordance with procedures set forth in this Article XXXI, any and all disputes involving the interpretation or application of, or compliance with, the provisions of this Agreement or the provisions of a Player Contract, including any dispute concerning the validity of a Player Contract or any dispute arising under the Joint NBA/NBPA Policy on Domestic Violence, Sexual Assault, and Child Abuse. Any such dispute subject to the exclusive jurisdiction of the Grievance Arbitrator shall hereinafter be referred to as a ``Grievance.''
  \item
    The Grievance Arbitrator shall also have jurisdiction to resolve disputes among the applicable trustees arising under the Agreement of Trust for the National Basketball Association Players' Health and Welfare Benefit Plan, the Agreement and Declaration of Trust Establishing the National Basketball Players Association/National Basketball Association Labor-Management Cooperation and Education Trust, and the Trust Agreements for the National Basketball Association Players' Qualified and Non-Qualified Post-Career Income Plans in accordance with the provisions of such agreements and declarations of trust. In connection with the resolution of such disputes, to the extent there is any conflict between the provisions of such agreements and declarations of trust and the provisions of this Agreement, the provisions of such agreements and declarations of trust shall control.
  \end{enumerate}
\item
  Notwithstanding the provisions of Section 1(a) above:

  \begin{enumerate}
  \def\labelenumii{(\roman{enumii})}
  \tightlist
  \item
    Disputes arising under Articles I, II, VII, VIII, X, XI, XII, XIII, XIV, XV, XVI, XXII, Section 14(j)(iii), Article XXIX, Section 6(b), XXXVII, XXXIX, and XL, as well as disputes arising under Article XXVIII and Paragraph 14 of the Uniform Player Contract regarding an Unauthorized Sponsor Promotion (as that term is defined in Paragraph 14(e) of the Uniform Player Contract) shall (except as otherwise specifically provided by Article VII, Section 3(d)(5)) be determined by the System Arbitrator provided for in Article XXXII; and
  \item
    Disputes involving (A) a fine or suspension imposed upon a player by the Commissioner (or his designee) for conduct on the playing court or in-game conduct involving another player (as those terms are defined in Section 9(c) below), or (B) action taken by the Commissioner (or his designee) concerning the preservation of the integrity of, or maintenance of public confidence in, the game of basketball, shall be resolved in accordance with the provisions set forth in Section 9 below.
  \end{enumerate}
\end{enumerate}

\hypertarget{initiation.}{%
\section{Initiation.}\label{initiation.}}

\begin{enumerate}
\def\labelenumi{(\alph{enumi})}
\tightlist
\item
  Grievances may be initiated, as set forth below, by a player, a Team, the NBA, or the Players Association, except that the Players Association may not initiate a Grievance involving player discipline without the approval of the player(s) concerned.
\item
  No party may initiate a Grievance until and unless it has first discussed the matter with the party or parties against whom the Grievance is to be initiated in an attempt to settle it.
\item
  A Grievance must be initiated, in accordance with the provisions of Section 2(d) below, within thirty (30) days from the date of the occurrence upon which the Grievance is based, or within thirty (30) days from the date upon which the facts of the matter became known or reasonably should have become known to the party initiating the Grievance, whichever is later.
\item
  Subject to the provisions of Sections 2(a)-(c) above: (i) a player or the Players Association may initiate a Grievance (A) against the NBA by filing written notice thereof with the NBA, and (B) against a Team, by filing written notice thereof with the Team and the NBA; (ii) a Team may initiate a Grievance by filing written notice thereof with the Players Association and furnishing copies of such notice to the player(s) involved and to the NBA; and (iii) the NBA may initiate a Grievance by filing written notice thereof with the Players Association and furnishing copies of such notice to the player(s) and Team(s) involved. Any such notice shall expressly state that the party is initiating a Grievance pursuant to Article XXXI, Section 2.
\end{enumerate}

\hypertarget{pre-hearing-motions.}{%
\section{Pre-Hearing Motions.}\label{pre-hearing-motions.}}

\begin{enumerate}
\def\labelenumi{(\alph{enumi})}
\tightlist
\item
  A party to a Grievance may file a pre-hearing motion with the Grievance Arbitrator under this Section 3 if that party is seeking to have the Grievance dismissed (i) because the Grievance Arbitrator does not have jurisdiction to hear the matter under Section 1 above, or (ii) for the opposing party's failure to properly initiate a Grievance or file the Grievance on a timely basis pursuant to Section 2 above.
\item
  Upon the filing of a motion under Section 3(a) above, the parties will schedule a conference call with the Grievance Arbitrator for the purposes of setting a schedule for the motion, including a date for the opposing party's opposition brief and a date for oral argument before the Grievance Arbitrator. Oral argument under this Section 3(b) shall be conducted by teleconference.
\item
  The opposing party may request a factual hearing on the motion in its opposition brief, but cannot request a factual hearing on the underlying merits of the Grievance. If the Grievance Arbitrator grants the request for a factual hearing, the hearing shall comply with the requirements of Sections 4, 5, and 6 below.
\item
  The Grievance Arbitrator shall render a decision on the motion (including any appropriate award) as soon as practicable and the decision shall be accompanied by a written opinion, or, if both the NBA and the Players Association agree, the written opinion may follow within a reasonable time thereafter. In no event shall the award and written opinion be issued more than thirty (30) days following the date of the oral argument or, where applicable, following the date designated by the Grievance Arbitrator for the submission of post-argument briefs. If the decision is dispositive, the award shall constitute full, final, and complete disposition of the Grievance, and shall be binding upon the player(s) and Team(s) involved and the parties to this Agreement.
\item
  The procedure set forth in this Section 3 shall not be applicable to disputes with respect to which the Expedited Procedure set forth in Section 13 is properly invoked by either the NBA or the Players Association; provided, however, that this Section 3(e) shall not preclude any party from asserting, in a proceeding to which such Expedited Procedure applies, that the Grievance should be dismissed (i) because the Grievance Arbitrator does not have jurisdiction to hear the matter under Section 1 above, or (ii) for the opposing party's failure to properly initiate a Grievance or file the Grievance on a timely basis pursuant to Section 2 above.
\item
  If a pre-hearing motion to dismiss is denied, the NBA and the Players Association shall schedule a hearing promptly with respect to the merits of the Grievance involved.
\end{enumerate}

\hypertarget{hearings.}{%
\section{Hearings.}\label{hearings.}}

\begin{enumerate}
\def\labelenumi{(\alph{enumi})}
\tightlist
\item
  Upon at least thirty (30) days' written notice to the other side, the NBA and the Players Association may arrange to have a hearing scheduled on a date that is mutually convenient to the parties to the dispute, the NBA, the Players Association, and the Grievance Arbitrator; provided, however, that if the NBA and the Players Association cannot agree on a hearing date, the Grievance Arbitrator shall set a reasonable hearing date that follows the expiration of the thirty-day notice period. Only the NBA and the Players Association may schedule or postpone hearings before the Grievance Arbitrator.
\item
  Notwithstanding the provisions of Section 4(a) above, during each Salary Cap Year covered by this Agreement, (i) the Players Association and the NBA shall each have the right, upon a showing of need, to have two (2) Grievances scheduled for hearing on or after the tenth day following service of the notice provided for by Section 4(a) above, and (ii) in addition to the foregoing, the Players Association shall have the right, upon a showing of need, to have one (1) additional Grievance scheduled for hearing on or after the tenth day following service of the notice provided for by Section 4(a) above for the purpose of challenging a suspension imposed on a player by a Team. The provisions of this Section 4(b) shall not limit or otherwise affect the rights of the NBA or the Players Association pursuant to Section 13 below.
\item
  If a Grievance is scheduled for hearing under this Article XXXI, and the hearing date is thereafter postponed at the request of either the NBA or the Players Association, the postponement fee (if any) of the Grievance Arbitrator will be borne by the party requesting the postponement, unless that party objects and the Grievance Arbitrator finds that the request for such postponement was for good cause. Should good cause be found, the parties will share any postponement fee equally.
\item
  In any Grievance matter, neither the NBA nor the Players Association may request or be granted more than one (1) postponement of a hearing previously scheduled under this Article XXXI. If a party which has been granted a postponement of a hearing fails to attend a subsequently scheduled hearing in the same Grievance matter, the Grievance shall be resolved against that party.
\item
  If (i) a hearing of a Grievance is not scheduled to take place within one (1) year from the initiation of the Grievance, or (ii) in the circumstance where the initial date set for the hearing has been postponed, if a second hearing in that Grievance is not scheduled to take place within two (2) years from the initiation of the Grievance, then the Grievance shall, upon written notice to the party or parties filing such Grievance, be deemed to have been dismissed with prejudice as of the thirtieth (30th) day following the delivery of such notice without the need for a hearing or for any action to be taken or decision to be issued by the Grievance Arbitrator, unless, upon written application made by the party or parties filing such Grievance within such thirty-day period, the Grievance Arbitrator determines that dismissal of the Grievance without prejudice would be unjust.
\item
  For purposes of computing time under this Section 4, the time shall be tolled during any period when there is no Grievance Arbitrator or when the grieving party has been unable to schedule a hearing (after making efforts to do so) because the Grievance Arbitrator is unavailable.
\item
  Hearings before the Grievance Arbitrator shall be held in New York (alternating between the NBA and Players Association offices). All such hearings shall be conducted in accordance with the Labor Arbitration Rules of the American Arbitration Association; provided, however, that in the event of any conflict between such Rules and the provisions of this Agreement, the provisions of this Agreement shall control.
\end{enumerate}

\hypertarget{procedure.}{%
\section{Procedure.}\label{procedure.}}

\begin{enumerate}
\def\labelenumi{(\alph{enumi})}
\tightlist
\item
  Not later than seven (7) days prior to the hearing, the parties shall submit to the Grievance Arbitrator a joint statement of the issue(s) in dispute. If the parties cannot agree on such a joint statement, each party may submit to the Grievance Arbitrator a separate statement setting forth the disputed issue(s), and such separate statement shall be delivered to the other party or parties at the same time it is submitted to the Grievance Arbitrator.
\item
  During each Salary Cap Year covered by this Agreement, the NBA and the Players Association shall each be entitled, as a matter of right, in connection with two (2) proceedings brought pursuant to this Article XXXI, to the discovery, in advance of a hearing, of non-privileged documents from any adverse party (or parties) in such proceeding. The party (or parties) to whom a request for document discovery is made shall have the obligation to produce only documents that are directly relevant and material to the core issue(s) in dispute, and shall not be obligated to produce documents merely because the production of such documents would be reasonably calculated to lead to the discovery of relevant or admissible evidence.
\item
  Not later than three (3) business days prior to the hearing, the parties shall exchange witness lists, relevant documents and other evidentiary materials, and citations of legal authorities that each side intends to rely on in its affirmative case. Absent a showing of good cause, no party may proffer, refer to, or rely on the testimony of any witness, any document, or other evidentiary material in its affirmative case that has not been identified to the other side as required by this subsection.
\item
  The Grievance Arbitrator shall grant the request of any party to file a pre-hearing and/or post-hearing brief, unless an opposing party demonstrates that the filing of briefs is unreasonable in the circumstances. If the Grievance Arbitrator grants a request to file pre-hearing briefs, such briefs shall be served on the adverse party (or parties) and filed with the Grievance Arbitrator not later than three (3) business days prior to the hearing. No pre-hearing brief shall exceed ten (10) pages in length, and the rules applicable in the United States District Court for the Southern District of New York with respect to the calculation of pages, the size of font, margins, and the like shall apply. If the Grievance Arbitrator grants a request to file post-hearing briefs, such briefs shall be served on the adverse party (or parties) and filed with the Grievance Arbitrator not later than seven (7) calendar days after the conclusion of the hearing (unless the parties otherwise agree).
\end{enumerate}

\hypertarget{arbitrators-decision-and-award.}{%
\section{Arbitrator's Decision and Award.}\label{arbitrators-decision-and-award.}}

\begin{enumerate}
\def\labelenumi{(\alph{enumi})}
\tightlist
\item
  Except as set forth in Section 13 below, the Grievance Arbitrator shall render an award as soon as practicable. The award shall be accompanied by a written opinion, or, if both the NBA and the Players Association agree, the written opinion may follow within a reasonable time thereafter. In no event shall the award and written opinion be issued more than thirty (30) days following the conclusion of a Grievance hearing (or, where applicable, following the date designated by the Grievance Arbitrator for the submission of post-hearing briefs). The award shall constitute full, final, and complete disposition of the Grievance, and shall be binding upon the player(s) and Team(s) involved and the parties to this Agreement.
\item
  In addition to such other limitations as may be imposed on him/her by this Agreement, the Grievance Arbitrator shall have jurisdiction and authority only to: (i) interpret, apply, or determine compliance with the provisions of this Agreement; (ii) interpret, apply, or determine compliance with the provisions of Player Contracts; (iii) determine the validity of Player Contracts; (iv) award damages in connection with a proceeding provided for in Section 12 below; (v) award declaratory relief in connection with a proceeding initiated by a Team to determine whether such Team may properly terminate a Player Contract pursuant to Paragraph 16(a) of such Contract, and what, if any, liability such Team would incur as a result of such termination; and (vi) resolve disputes arising under Article VII, Section 3(d)(5), Article XXVI, and Article XXXIII in the manner set forth therein. Notwithstanding the foregoing or any other provision of this Agreement or the Uniform Player Contract, the Grievance Arbitrator shall not have jurisdiction or authority to add to, detract from, or alter in any way the provisions of this Agreement (including the provisions of this Section 6(b)) or any Player Contract. Nor, in the absence of agreement by the NBA and the Players Association, shall the Grievance Arbitrator have jurisdiction or authority to resolve questions of substantive, as opposed to procedural, arbitrability. Questions of substantive arbitrability shall include the question of whether an arbitrator provided for by the terms of this Agreement, as opposed to the Commissioner (or his designee), has jurisdiction to hear or resolve a particular dispute and such questions shall be determined in a judicial proceeding to be venued in the United States District Court for the Southern District of New York.
\end{enumerate}

\hypertarget{appointment-and-replacement-of-grievance-arbitrator.}{%
\section{Appointment and Replacement of Grievance Arbitrator.}\label{appointment-and-replacement-of-grievance-arbitrator.}}

\begin{enumerate}
\def\labelenumi{(\alph{enumi})}
\tightlist
\item
  The parties to this Agreement shall agree upon the appointment of a Grievance Arbitrator, who shall serve for the duration of this Agreement; provided, however, that as of September 1, 2024, and as of each successive September 1, either of the parties to this Agreement may discharge the Grievance Arbitrator by serving written notice upon him/her and upon the other party to this Agreement during the period July 27 through August 1 immediately preceding each such September 1; and provided, further, that as of the April 30 of the last Season covered by this Agreement (or any extension thereof), either of the parties may discharge the Grievance Arbitrator by serving written notice upon him/her and upon the other party to this Agreement during the period March 26 through March 31 immediately preceding such April 30. A Grievance Arbitrator as to whom a notice of discharge has been served shall continue to have jurisdiction only with respect to (i) Grievances as to which a hearing has been commenced or scheduled for a date certain and (ii) Grievances filed within the thirty (30) day period preceding the service of a notice of discharge; provided, however, that a hearing with respect to Grievances referred to in this Section 7(a)(ii) must commence no later than thirty (30) days following the effective date of the Grievance Arbitrator's discharge.
\item
  If the Grievance Arbitrator is discharged (or resigns), the parties shall agree upon a successor Grievance Arbitrator. In the absence of such agreement, the parties shall jointly request the International Institute for Conflict Prevention and Resolution (the ``CPR Institute'') (or such other organization(s) as the parties may agree upon) to submit to the parties a list of eleven (11) attorneys, none of whom shall have, nor whose firm shall have, represented within the past five (5) years any professional athletes; agents or other representatives of professional athletes; labor organizations representing athletes; sports leagues, governing bodies, or their affiliates; sports teams or their affiliates; or owners in any professional sport. If, within seven (7) days from the receipt of such list, the parties fail to agree upon the selection of a Grievance Arbitrator from among the names on such list, they shall return that list, with up to five (5) names deleted therefrom by each party, to the CPR Institute (or such other organization as the parties may have agreed upon), and the CPR Institute (or such other organization) shall choose a new Grievance Arbitrator from the names remaining on such list.
\end{enumerate}

\hypertarget{injury-grievances.}{%
\section{Injury Grievances.}\label{injury-grievances.}}

\begin{enumerate}
\def\labelenumi{(\alph{enumi})}
\tightlist
\item
  If a party to a dispute arising under Paragraph 7, 16(a)(iii), 16(b), or 16(c) of a Uniform Player Contract so elects, the NBA and the Players Association shall agree upon a neutral physician or (in the absence of such agreement) jointly request that the President of the American College of Orthopedic Surgeons (or such other similar organization as the NBA and the Players Association agree may be most appropriate to the issues in dispute) designate a physician who has no relationship with any party covered by this Agreement who shall, for purposes of the dispute, serve as an independent medical expert and consultant to the Grievance Arbitrator. Such independent medical expert shall conduct a physical examination of the player; review such medical records and reports relating to the player that bear on the issues in dispute; and prepare a written report of the player's medical condition, which report shall address any specific medical questions submitted to the independent medical expert by joint agreement of the parties or by the Grievance Arbitrator. Any reports, opinions, or conclusions of the independent medical expert shall be provided in writing to the parties in advance of any hearing scheduled pursuant to Section 4 above. The opinions and conclusions of the independent medical expert shall be accorded such weight as the Grievance Arbitrator deems appropriate. The fees and costs of the independent medical expert shall be borne equally by both sides.
\item
  During the course of any arbitration proceeding, the Grievance Arbitrator may, by appropriate process, require any person (including, but not limited to, a Team and a Team physician, and a player and any physician consulted by such player) to provide to the player or that player's Team, as the case may be, all medical information in the possession of any such person relating to the subject matter of the arbitration.
\end{enumerate}

\hypertarget{special-procedures-with-respect-to-player-discipline.}{%
\section{Special Procedures with Respect to Player Discipline.}\label{special-procedures-with-respect-to-player-discipline.}}

\begin{enumerate}
\def\labelenumi{(\alph{enumi})}
\item
  A dispute involving (i) a fine of \$50,000 or less or a suspension of twelve (12) games or fewer (or both such fine and suspension) imposed upon a player by the Commissioner (or his designee) for (x) conduct on the playing court (as defined in Section 9(c)(i) below) or (y) for in-game conduct involving another player (as defined in Section 9(c)(ii) below), or (ii) action taken by the Commissioner (or his designee) (A) concerning the preservation of the integrity of, or the maintenance of public confidence in, the game of basketball and (B) resulting in a financial impact on the player of \$50,000 or less, shall not give rise to a Grievance, shall not be subject to a hearing before, or resolution by, the Grievance Arbitrator, and shall not be determined by arbitration; but instead shall be processed exclusively as follows:

  \begin{enumerate}
  \def\labelenumii{(\arabic{enumii})}
  \tightlist
  \item
    Within twenty (20) days following written notification of the action taken by the Commissioner (or his designee), the Players Association (with the approval of the player involved) may appeal in writing to the Commissioner.
  \item
    Upon the written request of the Players Association, the Commissioner shall designate a time and place for a hearing as soon as is reasonably practicable following his receipt of the notice of appeal.
  \item
    As soon as reasonably practicable, but not later than twenty (20) days, following the conclusion of such hearing, the Commissioner shall render a written decision, which decision shall, absent further proceedings pursuant to Section 9(a)(5) below, constitute full, final, and complete disposition of the dispute, and shall be binding upon the player(s) and Team(s) involved and the parties to this Agreement.
  \item
    In the event such appeal involves a fine and/or suspension imposed by the Commissioner's designee, the Commissioner, as a consequence of such appeal and hearing, shall have authority only to affirm or reduce such fine and/or suspension, and shall not have authority to increase such fine and/or suspension.
  \item
    If a dispute under Section 9(a)(i)(y) above is not resolved in a manner satisfactory to the player as a result of the procedures set forth in Sections 9(a)(1)-(4) above, then the Players Association may (with the approval of such player) seek review of the financial impact of the Commissioner's decision by filing a written request for such review with the Player Discipline Arbitrator (as provided for below) within ten (10) days following the issuance of such decision, and the following procedures shall apply:

    \begin{enumerate}
    \def\labelenumiii{(\alph{enumiii})}
    \tightlist
    \item
      Following receipt of the written request for review, the Player Discipline Arbitrator shall schedule a meeting with the player, the Players Association, and the NBA (and such representatives as each may designate), shall review the relevant facts and circumstances, and shall issue a decision affirming or reducing the financial penalty imposed by the Commissioner. All such meetings shall be in person, shall be held in New York (alternating between the NBA and Players Association offices), and shall be conducted during the month of September following the conclusion of the Season in which the in-game conduct involving another player occurred.
    \item
      In reviewing the fine and/or suspension imposed upon the player by the Commissioner, the Player Discipline Arbitrator shall have authority only to affirm or reduce the financial penalty associated with such fine and/or suspension (including lost salary). The Player Discipline Arbitrator shall have no authority to review financial penalties automatically imposed as a result of technical fouls, ejections, or the violation of other similar NBA rules that result in the imposition of an automatic penalty (such as the ``leaving the bench'' rule). The review by the Player Discipline Arbitrator shall be de novo.
    \item
      The decision of the Player Discipline Arbitrator shall constitute full, final, and complete disposition of the dispute, and shall be binding upon the player(s) and Team(s) involved and the parties to this Agreement. The Player Discipline Arbitrator shall make no public comment regarding the matter.
    \item
      The Player Discipline Arbitrator shall be selected by agreement between the NBA and the Players Association, and shall be (i) a person with experience in professional basketball (such as a former NBA coach, general manager, or player) or (ii) an attorney with experience as a private arbitrator and/or mediator. In the event that the NBA and the Players Association cannot agree on the identity of the Player Discipline Arbitrator, each party shall simultaneously serve upon the other a list of the names of five (5) individuals meeting the criteria set forth in this Section 9(a)(5)(d) and shall alternate in striking names from such list until only one (1) such name remains; and the individual whose name remains on the list shall be selected as the Player Discipline Arbitrator. (A coin-flip or such other procedure as agreed upon by the NBA and the Players Association shall determine which of such parties shall exercise the first strike.)
    \item
      The Player Discipline Arbitrator shall serve for the duration of this Agreement; provided, however, that as of January 1, 2024, and as of each successive January 1, either of the parties to this Agreement may discharge the Player Discipline Arbitrator by serving written notice upon him and upon the other party to this Agreement during the period from November 1 through December 1 immediately preceding each such January 1.
    \item
      If the Player Discipline Arbitrator is discharged (or resigns), the parties shall select a successor Player Discipline Arbitrator in accordance with the procedures set forth in Section 9(a)(5)(d) above.
    \end{enumerate}
  \end{enumerate}
\item
  A dispute involving (i) a fine of more than \$50,000 and/or a suspension of more than twelve (12) games that is imposed upon a player by the Commissioner (or his designee) for conduct on the playing court, or (ii) an action taken by the Commissioner (or his designee) that (A) concerns the preservation of the integrity of, or the maintenance of public confidence in, the game of basketball and (B) results in a financial impact on the player of more than \$50,000, shall be processed and determined in the same manner as a Grievance under Sections 2-7 above; provided, however, that the Grievance Arbitrator shall apply an ``arbitrary and capricious'' standard of review.
\item
  \begin{enumerate}
  \def\labelenumii{(\roman{enumii})}
  \tightlist
  \item
    As used in this Agreement, ``conduct on the playing court'' shall mean conduct in any area within an arena (including, but not limited to, locker rooms, vomitories, loading docks, and other back-of-house and underground areas, including those used by television production and other vehicles), at, during, or in connection with an NBA Exhibition, All-Star, Regular Season, Play-In, or playoff game. (By way of example and not limitation, conduct ``at'' and/or ``in connection with'' an NBA game shall include conduct engaged in by a player within an arena from the time the player arrives at the arena for an NBA game until the time the player has left the premises of the arena following the conclusion of such game.) Conduct engaged in by a player outside an arena such as, for example, in a parking lot adjacent to an arena, shall not constitute ``conduct on the playing court.''
  \item
    As used in this Agreement, ``in-game conduct involving another player'' shall mean conduct occurring during the course of an NBA Exhibition, All-Star, Regular Season, Play-In, or playoff game that is exclusively between or among players (and not, for example, involving in any manner a referee, fan, or coach) and that takes place on or adjacent to the playing floor (including the area of the benches), and shall include, but not be limited to, fights, altercations, flagrant fouls, and other similar conduct.
  \end{enumerate}
\item
  In the event a matter filed as a Grievance in accordance with the provisions of this Article XXXI gives rise to issues involving the integrity of, or public confidence in, the game of basketball, and the financial impact on the player of the action being grieved is \$50,000 or less, the Commissioner may, at any stage of its processing, order that the matter be withdrawn from such processing and thereafter be processed in accordance with the appeal procedure provided in Sections 9(a)(1)-(4) above.
\end{enumerate}

\hypertarget{procedure-with-respect-to-fine-and-suspension-amounts.}{%
\section{Procedure with Respect to Fine and Suspension Amounts.}\label{procedure-with-respect-to-fine-and-suspension-amounts.}}

In the event that a Grievance or an appeal challenging a Commissioner or Team-imposed fine and/or suspension is filed in accordance with this Article XXXI, the amount of any fine or salary lost by virtue of the suspension shall be deposited in a separate interest-bearing account maintained for such fines or suspension-related amounts. The NBA shall provide written notice to the Players Association of the date and amount of each deposit made pursuant to this Section 10 by delivering to the Players Association monthly statements reflecting the investment activity in such account. In the absence of agreement between the NBA and the Players Association, the Grievance Arbitrator (in resolving a Grievance, and in a manner consistent with his determination of such Grievance), or the Commissioner (or his designee) (in resolving an appeal, and in a manner consistent with his determination of such appeal), or the Player Discipline Arbitrator (in connection with his review of a decision by the Commissioner, and in a manner consistent with his determination following such review) shall determine the amount of the deposited funds to be payable to the player, the Team, or the NBA, and any interest earned on such deposit shall be allocated to the parties in proportion thereto.

\hypertarget{disputes-with-respect-to-the-terms-of-a-player-contract.}{%
\section{Disputes with Respect to the Terms of a Player Contract.}\label{disputes-with-respect-to-the-terms-of-a-player-contract.}}

\begin{enumerate}
\def\labelenumi{(\alph{enumi})}
\tightlist
\item
  If either the NBA or the Players Association asserts that a term or provision of a Player Contract is not permitted by this Agreement, either may have the dispute involving such Contract term or provision resolved by initiating a Grievance. If such a Grievance is initiated by the NBA, the thirty-day time period referred to in Section 2(c) above shall commence with the date upon which the NBA received the Player Contract (or amendment thereto) containing the disputed term or provision. If such a Grievance is initiated by the Players Association, the thirty-day time period referred to in Section 2(c) above shall commence with the date upon which the Player Contract (or amendment thereto) containing the disputed term or provision was first made available for inspection by the Players Association.
\item
  If, as a result of the Grievance and Arbitration procedure, a Player Contract is found to contain a term or provision that is not permitted by this Agreement, then (i) such term or provision shall be deleted from the Player Contract and have no force or effect, and the Player Contract shall in all other respects remain valid and binding upon the parties thereto, and (ii) if the Team and the player agree to reform or revise the Player Contract within thirty (30) days of the Grievance Arbitrator's decision, such reformation or revision shall be exempted from the rules governing Renegotiations contained in Article VII, Section 7(c).
\item
  Nothing set forth above shall affect in any manner the Commissioner's authority with respect to the approval or disapproval of Player Contracts pursuant to Paragraph 11 of the Uniform Player Contract; and the fact that the Commissioner has approved or not disapproved a Player Contract containing a term or provision not permitted by this Agreement shall not be referred to in the course of the Grievance and Arbitration procedure and shall not be considered in any manner or for any purpose by the Grievance Arbitrator in connection with a dispute concerning that Player Contract.
\end{enumerate}

\hypertarget{disputes-with-respect-to-players-under-contract-who-withhold-playing-services.}{%
\section{Disputes with Respect to Players Under Contract Who Withhold Playing Services.}\label{disputes-with-respect-to-players-under-contract-who-withhold-playing-services.}}

In addition to any other rights a Team may have under contract or law, including those under Paragraph 9 of a Uniform Player Contract, a Team may recover damages in a proceeding before the Grievance Arbitrator when a player who is party to a currently effective Player Contract fails or refuses to render the services called for under the Player Contract. In any such proceeding, where the Grievance Arbitrator determines that damages are continuing to accrue at the time of the hearing, the Arbitrator shall award such damages (if any) as the Team has by then sustained, and the hearing shall remain open to enable the submission of proof on the issue of continuing damages.

\hypertarget{expedited-procedure.}{%
\section{Expedited Procedure.}\label{expedited-procedure.}}

\begin{enumerate}
\def\labelenumi{(\alph{enumi})}
\tightlist
\item
  Notwithstanding the foregoing, in the event of a dispute arising under Article XVII, Article XXX, or Article XXXI, Section 12 of this Agreement, or under Paragraph 15 of a Uniform Player Contract (but only insofar as such Paragraph provides), or in the event of an alleged breach by a player of Paragraph 9 of a Uniform Player Contract, the NBA or the Players Association may request that such dispute or alleged breach be referred immediately to the Grievance Arbitrator. In any such case, the dispute or alleged breach shall be asserted by notice in writing given to the other party or parties, the NBA, the Players Association, and the Grievance Arbitrator.
\item
  The Grievance Arbitrator shall convene a hearing with respect to such dispute or alleged breach at the earliest possible time, but in no event later than twenty-four (24) hours following his receipt of such notice. If the Grievance Arbitrator is not immediately available and the parties are unable to agree upon another arbitrator to hear and resolve such dispute, the parties shall select an arbitrator in accordance with the procedures set forth in Section 7(b) above.
\item
  The award, which shall be issued not later than twenty-four (24) hours after the conclusion of the hearing, shall be in writing and may be issued with or without opinion. If any party desires an opinion, one shall be issued but its issuance shall not delay compliance with or enforcement of the award. The award shall constitute full, final, and complete disposition of the dispute or alleged breach, and shall be binding upon the player(s) and Team(s) involved and the parties to this Agreement.
\item
  The failure of any party to attend the hearing as scheduled shall not delay the hearing, and the Grievance Arbitrator (or an arbitrator selected in accordance with the procedures set forth in Section 7(b) above, as the case may be) shall be authorized to proceed to take evidence and issue an award as though such party were present.
\end{enumerate}

\hypertarget{threshold-amount-for-certain-grievances.}{%
\section{Threshold Amount for Certain Grievances.}\label{threshold-amount-for-certain-grievances.}}

A dispute concerning a fine or suspension (or a combination thereof) imposed by a Team may be heard and resolved by the Grievance Arbitrator only if it results in a financial impact on the player of more than \$5,000. A dispute concerning a fine or suspension (or a combination thereof) imposed by the Commissioner (or his designee) other than for conduct on the playing court (as defined in Section 9(c) above) may be heard and resolved by the Grievance Arbitrator only if it results in a financial impact on the player of more than \$50,000.

\hypertarget{miscellaneous.-1}{%
\section{Miscellaneous.}\label{miscellaneous.-1}}

\begin{enumerate}
\def\labelenumi{(\alph{enumi})}
\tightlist
\item
  Each of the time limits set forth herein may be extended by mutual agreement of the parties involved.
\item
  In any meeting or hearing provided for by this Article XXXI, a player may be accompanied by a representative of the Players Association who may participate in such meeting or hearing and represent the player. In any such meeting or hearing, the NBA and any Team involved may attend and be accompanied by a representative who may participate in such meeting or hearing and represent the NBA and any such Team.
\item
  The parties recognize that a player may be subjected to disciplinary action for just cause by his Team or by the Commissioner (or his designee). Therefore, in Grievances regarding discipline, the issue to be resolved shall be whether there has been just cause for the penalty imposed. Notwithstanding the foregoing, in all proceedings pursuant to Section 9(b) above, the Grievance Arbitrator shall apply an ``arbitrary and capricious'' standard of review as set forth in that Section.
\item
  Nothing contained herein shall excuse a player from prompt compliance with any discipline imposed upon him. If discipline imposed upon a player is determined to be improper by a final disposition under this Article XXXI, the player shall promptly be made whole.
\item
  Nothing contained in this Article XXXI shall be deemed to limit or impair the right of the NBA or any Team to impose discipline upon a player(s) or to take any other action not inconsistent with the provisions of a Player Contract or this Agreement.
\item
  Subject to Section 4(c) above, all costs of arbitration, including the fees and expenses of the Grievance Arbitrator, and all costs of the proceedings before the Player Discipline Arbitrator (including the fees and expenses of the Player Discipline Arbitrator), shall be borne equally by the parties thereto; but each party shall bear the cost of its own witnesses, counsel, and the like.
\item
  A Team shall not be required to terminate a Player Contract under the NBA waiver procedure as a condition precedent to the filing of a Grievance with respect to such Player Contract. To the extent that the decision of the Impartial Arbitrator in In re: Otis Birdsong, Dec.~No.~87-2 (May 14, 1987) is inconsistent with the foregoing, it is hereby overruled.
\item
  In a proceeding involving the interpretation of a Player Contract, no Uniform Player Contract (whether signed during the term of this Agreement or during the term of any prior collective bargaining agreement between the parties), or amendment thereto, other than the Player Contract or amendment that is the subject of dispute, shall be admissible as evidence of the meaning of, or of the parties' intentions with respect to, any individually-negotiated terms or provisions in the Player Contract or amendment that is the subject of dispute.
\end{enumerate}

\hypertarget{system-arbitration}{%
\chapter{SYSTEM ARBITRATION}\label{system-arbitration}}

\hypertarget{jurisdiction-and-authority.}{%
\section{Jurisdiction and Authority.}\label{jurisdiction-and-authority.}}

The NBA and the Players Association shall agree upon a System Arbitrator, who shall have exclusive jurisdiction to determine any and all disputes arising under Articles I, II, VII (except as otherwise specifically provided by Article VII, Section 3(d)(5)), VIII, X, XI, XII, XIII, XIV, XV, XVI, XXII, Section 14(j)(iii), Article XXIX, Section 6(b), XXXVII, XXXIX, and XL of this Agreement, any and all disputes arising under Article XXVIII and Paragraph 14 of the Uniform Player Contract regarding an Unauthorized Sponsor Promotion (as that term is defined in Paragraph 14(e) of the Uniform Player Contract), and those disputes made subject to his jurisdiction by Sections 9 and 10 of this Article. In addition, in the event of a disagreement between the NBA and the Players Association, the System Arbitrator shall have exclusive jurisdiction to determine whether the System Arbitrator, the Grievance Arbitrator, or some other arbitrator provided for by the provisions of this Agreement has jurisdiction to hear or resolve a particular dispute.

\hypertarget{initiation.-1}{%
\section{Initiation.}\label{initiation.-1}}

\begin{enumerate}
\def\labelenumi{(\alph{enumi})}
\tightlist
\item
  Subject to Article XIV, Section 5 and Article XXIX, Section 6(b), System Arbitrations may be initiated, as set forth below, only by the NBA or the Players Association.
\item
  No party may initiate a System Arbitration until and unless it has first discussed the matter with the other party in an attempt to settle it.
\item
  A System Arbitration must be initiated within three (3) years from the date of the act or omission upon which the System Arbitration is based, or within three (3) years from the date upon which such act or omission became known or reasonably should have become known to the party initiating the System Arbitration, whichever is later.
\item
  Either the NBA or the Players Association may initiate a System Arbitration by serving a written notice thereof on the other party, with a copy of such written notice to be filed with the System Arbitrator.
\end{enumerate}

\hypertarget{hearings.-1}{%
\section{Hearings.}\label{hearings.-1}}

\begin{enumerate}
\def\labelenumi{(\alph{enumi})}
\tightlist
\item
  The System Arbitrator shall hold hearings on alleged violations of the Articles set forth in Section 1 above. Except as otherwise provided in Article X, Section 10; Article XI, Section 5(p); Article XIII, Section 5; Article XXIX, Section 6(b); and Sections 9 and 10 below, awards issued by the System Arbitrator shall be subject to review by the Appeals Panel, in the manner and in accordance with the procedures set forth in Sections 3 and 8 of this Article XXXII.
\item
  The System Arbitrator shall make findings of fact and award appropriate relief including, without limitation, damages, injunctive relief, and specific performance; provided, however, that the System Arbitrator shall not have the authority to impose an award of punitive damages on any party. The System Arbitrator shall render an award as soon as practicable, and the award shall be accompanied by a written opinion. Notwithstanding the foregoing, if the System Arbitrator determines that expedition so requires, he/she shall accompany the award with a written summary of the grounds upon which the award is based, and a full written opinion may follow within a reasonable time thereafter. In no event shall the award and written opinion be issued more than thirty (30) days following the date upon which the record of a System Arbitration proceeding is closed (or, where applicable, the date designated by the System Arbitrator for the submission of post-hearing briefs).
\item
  The System Arbitrator shall have authority to order the production of documents, the conduct of pre-hearing depositions, and the attendance of witnesses at the hearing with respect to the NBA and the Players Association, and/or any player or Team. The System Arbitrator shall have the authority to compel the attendance of witnesses and the production of documents at any hearing within the jurisdiction of the System Arbitrator in accordance with the New York C.P.L.R.
\item
  An award of the System Arbitrator shall upon its issuance constitute the full, final, and complete disposition of the dispute, shall be binding upon the parties to this Agreement and upon any player(s) or Team(s) involved, and shall be followed by them unless (in cases where this Agreement provides for an appeal to the Appeals Panel) a notice of appeal is served by the appealing party upon the responding party and filed with the System Arbitrator within ten (10) days of the date of the award of the System Arbitrator appealed from. If and when an award of the System Arbitrator is reversed or modified by the Appeals Panel, the effect of such reversal or modification shall be deemed by the parties to be retroactive to the time of issuance of the award of the System Arbitrator. The parties may seek appropriate relief to effectuate and enforce this provision.
\item
  The System Arbitrator shall not have jurisdiction or authority to add to, detract from, or alter in any way the provisions of this Agreement or any Player Contract. Nor, except for the authority conferred upon him/her by the second sentence of Section 1 above (or unless the NBA and the Players Association otherwise agree), shall the System Arbitrator have jurisdiction or authority to resolve questions of substantive, as opposed to procedural, arbitrability (which shall include the question of whether an arbitrator provided for by the terms of this Agreement, as opposed to the Commissioner (or his/her designee), has jurisdiction to hear or resolve a particular dispute), which shall be determined in a judicial proceeding to be venued in the United States District Court for the Southern District of New York.
\end{enumerate}

\hypertarget{costs-relating-to-system-arbitration.}{%
\section{Costs Relating to System Arbitration.}\label{costs-relating-to-system-arbitration.}}

\begin{enumerate}
\def\labelenumi{(\alph{enumi})}
\tightlist
\item
  The compensation of the System Arbitrator and the costs and expenses incurred in connection with any proceeding brought before the System Arbitrator shall be borne equally by the parties to this Agreement; provided, however, that each participant in such proceeding shall bear its own attorneys' fees and litigation costs.
\item
  Notwithstanding the provisions of Section 4(a) above, if a matter is scheduled for hearing under this Article XXXII, and the hearing date is thereafter postponed at the request of either the NBA or the Players Association, the postponement fee (if any) of the System Arbitrator will be borne by the party requesting the postponement unless that party objects and the System Arbitrator finds that the request for such postponement was for good cause. Should good cause be found, the parties will share any postponement fee equally.
\end{enumerate}

\hypertarget{procedure-for-system-arbitration.}{%
\section{Procedure for System Arbitration.}\label{procedure-for-system-arbitration.}}

All matters before the System Arbitrator shall be heard and determined in an expedited manner, provided that such expedition is reasonable under the circumstances. A proceeding may be commenced upon seventy-two (72) hours' written notice (or upon shorter notice if ordered by the System Arbitrator) served upon the party against whom the proceeding is brought and filed with the System Arbitrator. All such notices and all orders and notices issued and directed by the System Arbitrator shall be served on the NBA, counsel for the NBA, the Players Association, counsel for the Players Association, and any counsel appearing for individual NBA players or individual NBA Teams. In any proceeding commenced pursuant to Article XIV, Section 5, the Players Association (on its own behalf and/or on behalf of a player) and the NBA (on its own behalf and/or on behalf of a Team) shall have the right to participate.

\hypertarget{selection-of-system-arbitrator.}{%
\section{Selection of System Arbitrator.}\label{selection-of-system-arbitrator.}}

\begin{enumerate}
\def\labelenumi{(\alph{enumi})}
\tightlist
\item
  In the event that the Players Association and the NBA cannot agree on the identity of a System Arbitrator, the parties shall jointly request the International Institute for Conflict Prevention and Resolution (the ``CPR Institute'') (or such other organization(s) as the parties may have agreed upon) to submit to the parties a list of eleven (11) attorneys, none of whom shall have, nor whose firm shall have, represented within the past five (5) years any professional athletes; agents or other representatives of professional athletes; labor organizations representing athletes; sports leagues, governing bodies, or their affiliates; sports teams or their affiliates; or owners in any professional sport. If the parties cannot within seven (7) days from the receipt of such list agree to the identity of the System Arbitrator from among the names on such list, they shall return said list, with up to five (5) names deleted therefrom by each party, to the CPR Institute (or such other organization as the parties may have agreed upon), which shall choose from the remaining name(s) on the list the identity of the System Arbitrator.
\item
  Effective July 1, 2023, the System Arbitrator selected by the parties shall serve for continually renewing two-year terms unless notice of termination is given either by the NBA or by the Players Association. Notice of termination of the System Arbitrator shall be given to the other party, and to the System Arbitrator, during the period May 10 through May 15 immediately preceding the end of any term. Following the giving of such notice, a new System Arbitrator shall be selected in accordance with the procedures set forth in Section 6(a) above. A System Arbitrator as to whom a notice of termination has been given shall continue to have jurisdiction only with respect to (i) System Arbitrations in which a hearing has been commenced or scheduled for a date certain, and (ii) System Arbitrations initiated (in accordance with the provisions of Section 2 above) within the thirty (30) day period preceding the service of the notice of termination; provided, however, that a hearing with respect to System Arbitrations referred to in this subsection (ii) must commence no later than thirty (30) days following the end of a System Arbitrator's term.
\end{enumerate}

\hypertarget{selection-of-appeals-panel.}{%
\section{Selection of Appeals Panel.}\label{selection-of-appeals-panel.}}

\begin{enumerate}
\def\labelenumi{(\alph{enumi})}
\tightlist
\item
  There shall be a three-member Appeals Panel for each appeal noticed from an award of the System Arbitrator. In the event the Players Association and the NBA cannot agree upon the members of such a panel, the parties will jointly request the CPR Institute (or such other organization(s) as the parties may agree) to submit to the parties a list of fifteen (15) attorneys (none of whom shall have, nor whose firm shall have, represented within the past five (5) years any professional athletes; agents or other representatives of professional athletes; labor organizations representing athletes; sports leagues, governing bodies, or their affiliates; sports teams or their affiliates; or owners in any professional sport). If the parties cannot within seven (7) days from the receipt of such list agree to the identity of the Appeals Panel from among the names on such list, they shall meet and alternate striking one (1) name at a time from the list until three (3) names on the list remain. The three (3) remaining names on the list shall comprise the Appeals Panel.
\item
  Effective July 1, 2023, the members of the Appeals Panel selected by the parties shall serve for continually renewing one-year terms unless notice of termination is given either by the NBA or by the Players Association. On or before June 30, 2024, and on or before each other successive June 30 during the term of this Agreement, either party may discharge one or more members from the Appeals Panel by serving notice of termination on him/her on or before that date and upon the other party to this Agreement, and the discharge shall be effective as of such June 30. A discharged Appeals Panel member may participate in decisions rendered by the Appeals Panel in all cases previously heard to closure of the record, but not participate in the consideration or decision of any other cases. If a member of the Appeals Panel is not discharged as provided above, the member's term will automatically be renewed for an additional year. The compensation of the members of the Appeals Panel and the costs of proceedings before the Appeals Panel shall be borne equally by the parties to this Agreement; provided, however, that each participant in an Appeals Panel proceeding shall bear its own attorneys' fees and litigation costs.
\end{enumerate}

\hypertarget{procedure-relating-to-appeals-of-determination-by-the-system-arbitrator.}{%
\section{Procedure Relating to Appeals of Determination by the System Arbitrator.}\label{procedure-relating-to-appeals-of-determination-by-the-system-arbitrator.}}

\begin{enumerate}
\def\labelenumi{(\alph{enumi})}
\tightlist
\item
  Any party seeking to appeal (in whole or in part) an award of the System Arbitrator must serve on the other party and file with the System Arbitrator a notice of appeal, within ten (10) days of the date of the award appealed from. The timely service and filing of a notice of appeal shall automatically stay the award of the System Arbitrator pending resolution by the Appeals Panel.
\item
  Following the timely service and filing of a notice of appeal, the NBA and the Players Association shall attempt to agree upon a briefing schedule. In the absence of such agreement, the briefing schedule shall be set by the Appeals Panel; provided, however, that any party seeking to appeal (in whole or in part) from an award of the System Arbitrator shall be afforded no less than fifteen (15) and no more than twenty-five (25) days from the date of the issuance of such award, or the date of the issuance of the System Arbitrator's written opinion, or the date upon which the members of the Appeals Panel have been selected in accordance with the provisions of Section 7 above, whichever is latest, to serve on the opposing party and file with the Appeals Panel its brief in support thereof; and provided further that the responding party or parties shall be afforded the same aggregate amount of time to serve and file its or their responding brief(s). The Appeals Panel shall schedule oral argument on the appeal(s) no less than five (5) and no more than ten (10) days following the service and filing of the responding brief(s), and shall issue a written decision within thirty (30) days from the date of argument.
\item
  The Appeals Panel shall review the findings of fact and conclusions of law made by the System Arbitrator using the standards of review employed by the United States Court of Appeals for the Second Circuit. The decision of the Appeals Panel shall constitute full, final, and complete disposition of the dispute, and shall be binding upon the parties to this Agreement and upon any player(s) or Team(s) involved.
\end{enumerate}

\hypertarget{special-procedure-for-disputes-with-respect-to-interim-audit-reports.}{%
\section{Special Procedure for Disputes with Respect to Interim Audit Reports.}\label{special-procedure-for-disputes-with-respect-to-interim-audit-reports.}}

\begin{enumerate}
\def\labelenumi{(\alph{enumi})}
\tightlist
\item
  Notwithstanding any of the other provisions of this Agreement, at the request of either the NBA or the Players Association, and irrespective of which party may commence the proceeding, the procedures set forth in this Section 9 shall apply to the resolution of any disputes with respect to an Interim Audit Report (as defined in Article VII, Section 10(a) above), including but not limited to disputes concerning any Designated Share Information set forth in an Interim Audit Report. If in connection with such disputes, there is any conflict between the procedures set forth in this Section 9 and those set forth elsewhere in this Agreement, the procedures set forth in this Section shall control.
\item
  A proceeding before the System Arbitrator shall be commenced, in the manner provided for by Sections 2(d) and 5 above, no more than thirty (30) days following the delivery by the Accountants (as defined in Article VII, Section 10(a) above) of the Interim Audit Report with respect to any dispute or claim concerning (i) the amount(s) of BRI or Total Salaries (or portions thereof) as to which the Accountants have completed their review and which is the subject of a good faith dispute between the parties, (ii) the amount(s) of BRI or Total Salaries (or portions thereof) as to which the Accountants have not completed their review and with respect to which the parties have a good faith disagreement, (iii) such Designated Share Information (as defined in Article VII, Section 10(a) above) as is included in the Interim Audit Report as to which the parties have a good faith disagreement, and/or (iv) all other disputes (including but not limited to disputes over the amounts and includability of any revenues or expenses included or excluded from the Interim Audit Report) of which the parties were aware or reasonably should have been aware, at the time the proceeding was commenced, based upon the contents of the BRI Reports, the Draft Audit Report, or Interim Audit Report, or other documents or writings provided to the parties by the Accountants in connection with their BRI audit.
\item
  A party's failure to commence a proceeding before the System Arbitrator within the thirty (30) day period provided for by Section 9(b) above with respect to the disputes or claims enumerated therein shall forever bar that party from asserting or seeking relief of any kind for any such dispute or claim; provided, however, that the provisions of Section 9(b) above and this Section 9(c) shall not bar a party from commencing a proceeding before the System Arbitrator and seeking appropriate relief, subject to the limitations imposed by Section 2 above:

  \begin{enumerate}
  \def\labelenumii{(\roman{enumii})}
  \tightlist
  \item
    With respect to a dispute or claim concerning an Interim Audit Report as to which such party was not aware or reasonably should not have been aware, based upon the materials referred to in Section 9(b) above, during the thirty (30) day period following the delivery of such Interim Audit Report; or
  \item
    With respect to any dispute or claim relating to a subsequent Salary Cap Year, including, but not limited to, any dispute concerning the includability or non-includability in BRI of a category or type of revenue or the allowance or disallowance of a category or type of expense, without regard to whether, based upon the materials referred to in Section 9(b) above (other than a BRI Report, Draft Audit Report, or Interim Audit Report), the party was or reasonably should have been aware of such dispute or claim during the thirty (30) day period following the delivery of such Interim Audit Report.
  \item
    Subject to Section 9(c)(ii) above, no determination made by the System Arbitrator or the Appeals Panel (as the case may be) in a proceeding commenced pursuant to Section 9(c)(i) or (ii) above shall affect any calculations made pursuant to Article VII, Section 12.
  \end{enumerate}
\item
  Where a hearing before the System Arbitrator is provided for by this Section 9, such hearing shall be conducted within fifteen (15) days from the commencement of the proceeding, and the System Arbitrator shall render an award and issue a written decision as soon as possible, but in no event later than fifteen (15) days following the close of the hearing. Where a right to appeal from the System Arbitrator's award is provided for by this Section 9, any party seeking to appeal (in whole or in part) from such an award shall serve and file a notice of appeal therefrom within five (5) days from the date of such award and shall serve and file its brief in support of such appeal within fifteen (15) days from the date of the System Arbitrator's award or within five (5) days from the date upon which the members of the Appeals Panel have been selected, whichever is later. The party opposing such appeal shall serve and file its brief in opposition within ten (10) days following its receipt of the brief in support of the appeal. The Appeals Panel shall schedule oral argument at its discretion, but shall issue a written decision within twenty (20) days following its receipt of the brief from the party opposing the appeal.
\item
  Any dispute concerning the amounts (as opposed to the includability) of any revenues or expenses to be included in an Interim Audit Report (hereinafter referred to as ``Disputed Adjustments'') shall, whenever such Disputed Adjustments for all Teams are adverse to the party asserting the dispute in an aggregate amount of less than \$10 million for any Season covered by this Agreement, be resolved by the Accountants; and the determination of the Accountants shall constitute full, final, and complete disposition of the dispute and shall be binding upon the parties to this Agreement. Notwithstanding the foregoing, any Disputed Adjustments that involve the interpretation, validity, or application of this Agreement shall be resolved by the System Arbitrator and shall be appealable to the Appeals Panel in accordance with the provisions of Section 9(d) above.
\item
  If the Disputed Adjustments for all Teams are adverse to the party asserting the dispute in an aggregate amount of \$10 million or more but less than \$15 million for any Season covered by this Agreement, the determination of the System Arbitrator shall constitute full, final, and complete disposition of the dispute and shall be binding upon the parties to this Agreement, and there shall be no appeal to the Appeals Panel. Notwithstanding the foregoing, any Disputed Adjustments that involve the interpretation, validity or application of this Agreement shall be resolved by the System Arbitrator and shall be appealable to the Appeals Panel in accordance with the provisions of Section 9(d) above.
\item
  If the Disputed Adjustments for all Teams are adverse to the party asserting the dispute in an aggregate amount of \$10 million or more but less than \$15 million for any Season covered by this Agreement, and if the party asserting such dispute does not prevail before the System Arbitrator, then that party shall pay all of the fees and expenses of the System Arbitrator and the reasonable costs and expenses, including attorneys' fees, of the other party for its defense of the proceeding; provided, however, that if each party has asserted a dispute upon which it has not prevailed, all such fees, expenses and costs shall be borne in the manner provided for by Section 4 above.
\item
  All other disputes involving an Interim Audit Report (including but not limited to disputes over the amounts and includability of any revenues or expenses to be included in such Reports) and the Designated Share Information shall be resolved by the System Arbitrator and shall be appealable to the Appeals Panel in accordance with the provisions of Section 9(d) above.
\end{enumerate}

\hypertarget{special-procedure-for-disputes-with-respect-to-the-adjustment-schedules.}{%
\section{Special Procedure for Disputes with Respect to the Adjustment Schedules.}\label{special-procedure-for-disputes-with-respect-to-the-adjustment-schedules.}}

\begin{enumerate}
\def\labelenumi{(\alph{enumi})}
\tightlist
\item
  Notwithstanding any of the other provisions of this Agreement, the procedures set forth in this Section 10 shall apply to the resolution of any disputes with respect to the Adjustment Schedules described in Article VII, Section 12. If in connection with such disputes, there is any conflict between the procedures set forth in this Section 10 and those set forth elsewhere in this Agreement, the procedures set forth in this Section shall control.
\item
  In the event of any dispute with respect to the Adjustment Schedules, the proceeding before the System Arbitrator shall be commenced, in the manner provided for by Sections 2(d) and 5 above, no more than seven (7) days following the transmittal to the Players Association of any of such schedules.
\item
  The hearing before the System Arbitrator with respect to a dispute concerning the Adjustment Schedules shall be conducted within ten (10) days following the commencement of the proceeding and the briefs of the parties, if any, shall be filed before the opening of the hearing on a date or dates set by the System Arbitrator. The hearing shall be conducted on an expedited basis and, unless the parties otherwise agree or a party demonstrates that such limitation will result in undue prejudice, will not last longer than two (2) full days.
\item
  If in connection with the Adjustment Schedules, there is a dispute between the NBA and the Players Association and the amount in controversy is \$5 million or less, the determination of the System Arbitrator shall constitute full, final, and complete disposition of the dispute and shall be binding upon the parties to this Agreement, and there shall be no appeal to the Appeals Panel. If with respect to such dispute the amount in controversy is more than \$5 million, either party may appeal a determination of the System Arbitrator to the Appeals Panel.
\item
  In connection with any dispute concerning the Adjustment Schedules, the System Arbitrator shall render an award and issue a written decision as soon as possible, but in no event later than ten (10) days following the close of the hearing. When the award is issued, the System Arbitrator shall set forth the basis therefore either in a written opinion or orally at a conference with the parties (which conference may be conducted by telephone) of which a stenographic record shall be made. Any party seeking to appeal (in whole or in part) from an award of the System Arbitrator rendered pursuant to Section 10(d) above shall serve and file a notice of appeal therefrom within two (2) business days from the date of such award. The party seeking to appeal shall serve and file its brief in support of such appeal within ten (10) days from the date of the System Arbitrator's award or within three (3) days from the date upon which the members of the Appeals Panel have been selected, whichever is later. The party opposing such appeal shall serve and file its brief in opposition within ten (10) days following its receipt of the brief in support of the appeal. The Appeals Panel shall schedule oral argument at its discretion, but shall issue a written decision within twenty (20) days following its receipt of the brief from the party opposing the appeal.
\end{enumerate}

\hypertarget{anti-drug-program-and-substance-abuse-treatment}{%
\chapter{ANTI-DRUG PROGRAM AND SUBSTANCE ABUSE TREATMENT}\label{anti-drug-program-and-substance-abuse-treatment}}

\hypertarget{definitions.-2}{%
\section{Definitions.}\label{definitions.-2}}

As used in this Article XXXIII, the following terms shall have the following meanings:

\begin{enumerate}
\def\labelenumi{(\alph{enumi})}
\tightlist
\item
  ``Authorization for Testing'' shall mean a notice issued by the Independent Expert pursuant to the provisions of Section 5 below in the form annexed hereto as Exhibit I-1 to this Agreement.
\item
  ``Benzodiazepines'' shall mean any of the substances listed as benzodiazepines on Exhibit I-2 to this Agreement.
\item
  ``Come Forward Voluntarily'' shall mean that a player has directly communicated to the Medical Director his desire to enter the Program and seek treatment for a problem involving the use of a Drug of Abuse or synthetic cannabinoid. Such communication may be facilitated by a representative of the NBA or the Players Association (e.g., by arranging a conference call among the player, the Medical Director, and such representative in which this communication occurs). A player may not Come Forward Voluntarily if, prior to his direct communication to the Medical Director, the NBA has been notified by the applicable Program laboratory that the player's most recent drug test was positive or produced an atypical finding for a Drug of Abuse or synthetic cannabinoid. A player may not Come Forward Voluntarily for the use of a SPED or Diuretic.
\item
  ``Counselors'' shall mean the persons selected by the Medical Director to provide counseling and other treatment to players in the Program.
\item
  ``Diuretics'' shall mean any of the substances listed as diuretics on Exhibit I-2 to this Agreement.
\item
  ``Drugs of Abuse'' shall mean any of the substances listed as drugs of abuse on Exhibit I-2 to this Agreement.
\item
  ``Drugs of Abuse Program'' shall mean (i) the testing program for Drugs of Abuse set forth in this Article XXXIII, and (ii) the education, treatment, and counseling program for Drugs of Abuse established by the Medical Director (after consultation with the NBA and the Players Association), which may contain such elements---including, but not limited to, urine, blood, breath, or other testing for Prohibited Substances other than SPEDs---as may be determined by the Medical Director in his or her professional judgment.
\item
  ``First-Year Player'' shall mean a player under Contract to an NBA Team who, prior to the then-current Season, has not been on the roster of an NBA Team following the first game of a Regular Season.
\item
  ``HGH Blood Testing'' shall mean the collection and testing of blood samples for Human Growth Hormone via dried blood spots.
\item
  ``In-Patient Facility'' shall mean such treatment center or other facility as may be selected by the Medical Director and agreed upon by the NBA and the Players Association.
\item
  ``Independent Expert'' or ``Expert'' shall mean the person selected by the NBA and the Players Association in accordance with Section 2(c) below.
\item
  ``Marijuana and Alcohol Treatment Programs'' shall mean the education, treatment, and counseling programs for marijuana and alcohol established by the Medical Director (after consultation with the NBA and the Players Association), which may contain such elements---including, but not limited to, urine, blood, breath, or other testing for marijuana, alcohol, or Prohibited Substances other than SPEDs---as may be determined by the Medical Director in his or her professional judgment.
\item
  ``Medical Director'' shall mean the person selected by the NBA and the Players Association in accordance with Section 2(a) below.
\item
  ``Off-Season'' shall mean, for any given player, the period beginning the day after the last game of that player's Team's Season and ending the day before the first day of that player's Team training camp.
\item
  ``Prohibited Substance'' shall mean any of the substances listed on Exhibit I-2 to this Agreement and any other substance added to such Exhibit under the provisions of Section 17 below.
\item
  ``Program'' shall mean this Anti-Drug Program, and shall include the Drugs of Abuse Program, the Marijuana and Alcohol Treatment Programs, the SPED Program, and the Synthetic Cannabinoid Program.
\item
  ``Prohibited Substances Committee'' shall mean the committee selected by the NBA and the Players Association in accordance with Section 2(e) below.
\item
  ``SPED'' shall mean any of the steroids, performance-enhancing drugs, and masking agents (other than Diuretics) listed on Exhibit I-2 to this Agreement.
\item
  ``SPED Medical Director'' shall mean the person selected by the NBA and the Players Association in accordance with Section 2(b) below.
\item
  ``SPED Program'' shall mean the (i) testing program for SPEDs and Diuretics (but not for any other Prohibited Substance) set forth in this Article XXXIII, and (ii) the education, treatment, and counseling program for SPEDs and Diuretics established by the SPED Medical Director (after consultation with the NBA and the Players Association), which may contain such elements---including, but not limited to, urine, blood, breath, or other testing for SPEDs and Diuretics (but not for any other Prohibited Substance)---as may be determined by the SPED Medical Director in his or her professional judgment.
\item
  ``Synthetic Cannabinoid Program'' shall mean the (i) testing program for synthetic cannabinoids (but not for any other Prohibited Substance) set forth in this Article XXXIII, and (ii) the education, treatment, and counseling program for synthetic cannabinoids established by the Medical Director (after consultation with the NBA and the Players Association), which may contain such elements---including, but not limited to, urine, blood, breath, or other testing for Prohibited Substances other than SPEDs---as may be determined by the Medical Director in his or her professional judgment.
\item
  ``Tender'' shall mean an offer of a Uniform Player Contract, signed by the Team, that is either personally delivered to the player or his representative or sent by prepaid certified, registered, or overnight mail to the last known address of the player or his representative.
\item
  ``Veteran Player'' shall mean any player who is not a First-Year Player.
\end{enumerate}

\hypertarget{administration.}{%
\section{Administration.}\label{administration.}}

\begin{enumerate}
\def\labelenumi{(\alph{enumi})}
\item
  The NBA and the Players Association shall jointly select a Medical Director who shall be a person experienced in the field of testing and treatment for substance abuse. The Medical Director shall have the responsibility, among other duties, for selecting and supervising the Counselors and other personnel necessary for the effective implementation of the Drugs of Abuse, Marijuana and Alcohol Treatment, and Synthetic Cannabinoid Programs; for making medical review determinations for Prohibited Substances other than SPEDs and Diuretics; for evaluating and treating players subject to the Drugs of Abuse, Marijuana and Alcohol Treatment, and Synthetic Cannabinoid Programs; and for otherwise managing and overseeing such Programs, subject to the control of the NBA and the Players Association. To the extent practicable, the Medical Director shall select qualified retired NBA players to serve as Counselors.
\item
  The NBA and the Players Association shall jointly select a SPED Medical Director who shall be a medical doctor, preferably specializing in internal or sports medicine, with experience in the field of testing and treatment for steroids and performance-enhancing drugs. The SPED Medical Director shall have the responsibility, among other duties, for making medical review determinations for SPEDs and Diuretics, for evaluating and treating players subject to the SPED Program, and for otherwise managing and overseeing the SPED Program, subject to the control of the NBA and the Players Association.
\item
  The NBA and the Players Association shall jointly select an Independent Expert who shall be a person experienced in the field of substance abuse detection and enforcement and shall be authorized to issue Authorizations for Testing in accordance with Section 5 below.
\item
  The Medical Director, the SPED Medical Director, and the Independent Expert shall all serve for the duration of this Agreement, unless either the NBA or the Players Association has, by September 1 of any year covered by this Agreement, served written notice of discharge upon the other party and, as appropriate, the Medical Director, SPED Medical Director, and/or the Independent Expert. Such notice of discharge shall be effective as of the immediately following September 30; provided, however, that if the parties do not reach agreement by such September 30 as to who shall serve thereafter as the Medical Director, SPED Medical Director, and/or the Independent Expert, as the case may be, each party shall, by the immediately following October 15, appoint a person who shall have no relationship to or affiliation with that party. Such persons shall then have until the immediately following December 1 to agree on the appointment of a new Medical Director, SPED Medical Director, and/or Independent Expert. Until a new Medical Director, SPED Medical Director, and/or Independent Expert has been appointed, the previous Medical Director, SPED Medical Director, and/or Independent Expert shall continue to serve.
\item
  \begin{enumerate}
  \def\labelenumii{(\roman{enumii})}
  \tightlist
  \item
    The NBA and the Players Association shall form a Prohibited Substance Committee, which shall be comprised of one (1) representative from the NBA, one (1) representative from the Players Association, and three (3) individuals jointly selected by the NBA and the Players Association who shall be experts in the field of testing and treatment for drugs of abuse and performance-enhancing substances. The members of this Committee shall serve for the duration of the Agreement.
  \item
    The members of the Prohibited Substances Committee shall meet (either in person or by conference call) at least once each Season and once each off-season (the ``Annual Meetings''). The Annual Meetings shall be scheduled by the NBA after consultation with the Players Association. At the Annual Meetings, the Committee shall review the Program's list of Prohibited Substances, and discuss general anti-doping issues (including, but not limited to, advances in drug testing science and technology, and modifications to relevant anti-doping policies of other sports organizations). The Committee shall also make recommendations to the NBA and the Players Association for changes to the list of Prohibited Substances (including the determination of laboratory analysis cutoff levels).
  \item
    As of September 1, 2023, and as of each successive September 1, either of the parties to this Agreement may discharge any jointly-selected member of the Prohibited Substances Committee by serving thirty (30) days' prior notice upon that person and upon the other party to this Agreement. In the case of such discharge, or in the event a Committee member resigns, and if the parties are unable to agree on a replacement Committee member within thirty (30) days, then the parties shall request a list of seven (7) names of potential replacements prepared by the Medical Director and any remaining jointly-selected Committee members, and, within seven (7) days, shall select the necessary replacement by alternately striking names from the list until only one (1) remains.
  \end{enumerate}
\item
  Unless specifically stated otherwise in this Article XXXIII, all costs of the Program in excess of those covered by the NBA Players Group Health Plan, including the fees and expenses of the Medical Director, the SPED Medical Director, the Independent Expert, and the Prohibited Substances Committee shall be shared equally by the NBA and Players Association. The Players Association's share shall be paid by the NBA and included in Player Benefits under Article IV, Section 6(g) of this Agreement. The fees and expenses incurred by the NBA in conducting testing pursuant to Sections 5, 6, and 16 below shall be borne by the NBA.
\item
  Any and all disputes arising under this Article XXXIII shall be resolved in accordance with Article XXXI, Sections 2-7 and 15 of this Agreement; provided, however, that in any challenge to a decision, recommendation, or other conduct of the Medical Director, SPED Medical Director, Independent Expert, or Prohibited Substances Committee, or in any challenge to an action or process over which the Medical Director or the SPED Medical Director has supervision, the Grievance Arbitrator shall apply an ``arbitrary and capricious'' standard of review; and provided further that nothing in this Section 2(g) shall limit or otherwise affect Paragraph 19 of the Uniform Player Contract. Notwithstanding the foregoing, neither party, nor any player or Team, may challenge a determination made by the applicable Program laboratory of whether the estimated or adjusted concentration of a Prohibited Substance that is subject to a confirmatory laboratory analysis level set forth in Exhibit I-6 exceeds the relevant single-point calibrator in a player's ``A'' and/or ``B'' sample.
\end{enumerate}

\hypertarget{confidentiality.}{%
\section{Confidentiality.}\label{confidentiality.}}

\begin{enumerate}
\def\labelenumi{(\alph{enumi})}
\tightlist
\item
  Other than as reasonably required in connection with the suspension or disqualification of a player, the NBA, the Teams, and the Players Association, and all of their members, affiliates, agents, consultants, and employees, are prohibited from publicly disclosing information about the diagnosis, treatment, prognosis, test results, compliance, or the fact of participation of a player in the Program (``Program Information''); provided, however, (i) if a player is suspended or disqualified for conduct involving a Drug of Abuse, Diuretic, synthetic cannabinoid, distribution of marijuana, or for failing to comply with his treatment program as prescribed and determined by the Medical Director or SPED Medical Director (as applicable), the NBA may publicly disclose the applicable penalty (but may not, for clarity, publicly disclose the particular Prohibited Substance involved, absent the agreement of the Players Association or the prior disclosure of such information by the player or by a person authorized by the player to disclose such information), and (ii) if a player is suspended or disqualified for conduct involving a SPED, the particular SPED shall be publicly disclosed along with the announcement of the applicable penalty.
\item
  The Medical Director, the SPED Medical Director, and the Counselors, and all of their affiliates, agents, consultants, and employees, are prohibited from publicly disclosing Program Information; provided, however, that the Medical Director and the SPED Medical Director shall not be prohibited from disclosing such information to the NBA and the Players Association.
\item
  The Independent Expert is prohibited from publicly disclosing any information supplied to him by the NBA or the Players Association pursuant to Section 5 below.
\item
  Members of the Prohibited Substances Committee are prohibited from publicly disclosing any information obtained by them in connection with their duties as Committee members. If a jointly-selected member of the Committee violates this Section 3(d), such member shall be immediately discharged from the Committee.
\item
  Any Program Information that is publicly disclosed (i) under Section 3(a) above, (ii) by the player, (iii) with the player's authorization, or (iv) through release by sources other than the NBA, NBA Teams, the Players Association, the Medical Director, the Counselors, the SPED Medical Director, the Independent Expert, or the Prohibited Substances Committee, or any of their members, affiliates, agents, consultants, and employees, will, after such disclosure, no longer be subject to the confidentiality provisions of this Section 3.
\item
  Other than as reasonably required by the Reasonable Cause Testing procedure set forth in Section 5 below, neither the NBA nor the Players Association shall divulge to any other person or entity (including their respective members, affiliates, agents, consultants, employees, and the player and Team involved):

  \begin{enumerate}
  \def\labelenumii{(\roman{enumii})}
  \tightlist
  \item
    that it has received information regarding the use, possession, or distribution of a Prohibited Substance by a player;
  \item
    that it is considering requesting, has requested, or has had a conference with the Independent Expert concerning the suspected use, possession, or distribution of a Prohibited Substance by a player;
  \item
    any information disclosed to the Independent Expert; or
  \item
    the results of any conference with the Independent Expert.
  \end{enumerate}
\item
  Notwithstanding anything to the contrary contained in Sections 3(a)-(f) above, the NBA and the Players Association shall promptly advise and make available to each other all information either of them may have in their possession, custody, or control that provides cause to believe that a player is engaged in the use, possession, or distribution of a Prohibited Substance.
\item
  Notwithstanding anything to the contrary contained in Sections 3(a)-(f) above, if a player (1) has tested positive for a Prohibited Substance and is subject to a potential suspension or dismissal and disqualification under this Article XXXIII by virtue of such positive test (e.g., because such positive test has not been deemed negative due to a determination by the Medical Director or SPED Medical Director (as applicable) that there is a valid alternative medical explanation for the test result, or (2) is otherwise in violation of the Program and, as a result, is subject to a potential suspension or dismissal and disqualification under this Article XXXIII by reason of such violation (e.g., because of his noncompliance with treatment or failure to cooperate with the testing process), then:

  \begin{enumerate}
  \def\labelenumii{(\roman{enumii})}
  \tightlist
  \item
    If, while the player is subject to a potential suspension or dismissal and disqualification pursuant to Section 3(h)(1) or (2) above, the player commences or is engaged in negotiations with a Team regarding a Player Contract or an amendment to a Player Contract (including an Extension, Renegotiation, or other amendment), the player shall immediately provide written notice of the positive test and/or the potential suspension or dismissal and disqualification to the Team. (For purposes of the foregoing sentence, ``immediately'' means (x) if the player commences such negotiations with the Team after being informed that he is subject to a potential suspension or dismissal and disqualification pursuant to Section 3(h)(1) or (2) above, upon commencement of such negotiations, and (y) if the player is informed that he is subject to a potential suspension or dismissal and disqualification pursuant to Section 3(h)(1) or (2) above after commencing such negotiations with the Team, within twenty-four (24) hours of being so informed, and in the case of either (x) or (y), prior to the execution of any Contract or amendment to such Contract.) If the player enters into a Player Contract or an amendment to a Player Contract with a Team, then the NBA shall promptly inform the Team of the player's positive test and/or potential suspension or dismissal and disqualification and notice obligation pursuant to this Section 3(h)(i). Within six (6) business days of being so informed by the NBA, the Team may request that the NBA render the player's Contract (or amendment, as the case may be) null and void if the Team believes that the player failed to provide the Team with written notice pursuant to this Section 3(h). If a Team makes such a request, and the NBA then determines that the player failed to provide the Team with such written notice, the player's Contract (or amendment, as the case may be) shall be rendered null and void and of no further force or effect.
  \item
    If, while the player is subject to a potential suspension or dismissal and disqualification pursuant to Section 3(h)(1) or (2) above, an assignment of the player's Contract is proposed to occur via a trade conference call with the NBA league office, then the NBA shall provide notice of the positive test and/or potential suspension or dismissal and disqualification, to the Teams involved in the trade of the player's Contract.
  \item
    If, while the player is subject to a potential suspension or dismissal and disqualification pursuant to Section 3(h)(1) or (2) above, the player has been placed on waivers and a Team claims the rights to the player, prior to notifying a Team that it has acquired such rights, the NBA shall provide written notice of the positive result and/or potential suspension or dismissal and disqualification to such Team. Upon receiving such notice, notwithstanding anything in this Agreement to the contrary, the Team shall have the right to withdraw its waiver claim pursuant to a process established by the NBA.
  \end{enumerate}

  In the event that the NBA provides notice to a Team pursuant to this Section 3(h), the NBA also shall inform the Team of (x) whether the testing of the split or ``B'' sample of the player's specimen is outstanding, and (y) the further process to which the player is subject under the Program.
\item
  Nothing contained in this Section 3 shall prohibit a Team from providing to the NBA information concerning whether a player is engaged in the use, possession, or distribution of a Prohibited Substance. For clarity, this Section 3(i) does not permit a Team to provide information to the NBA in violation of Section 18(d) below.
\end{enumerate}

\hypertarget{testing.}{%
\section{Testing.}\label{testing.}}

\begin{enumerate}
\def\labelenumi{(\alph{enumi})}
\tightlist
\item
  Testing conducted pursuant to this Article XXXIII, whether by the NBA, the Medical Director, or the SPED Medical Director, shall be conducted in compliance with scientifically accepted analytical techniques. Such testing shall also comply with Section 4(b) below, the collection procedures described in Exhibit I-3 (for urine collections) and Exhibit I-4 (for blood collections) to this Agreement, and such additional procedures and protocols as may be established by the NBA (after consultation with the Players Association) or the Medical Director or the SPED Medical Director, as applicable (after consultation with the NBA and the Players Association). The NBA and the Medical Director or the SPED Medical Director, as applicable (after consultation with the Players Association), are authorized to retain such consultants and support services as are necessary and appropriate to administer and conduct such testing.
\item
  If a player is selected for random drug testing pursuant to Section 6 below on a day he is scheduled to play a game, the following additional procedures will apply: (i) any blood testing must occur after the game; and (ii) for urine testing of a player on the visiting Team scheduled at game-day shoot-arounds, tests will be scheduled to occur before the shoot-around for that Team commences, and for any tests that are not completed by the time the visiting Team bus is scheduled to leave the arena or practice facility after the shoot-around is completed, the Team will provide alternate transportation to the team hotel for any player that must remain at the arena or practice facility to complete the testing process and will ensure that a Team staff member remains with the affected player(s) and accompanies him or them back to the Team's hotel.
\item
  All tests conducted pursuant to this Article XXXIII shall be analyzed by laboratories selected by the NBA and the Players Association, and certified by the World Anti-Doping Agency or the Substance Abuse and Mental Health Services Administration (SAMHSA).
\item
  Any test conducted pursuant to this Article XXXIII will be considered ``positive'' for a Prohibited Substance under the following circumstances:

  \begin{enumerate}
  \def\labelenumii{(\roman{enumii})}
  \tightlist
  \item
    If the test is for a Prohibited Substance other than a SPED or Diuretic and it is confirmed by laboratory analysis at the levels set forth in Exhibit I-5.
  \item
    If the test is for a SPED or Diuretic, and it is confirmed by laboratory analysis at the levels set forth in Exhibit I-6.
  \item
    If a player refuses to submit to a test or cooperate fully with the testing process, without a reasonable explanation satisfactory to the Medical Director or the SPED Medical Director (for testing under the SPED Program only); provided, however, that the NBA will use its best efforts (A) to have the drug testing collectors immediately notify the NBA when any player refuses to submit to a test or cooperate fully with the testing process, and (B) to provide such information to the Players Association as soon as possible thereafter; and provided, further, that (C) following any player's refusal to submit to a test or failure to cooperate fully with the testing process, the drug testing collector shall wait ninety (90) minutes at the collection site, and (D) if the player submits to the test and cooperates fully with the testing process within such additional time, then his earlier refusal or failure to cooperate shall be excused and the test shall not be deemed positive under this Section 4(d).
  \item
    If the player fails to submit to a scheduled test, without a reasonable explanation satisfactory to the Medical Director or SPED Medical Director (for testing under the SPED Program only).
  \item
    If the player attempts to substitute, dilute, or adulterate a specimen sample or in any other manner alter a test result (other than by testing positive for a Diuretic).
  \end{enumerate}
\item
  The NBA shall promptly notify the Players Association of any positive test conducted by the NBA, and shall thereafter notify the player. The Medical Director or the SPED Medical Director (as applicable) shall promptly notify the player of any positive test conducted by the Medical Director or SPED Medical Director (as applicable); provided, however, that if the positive test will result in a penalty to be imposed on the player, the Medical Director or SPED Medical Director (as applicable) shall notify the NBA and the Players Association of the positive test result and the NBA shall thereafter notify the player of such result and such penalty.
\item
  Upon notifying the Players Association of any positive test or atypical finding of an ``A'' sample, the NBA shall direct the testing of the split or ``B'' sample of the player's specimen. The test of the ``B'' sample will be performed at a laboratory other than the laboratory that performed the test of the original or ``A'' sample. Any such test shall be subject to the provisions of this Section 4. The NBA will notify the Players Association of the result of the test of the player's ``B'' sample and, if the result is positive, the Players Association may, within five (5) business days of the date of such notification, direct the NBA to request documentation package(s) for the player's ``A'' sample and ``B'' sample from the applicable Program laboratory. Within ten (10) business days of receiving a documentation package pursuant to the preceding sentence, the Players Association may hold a conference call among the NBA, the Players Association, and the applicable Program laboratory to request clarification of any information in such documentation package; provided, however, if it is impracticable to hold such conference call within ten (10) business days of receiving a documentation package, then the Players Association may instead seek clarification of any information in the documentation package via an email to the applicable Program laboratory (with the NBA copied), which email must be sent within ten (10) business days of receiving a respective documentation package.
\item
  Any positive test pursuant to Section 4(d)(i) above shall be reviewed by the Medical Director. Any positive test pursuant to Section 4(d)(ii) above shall be reviewed by the SPED Medical Director. If the Medical Director or SPED Medical Director (as applicable) determines, in his or her professional judgment, that there is a valid alternative medical explanation for such positive test result, then the test shall be deemed negative.
\item
  If the test result for any player is reported by the laboratory as ``invalid'' or ``endogenous steroids abnormally low,'' the NBA shall promptly notify the Players Association, and shall thereafter notify the player. In the event of such a test result, the player shall be required to submit to another test on a date determined by the NBA that is not more than thirty (30) days after the date of the original test (the ``Re-Test''). If the Re-Test results in (i) a positive test for a Drug of Abuse other than a Benzodiazepine or a positive test under Section 4(d)(iii), (iv), or (v) above, the player shall immediately be dismissed and disqualified from any association with the NBA or its Teams in accordance with the provisions of Section 12(a) below; (ii) a positive test for a synthetic cannabinoid, the player shall suffer the applicable consequences set forth in Section 10 below; (iii) a positive test for a SPED or Benzodiazepine, the player shall suffer the applicable consequences set forth in Section 9 below; or (iv) a positive test for a Diuretic, the player shall be deemed to have tested positive for a SPED and shall suffer the applicable consequences set forth in Section 9 below. The original test will not be counted towards the number of tests to be administered to that player for that Season under Section 6 (Random Testing) below.
\item
  For clarity, if the test result for any player reports a SPED or Diuretic at a detectable level below the confirmatory laboratory analysis levels set forth in Exhibit I-6 (and, for clenbuterol, above 0.2 ng/ml but below 1 ng/ml), the result shall be treated as an atypical finding under the Program and, as a result, the player shall be subject to testing for Prohibited Substances no more than four (4) times during the six-week period commencing on the date the NBA is notified by the applicable Program laboratory of the atypical finding. Such testing may be administered at any time, in the discretion of the NBA, without prior notice to the player.
\end{enumerate}

\hypertarget{reasonable-cause-testing-or-hearing.}{%
\section{Reasonable Cause Testing or Hearing.}\label{reasonable-cause-testing-or-hearing.}}

\begin{enumerate}
\def\labelenumi{(\alph{enumi})}
\tightlist
\item
  In the event that either the NBA or the Players Association has information that gives it reasonable cause to believe that a player is engaged in the use, possession, or distribution of a Prohibited Substance, including information that a First-Year Player may have been engaged in such conduct during the period beginning three (3) months prior to his entry into the NBA, such party shall request a conference with the other party and the Independent Expert, which shall be held within twenty-four (24) hours or as soon thereafter as the Expert is available. Upon hearing the information presented, the Independent Expert shall immediately decide whether there is reasonable cause to believe that the player in question has been engaged in the use, possession, or distribution of a Prohibited Substance. If the Independent Expert decides that such reasonable cause exists, the Expert shall thereupon issue an Authorization for Testing with respect to such player.
\item
  In evaluating the information presented to him, the Independent Expert shall use his or her independent judgment based upon his or her experience in substance abuse detection and enforcement. The parties acknowledge that the type of information to be presented to the Independent Expert is likely to consist of reports of conversations with third parties of the type generally considered by law enforcement authorities to be reliable sources, and that such sources might not otherwise come forward if their identities were to become known. Accordingly, neither the NBA nor the Players Association shall be required to divulge to each other or to the Independent Expert the names (or other identifying characteristics) of their sources of information regarding the use, possession, or distribution of a Prohibited Substance, and the absence of such identification of sources, standing alone, shall not constitute a basis for the Expert to refuse to issue an Authorization for Testing with respect to a player. In conferences with the Independent Expert, the player involved shall not be identified by name until such time as the Expert has determined to issue an Authorization for Testing with respect to such player in the form set forth in Exhibit I-1 to this Agreement.
\item
  Immediately upon the Independent Expert's issuance of an Authorization for Testing with respect to a particular player, the NBA shall arrange for such player to undergo testing for Drugs of Abuse (if the Authorization for Testing was based on information regarding the use, possession, or distribution of a Drug of Abuse), for synthetic cannabinoids (if the authorization for Testing was based on information regarding the player's use, possession, or distribution of synthetic cannabinoids), or for SPEDs (if the Authorization for Testing was based on information regarding the player's use, possession, or distribution of a SPED) no more than four (4) times during the six-week period commencing with the issuance of the Authorization for Testing. Such testing may be administered at any time, in the discretion of the NBA, without prior notice to the player.
\item
  In the event that the player tests positive for a Drug of Abuse other than a Benzodiazepine pursuant to this Section 5, or tests positive pursuant to Section 4(d)(iii), (iv), or (v) above in connection with testing conducted pursuant to this Section 5, he shall immediately be dismissed and disqualified from any association with the NBA or any of its Teams in accordance with the provisions of Section 12(a) below. If the player tests positive for a SPED, Benzodiazepine, or synthetic cannabinoid pursuant to this Section 5, he shall enter the Program and suffer the applicable consequences set forth in Section 9 or 10 below, as the case may be. If the player tests positive for a Diuretic, he shall suffer the applicable consequences of a positive test for the Prohibited Substance for which the Authorization for Testing was issued.
\item
  In the event that either the NBA or the Players Association determines that there is sufficient evidence to demonstrate that, within the previous year, a player has engaged in the use, possession, or distribution of a Prohibited Substance, has engaged in a felony involving the distribution of marijuana, or has received treatment for use of a Prohibited Substance other than in accordance with the terms of this Article XXXIII, it may, in lieu of requesting the testing procedure set forth in Sections 5(a)-(d) above, request a hearing on the matter before the Grievance Arbitrator. If the Grievance Arbitrator concludes that, within the previous year, the Player has used, possessed, or distributed a Prohibited Substance, has engaged in a felony involving the distribution of marijuana, or has received treatment other than in accordance with the terms of this Article XXXIII, the player shall immediately be dismissed and disqualified from any association with the NBA or any of its Teams in accordance with the provisions of Section 12(a) below, notwithstanding the fact that the player has not undergone the testing procedure set forth in this Section 5; provided, however, that if the Grievance Arbitrator concludes that the player has used or possessed a SPED, Benzodiazepine, Diuretic, or synthetic cannabinoid, he shall enter the Program and suffer the applicable consequences set forth in Section 9 or 10 below, as the case may be.
\end{enumerate}

\hypertarget{random-testing.}{%
\section{Random Testing.}\label{random-testing.}}

\begin{enumerate}
\def\labelenumi{(\alph{enumi})}
\item
  In addition to the testing procedures set forth in Section 5 above, a player shall be required to undergo urine testing for Prohibited Substances at any time, without prior notice to the player, no more than four (4) times each Season and no more than two (2) times during each Off-Season. For purposes of this Section 6, the last day of a Season for a player shall be the day before that player's Off-Season begins. During each Season, the NBA will conduct no more than 1,925 total urine tests. During the Off-Season, the NBA will conduct no more than 600 total urine tests. The scheduling of testing and collection of urine samples will be conducted according to a random player selection procedure by a third-party organization, and neither the NBA, the Players Association, any Team, or any player will have any involvement in selecting the players to be tested or will receive prior notice of the testing schedule; provided, however, that it shall not be a violation of the foregoing for the third-party organization (or a specimen collector for the same) to provide advance notice of a scheduled collection to an NBA Team Security Representative, so long as such notice does not identify the player(s) who will be tested and seeks merely to facilitate access of the collector to the testing location. Urine samples collected during the Season will be tested for all Prohibited Substances; urine samples collected during the Off-Season will be tested for SPEDs and Diuretics only and may not under any circumstances be tested with respect to any other Prohibited Substances.
\item
  \begin{enumerate}
  \def\labelenumii{(\roman{enumii})}
  \tightlist
  \item
    In the event that a First-Year Player tests positive for a Drug of Abuse other than a Benzodiazepine pursuant to this Section 6, he shall immediately be dismissed and disqualified from any association with the NBA or its Teams for a period of one (1) year, his Player Contract shall be rendered null and void and of no further force or effect (subject to the provisions of Paragraph 8 of the Uniform Player Contract), and he shall enter Stage 1 of the Drugs of Abuse Program. Such dismissal and disqualification shall be mandatory and may not be rescinded or reduced by the player's Team or the NBA; provided, however, that such dismissal and disqualification may be reduced or rescinded by the Grievance Arbitrator in accordance with Section 20 below.
  \item
    During any period while a First-Year Player is dismissed and disqualified from the NBA under Section 6(b)(i) above, and so long as such player is in compliance with his in-patient or aftercare obligations under the Program (as determined by the Medical Director), he shall receive from his Team a reasonable and necessary living expense stipend to be agreed upon by the NBA and the Players Association which (A) shall not exceed twenty-five percent (25\%) of the Salary that the player would otherwise have been entitled to earn for the period of his dismissal and disqualification and (B) shall not be payable for more than one (1) year from the date of such dismissal and disqualification.
  \item
    Any First-Year Player who tests positive for a SPED, Benzodiazepine, or synthetic cannabinoid pursuant to this Section 6, shall suffer the applicable consequences set forth in Section 9 or 10 below, as the case may be. Any First-Year Player who tests positive for a Diuretic pursuant to this Section 6 shall be deemed to have tested positive for a SPED and shall suffer the applicable consequences set forth in Section 9 below.
  \end{enumerate}
\item
  In the event that a Veteran Player tests positive for a Drug of Abuse other than a Benzodiazepine pursuant to this Section 6, he shall immediately be dismissed and disqualified from any association with the NBA or any of its Teams in accordance with the provisions of Section 12(a) below; provided, however, that such dismissal and disqualification may be reduced or rescinded by the Grievance Arbitrator in accordance with Section 20 below. If the player tests positive for a SPED, Benzodiazepine, or synthetic cannabinoid pursuant to this Section 6, he shall enter the Program and suffer the applicable consequences set forth in Section 9 or 10 below, as the case may be. If the player tests positive for a Diuretic pursuant to this Section 6, he shall be deemed to have tested positive for a SPED and shall suffer the applicable consequences set forth in Section 9 below.
\item
  In the event that any player tests ``positive'' pursuant to Section 4(d)(iii), (iv), or (v) above in connection with testing conducted pursuant to this Section 6, that positive test result shall be considered a positive test result for a Drug of Abuse, and the player shall immediately be dismissed and disqualified from any association with the NBA or any of its Teams in accordance with the provisions of Section 12(a) below.
\item
  If a player fails to submit to a scheduled test during the Off-Season pursuant to this Section 6, or to cooperate fully with the testing process for such test, without a reasonable explanation satisfactory to NBA, then (i) the drug testing collector shall provide notice to the player (with the NBA and Players Association copied) of such failure or lack of cooperation each time it occurs; (ii) if such failure or lack of cooperation continues, the player will be subject to a daily fine commencing on the third day of such continuing failure or lack of cooperation (with the fining period to commence at 5:00 p.m. (local time) at the site of such testing) , and (iii) the daily fine shall be \$1,000 for the first day, \$2,000 for the second day, \$3,000 for the third day, \$4,000 for the fourth day, and \$5,000 for the fifth and any additional days on which the player fails to submit to scheduled testing or cooperate fully with the testing process. Nothing in the foregoing shall prejudice in any manner the NBA's rights under Sections 4(d)(iii) and 4(d)(iv) above.
\end{enumerate}

\hypertarget{drugs-of-abuse-program.}{%
\section{Drugs of Abuse Program.}\label{drugs-of-abuse-program.}}

\begin{enumerate}
\def\labelenumi{(\alph{enumi})}
\tightlist
\item
  \textbf{Voluntary Entry.}

  \begin{enumerate}
  \def\labelenumii{(\roman{enumii})}
  \tightlist
  \item
    A player may enter the Drugs of Abuse Program voluntarily at any time by Coming Forward Voluntarily for a problem involving the use of a Drug of Abuse; provided, however, that a player may not Come Forward Voluntarily (A) until he has been selected in an NBA Draft or invited to an NBA training camp; (B) during any period in which an Authorization for Testing as to that player remains in effect pursuant to Section 5 above; (C) during any period in which he remains subject to in-patient or aftercare treatment in Stage 1 of the Drugs of Abuse Program; or (D) after he has reached Stage 2 of the Drugs of Abuse Program.
  \item
    If a player who has not previously entered the Drugs of Abuse Program Comes Forward Voluntarily for a problem involving the use of a Drug of Abuse, he shall enter Stage 1 of the Drugs of Abuse Program.
  \item
    If a player who has not previously entered Stage 2 of the Drugs of Abuse Program, but who has been notified by the Medical Director that he has successfully completed Stage 1 of that Program, Comes Forward Voluntarily for a problem involving the use of a Drug of Abuse, he shall enter Stage 2 of the Drugs of Abuse Program.
  \item
    No penalty of any kind will be imposed on a player as a result of having Come Forward Voluntarily for a problem involving the use of a Drug of Abuse. The foregoing sentence shall not preclude the imposition of a penalty under Section 7(c)(iv) below as a result of the player's entering Stage 2 of the Drugs of Abuse Program, or any penalty called for by this Article XXXIII as a result of conduct by the player that occurs after he has Come Forward Voluntarily.
  \end{enumerate}
\item
  \textbf{Stage 1.}

  \begin{enumerate}
  \def\labelenumii{(\roman{enumii})}
  \tightlist
  \item
    Any player who has entered Stage 1 of the Drugs of Abuse Program shall be required to submit to an evaluation by the Medical Director, provide (or cause to be provided) to the Medical Director such relevant medical and treatment records as the Medical Director may request, and commence the treatment and testing program prescribed by the Medical Director.
  \item
    If a player, within ten (10) days of the date on which he was notified that he had entered Stage 1 of the Drugs of Abuse Program and without a reasonable excuse, fails to comply (in the professional judgment of the Medical Director) with any of the obligations set forth in Section 7(b)(i) above, he shall be suspended until such time as the Medical Director determines that he has fully complied with Section 7(b)(i) above. If such noncompliance continues without a reasonable excuse (in the professional judgment of the Medical Director) for thirty (30) days from the date on which the player was notified that he had entered Stage 1 of the Drugs of Abuse Program, the player shall, following notice of the player's non-compliance by the Medical Director to the NBA and then by the NBA to the player's Team (notwithstanding the provisions of Section 3 above), (A) advance to Stage 2 of the Drugs of Abuse Program, or (B) the player's Team may, notwithstanding any term or provision in or amendment to the player's Uniform Player Contract, elect to terminate such Contract without any further obligation to pay Compensation, except to pay the Compensation (either Current or Deferred) that may have been earned by the player to the date of termination.
  \item
    Except as provided in this Article XXXIII, no penalty of any kind will be imposed on a player while he is in Stage 1 of the Drugs of Abuse Program and, provided he complies with the terms of his prescribed treatment, he will continue to receive his Compensation during the term of his treatment for a period of up to three (3) months of care in an In-Patient Facility and such aftercare as may be required by the Medical Director.
  \end{enumerate}
\item
  \textbf{Stage 2.}

  \begin{enumerate}
  \def\labelenumii{(\roman{enumii})}
  \tightlist
  \item
    Any player who has entered Stage 2 of the Drugs of Abuse Program shall be required to submit to an evaluation by the Medical Director, provide (or cause to be provided) to the Medical Director such relevant medical and treatment records as the Medical Director may request, and commence the treatment and testing program prescribed by the Medical Director.
  \item
    If a player, within thirty (30) days of the date on which he was notified that he had entered Stage 2 of the Drugs of Abuse Program and without a reasonable excuse, fails to comply (in the professional judgment of the Medical Director) with any of the obligations set forth in Section 7(c)(i) above, he shall immediately be dismissed and disqualified from any association with the NBA or any of its Teams in accordance with the provisions of Section 12(a) below.
  \item
    A player in Stage 2 of the Drugs of Abuse Program shall be suspended during the period of his in-patient treatment and for at least the first six (6) months of his aftercare treatment. The player shall remain suspended during any subsequent period in which he is undergoing treatment that, in the professional judgment of the Medical Director, prevents him from rendering the playing services called for by his Uniform Player Contract.
  \item
    Any subsequent use, possession, or distribution of a Drug of Abuse by a player in Stage 2, even if voluntarily disclosed, or any conduct by a player in Stage 2 that results in his advancing one (1) Stage in the Drugs of Abuse Program, shall result in the player being immediately dismissed and disqualified from any association with the NBA or any of its Teams in accordance with the provisions of Section 12(a) below.
  \end{enumerate}
\item
  \textbf{Treatment and Testing Program.} A player who enters the Drugs of Abuse Program shall be required to comply with such in-patient and aftercare program as may be prescribed and supplemented from time to time by the Medical Director. Such program may include random testing for Prohibited Substances other than SPEDs, and for marijuana and alcohol, and such non-testing elements as may be determined in the professional judgment of the Medical Director.
\end{enumerate}

\hypertarget{marijuana-and-alcohol-treatment-programs.}{%
\section{Marijuana and Alcohol Treatment Programs.}\label{marijuana-and-alcohol-treatment-programs.}}

\begin{enumerate}
\def\labelenumi{(\alph{enumi})}
\tightlist
\item
  \textbf{Team Referral.}

  \begin{enumerate}
  \def\labelenumii{(\roman{enumii})}
  \tightlist
  \item
    In the event that a player's Team has reasonable cause to believe that the player was under the influence of marijuana and/or alcohol while engaged in activities for such Team or for the NBA, or that the player has a dependency or other related issue involving the use of marijuana and/or alcohol, the Team may refer the player to the Medical Director for a mandatory evaluation. The Medical Director shall notify the player of the referral, with a copy of such notice to the NBA and the Players Association. In connection with this evaluation, the player shall provide (or cause to be provided) to the Medical Director such relevant medical and treatment records as the Medical Director may request.
  \item
    If, based on the mandatory evaluation described in Section 8(a)(i) above, the Medical Director determines that the player was under the influence of marijuana and/or alcohol while engaged in activities for his Team or for the NBA, or that the player has a dependency or other related issue involving the use of marijuana and/or alcohol, then the Medical Director shall provide notice of such determination to the player (with a copy to the NBA and the Players Association) and the player shall be required to commence and fully cooperate with a treatment and testing program prescribed by the Medical Director. Such program may include random testing for marijuana, alcohol, and/or Prohibited Substances other than SPEDs, and such non-testing elements as may be determined in the professional judgment of the Medical Director.
  \item
    If a player, within five (5) days of the date on which he was notified by the Medical Director of any of the obligations set forth in Section 8(a) above and without a reasonable excuse, fails to comply (in the professional judgment of the Medical Director) with any of such obligations, he shall be fined \$10,000; if the player thereafter fails to comply, without a reasonable excuse, with such obligations (in the professional judgment of the Medical Director) within eight (8) days of such notification, he shall be fined an additional \$10,000; and for each additional day beyond the 8th day that the player, without a reasonable excuse, fails to comply with such obligations (in the professional judgment of the Medical Director), he shall be fined an additional \$10,000. The total amount of such fines may not exceed the player's total Compensation.
  \end{enumerate}
\item
  \textbf{Voluntary Entry.}

  \begin{enumerate}
  \def\labelenumii{(\roman{enumii})}
  \tightlist
  \item
    A player may seek assistance from the Medical Director at any time for dependency on or any other issue related to the use of marijuana or alcohol.
  \end{enumerate}
\item
  \textbf{Non-Exclusivity.}

  \begin{enumerate}
  \def\labelenumii{(\roman{enumii})}
  \tightlist
  \item
    Nothing in this Section 8 or Section 14 below shall prohibit (or otherwise prejudice) a team or the NBA from imposing reasonable discipline on a player (subject to the One Penalty Rule) for being under the influence of marijuana and/or alcohol while engaged in team or NBA-related activities, or for not providing the services called for under his Player Contract as a result of a dependency or other related issue involving the use of marijuana and/or alcohol.
  \end{enumerate}
\end{enumerate}

\hypertarget{steroids-and-performance-enhancing-drugs-program.}{%
\section{Steroids and Performance-Enhancing Drugs Program.}\label{steroids-and-performance-enhancing-drugs-program.}}

\begin{enumerate}
\def\labelenumi{(\alph{enumi})}
\tightlist
\item
  \textbf{Treatment.}

  \begin{enumerate}
  \def\labelenumii{(\roman{enumii})}
  \tightlist
  \item
    A player who enters the SPED Program shall be required to submit to an evaluation by the SPED Medical Director, provide (or cause to be provided) to the SPED Medical Director such relevant medical and treatment records as the SPED Medical Director may request, and commence the treatment and testing program prescribed by the SPED Medical Director. Such program may include random testing for SPEDs and Diuretics and such non-testing elements as may be determined in the professional judgment of the SPED Medical Director.
  \item
    If a player, within five (5) days of the date on which he was notified that he had entered the SPED Program and without a reasonable excuse, fails to comply (in the professional judgment of the SPED Medical Director) with any of the obligations set forth in the first sentence of Section 9(a)(i) above, he shall be fined \$10,000; if the player, without a reasonable excuse, thereafter fails to comply with such obligations (in the professional judgment of the SPED Medical Director) within eight (8) days of such notification, he shall be fined an additional \$10,000; and for each additional day beyond the 8th day that the player, without a reasonable excuse, fails to comply with such obligations (in the professional judgment of the SPED Medical Director), he shall be fined an additional \$10,000. The total amount of such fines shall not exceed the player's total Compensation.
  \end{enumerate}
\item
  \textbf{Penalties.} Any player who (i) tests positive for a SPED, Benzodiazepine, or Diuretic pursuant to Section 5 (Reasonable Cause Testing or Hearing), Section 6 (Random Testing), or Section 16 (Additional Bases for Testing), or (ii) is adjudged by the Grievance Arbitrator pursuant to Section 5(e) above to have used or possessed a SPED, Benzodiazepine, or Diuretic, shall suffer the following penalties:

  \begin{enumerate}
  \def\labelenumii{(\Alph{enumii})}
  \tightlist
  \item
    For the first such violation, the player shall be suspended for twenty-five (25) games and required to enter the SPED Program (or the Drugs of Abuse Program if the positive test or the use or possession is for a Benzodiazepine);
  \item
    for the second such violation, the player shall be suspended for fifty-five (55) games and, if the player is not then subject to in-patient or aftercare treatment in the SPED or Drugs of Abuse Program (as applicable), be required to enter the SPED Program (or the Drugs of Abuse Program if the positive test or the use or possession is for a Benzodiazepine); and
  \item
    for the third such violation, the player shall be immediately dismissed and disqualified from any association with the NBA or any of its Teams in accordance with the provisions of Section 12(a) below.
  \end{enumerate}
\item
  The penalties set forth in Section 9(b) above with respect to a player's use of a SPED or Benzodiazepine may be reduced or rescinded by the Grievance Arbitrator in accordance with Section 20 below.
\end{enumerate}

\hypertarget{synthetic-cannabinoid-program.}{%
\section{Synthetic Cannabinoid Program.}\label{synthetic-cannabinoid-program.}}

\begin{enumerate}
\def\labelenumi{(\alph{enumi})}
\tightlist
\item
  \textbf{Voluntary Entry.}

  \begin{enumerate}
  \def\labelenumii{(\roman{enumii})}
  \tightlist
  \item
    A player may enter the Synthetic Cannabinoid Program voluntarily at any time by Coming Forward Voluntarily; provided, however, that a player may not Come Forward Voluntarily for a problem involving the use of a synthetic cannabinoid (A) until he has been selected in an NBA Draft or invited to an NBA training camp; (B) during any period in which an Authorization for Testing as to that player remains in effect pursuant to Section 5 above; or (C) during any period in which he remains subject to in-patient or aftercare treatment in the Synthetic Cannabinoid Program.
  \item
    If a player who has not previously entered the Synthetic Cannabinoid Program, or a player who has been notified by the Medical Director that he has successfully completed that Program, Comes Forward Voluntarily for a dependency or other related problem involving the use of a synthetic cannabinoid, he shall enter the Synthetic Cannabinoid Program.
  \item
    No penalty of any kind will be imposed on a player as a result of having Come Forward Voluntarily for a problem involving the use of a synthetic cannabinoid. The foregoing sentence shall not preclude the imposition of any penalty called for by this Article XXXIII as a result of conduct by the player that occurs after he has Come Forward Voluntarily.
  \end{enumerate}
\item
  \textbf{Treatment.}

  \begin{enumerate}
  \def\labelenumii{(\roman{enumii})}
  \tightlist
  \item
    A player who enters the Synthetic Cannabinoid Program shall be required to submit to an evaluation by the Medical Director, provide (or cause to be provided) to the Medical Director such relevant medical and treatment records as the Medical Director may request, and commence the treatment and testing program prescribed by the Medical Director. Such program may include random testing for Prohibited Substances other than SPEDs, and such non-testing elements as may be determined in the professional judgment of the Medical Director.
  \item
    If a player, within five (5) days of the date on which he was notified that he had entered the Synthetic Cannabinoid Program and without a reasonable excuse, fails to comply (in the professional judgment of the Medical Director) with any of the obligations set forth in the first sentence of Section 10(b)(i) above, he shall be fined \$10,000; if the player thereafter fails to comply, without a reasonable excuse, with such obligations (in the professional judgment of the Medical Director) within eight (8) days of such notification, he shall be fined an additional \$10,000; and for each additional day beyond the 8th day that the player, without a reasonable excuse, fails to comply with such obligations (in the professional judgment of the Medical Director), he shall be fined an additional \$10,000. The total amount of such fines may not exceed the player's total Compensation.
  \end{enumerate}
\item
  \textbf{Penalties.} Any player who (i) tests positive for a synthetic cannabinoid pursuant to Section 5 (Reasonable Cause Testing or Hearing), Section 6 (Random Testing), or Section 16 (Additional Bases for Testing), (ii) is adjudged by the Grievance Arbitrator pursuant to Section 5(e) above to have used or possessed a synthetic cannabinoid, or (iii) has been convicted of (including by a plea of guilty, no contest, or nolo contendere to) the use or possession of a synthetic cannabinoid in violation of the law, shall suffer the following penalties:

  \begin{enumerate}
  \def\labelenumii{(\Alph{enumii})}
  \tightlist
  \item
    For the first such violation, the player shall be required to enter the Synthetic Cannabinoid Program;
  \item
    For the second such violation, the player shall be fined \$25,000 and, if the player is not then subject to in-patient or aftercare treatment in the Synthetic Cannabinoid Program, be required to enter the Synthetic Cannabinoid Program;
  \item
    For the third such violation, the player shall be suspended for five (5) games and, if the player is not then subject to in-patient or aftercare treatment in the Synthetic Cannabinoid Program, be required to enter the Synthetic Cannabinoid Program; and
  \item
    For any subsequent violation, the player shall be suspended for five (5) games longer than his immediately-preceding suspension for violating the Synthetic Cannabinoid Program and, if the player is not then subject to in-patient or aftercare treatment in the Synthetic Cannabinoid Program, be required to enter the Synthetic Cannabinoid Program.
  \end{enumerate}
\end{enumerate}

\hypertarget{noncompliance-with-treatment.}{%
\section{Noncompliance with Treatment.}\label{noncompliance-with-treatment.}}

\begin{enumerate}
\def\labelenumi{(\alph{enumi})}
\tightlist
\item
  \textbf{Drugs of Abuse.}

  \begin{enumerate}
  \def\labelenumii{(\roman{enumii})}
  \tightlist
  \item
    Any player who, after entering Stage 1 or Stage 2 of the Drugs of Abuse Program, fails to comply with his treatment or his aftercare program as prescribed and determined by the Medical Director, shall be suspended. Such suspension shall continue until the player has, in the professional judgment of the Medical Director, resumed full compliance with his treatment program.
  \item
    Notwithstanding Section 11(a)(i) above, any player who in the professional judgment of the Medical Director, after entering Stage 1 or Stage 2 of the Drugs of Abuse Program, fails to comply with his treatment program through (A) a pattern of behavior that demonstrates a mindful disregard for his treatment responsibilities, or (B) a positive test for a Prohibited Substance other than a SPED that is not clinically expected by the Medical Director, shall suffer the following penalties:

    \begin{enumerate}
    \def\labelenumiii{(\arabic{enumiii})}
    \tightlist
    \item
      if the player is in Stage 1 of the Drugs of Abuse Program, he shall advance to Stage 2 and be suspended until, in the professional judgment of the Medical Director, he has resumed full compliance with his treatment program; or
    \item
      if the player already is in Stage 2 of the Drugs of Abuse Program, he shall immediately be dismissed and disqualified from any association with the NBA or any of its Teams in accordance with the provisions of Section 12(a) below.
    \end{enumerate}
  \end{enumerate}
\item
  \textbf{Marijuana and Alcohol.}

  \begin{enumerate}
  \def\labelenumii{(\roman{enumii})}
  \tightlist
  \item
    Any player who, after entering the Marijuana and/or Alcohol Treatment Program pursuant to Section 8(a) above, fails to comply (without a reasonable excuse) with his treatment program as prescribed and determined by the Medical Director, shall be fined \$5,000 for each day that he fails to comply. Such fines shall continue until the player has, in the professional judgment of the Medical Director, resumed full compliance with his treatment program. The total amount of such fines shall not exceed the player's total Compensation.
  \item
    Notwithstanding Section 11(b)(i) above, any player who, after entering the Marijuana and/or Alcohol Treatment Program pursuant to Section 8(a) above, fails to comply with his treatment program as prescribed and determined by the Medical Director through (A) a pattern of behavior that demonstrates a mindful disregard for his treatment responsibilities, or (B) a positive test for marijuana and/or alcohol (as applicable) that is not clinically expected by the Medical Director, shall suffer the following penalties:

    \begin{enumerate}
    \def\labelenumiii{(\arabic{enumiii})}
    \tightlist
    \item
      if the player has not previously been fined \$10,000 under Section 8(c) above or this Section 11(b)(ii), a fine of \$25,000;
    \item
      if the player has previously been fined \$10,000 under Section 8(c) above or \$25,000 under this Section 11(b)(ii), a suspension of five (5) games; or
    \item
      if the player has previously been suspended for five (5) or more games under this Section 11(b)(ii), a suspension that is at least five (5) games longer than his immediately-preceding suspension and that shall continue until, in the professional judgment of the Medical Director, the player resumes full compliance with his treatment program.
    \end{enumerate}
  \item
    In addition to any consequence to the player under Section 11(b)(ii) above, any player who has entered the Marijuana and/or Alcohol Treatment Program pursuant to Section 8 above but not the Drugs of Abuse Program, and tests positive for a Drug of Abuse in any test conducted by the Medical Director, shall enter Stage 1 of the Drugs of Abuse Program.
  \end{enumerate}
\item
  \textbf{SPEDs.}

  \begin{enumerate}
  \def\labelenumii{(\roman{enumii})}
  \tightlist
  \item
    Any player who, after entering the SPED Program, fails to comply (without a reasonable excuse) with his treatment program as prescribed and determined by the SPED Medical Director, shall be fined \$5,000 per day for each day that he fails to comply. Such fines shall continue until the player has, in the professional judgment of the SPED Medical Director, resumed full compliance with his treatment program. The total amount of such fines shall not exceed the player's total Compensation.
  \item
    Notwithstanding Section 11(c)(i) above, any player who, after entering the SPED Program, fails to comply with his treatment program as prescribed and determined by the SPED Medical Director through (A) a pattern of behavior that demonstrates a mindful disregard for his treatment responsibilities, or (B) a positive test for a SPED that is not clinically expected by the SPED Medical Director, shall suffer the following penalties:

    \begin{enumerate}
    \def\labelenumiii{(\arabic{enumiii})}
    \tightlist
    \item
      if the player has not previously been suspended for twenty-five (25) games under Section 9(b) above or this Section 11(c)(ii), a suspension of twenty-five (25) games;
    \item
      if the player has previously been suspended for twenty-five (25) games under Section 9(b) above or this Section 11(c)(ii), a suspension of fifty-five (55) games; or
    \item
      if the player has been previously suspended for fifty-five (55) games under Section 9(b) above or this Section 11(c)(ii), the player shall be immediately dismissed and disqualified from any association with the NBA or any of its Teams in accordance with the provisions of Section 12(a) below.
    \end{enumerate}
  \end{enumerate}
\item
  \textbf{Synthetic Cannabinoids.}

  \begin{enumerate}
  \def\labelenumii{(\roman{enumii})}
  \tightlist
  \item
    Any player who, after entering the Synthetic Cannabinoid Program, fails to comply (without a reasonable excuse) with his treatment program as prescribed and determined by the Medical Director, shall be fined \$5,000 for each day that he fails to comply. Such fines shall continue until the player has, in the professional judgment of the Medical Director, resumed full compliance with his treatment program. The total amount of such fines shall not exceed the player's total Compensation.
  \item
    Notwithstanding Section 11(d)(i) above, any player who, after entering the Synthetic Cannabinoid Program, fails to comply with his treatment program as prescribed and determined by the Medical Director through (A) a pattern of behavior that demonstrates a mindful disregard for his treatment responsibilities, or (B) a positive test for a synthetic cannabinoid that is not clinically expected by the Medical Director, shall suffer the following penalties:

    \begin{enumerate}
    \def\labelenumiii{(\arabic{enumiii})}
    \tightlist
    \item
      if the player has not previously been fined \$25,000 under Section 10(c) above or this Section 11(d)(ii), a fine of \$25,000;
    \item
      if the player has previously been fined \$25,000 under Section 10(c) above or this Section 11(d)(ii), a suspension of five (5) games; or
    \item
      if the player has previously been suspended for five (5) or more games under Section 10(c) above or this Section 11(d)(ii), a suspension that is at least five (5) games longer than his immediately-preceding suspension and that shall continue until, in the professional judgment of the Medical Director, the player resumes full compliance with his treatment program.
    \end{enumerate}
  \item
    In addition to any consequence to the player under Section 11(d)(ii) above, any player who has entered the Synthetic Cannabinoid Program but not the Drugs of Abuse Program, and tests positive for a Drug of Abuse in any test conducted by the Medical Director, shall enter Stage 1 of the Drugs of Abuse Program.
  \end{enumerate}
\item
  \textbf{Directed Testing.} Any player who, after entering the Program, and without a reasonable explanation satisfactory to the Medical Director, (i) fails to appear for any of his Team's scheduled games, or (ii) misses, during any consecutive seven-day period, any two (2) airplane flights on which his team is scheduled to travel, any two (2) Team practices, or a combination of any one (1) practice and any one (1) Team flight, shall immediately submit to a urine test to be conducted by the NBA. If any test conducted pursuant to this Section 11(e) is positive: (w) for a Drug of Abuse or pursuant to Section 4(d)(iii), (iv), or (v) above (for a player in the Drugs of Abuse Program), then the player shall suffer the applicable consequence set forth in Section 11(a)(ii) above; (x) for marijuana and/or alcohol pursuant to Section 4(d)(iii), (iv), or (v) above (for a player in the Marijuana and/or Alcohol Treatment Program), then the player shall suffer the applicable consequence set forth in Section 11(b)(ii) above; (y) for a SPED or pursuant to Section 4(d)(iii), (iv), or (v) above (for a player in the SPED Program), then the player will suffer the applicable consequence set forth in Section 11(c)(ii) above; or (z) for a synthetic cannabinoid or pursuant to Section 4(d)(iii), (iv), or (v) above (for a player in the Synthetic Cannabinoid Program), then the player shall suffer the applicable consequence set forth in Section 11(d)(ii) above. If any test conducted pursuant to this Section 11(e) is positive for a Diuretic, then the player shall suffer the applicable consequences of a positive test for the Prohibited Substance for which he entered the Program.
\end{enumerate}

\hypertarget{dismissal-and-disqualification.}{%
\section{Dismissal and Disqualification.}\label{dismissal-and-disqualification.}}

\begin{enumerate}
\def\labelenumi{(\alph{enumi})}
\tightlist
\item
  A player who, under the terms of this Agreement, is ``dismissed and disqualified from any association with the NBA or any of its Teams in accordance with the provisions of Section 12(a)'' shall, without exception, immediately be so dismissed and disqualified for a period of not less than one (1) year, and such player's Player Contract shall be rendered null and void and of no further force or effect (subject to the provisions of Paragraph 8 of the Uniform Player Contract). Such dismissal and disqualification shall be mandatory and may not be rescinded or reduced by the player's Team or the NBA.
\item
  In addition to any other provision of this Agreement requiring that a player be dismissed and disqualified from any association with the NBA or any of its Teams in accordance with the provisions of Section 12(a) above, a player will also be dismissed and disqualified under Section 12(a) above if he is convicted of (including by a plea of guilty, no contest, or nolo contendere to) a crime involving the use, possession, or distribution of a Prohibited Substance other than marijuana or a felony involving the distribution of marijuana.
\end{enumerate}

\hypertarget{reinstatement.}{%
\section{Reinstatement.}\label{reinstatement.}}

\begin{enumerate}
\def\labelenumi{(\alph{enumi})}
\item
  After a period of at least one (1) year from the time of a player's dismissal and disqualification under Section 12(a) above, such player may apply for reinstatement as a player in the NBA. However, such player shall have no right to reinstatement under any circumstance and the reinstatement shall be granted only with the prior approval of both the NBA and the Players Association, which shall not be unreasonably withheld. The approval of the NBA and the Players Association shall rest in their absolute and sole discretion, and their decision shall be final, binding, and unappealable. Among the factors that may be considered by the NBA and the Players Association in determining whether to grant reinstatement are (without limitation): the circumstances surrounding the player's dismissal and disqualification; whether the player has satisfactorily completed a treatment and rehabilitation program; the player's conduct since his dismissal, including the extent to which the player has since comported himself as a suitable role model for youth; and whether the player is judged to possess the requisite qualities of good character and morality.
\item
  For a First-Year Player, the NBA and the Players Association will consider an application for reinstatement only if the player has, in the opinion of the Medical Director or the SPED Medical Director (as applicable), successfully completed any in-patient treatment and/or aftercare prescribed by the Medical Director or the SPED Medical Director (as applicable). For a Veteran Player who was dismissed and disqualified under Section 12(a) above in connection with a Drug of Abuse, the NBA and the Players Association will consider any application for reinstatement only if the player can demonstrate, by proof of random urine testing acceptable to the Medical Director (conducted on at least a weekly basis), that he has not tested positive (i) for a Drug of Abuse or synthetic cannabinoid within the twelve (12) months prior to the submission of his application for reinstatement and during any period while his application is being reviewed, and (ii) if the Medical Director deems it necessary in his or her professional judgment, for marijuana and/or alcohol for the six (6) months prior to the submission of his application for reinstatement and during any period while his application is being reviewed. For a Veteran Player who was dismissed and disqualified under Section 12(a) above in connection with a SPED, the NBA and the Players Association will consider any application for reinstatement only if the player can demonstrate, by proof of random urine and/or blood testing acceptable to the SPED Medical Director (conducted on at least a weekly basis), that he has not tested positive for a SPED within the twelve (12) months prior to the submission of his application for reinstatement and during any period while his application is being reviewed.
\item
  The granting of an application for reinstatement may be conditioned upon random testing of the player or such other terms as may be agreed upon by the NBA and the Players Association, whether or not such terms are contemplated by the terms of this Article XXXIII.
\item
  Any player who has been reinstated pursuant to this Section 13 and is subsequently dismissed and disqualified from any association with the NBA or any of its Teams in accordance with the provisions of Section 12(a) above shall therefore be ineligible for reinstatement pursuant to this Section 13.
\item
  In the event that the application for reinstatement of a First-Year Player dismissed and disqualified pursuant to Section 6(b) above is approved, such player, by reason of his Player Contract having been rendered null and void pursuant to Section 6(b) above, shall be deemed not to have completed his Player Contract by rendering the playing services called for thereunder. Accordingly, such player shall not be a Free Agent and shall not be entitled to negotiate or sign a Player Contract with any NBA Team, except as specifically provided in this Section 13.
\item
  \begin{enumerate}
  \def\labelenumii{(\roman{enumii})}
  \tightlist
  \item
    A First-Year Player who has been reinstated pursuant to this Section 13 shall, immediately upon such reinstatement, notify the Team to which he was under contract at the time of his dismissal and disqualification (the ``previous Team''). Upon receipt of such notification, and subject to Section 13(f)(ii) below, the previous Team shall then have thirty (30) days in which to make a Tender to the player with a stated term of at least one (1) full NBA Season (or, in the event that the Tender is made during a Season, of at least the remainder of that Season) and calling for at least the Minimum Player Salary then applicable to that player but not more than the Salary provided for in Section 13(f)(iii) below. If the previous Team makes such a Tender, it shall, for a period of one (1) year from the date of the Tender, be the only NBA Team with which the player may negotiate and sign a Player Contract. If the player does not sign a Player Contract with the previous Team within the year following such Tender, the player shall thereupon be deemed a Restricted Free Agent, subject to a Right of First Refusal. If the previous Team fails to make a Tender, the player shall become an Unrestricted Free Agent.
  \item
    Notwithstanding anything to the contrary in Section 13(f)(i) above, the 30-day period for the previous Team to make a Tender shall be tolled if (A) on the date the player serves the notice required by Section 13(f)(i), he is under contract to a professional basketball team not in the NBA, or (B) the player signs a contract with a professional basketball team not in the NBA at any point after the date on which the player serves the notice required by Section 13(f)(i) and before the date on which the previous Team makes a Tender. If the 30-day period for making a Tender is tolled pursuant to the preceding sentence, the period shall remain tolled until the date on which the player notifies the Team that he is immediately available to sign and begin rendering playing services under a Player Contract with such Team, provided that such notice will not be effective until the player is under no contractual or other legal impediment to sign with and begin rendering playing services for such team.
  \item
    A First-Year Player who is reinstated pursuant to this Section 13 may enter into a Player Contract with his previous Team that provides for a Salary and Unlikely Bonuses for the first Season of up to the Player's Salary and Unlikely Bonuses, respectively, for the Salary Cap Year in which he was dismissed and disqualified (reduced on a pro rata basis if the first Season of the new Contract is a partial Season), even if the Team has a Team Salary at or above the Salary Cap or such Player Contract causes the Team to have a Team Salary above the Salary Cap. If the player and the previous Team enter into such Player Contract and such Contract covers more than one Season, increases and decreases in Salary for Seasons following the first Season shall be governed by Article VII, Section 5(a)(1); provided, however, that if the player who is reinstated was dismissed and disqualified during the term of his Rookie Scale Contract, then (A) the number of Seasons in the player's new Contract may not exceed two (2) Seasons plus two (2) Option Years in favor of the Team, and the Salary and Unlikely Bonuses called for in any Season of the player's new Contract, including any Option Year, may not exceed the Salary and Unlikely Bonuses called for during the corresponding Season of his Rookie Scale Contract, and (B) if the new Contract contains terms identical to those contained in the remaining Seasons of the Player's Rookie Scale Contract at the time he was dismissed and disqualified, and the Team exercises all Option Year(s) available under the new Contract, then the player's Team shall retain the same rights with respect to such new Contract as it would have retained under Article XI following the completion of the player's Rookie Scale Contract.
  \end{enumerate}
\item
  \begin{enumerate}
  \def\labelenumii{(\roman{enumii})}
  \tightlist
  \item
    A Veteran Player who has been reinstated pursuant to this Section 13 shall, immediately upon such reinstatement, notify the Team to which he was under contract at the time of his dismissal and disqualification (the ``previous Team''). Upon receipt of such notification, and subject to Section 13(g)(iii) below, the previous Team shall then have thirty (30) days in which to make a Tender to the player with a stated term of at least one (1) full NBA Season (or, in the event the Tender is made during a Season, of at least the rest of that Season) and calling for a Salary in the first Season covered by the Tender at least equal to the lesser of (A) the player's Salary for the Salary Cap Year in which he was dismissed and disqualified, or (B) the Estimated Average Player Salary during the then-current Season, in either case reduced on a pro rata basis if the first Season covered by the Tender is a partial Season, but not greater than the Salary provided in Section 13(g)(iv) below. If the previous Team makes such a Tender, it shall, for a period of one (1) year from the date of the Tender, be the only NBA Team with which the player may negotiate and sign a Player Contract. If the player does not sign a Player Contract with the previous Team within the year following such Tender, then the player shall thereupon be deemed a Restricted or an Unrestricted Free Agent, in accordance with the provisions of Article XI. If the previous Team fails to make a Required Tender, the player shall become an Unrestricted Free Agent.
  \item
    Notwithstanding anything to the contrary in Section 13(g)(i) above, a Veteran Player who has been reinstated pursuant to this Section 13 and who (A) had completed the playing services called for under his Player Contract with the previous Team at the time of his dismissal and disqualification, and (B) would have been an Unrestricted Free Agent on the July 1 following his dismissal and disqualification, shall be an Unrestricted Free Agent upon being reinstated pursuant to this Section 13 and need not serve the notice to his previous Team described in Section 13(g)(i) above. For clarity, a Veteran Player who has been reinstated pursuant to this Section 13 and would not have been an Unrestricted Free Agent on the July 1 following his dismissal and disqualification (including a Veteran Player who had completed the playing services called for under his Player Contract with his previous Team at the time of his dismissal and disqualification and who would have been a Restricted Free Agent on the July 1 following his dismissal and disqualification) shall be subject to the process described in Section 13(g)(i) above.
  \item
    Notwithstanding anything to the contrary in Section 13(g)(i) above, the 30-day period for the previous Team to make a Tender shall be tolled if (A) on the date the player serves the notice required by Section 13(g)(i), he is under contract to a professional basketball team or league not in the NBA, or (B) the player signs a contract with a professional basketball team or league not in the NBA at any point after the date on which he serves the notice required by Section 13(g)(i) and before the date on which the previous Team makes a Tender. If the 30-day period for making a Tender is tolled pursuant to the preceding sentence, the period shall remain tolled until the date on which the player notifies the Team that he is available to sign a Player Contract with and begin rendering playing services for such Team immediately, provided that such notice will not be effective until the player is under no contractual or other legal impediment to sign with and begin rendering playing services for such Team.
  \item
    A Veteran Player who is reinstated pursuant to this Section 13 and enters into a Player Contract with his previous Team may enter into a Player Contract with such Team that provides for a Salary and Unlikely Bonuses for the first Season of up to the player's Salary and Unlikely Bonuses, respectively, for the Salary Cap Year in which he was dismissed and disqualified (reduced on a pro rata basis if the first Season of the new Contract is a partial Season), even if the Team has a Team Salary at or above the Salary Cap or such Player Contract causes the Team to have a Team Salary above the Salary Cap. If the player and the previous Team enter into such Player Contract and such Contract covers more than one (1) Season, increases and decreases in Salary for Seasons following the first Season shall be governed by Article VII, Section 5(a)(1); provided, however, that if the player who is reinstated was dismissed and disqualified during the term of his Rookie Scale Contract, then (A) the number of Seasons in the Player's new Contract may not exceed the number of Seasons (including the Option Year in favor of the Team) that remained under the player's Rookie Scale Contract at the time he was dismissed and disqualified, and the Salary called for in any Season of the Player's new Contract (including any Option Year), may not exceed the Salary called for during the corresponding Season of his Rookie Scale Contract, and (B) if the new Contract contains terms identical to those contained in the remaining Seasons of the player's Rookie Scale Contract at the time he was dismissed and disqualified, and the player's Team ultimately exercises the Option Year available under the new Contract, then such Team shall retain the same rights with respect to such new Contract as it would have retained under Article XI following the completion of the player's Rookie Scale Contract.
  \end{enumerate}
\end{enumerate}

\hypertarget{exclusivity.}{%
\section{Exclusivity.}\label{exclusivity.}}

\begin{enumerate}
\def\labelenumi{(\alph{enumi})}
\tightlist
\item
  Except as expressly provided in this Article XXXIII, there shall be no other screening or testing for Prohibited Substances conducted by the NBA or any Team, and no player may undergo such screening or testing; provided, however, that, in a medical emergency, team physicians may test players solely for diagnostic purposes in order to provide satisfactory medical care. The results of any diagnostic drug testing conducted pursuant to the preceding sentence shall not be used for any other purpose by the player's Team or the NBA. If any Team is found to have tested a player in violation of this Section 14, the NBA will impose a substantial fine not to exceed \$750,000 upon such Team pursuant to the NBA's Constitution and By-Laws.
\item
  The penalties set forth in this Article XXXIII shall be the exclusive penalties to be imposed upon a player for the use, possession, or distribution of a Prohibited Substance.
\item
  No Uniform Player Contract entered into after the date hereof shall include any term or provision that modifies, contradicts, changes, or is inconsistent with Paragraph 8 of such Contract (including any condition or limitation on salary protection other than the standard conditions or limitations specifically provided for in Article II, Section 4) or provides for the testing of a player for illegal substances. Any term or provision of a currently effective Uniform Player Contract that is inconsistent with Paragraph 8 of such Contract shall be deemed null and void only to the extent of the inconsistency.
\end{enumerate}

\hypertarget{random-hgh-blood-testing.}{%
\section{Random HGH Blood Testing.}\label{random-hgh-blood-testing.}}

\begin{enumerate}
\def\labelenumi{(\alph{enumi})}
\tightlist
\item
  In addition to the testing procedures set forth in Section 5 above, a player shall be required to undergo HGH Blood Testing at any time, without prior notice to the player, no more than two (2) times each Season and no more than one (1) time during each Off-Season. For purposes of this Section 15, the last day of a Season for a player shall be the day before that player's Off-Season begins. The scheduling of testing and collection of blood samples will be conducted according to a random player selection procedure by a third-party organization, and neither the NBA, the Players Association, any Team, or any player will have any involvement in selecting the players to be tested or will receive prior notice of the testing schedule; provided, however, that it shall not be a violation of the foregoing for the third-party organization (or a specimen collector for the same) to provide advance notice of a scheduled collection to an NBA Team Security Representative, so long as such notice does not identify the player(s) who will be tested and seeks merely to facilitate access of the collector to the testing location. HGH Blood Testing may also take place under Section 5 (Reasonable Cause Testing or Hearing) and Section 9 (Steroids and Performance-Enhancing Drugs Program) above, and Section 16 (Additional Bases for Testing) below. (For clarity, the number of random blood tests for Human Growth Hormone pursuant to this Section 15 shall be in addition to the number of random urine tests for other Prohibited Substances called for in Section 6 above.) HGH Blood Testing may occur at the same time that players undergo random urine tests for other SPEDs, subject to the procedures governing game-day blood testing set forth in Section 4(b) above and Exhibit I-4 to this Agreement.
\item
  In the event that a player tests positive for a Human Growth Hormone pursuant to this Section 15, he shall enter the SPED Program and suffer the consequences set forth in Section 9 above.
\item
  The isoform test for HGH blood testing will be used with corresponding decision limits issued by the World Anti-Doping Agency in December of 2020 (the ``WADA Decision Limits'') for positive test results. (The WADA Decision Limits are 1.84 for kit 1 and 1.91 for kit 2 for male athletes.)
\end{enumerate}

\hypertarget{additional-bases-for-testing.}{%
\section{Additional Bases for Testing.}\label{additional-bases-for-testing.}}

\begin{enumerate}
\def\labelenumi{(\alph{enumi})}
\tightlist
\item
  Any player who seeks treatment outside the Program for a problem involving a Prohibited Substance, marijuana, or alcohol shall, as directed by the NBA (after notice to the Players Association), submit himself to an evaluation by the Medical Director or SPED Medical Director (as applicable) and provide (or cause to be provided) to the Medical Director or SPED Medical Director (as applicable) such medical and treatment records as the Medical Director or SPED Medical Director (as applicable) may request. The Medical Director or SPED Medical Director (as applicable) may, in his or her professional judgment, also require such a player, without prior notice, to submit to testing for Prohibited Substances, provided that the frequency of such testing shall not exceed three (3) times per week and the duration of such testing shall not exceed one (1) year from the date of the player's initial evaluation by the Medical Director or SPED Medical Director (as applicable).
\item
  Any player who is subject to in-patient or aftercare treatment in the Program and is formally charged with ``driving while intoxicated,'' ``driving under the influence of alcohol,'' or any other crime or offense involving suspected alcohol, marijuana, or illegal substance use shall, provided that the NBA has advised the Players Association, be required to submit to a urine test, to be conducted by the NBA, within seven (7) days of being so charged.
\item
  If, pursuant to Section 16(a) above, a player (i) tests positive for a Drug of Abuse other than a Benzodiazepine; (ii) tests positive pursuant to Section 4(d)(iii), (iv), or (v) above; or (iii) refuses or fails to submit to an evaluation or provide (or cause to be provided) the information requested by the Medical Director, but does not Come Forward Voluntarily within sixty (60) days of being requested to do so by the NBA (with notice to the Players Association), or if, pursuant to Section 16(b) above, a player tests positive for a Drug of Abuse other than a Benzodiazepine, then, in either case the player shall advance two stages in the Drugs of Abuse Program---i.e., the player shall enter Stage 2 of the Drugs of Abuse Program (if the player had not previously entered Stage 1 of such Program), and the player shall be dismissed and disqualified from any association with the NBA or any of its Teams in accordance with the provisions of Section 12(a) above (if the player had previously entered Stage 1 or Stage 2 of such Program).
\item
  If, pursuant to Section 16(a) or (b) above, a player tests positive for a SPED, Benzodiazepine, or synthetic cannabinoid, he shall suffer the applicable consequences set forth in Section 9 or 10 above, as the case may be. If, pursuant to Section 16(a) or (b) above, a player tests positive for a Diuretic, he shall be deemed to have tested positive for a SPED and shall suffer the applicable consequences set forth in Section 9 above.
\item
  If a player is or, within the previous six (6) months, (i) has been in possession of any device or product used or designed for substituting, diluting, or adulterating a specimen sample, or (ii) has been subject to a finding by another sports league or anti-doping organization that he has substituted, diluted, or adulterated a specimen sample and that finding has not been overturned on appeal, that player shall be required to undergo testing for Prohibited Substances no more than four (4) times during the six-week period following his notification by the NBA of the commencement of such testing. If the player (i) tests positive for a Drug of Abuse other than a Benzodiazepine or (ii) tests positive pursuant to Section 4(d)(iii), (iv), or (v) above, he shall be dismissed and disqualified from any association with the NBA or any of its Teams in accordance with the provisions of Section 12(a) above. If the player tests positive for a SPED, Benzodiazepine, or synthetic cannabinoid, he shall suffer the applicable consequences set forth in Section 9 or 10 above, as the case may be. If the player tests positive for a Diuretic, he shall be deemed to have tested positive for a SPED and shall suffer the applicable consequences set forth in Section 9 above. A player who tests positive for a Drug of Abuse or a SPED pursuant to this Section 16(e) may have his dismissal and disqualification or other penalty reduced or rescinded by the Grievance Arbitrator in accordance with Section 20 below.
\item
  Nothing in this Section 16 shall limit or otherwise affect any of the provisions of Section 5 (Reasonable Cause Testing or Hearing).
\end{enumerate}

\hypertarget{additional-prohibited-substances-and-testing-methods.}{%
\section{Additional Prohibited Substances and Testing Methods.}\label{additional-prohibited-substances-and-testing-methods.}}

\begin{enumerate}
\def\labelenumi{(\alph{enumi})}
\tightlist
\item
  Any steroid or performance-enhancing drug that is declared illegal during the term of this Agreement will automatically be added to the list of Prohibited Substances as a SPED.
\item
  At any time during the term of this Agreement, either the NBA or the Players Association may convene a meeting of the Prohibited Substances Committee to request that a substance or substances be added to the list of Prohibited Substances set forth on Exhibit I-2 to this Agreement. Any such addition of a Prohibited Substance may only include a substance that is or is reasonably likely to be harmful to Players and is or is reasonably likely to be improperly performance-enhancing. The determination of the Committee to add to the list of Prohibited Substances shall be made by a majority vote of all five (5) Committee members, and shall be final, binding, and unappealable.
\item
  Players will receive notice of any addition to the list of Prohibited Substances six (6) months prior to the date on which such addition becomes effective under this Article XXXIII.
\item
  At any time during the term of this Agreement, either the NBA or the Players Association may convene a meeting of the Prohibited Substances Committee to request that a testing method be added to the Program. Pursuant to this Section 17(d), the Prohibited Substances Committee shall have the authority to: (i) determine what testing methods will be used to detect newly added Prohibited Substances under the Program, if such Prohibited Substances are detected by methods not currently used by the Program's laboratories; and (ii) approve the use of new testing methods for current Prohibited Substances when such methods have been developed or validated during the term of this Agreement; provided, however, that the Prohibited Substances Committee shall not have the authority to add a testing method that would require a change to the manner in which specimens are collected from players (such as a change from urine collections to blood collections). Any determination of the Committee pursuant to this Section 17(d) shall be made by a majority vote of all five (5) Committee members, and shall be final, binding, and unappealable.
\end{enumerate}

\hypertarget{prescriptions-under-the-anti-drug-program.}{%
\section{Prescriptions Under the Anti-Drug Program.}\label{prescriptions-under-the-anti-drug-program.}}

\begin{enumerate}
\def\labelenumi{(\alph{enumi})}
\tightlist
\item
  Notwithstanding the confidentiality provisions of Section 3 of this Article XXXIII, before any player is prescribed a drug or substance (whether or not it is a Prohibited Substance) as part of his treatment in the Program, the Medical Director or SPED Medical Director (as applicable) will notify the designated physician of the player's team of the name of the drug or substance (the ``Proposed Substance''), the medical justification for the prescription of the Proposed Substance, and the name of the prescribing physician.
\item
  If the designated physician of the player's team advises the Medical Director or SPED Medical Director (as applicable) -- at that time or at any time thereafter -- that the Proposed Substance would create a possible adverse reaction with another prescription substance that the player is being administered, a discussion will be held among the Medical Director or SPED Medical Director (as applicable), the prescribing physician, and the designated team physician with respect to modifying one or both of the prescriptions so as to avoid the potential adverse reaction.
\item
  If the Medical Director or SPED Medical Director (as applicable) becomes aware that a player has been traded to or signed with another team after notification has been made to a designated team physician under Section 18(a) above, the Medical Director or SPED Medical Director (as applicable) is required to make the same notification to the designated team physician of the player's new team and to have the discussion required by Section 18(b) above.
\item
  A team physician who receives a notification from the Medical Director or SPED Medical Director (as applicable) under this Section 18 may only disclose the prescription for the Proposed Substance to other members of the team medical staff who are required to be advised of the prescription in order to ensure that the player is receiving proper medical care from the team's medical staff, and to no other person.
\end{enumerate}

\hypertarget{longitudinal-profile-for-exogenous-testosterone.}{%
\section{Longitudinal Profile for Exogenous Testosterone.}\label{longitudinal-profile-for-exogenous-testosterone.}}

\begin{enumerate}
\def\labelenumi{(\alph{enumi})}
\tightlist
\item
  A longitudinal profile for exogenous testosterone will be established for each player (the ``Longitudinal Profile''). The sole purpose of the Longitudinal Profile is to assist the laboratory selected by the parties to perform the analysis of primary specimens for urine tests under the Program (the ``Laboratory'') in determining which specimens shall be subjected to carbon isotope ratio mass spectrometry (``IRMS'') analysis.
\item
  Players' Longitudinal Profiles will be created pursuant to the protocol set forth in Exhibit I-7. The three (3) tests used to create the Longitudinal Profiles will be random tests conducted under Section 6 above, and the creation of the Longitudinal Profiles will not require any player to undergo any testing in addition to the required random testing set forth in Section 6 above.
\item
  Once a player's Longitudinal Profile is established, the director of the Laboratory will consider the player's Baseline Values (as defined in Exhibit I-7) in comparison to the normalized Testosterone concentration, normalized Epitestosterone concentration or the corresponding Baseline Testosterone/Epitestosterone ratio (collectively, the ``Specimen Values'') of the player's subsequent urine specimens and will determine, in his or her discretion, whether to conduct an IRMS analysis on a urine specimen. In addition, the Laboratory will randomly select urine specimens for IRMS analysis to ensure that such analysis is conducted on at least one (1) urine specimen from every player during each year covered by the Program (i.e., from October 1 through September 30). The decision regarding whether to conduct IRMS analysis on a urine specimen for any other reason will remain in the discretion of the director of the Laboratory.
\end{enumerate}

\hypertarget{no-significant-fault-or-negligence-by-player.}{%
\section{No Significant Fault or Negligence by Player.}\label{no-significant-fault-or-negligence-by-player.}}

\begin{enumerate}
\def\labelenumi{(\alph{enumi})}
\tightlist
\item
  If a player proves by clear and convincing evidence that he bears no significant fault or negligence for the presence of a Drug of Abuse or a SPED in his test result, the Grievance Arbitrator may, in a proceeding brought under Article XXXI of this Agreement, reduce or rescind the penalty otherwise applicable under this Article XXXIII. Such reduction or rescission (if any) will be determined at the discretion of the Grievance Arbitrator.
\item
  For purposes of this Section 20, ``no significant fault or negligence'' means the unusual circumstance in which the Player did not know or suspect, and could not reasonably have known or suspected, even with the exercise of considerable caution and diligence, that he was taking, ingesting, applying, or otherwise using the Drug of Abuse or SPED. To show that he bears no significant fault or negligence, the Player must also establish how the Drug of Abuse or SPED entered his system. A Player cannot satisfy his burden by merely denying that he intentionally used the Drug of Abuse or SPED.
\end{enumerate}

\hypertarget{g-league-suspensions.}{%
\section{G League Suspensions.}\label{g-league-suspensions.}}

Any player suspended under the NBAGL anti-drug program who signs a Uniform Player Contract before the full term of the suspension is served shall continue to serve the suspension in the NBA for the lesser of: (i) the number of games remaining on the suspension imposed under the NBAGL anti-drug program when the player signs the Player Contract, or (ii) the difference between the maximum number of games for which the player could have been suspended under the Program for the same violation and the number of games of the suspension already served by the player in the NBAGL. In addition, any player suspended under the NBAGL anti-drug program whose NBAGL contract ends before the full term of the suspension is served shall be subject to Article VI, Section 1(c) of this Agreement with respect to the NBAGL suspension if, at the start of the following NBA Regular Season, he is a Free Agent who has games remaining to be served on the NBAGL suspension. For purposes of Article VI, Section 1(c), the ``Team to which he was under contract when the suspension was imposed'' shall be deemed to be the NBA Team, if any, with which the player first signs a Player Contract following imposition of the NBAGL suspension.

\hypertarget{recognition-clause}{%
\chapter{RECOGNITION CLAUSE}\label{recognition-clause}}

The NBA recognizes the Players Association as the exclusive collective bargaining representative of all persons who are employed by NBA Teams as professional basketball players and/or who may become so employed during the term of any collective bargaining agreement between the parties or any extension thereof: (a) all persons who are employed by NBA Teams as professional basketball players; (b) all persons who have been previously employed by an NBA Team as professional basketball players who are seeking employment with an NBA Team as a professional basketball player; (c) all rookie players selected in each year's NBA Draft; and (d) all undrafted rookie players seeking employment with an NBA Team as a professional basketball player. The Players Association warrants that it is duly empowered to enter into this Agreement for and on behalf of such persons. The NBA and the Players Association agree that, notwithstanding the foregoing, such persons and NBA Teams may, on an individual basis, bargain with respect to and agree upon the provisions of Player Contracts, but only as and to the extent permitted by this Agreement.

\hypertarget{savings-clause}{%
\chapter{SAVINGS CLAUSE}\label{savings-clause}}

In the event that any provision hereof is found to be inconsistent with the Internal Revenue Code of 1986, as amended (or the rules and regulations issued thereunder (the ``Code'')), the National Labor Relations Act, any other federal, state, provincial, or local statute or ordinance, or the rules and regulations of any other government agency, or is determined to have an adverse effect upon the right of the NBA (or any successor entity) to a tax exemption under Section 501(c)(6) of the Code (or any successor section of like import), then the parties hereto agree to make such changes as are necessary to avoid such inconsistency or to obtain or maintain such exemption retaining, to the extent possible, the intention of such provision.

\hypertarget{player-agents}{%
\chapter{PLAYER AGENTS}\label{player-agents}}

\hypertarget{approval-of-player-contracts.}{%
\section{Approval of Player Contracts.}\label{approval-of-player-contracts.}}

The NBA shall not approve any Player Contract between a player and a Team unless such player (a) is represented in the negotiations with respect to such Player Contract by an agent or representative duly certified by the Players Association in accordance with the Players Association's Regulations Governing Player Agents and authorized to represent him, or (b) acts on his own behalf in negotiating such Player Contract.

\hypertarget{fines.}{%
\section{Fines.}\label{fines.}}

The NBA shall impose a fine of \$50,000 upon any Team that negotiates a Player Contract with an agent or representative not certified by the Players Association in accordance with the Players Association's Regulations Governing Player Agents if, at the time of such negotiations, such Team either (a) knows that such agent or representative has not been so certified, or (b) fails to make reasonable inquiry of the NBA as to whether such agent or representative has been so certified. Notwithstanding the preceding sentence, in no event shall any Team be subject to a fine if the Team negotiates a Player Contract with an agent or representative designated as the player's authorized agent on the then-current agent list provided by the Players Association to the NBA in accordance with Section 5 below.

\hypertarget{prohibition-on-players-as-agents.}{%
\section{Prohibition on Players as Agents.}\label{prohibition-on-players-as-agents.}}

For purposes of negotiating the terms of a Uniform Player Contract or otherwise dealing with a Team over any matter, players are prohibited from (a) representing other current or prospective NBA players as an agent certified under the Players Association's Regulations Governing Player Agents, or (b) holding an equity interest or position in a business entity that represents other current or prospective NBA players as an agent certified under the Players Association's Regulations Governing Player Agents.

\hypertarget{indemnity.}{%
\section{Indemnity.}\label{indemnity.}}

The Players Association agrees to indemnify and hold harmless the NBA, its Teams, and each of its and their respective past, present, and future owners (direct and indirect) acting in their capacity as Team owners, officers, directors, trustees, employees, successors, agents, attorneys, heirs, administrators, executors, and assigns, from any and all claims of any kind arising from or relating to (a) the Players Association's Regulations Governing Player Agents, and (b) the provisions of this Article, including, without limitation, any judgments, costs, and settlements, provided that the Players Association is immediately notified of such claim in writing (and, in no event later than five (5) days from the receipt thereof), is given the opportunity to assume the defense thereof, and the NBA and/or its Teams (whichever is sued) use their best efforts to defend such claim, and do not admit liability with respect to and do not settle such claim without the prior written consent of the Players Association.

\hypertarget{agent-lists.}{%
\section{Agent Lists.}\label{agent-lists.}}

The Players Association agrees to provide the NBA League Office with a list of (a) all agents certified under the Players Association's Regulations Governing Player Agents, and (b) the players represented by each such agent. Such list shall be updated once every two (2) weeks from the day after the NBA Finals to the first day of the next succeeding Regular Season and shall be updated once every month at all other times.

\hypertarget{confirmation-by-the-players-association.}{%
\section{Confirmation by the Players Association.}\label{confirmation-by-the-players-association.}}

If the NBA has reason to believe that the agent representing a player in Contract negotiations is not a certified agent or is not the agent authorized to represent the player, then the NBA may, at its election, request in writing from the Players Association confirmation as to whether the agent who represented the player in the Contract negotiations is in fact the player's certified representative. If within three (3) business days of the date the Players Association receives such written request, the NBA does not receive a written response from the Players Association stating that the agent who represented the player is not the player's certified representative, then the NBA shall be free to act as if the agent is the player's confirmed certified representative.

\hypertarget{agent-rules-compliance.}{%
\section{Agent Rules Compliance.}\label{agent-rules-compliance.}}

\begin{enumerate}
\def\labelenumi{(\alph{enumi})}
\tightlist
\item
  If the NBA notifies the Players Association that it has reasonable cause to believe that an agent or representative has engaged in conduct that violates this Agreement, the Players Association will review any information supplied by the NBA, determine whether to conduct an investigation of the alleged conduct, and, if the Players Association concludes that misconduct occurred, inform the NBA as to the result of its investigation and any discipline it has imposed.
\item
  The Players Association will amend the Players Association's Regulations Governing Player Agents to expressly provide that the Players Association may impose a fine of up to \$125,000 on an agent for a violation of the rules regarding tampering, public requests or demands for a trade, or the timing of Contract negotiations.
\end{enumerate}

\hypertarget{player-appearances-and-additional-content-activitiesuniform}{%
\chapter{PLAYER APPEARANCES AND ADDITIONAL CONTENT ACTIVITIES/UNIFORM}\label{player-appearances-and-additional-content-activitiesuniform}}

\hypertarget{player-activities-on-behalf-of-the-nba.}{%
\section{Player Activities on Behalf of the NBA.}\label{player-activities-on-behalf-of-the-nba.}}

\begin{enumerate}
\def\labelenumi{(\alph{enumi})}
\tightlist
\item
  \textbf{Appearances.} A player may, during each Salary Cap Year covered by a Player Contract to which he is a party, be required to make up to two (2) appearances at the request of NBA Properties, Inc.~in accordance with Paragraph 13(d) of a Uniform Player Contract and Article II, Section 8. Any appearance that a player is required to make shall comply with the terms of Article II, Section 8, and when a player makes an appearance in accordance with this Section, he shall be paid at least \$3,500.
\item
  \textbf{NBA Player Days.} Upon request by the NBA, a player shall be required to participate one (1) time each Salary Cap Year covered by a Player Contract to which he is a party in an NBA-arranged ``NBA Player Day'' content creation session (e.g., filming (x) a sit-down interview with the player where the player discusses basketball and/or non-basketball-related topics, (y) a player's visit to his hometown, or (z) a player's performance in a major NBA brand advertising campaign). A player's required participation in any such session will last no longer than two (2) hours and shall count as one (1) individual or group appearance as applicable under Article II, Section 8(a)(i). An NBA Player Day shall be planned by the NBA with consideration for the player's schedule in accordance with the following:

  \begin{enumerate}
  \def\labelenumii{(\roman{enumii})}
  \tightlist
  \item
    Participation Windows. A player's NBA Player Day for a Salary Cap Year may take place during any of the following periods: (1) during the off-season preceding the applicable Season; (2) during All-Star Weekend for a player who is participating in an All-Star Weekend game or event; or (3) if agreed to by the player and the NBA, on a date during the Season other than All-Star Weekend.
  \item
    Off-Season Scheduling. Prior to the NBA Draft (or, for players selected during such NBA Draft, within fifteen (15) days of such NBA Draft), the NBA will notify those players selected to participate in an NBA Player Day during the off-season, and each such player shall promptly provide either five (5) consecutive dates or eight (8) non-consecutive dates on which he is available to participate within the period beginning one (1) week after the Draft and ending at the start of training camp for the following Season. Within one (1) week of receiving such dates, the NBA will notify the player of which date of those provided has been selected for the player's participation in his NBA Player Day.
  \item
    All-Star Scheduling. At least fifteen (15) days prior to the commencement of All-Star Weekend, the NBA will notify those players who have been selected to participate in an NBA Player Day during that weekend, and the NBA and each such player shall cooperate to designate a mutually agreeable time during such weekend for such player to participate in his NBA Player Day.
  \item
    No Player Travel Required. Unless otherwise agreed to by the player, a player shall not be required to travel from his location (other than local ground transportation) to participate in his NBA Player Day.
  \item
    Advance Notice. Subject to consultation with the respective individual players, the NBA shall provide each player with advance notice of the specific content plans (e.g., one-on-one interview, local charity visit) for his NBA Player Day.
  \end{enumerate}
\item
  \textbf{Player Access to NBA Player Day and Player Appearance Content.} Footage captured by or on behalf of the NBA or Team during an NBA Player Day or a Team content creation activity described in Article II, Section 8(a)(iii)(B) shall, upon request, be made available by the NBA or the Team to the player solely for use in Player-Produced Content (as defined in Section 2(a) below) in accordance with the requirements and terms in Section 2 below. The NBA shall also accommodate a reasonable request from a player to allow a separate player-secured production crew to attend an NBA Player Day involving such player, subject to reasonable parameters provided by the NBA (e.g., relating to SAG rules for applicable content involving SAG talent).
\item
  When a player fails, without reasonable excuse, to appear or reasonably to cooperate during any of the activities referred to in this Section 1, he may be fined for each failure in an amount up to \$20,000.
\end{enumerate}

\hypertarget{additional-content-opportunities-for-the-nba-and-players.}{%
\section{Additional Content Opportunities for the NBA and Players.}\label{additional-content-opportunities-for-the-nba-and-players.}}

\begin{enumerate}
\def\labelenumi{(\alph{enumi})}
\tightlist
\item
  \textbf{Use of NBA Content for Player-Produced Content.} A player shall, upon request, receive footage of such player captured by or on behalf of the NBA or Team described in Section 1(c) above, via a secure content network that the NBA shall use good faith efforts to maintain in regular working order (the ``Content Network''), for the purpose of incorporating such footage into content produced by or on behalf of such player during such Season (``Player-Produced Content''), which may be used by such player subject to the requirements and terms set forth in this Section 2.
\item
  \textbf{Content Participants and Additional Player Content Opportunities.} For any Season, a ``Content Participant'' means (for purposes of this Article XXXVII) a player who agrees in writing by the September 1 prior to the Season either to (x) participate in three (3) content creation activities during the Season, that are in addition to activities otherwise required under this Agreement and/or the Uniform Player Contract and that are described in subsection 2(b)(i) below, or (y) license the use of his Player-Produced Content on NBA and/or Team platforms, as described in subsection 2(b)(ii) below. A Content Participant for a Season shall, upon request, receive footage captured by or on behalf of such player described in Section 2(b)(i) below and game footage and game photos as described in Sections 2(c) and 2(d) below for the purpose of incorporating such footage and photos into Player-Produced Content, and other such uses as described in Sections 2(c) and 2(d) below, during such Season and the immediately following offseason, subject to the requirements and terms set forth in this Section 2. For clarity, ``game footage'' and ``game photos'' include any images from a game, including game action and players on-court, in the tunnel, or elsewhere in the arena or other team facility.

  \begin{enumerate}
  \def\labelenumii{(\roman{enumii})}
  \tightlist
  \item
    Additional Content Creation Activities. If requested by the NBA, a Content Participant who has selected the option described in subsection 2(b)(x) above shall participate in three (3) of any of the following content creation activities each Season (in each case, as determined by the NBA, but for no longer than thirty (30) minutes per activity): (A) a live digital/social content interview or activity; (B) a recorded NBA content session (e.g., a performance in an NBA brand campaign); and (C) an off-court/out-of-arena interview during the Season (e.g., a live remote interview).
  \item
    Provision of Player-Produced Content to the NBA and Team. Upon the NBA's request, a Content Participant who has elected the option described in subsection 2(b)(y) above shall promptly make available to the NBA via the Content Network, and shall be deemed to have licensed the NBA and the Team on a non-exclusive, royalty free basis, Player-Produced Content that includes or depicts any NBA or Team footage described in Sections 1(c) above, any game footage, and/or any game photos, for use, during and after the term of this Agreement, on NBA and/or Team platforms (the ``Player-Produced Content License''), it being understood that the NBA or Team shall be responsible for the procurement of any applicable third-party content licenses to permit it to use such Player-Produced Content. Any Player-Produced Content that is used on NBA and/or Team platforms may not include any sponsorship unless agreed to by the player and the NBA. In addition, this Section 2 does not confer any right or authority for the NBA or any Team to use any Player-Produced Content in a way that constitutes an unauthorized Endorsement (as such term is defined and clarified in Article XXVIII).
  \end{enumerate}
\item
  \textbf{Use of Game Footage.}

  \begin{enumerate}
  \def\labelenumii{(\roman{enumii})}
  \tightlist
  \item
    Automated Delivery of Game Highlights. Promptly following the conclusion of each Regular Season, Play-In, and playoff game in which a Content Participant participates, the NBA shall make available to such Content Participant at least two (2) ``highlights'' (i.e., game footage clips) from that game featuring such player, subject to availability.
  \item
    Selection of Game Highlights by Content Participant. A Content Participant may request to substitute an automatedly-distributed game highlight for another game footage clip of such player from such game and the substituted clip shall be delivered within one (1) week of the request.
  \item
    Use by Content Participants. Game footage shall be made available to a Content Participant via the Content Network and, subject to the terms and requirements of this Section 2, may be posted by the applicable Content Participant on applicable Designated Distribution Platforms (as defined in Section 2(f) below), as follows:

    \begin{enumerate}
    \def\labelenumiii{(\arabic{enumiii})}
    \tightlist
    \item
      On an ``Unedited'' basis --- i.e., not presented in conjunction with or incorporated into any other content (e.g., non-game content, other game highlights) and without alteration other than formatting necessary to post on the applicable platform, it being understood that the foregoing shall not restrict non-video elements that would customarily accompany such a posting (e.g., a written caption accompanying a highlight on a Designated Distribution Platform); and/or
    \item
      As incorporated into Player-Produced Content, provided that (x) a maximum of five (5) minutes of game footage may be incorporated within one (1) or more pieces of Player-Produced Content each Salary Cap Year produced by the player and (y) no single piece of Player-Produced Content may include more than one (1) minute of game footage.
    \end{enumerate}
  \item
    NBA Promotional Messaging in Game Footage. The NBA may include or cause the inclusion of promotional messaging on behalf of the NBA or Team into the game footage that is made available to, and that may be used by, a Content Participant; provided that such promotional messaging is subject to the approval of the Players Association. Each of the following types of promotional messaging is hereby approved by the Players Association: (1) game tune-in messaging; (2) promotion of NBA-branded and controlled content platforms (e.g., the NBA mobile application); and (3) ticket sales promotion.
  \end{enumerate}
\item
  \textbf{Use of Game Photos.}

  \begin{enumerate}
  \def\labelenumii{(\roman{enumii})}
  \tightlist
  \item
    Automated Delivery of Game Photos. Promptly following the conclusion of each Regular Season, Play-In, and playoff game, the NBA shall make available to each Content Participant for such Season that is active during the game at least two (2) game photos of the player from that game, subject to availability, which, subject to the terms and requirements of this Section 2, the player may post on applicable Designated Distribution Platforms.
  \item
    Selection of Game Photos by Content Participant. A Content Participant may request to substitute an automatedly-distributed game photo for another selected game photo of such player from such game and the substituted photo shall be delivered within one week of the request.
  \item
    Incorporation into Player-Produced Content. A Content Participant may include game photos made available pursuant to this Section 2(d) within Player-Produced Content.
  \end{enumerate}
\item
  \textbf{Authorized Uses.} The NBA hereby licenses on a non-exclusive, royalty-free basis a player to use footage of such player captured by or on behalf of the NBA or Team described in Section 1(c) above for inclusion in Player-Produced Content and, in the case of a Content Participant, footage of such player captured by or on behalf of the NBA or Team described in Section 2(b)(i) above, game footage and game photos, which, in each case, may be made available by such player during and after the term of this Agreement, subject to the requirements and terms of this Section 2, including the following requirements and terms:

  \begin{enumerate}
  \def\labelenumii{(\roman{enumii})}
  \tightlist
  \item
    Player-Produced Content shall be (1) required to be primarily focused on the applicable player (and, for clarity, not on the NBA or any Team, it being understood that such content could include incidental appearances of other players) and (2) subject to any other parameters related to the content licensed from the NBA and reasonably established by the NBA (e.g., a player could not post a clip of an NBA commercial created during an ``NBA Player Day'' until the commercial has been officially launched by the NBA);
  \item
    The incorporation of game footage into Player-Produced Content or other uses of game footage, if applicable, shall be subject to NBA agreements with third parties that limit such use outside of the United States and Canada;
  \item
    Distribution of Player-Produced Content, game footage, and game photos, if applicable, shall be limited to the Designated Distribution Platforms set forth in Section 2(f) below; and
  \item
    Such player shall be responsible for the procurement of any applicable third-party content licenses to permit such player to use such Player-Produced Content, NBA- or team-captured content, game footage and/or game photos.
  \end{enumerate}
\item
  \textbf{Designated Distribution Platforms.} A player shall be permitted to distribute or ``post'' (but, for clarity, not to license in a commercial transaction) applicable Player-Produced Content and, with respect to Content Participants, game footage and game photos on the following distribution platforms (the ``Designated Distribution Platforms''): (i) the player's own ``official'' (i.e., controlled by the player and/or his representatives) website and mobile application; and (ii) the player's own ``official'' handles on Social Media Platforms. ``Social Media Platforms'' means social media platforms on which the NBA and/or NBA Teams then-currently post content through official handles. Upon request, the NBA shall provide from time to time the list of Social Media Platforms that meet the above definition.
\item
  \textbf{Limitation on Sponsorship and Advertising.} No game footage, game photos, and/or Player-Produced Content that includes any NBA- or Team-captured content, game footage and/or game photos, may be used or distributed by a player in connection with any sponsorship or advertising for any product, service, or brand (including, for clarity, sponsorship of a Designated Distribution Platform on which any such content appears).
\item
  \textbf{Other Permitted Uses Subject to NBA Approval.} A player may use game footage, game photos, or NBA-captured content outside of the parameters described in this Section 2, and/or for a commercial purpose, if approved in writing by the NBA. Players and the NBA each retain the right to enter into arm's length transactions, on mutually agreeable terms, to license game footage and game photos, and/or other NBA intellectual property, including without the player having to grant any corresponding licenses to the NBA.
\item
  \textbf{Content Advisory Council.} A Content Advisory Council, composed of players and representatives from the Players Association and NBA, shall be established to consult on the development and distribution of Player-Produced Content and other content described in this Section 2 and any process or operational matters related to the cooperation of the parties as described in this Section. The Content Advisory Council shall also discuss, among other things, best practices and opportunities in connection with the creation of content.
\item
  \textbf{Process Matters.} The NBA (on behalf of itself and the NBA Teams) shall designate a representative responsible for communicating and receiving the requests and notifications under this Section (e.g., a player's request for NBA Player Day footage) and shall communicate with players directly with respect to such requests and notifications.
\end{enumerate}

\hypertarget{indemnity.-1}{%
\section{Indemnity.}\label{indemnity.-1}}

Neither the NBA nor any Team shall be liable for any monetary or non-monetary claims arising out of or relating to Paragraphs 13(f), or 14(a) of the Uniform Player Contract (the ``Designated Player Contract Provisions''). The Players Association indemnifies, saves and holds harmless the NBA, each Team, and each of its and their respective affiliates, owners, directors, governors, officers and employees, and the successors, assigns and personal representatives of the foregoing parties, against any and all claims, demands, suits, or other forms of liability that may arise, directly or indirectly, in connection with the Designated Player Contract Provisions, including their enforcement and/or application, provided that any such claim is not due to the breach of the Group License (as defined in the Uniform Player Contract) by the NBA, any Team or any League-related entities that generate BRI; provided, further that, the Players Association is notified of such claim in writing (and, in no event later than five (5) days from the receipt thereof), is given the opportunity to assume the defense thereof, and the NBA, its Teams and/or other indemnified parties in accordance with this Section (whichever is sued) use their best efforts to defend such claim, and do not admit liability with respect to and do not settle such claim without the prior written consent of the Players Association.

\hypertarget{uniform.}{%
\section{Uniform.}\label{uniform.}}

\begin{enumerate}
\def\labelenumi{(\alph{enumi})}
\tightlist
\item
  During any NBA game or practice, including warm-up periods and going to and from the locker room to the playing floor, a player shall wear only the Uniform as supplied by his Team. For purposes of the preceding sentence only, ``Uniform'' means all clothing and other items (such as kneepads, wristbands, and headbands, but not including Sneakers) worn by a player during an NBA game or practice. ``Sneakers'' means athletic shoes of the type worn by players while playing an NBA game.
\item
  Other than as may be incorporated into his Uniform and the manufacturer's identification incorporated into his Sneakers, a player may not, during any NBA game, display any commercial, promotional, or charitable name, mark, logo, or other identification, including, but not limited to, on his body, in his hair, or otherwise.
\end{enumerate}

\hypertarget{integration-entire-agreement-interpretation-and-choice-of-law}{%
\chapter{INTEGRATION, ENTIRE AGREEMENT, INTERPRETATION, AND CHOICE OF LAW}\label{integration-entire-agreement-interpretation-and-choice-of-law}}

\chaptermark{INTEGRATION, ENTIRE AGREEMENT, INTERPRETATION, AND \ldots}

\hypertarget{integration-entire-agreement.}{%
\section{Integration, Entire Agreement.}\label{integration-entire-agreement.}}

This Agreement, together with the exhibits hereto, and all letter agreements executed contemporaneously herewith, constitutes the entire understanding between the parties and all understandings, conversations and communications, proposals, and counterproposals, oral and written (including any draft of this Agreement) between the Members of the NBA and the Players Association, or on behalf of them, are merged into and superseded by this Agreement and shall be of no force or effect, except as expressly provided herein. No such understandings, conversations, communications, proposals, counterproposals, or drafts shall be referred to in any proceeding by the parties. Further, no understanding contained in this Agreement shall be modified, altered, or amended, except by a writing signed by the party against whom enforcement is sought.

\hypertarget{interpretation.}{%
\section{Interpretation.}\label{interpretation.}}

\begin{enumerate}
\def\labelenumi{(\alph{enumi})}
\tightlist
\item
  The NBA and Players Association recognize and acknowledge that there is and may continue to be (i) a collective bargaining relationship between WNBA, LLC (``WNBA'') and the Women's National Basketball Players Association (``WNBPA''), and (ii) a collective bargaining relationship between the NBAGL and the Next Gen Basketball Players Union (``NGBPU''), each of which is separate and distinct from the collective bargaining relationship between the NBA and the Players Association.
\item
  The NBA and the Players Association agree that this Agreement shall be interpreted without reference: (i) to any past, present, or future WNBA/WNBPA collective bargaining agreement (or to any other past, present, or future agreement between the WNBA or WNBA Enterprises, LLC, on the one hand, and the WNBPA on the other) or to any past, present, or future Standard Player Contract, Team Marketing and Promotional Agreement, or WNBA Marketing and Promotional Agreement (collectively, ``WNBA Agreements''); (ii) to any past, present, or future NBAGL/NGBPU collective bargaining agreement (or to any other past, present, or future agreement between the NBAGL or any NBAGL commercial entity, on the one hand, and the NGBPU on the other) or to any past, present, or future employment or marketing agreements between the NBAGL and the players in the NBAGL (collectively, ``NBAGL Agreements''); (iii) to any of the provisions of such agreements or contracts; (iv) to the fact that a subject was not or is not covered by or included in any such agreements or contracts; and/or (v) to any judicial, arbitral, or administrative decision interpreting any of such agreements or contracts.
\item
  The parties agree that they will make no reference to any of the WNBA Agreements, NBAGL Agreements, contracts, or decisions referred to in Section 2(b) above, or to the fact that a particular provision was not or is not included in any such agreement or contract, or to any practice or policy of the WNBA (or WNBA Enterprises, LLC), the NBAGL (or any NBAGL commercial entity), the WNBPA, or the NGBPU, in any arbitral, judicial, administrative, or other proceeding concerning the interpretation or enforcement of this Agreement, including, without limitation, a proceeding brought under Articles XXXI or XXXII of this Agreement. The parties further agree that no such agreement, contract, provision (or absence of provisions), decision, practice, or policy may be relied upon by any decision maker in such proceedings.
\end{enumerate}

\hypertarget{choice-of-law.}{%
\section{Choice of Law.}\label{choice-of-law.}}

This Agreement (including the Uniform Player Contract and all other Exhibits to this Agreement) is made under and shall be governed by the internal law of the State of New York, except where federal law may govern.

\hypertarget{term-of-agreement}{%
\chapter{TERM OF AGREEMENT}\label{term-of-agreement}}

\hypertarget{effective-date-and-expiration-date.}{%
\section{Effective Date and Expiration Date.}\label{effective-date-and-expiration-date.}}

This Agreement shall be effective from July 1, 2023 (except with respect to provisions that the parties have specifically agreed herein will commence earlier) and, unless terminated pursuant to the provisions of this Article XXXIX, shall continue in full force and effect through June 30, 2030 (except with respect to provisions that the parties have specifically agreed herein will survive expiration or termination).

\hypertarget{mutual-options-to-terminate-following-sixth-season.}{%
\section{Mutual Options to Terminate Following Sixth Season.}\label{mutual-options-to-terminate-following-sixth-season.}}

The NBA and the Players Association shall each have the option to terminate this Agreement on June 30, 2029 by serving written notice of its exercise of such option on the other party on or before October 15, 2028.

\hypertarget{termination-by-players-associationanti-collusion.}{%
\section{Termination by Players Association/Anti-Collusion.}\label{termination-by-players-associationanti-collusion.}}

\begin{enumerate}
\def\labelenumi{(\alph{enumi})}
\tightlist
\item
  In the event the conditions of Article XIV, Section 15 are satisfied, the Players Association shall have the right to terminate this Agreement by serving written notice of its exercise of such right within thirty (30) days after the System Arbitrator's report finding the requisite conditions (pursuant to Article XIV, Section 15) becomes final and any appeals therefrom have been exhausted or, in the absence of a System Arbitrator, by serving such written notice upon the NBA within thirty (30) days after any decision by a court finding the requisite conditions (pursuant to Article XIV, Section 15). In the latter situation, if the finding of the court is reversed on appeal, the Agreement shall be immediately reinstated and both parties reserve their rights with respect to any conduct by the other party during the period from the date of service of the termination notice to the date upon which the Agreement was reinstated.
\item
  If the Players Association exercises the right accorded it by Section 3(a) above, this Agreement shall terminate as of the June 30 immediately following the service of the termination notice.
\end{enumerate}

\hypertarget{termination-by-nbanational-tv-revenues.}{%
\section{Termination by NBA/National TV Revenues.}\label{termination-by-nbanational-tv-revenues.}}

\begin{enumerate}
\def\labelenumi{(\alph{enumi})}
\tightlist
\item
  For the purposes of this provision: (i) ``National TV Revenues'' shall mean the rights fees or other non-contingent payments stated in the NBA's third-party national broadcast network (e.g., ABC) and cable network (e.g., TNT or ESPN) television agreements (each, a ``National TV Agreement''); and (ii) ``Other Media Income'' shall mean the aggregate net income earned by any League-related entity (as defined in Article VII, Section 1(a)(1)) (but excluding net income attributable to ownership interests in any such League-related entity that is not owned by the NBA, NBA Properties, Inc., NBA Media Ventures, LLC, and/or a group of NBA Teams) or by the NBA on behalf of the Teams from agreements that provide for the transmission of live (or delayed) NBA games, on a domestic or international basis, by means of television, radio, internet, and any other mode of delivery referenced in Article VII, Section 1(a)(1)(ii), net of reasonable and customary expenses related thereto.
\item
  If, during the term of this Agreement, (i) the sum of the average annual National TV Revenues provided for under the Successor Agreements (as defined in Article VII, Section 1(c)(2)), plus one hundred four and one-half percent (104.5\%) of Other Media Income for the most recent Salary Cap Year, will be at least thirty-five percent (35\%) less than (ii) the sum of the average annual National TV Revenues provided for under the NBA/ABC and NBA/TBS Agreements, plus Other Media Income for the 2022-23 Salary Cap Year, the NBA shall have the right to terminate this Agreement effective as of the June 30 immediately preceding the first Season covered by the Successor Agreements, by providing written notice of such termination to the Players Association at least sixty (60) days prior to such June 30. During the period following delivery of such written notice of termination and through such June 30, the NBA and the Players Association shall engage in good faith negotiations for the purpose of entering into a successor agreement and the provisions of Article XXX shall remain in full force and effect.
\end{enumerate}

\hypertarget{termination-by-nbaforce-majeure.}{%
\section{Termination by NBA/Force Majeure.}\label{termination-by-nbaforce-majeure.}}

\begin{enumerate}
\def\labelenumi{(\alph{enumi})}
\tightlist
\item
  ``Force Majeure Event'' shall mean the occurrence of any of the following events or conditions, provided that such event or condition either (i) makes it impossible for the NBA to perform its obligations under this Agreement, or (ii) frustrates the underlying purpose of this Agreement, or (iii) makes it economically impracticable for the NBA to perform its obligations under this Agreement: wars or war-like action (whether actual or threatened and whether conventional or other, including, but not limited to, chemical or biological wars or war-like action); sabotage, terrorism, or threats of sabotage or terrorism; explosions; epidemics; weather or natural disasters, including, but not limited to, fires, floods, droughts, hurricanes, tornados, storms, or earthquakes; and any governmental order or action (civil or military); provided, however, that none of the foregoing enumerated events or conditions is within the reasonable control of the NBA or an NBA Team.
\item
  In addition to any other rights a Team or the NBA may have by contract or by law, if a Force Majeure Event occurs and, as a result, one or more Teams are unable to play one or more games (whether Exhibition, Regular Season, Play-In, or playoff games), then, for each missed game during such period (the ``Force Majeure Period'') that was not rescheduled and replayed, the Compensation payable to each player who was on the roster of a Team that was unable to play one or more games during the Force Majeure Period shall be reduced by 1/92.6th of the player's Compensation for the Season(s) covering the Force Majeure Period. For purposes of the foregoing calculation, and notwithstanding the actual number of games that any Team played, was scheduled to play, or could have played during the Seasons(s) affected by the Force Majeure Event, each Team shall be deemed to play five (5) Exhibition games, eighty-two (82) Regular Season games, and 5.6 playoff games during each such Season.
\item
  In the event that Section 5(b) above applies, the applicable Compensation reduction from each player shall be withheld by the player's Team from the first Compensation payment (or payments, if the first such payment is insufficient to satisfy the reduction) that is (or are) due or to become due to such player following the commencement of the Force Majeure Period (whether under the Player Contract that was in existence at the commencement of the Force Majeure Period or any subsequent Player Contract between the player and the Team). If such Compensation payment (or payments) is (or are) insufficient to cover the Compensation reduction required by Section 5(b) above, then either (i) the player shall promptly pay the difference directly to the Team (``old Team''), or (ii) if he subsequently enters into a Player Contract with, or is traded to, another NBA Team (``new Team''), such difference shall be withheld from the first available Compensation payment (or payments, if the first such payment is insufficient to satisfy the remaining reduction) that is (or are) due to the player from the new Team and shall be remitted by the new Team to the old Team.
\item
  Upon the occurrence of a Force Majeure Event satisfying the terms of Section 5(a) above, the NBA shall have the right to terminate this Agreement as of the sixtieth (60th) day following delivery to the Players Association of a written notice of termination, which must be delivered to the Players Association within sixty (60) days of the Force Majeure Event. During the sixty-day period following delivery of such written notice of termination, the NBA and the Players Association shall engage in good faith negotiations for the purpose of entering into a successor agreement, and during such period the provisions of Article XXX shall remain in full force and effect.
\end{enumerate}

\hypertarget{mutual-right-of-termination.}{%
\section{Mutual Right of Termination.}\label{mutual-right-of-termination.}}

If at any time during the term of this Agreement any provision contained in Article VII, X, XI, or XIV of this Agreement is enjoined, vacated, declared null and void, or is rendered unenforceable by any court of competent jurisdiction, then the NBA and the Players Association shall each have the right to terminate this Agreement by serving upon the other party written notice of termination at least sixty (60) days prior to the effective date of such termination.

\hypertarget{mutual-right-of-termination-league-financial-results.}{%
\section{Mutual Right of Termination -- League Financial Results.}\label{mutual-right-of-termination-league-financial-results.}}

If, as determined by the Governing Audit Report for a Salary Cap Year (the ``Trigger Salary Cap Year''):

\begin{enumerate}
\def\labelenumi{(\alph{enumi})}
\tightlist
\item
  The sum of (i) Team Content Expenses, (ii) League Content Expenses, and (iii) any amortized amount of prior year expenses deductible pursuant to Article VII, Section 1(a)(6)(vi) above, deducted from BRI for the Trigger Salary Cap Year exceeds twenty-five percent (25\%) of the sum of Team Content Revenues and League Content Revenues; or
\item
  BRI for the Trigger Salary Cap Year is less than ninety-five percent (95\%) of the highest BRI amount for any prior Salary Cap Year;
\end{enumerate}

then (i) the NBA and Players Association shall negotiate in good faith to agree upon adjustments to the provisions of this Agreement to take effect beginning with the Salary Cap Year immediately following the Trigger Salary Cap Year, and (ii) if the parties are unable to agree to adjustments to this Agreement in accordance with the foregoing, then the NBA and Players Association will each have the option to terminate this Agreement effective as of the June 30 of the Salary Cap Year immediately following the Trigger Salary Cap Year, by serving written notice of its exercise of such option on the other party on or before the date that is sixty (60) days following the issuance of the Governing Audit Report for the Trigger Salary Cap Year. During the period following delivery of such written notice of termination through the last day of the Salary Cap Year immediately following the Trigger Salary Cap Year, the NBA and the Players Association shall engage in good faith negotiations for the purpose of entering into a successor agreement and during such period the provisions of Article XXX shall remain in full force and effect.

\hypertarget{mutual-right-of-termination-designated-share.}{%
\section{Mutual Right of Termination -- Designated Share.}\label{mutual-right-of-termination-designated-share.}}

If, as determined by the Governing Audit Report for a Trigger Salary Cap Year (and, with respect to clause (c) below, the Governing Audit Report for the Salary Cap Year immediately preceding the Trigger Salary Cap Year):

\begin{enumerate}
\def\labelenumi{(\alph{enumi})}
\tightlist
\item
  the Aggregate Team Overage Balance in respect of the Trigger Salary Cap Year or any Salary Cap Year preceding the Trigger Salary Cap Year, after giving effect to the processes set forth in Article VII, Section 12(e)(1)-(2) for the Trigger Salary Cap Year, is greater than zero (0); or
\item
  the Shortfall Amount for the Trigger Salary Cap Year is greater than twenty-five percent (25\%) of Adjusted Total Salaries for such Salary Cap Year; or
\item
  with respect to each of the Trigger Salary Cap Year and the Salary Cap Year immediately preceding the Trigger Salary Cap Year, the Shortfall Amount for the year is greater than ten percent (10\%) of Adjusted Total Salaries for such year;
\end{enumerate}

then (i) the NBA and Players Association shall negotiate in good faith to agree upon adjustments to the provisions of this Agreement as may be appropriate to effect (1) in the case of clause (a) above, a timely recoupment of the outstanding Aggregate Team Overage Balance and any potential future Aggregate Team Overage Balances, and (2) in the case of clauses (b) and (c) above, a more timely distribution of the Designated Share into Total Salaries, with such adjustments to take effect beginning with the Salary Cap Year immediately following the Trigger Salary Cap Year, and (ii) if the parties are unable to agree to adjustments to this Agreement in accordance with the foregoing, then the NBA and Players Association will each have the option to terminate this Agreement effective as of the June 30 of the Salary Cap Year immediately following the Trigger Salary Cap Year, by serving written notice of its exercise of such option on the other party on or before the date that is sixty (60) days following the issuance of the Governing Audit Report for the Trigger Salary Cap Year. During the period following delivery of such written notice of termination through the last day of the Salary Cap Year immediately following the Trigger Salary Cap Year, the NBA and the Players Association shall engage in good faith negotiations for the purpose of entering into a successor agreement and during such period the provisions of Article XXX shall remain in full force and effect.

\hypertarget{mutual-right-of-termination-league-entity-transaction.}{%
\section{Mutual Right of Termination -- League Entity Transaction.}\label{mutual-right-of-termination-league-entity-transaction.}}

In the event a sale or transfer of ownership interests in a League-related entity that, prior to such sale or transfer, generated \$50 million or more of annual revenues included in BRI results in such entity ceasing to be a League-related entity, including in circumstances where a League-related entity continues to hold a non-controlling minority ownership interest in such entity following such sale or transfer, the parties shall negotiate in good faith such modifications to the CBA as may be appropriate, to take effect beginning with the Salary Cap Year in which such sale or transfer occurs and taking into account all relevant facts and circumstances including the amounts included in BRI prior to such sale or transfer, to ensure a fair inclusion of amounts in BRI following such sale or transfer. In the event the parties are unable to reach an agreement on CBA modifications within thirty (30) days of such a sale or transfer, either party may thereafter for a period of thirty (30) days elect to terminate the CBA, by written notice to the other party, effective as of the June 30 immediately following the service of the termination notice (or, if later, as of the first June 30 that is at least sixty (60) days following the service of the termination notice). Should either party terminate the CBA in accordance with the foregoing effective as of any June 30, then:

\begin{enumerate}
\def\labelenumi{(\alph{enumi})}
\tightlist
\item
  For the Salary Cap Year encompassing such June 30:

  \begin{enumerate}
  \def\labelenumii{(\roman{enumii})}
  \tightlist
  \item
    BRI shall be calculated in accordance with the provisions of this Agreement, except as set forth in subsection (a)(ii) below;
  \item
    the treatment of BRI relating to such sale or transfer shall be determined by agreement of the parties in a successor Collective Bargaining Agreement; and
  \item
    the completion of the Audit Report, and the performance of the calculations and reconciliation processes described in Article VII, Sections 12(d)-(g) (including, for clarity, the distribution of any Overage Amount or Shortfall Amount) shall be deferred pending the completion of an agreement by the parties on a successor Collective Bargaining Agreement, which shall specify the time period for completing such Audit Report, calculations, and reconciliation processes in accordance with the provisions of subsections (a)(i) and (a)(ii) above; and
  \end{enumerate}
\item
  For the Salary Cap Year immediately following such June 30, each of the Salary Cap, Minimum Team Salary, Tax Level, First Apron Level, and Second Apron Level shall increase to an amount that is equal to one hundred five percent (105\%) of its amount for the Salary Cap Year encompassing such June 30, subject to any modification of the foregoing on which the parties agree in a successor Collective Bargaining Agreement.
\end{enumerate}

\hypertarget{no-obligation-to-terminate-no-waiver.}{%
\section{No Obligation to Terminate; No Waiver.}\label{no-obligation-to-terminate-no-waiver.}}

The grant to either party of a right or option to terminate pursuant to the provisions of this Article XXXIX shall not carry with it the obligation to exercise that right or option; and the failure of the NBA or the Players Association to exercise any right or option to terminate this Agreement with respect to any playing Season in accordance with this Article XXXIX shall not be deemed a waiver of or in any way impair or prejudice the NBA or the Players Association's right or option, if any, to terminate this Agreement in accordance with this Article XXXIX with respect to any succeeding Season.

\hypertarget{expansion-and-contraction}{%
\chapter{EXPANSION AND CONTRACTION}\label{expansion-and-contraction}}

\hypertarget{expansion.}{%
\section{Expansion.}\label{expansion.}}

The NBA may determine during the term of this Agreement to expand the number of Teams and to have existing Teams make available for assignment to any such Expansion Teams the Player Contracts of a certain number of Veterans under substantially the same terms and in substantially the same manner that Player Contracts were made available to the Charlotte expansion Team pursuant to the 1999 NBA/NBPA Collective Bargaining Agreement; provided, however, that any change shall be subject to the approval of the Players Association, which shall not be unreasonably withheld.

\hypertarget{contraction.}{%
\section{Contraction.}\label{contraction.}}

If, during the term of this Agreement, the NBA decides to contract the number of Teams, (a) the NBA shall provide written notice of such decision to the Players Association, and (b) the NBA and the Players Association shall negotiate and agree upon the effects of such decision on the players and the procedures to be followed in connection therewith.

\hypertarget{nba-g-league}{%
\chapter{NBA G LEAGUE}\label{nba-g-league}}

\hypertarget{nbagl-work-assignments.}{%
\section{NBAGL Work Assignments.}\label{nbagl-work-assignments.}}

\begin{enumerate}
\def\labelenumi{(\alph{enumi})}
\tightlist
\item
  An NBA Team may at any time assign a player (other than a Two-Way Player) on its Active List or Inactive List to an NBAGL team, provided that the player (i) has either zero (0), one (1), or two (2) Years of Service at the time of the assignment, or (ii) has more than two (2) Years of Service at the time of the assignment and the player and the Players Association consent to such assignment in writing. Upon such assignment (``NBAGL Work Assignment''), the player will be placed on the NBA Team's Inactive List, and shall (A) report to the NBAGL team (and render for the NBAGL team such services as the player is required to render for the NBA Team under his Uniform Player Contract and this Agreement), and (B) at the direction of the NBA Team, subsequently return and report to, and resume the performance of services for, the NBA Team. An NBAGL Work Assignment commences when the player reports in-person to the NBAGL team, and ends either when the player, upon being recalled, reports back to his NBA Team or when the NBAGL Season concludes.
\item
  There shall be no limit on the number of NBAGL Work Assignments given to a player. No NBA Team shall issue an NBAGL Work Assignment for the purpose of disciplining a player for misconduct or retaliating against a player for exercising any right that he has under this Agreement or the Uniform Player Contract.
\item
  The NBA may establish reasonable rules regarding the assignment and recall of players to the NBAGL provided that such rules do not violate the provisions of this Article XLI.
\end{enumerate}

\hypertarget{reporting-requirements-for-nbagl-work-assignments.}{%
\section{Reporting Requirements for NBAGL Work Assignments.}\label{reporting-requirements-for-nbagl-work-assignments.}}

\begin{enumerate}
\def\labelenumi{(\alph{enumi})}
\tightlist
\item
  In order to initiate an NBAGL Work Assignment or terminate such assignment and recall the player, the NBA Team shall provide the player, the NBA, and the Players Association with written notice. The player shall report to the NBAGL team or NBA Team, whichever is applicable, within forty-eight (48) hours after such notice is received by the player.
\item
  If the player, without a reasonable excuse, does not report to the NBAGL team or NBA Team, whichever is applicable, within the time provided in Section 2(a) above, the player may be fined and/or suspended without pay by the NBA Team until such time as he reports. In addition, such failure to report, without a reasonable excuse, shall constitute conduct prejudicial to the NBA under Article 35(d) of the NBA Constitution, subject however to the One Penalty rule set forth in Article VI, Section 10.
\end{enumerate}

\hypertarget{travel-and-relocation-expenses.}{%
\section{Travel and Relocation Expenses.}\label{travel-and-relocation-expenses.}}

A player's NBA Team shall be obligated to reimburse the player for his ordinary and reasonable expenses incurred in (a) traveling to and, when recalled, from the NBAGL team to begin and/or end any NBAGL Work Assignment or period of service on the Two-Way List (``NBAGL Two-Way Service''), and (b) relocating to and, if recalled, from the NBAGL team's home location to begin and/or end any NBAGL Work Assignment or NBAGL Two-Way Service that extends beyond a period of thirty (30) days. During any NBAGL Work Assignment or NBAGL Two-Way Service, the player will be provided with housing or a housing subsidy in accordance with the NBAGL housing policy.

\hypertarget{terms-of-nbagl-work-assignment-and-nbagl-two-way-service.}{%
\section{Terms of NBAGL Work Assignment and NBAGL Two-Way Service.}\label{terms-of-nbagl-work-assignment-and-nbagl-two-way-service.}}

\begin{enumerate}
\def\labelenumi{(\alph{enumi})}
\tightlist
\item
  \textbf{General Terms.} During or in connection with any NBAGL Work Assignment or NBAGL Two-Way Service, and except as expressly set forth in, or limited or modified by, this Article XLI, a player shall (i) accept and be subject to the work requirements and conditions applicable to NBAGL players (as such requirements and conditions may change from time to time), and (ii) continue to be subject to the terms and obligations and entitled to the benefits and rights (including, without limitation, Years of Service and free agency rights) of his Uniform Player Contract and this Agreement.
\item
  \textbf{Compensation and Benefits.}

  \begin{enumerate}
  \def\labelenumii{(\roman{enumii})}
  \tightlist
  \item
    During or in connection with any NBAGL Work Assignment or NBAGL Two-Way Service, a player (A) shall continue to receive the Compensation called for by his Uniform Player Contract, and (B) shall not receive (or accept) any compensation of any kind from the NBAGL or any NBAGL team other than as expressly set forth in this Article XLI. The player's performance in the NBAGL shall not be considered for purposes of any Incentive Compensation contained in his Uniform Player Contract.
  \item
    Any Compensation protection provided to a player in his Uniform Player Contract shall remain in effect during an NBAGL Work Assignment or NBAGL Two-Way Service. For purposes of Article II, Section 4, an injury sustained while participating in a basketball practice or game for an NBAGL team shall be deemed an injury sustained while participating in a basketball practice or game for the NBA Team.
  \item
    During or in connection with any NBAGL Work Assignment, a player (A) shall continue to be eligible to receive the benefits set forth in Article IV of this Agreement to the extent that such player would have been eligible to receive such benefits under this Agreement absent the NBAGL Work Assignment, and (B) shall not be eligible to receive (and shall not accept) any benefits from the NBAGL or any NBAGL team, unless expressly set forth in this Article XLI.
  \item
    To the extent necessary, any plans and/or policies described in Article IV of this Agreement shall be amended to implement the provisions of Section 4(b)(iii) of this Article.
  \end{enumerate}
\item
  \textbf{Meal Expense.} While on the road with his NBAGL team:

  \begin{enumerate}
  \def\labelenumii{(\roman{enumii})}
  \tightlist
  \item
    A player on an NBAGL Work Assignment, (i) shall receive the meal expense allowance applicable to NBA players, in accordance with the terms of Article III, Section 2 of this Agreement, and (ii) shall not receive (or accept) any meal expense or per diem from the NBAGL or any NBAGL team.
  \item
    A Two-Way Player shall receive the meal expense allowance
    applicable to NBAGL players.
  \end{enumerate}
\item
  \textbf{Travel Accommodations.} During an NBAGL Work Assignment or NBAGL Two-Way Service, the player shall be provided with the same travel accommodations (including, but not limited to, transportation and hotel arrangements for ``road'' games) that are provided to NBAGL players pursuant to applicable NBAGL policies, except that: (i) a player on an NBAGL Work Assignment shall not be required to share a hotel room; and (ii) a player on an NBAGL Work Assignment shall be permitted to fly first class when traveling by air with his NBAGL team to road games to the extent first class seats are available on his NBAGL team's flight.
\item
  \textbf{Conduct and Discipline.}

  \begin{enumerate}
  \def\labelenumii{(\roman{enumii})}
  \tightlist
  \item
    During any NBAGL Work Assignment or NBAGL Two-Way Service, the player will: (A) observe and comply with all rules and policies of the NBAGL or his NBAGL team at all times, whether on or off the playing floor; (B) give his best services, as well as his loyalty, to the NBAGL team; (C) be neatly and fully attired in public; (D) conduct himself on and off the court according to the highest standards of honesty, citizenship, and sportsmanship; and (E) not do anything that, in the opinion of the Commissioner of the NBA, is materially detrimental or materially prejudicial to the best interests of the NBA Team, the NBA, the NBAGL, or the NBAGL team.
  \item
    During or in connection with any NBAGL Work Assignment or NBAGL Two-Way Service, the NBAGL, the player's NBAGL team, the NBA, and the player's NBA Team may impose a fine and/or suspension on the player for the violation of NBAGL or NBAGL team rules or policies or for any conduct impairing the faithful and thorough discharge of the duties incumbent upon the player. Any disciplinary action taken by the NBA or an NBA Team in response to any act or conduct of a player during an NBAGL Work Assignment or NBAGL Two-Way Service will supersede disciplinary action taken by the NBAGL or any NBAGL team in response to such act or conduct. Further, with respect to discipline imposed by the NBA and/or the NBA Team, the One Penalty rule set forth in Article VI, Section 10 of this Agreement shall apply. The amount of any such fine and/or suspension that may be imposed by the NBA or an NBA Team shall be governed by the terms of this Agreement and the Uniform Player Contract and shall not be limited by any NBAGL rules, policies, practices, procedures, or fine schedules.
  \item
    All players on NBAGL Work Assignments and all Two-Way Players providing services to an NBAGL team shall be subject to the Joint NBA/NBPA Policy on Domestic Violence, Sexual Assault, and Child Abuse set forth as Exhibit F to this Agreement. Any evaluation, counseling, treatment, and/or discipline of such players for engaging in acts covered by this Policy shall be governed exclusively by the terms of the Policy. In the event any such player engages in other off-court conduct that is prohibited by both NBA and NBAGL rules, NBA rules shall apply.
  \item
    When a player on an NBAGL Work Assignment is suspended by his Team, the NBAGL team to which he has been assigned, the NBA, or the NBAGL, such player's Base Compensation for the Season of the Contract during which such suspension occurs shall be reduced in accordance with Article VI, Section 1 of this Agreement for each game missed as a result of such suspension, regardless of whether such suspension is expressed as a number of NBA games or as a number of NBAGL games. For clarity, for purposes of the foregoing sentence, during the term of any suspension, a player shall be considered to have missed either NBA games or NBAGL games, but not both. The player must remain on the NBA Inactive List during the term of the suspension but may be recalled at the option of the Team; provided, however, that the player may not play in any NBA or NBAGL games during the term of the suspension.
  \item
    When a Two-Way Player is suspended by his Team, his NBAGL team, the NBA, or the NBAGL, such player's Base Compensation for the Season of the Contract during which such suspension occurs shall be reduced in accordance with Article VI, Section 1 of this Agreement for each game missed as a result of such suspension, regardless of whether such suspension is expressed as a number of NBA games or as a number of NBAGL games. For clarity, for purposes of the foregoing sentence, during the term of any suspension, a player shall be considered to have missed either NBA games or NBAGL games, but not both. During the term of any suspension, the player (i) may not play in any NBA or NBAGL games, and (ii) may be maintained on the Team's Active List, Inactive List, or Two-Way List, provided that if the Two-Way Player was on the Active List when the actions that led to the suspension occurred and the player was suspended by the NBA then he must be maintained on the Team's Active List during the full term of the suspension, except if the suspension is for more than five (5) games, in which case the player must be transferred to the Team's Two-Way List following the fifth game of the suspension.
  \item
    A fine or suspension imposed by the NBAGL or NBAGL team in connection with a player's NBAGL Work Assignment or NBAGL Two-Way Service may be heard and resolved by the Grievance Arbitrator pursuant to Article XXXI of this Agreement only if it results in a financial impact to the player of more than \$5,000. For purposes of Paragraph 16(a)(ii) of a player's Uniform Player Contract, during or in connection with any NBAGL Work Assignment or NBAGL Two-Way Service, (A) the terms ``any official or employee of the Team or the NBA (other than another player)'' will be construed to include, without limitation, any official or employee of the NBAGL or the player's NBAGL team (other than another player), and (B) the terms ``any NBA game or event'' will be construed to include, without limitation, any NBAGL game or event.
  \end{enumerate}
\item
  \textbf{Medical Treatment and Physical Condition.}

  \begin{enumerate}
  \def\labelenumii{(\roman{enumii})}
  \tightlist
  \item
    The NBAGL and/or NBAGL team may make public medical information about a player on an NBAGL Work Assignment or NBAGL Two-Way Service to the same extent as an NBA Team would be able to, pursuant to Article XXII, Section 4.
  \item
    For purposes of Paragraphs 7, 16(a)(iii), 16(b), and 16(c) of the player's Uniform Player Contract, the terms ``basketball practice or game played for the Team'' or ``playing for the Team'' will be construed to include, without limitation, any practice or game played in the NBAGL during an NBAGL Work Assignment or NBAGL Two-Way Service.
  \end{enumerate}
\item
  \textbf{Prohibited Substances.} During any NBAGL Work Assignment or NBAGL Two-Way Service, the player (i) shall be subject to Article XXXIII (Anti-Drug Program) of this Agreement and Paragraph 8 of the Uniform Player Contract, and (ii) shall not be subject to any anti-drug program maintained by the NBAGL.
\item
  \textbf{Player Attributes and Performances.} Notwithstanding anything to the contrary in this Agreement or the Uniform Player Contract, with respect to any player who serves or has served on an NBAGL Work Assignment or provides or has provided NBAGL Two-Way Service:

  \begin{enumerate}
  \def\labelenumii{(\roman{enumii})}
  \tightlist
  \item
    The NBA and its related entities (including, without limitation, NBA Teams), and the NBAGL and its related entities (including, without limitation, NBAGL teams), shall have the right to use, and to license others to use, such player's Player Attributes (as defined in Paragraph 14(c) of the Uniform Player Contract) in connection with any advertising, marketing, or collateral materials or marketing programs conducted by the NBAGL or any NBAGL team that is intended to promote (1) any game in which an NBAGL team participates or any NBAGL game telecast or broadcast (including NBAGL pre-season, exhibition, regular season, or playoff games), (2) the NBAGL, its teams, or its players, or (3) the sport of basketball.
  \item
    The NBA and its related entities (including, without limitation, NBA Teams), and the NBAGL and its related entities (including, without limitation, NBAGL teams), shall have the right to use, and to license others to use, any performance of such player in connection with any form of broadcast or telecast, including over-the-air television, cable television, pay television, direct broadcast satellite television, and any form of cassette, cartridge, disk system, or other means of distribution known or unknown.
  \end{enumerate}

  The foregoing does not confer any right or authority for the NBA and its related entities (including, without limitation, NBA Teams), and/or the NBAGL and its related entities (including, without limitation, NBAGL teams), to use or authorize others to use the Player's Player Attributes in a manner that constitutes an unauthorized Endorsement or an Unauthorized Sponsor Promotion (as such terms are defined and clarified in Article XXVIII of this Agreement and Paragraph 14 of the Uniform Player Contract). For purposes of clarity and without limitation, any use of a player's Player Attributes that has been expressly authorized by the player (not including the Uniform Player Contract) shall not be an unauthorized Endorsement or an Unauthorized Sponsor Promotion. For the purposes of this Section 4(h), references to the NBA and NBA Teams in Article XXVIII, Section 3 (Unauthorized Endorsement/Sponsor Promotion) shall apply to the NBAGL and NBAGL teams.
\item
  \textbf{Promotional Activities.} In connection with a player's NBAGL Work Assignment or NBAGL Two-Way Service, the rights accorded to the NBA and his NBA Team under Paragraph 13(a) of the Uniform Player Contract shall extend, without limitation, to the NBAGL and his NBAGL team, and any promotional appearances such player is required to make during an NBAGL Work Assignment or NBAGL Two-Way Service shall count against the appearances the player is obligated to provide to the NBA and his NBA Team under Article II, Section 8; provided, however, that such player will be required to provide two (2) additional promotional appearances while on NBAGL Work Assignment or NBAGL Two-Way Service each Season to the NBAGL or his NBAGL team.
\end{enumerate}

\hypertarget{miscellaneous.-2}{%
\section{Miscellaneous.}\label{miscellaneous.-2}}

\begin{enumerate}
\def\labelenumi{(\alph{enumi})}
\tightlist
\item
  With respect to the duties and obligations of players under Paragraph 5 of the Uniform Player Contract (relating to Article 35 of the NBA Constitution) during or in connection with any NBAGL Work Assignment or NBAGL Two-Way Service:

  \begin{enumerate}
  \def\labelenumii{(\roman{enumii})}
  \tightlist
  \item
    the terms ``game'' or ``games'' in Article 35(b) and (c) of the Constitution will be construed to include, without limitation, any game played by an NBAGL team;
  \item
    the term ``basketball'' or ``game of basketball'' in Article 35(c) and (d) of the Constitution will be construed to include, without limitation, the NBAGL or any of its teams;
  \item
    the prohibition concerning wagering in Article 35(f) of the Constitution will extend, without limitation, to any game played by an NBAGL team; and
  \item
    the Commissioner's authority to act pursuant to Paragraph 5(e) of the Uniform Player Contract will extend, without limitation, to any game played by an NBAGL team.
  \end{enumerate}
\item
  A player shall not directly or indirectly own or hold any interest in the NBAGL or any NBAGL team unless authorized by the NBA.
\item
  At the conclusion of each Season covered by this Agreement, the NBA and the Players Association shall meet to discuss issues concerning the operation of this Article XLI.
\end{enumerate}

\hypertarget{career-opportunities-for-former-nba-players.}{%
\section{Career Opportunities for Former NBA Players.}\label{career-opportunities-for-former-nba-players.}}

\begin{enumerate}
\def\labelenumi{(\alph{enumi})}
\tightlist
\item
  The NBA and/or NBAGL will operate an apprenticeship program in the NBA/NBAGL League Office and/or on NBAGL team coaching staffs to provide business and/or basketball operations immersion training for former NBA players. Each session will last for approximately three (3) months and include basketball operations, community relations, sales and marketing, and/or team coaching rotations. There will be two (2) sessions (or one session lasting at least six (6) months) held annually, and each session will include up to two (2) former NBA players (based on player interest and, with respect to NBAGL team coaching apprenticeships, availability of NBAGL teams willing to participate). Participating former players in the League Office program will receive a monthly stipend to be agreed upon by the NBA and the Players Association. Participating former players in the NBAGL team coaching staff program will receive a monthly stipend to be agreed upon by the NBA and the Players Association, and housing or a housing subsidy in accordance with the NBAGL housing policy.
\item
  The NBA and/or NBAGL will operate an NBAGL coaching program to complement the existing NBA program and provide coaching training and experience for former NBA players. Up to fourteen (14) total coaching spots will be made available each year at the NBAGL Elite Mini Camp and other key basketball operations events such as the Portsmouth Invitational Tournament and the NBA Draft Combine. Participating former players will receive reimbursement for all reasonable expenses associated with participating in the coaching program.
\item
  The following programs will be created and/or maintained (as applicable) for former NBA players to have access to information about job opportunities in the NBAGL:

  \begin{enumerate}
  \def\labelenumii{(\roman{enumii})}
  \tightlist
  \item
    A database of NBAGL team job openings, along with a digital platform that gives NBA and NBAGL decision-makers access to information on prospective candidates, to be made available to former NBA players who have expressed interest in such positions.
  \item
    NBAGL teams will attend an annual job fair and/or career networking event held in connection with an NBA or NBAGL event (e.g., Draft Combine/NBAGL Elite Mini Camp or NBAGL Showcase) to facilitate discussions between NBAGL team executives and former NBA players. The NBA will use reasonable efforts to ensure that a representative from each NBAGL team attends each job fair.
  \end{enumerate}
\end{enumerate}

\hypertarget{other}{%
\chapter{OTHER}\label{other}}

\hypertarget{headings-and-organization.}{%
\section{Headings and Organization.}\label{headings-and-organization.}}

The headings and organization of this Agreement are solely for the convenience of the parties, and shall not be deemed part of, or considered in construing or interpreting, this Agreement.

\hypertarget{time-periods.}{%
\section{Time Periods.}\label{time-periods.}}

Unless specifically stated otherwise, the specification of any time period in this Agreement shall include any non-business days within such period, except that any deadline falling on a Saturday, Sunday, or Federal Holiday shall be deemed to fall on the following business day.

\hypertarget{exhibits.}{%
\section{Exhibits.}\label{exhibits.}}

All of the Exhibits hereto are an integral part of this Agreement and of the agreement of the parties thereto.

NATIONAL BASKETBALL ASSOCIATION

By:\\
\_\_\_\_\_\_\_\_\_\_\_\_\_\_\_\_\_\_\_\_\_\_\_\_\_\_\_\_\_\\
Adam Silver\\
Commissioner

NATIONAL BASKETBALL PLAYERS ASSOCIATION

By:\\
\_\_\_\_\_\_\_\_\_\_\_\_\_\_\_\_\_\_\_\_\_\_\_\_\_\_\_\_\_\\
Tamika Tremaglio
Executive Director

\hypertarget{appendix-appendix}{%
\appendix}


\hypertarget{national-basketball-association-uniform-player-contract}{%
\chapter{NATIONAL BASKETBALL ASSOCIATION UNIFORM PLAYER CONTRACT}\label{national-basketball-association-uniform-player-contract}}

THIS AGREEMENT made this \_\_\_\_\_ day of\_\_\_\_\_\_\_\_\_\_\_\_\_\_\_\_\_\_\_, is by and between \_\_\_\_\_\_\_\_\_\_\_\_\_\_\_\_\_\_\_\_\_\_\_\_ (hereinafter called the ``Team''), a member of the National Basketball Association (hereinafter called the ``NBA'' or ``League'') and \_\_\_\_\_\_\_\_\_\_\_\_\_\_\_\_\_\_\_, an individual whose address is shown below (hereinafter called the ``Player''). In consideration of the mutual promises hereinafter contained, the parties hereto promise and agree as follows:

\begin{enumerate}
\def\labelenumi{\arabic{enumi}.}
\tightlist
\item
  \textbf{TERM.}
\end{enumerate}

The Team hereby employs the Player as a skilled basketball player for a term of \_\_\_\_ year(s) from the 1st day of September \_\_\_\_.

\begin{enumerate}
\def\labelenumi{\arabic{enumi}.}
\setcounter{enumi}{1}
\tightlist
\item
  \textbf{SERVICES.}
\end{enumerate}

The services to be rendered by the Player pursuant to this Contract shall include: (a) training camp, (b) practices, meetings, workouts, and skill or conditioning sessions conducted by the Team during the Season, (c) games scheduled for the Team during any Regular Season, (d) Exhibition games scheduled by the Team or the League during and prior to any Regular Season, (e) if the Player is invited to participate, the NBA's All-Star Game (including the Rookie-Sophomore Game) and every event conducted in association with such All-Star Game, but only in accordance with Article XXI of the Collective Bargaining Agreement currently in effect between the NBA and the National Basketball Players Association (hereinafter the ``CBA''), (f) Play-In and playoff games scheduled by the League subsequent to any Regular Season, (g) promotional and commercial activities of the Team and the League as set forth in this Contract and the CBA, (h) any NBAGL Work Assignment in accordance with Article XLI of the CBA, and (i) any service in the NBAGL pursuant to a Two-Way Contract.

\begin{enumerate}
\def\labelenumi{\arabic{enumi}.}
\setcounter{enumi}{2}
\tightlist
\item
  \textbf{COMPENSATION.}
\end{enumerate}

\begin{enumerate}
\def\labelenumi{(\alph{enumi})}
\tightlist
\item
  Subject to Paragraph 3(b) below, the Team agrees to pay the Player for rendering the services and performing the obligations described herein the Compensation described in Exhibit 1, Exhibit 1A, Exhibit 1B, or Exhibit 10 hereto, as applicable (less all amounts required to be withheld by any governmental authority, and exclusive of any amount(s) which the Player shall be entitled to receive from the In-Season Tournament Prize Pool and the Player Playoff Pool). Unless otherwise provided in Exhibit 1 (or, with respect to advances, in Exhibit 1, Exhibit 1A, or Exhibit 1B), such Compensation shall be paid in twenty-four (24) equal semi-monthly payments beginning with the first of said payments on November 1st of each year covered by this Contract (``contract year'') and continuing with such payments on the first and fifteenth of each month until said Compensation is paid in full.
\item
  The Team agrees to pay the Player \$4,500 per week, pro rata, less all amounts required to be withheld by any governmental authority, for each week (up to a maximum of four (4) weeks for Veterans and up to a maximum of five (5) weeks for First-Year Players (as defined in Article XX, Section 1(c) of the CBA)) prior to the Team's first Regular Season game that the Player is in attendance at NBA training camp or Exhibition games; provided, however, that no such payments shall be made if, prior to the date on which he is required to attend training camp, the Player has been paid \$10,000 or more in Compensation with respect to the NBA Season scheduled to commence immediately following such training camp. Any Compensation paid by the Team pursuant to this subparagraph shall be considered an advance against any Compensation owed to the Player pursuant to Paragraph 3(a) above, and each of the first two (2) scheduled payments of such Compensation shall be reduced by fifty percent (50\%) of the amount of such advance.
\item
  The Team will not pay and the Player will not accept any bonus or anything of value on account of (i) the Team's winning any particular NBA game or series of games or attaining a certain position in the standings of the League as of a certain date, other than the final standing of the Team, or (ii) the player's or Team's performance in any particular NBA game or series of games; provided, however, that the preceding prohibition shall not apply to bonuses otherwise allowable under the CBA payable on account of the Team's qualifying for or winning a particular playoff series and/or the player's or Team's performance in a particular playoff series.
\end{enumerate}

\begin{enumerate}
\def\labelenumi{\arabic{enumi}.}
\setcounter{enumi}{3}
\tightlist
\item
  \textbf{EXPENSES.}
\end{enumerate}

The Team agrees to pay all proper and necessary expenses of the Player, including the reasonable lodging expenses of the Player while playing for the Team ``on the road'' and during the NBA training camp period (defined for this paragraph only to mean the period from the first day of training camp through the day of the Team's first Exhibition game) for as long as the Player is not then living at home. The Player, while ``on the road'' (and during the NBA training camp period, only if the Player is not then living at home and the Team does not pay for meals directly), shall be paid a meal expense allowance as set forth in the CBA. No deductions from such meal expense allowance shall be made for meals served on an airplane. During the NBA training camp period (and only if the Player is not then living at home and the Team does not pay for meals directly), the meal expense allowance shall be paid in weekly installments commencing with the first week of training camp. For the purposes of this paragraph, the Player shall be considered to be ``on the road'' from the time the Team leaves its home city until the time the Team arrives back at its home city.

\begin{enumerate}
\def\labelenumi{\arabic{enumi}.}
\setcounter{enumi}{4}
\tightlist
\item
  \textbf{CONDUCT.}
\end{enumerate}

\begin{enumerate}
\def\labelenumi{(\alph{enumi})}
\tightlist
\item
  The Player agrees to observe and comply with all Team rules, as maintained or promulgated in accordance with the CBA, at all times whether on or off the playing floor. Subject to the provisions of the CBA, such rules shall be part of this Contract as fully as if herein written and shall be binding upon the Player.
\item
  The Player agrees: (i) to give his best services, as well as his loyalty, to the Team, and to play basketball only for the Team and its assignees; (ii) to be neatly and fully attired in public; (iii) to conduct himself on and off the court according to the highest standards of honesty, citizenship, and sportsmanship; and (iv) not to do anything that is materially detrimental or materially prejudicial to the best interests of the Team or the League.
\item
  For any violation of Team rules, any breach of any provision of this Contract, or for any conduct impairing the faithful and thorough discharge of the duties incumbent upon the Player, the Team may reasonably impose fines and/or suspensions on the Player in accordance with the terms of the CBA.
\item
  The Player agrees to be bound by Article 35 of the NBA Constitution, a copy of which, as in effect on the date of this Contract, is attached hereto. The Player acknowledges that the Commissioner is empowered to impose fines upon and/or suspend the Player for causes and in the manner provided in such Article, provided that such fines and/or suspensions are consistent with the terms of the CBA.
\item
  The Player agrees that if the Commissioner, in his sole judgment, shall find that the Player has bet, or has offered or attempted to bet, money or anything of value on any game or event in the NBA or NBAGL, the Commissioner shall have the power in his sole discretion to suspend the Player indefinitely or to expel him as a player for any Team, and the Commissioner's finding and decision shall be final, binding, conclusive, and unappealable.
\item
  The Player agrees that he will not, during the term of this Contract, directly or indirectly, entice, induce, or persuade, or attempt to entice, induce, or persuade, any player or coach who is under contract to any NBA Team to enter into negotiations for or relating to his services as a basketball player or coach, nor shall he negotiate for or contract for such services, except with the prior written consent of such Team. Breach of this subparagraph, in addition to the remedies available to the Team, shall be punishable by fine and/or suspension to be imposed by the Commissioner.
\item
  When the Player is fined and/or suspended by the Team or the NBA, he shall be given notice in writing (with a copy to the Players Association), stating the amount of the fine or the duration of the suspension and the reasons therefor.
\end{enumerate}

\begin{enumerate}
\def\labelenumi{\arabic{enumi}.}
\setcounter{enumi}{5}
\tightlist
\item
  \textbf{WITHHOLDING.}
\end{enumerate}

\begin{enumerate}
\def\labelenumi{(\alph{enumi})}
\tightlist
\item
  In the event the Player (i) is fined and/or suspended by the Team or the NBA (or, as applicable, the NBAGL or an NBAGL team) or (ii) fails or refuses, without proper and reasonable cause or excuse, to render the services required by this Contract or the CBA, the Team shall withhold the amount of the fine or, in the case of a suspension or a failure or refusal to provide services, the amount provided in Article VI of the CBA (or, as applicable, Article XLI of the CBA) from any Current Base Compensation due or to become due to the Player with respect to the contract year in which the conduct resulting in the fine occurred, the suspension was served, and/or the failure or refusal to play occurred (or a subsequent contract year if the Player has received all Current Base Compensation due to him for the applicable contract year). If, at the applicable time for withholding in accordance with the preceding sentence, the Current Base Compensation remaining to be paid to the Player under this Contract is not sufficient to cover such withholding, then the Player agrees promptly to pay the applicable amount directly to the Team. In no case shall the Player permit any such amount to be paid on his behalf by anyone other than himself.
\item
  Any Current Base Compensation withheld from or paid by the Player pursuant to this Paragraph 6 shall be retained by the Team or the League, as the case may be, unless the Player contests the withholding (or the basis therefor) by initiating a timely Grievance in accordance with the provisions of the CBA. If such Grievance is initiated and it satisfies Article XXXI, Section 14 of the CBA, the amount withheld from the Player shall be placed in an interest-bearing account, pursuant to Article XXXI, Section 10 of the CBA, pending the resolution of the Grievance.
\end{enumerate}

\begin{enumerate}
\def\labelenumi{\arabic{enumi}.}
\setcounter{enumi}{6}
\tightlist
\item
  \textbf{HEALTH AND PHYSICAL CONDITION.}
\end{enumerate}

\begin{enumerate}
\def\labelenumi{(\alph{enumi})}
\item
  The Player agrees to report at the time and place fixed by the Team in good physical condition and to keep himself throughout each NBA Season in good physical condition.
\item
  If the Player, in the judgment of the Team's physician, is not in good physical condition at the date of his first scheduled game for the Team, or if, at the beginning of or during any Season, he fails to remain in good physical condition (unless such condition results directly from an injury sustained by the Player as a direct result of participating in any basketball practice or game played for the Team during such Season), so as to render the Player, in the judgment of the Team's physician, unfit to play skilled basketball, the Team shall have the right to suspend such Player until such time as, in the judgment of the Team's physician, the Player is in sufficiently good physical condition to play skilled basketball. In the event of such suspension, the Base Compensation payable to the Player for any Season during such suspension shall be reduced in the same proportion as the length of the period during which, in the judgment of the Team's physician, the Player is unfit to play skilled basketball, bears to the length of such Season. Nothing in this subparagraph shall authorize the Team to suspend the Player solely because the Player is injured or ill.
\item
  If, during the term of this Contract, the Player is injured as a direct result of participating in any basketball practice or game played for the Team, the Team will pay the Player's reasonable hospitalization and medical expenses (including doctor's bills), provided that the hospital and doctor are selected by the Team and that the Team shall be obligated to pay only those expenses incurred as a direct result of medical treatment caused solely by and relating directly to the injury sustained by the Player. The Team will also pay costs associated with a second opinion in accordance with Article XXII, Section 10 of the CBA. Subject to the provisions set forth in Exhibit 3, if in the judgment of the Team's physician, the Player's injuries resulted directly from playing for the Team and render him unfit to play skilled basketball, then, so long as such unfitness continues, but in no event after the Player has received his full Base Compensation for the Season in which the injury was sustained, the Team shall pay to the Player the Base Compensation prescribed in Exhibit 1, Exhibit 1A, or Exhibit 1B to this Contract, as applicable, for such Season. The Team's obligations hereunder shall be reduced by (x) any workers' compensation benefits, which, to the extent permitted by law, the Player hereby assigns to the Team, and (y) any insurance provided for by the Team whether paid or payable to the Player.
\item
  The Player agrees to provide to the Team's coach, trainer, or physician prompt notice of any injury, illness, or other medical condition (including, for clarity, any illness or other medical condition related to mental health) suffered by him that is likely to affect adversely the Player's ability to render the services required under this Contract, including the time, place, cause, and nature of such injury, illness, or other medical condition.
\item
  Should the Player suffer an injury, illness, or other medical condition, he will submit himself to a medical examination, appropriate medical treatment by a physician designated by the Team, and such rehabilitation activities as such physician may specify. Such examination when made at the request of the Team shall be at its expense, unless made necessary by some act or conduct of the Player contrary to the terms of this Contract.
\item
  The Player agrees (i) to submit to a physical examination at the commencement and conclusion of each contract year hereunder, and at such other times as reasonably determined by the Team to be medically necessary, and (ii) at the commencement of this Contract, and upon the request of the Team, to provide a complete prior medical history.
\item
  The Player agrees to supply complete and truthful information in connection with any medical examinations or requests for medical information authorized by this Contract.
\item
  \begin{enumerate}
  \def\labelenumii{(\roman{enumii})}
  \tightlist
  \item
    If the Player consults or is treated by a physician (including a psychiatrist) or a professional providing non-mental health related medical services (e.g., chiropractor, physical therapist) other than a physician or other professional designated by the Team, the Player shall give timely notice of such consultation or treatment to the Team and shall timely provide the Team with all information it may request concerning any condition that in the judgment of the Team's physician may affect the Player's ability to play skilled basketball.
  \item
    If Player engages in two (2) or more training, workout, or rehabilitation sessions with a trainer, performance coach, strength and conditioning coach, or any other similar coach or trainer other than at the direction of the Team (each a ``Third-Party Trainer'') during the Season, or five (5) or more such sessions with a Third-Party Trainer during the off-season, he shall give notice of such training, workout, or rehabilitation session to the Team prior to the first such training, workout, or rehabilitation session, provided that: (a) during the Season, if the Player does not initially plan to continue working with any Third-Party Trainer for two (2) or more sessions, such notice (and certification, if required pursuant to subsection (iii) below) must be provided no later than prior to the second such session; and (b) in the off-season, if the Player does not initially plan to continue working with any Third-Party Trainer for five (5) or more sessions, such notice (and certification, if required pursuant to subsection (iii) below) must be provided no later than prior to the fifth such session. This notice requirement shall not apply to workouts or training that exclusively involve jogging, road bicycling, swimming, yoga, Pilates, and/or dance; and the Player's failure to comply with such notice requirement shall not itself constitute a material breach of this Contract. For clarity with respect to counting multi-day training or workout sessions under this paragraph, any such session(s) shall be counted to equal the number of days on which such training or workouts occurred. Subject to the Team's other rights, and the Player's other obligations, under the CBA and this Contract, including, for example, the Player's obligations under this Paragraph 7 to report in good physical condition and to submit to treatment and rehabilitation specified by a physician designated by the Team, the Player will have the right in the off-season to work out with one or more Third-Party Trainers of his choosing and may not be disciplined for exercising that right.
  \item
    If the Player is receiving services from a Third-Party Trainer that are consistent with athletic training and/or strength and conditioning services, then in addition to providing notice as required pursuant to subsection (ii) above, the Player shall also certify to the Team (including by providing supporting documentation upon request) that such Third-Party Trainer meets the applicable standards for team athletic trainers and/or team strength and conditioning coaches set forth in Article XXII, Section 1 of the CBA, provided that this certification requirement in respect of a particular Third-Party Trainer shall not apply to the Player if he received services consistent with athletic training and/or strength and conditioning services from such particular Third-Party Trainer prior to the effective date of the CBA.
  \item
    If the Player fails to comply with the notice requirement set forth in subsection (ii) above and/or the certification requirement set forth in subsection (iii) above, or if the Team were to determine that a Third-Party Trainer did not meet the applicable standards, the Player may not, unless approved by the Team, engage in training, workout, or rehabilitation sessions with such Third-Party Trainer.
  \end{enumerate}
\item
  If and to the extent necessary to enable or facilitate the disclosure of medical information as provided for by this Contract or Article XXII or XXXIII of the CBA, the Player shall execute such individual authorization(s) as may be requested by the NBA, the Team, or the Medical Director or SPED Medical Director of the Anti-Drug Program, or as may be required by health care providers who examine or treat the Player.
\end{enumerate}

\begin{enumerate}
\def\labelenumi{\arabic{enumi}.}
\setcounter{enumi}{7}
\tightlist
\item
  \textbf{PROHIBITED SUBSTANCES/DOMESTIC VIOLENCE.}
\end{enumerate}

The Player acknowledges that this Contract may be terminated in accordance with the express provisions of (i) Article XXXIII (Anti-Drug Program) of the CBA or (ii) the Joint NBA/NBPA Policy on Domestic Violence, Sexual Assault, and Child Abuse, and that any such termination will result in the Player's immediate dismissal and disqualification from any employment by the NBA and any of its Teams. Notwithstanding any terms or provisions of this Contract (including any amendments hereto), in the event of such termination, all obligations of the Team, including obligations to pay Compensation, shall cease, except the obligation of the Team to pay the Player's earned Compensation (whether Current or Deferred) to the date of termination.

\begin{enumerate}
\def\labelenumi{\arabic{enumi}.}
\setcounter{enumi}{8}
\tightlist
\item
  \textbf{UNIQUE SKILLS.}
\end{enumerate}

The Player represents and agrees that he has extraordinary and unique skill and ability as a basketball player, that the services to be rendered by him hereunder cannot be replaced or the loss thereof adequately compensated for in money damages, and that any breach by the Player of this Contract will cause irreparable injury to the Team, and to its assignees. Therefore, it is agreed that in the event it is alleged by the Team that the Player is playing, attempting or threatening to play, or negotiating for the purpose of playing, during the term of this Contract, for any other person, firm, entity, or organization, the Team and its assignees (in addition to any other remedies that may be available to them judicially or by way of arbitration) shall have the right to obtain from any court or arbitrator having jurisdiction such equitable relief as may be appropriate, including a decree enjoining the Player from any further such breach of this Contract, and enjoining the Player from playing basketball for any other person, firm, entity, or organization during the term of this Contract. The Player agrees that this right may be enforced by the Team or the NBA. In any suit, action, or arbitration proceeding brought to obtain such equitable relief, the Player does hereby waive his right, if any, to trial by jury, and does hereby waive his right, if any, to interpose any counterclaim or set-off for any cause whatever.

\begin{enumerate}
\def\labelenumi{\arabic{enumi}.}
\setcounter{enumi}{9}
\tightlist
\item
  \textbf{ASSIGNMENT.}
\end{enumerate}

\begin{enumerate}
\def\labelenumi{(\alph{enumi})}
\tightlist
\item
  The Team shall have the right to assign this Contract to any other NBA Team, and the Player agrees to accept such assignment and to faithfully perform and carry out this Contract with the same force and effect as if it had been entered into by the Player with the assignee Team instead of with the Team.
\item
  In the event that this Contract is assigned to any other NBA Team, all reasonable expenses incurred by the Player in moving himself and his family to the home territory of the Team to which such assignment is made, as a result thereof, shall be paid by the assignee Team.
\item
  In the event that this Contract is assigned to another NBA Team, the Player (or his agent) shall forthwith be provided notice of such assignment by phone or email. With respect to an assignment by trade, notice of the trade must be provided to the Player (or his agent) by phone or email either before conclusion of the trade call with the NBA or as soon as possible after the conclusion of the trade call (but in no event may such notification be made more than one (1) hour after the conclusion of the trade call or less than one (1) hour prior to the public announcement of the assignment). The Player shall report to the assignee Team within forty-eight (48) hours after said notice has been received (if the assignment is made during a Season), within one (1) week after said notice has been received (if the assignment is made between Seasons), or within such longer time for reporting as may be specified in said notice. The NBA shall also notify the Players Association of any such assignment as soon as practicable but in no event later than one (1) business day after such assignment occurs. The Player further agrees that, immediately upon reporting to the assignee Team, he will submit upon request to a physical examination conducted by a physician designated by the assignee Team.
\item
  If the Player, without a reasonable excuse, does not report to the Team to which this Contract has been assigned within the time provided in subsection (c) above, then (i) upon consummation of the assignment, the Player may be disciplined by the assignee Team or, if the assignment is not consummated or is voided as a result of the Player's failure to so report, by the assignor Team, and (ii) such conduct shall constitute conduct prejudicial to the NBA under Article 35(d) of the NBA Constitution, and shall therefore subject the Player to discipline from the NBA in accordance with such Article.
\end{enumerate}

\begin{enumerate}
\def\labelenumi{\arabic{enumi}.}
\setcounter{enumi}{10}
\tightlist
\item
  \textbf{VALIDITY AND FILING.}
\end{enumerate}

\begin{enumerate}
\def\labelenumi{(\alph{enumi})}
\tightlist
\item
  This Contract shall be valid and binding upon the Team and the Player immediately upon its execution.
\item
  The Team agrees to file a copy of this Contract, and/or any amendment(s) thereto, with and as directed by the Commissioner of the NBA as soon as practicable by email, but in no event may such filing be made more than forty-eight (48) hours after the execution of this Contract and/or amendment(s).
\item
  If pursuant to the NBA Constitution and By-Laws or the CBA, the Commissioner disapproves this Contract (or any amendment(s) thereto) within ten (10) days from the first business day following the day on which this Contract (or amendment) is first received, as directed, in his office, this Contract (or amendment) shall thereupon terminate and be of no further force or effect and the Team and the Player shall thereupon be relieved of their respective rights and liabilities thereunder, provided that such ten (10) day period shall be fifteen (15) days for any Contract (or amendment) so received during the period each year from July 1 through the date that is fourteen (14) days following the last day of the Moratorium Period. If the Commissioner's disapproval is subsequently overturned in any proceeding brought under the arbitration provisions of the CBA (including any appeals), the Contract shall again be valid and binding upon the Team and the Player, and the Commissioner shall be afforded another ten (10) day period to disapprove the Contract (based on the Team's Room at the time the Commissioner's disapproval is overturned) as set forth in the foregoing sentence. The NBA will inform the Players Association if the Commissioner disapproves this Contract (or any amendment(s) thereto) no later than one (1) day following the date of such disapproval.
\end{enumerate}

\begin{enumerate}
\def\labelenumi{\arabic{enumi}.}
\setcounter{enumi}{11}
\tightlist
\item
  \textbf{PROHIBITED ACTIVITIES.}
\end{enumerate}

The Player and the Team acknowledge and agree that the Player's participation in certain other activities may impair or destroy his ability and skill as a basketball player, and the Player's participation in any game or exhibition of basketball other than at the request of the Team may result in injury to him. Accordingly, the Player agrees that he will not, without the written consent of the Team, engage in any activity that a reasonable person would recognize as involving or exposing the participant to a substantial risk of bodily injury including, but not limited to: (i) sky-diving, hang gliding, snow skiing, rock or mountain climbing (as distinguished from hiking), water or jet skiing, whitewater rafting, rappelling, bungee jumping, trampoline jumping, and mountain biking; (ii) any fighting, boxing, or wrestling; (iii) using fireworks or participating in any activity involving firearms or other weapons; (iv) riding on electric scooters or hoverboards; (v) driving or riding on a motorcycle or moped or four-wheeling/off-roading of any kind; (vi) riding in or on any motorized vehicle in any kind of race or racing contest; (vii) operating an aircraft of any kind; (viii) engaging in any other activity excluded or prohibited by or under any insurance policy which the Team procures against the injury, illness, or disability to or of the Player, or death of the Player, for which the Player has received written notice from the Team prior to the execution of this Contract; or (ix) participating in any game or exhibition of basketball, football, baseball, hockey, lacrosse, or other team sport or competition. If the Player violates this Paragraph 12, he shall be subject to discipline imposed by the Team and/or the Commissioner of the NBA. Nothing contained herein shall be intended to require the Player to obtain the written consent of the Team in order to enable the Player to participate in, as an amateur, the sports of golf, tennis, handball, swimming, hiking, softball, volleyball, and other similar sports that a reasonable person would not recognize as involving or exposing the participant to a substantial risk of bodily injury.

\begin{enumerate}
\def\labelenumi{\arabic{enumi}.}
\setcounter{enumi}{12}
\tightlist
\item
  \textbf{PROMOTIONAL ACTIVITIES.}
\end{enumerate}

\begin{enumerate}
\def\labelenumi{(\alph{enumi})}
\tightlist
\item
  The Player agrees to allow the Team, the NBA, or any League-related entity to take pictures of the Player, alone or together with others, for still photographs, motion pictures, television, or other Media (as such term is defined in Article XXVIII of the CBA), at such reasonable times as the Team, the NBA, or the League-related entity may designate. No matter by whom taken, such images may be used in any manner desired by either the Team, the NBA, or the League-related entity for publicity or promotional purposes for Teams or the NBA. The rights in any such images taken by the Team, the NBA, or the League-related entity shall belong to the Team, the NBA, or the League-related entity, as their interests may appear.
\item
  The Player agrees that, during any year of this Contract, he will not make public appearances, participate in radio or television programs, permit his picture to be taken, write or sponsor newspaper or magazine articles, or sponsor commercial products without the written consent of the Team, which shall not be withheld except in the reasonable interests of the Team or the NBA. The foregoing shall be interpreted in accordance with the decision in Portland Trail Blazers v. Darnell Valentine and Jim Paxson, Decision 86-2 (August 13, 1986).
\item
  Upon request, the Player shall consent to and make himself available for interviews by representatives of the media conducted at reasonable times and shall participate in an NBA Player Day as described in Article XXXVII, Section 1(b) of the CBA.
\item
  In addition to the foregoing, and subject to the conditions and limitations set forth in Article II, Section 8 of the CBA, the Player agrees to participate, upon request, in all other reasonable promotional activities of the Team, the NBA, and any League-related entity. For each such promotional appearance made on behalf of a commercial sponsor of the Team, the Team agrees to pay the Player \$3,500 subject to Article II, Section 8 of the CBA, or, if the Team agrees, such higher amount that is consistent with the Team's past practice and not otherwise unreasonable.
\item
  If, with respect to any Season, the Player elects to be a Content Participant and authorize the use of his Player-Produced Content, as those terms are defined in and in accordance with Article XXXVII, Section 2(b)(ii) of the CBA, the Player hereby grants the Player-Produced Content License (as defined in such Article XXXVII, Section 2(b)(ii) of the CBA) to each of the NBA, each League-related entity that generates BRI (as defined in Article VII of the CBA), and the Team.
\item
  The Player agrees to participate, upon request of the Players Association (or its wholly owned affiliate), in up to four (4) personal appearances each Salary Cap Year on behalf of the Players Association (or its wholly owned affiliate). The terms of this Paragraph 13(f) are in favor of the Players Association in order to facilitate the Players Association's marketing and licensing programs for the benefit of the NBA players.
\end{enumerate}

\begin{enumerate}
\def\labelenumi{\arabic{enumi}.}
\setcounter{enumi}{13}
\tightlist
\item
  \textbf{GROUP LICENSE AND LEAGUE PROMOTION.}
\end{enumerate}

\begin{enumerate}
\def\labelenumi{(\alph{enumi})}
\tightlist
\item
  For group licensing purposes only, the Player (i) exclusively grants to the Players Association the right to grant third parties the use of the Player's Player Attributes for Sponsorship Purposes and Product Licensing Purposes (as such terms are defined in the Group License Agreement); and (ii) authorizes the Players Association to grant non-exclusive group license rights for Sponsorship Purposes to the NBA Entities in accordance with the Group License Agreement, effective as of October 1, 2023, between NBA Properties, Inc.~and the Players Association (the ``Group License Agreement''), in each case during the term of the CBA or the Group License as described in the Group License Agreement, if longer; it being understood that the only parties that may grant group licensing rights during the term of the CBA shall be the Players Association (or its wholly-owned affiliate) and, for Sponsorship Purposes, the NBA Entities and the Players Association (or its wholly owned affiliate).
\item
  The Player acknowledges that (i) the Players Association or its wholly owned affiliate has granted group license rights for Sponsorship Purposes to the NBA Entities in accordance with the Group License Agreement during the term of the CBA or the Group License as described in the Group License Agreement, if longer, and (ii) the Player has not granted and will not grant during the term of the CBA rights for group licensing purposes that are in conflict with the Players Association's and NBA Entities' rights and/or Paragraph 14(a) above.
\item
  The NBA, all League-related entities, and the Teams may use, and may authorize others to use, in League Promotions, the Player's name, nickname, picture, portrait, likeness, signature, voice, caricature, biographical information, or other identifiable feature (collectively, ``Player Attributes''). The NBA, all League-related entities, and the Teams shall be entitled to use the Player's Player Attributes individually pursuant to the preceding sentence and may, but shall not be required to, use the Player's Player Attributes in a group or as one of multiple players. As used herein, ``League Promotion'' shall mean any and all uses intended to publicize, promote, or market (including in any and all Media) (i) the NBA, any League-related entity that generates BRI, any Team, or any Player, (ii) any game in which a Team participates (including an Exhibition, Regular Season, Play-In, and playoff game), including the sale of tickets to any such game, (iii) any telecast or other exhibition or distribution of (x) any such game or (y) any NBA-related or Team-related program or content, (iv) any NBA or Team facility, platform, or event, including the sale of tickets to any such event, or public service activity conducted by the NBA, a League-related entity that generates BRI, or a Team, or (v) the sport of basketball. For purposes of clarity, the foregoing rights of the NBA, League-related entities, and the Teams include the right and authority to use, and to authorize others to use, after the term of this Contract, any Player Attributes fixed in a tangible medium (e.g., filmed, photographed, recorded, or otherwise captured) during the term of this Contract solely for the purposes described herein.
\item
  Paragraph 14(c) above does not confer any right or authority to (i) use the Player's Player Attributes in a manner that constitutes an unauthorized Endorsement (as such term is defined and clarified in Article XXVIII of the CBA); (ii) use or authorize others to use the Player's Player Attributes (including in any program, content, platform, facility, or event) in a manner that constitutes an Unauthorized Sponsor Promotion (as such term is defined and clarified in Paragraph 14(e) below); or (iii) authorize others (including any NBA sponsor or Team sponsor) to use the Player's Player Attributes on any product, product packaging, service, or service-related materials sold or distributed by any third party, or any associated premiums.
\item
  An ``Unauthorized Sponsor Promotion'' shall mean a use of the Player's Player Attributes by a third party, or anyone on the third party's behalf (including, without limitation, the NBA, any League-related entity, or any NBA Team), to promote, market, or advertise the third party's product, service, or brand; provided, however, the term Unauthorized Sponsor Promotion does not include the use of the Player's Player Attributes (i) by, or on behalf of, a Telecaster (as defined in Paragraph 14(f) below) to promote the telecast or distribution of such NBA game, the fact that such third party is the telecaster or distributor of NBA content (e.g., an advertisement promoting MSG as the ``Home for New York Sports'' that includes a photograph of a Knicks player; or an ESPN advertisement promoting ESPN as the ``Worldwide Leader in Sports'' that includes footage of NBA players), or other sports-related programming of the Telecaster (but not related parties of the telecaster or distributor -- e.g., the Player's Player Attributes may be used to promote an e-commerce company's video service that carries games and may carry other sports content, but may not be used to promote other products or services of the e-commerce company), (ii) by, or on behalf of, a telecaster or distributor of NBA programs or content to promote such NBA programs or content, (iii) by a third party, or anyone on the third party's behalf, for use in the promotion of the sale of tickets to an NBA game or event, or (iv) by, or on behalf of, a third party to promote, market, or advertise the third party's product, service, or brand as part of a League Promotion or a promotional opportunity under Article XXVIII, Section 3(d)(y) of the CBA unless the execution (e.g., television advertisement, print ad, web ad) includes (x) more than (A) the third party's brand name and/or logo (either or both) (which use may not be persistent within such execution), provided that it shall not be considered persistent use of a third party's brand name and/or logo when used in conjunction with reference to the name and/or logo of the subject of such League Promotion for which the third party is a title or presenting sponsor (e.g., title sponsorship of the Slam Dunk Contest or a pre-game show) and (B) the subordinate and incidental promotion of the third party's products and services (e.g., not a call to action for a specific product or service), or (y) more than the subordinate and incidental promotion of the third party's products and services (it being understood that this clause (y) does not apply to an execution that includes the third party's brand name and/or logo, but clause (x) above does apply).
\item
  In addition to Paragraph 14(e)(i) above, it shall not be an Unauthorized Sponsor Promotion for a national, regional, local, or international telecaster or distributor of NBA games (such as ABC/ESPN, MSG, Bally Sports Ohio, or Tencent, Inc.) (each, a ``Telecaster'') to use Player Attributes to promote (i) itself and its sports programming or its other sports content and (ii) to the extent currently authorized by those contracts, its non-sports programming and content, in either case consistent with past practice as permitted under the 2017 CBA. The term Telecaster does not include related parties of the Telecaster. It shall be an Unauthorized Sponsor Promotion for the NBA, any League-related entity or any NBA Team to use the Player's Player Attributes as described in subparagraph (e), where such use (A) promotes the products, services, or brands of a third party that does not generate BRI, and (B) is not jointly licensed with the Players Association. Any dispute regarding whether a use of Player Attributes is or is not an Unauthorized Sponsor Promotion shall be determined by the System Arbitrator on an expedited basis, as soon as possible following a hearing conducted within seventy-two (72) hours after commencement of the proceeding.
\item
  The Player does not and will not contest during or after the term of this Contract, and the Player hereby acknowledges, the exclusive rights of the NBA, all League-related entities that generate BRI, and the Teams (i) to telecast, or otherwise distribute, transmit, exhibit, or perform, on a live, delayed, or archived basis, in any and all Media, any performance by the Player under this Contract or the CBA (including in NBA games or any excerpts thereof) and (ii) to produce, license, offer for sale, sell, market, or otherwise, exhibit, distribute, transmit, or perform (or authorize a third party to do any of the foregoing), on a live, delayed, or archived basis, any such performance in any and all Media, including, but not limited to, as part of programming or a content offering or in packaged or other electronic or digital media. The foregoing does not confer any right or authority to use the Player's Player Attributes in a manner that constitutes an unauthorized Endorsement or Unauthorized Sponsor Promotion (as such terms are defined and clarified in Article XXVIII of the CBA and Paragraph 14(e) above) or any right which would violate Article XXVIII, Section 3(f) of the CBA. For purposes of clarity and without limitation, any use of a Player's Player Attributes that has been expressly authorized by the Player (not including in this Contract) shall not be an unauthorized Endorsement or an Unauthorized Sponsor Promotion.
\end{enumerate}

\begin{enumerate}
\def\labelenumi{\arabic{enumi}.}
\setcounter{enumi}{14}
\tightlist
\item
  \textbf{TEAM DEFAULT.}
\end{enumerate}

In the event of an alleged default by the Team in the payments to the Player provided for by this Contract, or in the event of an alleged failure by the Team to perform any other material obligation that it has agreed to perform hereunder, the Player shall notify both the Team and the League in writing of the facts constituting such alleged default or alleged failure. If neither the Team nor the League shall cause such alleged default or alleged failure to be remedied within five (5) days after receipt of such written notice, the Players Association shall, on behalf of the Player, have the right to request that the dispute concerning such alleged default or alleged failure be referred immediately to the Grievance Arbitrator in accordance with the provisions of the CBA. If, as a result of such arbitration, an award issues in favor of the Player, and if neither the Team nor the League complies with such award within ten (10) days after the service thereof, the Player shall have the right, by a further written notice to the Team and the League, to terminate this Contract.

\begin{enumerate}
\def\labelenumi{\arabic{enumi}.}
\setcounter{enumi}{15}
\tightlist
\item
  \textbf{TERMINATION.}
\end{enumerate}

\begin{enumerate}
\def\labelenumi{(\alph{enumi})}
\tightlist
\item
  The Team may terminate this Contract upon written notice to the Player if the Player shall:

  \begin{enumerate}
  \def\labelenumii{(\roman{enumii})}
  \tightlist
  \item
    at any time, fail, refuse, or neglect to conform his personal conduct to standards of good citizenship, good moral character (defined here to mean not engaging in acts of moral turpitude, whether or not such acts would constitute a crime), and good sportsmanship, to keep himself in first class physical condition, or to obey the Team's training rules;
  \item
    at any time commit a significant and inexcusable physical attack against any official or employee of the Team or the NBA (other than another player), or any person in attendance at any NBA game or event, considering the totality of the circumstances, including (but not limited to) the degree of provocation (if any) that may have led to the attack, the nature and scope of the attack, the Player's state of mind at the time of the attack, and the extent of any injury resulting from the attack;
  \item
    at any time, fail, in the sole opinion of the Team's management, to exhibit sufficient skill or competitive ability to qualify to continue as a member of the Team; provided, however, (A) that if this Contract is terminated by the Team, in accordance with the provisions of this subparagraph, prior to January 10 of any Season, and the Player, at the time of such termination, is unfit to play skilled basketball as the result of an injury resulting directly from his playing for the Team, the Player shall (subject to the provisions set forth in Exhibit 3) continue to receive his full Base Compensation), less all workers' compensation benefits (which, to the extent permitted by law, and if not deducted from the Player's Compensation by the Team, the Player hereby assigns to the Team) and any insurance provided for by the Team paid or payable to the Player by reason of said injury, until such time as the Player is fit to play skilled basketball, but not beyond the Season during which such termination occurred; and provided, further, (B) that if this Contract is terminated by the Team, in accordance with the provisions of this subparagraph, during the period from the January 10 of any Season through the end of such Season, the Player shall be entitled to receive his full Base Compensation for said Season; or
  \item
    at any time, fail, refuse, or neglect to render his services hereunder or in any other manner materially breach this Contract.
  \end{enumerate}
\item
  If this Contract is terminated by the Team by reason of the Player's failure to render his services hereunder due to disability caused by an injury to the Player resulting directly from his playing for the Team and rendering him unfit to play skilled basketball, and notice of such injury is given by the Player as provided herein, the Player shall (subject to the provisions set forth in Exhibit 3) be entitled to receive his full Base Compensation for the Season in which the injury was sustained, less all workers' compensation benefits (which, to the extent permitted by law, and if not deducted from the Player's Compensation by the Team, the Player hereby assigns to the Team) and any insurance provided for by the Team paid or payable to the Player by reason of said injury.
\item
  Notwithstanding the provisions of Paragraph 16(b) above, if this Contract is terminated by the Team prior to the first game of a Regular Season by reason of the Player's failure to render his services hereunder due to an injury or condition sustained or suffered during a preceding Season, or after such Season but prior to the Player's participation in any basketball practice or game played for the Team, payment by the Team of any Compensation earned through the date of termination under Paragraph 3(b) above, payment of the Player's board, lodging, and expense allowance during the training camp period, payment of the reasonable traveling expenses of the Player to his home city, and the expert training and coaching provided by the Team to the Player during the training season shall be full payment to the Player.
\item
  If this Contract is terminated by the Team during the period designated by the Team for attendance at NBA training camp, payment by the Team of any Compensation earned through the date of termination under Paragraph 3(b) above, payment of the Player's board, lodging, and expense allowance during such period to the date of termination, payment of the reasonable traveling expenses of the Player to his home city, and the expert training and coaching provided by the Team to the Player during the training season shall be full payment to the Player.
\item
  If this Contract is terminated by the Team after the first game of a Regular Season, except in the case provided for in subparagraphs (a)(iii) and (b) of this paragraph 16, the Player shall be entitled to receive as full payment hereunder a sum of money which, when added to the salary which he has already received during such Season, will represent the same proportionate amount of the annual sum set forth in Exhibit 1, Exhibit 1A, or Exhibit 1B hereto, as applicable, as the number of days of such Regular Season then past bears to the total number of days of such Regular Season, plus the reasonable traveling expenses of the Player to his home.
\item
  If the Team proposes to terminate this Contract in accordance with subparagraph (a) of this Paragraph 16, it must first comply with the following waiver procedure:

  \begin{enumerate}
  \def\labelenumii{(\roman{enumii})}
  \tightlist
  \item
    The Team shall request the NBA Commissioner to request waivers from all other clubs. Such waiver request may not be withdrawn.
  \item
    Upon receipt of the waiver request, any other NBA Team may claim assignment of this Contract at such waiver price as may be fixed by the League, the priority of claims to be determined in accordance with the NBA Constitution and By-Laws.
  \item
    If this Contract is so claimed, the Team agrees that it shall, upon the assignment of this Contract to the claiming Team, notify the Player of such assignment as provided in Paragraph 10(c) hereof, and the Player agrees he shall report to the assignee Team as provided in said Paragraph 10(c).
  \item
    If the Contract is not claimed prior to the expiration of the waiver period, it shall terminate and the Team shall promptly deliver written notice of termination to the Player.
  \item
    The NBA shall promptly notify the Players Association of the disposition of any waiver request.
  \item
    To the extent not inconsistent with the foregoing provisions of this subparagraph (f), the waiver procedures set forth in the NBA Constitution and By-Laws, a copy of which, as in effect on the date of this Contract, is attached hereto, shall govern.
  \end{enumerate}
\item
  Upon any termination of this Contract by the Player, all obligations of the Team to pay Compensation shall cease on the date of termination, except the obligation of the Team to pay the Player's Compensation to said date.
\end{enumerate}

\begin{enumerate}
\def\labelenumi{\arabic{enumi}.}
\setcounter{enumi}{16}
\tightlist
\item
  \textbf{DISPUTES.}
\end{enumerate}

In the event of any dispute arising between the Player and the Team relating to any matter arising under this Contract, or concerning the performance or interpretation thereof (except for a dispute arising under Paragraph 9 hereof or as provided in Paragraph 14 above), such dispute shall be resolved in accordance with the Grievance and Arbitration Procedure set forth in Article XXXI of the CBA.

\begin{enumerate}
\def\labelenumi{\arabic{enumi}.}
\setcounter{enumi}{17}
\tightlist
\item
  \textbf{PLAYER NOT A MEMBER.}
\end{enumerate}

Nothing contained in this Contract or in any provision of the NBA Constitution and By-Laws shall be construed to constitute the Player a member of the NBA or to confer upon him any of the rights or privileges of a member thereof.

\begin{enumerate}
\def\labelenumi{\arabic{enumi}.}
\setcounter{enumi}{18}
\tightlist
\item
  \textbf{RELEASE.}
\end{enumerate}

The Player hereby releases and waives any and all claims he may have, or that may arise during the term of this Contract, against (a) the NBA and its related entities, the NBAGL and its related entities, and every member of the NBA or the NBAGL, and every director, officer, owner, stockholder, trustee, partner, and employee of the NBA, NBAGL, and their respective related entities and/or any member of the NBA or NBAGL and their related entities (excluding persons employed as players by any such member), and (b) any person retained by the NBA and/or the Players Association in connection with the NBA/NBPA Anti-Drug Program, the Grievance Arbitrator, the System Arbitrator, and any other arbitrator or expert retained by the NBA and/or the Players Association under the terms of the CBA, in both cases (a) and (b) above, arising out of, or in connection with, and whether or not by negligence, (i) any injury that is subject to the provisions of Paragraph 7 hereof, (ii) any fighting or other form of violent and/or unsportsmanlike conduct occurring during the course of any practice, any NBAGL game, and/or any NBA Exhibition, Regular Season, Play-In, and/or playoff game (in all cases on or adjacent to the playing floor or in or adjacent to any facility used for such practices or games), (iii) the testing procedures or the imposition of any penalties set forth in Paragraph 8 hereof and in the NBA/NBPA Anti-Drug Program, (iv) the provisions set forth in Paragraphs 13(f), 14(a), and 14(b) above, or (v) any injury suffered in the course of his employment as to which he has or would have a claim for workers' compensation benefits. The foregoing shall not apply to any claim of medical malpractice against a Team-affiliated physician or other medical personnel.

\begin{enumerate}
\def\labelenumi{\arabic{enumi}.}
\setcounter{enumi}{19}
\tightlist
\item
  \textbf{ENTIRE AGREEMENT.}
\end{enumerate}

This Contract (including any Exhibits hereto) contains the entire agreement between the parties and, except as provided in the CBA, sets forth all components of the Player's Compensation from the Team or any Team Affiliate, and there are no other agreements or transactions of any kind (whether disclosed or undisclosed to the NBA), express or implied, oral or written, or promises, undertakings, representations, commitments, inducements, assurances of intent, or understandings of any kind (whether disclosed or undisclosed to the NBA) (a) concerning any future Renegotiation, Extension, or other amendment of this Contract or the entry into any new Player Contract, or (b) involving compensation or consideration of any kind (including, without limitation, an investment or business opportunity) to be paid, furnished, or made available to the Player, or any person or entity controlled by, related to, or acting with authority on behalf of the Player, by the Team or any Team Affiliate.

\newpage

\textbf{\emph{EXAMINE THIS CONTRACT CAREFULLY BEFORE SIGNING IT.}}

THIS CONTRACT INCLUDES EXHIBITS \_\_\_\_\_\_\_\_, WHICH ARE ATTACHED HERETO AND MADE A PART HEREOF.

IN WITNESS WHEREOF the Player has hereunto signed his name and the Team has caused this Contract to be executed by its duly authorized officer.

\begin{longtable}[]{@{}ll@{}}
\toprule()
\endhead
Dated: \_\_\_\_\_\_\_\_\_\_\_\_\_\_\_\_\_\_\_\_\_ & By: \_\_\_\_\_\_\_\_\_\_\_\_\_\_\_\_\_\_\_\_\_\_\_\_\_\_\_\_ \\
& Title: \_\_\_\_\_\_\_\_\_\_\_\_\_\_\_\_\_\_\_\_\_\_\_\_\_\_\_\_ \\
& Team: \_\_\_\_\_\_\_\_\_\_\_\_\_\_\_\_\_\_\_\_\_\_\_\_\_\_\_\_ \\
& \\
Dated: \_\_\_\_\_\_\_\_\_\_\_\_\_\_\_\_\_\_\_\_\_ & By: \_\_\_\_\_\_\_\_\_\_\_\_\_\_\_\_\_\_\_\_\_\_\_\_\_\_\_\_ \\
& Player: \_\_\_\_\_\_\_\_\_\_\_\_\_\_\_\_\_\_\_\_\_\_\_\_\_\_\_\_ \\
& Player's Address: \\
& \_\_\_\_\_\_\_\_\_\_\_\_\_\_\_\_\_\_\_\_\_\_\_\_\_\_\_\_\_\_\_\_\_\_\_\_ \\
& \_\_\_\_\_\_\_\_\_\_\_\_\_\_\_\_\_\_\_\_\_\_\_\_\_\_\_\_\_\_\_\_\_\_\_\_ \\
\bottomrule()
\end{longtable}

\newpage

\hypertarget{excerpt-from-nba-constitution}{%
\subsection{EXCERPT FROM NBA CONSTITUTION}\label{excerpt-from-nba-constitution}}

\hypertarget{misconduct}{%
\subsubsection{MISCONDUCT}\label{misconduct}}

\begin{enumerate}
\def\labelenumi{\arabic{enumi}.}
\setcounter{enumi}{34}
\tightlist
\item
  The provisions of this Article 35 shall govern all Players in the Association, hereinafter referred to as ``Players.''

  \begin{enumerate}
  \def\labelenumii{(\alph{enumii})}
  \tightlist
  \item
    Each Member shall provide and require in every contract with any of its Players that they shall be bound and governed by the provisions of this Article. Each Member, at the direction of the Board of Governors or the Commissioner, as the case may be, shall take such action as the Board or the Commissioner may direct in order to effectuate the purposes of this Article.
  \item
    The Commissioner shall direct the dismissal and perpetual disqualification from any further association with the Association or any of its Members, of any Player found by the Commissioner after a hearing to have been guilty of offering, agreeing, conspiring, aiding, or attempting to cause any game of basketball to result otherwise than on its merits.
  \item
    If in the opinion of the Commissioner any act or conduct of a Player at or during an Exhibition, Regular Season, Play-In, or Playoff game has been prejudicial to or against the best interests of the Association or the game of basketball, the Commissioner shall impose upon such Player a fine not exceeding \$100,000, or may order for a time the suspension of any such Player from any connection or duties with Exhibition, Regular Season, Play-In, or Playoff games, or he may order both such fine and suspension.
  \item
    The Commissioner shall have the power to suspend for a definite or indefinite period, or to impose a fine not exceeding \$100,000, or inflict both such suspension and fine upon any Player who, in his opinion, (i) shall have made or caused to be made any statement having, or that was designed to have, an effect prejudicial or detrimental to the best interests of basketball or of the Association or of a Member, or (ii) shall have been guilty of conduct that does not conform to standards of morality or fair play, that does not comply at all times with all federal, state, and local laws, or that is prejudicial or detrimental to the Association.
  \item
    Any Player who, directly or indirectly, entices, induces, persuades or attempts to entice, induce, or persuade any Player, Coach, Trainer, General Manager, or any other person who is under contract to any other Member of the Association to enter into negotiations for or relating to his services or negotiates or contracts for such services shall, on being charged with such tampering, be given an opportunity to answer such charges after due notice and the Commissioner shall have the power to decide whether or not the charges have been sustained; in the event his decision is that the charges have been sustained, then the Commissioner shall have the power to suspend such Player for a definite or indefinite period, or to impose a fine not exceeding \$100,000, or inflict both such suspension and fine upon any such Player.
  \item
    Any Player who, directly or indirectly, wagers money or anything of value on any game or event in the Association or in the NBA G League shall, on being charged with such wagering, be given an opportunity to answer such charges after due notice, and the decision of the Commissioner shall be final, binding, and conclusive and unappealable. The penalty for such offense shall be within the absolute and sole discretion of the Commissioner and may include a fine, suspension, expulsion, and/or perpetual disqualification from further association with the Association or any of its Members.
  \item
    Except for a penalty imposed under Paragraph (f) of this Article 35: (i) any challenge by a Team to the decisions and acts of the Commissioner pursuant to Article 35 shall be appealable to the Board of Governors, who shall determine such appeals in accordance with such rules and regulations as may be adopted by the Board in its absolute and sole discretion, and (ii) any challenge by a Player to the decisions or acts of the Commissioner pursuant to Article 35 shall be governed by the provisions of Article XXXI of the NBA/NBPA Collective Bargaining Agreement then in effect.
  \end{enumerate}
\end{enumerate}

\hypertarget{excerpt-from-nba-by-laws}{%
\subsubsection{EXCERPT FROM NBA BY-LAWS}\label{excerpt-from-nba-by-laws}}

5.01. \emph{Waiver Right.} Except for sales and trading between Members in accordance with these By-Laws, no Member shall sell, option, or otherwise assign the contract with, right to the services of, or right to negotiate with, a Player without complying with the waiver procedure prescribed by this Constitution and By-Laws.

5.02. \emph{Waiver Price.} The waiver price shall be \$1,000 per Player.

5.03. \emph{Waiver Procedure.} A Member desiring to secure waivers on a Player shall notify the Commissioner or the Commissioner's designee, who shall, on behalf of such Member, immediately notify all other Members of the waiver request. Such Player shall be assumed to have been waived unless a Member shall notify the Commissioner or the Commissioner's designee in accordance with Section 5.04 of a claim to the rights to such Player. Once a Member has notified the Commissioner or the Commissioner's designee of its desire to secure waivers on a Player, such notice may not be withdrawn. A Player remains the financial responsibility of the Member placing him on waivers until the waiver period set by the Commissioner or the Commissioner's designee has expired.

5.04. \emph{Waiver Period.} If the Commissioner or the Commissioner's designee distributes notice of request for waiver, any Members wishing to claim rights to the Player shall do so by giving notice by telephone and in a Writing of such claim to the Commissioner or the Commissioner's designee within forty-eight (48) hours after the time of such notice. A Team may not withdraw a claim to the rights to a Player on waivers. Notwithstanding Article 40 of the NBA Constitution, Saturdays, Sundays, and legal holidays shall be included when computing the above-referenced waiver period.

5.05. \emph{Waiver Preferences.}
(a) In the event that more than one (1) Member shall have claimed the rights to a Player placed on waivers, the claiming Member with the lowest team standing at the time the waiver was requested shall be entitled to acquire the rights to such Player. If the request for waiver shall occur after the last day of the Season and before 11:59 p.m. eastern time on the following November 30, the standings at the close of the previous Season shall govern.
(b) If the winning percentage of two (2) claiming Teams are the same, then the tie shall be determined, if possible, on the basis of the Regular Season Games between the two (2) Teams during the Season or during the preceding Season, as the case may be. If still tied, a toss of a coin shall determine priority. For the purpose of determining standings, both Conferences of the Association shall be deemed merged and a consolidated standing shall control.

5.06. \emph{Players Acquired Through Waivers.} A Member who has acquired the rights and title to the contract of a Player through the waiver procedure may not sell or trade such rights for a period of thirty (30) days after the acquisition thereof; provided, however, that if the rights to such Player were acquired between Seasons, the 30-day period described herein shall begin on the first day of the next succeeding Season.

5.07. \emph{Additional Waiver Rules.} The Commissioner or the Board of Governors may from time to time adopt additional rules (supplementary to those set forth in this Section 5) with respect to the operation of the waiver procedure. Such rules shall not be inconsistent with the provisions of this Section 5 and shall apply to but shall not be limited to the mechanics of notice, inadvertent omission of notification to a Member, and rules of construction as to time.

\newpage

\hypertarget{agent-certification}{%
\subsection{AGENT CERTIFICATION}\label{agent-certification}}

(To be completed only if Player was represented by an agent who negotiated the terms of this Contract.)

I, the undersigned, having negotiated this Contract on behalf of \_\_\_\_\_\_\_\_\_\_\_\_\_\_\_, do hereby swear and certify, under penalties of perjury, that the terms of Paragraph 20 of this Contract (``Entire Agreement'') are true and correct to the best of my knowledge and belief.

\begin{longtable}[]{@{}r@{}}
\toprule()
\endhead
\_\_\_\_\_\_\_\_\_\_\_\_\_\_\_\_\_\_\_\_\_\_\_\_\_\_\_\_\_\_\_\_\_\_\_\_\_\_\_\_\_\_\_\_\_ \\
Player Reresentative \\
 \\
\_\_\_\_\_\_\_\_\_\_\_\_\_\_\_\_\_\_\_\_\_\_\_\_\_\_\_\_\_\_\_\_\_\_\_\_\_\_\_\_\_\_\_\_\_ \\
(Print or Type Name of Player Representative) \\
\bottomrule()
\end{longtable}

State of\\
County of

On \_\_\_\_\_\_\_\_\_\_\_\_\_\_\_\_\_\_\_\_\_\_\_, before me personally came \_\_\_\_\_\_\_\_\_\_\_\_\_\_\_\_\_\_\_\_\_\_\_\_\_\_\_ and acknowledged to me that he/she had executed the foregoing Agent Certification.

\begin{longtable}[]{@{}r@{}}
\toprule()
\endhead
\_\_\_\_\_\_\_\_\_\_\_\_\_\_\_\_\_\_\_\_\_\_\_\_\_\_\_\_\_\_\_\_\_\_\_\_\_\_\_\_\_\_\_\_\_ \\
Notary Public \\
\bottomrule()
\end{longtable}

\newpage

\hypertarget{uniform-player-contract-1}{%
\section{UNIFORM PLAYER CONTRACT}\label{uniform-player-contract-1}}

\hypertarget{exhibit-1-compensation}{%
\subsection{Exhibit 1 --- Compensation}\label{exhibit-1-compensation}}

Player: \_\_\_\_\_\_\_\_\_\_\_\_\_\_\_\_\_\_\_\_\_\_\_\_\_\_\\
Team: \_\_\_\_\_\_\_\_\_\_\_\_\_\_\_\_\_\_\_\_\_\_\_\_\_\_\_\_\\
Date: \_\_\_\_\_\_\_\_\_\_\_\_\_\_\_\_\_\_\_\_\_\_\_\_\_\_\_\_

\begin{longtable}[]{@{}
  >{\centering\arraybackslash}p{(\columnwidth - 4\tabcolsep) * \real{0.1071}}
  >{\centering\arraybackslash}p{(\columnwidth - 4\tabcolsep) * \real{0.4464}}
  >{\centering\arraybackslash}p{(\columnwidth - 4\tabcolsep) * \real{0.4464}}@{}}
\toprule()
\begin{minipage}[b]{\linewidth}\centering
Season
\end{minipage} & \begin{minipage}[b]{\linewidth}\centering
Current Base Compensation
\end{minipage} & \begin{minipage}[b]{\linewidth}\centering
Deferred Base Compensation
\end{minipage} \\
\midrule()
\endhead
\_\_\_\_\_\_\_\_ & \_\_\_\_\_\_\_\_\_\_\_\_\_\_\_\_\_\_\_\_\_\_\_ & \_\_\_\_\_\_\_\_\_\_\_\_\_\_\_\_\_\_\_\_\_\_\_\_\_ \\
\_\_\_\_\_\_\_\_ & \_\_\_\_\_\_\_\_\_\_\_\_\_\_\_\_\_\_\_\_\_\_\_ & \_\_\_\_\_\_\_\_\_\_\_\_\_\_\_\_\_\_\_\_\_\_\_\_\_ \\
\_\_\_\_\_\_\_\_ & \_\_\_\_\_\_\_\_\_\_\_\_\_\_\_\_\_\_\_\_\_\_\_ & \_\_\_\_\_\_\_\_\_\_\_\_\_\_\_\_\_\_\_\_\_\_\_\_\_ \\
\_\_\_\_\_\_\_\_ & \_\_\_\_\_\_\_\_\_\_\_\_\_\_\_\_\_\_\_\_\_\_\_ & \_\_\_\_\_\_\_\_\_\_\_\_\_\_\_\_\_\_\_\_\_\_\_\_\_ \\
\_\_\_\_\_\_\_\_ & \_\_\_\_\_\_\_\_\_\_\_\_\_\_\_\_\_\_\_\_\_\_\_ & \_\_\_\_\_\_\_\_\_\_\_\_\_\_\_\_\_\_\_\_\_\_\_\_\_ \\
\bottomrule()
\end{longtable}

\textbf{Payment Schedule} (if different from Paragraph 3):

Current Base:

Deferred Base:

\textbf{Signing Bonus} (include dates of payment):

\textbf{Incentive Compensation} (include dates of payment):

\textbf{Other Arrangements:}

\begin{longtable}[]{@{}ll@{}}
\toprule()
Initialed: & \\
\midrule()
\endhead
\_\_\_\_\_\_\_\_\_\_\_\_\_\_ & \_\_\_\_\_\_\_\_\_\_\_\_\_\_ \\
Player & Team \\
\bottomrule()
\end{longtable}

\newpage

\hypertarget{exhibit-1a-compensation-minimum-player-salary}{%
\subsubsection{Exhibit 1A --- Compensation: Minimum Player Salary}\label{exhibit-1a-compensation-minimum-player-salary}}

Player: \_\_\_\_\_\_\_\_\_\_\_\_\_\_\_\_\_\_\_\_\_\_\_\_\_\_\_\_\\
Team: \_\_\_\_\_\_\_\_\_\_\_\_\_\_\_\_\_\_\_\_\_\_\_\_\_\_\_\_\\
Date: \_\_\_\_\_\_\_\_\_\_\_\_\_\_\_\_\_\_\_\_\_\_\_\_\_\_\_\_

\begin{longtable}[]{@{}
  >{\centering\arraybackslash}p{(\columnwidth - 4\tabcolsep) * \real{0.1071}}
  >{\centering\arraybackslash}p{(\columnwidth - 4\tabcolsep) * \real{0.4464}}
  >{\centering\arraybackslash}p{(\columnwidth - 4\tabcolsep) * \real{0.4464}}@{}}
\toprule()
\begin{minipage}[b]{\linewidth}\centering
Season
\end{minipage} & \begin{minipage}[b]{\linewidth}\centering
Current Base Compensation
\end{minipage} & \begin{minipage}[b]{\linewidth}\centering
Deferred Base Compensation
\end{minipage} \\
\midrule()
\endhead
\_\_\_\_\_\_\_\_ & \_\_\_\_\_\_\_\_\_\_\_\_\_\_\_\_\_\_\_\_\_\_\_ & \_\_\_\_\_\_\_\_\_\_\_\_\_\_\_\_\_\_\_\_\_\_\_\_\_ \\
\_\_\_\_\_\_\_\_ & \_\_\_\_\_\_\_\_\_\_\_\_\_\_\_\_\_\_\_\_\_\_\_ & \_\_\_\_\_\_\_\_\_\_\_\_\_\_\_\_\_\_\_\_\_\_\_\_\_ \\
\_\_\_\_\_\_\_\_ & \_\_\_\_\_\_\_\_\_\_\_\_\_\_\_\_\_\_\_\_\_\_\_ & \_\_\_\_\_\_\_\_\_\_\_\_\_\_\_\_\_\_\_\_\_\_\_\_\_ \\
\_\_\_\_\_\_\_\_ & \_\_\_\_\_\_\_\_\_\_\_\_\_\_\_\_\_\_\_\_\_\_\_ & \_\_\_\_\_\_\_\_\_\_\_\_\_\_\_\_\_\_\_\_\_\_\_\_\_ \\
\_\_\_\_\_\_\_\_ & \_\_\_\_\_\_\_\_\_\_\_\_\_\_\_\_\_\_\_\_\_\_\_ & \_\_\_\_\_\_\_\_\_\_\_\_\_\_\_\_\_\_\_\_\_\_\_\_\_ \\
\bottomrule()
\end{longtable}

\textbf{This Contract is intended to provide for a Base Compensation for the \_\_\_\_\_\_\_\_\_\_\_\_\_\_ Season(s) equal to the Minimum Player Salary for such Season(s) (with no bonuses of any kind) and shall be deemed amended to the extent necessary to so provide.}

\textbf{Payment Schedule} (if different from Paragraph 3):

\textbf{Other Arrangements:}

\begin{longtable}[]{@{}ll@{}}
\toprule()
Initialed: & \\
\midrule()
\endhead
\_\_\_\_\_\_\_\_\_\_\_\_\_\_ & \_\_\_\_\_\_\_\_\_\_\_\_\_\_ \\
Player & Team \\
\bottomrule()
\end{longtable}

\newpage

\hypertarget{exhibit-1b-compensation-two-way-player-salary}{%
\subsubsection{Exhibit 1B --- Compensation: Two-Way Player Salary}\label{exhibit-1b-compensation-two-way-player-salary}}

Player: \_\_\_\_\_\_\_\_\_\_\_\_\_\_\_\_\_\_\_\_\_\_\_\_\_\_\\
Team: \_\_\_\_\_\_\_\_\_\_\_\_\_\_\_\_\_\_\_\_\_\_\_\_\_\_\_\_\\
Date: \_\_\_\_\_\_\_\_\_\_\_\_\_\_\_\_\_\_\_\_\_\_\_\_\_\_\_\_

\begin{longtable}[]{@{}cc@{}}
\toprule()
Season & Two-Way Player Salary \\
\midrule()
\endhead
\_\_\_\_\_\_\_\_\_\_\_\_\_ & \_\_\_\_\_\_\_\_\_\_\_\_\_\_\_\_\_\_\_\_\_\_\_\_\_\_\_\_ \\
\_\_\_\_\_\_\_\_\_\_\_\_\_ & \_\_\_\_\_\_\_\_\_\_\_\_\_\_\_\_\_\_\_\_\_\_\_\_\_\_\_\_ \\
\_\_\_\_\_\_\_\_\_\_\_\_\_ & \_\_\_\_\_\_\_\_\_\_\_\_\_\_\_\_\_\_\_\_\_\_\_\_\_\_\_\_ \\
\_\_\_\_\_\_\_\_\_\_\_\_\_ & \_\_\_\_\_\_\_\_\_\_\_\_\_\_\_\_\_\_\_\_\_\_\_\_\_\_\_\_ \\
\_\_\_\_\_\_\_\_\_\_\_\_\_ & \_\_\_\_\_\_\_\_\_\_\_\_\_\_\_\_\_\_\_\_\_\_\_\_\_\_\_\_ \\
\bottomrule()
\end{longtable}

\textbf{This Contract is intended to provide for a Base Compensation for the \_\_\_\_\_\_\_\_\_\_\_\_ Season(s) equal to the Two-Way Player Salary for such Season(s) (with no bonuses of any kind) and shall be deemed amended to the extent necessary to so provide.}

\textbf{Payment Schedule} (if different from Paragraph 3):

\textbf{Standard NBA Contract Conversion Option}: Team shall have the option to convert this Contract to a Standard NBA Contract (``Standard NBA Contract Conversion Option''). Team's Standard NBA Contract Conversion Option may be exercised by providing written notice to Player that is either personally delivered to Player or his representative or sent by email or pre-paid certified, registered, or overnight mail to the last known address of Player or his representative with a copy to the Players Association and the NBA. If Team exercises the Standard NBA Contract Conversion Option, the Base Compensation amount set forth above in this Exhibit 1B will immediately become null and void and of no further force or effect, Player's Compensation shall be equal to the Player's applicable Minimum Player Salary for a term equal to the remainder of the original term of this Contract beginning on the date such option is exercised, and all other terms and conditions of this Contract, including the Base Compensation protection set forth in Exhibit 2 (if any), shall remain applicable.

\begin{longtable}[]{@{}ll@{}}
\toprule()
Initialed: & \\
\midrule()
\endhead
\_\_\_\_\_\_\_\_\_\_\_\_\_\_ & \_\_\_\_\_\_\_\_\_\_\_\_\_\_ \\
Player & Team \\
\bottomrule()
\end{longtable}

\newpage

\hypertarget{exhibit-2-compensation-protection}{%
\subsection{Exhibit 2 --- Compensation Protection}\label{exhibit-2-compensation-protection}}

Player: \_\_\_\_\_\_\_\_\_\_\_\_\_\_\_\_\_\_\_\_\_\_\_\_\_\_\\
Team: \_\_\_\_\_\_\_\_\_\_\_\_\_\_\_\_\_\_\_\_\_\_\_\_\_\_\\
Date: \_\_\_\_\_\_\_\_\_\_\_\_\_\_\_\_\_\_\_\_\_\_\_\_\_\_\_

\begin{longtable}[]{@{}
  >{\centering\arraybackslash}p{(\columnwidth - 6\tabcolsep) * \real{0.0806}}
  >{\centering\arraybackslash}p{(\columnwidth - 6\tabcolsep) * \real{0.2419}}
  >{\centering\arraybackslash}p{(\columnwidth - 6\tabcolsep) * \real{0.2742}}
  >{\centering\arraybackslash}p{(\columnwidth - 6\tabcolsep) * \real{0.4032}}@{}}
\toprule()
\begin{minipage}[b]{\linewidth}\centering
Season
\end{minipage} & \begin{minipage}[b]{\linewidth}\centering
Type of Protection
\end{minipage} & \begin{minipage}[b]{\linewidth}\centering
Amount of Protection
\end{minipage} & \begin{minipage}[b]{\linewidth}\centering
Additional Conditions or Limitations
\end{minipage} \\
\midrule()
\endhead
\_\_\_\_\_ & \_\_\_\_\_\_\_\_\_\_\_ & \_\_\_\_\_\_\_\_\_\_\_\_\_\_\_ & \_\_\_\_\_\_\_\_\_\_\_\_\_\_\_\_\_\_ \\
\_\_\_\_\_ & \_\_\_\_\_\_\_\_\_\_\_ & \_\_\_\_\_\_\_\_\_\_\_\_\_\_\_ & \_\_\_\_\_\_\_\_\_\_\_\_\_\_\_\_\_\_ \\
\_\_\_\_\_ & \_\_\_\_\_\_\_\_\_\_\_ & \_\_\_\_\_\_\_\_\_\_\_\_\_\_\_ & \_\_\_\_\_\_\_\_\_\_\_\_\_\_\_\_\_\_ \\
\_\_\_\_\_ & \_\_\_\_\_\_\_\_\_\_\_ & \_\_\_\_\_\_\_\_\_\_\_\_\_\_\_ & \_\_\_\_\_\_\_\_\_\_\_\_\_\_\_\_\_\_ \\
\_\_\_\_\_ & \_\_\_\_\_\_\_\_\_\_\_ & \_\_\_\_\_\_\_\_\_\_\_\_\_\_\_ & \_\_\_\_\_\_\_\_\_\_\_\_\_\_\_\_\_\_ \\
\bottomrule()
\end{longtable}

\textbf{Automatic Stretch Provision:} In the event that the Team terminates this Contract (resulting in the Player's separation of service from the Team), and the Team is obligated thereafter to make payments to the Player pursuant to this Exhibit 2, such payments shall be made in accordance with the following schedule:

\begin{enumerate}
\def\labelenumi{(\arabic{enumi})}
\tightlist
\item
  If, as of the date of the Player's separation from service, the aggregate amount owed to the Player pursuant to this Exhibit 2 is five hundred thousand dollars (\$500,000) or less, such amount shall be paid in accordance with the semi-monthly installments prescribed by the payment schedule set forth in this Contract. Each installment shall equal the amount of Base Compensation that was due per pay period for the applicable Season immediately before the Player's separation until the aggregate amount of the remaining Base Compensation owed to the Player pursuant to this Exhibit 2 is paid in full.
\item
  If, as of the date of the Player's separation from service, the aggregate amount owed to the Player pursuant to this Exhibit 2 exceeds five hundred thousand dollars (\$500,000), such amount shall be paid as follows:

  \begin{enumerate}
  \def\labelenumii{(\roman{enumii})}
  \tightlist
  \item
    The Base Compensation, if any, owed to the Player pursuant to this Exhibit 2 with respect to the ``current season'' (as defined below) at the time when the request for waivers on the Player is made shall be paid in accordance with the payment schedule set forth in this Contract. Each installment shall equal the amount of Base Compensation that was due per pay period immediately before the Player's separation until the aggregate amount of the remaining Base Compensation owed to the Player pursuant to this Exhibit 2 with respect to the current season is paid in full. For purposes of this Paragraph 2 only, the ``current season'' means the period from September 1 through June 30.
  \item
    The remaining Base Compensation, if any, owed to the Player pursuant to this Exhibit 2 shall be aggregated and paid in equal amounts per year over a period equal to twice the number of NBA Seasons (including any Season covered by a Player Option Year) remaining on this Contract following the date upon which the request for waivers occurred, plus one NBA Season. For this purpose, if the request for waivers is made during the period from September 1 through June 30, the number of NBA Seasons remaining on this Contract shall not include the current season (as defined in subparagraph (i) above). The rescheduled payments described above shall be paid over the applicable number of NBA Seasons in equal semi-monthly installments on the pay dates prescribed by Paragraph 3(a) of this Contract.
  \end{enumerate}
\end{enumerate}

For purposes of Section 409A of the Internal Revenue Code, each installment of the amount payable pursuant to this Exhibit 2 shall be treated as a separate payment.

\textbf{Standard Conditions or Limitations:} The Player's Base Compensation protection for each Season hereunder shall not be applicable if the Player's lack of skill, death, injury or illness, and/or mental disability (as applicable) results from the Player's:

\begin{enumerate}
\def\labelenumi{(\arabic{enumi})}
\tightlist
\item
  participation in activities prohibited by Paragraph 12 of the Contract (as such Paragraph may be modified by Exhibit 5), which includes, among other things, engaging in any activity that a reasonable person would recognize as involving or exposing the participant to a substantial risk of bodily injury including, but not limited to (i) sky-diving, hang gliding, snow skiing, rock or mountain climbing (as distinguished from hiking), water or jet skiing, whitewater rafting, rappelling, bungee jumping, trampoline jumping and mountain biking; (ii) any fighting, boxing, or wrestling; (iii) using fireworks or participating in any activity involving firearms or other weapons; (iv) riding on electric scooters or hoverboards; (v) driving or riding on a motorcycle or moped or four-wheeling/off-roading of any kind; (vi) riding in or on any motorized vehicle in any kind of race or racing contest; (vii) operating an aircraft of any kind; (viii) engaging in any other activity excluded or prohibited by or under any insurance policy which the Team procures against the injury, illness, or disability to or of the Player, or death of the Player, for which the Player has received written notice from the Team prior to the execution of this Contract; or (ix) participating in any game or exhibition of basketball, football, baseball, hockey, lacrosse, or other team sport or competition;
\item
  intentional self-inflicted injury, attempted suicide, and/or suicide;
\item
  abuse of alcohol;
\item
  use of any Prohibited Substance or controlled substance;
\item
  abuse of or addiction to prescription drugs;
\item
  conduct occurring during a commission of any felony for which the player is convicted (including by a plea of guilty, no contest, or nolo contendere);
\item
  participation in any riot, insurrection, or war or other military activities; or
\item
  failure to comply with the requirements of Paragraphs 7(d)-(i) of this Contract.
\end{enumerate}

\textbf{Additional Conditions or Limitations:}

\begin{longtable}[]{@{}ll@{}}
\toprule()
Initialed: & \\
\midrule()
\endhead
\_\_\_\_\_\_\_\_\_\_\_\_\_\_ & \_\_\_\_\_\_\_\_\_\_\_\_\_\_ \\
Player & Team \\
\bottomrule()
\end{longtable}

\newpage

\hypertarget{exhibit-3-prior-injury-exclusion}{%
\subsection{Exhibit 3 --- Prior Injury Exclusion}\label{exhibit-3-prior-injury-exclusion}}

Player: \_\_\_\_\_\_\_\_\_\_\_\_\_\_\_\_\_\_\_\_\_\_\_\_\_\_\_\_\\
Team: \_\_\_\_\_\_\_\_\_\_\_\_\_\_\_\_\_\_\_\_\_\_\_\_\_\_\_\_\\
Date: \_\_\_\_\_\_\_\_\_\_\_\_\_\_\_\_\_\_\_\_\_\_\_\_\_\_\_\_

The Player's right to receive his Compensation as set forth in Paragraphs 7(c), 16(a)(iii), 16(b) of this Contract, or otherwise is limited or eliminated with respect to the following reinjury of the injury or aggravation of the condition set forth below:

\begin{longtable}[]{@{}l@{}}
\toprule()
Describe injury or condition: \\
\midrule()
\endhead
\_\_\_\_\_\_\_\_\_\_\_\_\_\_\_\_\_\_\_\_\_\_\_\_\_\_\_\_\_\_\_\_\_\_\_\_\_\_\_\_\_\_\_\_\_\_\_\_\_\_\_\_\_\_\_\_\_\_\_\_\_ \\
\_\_\_\_\_\_\_\_\_\_\_\_\_\_\_\_\_\_\_\_\_\_\_\_\_\_\_\_\_\_\_\_\_\_\_\_\_\_\_\_\_\_\_\_\_\_\_\_\_\_\_\_\_\_\_\_\_\_\_\_\_ \\
\_\_\_\_\_\_\_\_\_\_\_\_\_\_\_\_\_\_\_\_\_\_\_\_\_\_\_\_\_\_\_\_\_\_\_\_\_\_\_\_\_\_\_\_\_\_\_\_\_\_\_\_\_\_\_\_\_\_\_\_\_ \\
\_\_\_\_\_\_\_\_\_\_\_\_\_\_\_\_\_\_\_\_\_\_\_\_\_\_\_\_\_\_\_\_\_\_\_\_\_\_\_\_\_\_\_\_\_\_\_\_\_\_\_\_\_\_\_\_\_\_\_\_\_ \\
\_\_\_\_\_\_\_\_\_\_\_\_\_\_\_\_\_\_\_\_\_\_\_\_\_\_\_\_\_\_\_\_\_\_\_\_\_\_\_\_\_\_\_\_\_\_\_\_\_\_\_\_\_\_\_\_\_\_\_\_\_ \\
\_\_\_\_\_\_\_\_\_\_\_\_\_\_\_\_\_\_\_\_\_\_\_\_\_\_\_\_\_\_\_\_\_\_\_\_\_\_\_\_\_\_\_\_\_\_\_\_\_\_\_\_\_\_\_\_\_\_\_\_\_ \\
\_\_\_\_\_\_\_\_\_\_\_\_\_\_\_\_\_\_\_\_\_\_\_\_\_\_\_\_\_\_\_\_\_\_\_\_\_\_\_\_\_\_\_\_\_\_\_\_\_\_\_\_\_\_\_\_\_\_\_\_\_ \\
\_\_\_\_\_\_\_\_\_\_\_\_\_\_\_\_\_\_\_\_\_\_\_\_\_\_\_\_\_\_\_\_\_\_\_\_\_\_\_\_\_\_\_\_\_\_\_\_\_\_\_\_\_\_\_\_\_\_\_\_\_ \\
\_\_\_\_\_\_\_\_\_\_\_\_\_\_\_\_\_\_\_\_\_\_\_\_\_\_\_\_\_\_\_\_\_\_\_\_\_\_\_\_\_\_\_\_\_\_\_\_\_\_\_\_\_\_\_\_\_\_\_\_\_ \\
\_\_\_\_\_\_\_\_\_\_\_\_\_\_\_\_\_\_\_\_\_\_\_\_\_\_\_\_\_\_\_\_\_\_\_\_\_\_\_\_\_\_\_\_\_\_\_\_\_\_\_\_\_\_\_\_\_\_\_\_\_ \\
\_\_\_\_\_\_\_\_\_\_\_\_\_\_\_\_\_\_\_\_\_\_\_\_\_\_\_\_\_\_\_\_\_\_\_\_\_\_\_\_\_\_\_\_\_\_\_\_\_\_\_\_\_\_\_\_\_\_\_\_\_ \\
\bottomrule()
\end{longtable}

\begin{longtable}[]{@{}
  >{\raggedright\arraybackslash}p{(\columnwidth - 0\tabcolsep) * \real{1.0000}}@{}}
\toprule()
\begin{minipage}[b]{\linewidth}\raggedright
Describe the extent to which liability for Compensation is limited or eliminated:
\end{minipage} \\
\midrule()
\endhead
\_\_\_\_\_\_\_\_\_\_\_\_\_\_\_\_\_\_\_\_\_\_\_\_\_\_\_\_\_\_\_\_\_\_\_\_\_\_\_\_\_\_\_\_\_\_\_\_\_\_\_\_\_\_\_\_\_\_\_\_\_ \\
\_\_\_\_\_\_\_\_\_\_\_\_\_\_\_\_\_\_\_\_\_\_\_\_\_\_\_\_\_\_\_\_\_\_\_\_\_\_\_\_\_\_\_\_\_\_\_\_\_\_\_\_\_\_\_\_\_\_\_\_\_ \\
\_\_\_\_\_\_\_\_\_\_\_\_\_\_\_\_\_\_\_\_\_\_\_\_\_\_\_\_\_\_\_\_\_\_\_\_\_\_\_\_\_\_\_\_\_\_\_\_\_\_\_\_\_\_\_\_\_\_\_\_\_ \\
\_\_\_\_\_\_\_\_\_\_\_\_\_\_\_\_\_\_\_\_\_\_\_\_\_\_\_\_\_\_\_\_\_\_\_\_\_\_\_\_\_\_\_\_\_\_\_\_\_\_\_\_\_\_\_\_\_\_\_\_\_ \\
\_\_\_\_\_\_\_\_\_\_\_\_\_\_\_\_\_\_\_\_\_\_\_\_\_\_\_\_\_\_\_\_\_\_\_\_\_\_\_\_\_\_\_\_\_\_\_\_\_\_\_\_\_\_\_\_\_\_\_\_\_ \\
\_\_\_\_\_\_\_\_\_\_\_\_\_\_\_\_\_\_\_\_\_\_\_\_\_\_\_\_\_\_\_\_\_\_\_\_\_\_\_\_\_\_\_\_\_\_\_\_\_\_\_\_\_\_\_\_\_\_\_\_\_ \\
\_\_\_\_\_\_\_\_\_\_\_\_\_\_\_\_\_\_\_\_\_\_\_\_\_\_\_\_\_\_\_\_\_\_\_\_\_\_\_\_\_\_\_\_\_\_\_\_\_\_\_\_\_\_\_\_\_\_\_\_\_ \\
\_\_\_\_\_\_\_\_\_\_\_\_\_\_\_\_\_\_\_\_\_\_\_\_\_\_\_\_\_\_\_\_\_\_\_\_\_\_\_\_\_\_\_\_\_\_\_\_\_\_\_\_\_\_\_\_\_\_\_\_\_ \\
\_\_\_\_\_\_\_\_\_\_\_\_\_\_\_\_\_\_\_\_\_\_\_\_\_\_\_\_\_\_\_\_\_\_\_\_\_\_\_\_\_\_\_\_\_\_\_\_\_\_\_\_\_\_\_\_\_\_\_\_\_ \\
\_\_\_\_\_\_\_\_\_\_\_\_\_\_\_\_\_\_\_\_\_\_\_\_\_\_\_\_\_\_\_\_\_\_\_\_\_\_\_\_\_\_\_\_\_\_\_\_\_\_\_\_\_\_\_\_\_\_\_\_\_ \\
\_\_\_\_\_\_\_\_\_\_\_\_\_\_\_\_\_\_\_\_\_\_\_\_\_\_\_\_\_\_\_\_\_\_\_\_\_\_\_\_\_\_\_\_\_\_\_\_\_\_\_\_\_\_\_\_\_\_\_\_\_ \\
\bottomrule()
\end{longtable}

\begin{longtable}[]{@{}ll@{}}
\toprule()
Initialed: & \\
\midrule()
\endhead
\_\_\_\_\_\_\_\_\_\_\_\_\_\_ & \_\_\_\_\_\_\_\_\_\_\_\_\_\_ \\
Player & Team \\
\bottomrule()
\end{longtable}

\newpage

\hypertarget{exhibit-4-trade-payments}{%
\subsection{Exhibit 4 --- Trade Payments}\label{exhibit-4-trade-payments}}

Player: \_\_\_\_\_\_\_\_\_\_\_\_\_\_\_\_\_\_\_\_\_\_\_\_\_\_\_\_\\
Team: \_\_\_\_\_\_\_\_\_\_\_\_\_\_\_\_\_\_\_\_\_\_\_\_\_\_\_\_\\
Date: \_\_\_\_\_\_\_\_\_\_\_\_\_\_\_\_\_\_\_\_\_\_\_\_\_\_\_\_

In the event this Contract is traded by the Team executing the Contract to another NBA Team, the Player shall be entitled to receive from the assignor Team, within thirty (30) days of the date of such trade, the following payment:

\begin{longtable}[]{@{}l@{}}
\toprule()
\endhead
\_\_\_\_\_\_\_\_\_\_\_\_\_\_\_\_\_\_\_\_\_\_\_\_\_\_\_\_\_\_\_\_\_\_\_\_\_\_\_\_\_\_\_\_\_\_\_\_\_\_\_\_\_\_\_\_\_\_\_\_\_ \\
\_\_\_\_\_\_\_\_\_\_\_\_\_\_\_\_\_\_\_\_\_\_\_\_\_\_\_\_\_\_\_\_\_\_\_\_\_\_\_\_\_\_\_\_\_\_\_\_\_\_\_\_\_\_\_\_\_\_\_\_\_ \\
\_\_\_\_\_\_\_\_\_\_\_\_\_\_\_\_\_\_\_\_\_\_\_\_\_\_\_\_\_\_\_\_\_\_\_\_\_\_\_\_\_\_\_\_\_\_\_\_\_\_\_\_\_\_\_\_\_\_\_\_\_ \\
\_\_\_\_\_\_\_\_\_\_\_\_\_\_\_\_\_\_\_\_\_\_\_\_\_\_\_\_\_\_\_\_\_\_\_\_\_\_\_\_\_\_\_\_\_\_\_\_\_\_\_\_\_\_\_\_\_\_\_\_\_ \\
\_\_\_\_\_\_\_\_\_\_\_\_\_\_\_\_\_\_\_\_\_\_\_\_\_\_\_\_\_\_\_\_\_\_\_\_\_\_\_\_\_\_\_\_\_\_\_\_\_\_\_\_\_\_\_\_\_\_\_\_\_ \\
\_\_\_\_\_\_\_\_\_\_\_\_\_\_\_\_\_\_\_\_\_\_\_\_\_\_\_\_\_\_\_\_\_\_\_\_\_\_\_\_\_\_\_\_\_\_\_\_\_\_\_\_\_\_\_\_\_\_\_\_\_ \\
\_\_\_\_\_\_\_\_\_\_\_\_\_\_\_\_\_\_\_\_\_\_\_\_\_\_\_\_\_\_\_\_\_\_\_\_\_\_\_\_\_\_\_\_\_\_\_\_\_\_\_\_\_\_\_\_\_\_\_\_\_ \\
\_\_\_\_\_\_\_\_\_\_\_\_\_\_\_\_\_\_\_\_\_\_\_\_\_\_\_\_\_\_\_\_\_\_\_\_\_\_\_\_\_\_\_\_\_\_\_\_\_\_\_\_\_\_\_\_\_\_\_\_\_ \\
\_\_\_\_\_\_\_\_\_\_\_\_\_\_\_\_\_\_\_\_\_\_\_\_\_\_\_\_\_\_\_\_\_\_\_\_\_\_\_\_\_\_\_\_\_\_\_\_\_\_\_\_\_\_\_\_\_\_\_\_\_ \\
\_\_\_\_\_\_\_\_\_\_\_\_\_\_\_\_\_\_\_\_\_\_\_\_\_\_\_\_\_\_\_\_\_\_\_\_\_\_\_\_\_\_\_\_\_\_\_\_\_\_\_\_\_\_\_\_\_\_\_\_\_ \\
\_\_\_\_\_\_\_\_\_\_\_\_\_\_\_\_\_\_\_\_\_\_\_\_\_\_\_\_\_\_\_\_\_\_\_\_\_\_\_\_\_\_\_\_\_\_\_\_\_\_\_\_\_\_\_\_\_\_\_\_\_ \\
\_\_\_\_\_\_\_\_\_\_\_\_\_\_\_\_\_\_\_\_\_\_\_\_\_\_\_\_\_\_\_\_\_\_\_\_\_\_\_\_\_\_\_\_\_\_\_\_\_\_\_\_\_\_\_\_\_\_\_\_\_ \\
\_\_\_\_\_\_\_\_\_\_\_\_\_\_\_\_\_\_\_\_\_\_\_\_\_\_\_\_\_\_\_\_\_\_\_\_\_\_\_\_\_\_\_\_\_\_\_\_\_\_\_\_\_\_\_\_\_\_\_\_\_ \\
\_\_\_\_\_\_\_\_\_\_\_\_\_\_\_\_\_\_\_\_\_\_\_\_\_\_\_\_\_\_\_\_\_\_\_\_\_\_\_\_\_\_\_\_\_\_\_\_\_\_\_\_\_\_\_\_\_\_\_\_\_ \\
\_\_\_\_\_\_\_\_\_\_\_\_\_\_\_\_\_\_\_\_\_\_\_\_\_\_\_\_\_\_\_\_\_\_\_\_\_\_\_\_\_\_\_\_\_\_\_\_\_\_\_\_\_\_\_\_\_\_\_\_\_ \\
\bottomrule()
\end{longtable}

\begin{longtable}[]{@{}ll@{}}
\toprule()
Initialed: & \\
\midrule()
\endhead
\_\_\_\_\_\_\_\_\_\_\_\_\_\_ & \_\_\_\_\_\_\_\_\_\_\_\_\_\_ \\
Player & Team \\
\bottomrule()
\end{longtable}

\newpage

\hypertarget{exhibit-5-other-activities}{%
\subsection{Exhibit 5 --- Other Activities}\label{exhibit-5-other-activities}}

Player: \_\_\_\_\_\_\_\_\_\_\_\_\_\_\_\_\_\_\_\_\_\_\_\_\_\_\_\_\\
Team: \_\_\_\_\_\_\_\_\_\_\_\_\_\_\_\_\_\_\_\_\_\_\_\_\_\_\_\_\\
Date: \_\_\_\_\_\_\_\_\_\_\_\_\_\_\_\_\_\_\_\_\_\_\_\_\_\_\_\_

Notwithstanding the provisions of Paragraph 12 of this Contract, the Player and the Team agree that the Player need not obtain the consent of the Team in order to engage in the activities set forth below:

\begin{longtable}[]{@{}l@{}}
\toprule()
\endhead
\_\_\_\_\_\_\_\_\_\_\_\_\_\_\_\_\_\_\_\_\_\_\_\_\_\_\_\_\_\_\_\_\_\_\_\_\_\_\_\_\_\_\_\_\_\_\_\_\_\_\_\_\_\_\_\_\_\_\_\_\_ \\
\_\_\_\_\_\_\_\_\_\_\_\_\_\_\_\_\_\_\_\_\_\_\_\_\_\_\_\_\_\_\_\_\_\_\_\_\_\_\_\_\_\_\_\_\_\_\_\_\_\_\_\_\_\_\_\_\_\_\_\_\_ \\
\_\_\_\_\_\_\_\_\_\_\_\_\_\_\_\_\_\_\_\_\_\_\_\_\_\_\_\_\_\_\_\_\_\_\_\_\_\_\_\_\_\_\_\_\_\_\_\_\_\_\_\_\_\_\_\_\_\_\_\_\_ \\
\_\_\_\_\_\_\_\_\_\_\_\_\_\_\_\_\_\_\_\_\_\_\_\_\_\_\_\_\_\_\_\_\_\_\_\_\_\_\_\_\_\_\_\_\_\_\_\_\_\_\_\_\_\_\_\_\_\_\_\_\_ \\
\_\_\_\_\_\_\_\_\_\_\_\_\_\_\_\_\_\_\_\_\_\_\_\_\_\_\_\_\_\_\_\_\_\_\_\_\_\_\_\_\_\_\_\_\_\_\_\_\_\_\_\_\_\_\_\_\_\_\_\_\_ \\
\_\_\_\_\_\_\_\_\_\_\_\_\_\_\_\_\_\_\_\_\_\_\_\_\_\_\_\_\_\_\_\_\_\_\_\_\_\_\_\_\_\_\_\_\_\_\_\_\_\_\_\_\_\_\_\_\_\_\_\_\_ \\
\_\_\_\_\_\_\_\_\_\_\_\_\_\_\_\_\_\_\_\_\_\_\_\_\_\_\_\_\_\_\_\_\_\_\_\_\_\_\_\_\_\_\_\_\_\_\_\_\_\_\_\_\_\_\_\_\_\_\_\_\_ \\
\_\_\_\_\_\_\_\_\_\_\_\_\_\_\_\_\_\_\_\_\_\_\_\_\_\_\_\_\_\_\_\_\_\_\_\_\_\_\_\_\_\_\_\_\_\_\_\_\_\_\_\_\_\_\_\_\_\_\_\_\_ \\
\_\_\_\_\_\_\_\_\_\_\_\_\_\_\_\_\_\_\_\_\_\_\_\_\_\_\_\_\_\_\_\_\_\_\_\_\_\_\_\_\_\_\_\_\_\_\_\_\_\_\_\_\_\_\_\_\_\_\_\_\_ \\
\_\_\_\_\_\_\_\_\_\_\_\_\_\_\_\_\_\_\_\_\_\_\_\_\_\_\_\_\_\_\_\_\_\_\_\_\_\_\_\_\_\_\_\_\_\_\_\_\_\_\_\_\_\_\_\_\_\_\_\_\_ \\
\_\_\_\_\_\_\_\_\_\_\_\_\_\_\_\_\_\_\_\_\_\_\_\_\_\_\_\_\_\_\_\_\_\_\_\_\_\_\_\_\_\_\_\_\_\_\_\_\_\_\_\_\_\_\_\_\_\_\_\_\_ \\
\_\_\_\_\_\_\_\_\_\_\_\_\_\_\_\_\_\_\_\_\_\_\_\_\_\_\_\_\_\_\_\_\_\_\_\_\_\_\_\_\_\_\_\_\_\_\_\_\_\_\_\_\_\_\_\_\_\_\_\_\_ \\
\_\_\_\_\_\_\_\_\_\_\_\_\_\_\_\_\_\_\_\_\_\_\_\_\_\_\_\_\_\_\_\_\_\_\_\_\_\_\_\_\_\_\_\_\_\_\_\_\_\_\_\_\_\_\_\_\_\_\_\_\_ \\
\_\_\_\_\_\_\_\_\_\_\_\_\_\_\_\_\_\_\_\_\_\_\_\_\_\_\_\_\_\_\_\_\_\_\_\_\_\_\_\_\_\_\_\_\_\_\_\_\_\_\_\_\_\_\_\_\_\_\_\_\_ \\
\_\_\_\_\_\_\_\_\_\_\_\_\_\_\_\_\_\_\_\_\_\_\_\_\_\_\_\_\_\_\_\_\_\_\_\_\_\_\_\_\_\_\_\_\_\_\_\_\_\_\_\_\_\_\_\_\_\_\_\_\_ \\
\bottomrule()
\end{longtable}

\begin{longtable}[]{@{}ll@{}}
\toprule()
Initialed: & \\
\midrule()
\endhead
\_\_\_\_\_\_\_\_\_\_\_\_\_\_ & \_\_\_\_\_\_\_\_\_\_\_\_\_\_ \\
Player & Team \\
\bottomrule()
\end{longtable}

\newpage

\hypertarget{exhibit-6-physical-exam}{%
\subsection{Exhibit 6 --- Physical Exam}\label{exhibit-6-physical-exam}}

Player: \_\_\_\_\_\_\_\_\_\_\_\_\_\_\_\_\_\_\_\_\_\_\_\_\_\_\_\_\\
Team: \_\_\_\_\_\_\_\_\_\_\_\_\_\_\_\_\_\_\_\_\_\_\_\_\_\_\_\_\\
Date: \_\_\_\_\_\_\_\_\_\_\_\_\_\_\_\_\_\_\_\_\_\_\_\_\_\_\_\_

The Player and the Team agree that this Contract will be invalid and of no force and effect unless the Player passes, in the sole discretion of the Team, exercised in good faith, in consultation with one or more of the Team's physicians, a physical examination in accordance with Article II, Section 13(h) of the CBA that is (i) conducted within three (3) business days of the execution of this Contract, and (ii) the results of which are reported by the Team to the Player within six (6) business days of the execution of this Contract. The Player agrees to supply complete and truthful information in connection with any such examinations.

\begin{longtable}[]{@{}ll@{}}
\toprule()
Initialed: & \\
\midrule()
\endhead
\_\_\_\_\_\_\_\_\_\_\_\_\_\_ & \_\_\_\_\_\_\_\_\_\_\_\_\_\_ \\
Player & Team \\
\bottomrule()
\end{longtable}

\newpage

\hypertarget{exhibit-7-substitution-for-upc-paragraph-7b}{%
\subsection{Exhibit 7 --- Substitution for UPC Paragraph 7(b)}\label{exhibit-7-substitution-for-upc-paragraph-7b}}

Player: \_\_\_\_\_\_\_\_\_\_\_\_\_\_\_\_\_\_\_\_\_\_\_\_\_\_\_\_\\
Team: \_\_\_\_\_\_\_\_\_\_\_\_\_\_\_\_\_\_\_\_\_\_\_\_\_\_\_\_\\
Date: \_\_\_\_\_\_\_\_\_\_\_\_\_\_\_\_\_\_\_\_\_\_\_\_\_\_\_\_

Paragraph 7(b) is hereby deleted and the following shall be substituted in place and instead thereof:

\begin{quote}
``7. (b) The Player agrees, notwithstanding any other provision of this Contract, that he will to the best of his ability maintain himself in physical condition sufficient to play skilled basketball at all times. If the Player, in the reasonable judgment of the physician designated for that purpose by the Team, is not in good physical condition at the date of his first scheduled game for the Team, or if, at the beginning of or during any Season, he fails to remain in good physical condition, in either event so as to render the Player unfit in the reasonable judgment of said physician to play skilled basketball, the Team shall have the right to suspend the Player for successive one-week periods until the Player, in the reasonable judgment of the Team's physician, is in good physical condition; provided, however, that at the end of each such one-week period of suspension, if the Team notifies the Player, orally or in writing, that in its reasonable judgment it believes the Player is still not in good physical condition, and if the Player so requests, then the Player shall be examined by a physician or physicians designated for such purpose by the President, or any Vice President if the President is not available, of the American Society of Orthopedic Physicians, or equivalent organization (the''Reviewing Physician''), whose sole judgment concerning the physical condition of the Player to play skilled basketball shall be binding upon the Team and the Player for purposes of this Paragraph. The suspension of the Player shall be terminated promptly upon the failure of the Team to give the Player the notice required at the end of the one-week period or upon the finding of said Reviewing Physician that the Player is in physical condition sufficient to play skilled basketball. In the event of a suspension permitted hereunder, the Compensation (excluding any signing bonus or Incentive Compensation) payable to the Player for any Season during such suspension shall be reduced in the same proportion as the length of the period of disability so determined bears to the length of the Season. Nothing in this Paragraph 7(b) shall authorize the Team to suspend the Player solely because the Player is injured or ill.''
\end{quote}

\begin{longtable}[]{@{}ll@{}}
\toprule()
Initialed: & \\
\midrule()
\endhead
\_\_\_\_\_\_\_\_\_\_\_\_\_\_ & \_\_\_\_\_\_\_\_\_\_\_\_\_\_ \\
Player & Team \\
\bottomrule()
\end{longtable}

\newpage

\hypertarget{exhibit-8-sign-and-trade}{%
\subsection{Exhibit 8 --- Sign and Trade}\label{exhibit-8-sign-and-trade}}

Player: \_\_\_\_\_\_\_\_\_\_\_\_\_\_\_\_\_\_\_\_\_\_\_\_\_\_\_\_\\
Team: \_\_\_\_\_\_\_\_\_\_\_\_\_\_\_\_\_\_\_\_\_\_\_\_\_\_\_\_\\
Date: \_\_\_\_\_\_\_\_\_\_\_\_\_\_\_\_\_\_\_\_\_\_\_\_\_\_\_\_

The Player and the Team agree that this {[}Contract{]} {[}amendment{]} will be invalid and of no force and effect unless the {[}Contract{]} {[}amendment{]} is traded to the {[}assignee Team{]} within forty-eight (48) hours of its execution, and all conditions to such trade are ultimately satisfied.

\begin{longtable}[]{@{}ll@{}}
\toprule()
Initialed: & \\
\midrule()
\endhead
\_\_\_\_\_\_\_\_\_\_\_\_\_\_ & \_\_\_\_\_\_\_\_\_\_\_\_\_\_ \\
Player & Team \\
\bottomrule()
\end{longtable}

\newpage

\hypertarget{exhibit-9-one-season-non-guaranteed-training-camp-contracts}{%
\subsection{Exhibit 9 --- One-Season, Non-Guaranteed Training Camp Contracts}\label{exhibit-9-one-season-non-guaranteed-training-camp-contracts}}

Player: \_\_\_\_\_\_\_\_\_\_\_\_\_\_\_\_\_\_\_\_\_\_\_\_\_\_\_\_\\
Team: \_\_\_\_\_\_\_\_\_\_\_\_\_\_\_\_\_\_\_\_\_\_\_\_\_\_\_\_\\
Date: \_\_\_\_\_\_\_\_\_\_\_\_\_\_\_\_\_\_\_\_\_\_\_\_\_\_\_\_

The Player's right to receive any Compensation under this Contract (other than Compensation in accordance with Paragraph 3(b) and/or Exhibit 10 if such exhibit is contained in this Contract) is eliminated in the event the Contract is terminated prior to the first day of the Regular Season covered by the Contract; provided, however, that if the Player is injured as a direct result of playing for the Team and, accordingly, would have been entitled (but for this Exhibit 9) to Compensation pursuant to Paragraphs 7(c), 16(a)(iii), 16(b), or otherwise, the Team's sole liability (other than Compensation in accordance with Paragraph 3(b) and/or Exhibit 10 if such exhibit is contained in this Contract) shall be to pay the Player \$15,000 upon termination of the Player's Contract.

\begin{longtable}[]{@{}ll@{}}
\toprule()
Initialed: & \\
\midrule()
\endhead
\_\_\_\_\_\_\_\_\_\_\_\_\_\_ & \_\_\_\_\_\_\_\_\_\_\_\_\_\_ \\
Player & Team \\
\bottomrule()
\end{longtable}

\newpage

\hypertarget{exhibit-10-nbagl-bonus-and-two-way-player-conversion}{%
\subsection{Exhibit 10 --- NBAGL Bonus and Two-Way Player Conversion}\label{exhibit-10-nbagl-bonus-and-two-way-player-conversion}}

Player: \_\_\_\_\_\_\_\_\_\_\_\_\_\_\_\_\_\_\_\_\_\_\_\_\_\_\_\\
Team: \_\_\_\_\_\_\_\_\_\_\_\_\_\_\_\_\_\_\_\_\_\_\_\_\_\_\_\_\\
Bonus Amount*: \_\_\_\_\_\_\_\_\_\_\_\_\_\_\_\_\_\_\_\_\_\_\\
NBAGL Affiliate: \_\_\_\_\_\_\_\_\_\_\_\_\_\_\_\_\_\_\_\_\_\\
Conversion Protection Amount: \_\_\_\_\_\_\_\_\_\_\_\_\\
Date: \_\_\_\_\_\_\_\_\_\_\_\_\_\_\_\_\_\_\_\_\_\_\_\_\_\_\_\_

\textbf{Contract Termination/NBAGL:} In the event this Contract is terminated by the Team in accordance with the NBA waiver procedure prior to the first day of the NBA Regular Season, the Player shall be entitled to receive from the Team the Bonus Amount (if applicable) provided above, provided that the Player (a) signs with the NBAGL prior to the deadline set by the NBAGL for NBAGL teams to designate affiliate players, (b) is initially assigned by the NBAGL to the NBAGL affiliate listed above (or the NBAGL affiliate of any Team that acquires the Contract, if applicable) and timely reports to such affiliate, (c) does not leave the NBAGL (e.g., by buying out his contract with the NBAGL and signing a contract with an international team) prior to providing sixty (60) consecutive days of service during the NBAGL Season (the ``60-Day Service Period''), provided that, in the event the player is signed to one or more Contract(s) by the Team prior to completing the 60-Day Service Period, the Player shall still satisfy this clause (c) if he timely returns to the Team's NBAGL affiliate upon the completion or termination of such Contract(s) and completes the outstanding portion of the 60-Day Service Period, with such bonus payable (if applicable) within thirty (30) days of satisfying the above criteria. For clarity, a player will not satisfy clause (c) if at any time prior to completing the 60-Day Service Period he signs a contract with a professional basketball team other than the Team. In the event the Player fails to satisfy clause (c) because his contract with the NBAGL is terminated as a result of an injury resulting directly from his playing for the Team's NBAGL affiliate, such player shall nonetheless be entitled to receive from the Team the Bonus Amount.

\textbf{Contract Termination During Regular Season:} If this Contract is not terminated by the Team in accordance with the NBA waiver procedure prior to the first day of the NBA Regular Season, notwithstanding the absence of an Exhibit 2, the Contract shall be protected for lack of skill and injury or illness at an amount equal to the Conversion Protection Amount in this Exhibit 10.

\textbf{Two-Way Player Conversion Option:} Team shall have the option to convert this Contract to a Two-Way Contract (``Two-Way Player Conversion Option''); provided, however, that (a) such option must be exercised prior to the first day of the NBA Regular Season, and (b) may not be exercised if it would result in a violation of Article X, Section 4(d) of the CBA. Team's Two-Way Player Conversion Option may be exercised by providing written notice to Player that is either personally delivered to Player or his representative or sent by email or pre-paid certified, registered, or overnight mail to the last known address of Player or his representative with a copy to the Players Association and the NBA. If Team exercises the Two-Way Player Conversion Option, this Contract's Exhibit 1A will immediately become null and void and of no further force or effect and the Player's Compensation shall be equal to the Two-Way Player Salary applicable for such Season. Further, upon conversion, the Player's right to the Bonus Amount (if applicable) set forth above pursuant to this Exhibit 10 will be rescinded and the Player's Contract, notwithstanding the absence of an Exhibit 2, shall be protected for lack of skill and injury or illness at an amount equal to the Conversion Protection Amount in this Exhibit 10. All other terms and conditions of this Contract shall remain applicable.

\textbf{Standard NBA Contract Conversion Option:} In the event the Two-Way Player Conversion Option is exercised by the Team, Team shall thereafter have the option to convert the Contract to a Standard NBA Contract (``Standard NBA Contract Conversion Option''). Team's Standard NBA Contract Conversion Option may be exercised by providing written notice to Player that is either personally delivered to Player or his representative or sent by email or pre-paid certified, registered, or overnight mail to the last known address of Player or his representative with a copy to the Players Association and the NBA. If Team exercises the Standard NBA Contract Conversion Option, the Base Compensation amount applicable to the Two-Way Contract as set forth in this Exhibit 10 will immediately become null and void and of no further force or effect, Player's Compensation shall be equal to the Player's applicable Minimum Player Salary for such Season beginning on the date such option is exercised, and all other terms and conditions of this Contract, including the Base Compensation protection set forth in this Exhibit 10, shall remain applicable.

*Bonus Amount must be equal to the Conversion Protection Amount and may only be included if Team has an NBAGL Affiliate.

\begin{longtable}[]{@{}ll@{}}
\toprule()
Initialed: & \\
\midrule()
\endhead
\_\_\_\_\_\_\_\_\_\_\_\_\_\_ & \_\_\_\_\_\_\_\_\_\_\_\_\_\_ \\
Player & Team \\
\bottomrule()
\end{longtable}

\hypertarget{baseline-rookie-salary-scale-000s}{%
\chapter{BASELINE ROOKIE SALARY SCALE (\$000'S)}\label{baseline-rookie-salary-scale-000s}}

\begin{longtable}[]{@{}
  >{\centering\arraybackslash}p{(\columnwidth - 10\tabcolsep) * \real{0.0806}}
  >{\raggedright\arraybackslash}p{(\columnwidth - 10\tabcolsep) * \real{0.1452}}
  >{\raggedright\arraybackslash}p{(\columnwidth - 10\tabcolsep) * \real{0.1613}}
  >{\raggedright\arraybackslash}p{(\columnwidth - 10\tabcolsep) * \real{0.1613}}
  >{\centering\arraybackslash}p{(\columnwidth - 10\tabcolsep) * \real{0.2258}}
  >{\centering\arraybackslash}p{(\columnwidth - 10\tabcolsep) * \real{0.2258}}@{}}
\toprule()
\begin{minipage}[b]{\linewidth}\centering
Pick
\end{minipage} & \begin{minipage}[b]{\linewidth}\raggedright
1st Year Salary
\end{minipage} & \begin{minipage}[b]{\linewidth}\raggedright
2nd Year Salary
\end{minipage} & \begin{minipage}[b]{\linewidth}\raggedright
3rd Year Option Salary
\end{minipage} & \begin{minipage}[b]{\linewidth}\centering
4th Year Option: Percentage Increase Over 3rd Year Salary
\end{minipage} & \begin{minipage}[b]{\linewidth}\centering
Qualifying Offer: Percentage Increase Over 4th Year Salary
\end{minipage} \\
\midrule()
\endhead
1 & 9,212,600 & 9,673,400 & 10,134,000 & 26.1\% & 40.0\% \\
2 & 8,242,700 & 8,655,000 & 9,067,200 & 26.2\% & 40.5\% \\
3 & 7,402,200 & 7,772,100 & 8,142,400 & 26.4\% & 41.2\% \\
4 & 6,673,700 & 7,007,500 & 7,341,300 & 26.5\% & 41.9\% \\
5 & 6,043,500 & 6,345,400 & 6,647,700 & 26.7\% & 42.6\% \\
6 & 5,489,000 & 5,763,400 & 6,038,100 & 26.8\% & 43.4\% \\
7 & 5,010,800 & 5,261,500 & 5,511,800 & 27.0\% & 44.1\% \\
8 & 4,590,500 & 4,820,100 & 5,049,600 & 27.2\% & 44.8\% \\
9 & 4,219,600 & 4,430,800 & 4,641,700 & 27.4\% & 45.5\% \\
10 & 4,008,600 & 4,209,000 & 4,409,300 & 27.5\% & 46.2\% \\
11 & 3,808,200 & 3,998,700 & 4,189,200 & 32.7\% & 46.9\% \\
12 & 3,617,900 & 3,798,900 & 3,979,800 & 37.8\% & 47.6\% \\
13 & 3,436,900 & 3,608,900 & 3,780,700 & 42.9\% & 48.3\% \\
14 & 3,265,300 & 3,428,500 & 3,591,900 & 48.1\% & 49.1\% \\
15 & 3,101,700 & 3,256,800 & 3,411,900 & 53.3\% & 49.8\% \\
16 & 2,946,800 & 3,094,100 & 3,241,600 & 53.4\% & 50.5\% \\
17 & 2,799,300 & 2,939,300 & 3,079,300 & 53.6\% & 51.2\% \\
18 & 2,659,500 & 2,792,300 & 2,925,400 & 53.8\% & 51.9\% \\
19 & 2,539,700 & 2,666,600 & 2,793,900 & 54.0\% & 52.6\% \\
20 & 2,438,000 & 2,559,900 & 2,681,600 & 54.2\% & 53.3\% \\
21 & 2,340,500 & 2,457,600 & 2,574,700 & 59.3\% & 54.1\% \\
22 & 2,247,000 & 2,359,300 & 2,471,600 & 64.5\% & 54.8\% \\
23 & 2,157,200 & 2,265,200 & 2,372,700 & 69.7\% & 55.5\% \\
24 & 2,071,000 & 2,174,500 & 2,278,100 & 74.9\% & 56.2\% \\
25 & 1,987,900 & 2,087,200 & 2,186,900 & 80.1\% & 56.9\% \\
26 & 1,922,100 & 2,018,100 & 2,114,200 & 80.3\% & 57.6\% \\
27 & 1,866,600 & 1,960,000 & 2,053,500 & 80.4\% & 58.3\% \\
28 & 1,855,000 & 1,948,100 & 2,040,700 & 80.5\% & 59.0\% \\
29 & 1,841,700 & 1,933,700 & 2,025,900 & 80.5\% & 60.0\% \\
30 & 1,828,300 & 1,919,600 & 2,011,300 & 80.5\% & 60.0\% \\
\bottomrule()
\end{longtable}

\hypertarget{baseline-minimum-annual-salary-scale}{%
\chapter{BASELINE MINIMUM ANNUAL SALARY SCALE}\label{baseline-minimum-annual-salary-scale}}

\begin{longtable}[]{@{}
  >{\centering\arraybackslash}p{(\columnwidth - 10\tabcolsep) * \real{0.0781}}
  >{\raggedright\arraybackslash}p{(\columnwidth - 10\tabcolsep) * \real{0.1875}}
  >{\raggedright\arraybackslash}p{(\columnwidth - 10\tabcolsep) * \real{0.1875}}
  >{\raggedright\arraybackslash}p{(\columnwidth - 10\tabcolsep) * \real{0.1875}}
  >{\raggedright\arraybackslash}p{(\columnwidth - 10\tabcolsep) * \real{0.1875}}
  >{\raggedright\arraybackslash}p{(\columnwidth - 10\tabcolsep) * \real{0.1719}}@{}}
\toprule()
\begin{minipage}[b]{\linewidth}\centering
Years of Service
\end{minipage} & \begin{minipage}[b]{\linewidth}\raggedright
Year 1
\end{minipage} & \begin{minipage}[b]{\linewidth}\raggedright
Year 2
\end{minipage} & \begin{minipage}[b]{\linewidth}\raggedright
Year 3
\end{minipage} & \begin{minipage}[b]{\linewidth}\raggedright
Year 4
\end{minipage} & \begin{minipage}[b]{\linewidth}\raggedright
Year 5
\end{minipage} \\
\midrule()
\endhead
0 & 1,017,781 & - & - & - & - \\
1 & 1,637,966 & 1,719,864 & - & - & - \\
2 & 1,836,090 & 1,927,896 & 2,019,699 & - & - \\
3 & 1,902,133 & 1,997,238 & 2,092,344 & 2,187,451 & - \\
4 & 1,968,175 & 2,066,585 & 2,164,993 & 2,263,403 & 2,361,812 \\
5 & 2,133,278 & 2,239,943 & 2,346,606 & 2,453,270 & 2,559,934 \\
6 & 2,298,385 & 2,413,304 & 2,528,221 & 2,643,140 & 2,758,060 \\
7 & 2,463,490 & 2,586,665 & 2,709,839 & 2,833,013 & 2,956,189 \\
8 & 2,628,597 & 2,760,026 & 2,891,458 & 3,022,889 & 3,154,319 \\
9 & 2,641,682 & 2,773,765 & 2,905,850 & 3,037,934 & 3,170,018 \\
10+ & 2,905,851 & 3,051,144 & 3,196,438 & 3,341,730 & 3,487,023 \\
\bottomrule()
\end{longtable}

\hypertarget{bri-expense-ratios}{%
\chapter{BRI EXPENSE RATIOS}\label{bri-expense-ratios}}

\hypertarget{team-and-related-party-expenses-article-vii-section-1a6v}{%
\section{Team and Related Party Expenses, Article VII, Section 1(a)(6)(v)}\label{team-and-related-party-expenses-article-vii-section-1a6v}}

\begin{longtable}[]{@{}ll@{}}
\toprule()
\textbf{Category} & \textbf{Ratio of Expenses to Revenues} \\
\midrule()
\endhead
& \\
Uniform Expense Cap & 11.1\% \\
\bottomrule()
\end{longtable}

\hypertarget{league-expenses-article-vii-section-1a1ix}{%
\section{League Expenses, Article VII, Section 1(a)(1)(ix)}\label{league-expenses-article-vii-section-1a1ix}}

\begin{longtable}[]{@{}ll@{}}
\toprule()
\textbf{Category} & \textbf{Ratio of Expenses to Revenues} \\
\midrule()
\endhead
& \\
Sponsorships & 19\% \\
NBA Entertainment & 35\% \\
International Television & 22\% \\
Special Events & 100\% \\
\bottomrule()
\end{longtable}

\hypertarget{notice-to-veteran-players-concerning-summer-leagues}{%
\chapter{NOTICE TO VETERAN PLAYERS CONCERNING SUMMER LEAGUES}\label{notice-to-veteran-players-concerning-summer-leagues}}

\begin{enumerate}
\def\labelenumi{\arabic{enumi}.}
\item
  Under the Uniform Player Contract and the Collective Bargaining Agreement between the NBA and the Players Association, the Team cannot require players to participate in any summer league.
\item
  The failure of a player to participate in a summer league will not, by itself, prejudice or disadvantage such player in his Team standing or relationship.
\item
  The Team reserves the right to determine how many and which players it may enroll in any summer league.
\end{enumerate}

We would appreciate your signing in the space provided below to acknowledge that you have freely chosen to participate in summer league play on a voluntary basis during the summer of \_\_\_\_.

\begin{longtable}[]{@{}lc@{}}
\toprule()
Agreed to and Accepted: & \\
\midrule()
\endhead
\_\_\_\_\_\_\_\_\_\_\_\_\_\_\_\_\_\_\_\_\_ & \_\_\_\_\_\_ \\
(Name of Player) & \\
\_\_\_\_\_\_\_\_\_\_\_\_\_\_\_\_\_\_\_\_\_ & \_\_\_\_\_\_ \\
(Date) & \\
\bottomrule()
\end{longtable}

\hypertarget{joint-nbanbpa-policy-on-domestic-violence-sexual-assault-and-child-abuse}{%
\chapter{JOINT NBA/NBPA POLICY ON DOMESTIC VIOLENCE, SEXUAL ASSAULT, AND CHILD ABUSE}\label{joint-nbanbpa-policy-on-domestic-violence-sexual-assault-and-child-abuse}}

\chaptermark{JOINT NBA/NBPA POLICY ON DOMESTIC VIOLENCE, SEXUAL ASSAUL \ldots}

Through this Policy, the National Basketball Association (``NBA'') and the National Basketball Players Association (``NBPA'') (collectively, ``the Parties'') have agreed to work together to address domestic violence, sexual assault, and child abuse in the NBA.

\textbf{Covered Behavior}

Acts that constitute domestic violence, sexual assault, and child abuse are prohibited at all times and regardless of where they occur.

For purposes of this Policy, ``domestic violence'' includes, but is not limited to, any actual or attempted violent act that is committed by one party in an intimate or family relationship against another party in that relationship. Such an act may include physical assault or battery, sexual assault, stalking, harassment, or other forms of physical or psychological abuse. It may also include behavior that intimidates, manipulates, humiliates, isolates, frightens, terrorizes, coerces, threatens, injures, or places another person in fear of bodily harm. Domestic violence can be perpetrated by current or former spouses, current or former domestic or same sex partners, persons who are living together or have cohabitated, persons with children in common, persons who have or had an intimate or dating relationship, and family members. Domestic violence can be a single act or a pattern of behavior in a relationship.

For purposes of this Policy, ``sexual assault'' includes, but is not limited to, any actual or attempted sexual contact or act to which one party has not consented. Lack of consent is deemed to exist when a person uses or threatens the use of force, harassment, or any other form of coercion against another. Lack of consent is also deemed to exist when a person is mentally incapable of giving consent, as a result of disability, incapacitation, intoxication, or otherwise.

For purposes of this Policy, ``child abuse'' includes, but is not limited to, any act or failure to act by a parent, caregiver, or adult that results in death, serious physical or emotional harm, or sexual or other exploitation of a child. Child abuse also includes behavior that poses an imminent risk of such harm to a child.

\textbf{Policy Committee}

The Parties shall establish a joint committee to provide education, support, treatment, referrals, counseling, and other resources for players, their family members, and others at risk (the ``Policy Committee''). The Policy Committee will be comprised of two representatives from the NBA and two representatives from the NBPA (the ``Party Representatives''), as well as three independent experts with experience in domestic violence, sexual assault, and/or child abuse (the ``Expert Representatives''). All decisions of the Policy Committee shall be made by a majority vote, unless otherwise stated in this Policy, and shall be final, binding, and unappealable.

The Party Representatives shall jointly select the three Expert Representatives to serve on the Policy Committee within 60 days of the issuance of this Policy. There shall be at least one Expert Representative on the Policy Committee at all times with specific expertise in each of the three subject areas (i.e., domestic violence, sexual assault, and child abuse). The Expert Representatives will each serve for the duration of this Policy; provided, however, that either the NBA or the NBPA may discharge any of them on an annual basis by serving written notice upon the Expert Representative(s) and upon the other Party within 60 days of the anniversary of the appointment of such person. If an Expert Representative is discharged, the Party Representatives shall jointly select a successor Expert Representative within 30 days of the notice of discharge.

In the event that the Party Representatives are unable to agree upon and jointly select any or all of the Expert Representatives within 60 days of the issuance of this Policy or within 30 days of the notice of any discharge of an Expert Representative, the following process will be implemented. Within five days following the deadline to select the Expert Representative(s), the Party Representatives shall exchange lists containing the names and qualifications of three proposed Expert Representatives per open position. Within five days following the exchange of such lists, the Party Representatives shall jointly select from that group of individuals the Expert Representative(s) needed to serve on the Policy Committee. If they are unable to do so, then, within an additional three-day period, the Party Representatives shall engage in a process of alternatively striking names from the lists until one name remains for each open position, and such person(s) shall be appointed as the Expert Representative(s).

\textbf{Training and Education}

The Parties seek to prevent incidents of domestic violence, sexual assault, and child abuse from occurring through educational programs and awareness training.

The Policy Committee will implement and oversee all training and educational programs for NBA players that address issues of domestic violence, sexual assault, and child abuse, and shall make all determinations related thereto including, but not limited to, the staffing, content, format, and frequency of such programs. The Policy Committee will annually review such programs to ensure that they are effective and that the content is appropriate, thorough, and properly communicated to the players.

\textbf{Hotline}

Within 60 days of the issuance of this Policy, the Parties shall jointly select a service provider to support a 24-hour, confidential hotline that can be used by players, their families, and other victims of domestic violence, sexual assault, and child abuse as defined by this Policy to seek assistance and referrals (the ``Service Provider'').

If the Parties are unable to do so, then, within five days following the deadline to select the Service Provider, they shall exchange lists containing the names, qualifications, and cost of three proposed Service Providers. Within five days following the exchange of such lists, the Parties shall jointly select the Service Provider. If the Parties are unable to do so, then, within an additional three-day period, they shall engage in a process of alternatively striking names from the lists until one name remains, and such organization shall be appointed as the Service Provider.

\textbf{Treatment and Intervention}

\begin{enumerate}
\def\labelenumi{\arabic{enumi}.}
\item
  General

  The NBA or the NBPA may refer a player to the Policy Committee in any of the following circumstances:

  \begin{enumerate}
  \def\labelenumii{\alph{enumii}.}
  \tightlist
  \item
    As part of a disciplinary determination of the Commissioner for conduct in violation of this Policy; or
  \item
    After a Player is criminally convicted of an offense that involves conduct in violation of this Policy.
  \end{enumerate}

  The Policy Committee will also be available as a resource to any player who voluntarily seeks assistance.

  Once a player has been referred to the Policy Committee, an expert selected by the Policy Committee will conduct an initial evaluation of the player as soon as is practicable. Following such evaluation, the Policy Committee will develop a Treatment and Accountability Plan (``TAP'') for the player, as may be appropriate. As part of the TAP, the Policy Committee may require that the player submit to psychological or other evaluations and/or attend counseling sessions with a licensed professional, and take other steps that it deems necessary. In developing the TAP, the Policy Committee will take into account any treatment or counseling that the player may have initiated on his own or pursuant to a criminal resolution of any charges against him.

  The Policy Committee will oversee the player's compliance with any TAP, and shall provide additional support to the player as needed. Any treating professionals shall provide regular, written status reports to the Policy Committee that detail the player's progress and compliance with the TAP. The Policy Committee may periodically revise, modify, extend, or close the TAP on its own initiative, on the recommendation of the player's treating professional(s), or upon petition of the player. All information related to a player's involvement with the Policy Committee shall be kept confidential.

  The Policy Committee shall determine whether the player has successfully completed his TAP, and may also issue a revised TAP at any time. A player must receive a certification of completion from the Policy Committee in order to conclude his treatment and the oversight of the Policy Committee.
\item
  Non-Compliance

  Players are required to comply with the directives of the Policy Committee, including with his TAP. If the Policy Committee determines that a player has failed to comply without a reasonable explanation, it shall notify the NBA. For the first such instance of non-compliance, the NBA shall issue a warning to the player. If such non-compliance continues for three additional days after the warning is issued, or for the second or any additional instances of non-compliance as determined by the Policy Committee, the NBA shall fine the player in the amount of \$10,000 for each day that he fails to comply. Such fines shall continue until the player has, in the judgment of the Policy Committee, resumed full compliance.

  If the Policy Committee determines that a player has demonstrated substantial non-compliance, without a reasonable explanation, through a pattern of behavior that demonstrates a mindful disregard for his treatment responsibilities, it shall notify the NBA, which shall thereupon impose:

  \begin{enumerate}
  \def\labelenumii{\alph{enumii}.}
  \tightlist
  \item
    A one-game suspension for the first instance of substantial non-compliance; and
  \item
    A suspension that is at least one game longer than his immediately-preceding suspension for each additional instance of substantial non-compliance and that shall continue until, in the judgment of the Policy Committee, the player resumes full compliance with its directives, including with his TAP.
  \end{enumerate}
\end{enumerate}

\textbf{Costs}

Any and all costs of the training, education, treatment, intervention, and other resources described above including, but not limited to, the Policy Committee, Expert Representatives, education and training programs, hotline, experts, and counselors, will be shared equally by the Parties (unless otherwise covered by the NBA Players Group Health Plan or other insurance plan provided to NBA players). The NBPA's share shall be paid by the NBA and included in Player Benefits under Article IV, Section 6 of the CBA. The NBA's share will be excluded from the calculation of Benefits under the CBA.

\textbf{Investigation of Incidents}

The NBA will give the NBPA and the player prompt notice of the commencement of any investigation into an alleged violation of this Policy.

The NBA's investigation may include the use of third party resources including, but not limited to, outside legal counsel, outside investigators, or other individuals with relevant experience or expertise.

The NBA will notify the NBPA when it has concluded its investigation and report whether it believes a violation of the Policy has occurred.

\textbf{Cooperation}

Except in circumstances where the player has a reasonable apprehension of criminal prosecution, players shall cooperate fully with any NBA investigation under this Policy. Any player interviewed by the NBA as part of its investigation is entitled to have a representative from the NBPA present during the interview, and the NBA will provide the NBPA with at least 48 hours' notice before any in-person interview.

Failing to cooperate in full, or interfering in any manner, with an NBA investigation will subject the non-cooperative individual to discipline consistent with the terms of Article VI, Section 11(a) of the CBA. It may constitute a violation of this cooperation requirement for a player to attempt to or enter into any agreement with a witness, victim, or other party that would discourage or prevent that individual from cooperating with an NBA investigation. However, the player is under no obligation to demand, request, or otherwise encourage anyone to cooperate with an NBA investigation.

\textbf{Administrative Leave}

While an investigation is pending, the Commissioner may at any time place the player on administrative leave with pay for a reasonable period of time. The parties agree that administrative leave is not intended to be routinely applied during the pendency of every player investigation under this Policy. Instead, administrative leave should be applied in only those cases in which a balancing of all relevant factors clearly establishes that it is reasonable to do so under the totality of the circumstances.

In deciding whether to place a player on paid administrative leave, the Commissioner shall consider among other relevant factors the following non-exhaustive list of factors:

\begin{itemize}
\tightlist
\item
  The nature and severity of the allegation(s), including whether a weapon was involved and whether any injury was suffered by anyone (including the player);
\item
  Whether the allegations are supported by credible information;
\item
  The relationship between the player and accuser;
\item
  Information regarding the player's history of prior similar conduct, or lack thereof;
\item
  The prior criminal or disciplinary history of the player, or lack thereof;
\item
  The status of any criminal investigation and/or prosecution regarding the alleged incident, including whether any arrests have been made;
\item
  The character of the player;
\item
  The player's reputation within the NBA community;
\item
  The NBA's past practice regarding discipline imposed on a player for similar allegations; and
\item
  The risk of reputational damage to the NBA and/or the player's team.
\end{itemize}

The NBA will give prompt notice to the NBPA, the player's team, and the player of any decision to place a player on paid administrative leave pursuant to this Policy. The decision to place the player on paid administrative leave pending an investigation shall not preclude further disciplinary action by the Commissioner against the player in accordance with the provisions of this Policy.

While on administrative leave, the player shall be ineligible to play in any of his team's games. However, the player will continue to receive his salary and other welfare benefits to which he would be entitled as an active player. The player and the player's team may also request that the player be allowed to participate in non-public practices, workouts, or other team activities with the consent of the NBA, which shall not be unreasonably withheld.

A player may challenge the decision to be placed on paid administrative leave under the Grievance and Arbitration Procedure of the CBA. In evaluating such a challenge, the Grievance Arbitrator will determine whether it was reasonable for the Commissioner to place the player on administrative leave. A player may also request the Grievance Arbitrator review the length of a period of administrative leave that exceeds seven days. In such a proceeding, the Grievance Arbitrator will determine whether administrative leave in excess of seven days is reasonable based on the totality of the circumstances. Once a player challenges the decision to be placed on paid administrative leave, or the duration of such leave, the hearing before the Grievance Arbitrator must take place within 72 hours.

\textbf{Discipline}

Based on a finding of just cause, the Commissioner may fine, suspend, or dismiss and disqualify from any further association with the NBA and its teams a player who engages in prohibited conduct in violation of this Policy. Repeat offenders will be subject to enhanced discipline.

Notwithstanding the foregoing, an admission to, or conviction for, any offense that involves conduct that violates this Policy, whether after trial or upon a plea of guilty, as well as any plea of no contest or nolo contendere, will conclusively establish a violation of this Policy. A violation based on this ground, however, shall in no way limit or prevent the NBA from continuing to investigate the incident. Additionally, such admission, conviction, or plea is not required in order for a Policy violation to have occurred. However, a player who is acquitted after trial in a criminal proceeding may not be subject to disciplinary penalties under this Policy.

In conjunction with any discipline imposed by the Commissioner for a violation of this Policy, the NBA may also require the player to undergo an evaluation under the supervision of the Policy Committee, to participate in relevant training, education, or counseling programs as determined by the Policy Committee, and/or to perform community service. Any discipline determined by the Commissioner may be referred to the player's team for imposition.

Prior to the determination of any discipline, the Parties shall meet to discuss the matter. This conference shall be considered confidential, and no statements made during the discussion shall be admissible in any subsequent challenge to any discipline imposed on the player.

The Commissioner will determine all discipline under this Policy on a case-by-case basis, upon consideration of all facts and circumstances, including aggravating and mitigating factors.

Potential aggravating factors include, but are not limited to:

\begin{itemize}
\tightlist
\item
  Prior allegations of, or convictions for, prohibited conduct;
\item
  The use of a weapon or other means of coercion;
\item
  The use of, or threat to use, force or violence;
\item
  The vulnerability of the victim;
\item
  The presence of a minor;
\item
  The nature and extent of any injury to the victim; and
\item
  A civil verdict against the player for the underlying conduct.
\end{itemize}

Potential mitigating factors include, but are not limited to:

\begin{itemize}
\tightlist
\item
  Acceptance of responsibility;
\item
  Evidence of self-defense;
\item
  Complete and truthful cooperation with the investigation;
\item
  Voluntary participation in any treatment or counseling programs;
\item
  The player's overall good character;
\item
  The player's reputation in the NBA community; and
\item
  A civil verdict in favor of the player for the underlying conduct.
\end{itemize}

In cases where the Commissioner imposes a suspension, any period of time the player spent on paid administrative leave will be credited toward the suspension provided that the player remits to the League the applicable portion of salary that the player received while on paid administrative leave.

Challenges to any disciplinary action shall be made through the Grievance Arbitration process of the CBA.

\textbf{Confidentiality}

The Parties recognize the importance of confidentiality and privacy to the success of this Policy. Accordingly, the Parties will maintain confidentiality throughout the investigatory, disciplinary, and treatment process, and will take reasonable measures to protect the information gathered pursuant to this Policy, including by any outside advisors or experts. Any medical information obtained during the investigatory, disciplinary, and treatment process will be kept confidential as required by applicable law.

At the same time, the Parties recognize that disclosure of certain information may be necessary to further the NBA's investigation or may be required by law, including by court order or subpoena. Accordingly, the Parties cannot and do not guarantee that complete confidentiality will be maintained. The Parties also reserve the right to make notifications to law enforcement or other appropriate authorities if either the NBA or the NBPA becomes aware that there is a threat of imminent harm to any individual or in cases where the victim is a child or is either mentally or physically incapacitated. Additionally, in matters where a violation is found and discipline is imposed, such findings and discipline may be the subject of public statements by the NBA and/or the NBPA.

\textbf{Retaliation}

Under this Policy, it is prohibited to retaliate, or threaten to retaliate, against any individual who, in good faith, reports a potential violation of this Policy or who honestly participates in an investigation of such a report. It does not matter whether the investigation establishes that a violation of the Policy occurred, as long as the report of the violation or participation in the investigation is in good faith. Such retaliation includes, but is not limited to, threats, intimidation, harassment, and any adverse employment or other action, whether express or implied. Anyone who retaliates, or threatens to retaliate, against an individual who reports, or participates in an investigation into, an alleged violation of this Policy, or against any victim or other witness, will be subject to independent disciplinary action.

As with any complaint brought in bad faith, any individual, including coaches, general managers, or other team officials, who reports a violation of this Policy knowing such claim is malicious, false, or fundamentally frivolous shall be subject to disciplinary action.

\textbf{Reporting}

Anyone who is the victim of or acting on behalf of a victim of domestic violence, sexual assault, or child abuse, as defined by this Policy, is strongly encouraged to call the hotline established under this Policy as soon as possible after the incident to discuss the availability of counseling, treatment, security, and other appropriate resources.

If you are in immediate danger or involved in a situation in which another person is in immediate danger, the Parties recommend that you contact 911 or your local police department. Support and crisis intervention is also available from the National Domestic Violence Hotline at 1-800-799-SAFE (7233).

\hypertarget{offer-sheet}{%
\chapter{OFFER SHEET}\label{offer-sheet}}

\begin{longtable}[]{@{}
  >{\raggedright\arraybackslash}p{(\columnwidth - 2\tabcolsep) * \real{0.5000}}
  >{\raggedright\arraybackslash}p{(\columnwidth - 2\tabcolsep) * \real{0.5000}}@{}}
\toprule()
\endhead
Name of Player: & Date: \\
\_\_\_\_\_\_\_\_\_\_\_\_\_\_\_\_\_\_\_\_\_\_\_\_\_ & \_\_\_\_\_\_\_\_\_\_\_\_\_\_\_\_\_\_\_\_\_\_\_\_\_ \\
Address of Player and Email Address of Player: & Name of New Team: \\
\_\_\_\_\_\_\_\_\_\_\_\_\_\_\_\_\_\_\_\_\_\_\_\_\_ & \\
\_\_\_\_\_\_\_\_\_\_\_\_\_\_\_\_\_\_\_\_\_\_\_\_\_ & \\
\_\_\_\_\_\_\_\_\_\_\_\_\_\_\_\_\_\_\_\_\_\_\_\_\_ & \_\_\_\_\_\_\_\_\_\_\_\_\_\_\_\_\_\_\_\_\_\_\_\_\_ \\
Name, Address and Email Address of Player's Representative Authorized to Act for Player: & Name of ROFR Team: \\
\_\_\_\_\_\_\_\_\_\_\_\_\_\_\_\_\_\_\_\_\_\_\_\_\_ & \_\_\_\_\_\_\_\_\_\_\_\_\_\_\_\_\_\_\_\_\_\_\_\_\_ \\
\_\_\_\_\_\_\_\_\_\_\_\_\_\_\_\_\_\_\_\_\_\_\_\_\_ & Address of ROFR Team: \\
\_\_\_\_\_\_\_\_\_\_\_\_\_\_\_\_\_\_\_\_\_\_\_\_\_ & \_\_\_\_\_\_\_\_\_\_\_\_\_\_\_\_\_\_\_\_\_\_\_\_\_ \\
\_\_\_\_\_\_\_\_\_\_\_\_\_\_\_\_\_\_\_\_\_\_\_\_\_ & \_\_\_\_\_\_\_\_\_\_\_\_\_\_\_\_\_\_\_\_\_\_\_\_\_ \\
\bottomrule()
\end{longtable}

Attached hereto is an unsigned Player Contract that the New Team has offered to the Player and that the Player desires to accept. The attached Player Contract separately specifies in its exhibits those Principal Terms that will be included in the Player Contract with the ROFR Team if that Team gives the Player a timely First Refusal Exercise Notice.

\begin{longtable}[]{@{}ll@{}}
\toprule()
\endhead
Player: & New Team: \\
& \\
By \_\_\_\_\_\_\_\_\_\_\_\_\_\_\_\_\_\_\_\_\_\_\_\_\_ & By \_\_\_\_\_\_\_\_\_\_\_\_\_\_\_\_\_\_\_\_\_\_\_\_\_ \\
\bottomrule()
\end{longtable}

\hypertarget{first-refusal-exercise-notice}{%
\chapter{FIRST REFUSAL EXERCISE NOTICE}\label{first-refusal-exercise-notice}}

\begin{longtable}[]{@{}
  >{\raggedright\arraybackslash}p{(\columnwidth - 2\tabcolsep) * \real{0.5000}}
  >{\raggedright\arraybackslash}p{(\columnwidth - 2\tabcolsep) * \real{0.5000}}@{}}
\toprule()
\endhead
Name of Player: & Date: \\
\_\_\_\_\_\_\_\_\_\_\_\_\_\_\_\_\_\_\_\_\_\_\_\_\_ & \_\_\_\_\_\_\_\_\_\_\_\_\_\_\_\_\_\_\_\_\_\_\_\_\_ \\
Address of Player: & Name of New Team: \\
\_\_\_\_\_\_\_\_\_\_\_\_\_\_\_\_\_\_\_\_\_\_\_\_\_ & \\
\_\_\_\_\_\_\_\_\_\_\_\_\_\_\_\_\_\_\_\_\_\_\_\_\_ & \\
\_\_\_\_\_\_\_\_\_\_\_\_\_\_\_\_\_\_\_\_\_\_\_\_\_ & \_\_\_\_\_\_\_\_\_\_\_\_\_\_\_\_\_\_\_\_\_\_\_\_\_ \\
Name and Address of Player's Representative Authorized to Act for Player & Name of ROFR Team: \\
\_\_\_\_\_\_\_\_\_\_\_\_\_\_\_\_\_\_\_\_\_\_\_\_\_ & \_\_\_\_\_\_\_\_\_\_\_\_\_\_\_\_\_\_\_\_\_\_\_\_\_ \\
\_\_\_\_\_\_\_\_\_\_\_\_\_\_\_\_\_\_\_\_\_\_\_\_\_ & Address of ROFR Team: \\
\_\_\_\_\_\_\_\_\_\_\_\_\_\_\_\_\_\_\_\_\_\_\_\_\_ & \_\_\_\_\_\_\_\_\_\_\_\_\_\_\_\_\_\_\_\_\_\_\_\_\_ \\
\_\_\_\_\_\_\_\_\_\_\_\_\_\_\_\_\_\_\_\_\_\_\_\_\_ & \_\_\_\_\_\_\_\_\_\_\_\_\_\_\_\_\_\_\_\_\_\_\_\_\_ \\
\bottomrule()
\end{longtable}

The undersigned member of the NBA hereby exercises its Right of First Refusal so as to create a binding agreement with the Player containing the Principal Terms set forth in the Player Contract annexed to the Player's Offer Sheet (a copy of which is attached hereto).

\begin{longtable}[]{@{}l@{}}
\toprule()
\endhead
ROFR Team: \\
By \_\_\_\_\_\_\_\_\_\_\_\_\_\_\_\_\_\_\_\_\_ \\
\bottomrule()
\end{longtable}

\hypertarget{section}{%
\chapter{}\label{section}}

\hypertarget{authorization-for-testing}{%
\section{AUTHORIZATION FOR TESTING}\label{authorization-for-testing}}

\begin{longtable}[]{@{}lc@{}}
\toprule()
\endhead
To: & \_\_\_\_\_\_\_\_\_\_\_\_\_\_\_\_\_\_\_ \\
& \\
Player & \_\_\_\_\_\_\_\_\_\_\_\_\_\_\_\_\_\_\_ \\
\bottomrule()
\end{longtable}

Please be advised that on \_\_\_\_\_\_\_\_\_\_\_\_\_\_\_\_\_\_\_\_\_\_\_\_\_\_, you were the subject of a meeting or conference call held pursuant to the Anti-Drug Program set forth in Article XXXIII of the Collective Bargaining Agreement between the NBA and the National Basketball Players Association, dated June 28, 2023, (the ``Agreement''). Following the meeting or conference call, I authorized the NBA to conduct the testing procedures set forth in the Agreement, and you are hereby directed to submit to those testing procedures, on demand, no more than four (4) times during the next six (6) weeks.

Please be advised that your failure to submit to these procedures may result in substantial penalties, including but not limited to your dismissal and disqualification from the NBA.

\begin{longtable}[]{@{}l@{}}
\toprule()
\endhead
\_\_\_\_\_\_\_\_\_\_\_\_\_\_\_\_\_\_\_\_\_\_\_\_ \\
Independent Expert \\
Dated: \\
\_\_\_\_\_\_\_\_\_\_\_\_\_\_\_\_\_\_\_\_\_\_\_\_ \\
\bottomrule()
\end{longtable}

\newpage

\hypertarget{prohibited-substances}{%
\section{PROHIBITED SUBSTANCES}\label{prohibited-substances}}

A. Drugs of Abuse

\begin{itemize}
\tightlist
\item
  Benzodiazepines:

  \begin{itemize}
  \tightlist
  \item
    Alprazolam (also called Xanax or Niravam)
  \item
    Chlordiazepoxide (also called Librium, Mitran, Poxi or H-Tran)
  \item
    Clonazepam (also called Klonopin, Ceberclon or Valpaz)
  \item
    Diazepam (also called Valium)
  \item
    Lorazepam (also called Ativan)
  \end{itemize}
\item
  Synthetic Cathinones

  \begin{itemize}
  \tightlist
  \item
    4-methyl-N-ethylcathinone (also called 4-MEC)
  \item
    4-methyl-alpha-pyrrolidinopropiophenone (also called 4-MePPP)
  \item
    Alpha-pyrrolidinopentiophenone (also called alpha-PVP)
  \item
    1-(1,3-benzodioxol-5-yl)-2-(methylamino)butan-1-one (also called butylone)
  \item
    2-(methylamino)-1-phenylpentan-1-one (also called pentedrone)
  \item
    1-(1,3-benzodioxol-5-yl)-2-(methylamino)pentan-1-one (also called pentylone)
  \item
    4-fluoro-N-methylcathinone (also called 4-FMC)
  \item
    3-fluoro-N-methylcathinone (also called 3-FMC)
  \item
    1-(naphthalen-2-yl)-2-(pyrrolidin-1-yl)pentan-1-one (also called naphyrone)
  \item
    Alpha-pyrrolidinobutiophenone (also called alpha-PBP)
  \end{itemize}
\item
  Cocaine
\item
  Dimethyltryptamine (DMT)
\item
  Gamma Hydroxybutyrate (GHB)
\item
  Ketamine
\item
  LSD
\item
  Methamphetamine, MDMA, MDA and MDEA
\item
  Opiates:

  \begin{itemize}
  \tightlist
  \item
    Heroin
  \item
    Codeine
  \item
    Morphine
  \item
    Oxycodone (also called Oxycontin, Percocet, Percodan, Roxicet,
  \item
    Tylox, Dazidox, Endocet or Endodan)
  \item
    Hydrocodone (also called Vicodin, Lorcet, Lortab, Hydocan or Norco)
  \item
    Methadone (also called Methadose or Dolophine)
  \item
    Hydromorphone (also called Dilaudid)
  \item
    Fentanyl (also called Actiq or Duragesic) and its analogs (for example, Acetylfentanyl, Methylfentanyl, Alfenanyl, Carfentanyl, and Sufentanyl)
  \item
    Propoxyphene (also called Darvon or Darvocet)
  \item
    Dextromoramide
  \item
    Nicomorphine
  \item
    Oxymorphone
  \item
    Pethidine
  \item
    Phencyclidine (PCP)
  \item
    Psilocin
  \item
    Psilocybin
  \end{itemize}
\end{itemize}

B. Synthetic Cannabinoids

Synthetic Cannabinoids (including, but not limited to, Delta-8-tetrahydrocannabinol (also called delta-8-THC)) and their By-Products

C. Steroids and Performance Enhancing Drugs (SPEDs)

\begin{itemize}
\tightlist
\item
  Adrafinil
\item
  AICAR
\item
  Alexamorelin
\item
  Aminoglutethimide
\item
  Amiphenazole
\item
  Amphetamine and its analogs (with the exceptions of Methamphetamine, MDMA, MDA and MDEA)
\item
  Anamorelin
\item
  Anastrozole
\item
  Androsta-1,4,6-triene-3,17-dione (also called Androstatrienedione or ATD)
\item
  Androsta-3, 5-diene-7, 17-dione (also called Arimistane)
\item
  Androst-2-en-17-one (also called 2-Androstenone and Delta-2)
\item
  Androst-4-ene-3,11,17-trione (also called 11-ketoandrostenedione or adrenosterone)
\item
  Androstanediol
\item
  Androstanedione
\item
  Androstenediol
\item
  Androstenedione
\item
  Androstene-3,6,17-trione (also called called 6-OXO or 4-AT)
\item
  AOD 9604
\item
  BAY 87-2243
\item
  Bolasterone
\item
  Boldenone
\item
  Boldione
\item
  BPC-157
\item
  Bromantan

  \begin{itemize}
  \tightlist
  \item
    6-bromo-androstan-3,17-dione (also called 6-Bromo)
  \item
    6-bromo-androsta-1,4-diene,3,17-dione (also called Aromadrol)
  \end{itemize}
\item
  Buserelin
\item
  Calusterone

  \begin{itemize}
  \tightlist
  \item
    4-chloro-17a-methyl-androsta-1,4-diene-3,17b-diol (also called Halodrol, Halovar and Helladrol)
  \item
    4-chloro-17a-methyl-androst-4-ene-3b,17b-diol (also called P-Mag and Promagnon)
  \item
    4-chloro-17a-methyl-17b-hydroxy-androst-4-ene-3-one (also called Mechabol)
  \item
    4-chloro-17a-methyl-17b-hydroxy-androst-4-ene-3,11-dione (also called Oxyguno)
  \end{itemize}
\item
  Clenbuterol
\item
  Clobenzorex
\item
  Clomiphene
\item
  Clostebol
\item
  Cyclofenil
\item
  Danazol
\item
  Daprodustat
\item
  Dehydrochloromethyltestosterone (also called DHCMT and Turinabol)
\item
  Dehydroepiandrosterone (DHEA)
\item
  Deslorelin
\item
  Desoxymethyltestosterone (DMT)
\item
  Dihydrotestosterone

  \begin{itemize}
  \tightlist
  \item
    4-dihydrotestosterone
  \item
    1, 3-dimethylamylamine (also called DMAA, Methylhexaneamine and Dimethylpentylamine)
  \item
    1, 3-dimethylbutylamine (also called DMBA and 3-DMBA)
  \item
    1, 4-dimethylpentylamine (also called 5-methyl-hexan-2-amine)
  \item
    2a,17a-dimethyl-17b-hydroxy-5b-androstan-3-one (also called Superdrol)
  \end{itemize}
\item
  Ephedra (also called Ma Huang, Bishop's Tea and Chi Powder)
\item
  Ephedrine
\item
  1-Epiandrosterone (also called 1-Andro and 1-DHEA)
\item
  Epitestosterone 2a,3a-epithio-17a-methyl-5a-androstan-17b-ol (also called Epistane and Havoc)
\item
  Erythropoietin (EPO)
\item
  Estra-4,9,11-triene, 17-dione (also called Tren, Trenavar, Trendione and Trenazone)
\item
  13a-ethyl-17a-hydroxygon-4-en-3-one
\item
  Ethylestrenol
\item
  Etilefrine
\item
  Exemestane
\item
  Fadrozole
\item
  Fencamfamin
\item
  Fenethylline
\item
  Fenfluramine
\item
  Fenproporex
\item
  FG-2216
\item
  Fluoxymesterone
\item
  Follistatin 344
\item
  Formebolone
\item
  Formestane (also called 4-hydroxyandrostenedione)
\item
  Furazabol {[}3,2-c{]}-furazan-5a-androstan-17b-ol (also called Furazan or Furuza)
\item
  Gestrinone
\item
  Ghrelin
\item
  Gonadorelin
\item
  Goserelin
\item
  Growth Hormone Releasing Peptide (GHRP)-1*

  \begin{itemize}
  \tightlist
  \item
    GHRP-2 (also called Pralmorelin)*
  \item
    GHRP-3*
  \item
    GHRP-4*
  \item
    GHRP-5*
  \item
    GHRP-6*
  \end{itemize}
\item
  GW 0742
\item
  GW 1516
\item
  Heptaminol
\item
  Hexarelin 18a-homo-17b-hydroxyestr-4-en-3-one18a-homo-3-hydroxy-estra-2,5(10)-dien-17-one (also called M-LMG)
\item
  Human Chorionic Gonadotropin
\item
  Human Growth Hormone (HGH)

  \begin{itemize}
  \tightlist
  \item
    17b-hydroxy-5a-androstano{[}2,3-d{]}isoxazole (also called Androisoxazole or Prostanozol)
  \item
    17b-hydroxy-5a-androstano{[}3,2-c{]}isoxazole
  \item
    17b-hydroxy-17a-methyl- 5a-androst-1-en-3-one (also called Methyl-1-testosterone)
  \item
    3b-hydroxy-estra-4,9,11-trien-17-one
  \item
    4-hydroxytestosterone
  \end{itemize}
\item
  Ibutamoren
\item
  Insulin-like Growth Factor (IGF-1)
\item
  Ipamorelin
\item
  Isometheptene
\item
  Letrozole
\item
  Luteinizing Hormone (LH)
\item
  Mefenorex
\item
  Meldonium
\item
  Mephedrone
\item
  Mestanolone
\item
  Mesterolone
\item
  Methandienone (also called Methandrostenolone)
\item
  Methandriol
\item
  Methasterone
\item
  Methenolone (also called Metenolone)

  \begin{itemize}
  \tightlist
  \item
    7a-Methyl-19-nortestosterone (also called MENT and Trestolone)
  \item
    17a-methyl-19-nortestosterone (also called Methylnortestosterone and Normethandrone)
  \item
    17a-methyl-3a,17b-dihydroxy-5a-androstane
  \item
    17a-methyl-3b,17b-dihydroxy-5a-androstane
  \item
    17a-methyl-3b,17b-dihydroxyandrost-4-ene
  \item
    17a-methyl-4-hydroxynandrolone
  \item
    17a-methyl-5a-androstan-17b-ol (also called Methylandrostanol and Protobol)
  \item
    17a-methyl-androst-2-ene-3,17b-diol
  \item
    17a-methyl-androsta-1,4-diene-3,17b-diol (also called M1 and 4ADD)
  \item
    17a-methyl-androstan-3-hydroxyimine-17b-ol (also called D-Plex)
  \item
    2a-methyl-17b-hydroxy-5b-androstan-3-one (also called Drostanolone and Dromostanolone)
  \item
    6a-methyl-androst-4-ene-3,17-dione
  \end{itemize}
\item
  Methyldienolone
\item
  Methylephedrine
\item
  Methylphenidate
\item
  Methylstenbolone
\item
  Methyltestosterone
\item
  Methyltrienolone (also called Metribolone)
\item
  MHP MYO-X
\item
  Mibolerone
\item
  Modafinil
\item
  Molidustat (also called BAY 85-3934)
\item
  Myostatin Propeptide GDF-8
\item
  N-Benzylpiperazine (also called BZP or 1-benzylpiperazine)
\item
  Nafarelin
\item
  Nandrolone (also called 19-nortestosterone)
\item
  Nikethamide

  \begin{itemize}
  \tightlist
  \item
    19-norandrostenediol (also called Boldandiol)
  \item
    19-norandrostenedione
  \end{itemize}
\item
  Norbolethone (also called Norboletone)
\item
  Norclostebol
\item
  Norethandrolone
\item
  Norfenfluramine
\item
  Norpseudoephedrine (also called Cathine)
\item
  Octodrine
\item
  Oxabolone (also called 4-hydroxy-19-nortestosterone)
\item
  Oxandrolone
\item
  Oxilofrine
\item
  Oxymesterone
\item
  Oxymetholone
\item
  Pemoline
\item
  Pentetrazol
\item
  Phendimetrazine
\item
  Phenmetrazine
\item
  Phentermine
\item
  Phenylpropanolamine (PPA)
\item
  Probenecid
\item
  Prostanozol
\item
  Pseudoephedrine
\item
  {[}3,2,c{]}pyrazole-androst-4-en-17b-ol
\item
  Raloxifene
\item
  Roxadustat (also called FG-4592)
\item
  Quinbolone
\item
  Selective Androgen Receptor Modulator (SARM) S-1*

  \begin{itemize}
  \tightlist
  \item
    SARM S-4 (also called Andarine)*
  \item
    SARM S-9*
  \item
    SARM S-22 (also called Ostarine)*
  \item
    SARM S-23*Fulvestrant SARM S-24*
  \item
    SARM BMS-564,929*
  \item
    SARM LGD-2226*
  \item
    SARM LGD-4033 (also called Ligandrol)*
  \item
    SARM RAD-140 (also called Testolone)*
  \end{itemize}
\item
  Sermorelin
\item
  SR9009 (also called Stenabolic)
\item
  Stanozolol
\item
  Stenbolone
\item
  Strychnine
\item
  Tabimorelin
\item
  Tamoxifen
\item
  TB-500
\item
  Tesamorelin
\item
  Testolactone
\item
  Testosterone
\item
  1-Testosterone
\item
  Tetrahydrogestrinone (THG)
\item
  Tibolone
\item
  Toremifene
\item
  Trenbolone
\item
  Trimetazidine
\item
  Triptorelin
\item
  Vadadustat (also called AKB-6548)
\item
  Zeranol
\item
  Zilpaterol
\end{itemize}

* and any other substance with a similar chemical structure and similar biological effect(s)

D. Diuretics

\begin{itemize}
\tightlist
\item
  Acetazolamide
\item
  Altizide
\item
  Amiloride
\item
  Bendroflumethiazide
\item
  Benzthiazide
\item
  Bumetanide
\item
  Canrenone
\item
  Chlorothiazide
\item
  Chlorthalidone
\item
  Clopamide
\item
  Cyclothiazide
\item
  Dichlorphenamide
\item
  Eplerenone
\item
  Ethacrynic Acid
\item
  Flumethiazide
\item
  Furosemide
\item
  Hydrochlorothiazide
\item
  Hydroflumethiazide
\item
  Indapamide
\item
  Methyclothiazide
\item
  Metolazone
\item
  Polythiazide
\item
  Quinethazone
\item
  Spironolactone
\item
  Torasemide
\item
  Triamterene
\item
  Trichlormethiazide
\end{itemize}

\hypertarget{urine-collection-procedures}{%
\section{URINE COLLECTION PROCEDURES}\label{urine-collection-procedures}}

During the Season, collections for random testing will be scheduled to occur before practices on non-game days, and before shoot-arounds and games on game days. For random drug testing of a visiting team scheduled at game-day shoot-arounds, tests will be scheduled to occur before the shoot-around for that team commences, and for any tests that are not completed by the time the visiting team bus is scheduled to leave the arena or practice facility after the shoot-around is completed, the team will provide alternate transportation to the team hotel for any player that must remain at the arena or practice facility to complete the testing process and will ensure that a Team staff member remains with the affected player(s) and accompanies him or them back to the Team's hotel. Random drug tests can be scheduled to occur at any time during the Off-Season.

When the player arrives at the collection site, the collector will ensure that the player is positively identified through presentation of photo ID or identification by a team representative. If the player's identity cannot be established, the collector shall not proceed with the collection.

The player will be asked to select a sealed urine specimen cup. The player will then provide his urine specimen under the direct observation of the collector.

The collector shall ensure that the player has provided a urine specimen of sufficient volume for accurate testing. If such a sample cannot immediately be provided by the player, he shall be instructed to remain at the testing site for a reasonable period of time until he can provide such a specimen. Once the specimen has been obtained, the player will select a sealed specimen kit, which contains two bottles. The collector, in the presence of the player, will pour the specimen into two bottles. One bottle will be used as the primary or ``A'' specimen and the other will be used as the split or ``B'' specimen. The specimen bottles will be sealed with tamper-proof seals in the presence of the player. The seals will contain a unique identification number that corresponds to the number on the chain of custody form.The player and collector will complete the chain of custody form (which may be in hard copy or electronic form) that documents the handling of the specimen. The collector will note any irregularities concerning the specimen on the chain of custody form. Both the player and collector will sign the chain of custody form. The kit will be sealed and sent via an overnight delivery service to the laboratory for testing. If a hard copy chain-of-custody form is used, it will be included in the kit containing the two specimens that is sent by overnight delivery service to the laboratory. If an electronic chain-of-custody form is used, it will be sent to the laboratory electronically. Once the specimens arrive at the laboratory, the primary specimen will be analyzed. If the primary specimen tests positive or produces an atypical finding, the split sample will be placed in frozen storage and will be available for testing by a different laboratory, if directed by the NBA.

\hypertarget{blood-collection-procedures}{%
\section{BLOOD COLLECTION PROCEDURES}\label{blood-collection-procedures}}

During the Season, collections for random testing will be scheduled to occur after practices on non-game days, and after games on game days. Random tests can be scheduled to occur at any time during the Off-Season.

When the player arrives at the collection site, the collector will ensure that the player is positively identified through presentation of photo ID or identification by a team representative. If the player's identity cannot be established, the collector shall not proceed with the collection.

The player will be asked to select one (1) dried blood spot collection kit and one (1) security kit that will be used to transport the specimen.

The collector shall collect a total of four (4) blood spots. The player's non-dominant arm will be used to make the initial blood draw attempt. If the blood draw is not possible or successful from the non-dominant arm, the dominant arm may be used. In the event that the collector determines that there is no suitable location on the dominant or non-dominant arm due to excessive hair or heavy tattoo ink, the player's thigh may be used as an alternate draw site. No more than three (3) attempts will be made to draw a blood specimen. After that, the collection will be discontinued. Upon completing the blood draw, the collector will ensure that the draw site is not bleeding and bandage the site.

The player and collector will complete the chain of custody form (which may be in hard copy or electronic form) that documents the handling of the specimens. Both the player and collector will sign the chain of custody form. The specimen will be sealed in a blood specimen bag and sent via an overnight delivery service to the laboratory for testing. If a hard copy chain-of-custody form is used, it will be included in the kit containing the two specimens that is sent by overnight delivery service to the laboratory. If an electronic chain-of-custody form is used, it will be sent to the laboratory electronically.

Once the specimens arrive at the laboratory, the primary specimen will be analyzed. If the primary specimen tests positive or produces an atypical finding, the split sample will be placed in frozen storage and will be available for testing by a different laboratory, if directed by the NBA.

\hypertarget{drugs-of-abuse-and-synthetic-cannabinoids-confirmatory-laboratory-analysis-levels}{%
\section{DRUGS OF ABUSE AND SYNTHETIC CANNABINOIDS CONFIRMATORY LABORATORY ANALYSIS LEVELS}\label{drugs-of-abuse-and-synthetic-cannabinoids-confirmatory-laboratory-analysis-levels}}

\textbf{Drugs of Abuse}

\begin{itemize}
\item
  Benzodiazepines 100 ng/ml
\item
  Synthetic Cathinones Any detectable level
\item
  Cocaine Metabolites 150 ng/ml
\item
  Gamma Hydroxybutyrate (GHB) 10 mcg/ml
\item
  Ketamine 100 ng/ml
\item
  LSD 200 pg/ml
\item
  Methamphetamine 500 ng/ml (must also contain amphetamine at a concentration equal to or greater than 200 ng/ml)
\item
  MDMA, MDA and MDEA 500 ng/ml
\item
  Opiates:

  \begin{itemize}
  \tightlist
  \item
    Heroin Metabolite 6-acetylmorphine---10 ng/ml (only if the opiate metabolites are in excess of 2,000 ng/ml)
  \item
    Codeine Metabolites 2,000 ng/ml
  \item
    Morphine Metabolites 2,000 ng/ml
  \item
    Oxycodone 100 ng/ml
  \item
    Hydrocodone 300 ng/ml
  \item
    Methadone 300 ng/ml
  \item
    Hydromorphone 300 ng/ml
  \item
    Fentanyl and its analogs 300 pg/ml
  \item
    Propoxyphene 200 ng/ml
  \item
    Phencyclidine (PCP) 25 ng/ml
  \end{itemize}
\end{itemize}

\textbf{Synthetic Cannabinoids Any detectable level}

\hypertarget{steroids-and-performance-enhancing-drugs-and-diuretics-confirmatory-laboratory-analysis-levels}{%
\section{STEROIDS AND PERFORMANCE-ENHANCING DRUGS AND DIURETICS CONFIRMATORY LABORATORY ANALYSIS LEVELS}\label{steroids-and-performance-enhancing-drugs-and-diuretics-confirmatory-laboratory-analysis-levels}}

All SPEDs and Diuretics (including Human Growth Hormone in its synthetic form and Testosterone in its synthetic form detected through IRMS analysis), except those listed below, at any detectable level.

\begin{itemize}
\tightlist
\item
  Acetazolamide 20 ng/ml
\item
  Amphetamines and their analogs 500 ng/ml
\item
  Bumetanide 20 ng/ml
\item
  Clenbuterol 1 ng/ml
\item
  Clostebol 0.5 ng/ml
\item
  Dehydrochloromethyltestosterone (DHCMT or turinabol) 0.02 ng/ml
\item
  Ephedra/Ephedrine 10 mcg/ml
\item
  Furosemide 20 ng/ml
\item
  GW 1516 0.05 ng/ml
\item
  GW 0742 0.05 ng/ml
\item
  Hydrochlorothiazide 20 ng/ml
\item
  Methylephedrine 10 mcg/ml
\item
  Nandrolone 2 ng/ml
\item
  Norpseudoephedrine 5 mcg/ml
\item
  Phenylpropanolamine (PPA) 25 mcg/ml
\item
  Pseudoephedrine 150 mcg/ml
\item
  SARM S-22 0.05 ng/ml
\item
  SARM LGD-4033 0.05 ng/ml
\item
  Torasemide 20 ng/ml
\item
  Trenbolone 0.5 ng/ml
\item
  Triamterene 20 ng/ml
\item
  Zeranol 5 ng/ml
\item
  Zilpaterol 5 ng/ml
\end{itemize}

A sample will only be reported as positive by the laboratory if the estimated concentration of the Prohibited Substance in this Exhibit I-6 exceeds the relevant single-point calibrator, which will be set at 1.2 times the substance's confirmatory lab analysis level. The estimated concentration of such Prohibited Substance in a sample with a measured specific gravity (``SG'') greater than 1.018 will be adjusted as follows (where SGSample\_Max = SGSample + 0.002):

adj. concentration = ((1.020 -- 1)/(SGSample\_Max -- 1)) * est. concentration

\hypertarget{creation-of-player-longitudinal-profiles}{%
\section{CREATION OF PLAYER LONGITUDINAL PROFILES}\label{creation-of-player-longitudinal-profiles}}

The following protocol will be used to create the Longitudinal Profiles described in Article XXXIII, Section 19 above:

\begin{enumerate}
\def\labelenumi{\arabic{enumi}.}
\item
  \textbf{Step 1:} The Program's drug collection company will assign each player a unique personal identification number. A player's personal identification number will remain the same for all periods of time he is covered by the Program, and will only be used for the purposes of the Longitudinal Profile. Other than to the designated representatives or employees within the drug collection company and the Laboratory, the drug collection company will not disclose the personal identification number that corresponds to the player's name to any individual other than one representative each of the NBA and the Players Association.
\item
  \textbf{Step 2:} The Laboratory (as defined in Article XXXIII, Section 19(a)) will maintain a secure, separate database for each player's personal identification number that contains
  his corresponding Testosterone concentration, epitestosterone concentration and Testosterone/Epitestosterone (``T/E'') ratio (referred to collectively as the ``Baseline Values''). This database will not contain any identifying information for the players.
\item
  \textbf{Step 3:} The Baseline Values will be calculated, pursuant to the Laboratory's operating standards, by averaging a player's T/E ratio, Testosterone concentration and Epitestosterone concentration, respectively, from three (3) negative tests conducted under the Program. After a player's Baseline Values are established, those values will be considered a player's Longitudinal Profile for the duration of his coverage under the Program. New Baseline Values will be calculated for a player upon the recommendation of the director of the Laboratory.
\item
  \textbf{Step 4:} The Laboratory will compare the Baseline Values to the corresponding Specimen Values (as defined in Article XXXIII, Section 19(c)) in subsequent tests identified with a player's personal identification number in determining whether it will conduct IRMS analysis (as defined in Section 19(a)) on a urine specimen.
\end{enumerate}

\hypertarget{section-1}{%
\chapter{}\label{section-1}}

\hypertarget{form-of-confidentiality-agreement}{%
\section{FORM OF CONFIDENTIALITY AGREEMENT}\label{form-of-confidentiality-agreement}}

{[}Date{]}

National Basketball Players Association\\
1133 Avenue of the Americas\\
New York, New York 10036

Re: Confidentiality Agreement

Sir/Madam:

This will confirm the agreement of the National Basketball Players Association (on behalf of itself and its employees, officers, NBA team player representatives (``Player Representatives''), and outside advisors (collectively, the ``Players Association'')) to maintain the confidentiality of all Confidential Information (as defined in Paragraph 6 below) provided to the Players Association in connection with the audit, with respect to the 20\_\_-20\_\_ Salary Cap Year, of (i) the National Basketball Association (``NBA''), and any League-related entities associated with generating BRI, (ii) any NBA team that is included in such audit with respect to such Salary Cap Year (the ``Team(s)''), under the Collective Bargaining Agreement entered into June 28, 2023 (``CBA''), between the Players Association and the NBA (collectively, the ``Audit''). Capitalized terms not defined herein shall have the meaning ascribed to such terms in the CBA.

\begin{enumerate}
\def\labelenumi{\arabic{enumi}.}
\item
  The NBA and the Team(s) shall make available Confidential Information for purposes of the Audit based on the Players Association's representation that it (and its employees, officers, Player Representatives, and outside advisors) shall comply with the terms of this Confidentiality Agreement at all times during and after the Audit. To that end, before any employee, officer, Player Representative, or outside advisor of the Players Association may be permitted to review any Confidential Information, the Players Association shall require such employee, officer, Player Representative, or outside advisor to agree, in writing (in the form of acknowledgment annexed hereto), to comply with the terms of this Confidentiality Agreement, and the Players Association shall promptly provide copies of such writings to the NBA.
\item
  The Players Association shall maintain the absolute confidentiality of all Confidential Information at all times and shall not disclose, disseminate, or provide Confidential Information to any person or entity (including, but not limited to, any NBA players who are not officers of the Players Association and any representative of any player) at any time or for any purpose, except as permitted herein. The Players Association agrees that it may use or refer to Confidential Information only during the course of the Audit and solely for the purpose of conducting the Audit in accordance with the terms and conditions of the CBA and this Confidentiality Agreement, and that Confidential Information may not be used or referred to by the Players Association, at any time, for any other purpose. Notwithstanding the foregoing, or anything else in this letter agreement, the Players Association may only disclose or provide a summary of Confidential Information to Player Representatives in aggregate form without identifying any specific information (e.g., by sponsor). Notwithstanding anything to the contrary in this Confidentiality Agreement, the Players Association shall not be deemed to have violated any provision herein if the Players Association discloses to such third party that the Audit is being undertaken and that the Players Association is subject to a confidentiality agreement and, therefore, not permitted to discuss the Audit. The foregoing shall not foreclose the Players Association from disclosing Confidential Information during the course of a proceeding before the System Arbitrator, an appeal to the Appeals Panel of an award of the System Arbitrator, or a judicial action to enforce any such proceeding or award.
\item
  The Players Association shall adopt and implement such procedures to ensure the confidentiality of Confidential Information as would be employed by a reasonable and prudent person to safeguard the confidentiality of his or her own most confidential information, or, if more stringent, such procedures as are employed for such purposes by the Players Association for such information. Such procedures shall include, but not be limited to, steps to ensure that: (a) such Confidential Information is disclosed only to those Players Association employees, officers, outside advisors, and, subject to the restrictions set forth in Paragraph 2 above, Player Representatives who have a need to have access to such Confidential Information and only for the purpose of conducting the Audit in accordance with the terms of the CBA and this Confidentiality Agreement; and (b) before any such person is permitted to review any Confidential Information, he or she agrees in writing to comply with the terms of this Confidentiality Agreement by signing the form of acknowledgment annexed hereto as provided for in Paragraph 1 above. The foregoing shall not foreclose the Players Association from disclosing Confidential Information during the course of a proceeding before the System Arbitrator, an appeal to the Appeals Panel of an award of the System Arbitrator, or a judicial action to enforce any such proceeding or award.
\item
  The Players Association agrees that no copies of Confidential Information made available by the NBA and the Teams at their respective offices in connection with the Audit may be removed from such offices without the express written consent of the NBA or the Teams (as applicable) (for example, in connection with the use of online data rooms to permit access to information provided electronically during the on-site audit or to respond to information requests). Should the NBA or the Teams permit copies of Confidential Information to be removed from their offices in connection with the Audit, then at the request of the NBA, all such copies shall be returned to the NBA within thirty (30) days following completion of the Audit. Notwithstanding the foregoing, the Players Association shall be under no obligation to return copies of the final Audit Report or any debriefing memoranda (except to the extent such memoranda append contract documents) prepared by the Accountants and provided to the Players Association in connection with any audit pursuant to Article VII, Section 10.
\item
  If the Players Association is required by governmental or judicial authorities (by oral questions, interrogatories, requests for information or documents, subpoena, civil investigative demand, or any other similar process) to disclose any Confidential Information, it shall provide the NBA and/or the Teams with prompt notice so that the NBA and/or the Teams may seek an appropriate protective order. If, in the absence of a protective order, the Players Association is, after giving notice in accordance with the preceding sentence, compelled to disclose Confidential Information or else stand liable for contempt or suffer other censure or penalty, the Players Association may disclose only such Confidential Information as is necessary to avoid such liability without incurring liability hereunder.
\item
  For purposes of this Confidentiality Agreement, ``Confidential Information'' shall mean all documents, materials, and other information reviewed or made available (whether in written or oral form) in connection with the Audit (including, without limitation, all documents, debriefing memoranda, materials, and other information made available by PricewaterhouseCoopers, LLP (``PwC'')), and shall include all excerpts, extracts, summaries, and contents thereof and notes taken by the Players Association during the Audit; provided, however that Confidential Information shall not include information that (a) is or becomes generally available to the public other than as a result of disclosure by the Players Association (including Players Association affiliates or representatives), (b) was available to the Players Association prior to its disclosure by the NBA, the Team(s), or PwC (as applicable), or (c) becomes available to the Players Association from a source other than the NBA, the Team(s), or PwC, provided that such source is not bound by a confidentiality agreement with the NBA, the Teams, the Players Association, or PwC.
\item
  The Players Association acknowledges that the terms and conditions contained in this Confidentiality Agreement are reasonable and necessary to protect the legitimate interests of the NBA and the Teams, do not cause the Players Association undue hardship, and that any violation of the provisions of this Confidentiality Agreement or disclosure of any Confidential Information without the NBA's or the Teams' (as applicable) prior written consent will result in irreparable injury to the NBA and/or the Teams for which there is no adequate remedy at law. Accordingly, in the event of any such violation or disclosure, the NBA and/or the Teams shall be entitled to preliminary and permanent injunctive relief from any federal or state court of competent jurisdiction located in New York, New York, and the Players Association hereby consents to, and waives any objection to, venue and jurisdiction in such courts. In addition, the Players Association shall indemnify and hold harmless the NBA and its member Teams and their respective affiliates, owners, directors, governors, officers, and employees, and the successors, assigns, and personal representatives of the foregoing parties (``NBA indemnified parties''), from and against all liability, damages, and costs (including attorneys' fees) arising out of any claim asserted against any NBA indemnified party relating to any violation of this Confidentiality Agreement by the Players Association, provided that: (a) such violation resulted from the Players Association's negligent or intentional use or disclosure of Confidential Information; (b) the Players Association is given prompt notice of any such claim; (c) the Players Association has the right to approve counsel and/or has the opportunity to undertake the defense of such claim; and (d) the indemnified party does not admit liability with respect to and does not settle such claim without the prior written consent of the Players Association. The Players Association also agrees that the relief provided for in this Paragraph 7 shall be cumulative and in addition to any other rights or remedies to which the NBA and the Teams may be entitled.
\item
  This Confidentiality Agreement is the final and complete agreement between the parties with respect to its subject matter. Any waiver of or modification to this Confidentiality Agreement must be in a writing and signed by each party. Any waiver in any particular instance of the rights and limitations contained herein shall not be deemed and is not intended to be a general waiver of any rights or limitations contained herein and shall not operate as a waiver beyond the particular instance.
\item
  This Confidentiality Agreement shall be governed by and construed and enforced in accordance with the laws of the State of New York, without giving effect to the principles of conflicts of law thereof. If the foregoing coincides with your understanding of our agreement, please sign the this letter in the space provided below.
\end{enumerate}

Sincerely,

\begin{longtable}[]{@{}l@{}}
\toprule()
\endhead
NATIONAL BASKETBALL ASSOCIATION \\
By: \_\_\_\_\_\_\_\_\_\_\_\_\_\_\_\_\_\_\_\_\_\_\_\_\_\_\_\_\_\_\_\_\_\_\_ \\
 \\
AGREED TO AND ACCEPTED: \\
NATIONAL BASKETBALL PLAYERS ASSOCIATION \\
By: \_\_\_\_\_\_\_\_\_\_\_\_\_\_\_\_\_\_\_\_\_\_\_\_\_\_\_\_\_\_\_\_\_\_\_ \\
\bottomrule()
\end{longtable}

\newpage

\hypertarget{letter-agreement-regarding-accounting-procedures}{%
\section{LETTER AGREEMENT REGARDING ACCOUNTING PROCEDURES}\label{letter-agreement-regarding-accounting-procedures}}

June 28, 2023

Tamika Tremaglio\\
Executive Director\\
National Basketball Players Association\\
1133 Avenue of the Americas\\
New York, New York 10036

Dear Tamika:

This will confirm our agreement that the attached accounting procedures are the procedures that will be in effect for purposes of Article VII, Section 10 of the Collective Bargaining Agreement entered into on June 28, 2023, unless such procedures shall be modified by agreement of the parties. If the foregoing coincides with your understanding of our agreement, please sign this letter in the space provided below.

Sincerely,\\
/s/ RICHARD W. BUCHANAN\\
Richard W. Buchanan

AGREED TO AND\\
ACCEPTED:

NATIONAL BASKETBALL PLAYERS ASSOCIATION

By:\\
/s/ TAMIKA TREMAGLIO\\
Tamika Tremaglio\\
Executive Director

\newpage

\hypertarget{minimum-procedures-to-be-provided-by-the-accountants}{%
\subsection{Minimum Procedures to Be Provided by the Accountants}\label{minimum-procedures-to-be-provided-by-the-accountants}}

\textbf{General}

\begin{itemize}
\tightlist
\item
  The Audit Report (and any Interim Audit Report or Interim Escrow Audit Report) must be prepared in accordance with the relevant terms of the Collective Bargaining Agreement (``CBA''), which should be reviewed and understood by all auditors.
\item
  The Basketball Related Income Reporting Package and instructions should be reviewed and understood by all auditors.
\item
  All audit workpapers should be made available for review by representatives of the NBA and Players Association prior to issuance of the report.
\item
  A summary of all audit findings (including any unusual or non-recurring transactions) and proposed adjustments must be jointly reviewed with representatives of the NBA and Players Association prior to issuance of the report.
\item
  Any problems or questions raised during the audit should be resolved jointly with representatives of the NBA and Players Association (or by the Accountants, to the extent called for under the CBA).
\item
  All estimates should be reviewed in accordance with the CBA. Estimates are to be reviewed based upon the previous year's actual results and current year activity. All estimates should be confirmed with third parties when possible.
\item
  Revenue and expense amounts that have been estimated should be reconfirmed with the controller or other team representatives prior to the issuance of the Audit Report on or before the last day of the Moratorium Period.
\item
  Where appropriate, team and NBA revenues and expenses should be reconciled to audited financial statements.
\item
  All reporting packages and supporting schedules are to be completed in U.S. dollars.
\item
  The Auditors may consider, but are not bound by, the value attributed to or treatment of revenue or expense items in prior years.
\item
  Auditors should be aware of revenues excluded from BRI. The Teams should be instructed to make available to the Auditors all information necessary to determine categories of revenues they have excluded from BRI. Questions regarding whether revenues or expenses are includable or excludable from BRI should be reviewed with both parties to determine proper treatment. Auditors should perform a review for revenues improperly excluded from, or included in, BRI.
\end{itemize}

\textbf{Team Salaries}

\begin{itemize}
\tightlist
\item
  Trace amounts to the team's general ledger or other supporting documentation for agreement.
\item
  Foot all schedules and perform other clerical tests.
\item
  Examine an appropriate sample of player contracts, noting agreement of all salary amounts, in accordance with the definition of Salary in the CBA.
\item
  Compare player names with all player lists for the season in question.
\item
  Inquire of controller or other representative of each team if any additional compensation was paid to players and not included on the schedule, and, if so, whether or not such amounts were paid for basketball services. Also inquire if any business arrangements were entered into by the team or team affiliate with players or their affiliates, including with retired players who played for the team within the past five (5) years.
\item
  Review performance bonuses to determine whether such bonuses were actually earned for such season.
\item
  Review signing bonuses to determine if they have been properly allocated in accordance with the terms of the CBA.
\item
  Confirm that, where provided in the CBA, certain contracts have been averaged.
\end{itemize}

\textbf{Benefits}

\begin{itemize}
\tightlist
\item
  Trace amounts to the team's general ledger or other supporting documentation for agreement.
\item
  Foot all schedules and perform other clerical tests.
\item
  Investigate variations in amounts from the prior year through discussion with the controller or other representative of the team.
\item
  Review each team's insurance expenses for premium credits (refunds) received from Planet Insurance Ltd.~(owned by Teams) and the players' medical and dental insurance carriers (amounts can be obtained from League Office).
\item
  Review League Office supporting documentation with respect to Benefits.
\end{itemize}

\textbf{Basketball Related Income}

\begin{itemize}
\tightlist
\item
  Trace amounts to team's general ledger or other supporting documentation for agreement.
\item
  Foot all schedules and perform other clerical tests.
\item
  Trace gate receipts to general ledger and test supporting documentation where appropriate.
\item
  Gate receipts should be reviewed and reconciled to League Office gate receipts summary.
\item
  Verify amounts reported as luxury suite revenues with supporting documentation from the entity that sold, leased, or licensed such luxury suites.
\item
  Verify amounts reported as complimentary tickets and tickets traded for goods or services with supporting documentation from the team.
\item
  Trace amounts reported for novelties and concessions, game parking, game programs, Team sponsorships and promotions, arena signage, and arena club sales to general ledgers and test supporting documentation where appropriate.
\item
  Where reported amounts include proceeds received by a Related Party, verify the amounts reported with supporting documentation from the Related Party.
\item
  Examine the National Television and Cable contracts at the League Office, and agree to amounts reported.
\item
  Review, at League Office, expenses deducted from the National contracts in accordance with the terms of the CBA. Review supporting documentation and test where applicable.
\item
  Examine local television, local cable, and local radio contracts. Verify to amounts reported by teams.
\item
  When local broadcast revenues are not verifiable by reviewing a contract, detailed supporting documentation should be reviewed and tested.
\item
  All loans, advances, bonuses, etc. received by the League Office or its teams should be noted in the report and included in BRI where appropriate.
\item
  Schedules of NBA Radio, NBA TV, international broadcast, NBA Media Ventures, copyright royalty revenues and expenses should be obtained from the NBA. Schedules should be verified by agreeing to general ledgers and examining supporting documentation where applicable.
\item
  Schedules of revenues and expenses reported by Properties for sponsorship, NBA related revenues from NBA Entertainment, and NBA Special Events should be obtained from the NBA. Schedules should be verified by agreeing to general ledgers and examining supporting documentation where applicable.
\item
  Net exhibition revenues and expenses should be verified to supporting documentation where appropriate.
\item
  All amounts of other revenues should be reviewed for proper inclusion/exclusion in BRI. Test appropriateness of balances where appropriate.
\item
  Determine the ratio of expenses to revenues for those categories of proceeds that come within the provisions of Article VII, Section 1(a)(6) of the CBA and determine the extent to which expenses should be disallowed, if at all, pursuant to the provisions of that Section.
\end{itemize}

\textbf{Playoff Revenues}

\begin{itemize}
\tightlist
\item
  All sources of playoff revenues and expenses should be verified per the procedure outlined for Basketball Related Income.
\item
  Because of the late timing of the Playoffs, special attention should be given to revenue and expense estimates.
\item
  Playoff gate receipts should be recorded net of Taxes. Payments made to the Playoff Pool should not be deducted. Odd game payments should not be either deducted by the paying team or recorded by the receiving team.
\item
  Other playoff expenses should be reviewed in accordance with the terms of the CBA.
\item
  Team expenses paid by the League Playoff Pool, including travel expenses, should not be deducted by teams.
\item
  Review League Office supporting documentation as to expenses deducted from the Playoff Pool.Related Party Transactions
\item
  Inquire of the controller or other representative of the team what, if any, Related Parties exist, and discuss with the parties what, if any, amounts should be included in BRI.
\item
  Review information provided as to the team's Related Parties and revenues that arise from Related Party transactions, and request supporting details where appropriate.
\item
  Any revenue from a Related Party should be reviewed with both parties to determine proper treatment under the CBA.
\item
  Request that details be provided, where appropriate.
\item
  Prepare a summary of any changes, corrections, or additions to Related Party information previously reported.
\end{itemize}

\end{document}
